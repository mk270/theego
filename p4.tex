
\chapter[All Things Are Nothing To Me]{\centering All Things 
Are Nothing To Me\footnote{\textit{``Ich hab' 
Mein' Sach' auf Nichts gestellt''}, first line of Goethe's poem, 
\textit{``Vanitas! Vanitatum Vanitas!''} Literal translation: ``I have set 
my affair on nothing.''}}

What is not supposed to be my concern!\footnote{\textit{Sache}} First and 
foremost, the Good Cause,\footnote{\textit{Sache}} then God's cause, the cause 
of mankind, of truth, of freedom, of humanity, of justice; further, the cause 
of my people, my prince, my fatherland; finally, even the cause of Mind, and a 
thousand other causes. Only \textit{my} cause is never to be my concern. 
``Shame on the egoist who thinks only of himself!''

Let us look and see, then, how they manage \textit{their} concerns ---they for 
whose cause we are to labor, devote ourselves, and grow enthusiastic.

You have much profound information to give about God, and have for thousands 
of years ``searched the depths of the Godhead,'' and looked into its heart, 
so that you can doubtless tell us how God himself attends to ``God's 
cause,'' which we are called to serve. And you do not conceal the Lord's 
doings, either. Now, what is his cause? Has he, as is demanded of us, made an 
alien cause, the cause of truth or love, his own? You are shocked by this 
misunderstanding, and you instruct us that God's cause is indeed the cause of 
truth and love, but that this cause cannot be called alien to him, because God 
is himself truth and love; you are shocked by the assumption that God could be 
like us poor worms in furthering an alien cause as his own. ``Should God take 
up the cause of truth if he were not himself truth?'' He cares only for 
\textit{his} cause, but, because he is all in all, therefore all is his cause! 
But we, we are not all in all, and our cause is altogether little and 
contemptible; therefore we must ``serve a higher cause.'' ---Now it is 
clear, God cares only for what is his, busies himself only with himself, 
thinks only of himself, and has only himself before his eyes; woe to all that 
is not well-pleasing to him. He serves no higher person, and satisfies only 
himself. His cause is ---a purely egoistic cause.

How is it with mankind, whose cause we are to make our own? Is its cause that 
of another, and does mankind serve a higher cause? No, mankind looks only at 
itself, mankind will promote the interests of mankind only, mankind is its own 
cause. That it may develop, it causes nations and individuals to wear 
themselves out in its service, and, when they have accomplished what mankind 
needs, it throws them on the dung-heap of history in gratitude. Is not 
mankind's cause ---a purely egoistic cause?

I have no need to take up each thing that wants to throw its cause on us and 
show that it is occupied only with itself, not with us, only with its good, 
not with ours. Look at the rest for yourselves. Do truth, freedom, humanity, 
justice, desire anything else than that you grow enthusiastic and serve them?

They all have an admirable time of it when they receive zealous homage. Just 
observe the nation that is defended by devoted patriots. The patriots fall in 
bloody battle or in the fight with hunger and want; what does the nation care 
for that? By the manure of their corpses the nation comes to ``its bloom''! 
The individuals have died ``for the great cause of the nation,'' and the 
nation sends some words of thanks after them and ---has the profit of it. I 
call that a paying kind of egoism.

But only look at that Sultan who cares so lovingly for his people. Is he not 
pure unselfishness itself, and does he not hourly sacrifice himself for his 
people? Oh, yes, for ``his people.'' Just try it; show yourself not as his, 
but as your own; for breaking away from his egoism you will take a trip to 
jail. The Sultan has set his cause on nothing but himself; he is to himself 
all in all, he is to himself the only one, and tolerates nobody who would dare 
not to be one of ``his people.''

And will you not learn by these brilliant examples that the egoist gets on 
best? I for my part take a lesson from them, and propose, instead of further 
unselfishly serving those great egoists, rather to be the egoist myself.

God and mankind have concerned themselves for nothing, for nothing but 
themselves. Let me then likewise concern myself for \textit{myself,} who am 
equally with God the nothing of all others, who am my all, who am the only 
one.\footnote{\textit{Der Einzige}}

If God, if mankind, as you affirm, have substance enough in themselves to be 
all in all to themselves, then I feel that I shall still less lack that, and 
that I shall have no complaint to make of my ``emptiness.'' I am not nothing 
in the sense of emptiness, but I am the creative nothing, the nothing out of 
which I myself as creator create everything.

Away, then, with every concern that is not altogether my concern! You think at 
least the ``good cause'' must be my concern? What's good, what's bad? Why, I 
myself am my concern, and I am neither good nor bad. Neither has meaning for 
me.

The divine is God's concern; the human, man's. My concern is neither the 
divine nor the human, not the true, good, just, free, etc., but solely what is 
\textit{mine}, and it is not a general one, but is --- 
unique,\footnote{\textit{Einzig}} as I am unique.

Nothing is more to me than myself!

