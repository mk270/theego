\documentclass[12pt,a4paper]{book}
%\addtolength{\topmargin}{-1cm}
%\addtolength{\evensidemargin}{-3cm}
%\addtolength{\oddsidemargin}{-1cm}
%\addtolength{\textwidth}{4cm}
\usepackage[T1]{fontenc}
\newif\ifpdf
\ifx\pdfoutput\undefined
        \pdffalse
\else
        \pdftrue
\fi

\ifpdf
        % Saves a lot of space in PDF files
        \usepackage{times}
        \pdfcompresslevel=9
\fi

\title{The Ego and His Own}
\author{Max Stirner}
\date{1910}
\begin{document}

\maketitle

\tableofcontents

\newpage{}

\begin{center}
{\Huge THE EGO AND HIS\\
\medskip{}
OWN}

\medskip{}

by

\medskip{}

{\Huge MAX STIRNER}

\medskip{}

A Reproduction of the First English Edition.

\medskip{}

Translated from the German by\\
Steven T. Byington

\medskip{}

With an Introduction by\\
 J. L. Walker

\medskip{}

New York\\
BENJ. R. TUCKER, Publisher\\
1907

\medskip{}

Brought to you through the cooperative efforts of\\
Larry Schiereck, Svein Olav Nyberg, and Daniel T. Davis

\end{center}

\medskip{}

\noindent{}This transcription is a copy intended for 
electronic reading. Page numbering 
etc are not consistent to that of the original text. This electronic edition 
was created by Linus Walleij the year 2001, for supporting electronic books, 
PDF file generation and the like, to serve the casual reader. For this reason, 
and others, there is no index available in this version. The scientifically 
intresested are recommended to consult the HTML version originally created by 
Schiereck, Nyberg and Davis.

\medskip{}

\begin{center}
Version 0.1\end{center}

\medskip{}

\begin{center}
Copyright, 1907, by\\
BENJAMIN R. TUCKER
\end{center}

\newpage{}

~

\vspace{200pt}

\begin{center}
TO MY SWEETHEART\\
MARIE D\"AHNHARDT
\end{center}

\medskip{}

\setcounter{secnumdepth}{-1}
\chapter[Publisher's Preface]{\centering PUBLISHER'S PREFACE}

For more than twenty years I have entertained the design of publishing an 
English translation of \textit{"{}Der Einzige und Sein Eigentum}."{} When I 
formed this design, the number of English-speaking persons who had ever heard 
of the book was very limited. The memory of Max Stirner had been virtually 
extinct for an entire generation. But in the last two decades there has been a 
remarkable revival of interest both in the book and in its author. It began in 
this country with a discussion in the pages of the Anarchist periodical, 
"{}Liberty,"{} in which Stirner's thought was clearly expounded and vigorously 
championed by Dr. James L. Walker, who adopted for this discussion the 
pseudonym "{}Tak Kak."{} At that time Dr. Walker was the chief editorial 
writer for the Galveston "{}News."{} Some years later he became a practicing 
physician in Mexico, where he died in 1904. A series of essays which he began 
in an Anarchist periodical, "{}Egoism,"{} and which he lived to complete, was 
published after his death in a small volume, "{}The Philosophy of Egoism."{} 
It is a very able and convincing exposition of Stirner's teachings, and almost 
the only one that exists in the English language. But the chief instrument in 
the revival of Stirnerism was and is the German poet, John Henry Mackay. Very 
early in his career he met Stirner's name in Lange's "{}History of 
Materialism,"{} and was moved thereby to read his book. The work made such an 
impression on him that he resolved to devote a portion of his life to the 
rediscovery and rehabilitation of the lost and forgotten genius. Through years 
of toil and correspondence and travel, and triumphing over tremendous 
obstacles, he carried his task to completion, and his biography of Stirner 
appeared in Berlin in 1898. It is a tribute to the thoroughness of Mackay's 
work that since its publication not one important fact about Stirner has been 
discovered by anybody. During his years of investigation Mackay's advertising 
for information had created a new interest in Stirner, which was enhanced by 
the sudden fame of the writings of Friedrich Nietzsche, an author whose 
intellectual kinship with Stirner has been a subject of much controversy. 
\textit{"{}Der Einzige,"{}} previously obtainable only in an expensive form, 
was included in Philipp Reclam's Universal-Bibliothek, and this cheap edition 
has enjoyed a wide and ever-increasing circulation. During the last dozen 
years the book has been translated twice into French, once into Italian, once 
into Russian, and possibly into other languages. The Scandinavian critic, 
Brandes, has written on Stirner. A large and appreciative volume, entitled 
\textit{"{}L'Individualisme Anarchiste: Max Stirner,"{}} from the pen of Prof 
Victor Basch, of the University of Rennes, has appeared in Paris. Another 
large and sympathetic volume, "{}Max Stirner,"{} written by Dr. Anselm Ruest, 
has been published very recently in Berlin. Dr. Paul Eltzbacher, in his work, 
\textit{"{}Der Anarchismus},"{} gives a chapter to Stirner, making him one of 
the seven typical Anarchists, beginning with William Godwin and ending with 
Tolstoi, of whom his book treats. There is hardly a notable magazine or a 
review on the Continent that has not given at least one leading article to the 
subject of Stirner. Upon the initiative of Mackay and with the aid of other 
admirers a suitable stone has been placed above the philosopher's previously 
neglected grave, and a memorial tablet upon the house in Berlin where he died 
in 1856; and this spring another is to be placed upon the house in Bayreuth 
where he was born in 1806. As a result of these various efforts, and though 
but little has been written about Stirner in the English language, his name is 
now known at least to thousands in America and England where formerly it was 
known only to hundreds.

Therefore conditions are now more favorable for the reception of this volume 
than they were when I formed the design of publishing it, more than twenty 
years ago.

The problem of securing a reasonably good translation (for in the case of a 
work presenting difficulties so enormous it was idle to hope for an adequate 
translation) was finally solved by entrusting the task to Steven T. Byington, 
a scholar of remarkable attainments, whose specialty is philology, and who is 
also one of the ablest workers in the propaganda of Anarchism. But, for 
further security from error, it was agreed with Mr. Byington that his 
translation should have the benefit of revision by Dr. Walker, the most 
thorough American student of Stirner, and by Emma Heller Schumm and George 
Schumm, who are not only sympathetic with Stirner, but familiar with the 
history of his time, and who enjoy a knowledge of English and German that 
makes it difficult to decide which is their native tongue. It was also agreed 
that, upon any point of difference between the translator and his revisers 
which consultation might fail to solve, the publisher should decide. This 
method has been followed, and in a considerable number of instances it has 
fallen to me to make a decision. It is only fair to say, therefore, that the 
responsibility for special errors and imperfections properly rests on my 
shoulders, whereas, on the other hand, the credit for whatever general 
excellence the translation may possess belongs with the same propriety to Mr. 
Byington and his coadjutors. One thing is certain: its defects are due to no 
lack of loving care and pains. And I think I may add with confidence, while 
realizing fully how far short of perfection it necessarily falls, that it may 
safely challenge comparison with the translations that have been made into 
other languages.

In particular, I am responsible for the admittedly erroneous rendering of the 
title. "{}The Ego and His Own"{} is not an exact English equivalent of 
\textit{"{}Der Einzige und Sein Eigentum."{}} But then, there is no exact 
English equivalent. Perhaps the nearest is "{}The Unique One and His 
Property."{} But the unique one is not strictly the \textit{Einzige,} for 
uniqueness connotes not only singleness but an admirable singleness, while 
Stirner's \textit{Einzigkeit} is admirable in his eyes only as such, it being 
no part of the purpose of his book to distinguish a particular 
\textit{Einzigkeit} as more excellent than another. Moreover, "{}The Unique 
One and His Property "{} has no graces to compel our forgiveness of its slight 
inaccuracy. It is clumsy and unattractive. And the same objections may be 
urged with still greater force against all the other renderings that have been 
suggested, -- "{}The Single One and His Property,"{} "{}The Only One and His 
Property,"{} "{}The Lone One and His Property,"{} "{}The Unit and His 
Property,"{} and, last and least and worst, "{}The Individual and His 
Prerogative."{} "{} The Ego and His Own,"{} on the other hand, if not a 
precise rendering, is at least an excellent title in itself; excellent by its 
euphony, its monosyllabic incisiveness, and its telling -- 
\textit{Einzigkeit}. Another strong argument in its favor is the emphatic 
correspondence of the phrase "{}his own"{} with Mr. Byington's renderings of 
the kindred words, \textit{Eigenheit} and \textit{Eigner.} Moreover, no reader 
will be led astray who bears in mind Stirner's distinction: "{}I am not an ego 
along with other egos, but the sole ego; I am unique."{} And, to help the 
reader to bear this in mind, the various renderings of the word 
\textit{Einzige} that occur through the volume are often accompanied by 
foot-notes showing that, in the German, one and the same word does duty for 
all.

If the reader finds the first quarter of this book somewhat forbidding and 
obscure, he is advised nevertheless not to falter. Close attention will master 
almost every difficulty, and, if he will but give it, he will find abundant 
reward in what follows. For his guidance I may specify one defect in the 
author's style. When controverting a view opposite to his own, he seldom 
distinguishes with sufficient clearness his statement of his own view from his 
re-statement of the antagonistic view. As a result, the reader is plunged into 
deeper and deeper mystification, until something suddenly reveals the cause of 
his misunderstanding, after which he must go back and read again. I therefore 
put him on his guard. The other difficulties lie, as a rule, in the structure 
of the work. As to these I can hardly do better than translate the following 
passage from Prof. Basch's book, alluded to above: "{}There is nothing more 
disconcerting than the first approach to this strange work. Stirner does not 
condescend to inform us as to the architecture of his edifice, or furnish us 
the slightest guiding thread. The apparent divisions of the book are few and 
misleading. From the first page to the last a unique thought circulates, but 
it divides itself among an infinity of vessels and arteries in each of which 
runs a blood so rich in ferments that one is tempted to describe them all. 
There is no progress in the development, and the repetitions are 
innumerable... The reader who is not deterred by this oddity, or rather 
absence, of composition gives proof of genuine intellectual courage. At first 
one seems to be confronted with a collection of essays strung together, with a 
throng of aphorisms... But, if you read this book several times; if, after 
having penetrated the intimacy of each of its parts, you then traverse it as a 
whole, -- gradually the fragments weld themselves together, and Stirner's 
thought is revealed in all its unity, in all its force, and in all its 
depth."{}

A word about the dedication. Mackay's investigations have brought to light 
that Marie D\"ahnhardt had nothing whatever in common with Stirner, and so was 
unworthy of the honor conferred upon her. She was no \textit{Eigene.} I 
therefore reproduce the dedication merely in the interest of historical 
accuracy.

Happy as I am in the appearance of this book, my joy is not unmixed with 
sorrow. The cherished project was as dear to the heart of Dr. Walker as to 
mine, and I deeply grieve that he is no longer with us to share our delight in 
the fruition. Nothing, however, can rob us of the masterly introduction that 
he wrote for this volume (in 1903, or perhaps earlier), from which I will not 
longer keep the reader. This introduction, no more than the book itself, shall 
that \textit{Einzige}, Death, make his \textit{Eigentum.}

\begin{flushright}
February, 1907.\\
 \textit{B. R. T.}\end{flushright}

\chapter[Introduction]{\centering INTRODUCTION}

Fifty years sooner or later can make little difference in the; case of a book 
so revolutionary as this. It saw the light when a so-called revolutionary 
movement was preparing in men's minds which agitation was, however, only a 
disturbance due to desires to participate in government, and to govern and to 
be governed, in a manner different to that which prevails. The 
"{}revolutionists"{} of 1848 were bewitched with an idea. They were not at all 
the masters of ideas. Most of those who since that time have prided themselves 
upon being revolutionists have been and are likewise but the bondmen of an 
idea, -- that of the different lodgment of authority.

The temptation is, of course, present to attempt an explanation of the central 
thought of this work; but such an effort appears to be unnecessary to one who 
has the volume in his hand. The author's care in illustrating his meaning 
shows that he realized how prone the possessed man is to misunderstand 
whatever is not moulded according to the fashions in thinking. The author's 
learning was considerable, his command of words and ideas may never be 
excelled by another, and he judged it needful to develop his argument in 
manifold ways. So those who enter into the spirit of it will scarcely hope to 
impress others with the same conclusion in a more summary manner. Or, if one 
might deem that possible after reading Stirner, still one cannot think that it 
could be done so surely. The author has made certain work of it, even though 
he has to wait for his public; but still, the reception of the book by its 
critics amply proves the truth of the saying that one can give another 
arguments, but not understanding. The system-makers and system-believers thus 
far cannot get it out of their heads that any discourse about the nature of an 
ego must turn upon the common characteristics of egos, to make a systematic 
scheme of what they share as a generality. The critics inquire what kind of 
man the author is talking about. They repeat the question: What does he 
believe in? They fail to grasp the purport of the recorded answer: "{}I 
believe in myself"{}; which is attributed to a common soldier long before the 
time of Stirner. They ask, what is the principle of the self-conscious egoist, 
the Einzige? To this perplexity Stirner says: Change the question; put 
"{}who?"{} instead of "{}what?"{} and an answer can then be given by naming 
him!

This, of course, is too simple for persons governed by ideas, and for persons 
in quest of new governing ideas. They wish to classify the man. Now, that in 
me which you can classify is not my distinguishing self. "{}Man"{} is the 
horizon or zero of my existence as an individual. Over that I rise as I can. 
At least I am something more than "{}man in general."{} Pre-existing worship 
of ideals and disrespect for self had made of the ego at the very most a 
Somebody, oftener an empty vessel to be filled with the grace or the leavings 
of a tyrannous doctrine; thus a Nobody. Stirner dispels the morbid subjection, 
and recognizes each one who knows and feels himself as his own property to be 
neither humble Nobody nor befogged Somebody, but henceforth flat-footed and 
level-headed Mr. Thisbody, who has a character and good pleasure of his own, 
just as he has a name of his own. The critics who attacked this work and were 
answered in the author's minor writings, rescued from oblivion by John Henry 
Mackay, nearly all display the most astonishing triviality and impotent 
malice.

We owe to Dr. Eduard von Hartmann the unquestionable service which he rendered 
by directing attention to this book in his "{}Philosophie des 
Unbewu\ss{}ten,"{} the first edition of which was published in 1869, and in 
other writings. I do not begrudge Dr. von Hartmann the liberty of criticism 
which he used; and I think the admirers of Stirner's teaching must quite 
appreciate one thing which Von Hartmann did at a much later date. In "{}Der 
Eigene"{} of August 10, 1896, there appeared a letter written by him and 
giving, among other things, certain data from which to judge that, when 
Friedrich Nietzsche wrote his later essays, Nietzsche was not ignorant of 
Stirner's book.

Von Hartmann wishes that Stirner had gone on and developed his principle. Von 
Hartmann suggests that you and I are really the same spirit, looking out 
through two pairs of eyes. Then, one may reply, I need not concern myself 
about you, for in myself I have -- us; and at that rate Von Hartmann is merely 
accusing himself of inconsistency: for, when Stirner wrote this book, Von 
Hartmann's spirit was writing it; and it is just the pity that Von Hartmann in 
his present form does not indorse what he said in the form of Stirner, -- that 
Stirner was different from any other man; that his ego was not Fichte's 
transcendental generality, but "{}this transitory ego of flesh and blood."{} 
It is not as a generality that you and I differ, but as a couple of facts 
which are not to be reasoned into one. "{}I"{} is somewise Hartmann, and thus 
Hartmann is "{}I"{}; but I am not Hartmann, and Hartmann is not -- I. Neither 
am I the "{}I"{} of Stirner; only Stirner himself was Stirner's "{}I."{} Note 
how comparatively indifferent a matter it is with Stirner that one is an ego, 
but how all-important it is that one be a self-conscious ego, -- a 
self-conscious, self-willed person.

Those not self-conscious and self-willed are constantly acting from 
self-interested motives, but clothing these in various garbs. Watch those 
people closely in the light of Stirner's teaching, and they seem to be 
hypocrites, they have so many good moral and religious plans of which 
self-interest is at the end and bottom; but they, we may believe, do not know 
that this is more than a coincidence.

In Stirner we have the philosophical foundation for political liberty. His 
interest in the practical development of egoism to the dissolution of the 
State and the union of free men is clear and pronounced, and harmonizes 
perfectly with the economic philosophy of Josiah Warren. Allowing for 
difference of temperament and language, there is a substantial agreement 
between Stirner and Proudhon. Each would be free, and sees in every increase 
of the number of free people and their intelligence an auxiliary force against 
the oppressor. But, on the other hand, will any one for a moment seriously 
contend that Nietzsche and Proudhon march together in general aim and 
tendency, -- that they have anything in common except the daring to profane 
the shrine and sepulchre of superstition?

Nietzsche has been much spoken of as a disciple of Stirner, and, owing to 
favorable cullings from Nietzsche's writings, it has occurred that one of his 
books has been supposed to contain more sense than it really does -- so long 
as one had read only the extracts.

Nietzsche cites scores or hundreds of authors. Had he read everything, and not 
read Stirner?

But Nietzsche is as unlike Stirner as a tight-rope performance is unlike an 
algebraic equation.

Stirner loved liberty for himself, and loved to see any and all men and women 
taking liberty, and he had no lust of power. Democracy to him was sham 
liberty, egoism the genuine liberty.

Nietzsche, on the contrary, pours out his contempt upon democracy because it 
is not aristocratic. He is predatory to the point of demanding that those who 
must succumb to feline rapacity shall be taught to submit with resignation. 
When he speaks of "{}Anarchistic dogs"{} scouring the streets of great 
civilized cities; it is true, the context shows that he means the Communists; 
but his worship of Napoleon, his bathos of anxiety for the rise of an 
aristocracy that shall rule Europe for thousands of years, his idea of 
treating women in the oriental fashion, show that Nietzsche has struck out in 
a very old path -- doing the apotheosis of tyranny. We individual egoistic 
Anarchists, however, may say to the Nietzsche school, so as not to be 
misunderstood: We do not ask of the Napoleons to have pity, nor of the 
predatory barons to do justice. They will find it convenient for their own 
welfare to make terms with men who have learned of Stirner what a man can be 
who worships nothing, bears allegiance to nothing. To Nietzsche's rhodomontade 
of eagles in baronial form, born to prey on industrial lambs, we rather 
tauntingly oppose the ironical question: Where are your claws? What if the 
"{}eagles"{} are found to be plain barn-yard fowls on which more silly fowls 
have fastened steel spurs to hack the victims, who, however, have the power to 
disarm the sham "{}eagles"{} between two suns? Stirner shows that men make 
their tyrants as they make their gods, and his purpose is to unmake tyrants.

Nietzsche dearly loves a tyrant.

In style Stirner's work offers the greatest possible contrast to the puerile, 
padded phraseology of Nietzsche's "{}Zarathustra"{} and its false imagery. Who 
ever imagined such an unnatural conjuncture as an eagle "{}toting"{} a serpent 
in friendship? which performance is told of in bare words, but nothing comes 
of it. In Stirner we are treated to an enlivening and earnest discussion 
addressed to serious minds, and every reader feels that the word is to him, 
for his instruction and benefit, so far as he has mental independence and 
courage to take it and use it. The startling intrepidity of this book is 
infused with a whole-hearted love for all mankind, as evidenced by the fact 
that the author shows not one iota of prejudice or any idea of division of men 
into ranks. He would lay aside government, but would establish any regulation 
deemed convenient, and for this only our convenience in consulted. Thus there 
will be general liberty only when the disposition toward tyranny is met by 
intelligent opposition that will no longer submit to such a rule. Beyond this 
the manly sympathy and philosophical bent of Stirner are such that rulership 
appears by contrast a vanity, an infatuation of perverted pride. We know not 
whether we more admire our author or more love him.

Stirner's attitude toward woman is not special. She is an individual if she 
can be, not handicapped by anything he says, feels, thinks, or plans. This was 
more fully exemplified in his life than even in this book; but there is not a 
line in the book to put or keep woman in an inferior position to man, neither 
is there anything of caste or aristocracy in the book. Likewise there is 
nothing of obscurantism or affected mysticism about it. Everything in it is 
made as plain as the author could make it. He who does not so is not Stirner's 
disciple nor successor nor co-worker. Some one may ask: How does plumb-line 
Anarchism train with the unbridled egoism proclaimed by Stirner? The 
plumb-line is not a fetish, but an intellectual conviction, and egoism is a 
universal fact of animal life. Nothing could seem clearer to my mind than that 
the reality of egoism must first come into the consciousness of men, before we 
can have the unbiased Einzige in place of the prejudiced biped who lends 
himself to the support of tyrannies a million times stronger over me than the 
natural self-interest of any individual. When plumb-line doctrine is 
misconceived as duty between unequal-minded men, -- as a religion of humanity, 
-- it is indeed the confusion of trying to read without knowing the alphabet 
and of putting philanthropy in place of contract. But, if the plumb-line be 
scientific, it is or can be my possession, my property, and I choose it for 
its use -- when circumstances admit of its use. I do not feel bound to use it 
because it is scientific, in building my house; but, as my will, to be 
intelligent, is not to be merely wilful, the adoption of the plumb-line 
follows the discarding of incantations. There is no plumb-line without the 
unvarying lead at the end of the line; not a fluttering bird or a clawing cat.

On the practical side of the question of egoism versus self-surrender and for 
a trial of egoism in politics, this may be said: the belief that men not moved 
by a sense of duty will be unkind or unjust to others is but an indirect 
confession that those who hold that belief are greatly interested in having 
others live for them rather than for themselves. But I do not ask or expect so 
much.

I am content if others individually live for themselves, and thus cease in so 
many ways to act in opposition to my living for myself, -- to our living for 
ourselves.

If Christianity has failed to turn the world from evil, it is not to be 
dreamed that rationalism of a pious moral stamp will succeed in the same task. 
Christianity, or all philanthropic love, is tested in non-resistance. It is a 
dream that example will change the hearts of rulers, tyrants, mobs. If the 
extremest self-surrender fails, how can a mixture of Christian love and 
worldly caution succeed? This at least must be given up. The policy of Christ 
and Tolstoi can soon be tested, but Tolstoi's belief is not satisfied with a 
present test and failure. He has the infatuation of one who persists because 
this ought to be. The egoist who thinks "{}I should like this to be"{} still 
has the sense to perceive that it is not accomplished by the fact of some 
believing and submitting, inasmuch as others are alert to prey upon the 
unresisting. The Pharaohs we have ever with us.

Several passages in this most remarkable book show the author as a man full of 
sympathy. When we reflect upon his deliberately expressed opinions and 
sentiments, -- his spurning of the sense of moral obligation as the last form 
of superstition, -- may we not be warranted in thinking that the total 
disappearance of the sentimental supposition of duty liberates a quantity of 
nervous energy for the purest generosity and clarifies the intellect for the 
more discriminating choice of objects of merit?

\begin{flushright}
J. L. WALKER.\end{flushright}

\chapter[Translator's Preface]{\centering TRANSLATOR'S PREFACE}

If the style of this book is found unattractive, it will show that I have done 
my work ill and not represented the author truly; but, if it is found odd, I 
beg that I may not bear all the blame. I have simply tried to reproduce the 
author's own mixture of colloquialisms and technicalities, and his preference 
for the precise expression of his thought rather than the word conventionally 
expected.

One especial feature of the style, however, gives the reason why this preface 
should exist. It is characteristic of Stirner's writing that the thread of 
thought is carried on largely by the repetition of the same word in a modified 
form or sense. That connection of ideas which has guided popular instinct in 
the formation of words is made to suggest the line of thought which the writer 
wishes to follow. If this echoing of words is missed, the bearing of the 
statements on each other is in a measure lost; and, where the ideas are very 
new, one cannot afford to throw away any help in following their connection. 
Therefore, where a useful echo (and then are few useless ones in the book) 
could not be reproduced in English, I have generally called attention to it in 
a note. My notes are distinguished from the author's by being enclosed in 
parentheses.

One or two of such coincidences of language, occurring in words which are 
prominent throughout the book, should be borne constantly in mind as a sort of 
\textit{Keri perpetuum;} for instance, the identity in the original of the 
words "{}spirit"{} and "{}mind,"{} and of the phrases "{}supreme being"{} and 
"{}highest essence."{} In such cases I have repeated the note where it seemed 
that such repetition might be absolutely necessary, but have trusted the 
reader to carry it in his head where a failure of his memory would not be 
ruinous or likely.

For the same reason--that is, in order not to miss any indication of the drift 
of the thought -- I have followed the original in the very liberal use of 
italics, and in the occasional eccentric use of a punctuation mark, as I might 
not have done in translating a work of a different nature.

I have set my face as a flint against the temptation to add notes that were 
not part of the translation. There is no telling how much I might have 
enlarged the book if I had put a note at every sentence which deserved to have 
its truth brought out by fuller elucidation -- or even at every one which I 
thought needed correction. It might have been within my province, if I had 
been able, to explain all the allusions to contemporary events, but I doubt 
whether any one could do that properly without having access to the files of 
three or four well-chosen German newspapers of Stirner's time. The allusions 
are clear enough, without names and dates, to give a vivid picture of certain 
aspects of German life then. The tone of some of them is explained by the fact 
that the book was published under censorship.

I have usually preferred, for the sake of the connection, to translate 
Biblical quotations somewhat as they stand in the German, rather than conform 
them altogether to the English Bible. I am sometimes quite as near the 
original Greek as if I had followed the current translation.

Where German books are referred to, the pages cited are those of the German 
editions even when (usually because of some allusions in the text) the titles 
of the books are translated.

\begin{flushright}
Steven T. Byington\end{flushright}


\newpage{}

~

\vspace{200pt}

\begin{center}
\textbf{THE EGO AND HIS OWN}
\end{center}


\chapter[All Things Are Nothing To Me]{\centering All Things 
Are Nothing To Me\footnote{\textit{"{}Ich hab' 
Mein' Sach' auf Nichts gestellt}, first line of Goethe's poem, 
\textit{"{}Vanitas! Vanitatum Vanitas!}"{} Literal translation: "{}I have set 
my affair on nothing."{}}}

What is not supposed to be my concern!\footnote{\textit{Sache}} First and 
foremost, the Good Cause,\footnote{\textit{Sache}} then God's cause, the cause 
of mankind, of truth, of freedom, of humanity, of justice; further, the cause 
of my people, my prince, my fatherland; finally, even the cause of Mind, and a 
thousand other causes. Only \textit{my} cause is never to be my concern. 
"{}Shame on the egoist who thinks only of himself!"{}

Let us look and see, then, how they manage \textit{their} concerns -- they for 
whose cause we are to labor, devote ourselves, and grow enthusiastic.

You have much profound information to give about God, and have for thousands 
of years "{}searched the depths of the Godhead,"{} and looked into its heart, 
so that you can doubtless tell us how God himself attends to "{}God's 
cause,"{} which we are called to serve. And you do not conceal the Lord's 
doings, either. Now, what is his cause? Has he, as is demanded of us, made an 
alien cause, the cause of truth or love, his own? You are shocked by this 
misunderstanding, and you instruct us that God's cause is indeed the cause of 
truth and love, but that this cause cannot be called alien to him, because God 
is himself truth and love; you are shocked by the assumption that God could be 
like us poor worms in furthering an alien cause as his own. "{}Should God take 
up the cause of truth if he were not himself truth?"{} He cares only for 
\textit{his} cause, but, because he is all in all, therefore all is his cause! 
But we, we are not all in all, and our cause is altogether little and 
contemptible; therefore we must "{}serve a higher cause."{} -- Now it is 
clear, God cares only for what is his, busies himself only with himself, 
thinks only of himself, and has only himself before his eyes; woe to all that 
is not well-pleasing to him. He serves no higher person, and satisfies only 
himself. His cause is -- a purely egoistic cause.

How is it with mankind, whose cause we are to make our own? Is its cause that 
of another, and does mankind serve a higher cause? No, mankind looks only at 
itself, mankind will promote the interests of mankind only, mankind is its own 
cause. That it may develop, it causes nations and individuals to wear 
themselves out in its service, and, when they have accomplished what mankind 
needs, it throws them on the dung-heap of history in gratitude. Is not 
mankind's cause -- a purely egoistic cause?

I have no need to take up each thing that wants to throw its cause on us and 
show that it is occupied only with itself, not with us, only with its good, 
not with ours. Look at the rest for yourselves. Do truth, freedom, humanity, 
justice, desire anything else than that you grow enthusiastic and serve them?

They all have an admirable time of it when they receive zealous homage. Just 
observe the nation that is defended by devoted patriots. The patriots fall in 
bloody battle or in the fight with hunger and want; what does the nation care 
for that? By the manure of their corpses the nation comes to "{}its bloom"{}! 
The individuals have died "{}for the great cause of the nation,"{} and the 
nation sends some words of thanks after them and -- has the profit of it. I 
call that a paying kind of egoism.

But only look at that Sultan who cares so lovingly for his people. Is he not 
pure unselfishness itself, and does he not hourly sacrifice himself for his 
people? Oh, yes, for "{}his people."{} Just try it; show yourself not as his, 
but as your own; for breaking away from his egoism you will take a trip to 
jail. The Sultan has set his cause on nothing but himself; he is to himself 
all in all, he is to himself the only one, and tolerates nobody who would dare 
not to be one of "{}his people."{}

And will you not learn by these brilliant examples that the egoist gets on 
best? I for my part take a lesson from them, and propose, instead of further 
unselfishly serving those great egoists, rather to be the egoist myself.

God and mankind have concerned themselves for nothing, for nothing but 
themselves. Let me then likewise concern myself for \textit{myself,} who am 
equally with God the nothing of all others, who am my all, who am the only 
one.\footnote{\textit{Der Einzige}}

If God, if mankind, as you affirm, have substance enough in themselves to be 
all in all to themselves, then I feel that I shall still less lack that, and 
that I shall have no complaint to make of my "{}emptiness."{} I am not nothing 
in the sense of emptiness, but I am the creative nothing, the nothing out of 
which I myself as creator create everything.

Away, then, with every concern that is not altogether my concern! You think at 
least the "{}good cause"{} must be my concern? What's good, what's bad? Why, I 
myself am my concern, and I am neither good nor bad. Neither has meaning for 
me.

The divine is God's concern; the human, man's. My concern is neither the 
divine nor the human, not the true, good, just, free, etc., but solely what is 
\textit{mine}, and it is not a general one, but is -- 
unique,\footnote{\textit{Einzig}} as I am unique.

Nothing is more to me than myself!


\chapter[Part First: Man]{\centering{\normalsize Part First}\\
MAN}

\newpage{}

~

\vspace{200pt}

\begin{quote}

\textit{Man is to man the supreme being,}, says Feuerbach.

\textit{Man has just been discovered,}says Bruno Bauer.

Then let us take a more careful look at this supreme being and this new 
discovery.

\end{quote}

\medskip{}

\chapter[I. A Human Life]{\centering I.\\
A HUMAN LIFE}

From the moment when he catches sight of the light of the world a man seeks to 
find out \textit{himself} and get hold of \textit{himself} out of its 
confusion, in which he, with everything else, is tossed about in motley 
mixture.

But everything that comes in contact with the child defends itself in turn 
against his attacks, and asserts its own persistence.

Accordingly, because each thing\textit{cares for itself} at the same time 
comes into constant collision with other things, the \textit{combat} of 
self-assertion is unavoidable.

\textit{Victory or defeat} -- between the two alternatives the fate of the 
combat wavers. The victor becomes the \textit{lord,} the vanquished one the 
\textit{subject}: the former exercises \textit{supremacy} and "{}rights of 
supremacy,"{} the latter fulfills in awe and deference the "{}duties of a 
subject.

But both remain \textit{enemies}, and always lie in wait: they watch for each 
other's \textit{weaknesses} -- children for those of their parents and parents 
for those of their children (\textit{e.g.,} their fear); either the stick 
conquers the man, or the man conquers the stick.

In childhood liberation takes the direction of trying to get to the bottom of 
things, to get at what is "{}back of"{} things; therefore we spy out the weak 
points of everybody, for which, it is well known, children have a sure 
instinct; therefore we like to smash things, like to rummage through hidden 
corners, pry after what is covered up or out of the way, and try what we can 
do with everything. When we once get at what is back of the things, we know we 
are safe; when, \textit{e.g.,} we have got at the fact that the rod is too 
weak against our obduracy, then we no longer fear it, "{}have out-grown it."{}

Back of the rod, mightier than it, stands our -- obduracy, our obdurate 
courage. By degrees we get at what is back of everything that was mysterious 
and uncanny to us, the mysteriously-dreaded might of the rod, the father's 
stern look, etc., and back of all we find our ataraxia, \textit{i. e.} 
imperturbability, intrepidity, our counter force, our odds of strength, our 
invincibility. Before that which formerly inspired in us fear and deference we 
no longer retreat shyly, but take \textit{courage}. Back of everything we find 
our \textit{courage}, our superiority; back of the sharp command of parents 
and authorities stands, after all, our courageous choice or our outwitting 
shrewdness. And the more we feel ourselves, the smaller appears that which 
before seemed invincible. And what is our trickery, shrewdness, courage, 
obduracy? What else but -- \textit{mind!}\footnote{\textit{Geist}. This word 
will be translated sometimes "{}mind"{} and sometimes "{}spirit"{} in the 
following pages.}

Through a considerable time we are spared a fight that is so exhausting later 
-- the fight against \textit{reason.} The fairest part of childhood passes 
without the necessity of coming to blows with reason. We care nothing at all 
about it, do not meddle with it, admit no reason. We are not to be persuaded 
to anything by \textit{conviction}, and are deaf to good arguments, 
principles, etc.; on the other hand, coaxing, punishment, etc. are hard for us 
to resist.

This stern life-and-death combat with \textit{reason} enters later, and begins 
a new phase; in childhood we scamper about without racking our brains much.

\textit{Mind} is the name of the \textit{first} self-discovery, the first 
self-discovery, the first undeification of the divine; \textit{i. e.}, of the 
uncanny, the spooks, the "{}powers above."{} Our fresh feeling of youth, this 
feeling of self, now defers to nothing; the world is discredited, for we are 
above it, we are \textit{mind}.

Now for the first time we see that hitherto we have not looked at the world 
\textit{intelligently} at all, but only stared at it.

We exercise the beginnings of our strength on \textit{natural powers}. We 
defer to parents as a natural power; later we say: Father and mother are to be 
forsaken, all natural power to be counted as riven. They are vanquished. For 
the rational, \textit{i.e.} the "{}intellectual"{} man, there is no family as 
a natural power; a renunciation of parents, brothers, etc., makes its 
appearance. If these are "{}born again"{} as \textit{intellectual, rational 
powers}, they are no longer at all what they were before.

And not only parents, but \textit{men in general}, are conquered by the young 
man; they are no hindrance to him, and are no longer regarded; for now he 
says: One must obey God rather than men.

From this high standpoint everything \textit{"{}earthly"{}} recedes into 
contemptible remoteness; for the standpoint is -- the \textit{heavenly}.

The attitude is now altogether reversed; the youth takes up an 
\textit{intellectual} position, while the boy, who did not yet feel himself as 
mind, grew up on mindless learning. The former does not try to get hold of 
\textit{things} (\textit{e.g.} to get into his head the \textit{data} of 
history), but of the \textit{thoughts} that lie hidden in things, and so, 
\textit{e.g.}, of the \textit{spirit} of history. On the other hand, the boy 
understands \textit{connections} no doubt, but not ideas, the spirit; 
therefore he strings together whatever can be learned, without proceeding 
\textit{a priori} and theoretically, \textit{i.e.} without looking for ideas.

As in childhood one had to overcome the resistance of the \textit{laws of the 
world}, so now in everything that he proposes he is met by an objection of the 
mind, of reason, of his \textit{own conscience}. "{}That is unreasonable, 
unchristian, unpatriotic,"{} etc., cries conscience to us, and -- frightens us 
away from it. Not the might of the avenging Eumenides, not Poseidon's wrath, 
not God, far as he sees the hidden, not the father's rod of punishment, do we 
fear, but -- \textit{conscience.}

We "{}run after our thoughts"{} now, and follow their commands just as before 
we followed parental, human ones. Our course of action is determined by our 
thoughts (ideas, conceptions, \textit{faith}) as it is in childhood by the 
commands of our parents.

For all that, we were already thinking when we were children, only our 
thoughts were not fleshless, abstract, \textit{absolute}, \textit{i. e.}, 
NOTHING BUT THOUGHTS, a heaven in themselves, a pure world of thought, 
\textit{logical} thoughts.

On the contrary, they had been only thoughts that we had about a 
\textit{thing}; we thought of the thing so or so. Thus we may have thought 
"{}God made the world that we see there,"{} but we did not think of 
("{}search"{}) the "{}depths of the Godhead itself"{}; we may have thought 
"{}that is the truth about the matter,"{} but we do not think of Truth itself, 
nor unite into one sentence "{}God is truth."{} The "{}depths of the Godhead, 
who is truth,"{} we did not touch. Over such purely logical, \textit{i.e.} 
theological questions, "{}What is truth?"{} Pilate does not stop, though he 
does not therefore hesitate to ascertain in an individual case "{}what truth 
there is in the thing,"{} \textit{i.e.} whether the \textit{thing} is true.

Any thought bound to a \textit{thing} is not yet \textit{nothing but a 
thought}, absolute thought.

To bring to light the \textit{pure thought}, or to be of its party, is the 
delight of youth; and all the shapes of light in the world of thought, like 
truth, freedom, humanity, Man, etc., illumine and inspire the youthful soul.

But, when the spirit is recognized as the essential thing, it still makes a 
difference whether the spirit is poor or rich, and therefore one seeks to 
become rich in spirit; the spirit wants to spread out so as to found its 
empire -- an empire that is not of this world, the world just conquered. Thus, 
then, it longs to become all in all to itself; \textit{i.e.}, although I am 
spirit, I am not yet \textit{perfected} spirit, and must first seek the 
complete spirit.

But with that I, who had just now found myself as spirit, lose myself again at 
once, bowing before the complete spirit as one not my own but 
\textit{supernal}, and feeling my emptiness.

Spirit is the essential point for everything, to be sure; but then is every 
spirit the "{}right"{} spirit? The right and true spirit is the ideal of 
spirit, the "{}Holy Spirit."{} It is not my or your spirit, but just -- an 
ideal, supernal one, it is "{}God."{} "{}God is spirit."{} And this supernal 
"{}Father in heaven gives it to those that pray to him."{}\footnote{Luke 11, 
13.}

The man is distinguished from the youth by the fact that he takes the world as 
it is, instead of everywhere fancying it amiss and wanting to improve it, 
\textit{i.e.} model it after his ideal; in him the view that one must deal 
with the world according to his \textit{interest,} not according to his 
\textit{ideals}, becomes confirmed.

So long as one knows himself only as \textit{spirit}, and feels that all the 
value of his existence consists in being spirit (it becomes easy for the youth 
to give his life, the "{}bodily life,"{} for a nothing, for the silliest point 
of honor), so long it is only \textit{thoughts} that one has, ideas that he 
hopes to be able to realize some day when he has found a sphere of action; 
thus one has meanwhile only \textit{ideals}, unexecuted ideas or thoughts.

Not till one has fallen in love with his \textit{corporeal} self, and takes a 
pleasure in himself as a living flesh-and-blood person -- but it is in mature 
years, in the man, that we find it so -- not till then has one a personal or 
\textit{egoistic} interest, \textit{i.e.} an interest not only of our spirit, 
\textit{e. g.}, but of total satisfaction, satisfaction of the whole chap, a 
\textit{selfish} interest. Just compare a man with a youth, and see if he will 
not appear to you harder, less magnanimous, more selfish. Is he therefore 
worse? No, you say; he has only become more definite, or, as you also call it, 
more "{}practical."{} But the main point is this, that he makes 
\textit{himself} more the center than does the youth, who is infatuated about 
other things, \textit{e.g.} God, fatherland, etc.

Therefore the man shows a \textit{second} self-discovery. The youth found 
himself as \textit{spirit} and lost himself again in the \textit{general} 
spirit, the complete, holy spirit, Man, mankind -- in short, all ideals; the 
man finds himself as \textit{embodied} spirit.

Boys had only \textit{unintellectual} interests (\textit{i.e.} interests 
devoid of thoughts and ideas), youths only \textit{intellectual} ones; the man 
has bodily, personal, egoistic interests.

If the child has not an \textit{object} that it can occupy itself with, it 
feels \textit{ennui}; for it does not yet know how to occupy itself with 
\textit{itself}. The youth, on the contrary, throws the object aside, because 
for him \textit{thoughts} arose out of the object; he occupies himself with 
his \textit{thoughts}, his dreams, occupies himself intellectually, or "{}his 
mind is occupied."{}

The young man includes everything not intellectual under the contemptuous name 
of "{}externalities."{} If he nevertheless sticks to the most trivial 
externalities (\textit{e.g.} the customs of students' clubs and other 
formalities), it is because, and when, he discovers \textit{mind} in them, 
\textit{i.e.} when they are \textit{symbols} to him.

As I find myself back of things, and that as mind, so I must later find 
\textit{myself} also back of \textit{thoughts} -- to wit, as their creator and 
owner. In the time of spirits thoughts grew till they overtopped my head, 
whose offspring they yet were; they hovered about me and convulsed me like 
fever-phantasies -- an awful power. The thoughts had become \textit{corporeal} 
on their own account, were ghosts, \textit{e. g.} God, Emperor, Pope, 
Fatherland, etc. If I destroy their corporeity, then I take them back into 
mine, and say: "{}I alone am corporeal."{} And now I take the world as what it 
is to me, as \textit{mine}, as my property; I refer all to myself.

If as spirit I had thrust away the world in the deepest contempt, so as owner 
I thrust spirits or ideas away into their "{}vanity."{} They have no longer 
any power over me, as no "{}earthly might"{} has power over the spirit.

The child was realistic, taken up with the things of this world, till little 
by little he succeeded in getting at what was back of these very things; the 
youth was idealistic, inspired by thoughts, till he worked his way up to where 
he became the man, the egoistic man, who deals with things and thoughts 
according to his heart's pleasure, and sets his personal interest above 
everything. Finally, the old man? When I become one, there will still be time 
enough to speak of that.

\chapter[II. Men Of The Old And The New]{\centering II.\\
MEN OF THE OLD TIME AND THE NEW}

\medskip{}

\noindent{}How each of us developed himself, what he strove for, attained, or 
missed, what objects he formerly pursued and what plans and wishes his heart 
is now set on, what transformation his views have experienced, what 
perturbations his principles -- in short, how he has today become what 
yesterday or years ago he was not -- this he brings out again from his memory 
with more or less ease, and he feels with especial vividness what changes have 
taken place in himself when he has before his eyes the unrolling of another's 
life.

Let us therefore look into the activities our forefathers busied themselves 
with.

\medskip{}

\section[1. The Ancients]{\centering 1. THE ANCIENTS}

Custom having once given the name of "{}the ancients"{} to our pre-Christian 
ancestors, we will not throw it up against them that, in comparison with us 
experienced people, they ought properly to be called children, but will rather 
continue to honor them as our good old fathers. But how have they come to be 
antiquated, and who could displace them through his pretended newness?

We know, of course, the revolutionary innovator and disrespectful heir, who 
even took away the sanctity of the fathers' sabbath to hallow his Sunday, and 
interrupted the course of time to begin at himself with a new chronology; we 
know him, and know that it is -- the Christian. But does he remain forever 
young, and is he today still the new man, or will he too be superseded, as he 
has superseded the "{}ancients"{}?

The fathers must doubtless have themselves begotten the young one who entombed 
them. Let us then peep at this act of generation.

"{}To the ancients the world was a truth,"{} says Feuerbach, but he forgets to 
make the important addition, "{}a truth whose untruth they tried to get back 
of, and at last really did."{} What is meant by those words of Feuerbach will 
be easily recognized if they are put alongside the Christian thesis of the 
"{}vanity and transitoriness of the world."{} For, as the Christian can never 
convince himself of the vanity of the divine word, but believes in its eternal 
and unshakable truth, which, the more its depths are searched, must all the 
more brilliantly come to light and triumph, so the ancients on their side 
lived in the feeling that the world and mundane relations (\textit{e.g.} the 
natural ties of blood) were the truth before which their powerless "{}I"{} 
must bow. The very thing on which the ancients set the highest value is 
spurned by Christians as the valueless, and what they recognized as truth 
these brand as idle lies; the high significance of the fatherland disappears, 
and the Christian must regard himself as "{}a stranger on 
earth"{};\footnote{Heb. 11. 13.} the sanctity of funeral rites, from which 
sprang a work of art like the Antigone of Sophocles, is designated as a paltry 
thing ("{}Let the dead bury their dead"{}); the infrangible truth of family 
ties is represented as an untruth which one cannot promptly enough get clear 
of;\footnote{Mark 10. 29.} and so in everything.

If we now see that to the two sides opposite things appear as truth, to one 
the natural, to the other the intellectual, to one earthly things and 
relations, to the other heavenly (the heavenly fatherland, "{}Jerusalem that 
is above,"{} etc.), it still remains to be considered how the new time and 
that undeniable reversal could come out of antiquity. But the ancients 
themselves worked toward making their truth a lie.

Let us plunge at once into the midst of the most brilliant years of the 
ancients, into the Periclean century. Then the Sophistic culture was 
spreading, and Greece made a pastime of what had hitherto been to her a 
monstrously serious matter.

The fathers had been enslaved by the undisturbed power of existing things too 
long for the posterity not to have to learn by bitter experience to 
\textit{feel themselves}. Therefore the Sophists, with courageous sauciness, 
pronounce the reassuring words, "{}Don't be bluffed!"{} and diffuse the 
rationalistic doctrine, "{}Use your understanding, your wit, your mind, 
against everything; it is by having a good and well-drilled understanding that 
one gets through the world best, provides for himself the best lot, the most 
pleasant \textit{life."{}} Thus they recognize in \textit{mind} man's true 
weapon against the world. This is why they lay such stress on dialectic skill, 
command of language, the art of disputation, etc. They announce that mind is 
to be used against everything; but they are still far removed from the 
holiness of the Spirit, for to them it is a \textit{means}, a weapon, as 
trickery and defiance serve children for the same purpose; their mind is the 
unbribable \textit{understanding}.

Today we should call that a one-sided culture of the understanding, and add 
the warning, "{}Cultivate not only your understanding, but also, and 
especially, your heart."{} Socrates did the same. For, if the heart did not 
become free from its natural impulses, but remained filled with the most 
fortuitous contents and, as an uncriticized \textit{avidity}, altogether in 
the power of things, \textit{i.e.} nothing but a vessel of the most various 
\textit{appetites} -- then it was unavoidable that the free understanding must 
serve the "{}bad heart"{} and was ready to justify everything that the wicked 
heart desired.

Therefore Socrates says that it is not enough for one to use his understanding 
in all things, but it is a question of what \textit{cause} one exerts it for. 
We should now say, one must serve the "{}good cause."{} But serving the good 
cause is -- being moral. Hence Socrates is the founder of ethics.

Certainly the principle of the Sophistic doctrine must lead to the possibility 
that the blindest and most dependent slave of his desires might yet be an 
excellent sophist, and, with keen understanding, trim and expound everything 
in favor of his coarse heart. What could there be for which a "{}good 
reason"{} might not be found, or which might not be defended through thick and 
thin?

Therefore Socrates says: "{}You must be 'pure-hearted' if your shrewdness is 
to be valued."{} At this point begins the second period of Greek liberation of 
the mind, the period of \textit{purity of heart}. For the first was brought to 
a close by the Sophists in their proclaiming the omnipotence of the 
understanding. But the heart remained \textit{worldly-minded}, remained a 
servant of the world, always affected by worldly wishes. This coarse heart was 
to be cultivated from now on -- the era of \textit{culture of the heart}. But 
how is the heart to be cultivated? What the understanding; this one side of 
the mind, has reached -- to wit, the capability of playing freely with and 
over every concern -- awaits the heart also; everything \textit{worldly} must 
come to grief before it, so that at last family, commonwealth, fatherland, 
etc., are given up for the sake of the heart, \textit{i. e.}, \textit{of 
blessedness}, the heart's blessedness.

Daily experience confirms the truth that the understanding may have renounced 
a thing many years before the heart has ceased to beat for it. So the 
Sophistic understanding too had so far become master over the dominant, 
ancient powers that they now needed only to be driven out of the heart, in 
which they dwelt unmolested, to have at last no part at all left in man. This 
war is opened by Socrates, and not till the dying day of the old world does it 
end in peace.

The examination of the heart takes its start with Socrates, and all the 
contents of the heart are sifted. In their last and extremest struggles the 
ancients threw all contents out of the heart and let it no longer beat for 
anything; this was the deed of the Skeptics. The same purgation of the heart 
was now achieved in the Skeptical age, as the understanding had succeeded in 
establishing in the Sophistic age.

The Sophistic culture has brought it to pass that one's understanding no 
longer \textit{stands still} before anything, and the Skeptical, that his 
heart is no longer \textit{moved} by anything.

So long as man is entangled in the movements of the world and embarrassed by 
relations to the world -- and he is so till the end of antiquity, because his 
heart still has to struggle for independence from the worldly -- so long he is 
not yet spirit; for spirit is without body, and has no relations to the world 
and corporeality; for it the world does not exist, nor natural bonds, but only 
the spiritual, and spiritual bonds. Therefore man must first become so 
completely unconcerned and reckless, so altogether without relations, as the 
Skeptical culture presents him -- so altogether indifferent to the world that 
even its falling in ruins would not move him -- before he could feel himself 
as worldless; \textit{i. e.}, as spirit. And this is the result of the 
gigantic work of the ancients: that man knows himself as a being without 
relations and without a world, as \textit{spirit}.

Only now, after all worldly care has left him, is he all in all to himself, is 
he only for himself, \textit{i.e.} he is he spirit for the spirit, or, in 
plainer language, he cares only for the spiritual.

In the Christian wisdom of serpents and innocence of doves the two sides -- 
understanding and heart -- of the ancient liberation of mind are so completed 
that they appear young and new again, and neither the one nor the other lets 
itself be bluffed any longer by the worldly and natural.

Thus the ancients mounted to \textit{spirit}, and strove to become 
\textit{spiritual}. But a man who wishes to be active as spirit is drawn to 
quite other tasks than he was able to set himself formerly: to tasks which 
really give something to do to the spirit and not to mere sense or 
acuteness,\footnote{Italicized in the original for the sake of its etymology, 
\textit{Scharfsinn} -- "{}sharp-sense"{}. Compare next paragraph.} which 
exerts itself only to become master \textit{of things}. The spirit busies 
itself solely about the spiritual, and seeks out the "{}traces of mind"{} in 
everything; to the \textit{believing} spirit "{}everything comes from God,"{} 
and interests him only to the extent that it reveals this origin; to the 
\textit{philosophic} spirit everything appears with the stamp of reason, and 
interests him only so far as he is able to discover in it reason, \textit{i. 
e.}, spiritual content.

Not the spirit, then, which has to do with absolutely nothing unspiritual, 
with no \textit{thing}, but only with the essence which exists behind and 
above things, with \textit{thoughts --} not that did the ancients exert, for 
they did not yet have it; no, they had only reached the point of struggling 
and longing for it, and therefore sharpened it against their too-powerful foe, 
the world of sense (but what would not have been sensuous for them, since 
Jehovah or the gods of the heathen were yet far removed from the conception 
"{}God is \textit{spirit},"{} since the "{}heavenly fatherland"{} had not yet 
stepped into the place of the sensuous, etc.?) -- they sharpened against the 
world of sense their \textit{sense}, their acuteness. To this day the Jews, 
those precocious children of antiquity, have got no farther; and with all the 
subtlety and strength of their prudence and understanding, which easily 
becomes master of things and forces them to obey it, they cannot discover 
\textit{spirit}, which \textit{takes no account whatever of things}.

The Christian has spiritual interests, because he allows himself to be a 
\textit{spiritual} man; the Jew does not even understand these interests in 
their purity, because he does not allow himself to assign \textit{no value} to 
things. He does not arrive at pure \textit{spirituality}, a spirituality 
\textit{e. g.} is religiously expressed, \textit{e. g.}, in the \textit{faith} 
of Christians, which alone (\textit{i.e.} without works) justifies. Their 
\textit{unspirituality} sets Jews forever apart from Christians; for the 
spiritual man is incomprehensible to the unspiritual, as the unspiritual is 
contemptible to the spiritual. But the Jews have only "{}the spirit of this 
world."{}

The ancient acuteness and profundity lies as far from the spirit and the 
spirituality of the Christian world as earth from heaven.

He who feels himself as free spirit is not oppressed and made anxious by the 
things of this world, because he does not care for them; if one is still to 
feel their burden, he must be narrow enough to attach \textit{weight} to them 
-- as is evidently the case, \textit{e. g.}, when one is still concerned for 
his "{}dear life."{} He to whom everything centers in knowing and conducting 
himself as a free spirit gives little heed to how scantily he is supplied 
meanwhile, and does not reflect at all on how he must make his arrangements to 
have a thoroughly inconveniences of the life that depends on things, because 
he lives only spiritually and on spiritual food, while aside from this he only 
gulps things down like a beast, hardly knowing it, and dies bodily, to be 
sure, when his fodder gives out, but knows himself immortal as spirit, and 
closes his eyes with an adoration or a thought. His life is occupation with 
the spiritual, is -- thinking; the rest does not bother him; let him busy 
himself with the spiritual in any way that he can and chooses -- in devotion, 
in contemplation, or in philosophic cognition -- his doing is always thinking; 
and therefore Descartes, to whom this had at last become quite clear, could 
lay down the proposition: "{}I think, that is -- I am."{} This means, my 
thinking is my being or my life; only when I live spiritually do I live; only 
as spirit am I really, or -- I am spirit through and through and nothing but 
spirit. Unlucky Peter Schlemihl, who has lost his shadow, is the portrait of 
this man become a spirit; for the spirit's body is shadowless. -- Over against 
this, how different among the ancients! Stoutly and manfully as they might 
bear themselves against the might of things, they must yet acknowledge the 
might itself, and got no farther than to protect their \textit{life} against 
it as well as possible. Only at a late hour did they recognize that their 
"{}true life"{} was not that which they led in the fight against the things of 
the world, but the "{}spiritual life,"{} "{}turned away"{} from these things; 
and, when they saw this, they became Christians, \textit{i.e.} the moderns, 
and innovators upon the ancients. But the life turned away from things, the 
spiritual life, no longer draws any nourishment from nature, but "{}lives only 
on thoughts,"{} and therefore is no longer "{}life,"{} but -- 
\textit{thinking}.

Yet it must not be supposed now that the ancients were \textit{without 
thoughts}, just as the most spiritual man is not to be conceived of as if he 
could be without life. Rather, they had their thoughts about everything, about 
the world, man, the gods, etc., and showed themselves keenly active in 
bringing all this to their consciousness. But they did not know 
\textit{thought}, even though they thought of all sorts of things and 
"{}worried themselves with their thoughts."{} Compare with their position the 
Christian saying, "{}My thoughts are not your thoughts; as the heaven is 
higher than the earth, so are my thoughts higher than your thoughts,"{} and 
remember what was said above about our child-thoughts.

What is antiquity seeking, then? The true \textit{enjoyment of life!} You will 
find that at bottom it is all the same as "{}the true life."{}

The Greek poet Simonides sings: "{}Health is the noblest good for mortal man, 
the next to this is beauty, the third riches acquired without guile, the 
fourth the enjoyment of social pleasures in the company of young friends."{} 
These are all \textit{good things of life}, pleasures of life. What else was 
Diogenes of Sinope seeking for than the true enjoyment of life, which he 
discovered in having the least possible wants? What else Aristippus, who found 
it in a cheery temper under all circumstances? They are seeking for cheery, 
unclouded \textit{life-courage}, for \textit{cheeriness}; they are seeking to 
"{}be of good \textit{cheer}."{}

The Stoics want to realize the \textit{wise man}, the man with 
\textit{practical philosophy}, the man who \textit{knows how to live --} a 
wise life, therefore; they find him in contempt for the world, in a life 
without development, without spreading out, without friendly relations with 
the world, thus in the \textit{isolated life}, in life as life, not in life 
with others; only the Stoic \textit{lives}, all else is dead for him. The 
Epicureans, on the contrary, demand a moving life.

The ancients, as they want to be of good cheer, desire \textit{good living} 
(the Jews especially a long life, blessed with children and goods), 
\textit{eudaemonia}, well-being in the most various forms. Democritus, 
\textit{e. g.}, praises as such the "{}calm of the soul"{} in which one 
\textit{"{}lives} smoothly, without fear and without excitement."{}

So what he thinks is that with this he gets on best, provides for himself the 
best lot, and gets through the world best. But as he cannot get rid of the 
world -- and in fact cannot for the very reason that his whole activity is 
taken up in the effort to get rid of it, \textit{i. e.}, in \textit{repelling 
the world} (for which it is yet necessary that what can be and is repelled 
should remain existing, otherwise there would be no longer anything to repel) 
-- he reaches at most an extreme degree of liberation, and is distinguishable 
only in degree from the less liberated. If he even got as far as the deadening 
of the earthly sense, which at last admits only the monotonous whisper of the 
word "{}Brahm,"{} he nevertheless would not be essentially distinguishable 
from the \textit{sensual} man.

Even the stoic attitude and manly virtue amounts only to this -- that one must 
maintain and assert himself against the world; and the ethics of the Stoics 
(their only science, since they could tell nothing about the spirit but how it 
should behave toward the world, and of nature (physics) only this, that the 
wise man must assert himself against it) is not a doctrine of the spirit, but 
only a doctrine of the repelling of the world and of self-assertion against 
the world. And this consists in "{}imperturbability and equanimity of life,"{} 
and so in the most explicit Roman virtue.

The Romans too (Horace, Cicero, etc.) went no further than this 
\textit{practical philosophy}.

The \textit{comfort} (\textit{hedone}) of the Epicureans is the same 
\textit{practical philosophy} the Stoics teach, only trickier, more deceitful. 
They teach only another \textit{behavior} toward the world, exhort us only to 
take a shrewd attitude toward the world; the world must be deceived, for it is 
my enemy.

The break with the world is completely carried through by the Skeptics. My 
entire relation to the world is "{}worthless and truthless."{} Timon says, 
"{}The feelings and thoughts which we draw from the world contain no truth."{} 
"{}What is truth?"{} cries Pilate. According to Pyrrho's doctrine the world is 
neither good nor bad, neither beautiful nor ugly, etc., but these are 
predicates which I give it. Timon says that "{}in itself nothing is either 
good or bad, but man only \textit{thinks} of it thus or thus"{}; to face the 
world only \textit{ataraxia} (unmovedness) and \textit{aphasia} 
(speechlessness -- or, in other words, isolated \textit{inwardness)} are left. 
There is "{}no longer any truth to be recognized"{} in the world; things 
contradict themselves; thoughts about things are without distinction (good and 
bad are all the same, so that what one calls good another finds bad); here the 
recognition of "{}truth"{} is at an end, and only the man \textit{without 
power of recognition}, the \textit{man} who finds in the world nothing to 
recognize, is left, and this man just leaves the truth-vacant world where it 
is and takes no account of it.

So antiquity gets through with the \textit{world of things}, the order of the 
world, the world as a whole; but to the order of the world, or the things of 
this world, belong not only nature, but all relations in which man sees 
himself placed by nature, \textit{e. g.} the family, the community -- in 
short, the so-called "{}natural bonds."{} With the \textit{world of the 
spirit} Christianity then begins. The man who still faces the world 
\textit{armed} is the ancient, the -- \textit{heathen} (to which class the 
Jew, too, as non-Christian, belongs); the man who has come to be led by 
nothing but his "{}heart's pleasure,"{} the interest he takes, his 
fellow-feeling, his --\textit{spirit}, is the modern, the -- Christian.

As the ancients worked toward the \textit{conquest of the world} and strove to 
release man from the heavy trammels of connection with \textit{other things}, 
at last they came also to the dissolution of the State and giving preference 
to everything private. Of course community, family, etc., as \textit{natural} 
relations, are burdensome hindrances which diminish my \textit{spiritual 
freedom.}

\medskip{}

\section[2. The Moderns]{\centering 2. THE MODERNS}

"{}If any man be in Christ, he is a \textit{new creature}; the old is passed 
away, behold, all is become new."{}\footnote{2 Cor. 5. 17. [The words 
"{}new"{} and "{}modern"{} are the same in German.]}

As it was said above, "{}To the ancients the world was a truth,"{} we must say 
here, "{}To the moderns the spirit was a truth"{}; but here, as there, we must 
not omit the supplement, "{}a truth whose untruth they tried to get back of, 
and at last they really do."{}

A course similar to that which antiquity took may be demonstrated in 
Christianity also, in that the \textit{understanding} was held a prisoner 
under the dominion of the Christian dogmas up to the time preparatory to the 
Reformation, but in the pre-Reformation century asserted itself 
\textit{sophistically} and played heretical pranks with all tenets of the 
faith. And the talk then was, especially in Italy and at the Roman court, 
"{}If only the heart remains Christian-minded, the understanding may go right 
on taking its pleasure."{}

Long before the Reformation, people were so thoroughly accustomed to fine-spun 
"{}wranglings"{} that the pope, and most others, looked on Luther's appearance 
too as a mere "{}wrangling of monks"{} at first. Humanism corresponds to 
Sophisticism, and, as in the time of the Sophists Greek life stood in its 
fullest bloom (the Periclean age), so the most brilliant things happened in 
the time of Humanism, or, as one might perhaps also say, of Machiavellianism 
(printing, the New World, etc.). At this time the heart was still far from 
wanting to relieve itself of its Christian contents.

But finally the Reformation, like Socrates, took hold seriously of the 
\textit{heart} itself, and since then hearts have kept growing visibly -- more 
unchristian. As with Luther people began to take the matter to heart, the 
outcome of this step of the Reformation must be that the heart also gets 
lightened of the heavy burden of Christian faith. The heart, from day to day 
more unchristian, loses the contents with which it had busied itself, till at 
last nothing but empty \textit{warmheartedness} is left it, the quite general 
love of men, the love of \textit{Man}, the consciousness of freedom, 
"{}self-consciousness."{}

Only so is Christianity complete, because it has become bald, withered, and 
void of contents. There are now no contents whatever against which the heart 
does not mutiny, unless indeed the heart unconsciously or without "{}self- 
consciousness"{} lets them slip in. The heart \textit{criticises} to death 
with \textit{hard-hearted} mercilessness everything that wants to make its way 
in, and is capable (except, as before, unconsciously or taken by surprise) of 
no friendship, no love. What could there be in men to love, since they are all 
alike "{}egoists,"{} none of them man as such, \textit{i.e.} none 
\textit{spirit only}? The Christian loves only the spirit; but where could one 
be found who should be really nothing but spirit?

To have a liking for the corporeal man with hide and hair -- why, that would 
no longer be a "{}spiritual"{} warmheartedness, it would be treason against 
"{}pure"{} warmheartedness, the "{}theoretical regard."{} For pure 
warmheartedness is by no means to be conceived as like that kindliness that 
gives everybody a friendly hand-shake; on the contrary, pure warmheartedness 
is warm-hearted toward nobody, it is only a theoretical interest, concern for 
man as man, not as a person. The person is repulsive to it because of being 
"{}egoistic,"{} because of not being that abstraction, Man. But it is only for 
the abstraction that one can have a theoretical regard. To pure 
warmheartedness or pure theory men exist only to be criticized, scoffed at, 
and thoroughly despised; to it, no less than to the fanatical parson, they are 
only "{}filth"{} and other such nice things.

Pushed to this extremity of disinterested warmheartedness, we must finally 
become conscious that the spirit, which alone the Christian loves, is nothing; 
in other words, that the spirit is -- a lie.

What has here been set down roughly, summarily, and doubtless as yet 
incomprehensibly, will, it is to be hoped, become clear as we go on.

Let us take up the inheritance left by the ancients, and, as active workmen, 
do with it as much as -- can be done with it! The world lies despised at our 
feet, far beneath us and our heaven, into which its mighty arms are no longer 
thrust and its stupefying breath does not come. Seductively as it may pose, it 
can delude nothing but our \textit{sense}; it cannot lead astray the spirit -- 
and spirit alone, after all, we really are. Having once got \textit{back of} 
things, the spirit has also got \textit{above} them, and become free from 
their bonds, emancipated, supernal, free. So speaks "{}spiritual freedom."{}

To the spirit which, after long toil, has got rid of the world, the worldless 
spirit, nothing is left after the loss of the world and the worldly but -- the 
spirit and the spiritual.

Yet, as it has only moved away from the world and made of itself a being 
\textit{free from the world}, without being able really to annihilate the 
world, this remains to it a stumbling-block that cannot be cleared away, a 
discredited existence; and, as, on the other hand, it knows and recognizes 
nothing but the spirit and the spiritual, it must perpetually carry about with 
it the longing to spiritualize the world, \textit{i.e.} to redeem it from the 
"{}black list."{} Therefore, like a youth, it goes about with plans for the 
redemption or improvement of the world.

The ancients, we saw, served the natural, the worldly, the natural order of 
the world, but they incessantly asked themselves of this service; and, when 
they had tired themselves to death in ever-renewed attempts at revolt, then, 
among their last sighs, was born to them the \textit{God}, the "{}conqueror of 
the world."{} All their doing had been nothing but \textit{wisdom of the 
world}, an effort to get back of the world and above it. And what is the 
wisdom of the many following centuries? What did the moderns try to get back 
of? No longer to get back of the world, for the ancients had accomplished 
that; but back of the God whom the ancients bequeathed to them, back of the 
God who "{}is spirit,"{} back of everything that is the spirit's, the 
spiritual. But the activity of the spirit, which "{}searches even the depths 
of the Godhead,"{} is \textit{theology}. If the ancients have nothing to show 
but wisdom of the world, the moderns never did nor do make their way further 
than to theology. We shall see later that even the newest revolts against God 
are nothing but the extremest efforts of "{}theology,"{} \textit{i. e.}, 
theological insurrections.

\subsection[\S{}1. The Spirit]{\centering \S{}1. The Spirit}

The realm of spirits is monstrously great, there is an infinite deal of the 
spiritual; yet let us look and see what the spirit, this bequest of the 
ancients, properly is.

Out of their birth-pangs it came forth, but they themselves could not utter 
themselves as spirit; they could give birth to it, it itself must speak. The 
"{}born God, the Son of Man,"{} is the first to utter the word that the 
spirit, \textit{i.e.} he, God, has to do with nothing earthly and no earthly 
relationship, but solely, with the spirit and spiritual relationships.

Is my courage, indestructible under all the world's blows, my inflexibility 
and my obduracy, perchance already spirit in the full sense, because the world 
cannot touch it? Why, then it would not yet be at enmity with the world, and 
all its action would consist merely in not succumbing to the world! No, so 
long as it does not busy itself with itself alone, so long as it does not have 
to do with \textit{its} world, the spiritual, alone, it is not \textit{free} 
spirit, but only the "{}spirit of this world,"{} the spirit fettered to it. 
The spirit is free spirit, \textit{i. e.}, really spirit, only in a world of 
\textit{its own}; in "{}this,"{} the earthly world, it is a stranger. Only 
through a spiritual world is the spirit really spirit, for "{}this"{} world 
does not understand it and does not know how to keep "{}the maiden from a 
foreign land"{}\footnote{[Title of a poem by Schiller]} from departing.

But where is it to get this spiritual world? Where but out of itself? It must 
reveal itself; and the words that it speaks, the revelations in which it 
unveils itself, these are \textit{its} world. As a visionary lives and has his 
world only in the visionary pictures that he himself creates, as a crazy man 
generates for himself his own dream-world, without which he could not be 
crazy, so the spirit must create for itself its spirit world, and is not 
spirit till it creates it.

Thus its creations make it spirit, and by its creatures we know it, the 
creator; in them it lives, they are its world.

Now, what is the spirit? It is the creator of a spiritual world! Even in you 
and me people do not recognize spirit till they see that we have appropriated 
to ourselves something spiritual, -- \textit{i.e.} though thoughts may have 
been set before us, we have at least brought them to live in ourselves; for, 
as long as we were children, the most edifying thoughts might have been laid 
before us without our wishing, or being able, to reproduce them in ourselves. 
So the spirit also exists only when it creates something spiritual; it is real 
only together with the spiritual, its creature.

As, then, we know it by its works, the question is what these works are. But 
the works or children of the spirit are nothing else but -- spirits.

If I had before me Jews, Jews of the true metal, I should have to stop here 
and leave them standing before this mystery as for almost two thousand years 
they have remained standing before it, unbelieving and without knowledge. But, 
as you, my dear reader, are at least not a full-blooded Jew -- for such a one 
will not go astray as far as this -- we will still go along a bit of road 
together, till perhaps you too turn your back on me because I laugh in your 
face.

If somebody told you were altogether spirit, you would take hold of your body 
and not believe him, but answer: "{}I \textit{have} a spirit, no doubt, but do 
not exist only as spirit, but as a man with a body."{} You would still 
distinguish \textit{yourself} from "{}your spirit."{} "{}But,"{} replies he, 
"{}it is your destiny, even though now you are yet going about in the fetters 
of the body, to be one day a 'blessed spirit,' and, however you may conceive 
of the future aspect of your spirit, so much is yet certain, that in death you 
will put off this body and yet keep yourself, \textit{i.e.} your spirit, for 
all eternity; accordingly your spirit is the eternal and true in you, the body 
only a dwelling here below, which you may leave and perhaps exchange for 
another."{}

Now you believe him! For the present, indeed, you are not spirit only; but, 
when you emigrate from the mortal body, as one day you must, then you will 
have to help yourself without the body, and therefore it is needful that you 
be prudent and care in time for your proper self. "{}What should it profit a 
man if he gained the whole world and yet suffered damage in his soul?"{}

But, even granted that doubts, raised in the course of time against the tenets 
of the Christian faith, have long since robbed you of faith in the immortality 
of your spirit, you have nevertheless left one tenet undisturbed, and still 
ingenuously adhere to the one truth, that the spirit is your better part, and 
that the spiritual has greater claims on you than anything else. Despite all 
your atheism, in zeal against \textit{egoism} you concur with the believers in 
immortality.

But whom do you think of under the name of egoist? A man who, instead of 
living to an idea, \textit{i. e.}, a spiritual thing, and sacrificing to it 
his personal advantage, serves the latter. A good patriot brings his sacrifice 
to the altar of the fatherland; but it cannot be disputed that the fatherland 
is an idea, since for beasts incapable of mind,\footnote{[The reader will 
remember (it is to be hoped has never forgotten) that "{}mind"{} and 
"{}spirit"{} are one and the same word in German. For several pages back the 
connection of the discourse has seemed to require the almost exclusive use of 
the translation "{}spirit,"{} but to complete the sense it has often been 
necessary that the reader recall the thought of its identity with "{}mind,"{} 
as stated in a previous note.]} or children as yet without mind, there is no 
fatherland and no patriotism. Now, if any one does not approve himself as a 
good patriot, he betrays his egoism with reference to the fatherland. And so 
the matter stands in innumerable other cases: he who in human society takes 
the benefit of a prerogative sins egoistically against the idea of equality; 
he who exercises dominion is blamed as an egoist against the idea of liberty, 
-- etc.

You despise the egoist because he puts the spiritual in the background as 
compared with the personal, and has his eyes on himself where you would like 
to see him act to favor an idea. The distinction between you is that he makes 
himself the central point, but you the spirit; or that you cut your identity 
in two and exalt your "{}proper self,"{} the spirit, to be ruler of the 
paltrier remainder, while he will hear nothing of this cutting in two, and 
pursues spiritual and material interests just \textit{as he pleases}. You 
think, to be sure, that you are falling foul of those only who enter into no 
spiritual interest at all, but in fact you curse at everybody who does not 
look on the spiritual interest as his "{}true and highest"{} interest. You 
carry your knightly service for this beauty so far that you affirm her to be 
the only beauty of the world. You live not to \textit{yourself}, but to your 
\textit{spirit} and to what is the spirit's, \textit{i. e.} ideas.

As the spirit exists only in its creating of the spiritual, let us take a look 
about us for its first creation. If only it has accomplished this, there 
follows thenceforth a natural propagation of creations, as according to the 
myth only the first human beings needed to be created, the rest of the race 
propagating of itself. The first creation, on the other hand, must come forth 
"{}out of nothing"{} -- \textit{i.e.} the spirit has toward its realization 
nothing but itself, or rather it has not yet even itself, but must create 
itself; hence its first creation is itself, \textit{the spirit}. Mystical as 
this sounds, we yet go through it as an every-day experience. Are you a 
thinking being before you think? In creating the first thought you create 
yourself, the thinking one; for you do not think before you think a thought, 
\textit{i.e.} have a thought. Is it not your singing that first makes you a 
singer, your talking that makes you a talker? Now, so too it is the production 
of the spiritual that first makes you a spirit.

Meantime, as you distinguish \textit{yourself} from the thinker, singer, and 
talker, so you no less distinguish yourself from the spirit, and feel very 
clearly that you are something beside spirit. But, as in the thinking ego 
hearing and sight easily vanish in the enthusiasm of thought, so you also have 
been seized by the spirit-enthusiasm, and you now long with all your might to 
become wholly spirit and to be dissolved in spirit. The spirit is your 
\textit{ideal}, the unattained, the other-worldly; spirit is the name of your 
-- god, "{}God is spirit."{}

Against all that is not spirit you are a zealot, and therefore you play the 
zealot against \textit{yourself} who cannot get rid of a remainder of the 
non-spiritual. Instead of saying, "{}I am \textit{more} than spirit,"{} you 
say with contrition, "{}I am less than spirit; and spirit, pure spirit, or the 
spirit that is nothing but spirit, I can only think of, but am not; and, since 
I am not it, it is another, exists as another, whom I call 'God'."{}

It lies in the nature of the case that the spirit that is to exist as pure 
spirit must be an otherworldly one, for, since I am not it, it follows that it 
can only be \textit{outside} me; since in any case a human being is not fully 
comprehended in the concept "{}spirit,"{} it follows that the pure spirit, the 
spirit as such, can only be outside of men, beyond the human world -- not 
earthly, but heavenly.

Only from this disunion in which I and the spirit lie; only because "{}I"{} 
and "{}spirit"{} are not names for one and the same thing, but different names 
for completely different things; only because I am not spirit and spirit not I 
-- only from this do we get a quite tautological explanation of the necessity 
that the spirit dwells in the other world, \textit{i. e.} is God.

But from this it also appears how thoroughly theological is the liberation 
that Feuerbach\footnote{"{}Essence of Christianity"{}} is laboring to give us. 
What he says is that we had only mistaken our own essence, and therefore 
looked for it in the other world, but that now, when we see that God was only 
our human essence, we must recognize it again as ours and move it back out of 
the other world into this. To God, who is spirit, Feuerbach gives the name 
"{}Our Essence."{} Can we put up with this, that "{}Our Essence"{} is brought 
into opposition to \textit{us} -- that we are split into an essential and an 
unessential self? Do we not therewith go back into the dreary misery of seeing 
ourselves banished out of ourselves?

What have we gained, then, when for a variation we have transferred into 
ourselves the divine outside us? \textit{Are we} that which is in us? As 
little as we are that which is outside us. I am as little my heart as I am my 
sweetheart, this "{}other self"{} of mine. Just because we are not the spirit 
that dwells in us, just for that reason we had to take it and set it outside 
us; it was not we, did not coincide with us, and therefore we could, not think 
of it as existing otherwise than outside us, on the other side from us, in the 
other world.

With the strength of \textit{despair} Feuerbach clutches at the total 
substance of Christianity, not to throw it away, no, to drag it to himself, to 
draw it, the long-yearned-for, ever-distant, out of its heaven with a last 
effort, and keep it by him forever. Is not that a clutch of the uttermost 
despair, a clutch for life or death, and is it not at the same time the 
Christian yearning and hungering for the other world? The hero wants not to go 
into the other world, but to draw the other world to him, and compel it to 
become this world! And since then has not all the world, with more or less 
consciousness, been crying that "{}this world"{} is the vital point, and 
heaven must come down on earth and be experienced even here?

Let us, in brief, set Feuerbach's theological view and our contradiction over 
against each other! "{}The essence of man is man's supreme 
being;\footnote{[Or, "{}highest essence."{} The word \textit{Wesen}, which 
means both "{}essence"{} and "{}being,"{} will be translated now one way and 
now the other in the following pages. The reader must bear in mind that these 
two words are identical in German; and so are "{}supreme"{} and 
"{}highest."{}]} now by religion, to be sure, the \textit{supreme being is} 
called \textit{God} and regarded as an objective essence, but in truth it is 
only man's own essence; and therefore the turning point of the world's history 
is that henceforth no longer \textit{God}, but man, is to appear to man as 
God."{}\footnote{Cf. \textit{e. g.} "{}Essence of Christianity"{}, p. 402.}

To this we reply: The supreme being is indeed the essence of man, but, just 
because it is his \textit{essence} and not he himself, it remains quite 
immaterial whether we see it outside him and view it as "{}God,"{} or find it 
in him and call it "{}Essence of Man"{} or "{}Man."{} I am neither God nor 
Man,\footnote{[That is, the abstract conception of man, as in the preceding 
sentence.]} neither the supreme essence nor my essence, and therefore it is 
all one in the main whether I think of the essence as in me or outside me. 
Nay, we really do always think of the supreme being as in both kinds of 
otherworldliness, the inward and outward, at once; for the "{}Spirit of God"{} 
is, according to the Christian view, also "{}our spirit,"{} and "{}dwells in 
us."{}\footnote{\textit{E.g.}Rom. 8. 9, 1 Cor. 3. 16, John 20. 22 and 
innumerable other passages.} It dwells in heaven and dwells in us; we poor 
things are just its "{}dwelling,"{} and, if Feuerbach goes on to destroy its 
heavenly dwelling and force it to move to us bag and baggage, then we, its 
earthly apartments, will be badly overcrowded.

But after this digression (which, if we were at all proposing to work by line 
and level, we should have had to save for later pages in order to avoid 
repetition) we return to the spirit's first creation, the spirit itself.

The spirit is something other than myself. But this other, what is it?

\medskip{}

\subsection[\S{}2. The Possessed]{\centering \S{}2. The Possessed.}

Have you ever seen a spirit? "{}No, not I, but my grandmother."{} Now, you 
see, it's just so with me too; I myself haven't seen any, but my grandmother 
had them running between her feet all sorts of ways, and out of confidence in 
our grandmothers' honesty we believe in the existence of spirits.

But had we no grandfathers then, and did they not shrug their shoulders every 
time our grandmothers told about their ghosts? Yes, those were unbelieving men 
who have harmed our good religion much, those rationalists! We shall feel 
that! What else lies at the bottom of this warm faith in ghosts, if not the 
faith in "{}the existence of spiritual beings in general,"{} and is not this 
latter itself disastrously unsettled if saucy men of the understanding may 
disturb the former? The Romanticists were quite conscious what a blow the very 
belief in God suffered by the laying aside of the belief in spirits or ghosts, 
and they tried to help us out of the baleful consequences not only by their 
reawakened fairy world, but at last, and especially, by the "{}intrusion of a 
higher world,"{} by their somnambulists of Prevorst, etc. The good believers 
and fathers of the church did not suspect that with the belief in ghosts the 
foundation of religion was withdrawn, and that since then it had been floating 
in the air. He who no longer believes in any ghost needs only to travel on 
consistently in his unbelief to see that there is no separate being at all 
concealed behind things, no ghost or -- what is naively reckoned as synonymous 
even in our use of words -- no \textit{"{}spirit."{}}

"{}Spirits exist!"{} Look about in the world, and say for yourself whether a 
spirit does not gaze upon you out of everything. Out of the lovely little 
flower there speaks to you the spirit of the Creator, who has shaped it so 
wonderfully; the stars proclaim the spirit that established their order; from 
the mountain-tops a spirit of sublimity breathes down; out of the waters a 
spirit of yearning murmurs up; and -- out of men millions of spirits speak. 
The mountains may sink, the flowers fade, the world of stars fall in ruins, 
the men die -- what matters the wreck of these visible bodies? The spirit, the 
"{}invisible spirit,"{} abides eternally!

Yes, the whole world is haunted! Only is haunted? Nay, it itself "{}walks,"{} 
it is uncanny through and through, it is the wandering seeming-body of a 
spirit, it is a spook. What else should a ghost be, then, than an apparent 
body, but real spirit? Well, the world is "{}empty,"{} is "{}naught,"{} is 
only glamorous "{}semblance"{}; its truth is the spirit alone; it is the 
seeming-body of a spirit.

Look out near or far, a \textit{ghostly} world surrounds you everywhere; you 
are always having "{}apparitions"{} or visions. Everything that appears to you 
is only the phantasm of an indwelling spirit, is a ghostly "{}apparition"{}; 
the world is to you only a "{}world of appearances,"{} behind which the spirit 
walks. You "{}see spirits."{}

Are you perchance thinking of comparing yourself with the ancients, who saw 
gods everywhere? Gods, my dear modern, are not spirits; gods do not degrade 
the world to a semblance, and do not spiritualize it.

But to you the whole world is spiritualized, and has become an enigmatical 
ghost; therefore do not wonder if you likewise find in yourself nothing but a 
spook. Is not your body haunted by your spirit, and is not the latter alone 
the true and real, the former only the "{}transitory, naught"{} or a 
"{}semblance"{}? Are we not all ghosts, uncanny beings that wait for 
"{}deliverance"{} -- to wit, "{}spirits"{}?

Since the spirit appeared in the world, since "{}the Word became flesh,"{} 
since then the world has been spiritualized, enchanted, a spook.

You have spirit, for you have thoughts. What are your thoughts? "{}Spiritual 
entities."{} Not things, then? "{}No, but the spirit of things, the main point 
in all things, the inmost in them, their -- idea."{} Consequently what you 
think is not only your thought?

"{}On the contrary, it is that in the world which is most real, that which is 
properly to be called true; it is the truth itself; if I only think truly, I 
think the truth. I may, to be sure, err with regard to the truth, and 
\textit{fail to recognize} it; but, if I \textit{recognize} truly, the object 
of my cognition is the truth."{} So, I suppose, you strive at all times to 
recognize the truth? "{}To me the truth is sacred. It may well happen that I 
find a truth incomplete and replace it with a better, but \textit{the} truth I 
cannot abrogate. I \textit{believe} in the truth, therefore I search in it; 
nothing transcends it, it is eternal."{}

Sacred, eternal is the truth; it is the Sacred, the Eternal. But you, who let 
yourself be filled and led by this sacred thing, are yourself hallowed. 
Further, the sacred is not for your senses -- and you never as a sensual man 
discover its trace -- but for your faith, or, more definitely still, for your 
\textit{spirit}; for it itself, you know, is a spiritual thing, a spirit -- is 
spirit for the spirit.

The sacred is by no means so easily to be set aside as many at present affirm, 
who no longer take this "{}unsuitable"{} word into their mouths. If even in a 
single respect I am still \textit{upbraided} as an "{}egoist,"{} there is left 
the thought of something else which I should serve more than myself, and which 
must be to me more important than everything; in short, somewhat in which I 
should have to seek my true welfare,\footnote{[Heil]} something -- 
"{}sacred."{}\footnote{[heiling]} However human this sacred thing may look, 
though it be the Human itself, that does not take away its sacredness, but at 
most changes it from an unearthly to an earthly sacred thing, from a divine 
one to a human.

Sacred things exist only for the egoist who does not acknowledge himself, the 
\textit{involuntary egoist}, for him who is always looking after his own and 
yet does not count himself as the highest being, who serves only himself and 
at the same time always thinks he is serving a higher being, who knows nothing 
higher than himself and yet is infatuated about something higher; in short, 
for the egoist who would like not to be an egoist, and abases himself 
(\textit{i.e.} combats his egoism), but at the same time abases himself only 
for the sake of "{}being exalted,"{} and therefore of gratifying his egoism. 
Because he would like to cease to be an egoist, he looks about in heaven and 
earth for higher beings to serve and sacrifice himself to; but, however much 
he shakes and disciplines himself, in the end he does all for his own sake, 
and the disreputable egoism will not come off him. On this account I call him 
the involuntary egoist.

His toil and care to get away from himself is nothing but the misunderstood 
impulse to self-dissolution. If you are bound to your past hour, if you must 
babble today because you babbled yesterday,\footnote{[How the priests tinkle! 
how important they\\
 Would make it out, that men should come their way\\
 And babble, just as yesterday, today!

Oh, blame them not! They know man's need, I say!\\
 For he takes all his happiness this way,\\
 To babble just tomorrow as today.

Translated from Goethe's "{}Venetian Epigrams."{}]

} if you cannot transform yourself each instant, you feel yourself fettered in 
slavery and benumbed. Therefore over each minute of your existence a fresh 
minute of the future beckons to you, and, developing yourself, you get away 
"{}from yourself,"{} \textit{i. e.}, from the self that was at that moment. As 
you are at each instant, you are your own creature, and in this very 
"{}creature"{} you do not wish to lose yourself, the creator. You are yourself 
a higher being than you are, and surpass yourself. But that you are the one 
who is higher than you, \textit{i. e.}, that you are not only creature, but 
likewise your creator -- just this, as an involuntary egoist, you fail to 
recognize; and therefore the "{}higher essence"{} is to you -- an 
alien\footnote{[\textit{fremd}]} essence. Every higher essence, \textit{e. g.} 
truth, mankind, etc., is an essence \textit{over} us.

Alienness is a criterion of the "{}sacred."{} In everything sacred there lies 
something "{}uncanny,"{} \textit{i.e.} strange,\footnote{[\textit{fremd}]} 
\textit{e. g.} we are not quite familiar and at home in. What is sacred to me 
is \textit{not my own}; and if, \textit{e. g.,}, the property of others was 
not sacred to me, I should look on it as \textit{mine}, which I should take to 
myself when occasion offered. Or, on the other side, if I regard the face of 
the Chinese emperor as sacred, it remains strange to my eye, which I close at 
its appearance.

Why is an incontrovertible mathematical truth, which might even be called 
eternal according to the common understanding of words, not -- sacred? Because 
it is not revealed, or not the revelation of, a higher being. If by revealed 
we understand only the so-called religious truths, we go far astray, and 
entirely fail to recognize the breadth of the concept "{}higher being."{} 
Atheists keep up their scoffing at the higher being, which was also honored 
under the name of the "{}highest"{} or \textit{\^Etre supr\^eme}, and trample 
in the dust one "{}proof of his existence"{} after another, without noticing 
that they themselves, out of need for a higher being, only annihilate the old 
to make room for a new. Is "{}Man"{} perchance not a higher essence than an 
individual man, and must not the truths, rights, and ideas which result from 
the concept of him be honored and --counted sacred, as revelations of this 
very concept? For, even though we should abrogate again many a truth that 
seemed to be made manifest by this concept, yet this would only evince a 
misunderstanding on our part, without in the least degree harming the sacred 
concept itself or taking their sacredness from those truths that must 
"{}rightly"{} be looked upon as its revelations. \textit{Man} reaches beyond 
every individual man, and yet -- though he be "{}his essence"{} -- is not in 
fact \textit{his} essence (which rather would be as 
single\footnote{[\textit{einzig}]} as he the individual himself), but a 
general and "{}higher,"{} yes, for atheists "{}the highest 
essence."{}\footnote{[\textit{"{}the supreme being}."{}]} And, as the divine 
revelations were not written down by God with his own hand, but made public 
through "{}the Lord's instruments,"{} so also the new highest essence does not 
write out its revelations itself, but lets them come to our knowledge through 
"{}true men."{} Only the new essence betrays, in fact, a more spiritual style 
of conception than the old God, because the latter was still represented in a 
sort of embodiedness or form, while the undimmed spirituality of the new is 
retained, and no special material body is fancied for it. And withal it does 
not lack corporeity, which even takes on a yet more seductive appearance 
because it looks more natural and mundane and consists in nothing less than in 
every bodily man -- yes, or outright in "{}humanity"{} or "{}all men."{} 
Thereby the spectralness of the spirit in a seeming body has once again become 
really solid and popular.

Sacred, then, is the highest essence and everything in which this highest 
essence reveals or will reveal itself; but hallowed are they who recognize 
this highest essence together with its own, \textit{i.e.} together with its 
revelations. The sacred hallows in turn its reverer, who by his worship 
becomes himself a saint, as Likewise what he does is saintly, a saintly walk, 
saintly thoughts and actions, imaginations and aspirations.

It is easily understood that the conflict over what is revered as the highest 
essence can be significant only so long as even the most embittered opponents 
concede to each other the main point -- that there is a highest essence to 
which worship or service is due. If one should smile compassionately at the 
whole struggle over a highest essence, as a Christian might at the war of 
words between a Shiite and a Sunnite or between a Brahman and a Buddhist, then 
the hypothesis of a highest essence would be null in his eyes, and the 
conflict on this basis an idle play. Whether then the one God or the three in 
one. whether the Lutheran God or the \textit{\^Etre supr\^eme} or not God at 
all, but "{}Man,"{} may represent the highest essence, that makes no 
difference at all for him who denies the highest essence itself, for in his 
eyes those servants of a highest essence are one and all-pious people, the 
most raging atheist not less than the most faith-filled Christian.

In the foremost place of the sacred,\footnote{[\textit{heilig}]} then, stands 
the highest essence and the faith in this essence, our 
"{}holy\footnote{[\textit{heilig}]} faith."{}

\medskip{}

\subsection[The Spook]{\centering The Spook}

With ghosts we arrive in the spirit-realm, in the realm of \textit{essences}.

What haunts the universe, and has its occult, "{}incomprehensible"{} being 
there, is precisely the mysterious spook that we call highest essence. And to 
get to the bottom of this \textit{spook}, to comprehend it, to discover 
\textit{reality} in it (to prove "{}the existence of God"{}) -- this task men 
set to themselves for thousands of years; with the horrible impossibility, the 
endless Danaid-labor, of transforming the spook into a non-spook, the unreal 
into something real, the \textit{spirit} into an entire and \textit{corporeal} 
person -- with this they tormented themselves to death. Behind the existing 
world they sought the "{}thing in itself,"{} the essence; behind the 
\textit{thing} they sought the \textit{un-thing}.

When one looks to the \textit{bottom} of anything, \textit{i.e.} searches out 
its \textit{essence}, one often discovers something quite other than what it 
\textit{seems} to be; honeyed speech and a lying heart, pompous words and 
beggarly thoughts, etc. By bringing the essence into prominence one degrades 
the hitherto misapprehended appearance to a bare \textit{semblance}, a 
deception. The essence of the world, so attractive and splendid, is for him 
who looks to the bottom of it -- emptiness; emptiness is = world's essence 
(world's doings). Now, he who is religious does not occupy himself with the 
deceitful semblance, with the empty appearances, but looks upon the essence, 
and in the essence has -- the truth.

The essences which are deduced from some appearances are the evil essences, 
and conversely from others the good. The essence of human feeling, \textit{e. 
g.}, is love; the essence of human will is the good; that of one's thinking, 
the true, etc.

What at first passed for existence, \textit{e. g.} the world and its like, 
appears now as bare semblance, and the \textit{truly existent} is much rather 
the essence, whose realm is filled with gods, spirits, demons, with good or 
bad essences. Only this inverted world, the world of essences, truly exists 
now. The human heart may be loveless, but its essence exists, God, "{}who is 
love"{}; human thought may wander in error, but its essence, truth, exists; 
"{}God is truth,"{} and the like.

To know and acknowledge essences alone and nothing but essences, that is 
religion; its realm is a realm of essences, spooks, and ghosts.

The longing to make the spook comprehensible, or to realize 
\textit{non-sense}, has brought about a \textit{corporeal ghost}, a ghost or 
spirit with a real body, an embodied ghost. How the strongest and most 
talented Christians have tortured themselves to get a conception of this 
ghostly apparition! But there always remained the contradiction of two 
natures, the divine and human, \textit{i. e.,} the ghostly and sensual; there 
remained the most wondrous spook, a thing that was not a thing. Never yet was 
a ghost more soul torturing, and no shaman, who pricks himself to raving fury 
and nerve-lacerating cramps to conjure a ghost, can endure such soul-torment 
as Christians suffered from that most incomprehensible ghost.

But through Christ the truth of the matter had at the same time come to light, 
that the veritable spirit or ghost is -- man. The \textit{corporeal} or 
embodied spirit is just man; he himself is the ghostly being and at the same 
time the being's appearance and existence. Henceforth man no longer, in 
typical cases, shudders at ghosts \textit{outside} him, but at himself; he is 
terrified at himself. In the depth of his breast dwells the \textit{spirit of 
sin}; even the faintest thought (and this is itself a spirit, you know) may be 
a \textit{devil}, etc. -- The ghost has put on a body, God has become man, but 
now man is himself the gruesome spook which he seeks to get back of, to 
exorcise, to fathom, to bring to reality and to speech; man is -- 
\textit{spirit}. What matter if the body wither, if only the spirit is saved? 
Everything rests on the spirit, and the spirit's or "{}soul's"{} welfare 
becomes the exclusive goal. Man has become to himself a ghost, an uncanny 
spook, to which there is even assigned a distinct seat in the body (dispute 
over the seat of the soul, whether in the head, etc.).

You are not to me, and I am not to you, a higher essence. Nevertheless a 
higher essence may be hidden in each of us, and call forth a mutual reverence. 
To take at once the most general, Man lives in you and me. If I did not see 
Man in you, what occasion should I have to respect you? To be sure, you are 
not Man and his true and adequate form, but only a mortal veil of his, from 
which he can withdraw without himself ceasing; but yet for the present this 
general and higher essence is housed in you, and you present before me 
(because an imperishable spirit has in you assumed a perishable body, so that 
really your form is only an "{}assumed"{} one) a spirit that appears, appears 
in you, without being bound to your body and to this particular mode of 
appearance -- therefore a spook. Hence I do not regard you as a higher essence 
but only respect that higher essence which "{}walks"{} in you; I "{}respect 
Man in you."{} The ancients did not observe anything of this sort in their 
slaves, and the higher essence "{}Man"{} found as yet little response. To make 
up for this, they saw in each other ghosts of another sort. The People is a 
higher essence than an individual, and, like Man or the Spirit of Man, a 
spirit haunting the individual -- the Spirit of the People. For this reason 
they revered this spirit, and only so far as he served this or else a spirit 
related to it (\textit{e. g.} the Spirit of the Family) could the individual 
appear significant; only for the sake of the higher essence, the People, was 
consideration allowed to the "{}member of the people."{} As you are hallowed 
to us by "{}Man"{} who haunts you, so at every time men have been hallowed by 
some higher essence or other, like People, Family, and such. Only for the sake 
of a higher essence has any one been honored from of old, only as a ghost has 
he been regarded in the light of a hallowed, \textit{i.e.}, protected and 
recognized person. If I cherish you because I hold you dear, because in you my 
heart finds nourishment, my need satisfaction, then it is not done for the 
sake of a higher essence, whose hallowed body you are, not on account of my 
beholding in you a ghost, \textit{i.e.} an appearing spirit, but from egoistic 
pleasure; you yourself with \textit{your} essence are valuable to me, for your 
essence is not a higher one, is not higher and more general than you, is 
unique\footnote{[\textit{einzig}]} like you yourself, because it is you.

But it is not only man that "{}haunts"{}; so does everything. The higher 
essence, the spirit, that walks in everything, is at the same time bound to 
nothing, and only -- "{}appears"{} in it. Ghosts in every corner!

Here would be the place to pass the haunting spirits in review, if they were 
not to come before us again further on in order to vanish before egoism. Hence 
let only a few of them be particularized by way of example, in order to bring 
us at once to our attitude toward them.

Sacred above all, \textit{e. g.}, is the "{}holy Spirit,"{} sacred the truth, 
sacred are right, law, a good cause, majesty, marriage, the common good, 
order, the fatherland, etc.

\subsection[Wheels In The Head]{\centering Wheels In The Head}

Man, your head is haunted; you have wheels in your head! You imagine great 
things, and depict to yourself a whole world of gods that has an existence for 
you, a spirit-realm to which you suppose yourself to be called, an ideal that 
beckons to you. You have a fixed idea!

Do not think that I am jesting or speaking figuratively when I regard those 
persons who cling to the Higher, and (because the vast majority belongs under 
this head) almost the whole world of men, as veritable fools, fools in a 
madhouse. What is it, then, that is called a "{}fixed idea"{}? An idea that 
has subjected the man to itself. When you recognize, with regard to such a 
fixed idea, that it is a folly, you shut its slave up in an asylum. And is the 
truth of the faith, say, which we are not to doubt; the majesty of (\textit{e. 
g.}) the people, which we are not to strike at (he who does is guilty of -- 
lese-majesty); virtue, against which the censor is not to let a word pass, 
that morality may be kept pure; -- are these not "{}fixed ideas"{}? Is not all 
the stupid chatter of (\textit{e. g.}) most of our newspapers the babble of 
fools who suffer from the fixed idea of morality, legality, Christianity, 
etc., and only seem to go about free because the madhouse in which they walk 
takes in so broad a space? Touch the fixed idea of such a fool, and you will 
at once have to guard your back against the lunatic's stealthy malice. For 
these great lunatics are like the little so-called lunatics in this point too 
-- that they assail by stealth him who touches their fixed idea. They first 
steal his weapon, steal free speech from him, and then they fall upon him with 
their nails. Every day now lays bare the cowardice and vindictiveness of these 
maniacs, and the stupid populace hurrahs for their crazy measures. One must 
read the journals of this period, and must hear the Philistines talk, to get 
the horrible conviction that one is shut up in a house with fools. "{}Thou 
shalt not call thy brother a fool; if thou dost -- etc."{} But I do not fear 
the curse, and I say, my brothers are arch-fools. Whether a poor fool of the 
insane asylum is possessed by the fancy that he is God the Father, Emperor of 
Japan, the Holy Spirit, etc., or whether a citizen in comfortable 
circumstances conceives that it is his mission to be a good Christian, a 
faithful Protestant, a loyal citizen, a virtuous man -- both these are one and 
the same "{}fixed idea."{} He who has never tried and dared not to be a good 
Christian, a faithful Protestant, a virtuous man, etc., is \textit{possessed} 
and prepossessed\footnote{[\textit{gefangen und befangen}, literally 
"{}imprisoned and prepossessed."{}]} by faith, virtuousness, etc. Just as the 
schoolmen philosophized only \textit{inside} the belief of the church; as Pope 
Benedict XIV wrote fat books \textit{inside} the papist superstition, without 
ever throwing a doubt upon this belief; as authors fill whole folios on the 
State without calling in question the fixed idea of the State itself; as our 
newspapers are crammed with politics because they are conjured into the fancy 
that man was created to be a \textit{zoon politicon} -- so also subjects 
vegetate in subjection, virtuous people in virtue, liberals in humanity, 
without ever putting to these fixed ideas of theirs the searching knife of 
criticism. Undislodgeable, like a madman's delusion, those thoughts stand on a 
firm footing, and he who doubts them -- lays hands on the \textit{sacred!} 
Yes, the "{}fixed idea,"{} that is the truly sacred!

Is it perchance only people possessed by the devil that meet us, or do we as 
often come upon people \textit{possessed} in the contrary way -- possessed by 
"{}the good,"{} by virtue, morality, the law, or some "{}principle"{} or 
other? Possessions of the devil are not the only ones. God works on us, and 
the devil does; the former "{}workings of grace,"{} the latter "{}workings of 
the devil."{} Possessed\footnote{[\textit{besessene}]} people are 
set\footnote{[\textit{versessen}]} in their opinions.

If the word "{}possession"{} displeases you, then call it prepossession; yes, 
since the spirit possesses you, and all "{}inspirations"{} come from it, call 
it -- inspiration and enthusiasm. I add that complete enthusiasm -- for we 
cannot stop with the sluggish, half- way kind -- is called fanaticism.

It is precisely among cultured people that \textit{fanaticism} is at home; for 
man is cultured so far as he takes an interest in spiritual things, and 
interest in spiritual things, when it is alive, is and must be 
\textit{fanaticism}; it is a fanatical interest in the sacred 
\textit{(fanum)}. Observe our liberals, look into the \textit{S\"achsischen 
Vaterlandsbl\"atter}, hear what Schlosser 
says:\footnote{\textit{"{}Achtzehntes Jahrhundert}"{}, II, 519.} "{}Holbach's 
company constituted a regular plot against the traditional doctrine and the 
existing system, and its members were as fanatical on behalf of their unbelief 
as monks and priests, Jesuits and Pietists, Methodists, missionary and Bible 
societies, commonly are for mechanical worship and orthodoxy."{}

Take notice how a "{}moral man"{} behaves, who today often thinks he is 
through with God and throws off Christianity as a bygone thing. If you ask him 
whether he has ever doubted that the copulation of brother and sister is 
incest, that monogamy is the truth of marriage, that filial piety is a sacred 
duty, then a moral shudder will come over him at the conception of one's being 
allowed to touch his sister as wife also, etc. And whence this shudder? 
Because he \textit{believes} in those moral commandments. This moral 
\textit{faith} is deeply rooted in his breast. Much as he rages against the 
\textit{pious} Christians, he himself has nevertheless as thoroughly remained 
a Christian -- to wit, a \textit{moral} Christian. In the form of morality 
Christianity holds him a prisoner, and a prisoner under \textit{faith}. 
Monogamy is to be something sacred, and he who may live in bigamy is punished 
as a \textit{criminal}; he who commits incest suffers as a \textit{criminal}. 
Those who are always crying that religion is not to be regarded in the State, 
and the Jew is to be a citizen equally with the Christian, show themselves in 
accord with this. Is not this of incest and monogamy a \textit{dogma of 
faith?} Touch it, and you will learn by experience how this moral man is a 
\textit{hero of faith} too, not less than Krummacher, not less than Philip II. 
These fight for the faith of the Church, he for the faith of the State, or the 
moral laws of the State; for articles of faith, both condemn him who acts 
otherwise than \textit{their faith will} allow. The brand of "{}crime"{} is 
stamped upon him, and he may languish in reformatories, in jails. Moral faith 
is as fanatical as religious faith! They call that "{}liberty of faith"{} 
then, when brother and sister, on account of a relation that they should have 
settled with their "{}conscience,"{} are thrown into prison. "{}But they set a 
pernicious example."{} Yes, indeed: others might have taken the notion that 
the State had no business to meddle with their relation, and thereupon 
"{}purity of morals"{} would go to ruin. So then the religious heroes of faith 
are zealous for the "{}sacred God,"{} the moral ones for the "{}sacred 
good."{}

Those who are zealous for something sacred often look very little like each 
other. How the strictly orthodox or old-style believers differ from the 
fighters for "{}truth, light, and justice,"{} from the Philalethes, the 
Friends of Light, the Rationalists, and others. And yet, how utterly 
unessential is this difference! If one buffets single traditional truths 
(\textit{i.e.} miracles, unlimited power of princes), then the Rationalists 
buffet them too, and only the old-style believers wail. But, if one buffets 
truth itself, he immediately has both, as \textit{believers}, for opponents. 
So with moralities; the strict believers are relentless, the clearer heads are 
more tolerant. But he who attacks morality itself gets both to deal with. 
"{}Truth, morality, justice, light, etc.,"{} are to be and remain 
"{}sacred."{} What any one finds to censure in Christianity is simply supposed 
to be "{}unchristian"{} according to the view of these rationalists, but 
Christianity must remain a "{}fixture,"{} to buffet it is outrageous, "{}an 
outrage."{} To be sure, the heretic against pure faith no longer exposes 
himself to the earlier fury of persecution, but so much the more does it now 
fall upon the heretic against pure morals.

\begin{center}
--------\end{center}


Piety has for a century received so many blows, and had to hear its superhuman 
essence reviled as an "{}inhuman"{} one so often, that one cannot feel tempted 
to draw the sword against it again. And yet it has almost always been only 
moral opponents that have appeared in the arena, to assail the supreme essence 
in favor of -- another supreme essence. So Proudhon, unabashed, 
says:\footnote{\textit{"{}De la Cr\'eation de l'Ordre}"{} etc., p. 36.} "{}Man 
is destined to live without religion, but the moral law is eternal and 
absolute. Who would dare today to attack morality?"{} Moral people skimmed off 
the best fat from religion, ate it themselves, and are now having a tough job 
to get rid of the resulting scrofula. If, therefore, we point out that 
religion has not by any means been hurt in its inmost part so long as people 
reproach it only with its superhuman essence, and that it takes its final 
appeal to the "{}spirit"{} alone (for God is spirit), then we have 
sufficiently indicated its final accord with morality, and can leave its 
stubborn conflict with the latter lying behind us. It is a question of a 
supreme essence with both, and whether this is a superhuman or a human one can 
make (since it is in any case an essence over me, a super-mine one, so to 
speak) but little difference to me. In the end the relation to the human 
essence, or to "{}Man,"{} as soon as ever it has shed the snake-skin of the 
old religion, will yet wear a religious snake-skin again.

So Feuerbach instructs us that, "{}if one only \textit{inverts} speculative 
philosophy, \textit{i.e.} always makes the predicate the subject, and so makes 
the subject the object and principle, one has the undraped truth, pure and 
clean."{}\footnote{\textit{"{}Anekdota}, II, 64.} Herewith, to be sure, we 
lose the narrow religious standpoint, lost the \textit{God}, who from this 
standpoint is subject; but we take in exchange for it the other side of the 
religious standpoint, the \textit{moral} standpoint. Thus we no longer say 
"{}God is love,"{} but "{}Love is divine."{} If we further put in place of the 
predicate "{}divine"{} the equivalent "{}sacred,"{} then, as far as concerns 
the sense, all the old comes back-again. According to this, love is to be the 
\textit{good} in man, his divineness, that which does him honor, his true 
\textit{humanity} (it "{}makes him Man for the first time,"{} makes for the 
first time a man out of him). So then it would be more accurately worded thus: 
Love is what is \textit{human} in man, and what is inhuman is the loveless 
egoist. But precisely all that which Christianity and with it speculative 
philosophy (\textit{i.e.}, theology) offers as the good, the absolute, is to 
self-ownership simply not the good (or, what means the same, it is 
\textit{only the good)}. Consequently, by the transformation of the predicate 
into the subject, the Christian \textit{essence} (and it is the predicate that 
contains the essence, you know) would only be fixed yet more oppressively. God 
and the divine would entwine themselves all the more inextricably with me. To 
expel God from his heaven and to rob him of his \textit{"{}transcendence"{}} 
cannot yet support a claim of complete victory, if therein he is only chased 
into the human breast and gifted with indelible \textit{immanence}. Now they 
say, "{}The divine is the truly human!"{}

The same people who oppose Christianity as the basis of the State, 
\textit{i.e.} oppose the so-called Christian State, do not tire of repeating 
that morality is "{}the fundamental pillar of social life and of the State."{} 
As if the dominion of morality were not a complete dominion of the sacred, a 
"{}hierarchy."{}

So we may here mention by the way that rationalist movement which, after 
theologians had long insisted that only faith was capable of grasping 
religious truths, that only to believers did God reveal himself, and that 
therefore only the heart, the feelings, the believing fancy was religious, 
broke out with the assertion that the "{}natural understanding,"{} human 
reason, was also capable of discerning God. What does that mean but that the 
reason laid claim to be the same visionary as the 
fancy?\footnote{[\textit{dieselbe Phantastin wie die Phantasie.}]} In this 
sense Reimarus wrote his \textit{Most Notable Truths of Natural Religion}. It 
had to come to this -- that the \textit{whole} man with all his faculties was 
found to be \textit{religious}; heart and affections, understanding and 
reason, feeling, knowledge, and will -- in short, everything in man -- 
appeared religious. Hegel has shown that even philosophy is religious. And 
what is not called religion today? The "{}religion of love,"{} the "{}religion 
of freedom,"{} "{}political religion"{} -- in short, every enthusiasm. So it 
is, too, in fact.

To this day we use the Romance word "{}religion,"{} which expresses the 
concept of a condition of being \textit{bound}. To be sure, \textit{we} remain 
bound, so far as religion takes possession of our inward parts; but is the 
mind also bound? On the contrary, that is free, is sole lord, is not our mind, 
but absolute. Therefore the correct affirmative translation of the word 
religion would be \textit{"{}freedom of mind"{}}! In whomsoever the mind is 
free, he is religious in just the same way as he in whom the senses have free 
course is called a sensual man. The mind binds the former, the desires the 
latter. Religion, therefore, is boundness or \textit{religion} with reference 
to me -- I am bound; it is freedom with reference to the mind -- the mind is 
free, or has freedom of mind. Many know from experience how hard it is on 
\textit{us} when the desires run away with us, free and unbridled; but that 
the free mind, splendid intellectuality, enthusiasm for intellectual 
interests, or however this jewel may in the most various phrase be named, 
brings \textit{us} into yet more grievous straits than even the wildest 
impropriety, people will not perceive; nor can they perceive it without being 
consciously egoists.

Reimarus, and all who have shown that our reason, our heart, etc., also lead 
to God, have therewithal shown that we are possessed through and through. To 
be sure, they vexed the theologians, from whom they took away the prerogative 
of religious exaltation; but for religion, for freedom of mind, they thereby 
conquered yet more ground. For, when the mind is no longer limited to feeling 
or faith, but also, as understanding, reason, and thought in general, belongs 
to itself the mind -- when therefore, it may take part in the 
spiritual\footnote{[The same word as "{}intellectual"{}, as "{}mind"{} and 
"{}spirit"{} are the same.]} and heavenly truths in the form of understanding, 
as well as in its other forms -- then the whole mind is occupied only with 
spiritual things, \textit{i. e.}, with itself, and is therefore free. Now we 
are so through-and-through religious that "{}jurors,"{} \textit{i.e.} "{}sworn 
men,"{} condemn us to death, and every policeman, as a good Christian, takes 
us to the lock-up by virtue of an "{}oath of office."{}

Morality could not come into opposition with piety till after the time when in 
general the boisterous hate of everything that looked like an "{}order"{} 
(decrees, commandments, etc.) spoke out in revolt, and the personal 
"{}absolute lord"{} was scoffed at and persecuted; consequently it could 
arrive at independence only through liberalism, whose first form acquired 
significance in the world's history as "{}citizenship,"{} and weakened the 
specifically religious powers (see "{}Liberalism"{} below). For, when morality 
not merely goes alongside of piety, but stands on feet of its own, then its 
principle lies no longer in the divine commandments, but in the law of reason, 
from which the commandments, so far as they are still to remain valid, must 
first await justification for their validity. In the law of reason man 
determines himself out of himself, for "{}Man"{} is rational, and out of the 
"{}essence of Man"{} those laws follow of necessity. Piety and morality part 
company in this -- that the former makes God the law-giver, the latter Man.

From a certain standpoint of morality people reason about as follows: Either 
man is led by his sensuality, and is, following it, \textit{immoral}, or he is 
led by the good, which, taken up into the will, is called moral sentiment 
(sentiment and prepossession in favor of the good); then he shows himself 
\textit{moral}. From this point of view how, \textit{e. g.}, can Sand's act 
against Kotzebue be called immoral? What is commonly understood by unselfish 
it certainly was, in the same measure as (among other things) St. Crispin's 
thieveries in favor of the poor. "{}He should not have murdered, for it stands 
written, Thou shalt not murder!"{} Then to serve the good, the welfare of the 
people, as Sand at least intended, or the welfare of the poor, like Crispin -- 
is moral; but murder and theft are immoral; the purpose moral, the means 
immoral. Why? "{}Because murder, assassination, is something absolutely 
bad."{} When the Guerrillas enticed the enemies of the country into ravines 
and shot them down unseen from the bushes, do you suppose that was 
assassination? According to the principle of morality, which commands us to 
serve the good, you could really ask only whether murder could never in any 
case be a realization of the good, and would have to endorse that murder which 
realized the good. You cannot condemn Sand's deed at all; it was moral, 
because in the service of the good, because unselfish; it was an act of 
punishment, which the individual inflicted, an -- \textit{execution} inflicted 
at the risk of the executioner's life. What else had his scheme been, after 
all, but that he wanted to suppress writings by brute force? Are you not 
acquainted with the same procedure as a "{}legal"{} and sanctioned one? And 
what can be objected against it from your principle of morality? -- "{}But it 
was an illegal execution."{} So the immoral thing in it was the illegality, 
the disobedience to law? Then you admit that the good is nothing else than -- 
law, morality nothing else than \textit{loyalty}. And to this externality of 
"{}loyalty"{} your morality must sink, to this righteousness of works in the 
fulfillment of the law, only that the latter is at once more tyrannical and 
more revolting than the old-time righteousness of works. For in the latter 
only the \textit{act} is needed, but you require the \textit{disposition} too; 
one must carry \textit{in himself} the law, the statute; and he who is most 
legally disposed is the most moral. Even the last vestige of cheerfulness in 
Catholic life must perish in this Protestant legality. Here at last the 
domination of the law is for the first time complete. "{}Not I live, but the 
law lives in me."{} Thus I have really come so far to be only the "{}vessel of 
its glory."{} "{}Every Prussian carries his \textit{gendarme} in his 
breast,"{} says a high Prussian officer.

Why do certain \textit{opposition parties} fail to flourish? Solely for the 
reason that they refuse to forsake the path of morality or legality. Hence the 
measureless hypocrisy of devotion, love, etc., from whose repulsiveness one 
may daily get the most thorough nausea at this rotten and hypocritical 
relation of a "{}lawful opposition."{} -- In the \textit{moral} relation of 
love and fidelity a divided or opposed will cannot have place; the beautiful 
relation is disturbed if the one wills this and the other the reverse. But 
now, according to the practice hitherto and the old prejudice of the 
opposition, the moral relation is to be preserved above all. What is then left 
to the opposition? Perhaps the will to have a liberty, if the beloved one sees 
fit to deny it? Not a bit! It may not \textit{will} to have the freedom, it 
can only \textit{wish} for it, "{}petition"{} for it, lisp a "{}Please, 
please!"{} What would come of it, if the opposition really \textit{willed}, 
willed with the full energy of the will? No, it must renounce will in order to 
live to \textit{love}, renounce liberty -- for love of morality. It may never 
"{}claim as a right"{} what it is permitted only to "{}beg as a favor."{} 
Love, devotion. etc., demand with undeviating definiteness that there be only 
one will to which the others devote themselves, which they serve, follow, 
love. Whether this will is regarded as reasonable or as unreasonable, in both 
cases one acts morally when one follows it, and immorally when one breaks away 
from it. The will that commands the censorship seems to many unreasonable; but 
he who in a land of censorship evades the censoring of his book acts 
immorally, and he who submits it to the censorship acts morally. If some one 
let his moral judgment go, and set up \textit{e. g.} a secret press, one would 
have to call him immoral, and imprudent in the bargain if he let himself be 
caught; but will such a man lay claim to a value in the eyes of the 
"{}moral"{}? Perhaps! -- That is, if he fancied he was serving a "{}higher 
morality."{}

The web of the hypocrisy of today hangs on the frontiers of two domains, 
between which our time swings back and forth, attaching its fine threads of 
deception and self-deception. No longer vigorous enough to serve 
\textit{morality} without doubt or weakening, not yet reckless enough to live 
wholly to egoism, it trembles now toward the one and now toward the other in 
the spider-web of hypocrisy, and, crippled by the curse of \textit{halfness}, 
catches only miserable, stupid flies. If one has once dared to make a 
"{}free"{} motion, immediately one waters it again with assurances of love, 
and -- \textit{shams resignation}; if, on the other side, they have had the 
face to reject the free motion with \textit{moral} appeals to confidence, 
immediately the moral courage also sinks, and they assure one how they hear 
the free words with special pleasure, etc.; they -- \textit{sham approval}. In 
short, people would like to have the one, but not go without the other; they 
would like to have a \textit{free will}, but not for their lives lack the 
\textit{moral will}. Just come in contact with a servile loyalist, you 
Liberals. You will sweeten every word of freedom with a look of the most loyal 
confidence, and he will clothe his servilism in the most flattering phrases of 
freedom. Then you go apart, and he, like you, thinks "{}I know you, fox!"{} He 
scents the devil in you as much as you do the dark old Lord God in him.

A Nero is a "{}bad"{} man only in the eyes of the "{}good"{}; in mine he is 
nothing but a \textit{possessed} man, as are the good too. The good see in him 
an arch-villain, and relegate him to hell. Why did nothing hinder him in his 
arbitrary course? Why did people put up with so much? Do you suppose the tame 
Romans, who let all their will be bound by such a tyrant, were a hair the 
better? In old Rome they would have put him to death instantly, would never 
have been his slaves. But the contemporary "{}good"{} among the Romans opposed 
to him only moral demands, not their \textit{will}; they sighed that their 
emperor did not do homage to morality, like them; they themselves remained 
"{}moral subjects,"{} till at last one found courage to give up "{}moral, 
obedient subjection."{} And then the same "{}good Romans"{} who, as 
"{}obedient subjects,"{} had borne all the ignominy of having no will, 
hurrahed over the nefarious, immoral act of the rebel. Where then in the 
"{}good"{} was the courage for the \textit{revolution}, that courage which 
they now praised, after another had mustered it up? The good could not have 
this courage, for a revolution, and an insurrection into the bargain, is 
always something "{}immoral,"{} which one can resolve upon only when one 
ceases to be "{}good"{} and becomes either "{}bad"{} or -- neither of the two. 
Nero was no viler than his time, in which one could only be one of the two, 
good or bad. The judgment of his time on him had to be that he was bad, and 
this in the highest degree: not a milksop, but an arch-scoundrel. All moral 
people can pronounce only this judgment on him. Rascals \textit{e. g.} he was 
are still living here and there today (see \textit{e. g.} the \textit{Memoirs} 
of Ritter von Lang) in the midst of the moral. It is not convenient to live 
among them certainly, as one is not sure of his life for a moment; but can you 
say that it is more convenient to live among the moral? One is just as little 
sure of his life there, only that one is hanged "{}in the way of justice,"{} 
but least of all is one sure of his honor, and the national cockade is gone 
before you can say Jack Robinson. The hard fist of morality treats the noble 
nature of egoism altogether without compassion.

"{}But surely one cannot put a rascal and an honest man on the same level!"{} 
Now, no human being does that oftener than you judges of morals; yes, still 
more than that, you imprison as a criminal an honest man who speaks openly 
against the existing constitution, against the hallowed institutions, and you 
entrust portfolios and still more important things to a crafty rascal. So 
\textit{in praxi} you have nothing to reproach me with. "{}But in theory!"{} 
Now there I do put both on the same level, as two opposite poles -- to wit, 
both on the level of the moral law. Both have meaning only in the "{}moral 
world, just as in the pre-Christian time a Jew who kept the law and one who 
broke it had meaning and significance only in respect to the Jewish law; 
before Jesus Christ, on the contrary, the Pharisee was no more than the 
"{}sinner and publican."{} So before self-ownership the moral Pharisee amounts 
to as much as the immoral sinner.

Nero became very inconvenient by his possessedness. But a self-owning man 
would not sillily oppose to him the "{}sacred,"{} and whine if the tyrant does 
not regard the sacred; he would oppose to him his will. How often the 
sacredness of the inalienable rights of man has been held up to their foes, 
and some liberty or other shown and demonstrated to be a "{}sacred right of 
man!"{} Those who do that deserve to be laughed out of court -- as they 
actually are -- were it not that in truth they do, even though unconsciously, 
take the road that leads to the goal. They have a presentiment that, if only 
the majority is once won for that liberty, it will also will the liberty, and 
will then take what it \textit{will} have. The sacredness of the liberty, and 
all possible proofs of this sacredness, will never procure it; lamenting and 
petitioning only shows beggars.

The moral man is necessarily narrow in that he knows no other enemy than the 
"{}immoral"{} man. "{}He who is not moral is immoral!"{} and accordingly 
reprobate, despicable, etc. Therefore the moral man can never comprehend the 
egoist. Is not unwedded cohabitation an immorality? The moral man may turn as 
he pleases, he will have to stand by this verdict; Emilia Galotti gave up her 
life for this moral truth. And it is true, it is an immorality. A virtuous 
girl may become an old maid; a virtuous man may pass the time in fighting his 
natural impulses till he has perhaps dulled them, he may castrate himself for 
the sake of virtue as St. Origen did for the sake of heaven: he thereby honors 
sacred wedlock, sacred chastity, as inviolable; he is -- moral. Unchastity can 
never become a moral act. However indulgently the moral man may judge and 
excuse him who committed it, it remains a transgression, a sin against a moral 
commandment; there clings to it an indelible stain. As chastity once belonged 
to the monastic vow, so it does to moral conduct. Chastity is a -- good. -- 
For the egoist, on the contrary, even chastity is not a good without which he 
could not get along; he cares nothing at all about it. What now follows from 
this for the judgment of the moral man? This: that he throws the egoist into 
the only class of men that he knows besides moral men, into that of the -- 
immoral. He cannot do otherwise; he must find the egoist immoral in everything 
in which the egoist disregards morality. If he did not find him so, then he 
would already have become an apostate from morality without confessing it to 
himself, he would already no longer be a truly moral man. One should not let 
himself be led astray by such phenomena, which at the present day are 
certainly no longer to be classed as rare, but should reflect that he who 
yields any point of morality can as little be counted among the truly moral as 
Lessing was a pious Christian when, in the well-known parable, he compared the 
Christian religion, as well as the Mohammedan and Jewish, to a "{}counterfeit 
ring."{} Often people are already further than they venture to confess to 
themselves. For Socrates, because in culture he stood on the level of 
morality, it would have been an immorality if he had been willing to follow 
Crito's seductive incitement and escape from the dungeon; to remain was the 
only moral thing. But it was solely because Socrates was -- a moral man. The 
"{}unprincipled, sacrilegious"{} men of the Revolution, on the contrary, had 
sworn fidelity to Louis XVI, and decreed his deposition, yes, his death; but 
the act was an immoral one, at which moral persons will be horrified to all 
eternity.

Yet all this applies, more or less, only to "{}civic morality,"{} on which the 
freer look down with contempt. For it (like civism, its native ground, in 
general) is still too little removed and free from the religious heaven not to 
transplant the latter's laws without criticism or further consideration to its 
domain instead of producing independent doctrines of its own. Morality cuts a 
quite different figure when it arrives at the consciousness of its dignity, 
and raises its principle, the essence of man, or "{}Man,"{} to be the only 
regulative power. Those who have worked their way through to such a decided 
consciousness break entirely with religion, whose God no longer finds any 
place alongside their "{}Man,"{} and, as they (see below) themselves scuttle 
the ship of State, so too they crumble away that "{}morality"{} which 
flourishes only in the State, and logically have no right to use even its name 
any further. For what this "{}critical"{} party calls morality is very 
positively distinguished from the so-called "{}civic or political morality,"{} 
and must appear to the citizen like an "{}insensate and unbridled liberty."{} 
But at bottom it has only the advantage of the "{}purity of the principle,"{} 
which, freed from its defilement with the religious, has now reached universal 
power in its clarified definiteness as "{}humanity."{}

Therefore one should not wonder that the name "{}morality"{} is retained along 
with others, like freedom, benevolence, self-consciousness, and is only 
garnished now and then with the addition, a "{}free"{} morality -- just as, 
though the civic State is abused, yet the State is to arise again as a "{}free 
State,"{} or, if not even so, yet as a "{}free society."{}

Because this morality completed into humanity has fully settled its accounts 
with the religion out of which it historically came forth, nothing hinders it 
from becoming a religion on its own account. For a distinction prevails 
between religion and morality only so long as our dealings with the world of 
men are regulated and hallowed by our relation to a superhuman being, or so 
long as our doing is a doing "{}for God's sake."{} If, on the other hand, it 
comes to the point that "{}man is to man the supreme being,"{} then that 
distinction vanishes, and morality, being removed from its subordinate 
position, is completed into -- religion. For then the higher being who had 
hitherto been subordinated to the highest, Man, has ascended to absolute 
height, and we are related to him as one is related to the highest being, 
\textit{i.e.} religiously. Morality and piety are now as synonymous as in the 
beginning of Christianity, and it is only because the supreme being has come 
to be a different one that a holy walk is no longer called a "{}holy"{} one, 
but a "{}human"{} one. If morality has conquered, then a complete -- 
\textit{change of masters} has taken place.

After the annihilation of faith Feuerbach thinks to put in to the supposedly 
safe harbor of \textit{love}. "{}The first and highest law must be the love of 
man to man. \textit{Homo homini Deus est --} this is the supreme practical 
maxim, this is the turning point of the world's 
history."{}\footnote{"{}Essence of Christianity,"{} second edition, p. 402.} 
But, properly speaking, only the god is changed -- the \textit{deus}; love has 
remained: there love to the superhuman God, here love to the human God, to 
\textit{homo as Deus}. Therefore man is to me -- sacred. And everything 
"{}truly human"{} is to me -- sacred! "{}Marriage is sacred of itself. And so 
it is with all moral relations. Friendship is and must be \textit{sacred} for 
you, and property, and marriage, and the good of every man, but sacred 
\textit{in and of itself}.\footnote{P. 403.} "{} Haven't we the priest again 
there? Who is his God? Man with a great M! What is the divine? The human! Then 
the predicate has indeed only been changed into the subject, and, instead of 
the sentence "{}God is love,"{} they say "{}love is divine"{}; instead of 
"{}God has become man,"{} "{}Man has become God,"{} etc. It is nothing more or 
less than a new -- \textit{religion}. "{}All moral relations are ethical, are 
cultivated with a moral mind, only where of themselves (without religious 
consecration by the priest's blessing) they are counted \textit{religious}. 
"{} Feuerbach's proposition, "{}Theology is anthropology,"{} means only 
"{}religion must be ethics, ethics alone is religion."{}

Altogether Feuerbach accomplishes only a transposition of subject and 
predicate, a giving of preference to the latter. But, since he himself says, 
"{}Love is not (and has never been considered by men) sacred through being a 
predicate of God, but it is a predicate of God because it is divine in and of 
itself,"{} he might judge that the fight against the predicates themselves, 
against love and all sanctities, must be commenced. How could he hope to turn 
men away from God when he left them the divine? And if, as Feuerbach says, God 
himself has never been the main thing to them, but only his predicates, then 
he might have gone on leaving them the tinsel longer yet, since the doll, the 
real kernel, was left at any rate. He recognizes, too, that with him it is 
"{}only a matter of annihilating an illusion"{};\footnote{P. 408.} he thinks, 
however, that the effect of the illusion on men is "{}downright ruinous, since 
even love, in itself the truest, most inward sentiment, becomes an obscure, 
illusory one through religiousness, since religious love loves 
man\footnote{[Literally ``the man.'']} only for God's sake, therefore loves 
man only apparently, but in truth God only."{} Is this different with moral 
love? Does it love the man, \textit{this} man for \textit{this} man's sake, or 
for morality's sake, and so -- for \textit{homo homini Deus --} for God's 
sake?

\begin{center}
--------\end{center}


The wheels in the head have a number of other formal aspects, some of which it 
may be useful to indicate here.

Thus \textit{self-renunciation is} common to the holy with the unholy, to the 
pure and the impure. The impure man \textit{renounces} all "{}better 
feelings,"{} all shame, even natural timidity, and follows only the appetite 
that rules him. The pure man renounces his natural relation to the world 
("{}renounces the world"{}) and follows only the "{}desire"{} which rules him. 
Driven by the thirst for money, the avaricious man renounces all admonitions 
of conscience, all feeling of honor, all gentleness and all compassion; he 
puts all considerations out of sight; the appetite drags him along. The holy 
man behaves similarly. He makes himself the "{}laughing-stock of the world,"{} 
is hard-hearted and "{}strictly just"{}; for the desire drags him along. As 
the unholy man renounces \textit{himself} before Mammon, so the holy man 
renounces \textit{himself} before God and the divine laws. We are now living 
in a time when the \textit{shamelessness} of the holy is every day more and 
more felt and uncovered, whereby it is at the same time compelled to unveil 
itself, and lay itself bare, more and more every day. Have not the 
shamelessness and stupidity of the reasons with which men antagonize the 
"{}progress of the age"{} long surpassed all measure and all expectation? But 
it must be so. The self-renouncers must, as holy men, take the same course 
that they do so as unholy men; as the latter little by little sink to the 
fullest measure of self-renouncing vulgarity and \textit{lowness}, so the 
former must ascend to the most dishonorable \textit{exaltation}. The mammon of 
the earth and the \textit{God} of heaven both demand exactly the same degree 
of -- self-renunciation. The low man, like the exalted one, reaches out for a 
"{}good"{} -- the former for the material good, the latter for the ideal, the 
so-called "{}supreme good"{}; and at last both complete each other again too, 
as the "{}materially-minded"{} man sacrifices everything to an ideal phantasm, 
his \textit{vanity}, and the "{}spiritually-minded"{} man to a material 
gratification, the \textit{life of enjoyment}.

Those who exhort men to 
"{}unselfishness"{}\footnote{[\textit{uneigenn\"utzigkeit}, literally 
"{}un-self-benefitingness."{}]} think they are saying an uncommon deal. What 
do they understand by it? Probably something like what they understand by 
"{}self-renunciation."{} But who is this self that is to be renounced and to 
have no benefit? It seems that you yourself are supposed to be it. And for 
whose benefit is unselfish self-renunciation recommended to you? Again for 
\textit{your} benefit and behoof, only that through unselfishness you are 
procuring your "{}true benefit."{}

You are to benefit \textit{yourself}, and yet you are not to seek 
\textit{your} benefit.

People regard as unselfish the \textit{benefactor} of men, a Francke who 
founded the orphan asylum, an O'Connell who works tirelessly for his Irish 
people; but also the \textit{fanatic} who, like St. Boniface, hazards his life 
for the conversion of the heathen, or, like Robespierre, "{}sacrifices 
everything to virtue"{} -- like K\"orner, dies for God, king, and fatherland. 
Hence, among others, O'Connell's opponents try to trump up against him some 
selfishness or mercenariness, for which the O'Connell fund seemed to give them 
a foundation; for, if they were successful in casting suspicion on his 
"{}unselfishness,"{} they would easily separate him from his adherents.

Yet what could they show further than that O'Connell was working for another 
\textit{end} than the ostensible one? But, whether he may aim at making money 
or at liberating the people, it still remains certain, in one case as in the 
other, that he is striving for an end, and that \textit{his} end; selfishness 
here as there, only that his national self-interest would be beneficial to 
\textit{others too}, and so would be for the \textit{common} interest.

Now, do you suppose unselfishness is unreal and nowhere extant? On the 
contrary, nothing is more ordinary! One may even call it an article of fashion 
in the civilized world, which is considered so indispensable that, if it costs 
too much in solid material, people at least adorn themselves with its tinsel 
counterfeit and feign it. Where does unselfishness begin? Right where an end 
ceases to be \textit{our} end and our \textit{property}, which we, as owners, 
can dispose of at pleasure; where it becomes a fixed end or a -- fixed idea; 
where it begins to inspire, enthuse, fantasize us; in short, where it passes 
into our \textit{stubbornness} and becomes our -- master. One is not unselfish 
so long as he retains the end in his power; one becomes so only at that 
"{}Here I stand, I cannot do otherwise,"{} the fundamental maxim of all the 
possessed; one becomes so in the case of a \textit{sacred} end, through the 
corresponding sacred zeal.

I am not unselfish so long as the end remains my own, and I, instead of giving 
myself up to be the blind means of its fulfillment, leave it always an open 
question. My zeal need not on that account be slacker than the most fanatical, 
but at the same time I remain toward it frostily cold, unbelieving, and its 
most irreconcilable enemy; I remain its \textit{judge}, because I am its 
owner.

Unselfishness grows rank as far as possessedness reaches, as much on 
possessions of the devil as on those of a good spirit; there vice, folly, 
etc.; here humility, devotion, etc.

Where could one look without meeting victims of self-renunciation? There sits 
a girl opposite me, who perhaps has been making bloody sacrifices to her soul 
for ten years already. Over the buxom form droops a deathly-tired head, and 
pale cheeks betray the slow bleeding away of her youth. Poor child, how often 
the passions may have beaten at your heart, and the rich powers of youth have 
demanded their right! When your head rolled in the soft pillow, how awakening 
nature quivered through your limbs, the blood swelled your veins, and fiery 
fancies poured the gleam of voluptuousness into your eyes! Then appeared the 
ghost of the soul and its eternal bliss. You were terrified, your hands folded 
themselves, your tormented eyes turned their look upward, you -- prayed. The 
storms of nature were hushed, a calm glided over the ocean of your appetites. 
Slowly the weary eyelids sank over the life extinguished under them, the 
tension crept out unperceived from the rounded limbs, the boisterous waves 
dried up in the heart, the folded hands themselves rested a powerless weight 
on the unresisting bosom, one last faint "{}Oh dear!"{} moaned itself away, 
and -- \textit{the soul was at rest}. You fell asleep, to awake in the morning 
to a new combat and a new -- prayer. Now the habit of renunciation cools the 
heat of your desire, and the roses of your youth are growing pale in the -- 
chlorosis of your heavenliness. The soul is saved, the body may perish! O 
Lais, O Ninon, how well you did to scorn this pale virtue! One free 
\textit{grisette} against a thousand virgins grown gray in virtue!

The fixed idea may also be perceived as "{}maxim,"{} "{}principle,"{} 
"{}standpoint,"{} etc. Archimedes, to move the earth, asked for a standpoint 
\textit{outside} it. Men sought continually for this standpoint, and every one 
seized upon it as well as he was able. This foreign standpoint is the 
\textit{world of mind}, of ideas, thoughts, concepts, essences; it is 
\textit{heaven}. Heaven is the "{}standpoint"{} from which the earth is moved, 
earthly doings surveyed and -- despised. To assure to themselves heaven, to 
occupy the heavenly standpoint firmly and for ever -- how painfully and 
tirelessly humanity struggled for this!

Christianity has aimed to deliver us from a life determined by nature, from 
the appetites as actuating us, and so has meant that man should not let 
himself be determined by his appetites. This does not involve the idea that 
\textit{he} was not to have appetites, but that the appetites were not to have 
him, that they were not to become \textit{fixed}, uncontrollable, 
indissoluble. Now, could not what Christianity (religion) contrived against 
the appetites be applied by us to its own precept that \textit{mind} (thought, 
conceptions, ideas, faith) must determine us; could we not ask that neither 
should mind, or the conception, the idea, be allowed to determine us, to 
become fixed and inviolable or "{}sacred"{}? Then it would end in the 
\textit{dissolution of mind}, the dissolution of all thoughts, of all 
conceptions. As we there had to say, "{}We are indeed to have appetites, but 
the appetites are not to have us,"{} so we should now say, "{}We are indeed to 
have \textit{mind}, but mind is not to have us."{} If the latter seems lacking 
in sense, think \textit{e. g.} of the fact that with so many a man a thought 
becomes a "{}maxim,"{} whereby he himself is made prisoner to it, so that it 
is not he that has the maxim, but rather it that has him. And with the maxim 
he has a "{}permanent standpoint"{} again. The doctrines of the catechism 
become our \textit{principles} before we find it out, and no longer brook 
rejection. Their thought, or -- mind, has the sole power, and no protest of 
the "{}flesh"{} is further listened to. Nevertheless it is only through the 
"{}flesh"{} that I can break tyranny of mind; for it is only when a man hears 
his flesh along with the rest of him that he hears himself wholly, and it is 
only when he wholly hears \textit{himself} that he is a hearing or 
rational\footnote{[\textit{vern\"unftig}, derived from \textit{vernehmen}, to 
hear.]} being. The Christian does not hear the agony of his enthralled nature, 
but lives in "{}humility"{}; therefore he does not grumble at the wrong which 
befalls his \textit{person}; he thinks himself satisfied with the "{}freedom 
of the spirit."{} But, if the flesh once takes the floor, and its tone is 
"{}passionate,"{} "{}indecorous,"{} "{}not well-disposed,"{} "{}spiteful"{} 
(as it cannot be otherwise), then he thinks he hears voices of devils, voices 
\textit{against the spirit} (for decorum, passionlessness, kindly disposition, 
and the like, is -- spirit), and is justly zealous against them. He could not 
be a Christian if he were willing to endure them. He listens only to morality, 
and slaps unmorality in the mouth; he listens only to legality, and gags the 
lawless word. The \textit{spirit} of morality and legality holds him a 
prisoner; a rigid, unbending \textit{master}. They call that the "{}mastery of 
the spirit"{} -- it is at the same time the \textit{standpoint} of the spirit.

And now whom do the ordinary liberal gentlemen mean to make free? Whose 
freedom is it that they cry out and thirst for? The \textit{spirit's!} That of 
the spirit of morality, legality, piety, the fear of God. That is what the 
anti-liberal gentlemen also want, and the whole contention between the two 
turns on a matter of advantage -- whether the latter are to be the only 
speakers, or the former are to receive a "{}share in the enjoyment of the same 
advantage."{} The \textit{spirit} remains the absolute \textit{lord} for both, 
and their only quarrel is over who shall occupy the hierarchical throne that 
pertains to the "{}Viceregent of the Lord."{} The best of it is that one can 
calmly look upon the stir with the certainty that the wild beasts of history 
will tear each other to pieces just like those of nature; their putrefying 
corpses fertilize the ground for -- our crops.

We shall come back later to many another wheel in the head -- \textit{e. g.}, 
those of vocation, truthfulness, love, etc.

\begin{center}
--------\end{center}


When one's own is contrasted with what is \textit{imparted} to him, there is 
no use in objecting that we cannot have anything isolated, but receive 
everything as a part of the universal order, and therefore through the 
impression of what is around us, and that consequently we have it as something 
"{}imparted"{}; for there is a great difference between the feelings and 
thoughts which are \textit{aroused} in me by other things and those which are 
\textit{given} to me. God, immortality, freedom, humanity, etc. are drilled 
into us from childhood as thoughts and feelings which move our inner being 
more or less strongly, either ruling us without our knowing it, or sometimes 
in richer natures manifesting themselves in systems and works of art; but are 
always not aroused, but imparted, feelings, because we must believe in them 
and cling to them. That an Absolute existed, and that it must be taken in, 
felt, and thought by us, was settled as a faith in the minds of those who 
spent all the strength of their mind on recognizing it and setting it forth. 
The \textit{feeling} for the Absolute exists there as an imparted one, and 
thenceforth results only in the most manifold revelations of its own self. So 
in Klopstock the religious feeling was an imparted one, which in the 
\textit{Messiad} simply found artistic expression. If, on the other hand, the 
religion with which he was confronted had been for him only an incitation to 
feeling and thought, and if he had known how to take an attitude completely 
\textit{his own} toward it, then there would have resulted, instead of 
religious inspiration, a dissolution and consumption of the religion itself. 
Instead of that, he only continued in mature years his childish feelings 
received in childhood, and squandered the powers of his manhood in decking out 
his childish trifles.

The difference is, then, whether feelings are imparted to me or only aroused. 
Those which are aroused are my own, egoistic, because they are not \textit{as 
feelings} drilled into me, dictated to me, and pressed upon me; but those 
which are imparted to me I receive, with open arms -- I cherish them in me as 
a heritage, cultivate them, and am \textit{possessed} by them. Who is there 
that has never, more or less consciously, noticed that our whole education is 
calculated to produce \textit{feelings} in us, \textit{i.e.} impart them to 
us, instead of leaving their production to ourselves however they may turn 
out? If we hear the name of God, we are to feel veneration; if we hear that of 
the prince's majesty, it is to be received with reverence, deference, 
submission; if we hear that of morality, we are to think that we hear 
something inviolable; if we hear of the Evil One or evil ones, we are to 
shudder. The intention is directed to these \textit{feelings}, and he who 
\textit{e. g.} should hear with pleasure the deeds of the "{}bad"{} would have 
to be "{}taught what's what"{} with the rod of discipline. Thus stuffed with 
\textit{imparted feelings}, we appear before the bar of majority and are 
"{}pronounced of age."{} Our equipment consists of "{}elevating feelings, 
lofty thoughts, inspiring maxims, eternal principles,"{} etc. The young are of 
age when they twitter like the old; they are driven through school to learn 
the old song, and, when they have this by heart, they are declared of age.

We \textit{must not} feel at every thing and every name that comes before us 
what we could and would like to feel thereat; \textit{e. g.} at the name of 
God we must think of nothing laughable, feel nothing disrespectful, it being 
prescribed and imparted to us what and how we are to feel and think at mention 
of that name. That is the meaning of the \textit{care of souls --} that my 
soul or my mind be tuned as others think right, not as I myself would like it. 
How much trouble does it not cost one, finally to secure to oneself a feeling 
of one's own at the mention of at least this or that name, and to laugh in the 
face of many who expect from us a holy face and a composed expression at their 
speeches. What is imparted is \textit{alien} to us, is not our own, and 
therefore is "{}sacred,"{} and it is hard work to lay aside the "{}sacred 
dread of it."{}

Today one again hears "{}seriousness"{} praised, "{}seriousness in the 
presence of highly important subjects and discussions,"{} "{}German 
seriousness,"{} etc. This sort of seriousness proclaims clearly how old and 
grave lunacy and possession have already become. For there is nothing more 
serious than a lunatic when he comes to the central point of his lunacy; then 
his great earnestness incapacitates him for taking a joke. (See madhouses.)

\medskip{}

\subsection[\S{}3. The Hierarchy]{\centering \S{}3. The Hierarchy}

The historical reflections on our Mongolism which I propose to insert 
episodically at this place are not given with the claim of thoroughness, or 
even of approved soundness, but solely because it seems to me that they may 
contribute toward making the rest clear.

The history of the world, whose shaping properly belongs altogether to the 
Caucasian race, seems till now to have run through two Caucasian ages, in the 
first of which we had to work out and work off our innate \textit{negroidity}; 
this was followed in the second by \textit{Mongoloidity} (Chineseness), which 
must likewise be terribly made an end of. Negroidity represents 
\textit{antiquity}, the time of dependence on \textit{things} (on cocks' 
eating, birds' flight, on sneezing, on thunder and lightning, on the rustling 
of sacred trees, etc.); Mongoloidity the time of dependence on thoughts, the 
\textit{Christian} time. Reserved for the future are the words, "{}I am the 
owner of the world of things, I am the owner of the world of mind."{}

In the negroid age fall the campaigns of Sesostris and the importance of Egypt 
and of northern Africa in general. To the Mongoloid age belong the invasions 
of the Huns and Mongols, up to the Russians.

The value of \textit{me} cannot possibly be rated high so long as the hard 
diamond of the \textit{not-me} bears so enormous a price as was the case both 
with God and with the world. The not-me is still too stony and indomitable to 
be consumed and absorbed by me; rather, men only creep about with 
extraordinary \textit{bustle} on this \textit{immovable} entity, on this 
\textit{substance}, like parasitic animals on a body from whose juices they 
draw nourishment, yet without consuming it. It is the bustle of vermin, the 
assiduity of Mongolians. Among the Chinese, we know, everything remains as it 
used to be, and nothing "{}essential"{} or "{}substantial"{} suffers a change; 
all the more actively do they work away \textit{at} that which remains, which 
bears the name of the "{}old,"{} "{}ancestors,"{} etc.

Accordingly, in our Mongolian age all change has been only reformatory or 
ameliorative, not destructive or consuming and annihilating. The substance, 
the object, \textit{remains}. All our assiduity was only the activity of ants 
and the hopping of fleas, jugglers' tricks on the immovable tight-rope of the 
objective, \textit{corv\'ee} -service under the leadership of the unchangeable 
or "{}eternal."{} The Chinese are doubtless the most \textit{positive} nation, 
because totally buried in precepts; but neither has the Christian age come out 
from the \textit{positive}, \textit{i.e.} from "{}limited freedom,"{} freedom 
"{}within certain limits."{} In the most advanced stage of civilization this 
activity earns the name of \textit{scientific} activity, of working on a 
motionless presupposition, a \textit{hypothesis} that is not to be upset.

In its first and most unintelligible form morality shows itself as 
\textit{habit}. To act according to the habit and usage \textit{(mores)} of 
one's country -- is to be moral there. Therefore pure moral action, clear, 
unadulterated morality, is most straightforwardly practiced in China; they 
keep to the old habit and usage, and hate each innovation as a crime worthy of 
death. For \textit{innovation} is the deadly enemy of \textit{habit}, of the 
\textit{old}, of \textit{permanence}. In fact, too, it admits of no doubt that 
through habit man secures himself against the obtrusiveness of things, of the 
world, and founds a world of his own in which alone he is and feels at home, 
builds himself a \textit{heaven}. Why, heaven has no other meaning than that 
it is man's proper home, in which nothing alien regulates and rules him any 
longer, no influence of the earthly any longer makes him himself alien; in 
short, in which the dross of the earthly is thrown off, and the combat against 
the world has found an end -- in which, therefore, nothing is any longer 
\textit{denied} him. Heaven is the end of \textit{abnegation}, it is 
\textit{free enjoyment}. There man no longer denies himself anything, because 
nothing is any longer alien and hostile to him. But now habit is a "{}second 
nature,"{} which detaches and frees man from his first and original natural 
condition, in securing him against every casualty of it. The fully elaborated 
habit of the Chinese has provided for all emergencies, and everything is 
"{}looked out for"{}; whatever may come, the Chinaman always knows how he has 
to behave, and does not need to decide first according to the circumstances; 
no unforeseen case throws him down from the heaven of his rest. The morally 
habituated and inured Chinaman is not surprised and taken off his guard; he 
behaves with equanimity (\textit{i. e.}, with equal spirit or temper) toward 
everything, because his temper, protected by the precaution of his traditional 
usage, does not lose its balance. Hence, on the ladder of culture or 
civilization humanity mounts the first round through habit; and, as it 
conceives that, in climbing to culture, it is at the same time climbing to 
heaven, the realm of culture or second nature, it really mounts the first 
round of the -- ladder to heaven.

If Mongoldom has settled the existence of spiritual beings -- if it has 
created a world of spirits, a heaven -- the Caucasians have wrestled for 
thousands of years with these spiritual beings, to get to the bottom of them. 
What were they doing, then, but building on Mongolian ground? They have not 
built on sand, but in the air; they have wrestled with Mongolism, stormed the 
Mongolian heaven, Tien. When will they at last annihilate this heaven? When 
will they at last become \textit{really Caucasians}, and find themselves? When 
will the "{}immortality of the soul,"{} which in these latter days thought it 
was giving itself still more security if it presented itself as "{}immortality 
of mind,"{} at last change to the \textit{mortality of mind?}

It was when, in the industrious struggle of the Mongolian race, men had 
\textit{built a heaven}, that those of the Caucasian race, since in their 
Mongolian complexion they have to do with heaven, took upon themselves the 
opposite task, the task of storming that heaven of custom, 
\textit{heaven-storming}\footnote{[A German idiom for destructive 
radicalism.]} activity. To dig under all human ordinance, in order to set up a 
new and -- better one on the cleared site, to wreck all customs in order to 
put new and -- better customs in their place -- their act is limited to this. 
But is it thus already purely and really what it aspires to be, and does it 
reach its final aim? No, in this creation of \textit{a "{}better"{}} it is 
tainted with Mongolism. It storms heaven only to make a heaven again, it 
overthrows an old power only to legitimate a new power, it only -- 
\textit{improves}. Nevertheless the point aimed at, often as it may vanish 
from the eyes at every new attempt, is the real, complete downfall of heaven, 
customs, etc. -- in short, of man secured only against the world, of the 
\textit{isolation} or \textit{inwardness} of man. Through the heaven of 
culture man seeks to isolate himself from the world, to break its hostile 
power. But this isolation of heaven must likewise be broken, and the true end 
of heaven-storming is the -- downfall of heaven, the annihilation of heaven. 
\textit{Improving} and \textit{reforming} is the Mongolism of the Caucasian, 
because thereby he is always getting up again what already existed -- to wit, 
a \textit{precept}, a generality, a heaven. He harbors the most irreconcilable 
enmity to heaven, and yet builds new heavens daily; piling heaven on heaven, 
he only crushes one by another; the Jews' heaven destroys the Greeks', the 
Christians' the Jews', the Protestants' the Catholics', etc. -- If the 
\textit{heaven-storming} men of Caucasian blood throw off their Mongolian 
skin, they will bury the emotional man under the ruins of the monstrous world 
of emotion, the isolated man under his isolated world, the paradisiacal man 
under his heaven. And heaven is the \textit{realm of spirits}, the realm 
\textit{of freedom of the spirit}.

The realm of heaven, the realm of spirits and ghosts, has found its right 
standing in the speculative philosophy. Here it was stated as the realm of 
thoughts, concepts, and ideas; heaven is peopled with thoughts and ideas, and 
this "{}realm of spirits"{} is then the true reality.

To want to win freedom for the \textit{spirit} is Mongolism; freedom of the 
spirit is Mongolian freedom, freedom of feeling, moral freedom, etc.

We may find the word "{}morality"{} taken as synonymous with spontaneity, 
self-determination. But that is not involved in it; rather has the Caucasian 
shown himself spontaneous only \textit{in spite} of his Mongolian morality. 
The Mongolian heaven, or morals,\footnote{[The same word that has been 
translated "{}custom"{} several times in this section.]} remained the strong 
castle, and only by storming incessantly at this castle did the Caucasian show 
himself moral; if he had not had to do with morals at all any longer, if he 
had not had therein his indomitable, continual enemy, the relation to morals 
would cease, and consequently morality would cease. That his spontaneity is 
still a moral spontaneity, therefore, is just the Mongoloidity of it -- is a 
sign that in it he has not arrived at himself. "{}Moral spontaneity"{} 
corresponds entirely with "{}religious and orthodox philosophy,"{} 
"{}constitutional monarchy,"{} "{}the Christian State,"{} "{}freedom within 
certain limits,"{} "{}the limited freedom of the press,"{} or, in a figure, to 
the hero fettered to a sick-bed.

Man has not really vanquished Shamanism and its spooks till he possesses the 
strength to lay aside not only the belief in ghosts or in spirits, but also 
the belief in the spirit.

He who believes in a spook no more assumes the "{}introduction of a higher 
world"{} than he who believes in the spirit, and both seek behind the sensual 
world a supersensual one; in short, they produce and believe \textit{another} 
world, and this other \textit{world, the product of their mind}, is a 
spiritual world; for their senses grasp and know nothing of another, a 
non-sensual world, only their spirit lives in it. Going on from this Mongolian 
belief in the \textit{existence of spiritual beings} to the point that the 
\textit{proper being} of man too is his \textit{spirit}, and that all care 
must be directed to this alone, to the "{}welfare of his soul,"{} is not hard. 
Influence on the spirit, so-called "{}moral influence,"{} is hereby assured.

Hence it is manifest that Mongolism represents utter absence of any rights of 
the sensuous, represents non-sensuousness and unnature, and that sin and the 
consciousness of sin was our Mongolian torment that lasted thousands of years.

But who, then, will dissolve the spirit into its \textit{nothing?} He who by 
means of the spirit set forth nature as the \textit{null}, finite, transitory, 
he alone can bring down the spirit too to like nullity. I can; each one among 
you can, who does his will as an absolute I; in a word, the \textit{egoist} 
can.

\begin{center}
--------\end{center}


Before the sacred, people lose all sense of power and all confidence; they 
occupy a \textit{powerless} and \textit{humble} attitude toward it. And yet no 
thing is sacred of itself, but by my \textit{declaring it sacred}, by my 
declaration, my judgment, my bending the knee; in short, by my -- conscience.

Sacred is everything which for the egoist is to be unapproachable, not to be 
touched, outside his \textit{power --} \textit{i.e.} above \textit{him}; 
sacred, in a word, is every \textit{matter of conscience}, for "{}this is a 
matter of conscience to me"{} means simply, "{}I hold this sacred."{}

For little children, just as for animals, nothing sacred exists, because, in 
order to make room for this conception, one must already have progressed so 
far in understanding that he can make distinctions like "{}good and bad,"{} 
"{}warranted and unwarranted"{}; only at such a level of reflection or 
intelligence -- the proper standpoint of religion -- can unnatural (\textit{i. 
e.}, brought into existence by thinking) \textit{reverence}, "{}sacred 
dread,"{} step into the place of natural fear. To this sacred dread belongs 
holding something outside oneself for mightier, greater, better warranted, 
better, etc.; \textit{i.e.} the attitude in which one acknowledges the might 
of something alien -- not merely feels it, then, but expressly acknowledges 
it, \textit{i.e.} admits it, yields, surrenders, lets himself be tied 
(devotion, humility, servility, submission). Here walks the whole ghostly 
troop of the "{}Christian virtues."{}

Everything toward which you cherish any respect or reverence deserves the name 
of sacred; you yourselves, too, say that you would feel a \textit{"{}sacred 
dread"{}} of laying hands on it. And you give this tinge even to the unholy 
(gallows, crime, etc.). You have a horror of touching it. There lies in it 
something uncanny, that is, unfamiliar or not \textit{your own}.

 "{}If something or other did not rank as sacred in a man's mind, why, then 
all bars would be let down to self-will, to unlimited subjectivity!"{} Fear 
makes the beginning, and one can make himself fearful to the coarsest man; 
already, therefore, a barrier against his insolence. But in fear there always 
remains the attempt to liberate oneself from what is feared, by guile, 
deception, tricks, etc. In reverence,\footnote{[\textit{Ehrfurcht}]} on the 
contrary, it is quite otherwise. Here something is not only 
feared,\footnote{[\textit{gef\"urchtet}]} but also 
honored\footnote{[\textit{geehrt}]}: what is feared has become an inward power 
which I can no longer get clear of; I honor it, am captivated by it and 
devoted to it, belong to it; by the honor which I pay it I am completely in 
its power, and do not even attempt liberation any longer. Now I am attached to 
it with all the strength of faith; I \textit{believe}. I and what I fear are 
one; "{}not I live, but the respected lives in me!"{} Because the spirit, the 
infinite, does not allow of coming to any end, therefore it is stationary; it 
fears \textit{dying}, it cannot let go its dear Jesus, the greatness of 
finiteness is no longer recognized by its blinded eye; the object of fear, now 
raised to veneration, may no longer be handled; reverence is made eternal, the 
respected is deified. The man is now no longer employed in creating, but in 
\textit{learning} (knowing, investigating, etc.), \textit{i.e.} occupied with 
a fixed \textit{object}, losing himself in its depths, without return to 
himself. The relation to this object is that of knowing, fathoming, basing, 
not that of \textit{dissolution} (abrogation, etc.). "{}Man is to be 
religious,"{} that is settled; therefore people busy themselves only with the 
question how this is to be attained, what is the right meaning of 
religiousness, etc. Quite otherwise when one makes the axiom itself doubtful 
and calls it in question, even though it should go to smash. Morality too is 
such a sacred conception; one must be moral, and must look only for the right 
"{}how,"{} the right way to be so. One dares not go at morality itself with 
the question whether it is not itself an illusion; it remains exalted above 
all doubt, unchangeable. And so we go on with the sacred, grade after grade, 
from the "{}holy"{} to the "{}holy of holies."{}

\begin{center}
--------\end{center}


Men are sometimes divided into two classes: \textit{cultured} and 
\textit{uncultured}. The former, so far as they were worthy of their name, 
occupied themselves with thoughts, with mind, and (because in the time since 
Christ, of which the very principle is thought, they were the ruling ones) 
demanded a servile respect for the thoughts recognized by them. State, 
emperor, church, God, morality, order, are such thoughts or spirits, that 
exist only for the mind. A merely living being, an animal, cares as little for 
them as a child. But the uncultured are really nothing but children, and he 
who attends only to the necessities of his life is indifferent to those 
spirits; but, because he is also weak before them, he succumbs to their power, 
and is ruled by -- thoughts. This is the meaning of hierarchy.

\textit{Hierarchy is dominion of thoughts, dominion of mind!}

We are hierarchic to this day, kept down by those who are supported by 
thoughts. Thoughts are the sacred.

But the two are always clashing, now one and now the other giving the offence; 
and this clash occurs, not only in the collision of two men, but in one and 
the same man. For no cultured man is so cultured as not to find enjoyment in 
things too, and so be uncultured; and no uncultured man is totally without 
thoughts. In Hegel it comes to light at last what a longing for things even 
the most cultured man has, and what a horror of every "{}hollow theory"{} he 
harbors. With him reality, the world of things, is altogether to correspond to 
the thought, and no concept is to be without reality. This caused Hegel's 
system to be known as the most objective, as if in it thought and thing 
celebrated their union. But this was simply the extremest case of violence on 
the part of thought, its highest pitch of despotism and sole dominion, the 
triumph of mind, and with it the triumph of \textit{philosophy}. Philosophy 
cannot hereafter achieve anything higher, for its highest is the 
\textit{omnipotence of mind}, the almightiness of mind.\footnote{[Rousseau, 
the Philanthropists; and others were hostile to culture and intelligence, but 
they overlooked the fact that this is present in \textit{all}men of the 
Christian type, and assailed only learned and refined culture.]}

Spiritual men have \textit{taken into their head} something that is to be 
realized. They have \textit{concepts} of love, goodness, etc., which they 
would like to see \textit{realized}; therefore they want to set up a kingdom 
of love on earth, in which no one any longer acts from selfishness, but each 
one "{}from love."{} Love is to \textit{rule}. What they have taken into their 
head, what shall we call it but -- \textit{fixed idea?} Why, "{}their head is 
\textit{haunted."{}} The most oppressive spook is \textit{Man}. Think of the 
proverb, "{}The road to ruin is paved with good intentions."{} The intention 
to realize humanity altogether in oneself, to become altogether man, is of 
such ruinous kind; here belong the intentions to become good, noble, loving, 
etc.

In the sixth part of the \textit{"{}Denkw\"urdigkeiten,"{}} p. 7, Bruno Bauer 
says: "{}That middle class, which was to receive such a terrible importance 
for modern history, is capable of no self-sacrificing action, no enthusiasm 
for an idea, no exaltation; it devotes itself to nothing but the interests of 
its mediocrity; \textit{i.e.} it remains always limited to itself, and 
conquers at last only through its bulk, with which it has succeeded in tiring 
out the efforts of passion, enthusiasm, consistency -- through its surface, 
into which it absorbs a part of the new ideas."{} And (p. 6) "{}It has turned 
the revolutionary ideas, for which not it, but unselfish or impassioned men 
sacrificed themselves, solely to its own profit, has turned spirit into money. 
-- That is, to be sure, after it had taken away from those ideas their point, 
their consistency, their destructive seriousness, fanatical against all 
egoism."{} These people, then, are not self-sacrificing, not enthusiastic, not 
idealistic, not consistent, not zealots; they are egoists in the usual sense, 
selfish people, looking out for their advantage, sober, calculating, etc.

Who, then, is "{}self-sacrificing?"{}\footnote{[\textit{Literally, 
"{}sacrificing"{}; the German word has not the prefix "{}self."{}}]} In the 
full sense, surely, he who ventures everything else for \textit{one thing}, 
one object, one will, one passion. Is not the lover self-sacrificing who 
forsakes father and mother, endures all dangers and privations, to reach his 
goal? Or the ambitious man, who offers up all his desires, wishes, and 
satisfactions to the single passion, or the avaricious man who denies himself 
everything to gather treasures, or the pleasure-seeker, etc.? He is ruled by a 
passion to which he brings the rest as sacrifices.

And are these self-sacrificing people perchance not selfish, not egoist? As 
they have only one ruling passion, so they provide for only one satisfaction, 
but for this the more strenuously, they are wholly absorbed in it. Their 
entire activity is egoistic, but it is a one-sided, unopened, narrow egoism; 
it is possessedness.

"{}Why, those are petty passions, by which, on the contrary, man must not let 
himself be enthralled. Man must make sacrifices for a great idea, a great 
cause!"{} A "{}great idea,"{} a "{}good cause,"{} is, it may be, the honor of 
God, for which innumerable people have met death; Christianity, which has 
found its willing martyrs; the Holy Catholic Church, which has greedily 
demanded sacrifices of heretics; liberty and equality, which were waited on by 
bloody guillotines.

He who lives for a great idea, a good cause, a doctrine, a system, a lofty 
calling, may not let any worldly lusts, any self-seeking interest, spring up 
in him. Here we have the concept of \textit{clericalism}, or, as it may also 
be called in its pedagogic activity, school-masterliness; for the idealists 
play the schoolmaster over us. The clergyman is especially called to live to 
the idea and to work for the idea, the truly good cause. Therefore the people 
feel how little it befits him to show worldly haughtiness, to desire good 
living, to join in such pleasures as dancing and gaming -- in short, to have 
any other than a "{}sacred interest."{} Hence, too, doubtless, is derived the 
scanty salary of teachers, who are to feel themselves repaid by the sacredness 
of their calling alone, and to "{}renounce"{} other enjoyments.

Even a directory of the sacred ideas, one or more of which man is to look upon 
as his calling, is not lacking. Family, fatherland, science, etc., may find in 
me a servant faithful to his calling.

Here we come upon the old, old craze of the world, which has not yet learned 
to do without clericalism -- that to live and work \textit{for an idea} is 
man's calling, and according to the faithfulness of its fulfillment his 
\textit{human} worth is measured.

This is the dominion of the idea; in other words, it is clericalism. Thus 
Robespierre and St. Just were priests through and through, inspired by the 
idea, enthusiasts, consistent instruments of this idea, idealistic men. So St. 
Just exclaims in a speech, "{}There is something terrible in the sacred love 
of country; it is so exclusive that it sacrifices everything to the public 
interest without mercy, without fear, without human consideration. It hurls 
Manlius down the precipice; it sacrifices its private inclinations; it leads 
Regulus to Carthage, throws a Roman into the chasm, and sets Marat, as a 
victim of his devotion, in the Pantheon."{}

Now, over against these representatives of ideal or sacred interests stands a 
world of innumerable "{}personal"{} profane interests. No idea, no system, no 
sacred cause is so great as never to be outrivaled and modified by these 
personal interests. Even if they are silent momentarily, and in times of rage, 
and fanaticism, yet they soon come uppermost again through "{}the sound sense 
of the people."{} Those ideas do not completely conquer till they are no 
longer hostile to personal interests, till they satisfy egoism.

The man who is just now crying herrings in front of my window has a personal 
interest in good sales, and, if his wife or anybody else wishes him the like, 
this remains a personal interest all the same. If, on the other hand, a thief 
deprived him of his basket, then there would at once arise an interest of 
many, of the whole city, of the whole country, or, in a word, of all who abhor 
theft; an interest in which the herring-seller's person would become 
indifferent, and in its place the category of the "{}robbed man"{} would come 
into the foreground. But even here all might yet resolve itself into a 
personal interest, each of the partakers reflecting that he must concur in the 
punishment of the thief because unpunished stealing might otherwise become 
general and cause him too to lose his own. Such a calculation, however, can 
hardly be assumed on the part of many, and we shall rather hear the cry that 
the thief is a "{}criminal."{} Here we have before us a judgment, the thief's 
action receiving its expression in the concept "{}crime."{} Now the matter 
stands thus: even if a crime did not cause the slightest damage either to me 
or to any of those in whom I take an interest, I should nevertheless denounce 
it. Why? Because I am enthusiastic for \textit{morality}, filled with the 
\textit{idea} of morality; what is hostile to it I everywhere assail. Because 
in his mind theft ranks as abominable without any question, Proudhon, 
\textit{e. g.}, thinks that with the sentence "{}Property is theft"{} he has 
at once put a brand on property. In the sense of the priestly, theft is always 
a \textit{crime}, or at least a misdeed.

Here the personal interest is at an end. This particular person who has stolen 
the basket is perfectly indifferent to my person; it is only the thief, this 
concept of which that person presents a specimen, that I take an interest in. 
The thief and man are in my mind irreconcilable opposites; for one is not 
truly man when one is a thief; one degrades \textit{Man} or "{}humanity"{} in 
himself when one steals. Dropping out of personal concern, one gets into 
\textit{philanthropy}, friendliness to man, which is usually misunderstood as 
if it was a love to men, to each individual, while it is nothing but a love of 
\textit{Man}, the unreal concept, the spook. It is not \textit{tous 
anthropous,} men, but \textit{ton anthropon}, Man, that the philanthropist 
carries in his heart. To be sure, he cares for each individual, but only 
because he wants to see his beloved ideal realized everywhere.

So there is nothing said here of care for me, you, us; that would be personal 
interest, and belongs under the head of "{}worldly love."{} Philanthropy is a 
heavenly, spiritual, a -- priestly love. \textit{Man} must be restored in us, 
even if thereby we poor devils should come to grief. It is the same priestly 
principle as that famous \textit{fiat justitia, pereat mundus}; man and 
justice are ideas, ghosts, for love of which everything is sacrificed; 
therefore, the priestly spirits are the "{}self-sacrificing"{} ones.

He who is infatuated with \textit{Man} leaves persons out of account so far as 
that infatuation extends, and floats in an ideal, sacred interest. 
\textit{Man}, you see, is not a person, but an ideal, a spook.

Now, things as different as possible can belong to \textit{Man} and be so 
regarded. If one finds Man's chief requirement in piety, there arises 
religious clericalism; if one sees it in morality, then moral clericalism 
raises its head. On this account the priestly spirits of our day want to make 
a "{}religion"{} of everything, a "{}religion of liberty,"{} "{}religion of 
equality,"{} etc., and for them every idea becomes a "{}sacred cause,"{} 
\textit{e. g.} even citizenship, politics, publicity, freedom of the press, 
trial by jury, etc.

Now, what does "{}unselfishness"{} mean in this sense? Having only an ideal 
interest, before which no respect of persons avails!

The stiff head of the worldly man opposes this, but for centuries has always 
been worsted at least so far as to have to bend the unruly neck and "{}honor 
the higher power"{}; clericalism pressed it down. When the worldly egoist had 
shaken off a higher power (\textit{e. g.} the Old Testament law, the Roman 
pope, etc.), then at once a seven times higher one was over him again, 
\textit{e. g.} faith in the place of the law, the transformation of all laymen 
into divines in place of the limited body of clergy, etc. His experience was 
like that of the possessed man into whom seven devils passed when he thought 
he had freed himself from one.

In the passage quoted above, all ideality is denied to the middle class. It 
certainly schemed against the ideal consistency with which Robespierre wanted 
to carry out the principle. The instinct of its interest told it that this 
consistency harmonized too little with what its mind was set on, and that it 
would be acting against itself if it were willing to further the enthusiasm 
for principle. Was it to behave so unselfishly as to abandon all its aims in 
order to bring a harsh theory to its triumph? It suits the priests admirably, 
to be sure, when people listen to their summons, "{}Cast away everything and 
follow me,"{} or "{}Sell all that thou hast and give to the poor, and thou 
shalt have treasure in heaven; and come, follow me."{} Some decided idealists 
obey this call; but most act like Ananias and Sapphira, maintaining a behavior 
half clerical or religious and half worldly, serving God and Mammon.

I do not blame the middle class for not wanting to let its aims be frustrated 
by Robespierre, \textit{i.e.} for inquiring of its egoism how far it might 
give the revolutionary idea a chance. But one might blame (if blame were in 
place here anyhow) those who let their own interests be frustrated by the 
interests of the middle class. However, will not they likewise sooner or later 
learn to understand what is to their advantage? August Becker 
says:\footnote{\textit{"{}Die Volksphilosophie unserer Tage}"{}, p. 22.} "{}To 
win the producers (proletarians) a negation of the traditional conception of 
right is by no means enough. Folks unfortunately care little for the 
theoretical victory of the idea. One must demonstrate to them \textit{ad 
oculos} how this victory can be practically utilized in life."{} And (p.32): 
"{}You must get hold of folks by their real interests if you want to work upon 
them."{} Immediately after this he shows how a fine looseness of morals is 
already spreading among our peasants, because they prefer to follow their real 
interests rather than the commands of morality.

Because the revolutionary priests or schoolmasters served \textit{Man}, they 
cut off the heads of \textit{men}. The revolutionary laymen, those outside the 
sacred circle, did not feel any greater horror of cutting off heads, but were 
less anxious about the rights of Man than about their own.

How comes it, though, that the egoism of those who affirm personal interest, 
and always inquire of it, is nevertheless forever succumbing to a priestly or 
schoolmasterly (\textit{i. e.} an ideal) interest? Their person seems to them 
too small, too insignificant -- and is so in fact -- to lay claim to 
everything and be able to put itself completely in force. There is a sure sign 
of this in their dividing themselves into two persons, an eternal and a 
temporal, and always caring either only for the one or only for the other, on 
Sunday for the eternal, on the work-day for the temporal, in prayer for the 
former, in work for the latter. They have the priest in themselves, therefore 
they do not get rid of him, but hear themselves lectured inwardly every 
Sunday.

How men have struggled and calculated to get at a solution regarding these 
dualistic essences! Idea followed upon idea, principle upon principle, system 
upon system, and none knew how to keep down permanently the contradiction of 
the "{}worldly"{} man, the so-called "{}egoist."{} Does not this prove that 
all those ideas were too feeble to take up my whole will into themselves and 
satisfy it? They were and remained hostile to me, even if the hostility lay 
concealed for a considerable time. Will it be the same with 
\textit{self-ownership?} Is it too only an attempt at mediation? Whatever 
principle I turned to, it might be to that of \textit{reason}, I always had to 
turn away from it again. Or can I always be rational, arrange my life 
according to reason in everything? I can, no doubt, \textit{strive} after 
rationality, I can \textit{love} it, just as I can also love God and every 
other idea. I can be a philosopher, a lover of wisdom, as I love God. But what 
I love, what I strive for, is only in my idea, my conception, my thoughts; it 
is in my heart, my head, it is in me like the heart, but it is not I, I am not 
it.

To the activity of priestly minds belongs especially what one often hears 
called \textit{"{}moral influence."{}}

Moral influence takes its start where \textit{humiliation} begins; yes, it is 
nothing else than this humiliation itself, the breaking and bending of the 
temper\footnote{[\textit{Muth}]} down to humility.\footnote{[\textit{Demuth}]} 
If I call to some one to run away when a rock is to be blasted, I exert no 
moral influence by this demand; if I say to a child "{}You will go hungry if 
you will not eat what is put on the table,"{} this is not moral influence. 
But, if I say to it, "{}You will pray, honor your parents, respect the 
crucifix, speak the truth, for this belongs to man and is man's calling,"{} or 
even "{}this is God's will,"{} then moral influence is complete; then a man is 
to bend before the \textit{calling} of man, be tractable, become humble, give 
up his will for an alien one which is set up as rule and law; he is to 
\textit{abase} himself before something \textit{higher}: self-abasement. "{}He 
that abaseth himself shall be exalted."{} Yes, yes, children must early be 
\textit{made} to practice piety, godliness, and propriety; a person of good 
breeding is one into whom "{}good maxims"{} have been \textit{instilled} and 
\textit{impressed}, poured in through a funnel, thrashed in and preached in.

If one shrugs his shoulders at this, at once the good wring their hands 
despairingly, and cry: "{}But, for heaven's sake, if one is to give children 
no good instruction, why, then they will run straight into the jaws of sin, 
and become good-for-nothing hoodlums!"{} Gently, you prophets of evil. 
Good-for-nothing in your sense they certainly will become; but your sense 
happens to be a very good-for-nothing sense. The impudent lads will no longer 
let anything be whined and chattered into them by you, and will have no 
sympathy for all the follies for which you have been raving and driveling 
since the memory of man began; they will abolish the law of inheritance; they 
will not be willing to \textit{inherit} your stupidities as you inherited them 
from your fathers; they destroy \textit{inherited sin}.\footnote{[Called in 
English theology "{}original sin."{}]} If you command them, "{}Bend before the 
Most High,"{} they will answer: "{}If he wants to bend us, let him come 
himself and do it; we, at least, will not bend of our own accord."{} And, if 
you threaten them with his wrath and his punishment, they will take it like 
being threatened with the bogie-man. If you are no more successful in making 
them afraid of ghosts, then the dominion of ghosts is at an end, and nurses' 
tales find no -- \textit{faith}.

And is it not precisely the liberals again that press for good education and 
improvement of the educational system? For how could their liberalism, their 
"{}liberty within the bounds of law,"{} come about without discipline? Even if 
they do not exactly educate to the fear of God, yet they demand the 
\textit{fear of Man} all the more strictly, and awaken "{}enthusiasm for the 
truly human calling"{} by discipline.

\begin{center}
--------\end{center}


A long time passed away, in which people were satisfied with the fancy that 
they had the \textit{truth}, without thinking seriously whether perhaps they 
themselves must be true to possess the truth. This time was the \textit{Middle 
Ages}. With the common consciousness -- \textit{i.e.} the consciousness which 
deals with things, that consciousness which has receptivity only for things, 
or for what is sensuous and sense-moving -- they thought to grasp what did not 
deal with things and was not perceptible by the senses. As one does indeed 
also exert his eye to see the remote, or laboriously exercise his hand till 
its fingers have become dexterous enough to press the keys correctly, so they 
chastened themselves in the most manifold ways, in order to become capable of 
receiving the supersensual wholly into themselves. But what they chastened 
was, after all, only the sensual man, the common consciousness, so-called 
finite or objective thought. Yet as this thought, this understanding, which 
Luther decries under the name of reason, is incapable of comprehending the 
divine, its chastening contributed just as much to the understanding of the 
truth as if one exercised the feet year in and year out in dancing, and hoped 
that in this way they would finally learn to play the flute. Luther, with whom 
the so-called Middle Ages end, was the first who understood that the man 
himself must become other than he was if he wanted to comprehend truth -- must 
become as true as truth itself. Only he who already has truth in his belief, 
only he who \textit{believes} in it, can become a partaker of it; 
\textit{i.e.} only the believer finds it accessible and sounds its depths. 
Only that organ of man which is able to blow can attain the further capacity 
of flute-playing, and only that man can become a partaker of truth who has the 
right organ for it. He who is capable of thinking only what is sensuous, 
objective, pertaining to things, figures to himself in truth only what 
pertains to things. But truth is spirit, stuff altogether inappreciable by the 
senses, and therefore only for the "{}higher consciousness,"{} not for that 
which is "{}earthly-minded."{}

With Luther, accordingly, dawns the perception that truth, because it is a 
\textit{thought}, is only for the \textit{thinking} man. And this is to say 
that man must henceforth take an utterly different standpoint, to wit, the 
heavenly, believing, scientific standpoint, or that of \textit{thought} in 
relation to its object, the -- \textit{thought} -- that of mind in relation to 
mind. Consequently: only the like apprehend the like. "{}You are like the 
spirit that you understand."{}\footnote{[Goethe, "{}Faust"{}.]}

Because Protestantism broke the medieval hierarchy, the opinion could take 
root that hierarchy in general had been shattered by it, and it could be 
wholly overlooked that it was precisely a "{}reformation,"{} and so a 
reinvigoration of the antiquated hierarchy. That medieval hierarchy had been 
only a weakly one, as it had to let all possible barbarism of unsanctified 
things run on uncoerced beside it, and it was the Reformation that first 
steeled the power of hierarchy. If Bruno Bauer 
thinks:\footnote{\textit{"{}Anekdota"{}}, II, 152.} "{}As the Reformation was 
mainly the abstract rending of the religious principle from art, State, and 
science, and so its liberation from those powers with which it had joined 
itself in the antiquity of the church and in the hierarchy of the Middle Ages, 
so too the theological and ecclesiastical movements which proceeded from the 
Reformation are only the consistent carrying out of this abstraction of the 
religious principle from the other powers of humanity,"{} I regard precisely 
the opposite as correct, and think that the dominion of spirits, or freedom of 
mind (which comes to the same thing), was never before so all-embracing and 
all-powerful, because the present one, instead of rending the religious 
principle from art, State, and science, lifted the latter altogether out of 
secularity into the "{}realm of spirit"{} and made them religious.

Luther and Descartes have been appropriately put side by side in their "{}He 
who believes in God"{} and "{}I think, therefore I am"{} (\textit{cogito, ergo 
sum}). Man's heaven is thought -- mind. Everything can be wrested from him, 
except thought, except faith. \textit{Particular} faith, like faith of Zeus, 
Astarte, Jehovah, Allah, may be destroyed, but faith itself is indestructible. 
In thought is freedom. What I need and what I hunger for is no longer granted 
to me by any \textit{grace}, by the Virgin Mary. by intercession of the 
saints, or by the binding and loosing church, but I procure it for myself. In 
short, my being (the \textit{sum}) is a living in the heaven of thought, of 
mind, a \textit{cogitare}. But I myself am nothing else than mind, thinking 
mind (according to Descartes), believing mind (according to Luther). My body I 
am not; my flesh may \textit{suffer} from appetites or pains. I am not my 
flesh, but I am \textit{mind}, only mind.

This thought runs through the history of the Reformation till today.

Only by the more modern philosophy since Descartes has a serious effort been 
made to bring Christianity to complete efficacy, by exalting the "{}scientific 
consciousness."{} to be the only true and valid one. Hence it begins with 
absolute \textit{doubt, dubitare}, with grinding common consciousness to 
atoms, with turning away from everything that "{}mind,"{} "{}thought,"{} does 
not legitimate. To it \textit{Nature} counts for nothing; the opinion of men, 
their "{}human precepts,"{} for nothing: and it does not rest till it has 
brought reason into everything, and can say "{}The real is the rational, and 
only the rational is the real."{} Thus it has at last brought mind, reason, to 
victory; and everything is mind, because everything is rational, because all 
nature, as well as even the most perverse opinions of men, contains reason; 
for "{}all must serve for the best,"{} \textit{i. e.}, lead to the victory of 
reason.

Descartes's \textit{dubitare} contains the decided statement that only 
\textit{cogitare}, thought, mind -- \textit{is}. A complete break with 
"{}common"{} consciousness, which ascribes reality to \textit{irrational} 
things! Only the rational is, only mind is! This is the principle of modern 
philosophy, the genuine Christian principle. Descartes in his own time 
discriminated the body sharply from the mind, and "{}the spirit 'tis that 
builds itself the body,"{} says Goethe.

But this philosophy itself, Christian philosophy, still does not get rid of 
the rational, and therefore inveighs against the "{}merely subjective,"{} 
against "{}fancies, fortuities, arbitrariness,"{} etc. What it wants is that 
the \textit{divine} should become visible in everything, and all consciousness 
become a knowing of the divine, and man behold God everywhere; but God never 
is, without the \textit{devil}. For this very reason the name of philosopher 
is not to be given to him who has indeed open eyes for the things of the 
world, a clear and undazzled gaze, a correct judgment about the world, but who 
sees in the world just the world, in objects only objects, and, in short, 
everything prosaically as it is; but he alone is a philosopher who sees, and 
points out or demonstrates, heaven in the world, the supernal in the earthly, 
the -- \textit{divine} in the mundane. The former may be ever so wise, there 
is no getting away from this:

\begin{quotation}

\noindent{}What wise men see not by their wisdom's art\\
 Is practiced simply by a childlike heart.\footnote{[Schiller, \textit{"{}Die 
Worte des Glaubens}"{}.]}\end{quotation}

\noindent{}It takes this childlike heart, this eye for the divine, to make a 
philosopher. The first-named man has only a "{}common"{} consciousness, but he 
who knows the divine, and knows how to tell it, has a "{}scientific"{} one. On 
this ground Bacon was turned out of the realm of philosophers. And certainly 
what is called English philosophy seems to have got no further than to the 
discoveries of so-called "{}clear heads,"{} \textit{e. g.} Bacon and Hume. The 
English did not know how to exalt the simplicity of the childlike heart to 
philosophic significance, did not know how to make -- philosophers out of 
childlike hearts. This is as much as to say, their philosophy was not able to 
become \textit{theological} or \textit{theology}, and yet it is only as 
theology that it can really \textit{live itself} out, complete itself. The 
field of its battle to the death is in theology. Bacon did not trouble himself 
about theological questions and cardinal points.

Cognition has its object in life. German thought seeks, more than that of 
others, to reach the beginnings and fountain-heads of life, and sees no life 
till it sees it in cognition itself. Descartes's \textit{cogito, ergo sum} has 
the meaning "{}One lives only when one thinks."{} Thinking life is called 
"{}intellectual life"{}! Only mind lives, its life is the true life. Then, 
just so in nature only the "{}eternal laws,"{} the mind or the reason of 
nature, are its true life. In man, as in nature, only the thought lives; 
everything else is dead! To this abstraction, to the life of generalities or 
of that which is \textit{lifeless}, the history of mind had to come. God, who 
is spirit, alone lives. Nothing lives but the ghost.

How can one try to assert of modern philosophy or modern times that they have 
reached freedom, since they have not freed us from the power of objectivity? 
Or am I perhaps free from a despot when I am not afraid of the personal 
potentate, to be sure, but of every infraction of the loving reverence which I 
fancy I owe him? The case is the same with modern times. They only changed the 
\textit{existing} objects, the real ruler, into \textit{conceived} objects, 
\textit{i.e.} into \textit{ideas}, before which the old respect not only was 
not lost, but increased in intensity. Even if people snapped their fingers at 
God and the devil in their former crass reality, people devoted only the 
greater attention to their ideas. "{}They are rid of the Evil One; evil is 
left."{}\footnote{[Parodied from the words of Mephistopheles in the witch's 
kitchen in "{}Faust"{}.]} The decision having once been made not to let 
oneself be imposed on any longer by the extant and palpable, little scruple 
was felt about revolting against the existing State or overturning the 
existing laws; but to sin against the \textit{idea} of the State, not to 
submit to the \textit{idea} of law, who would have dared that? So one remained 
a "{}citizen"{} and a "{}law-respecting,"{} loyal man; yes, one seemed to 
himself to be only so much more law-respecting, the more rationalistically one 
abrogated the former defective law in order to do homage to the "{}spirit of 
the law."{} In all this the objects had only suffered a change of form; they 
had remained in their preponderance and pre-eminence; in short, one was still 
involved in obedience and possessedness, lived in reflection, and had an 
object on which one reflected, which one respected, and before which one felt 
reverence and fear. One had done nothing but transform the \textit{things} 
into \textit{conceptions} of the things, into thoughts and ideas, whereby 
one's \textit{dependence} became all the more intimate and indissoluble. So, 
\textit{e. g.}, it is not hard to emancipate oneself from the commands of 
parents, or to set aside the admonitions of uncle and aunt, the entreaties of 
brother and sister; but the renounced obedience easily gets into one's 
conscience, and the less one does give way to the individual demands, because 
he rationalistically, by his own reason, recognizes them to be unreasonable, 
so much the more conscientiously does he hold fast to filial piety and family 
love, and so much the harder is it for him to forgive himself a trespass 
against the \textit{conception} which he has formed of family love and of 
filial duty. Released from dependence as regards the existing family, one 
falls into the more binding dependence on the idea of the family; one is ruled 
by the spirit of the family. The family consisting of John, Maggie, etc., 
whose dominion has become powerless, is only internalized, being left as 
"{}family"{} in general, to which one just applies the old saying, "{}We must 
obey God rather than man,"{} whose significance here is this: "{}I cannot, to 
be sure, accommodate myself to your senseless requirements, but, as my 
'family,' you still remain the object of my love and care"{}; for "{}the 
family"{} is a sacred idea, which the individual must never offend against. -- 
And this family internalized and desensualized into a thought, a conception, 
now ranks as the "{}sacred,"{} whose despotism is tenfold more grievous 
because it makes a racket in my conscience. This despotism is broken when the 
conception, family, also becomes a \textit{nothing} to me The Christian dicta, 
"{}Woman, what have I to do with thee?"{}\footnote{Matt. 10. 35.} "{}I am come 
to stir up a man against his father, and a daughter against her 
mother,"{}\footnote{John 2. 4.} and others, are accompanied by something that 
refers us to the heavenly or true family, and mean no more than the State's 
demand, in case of a collision between it and the family, that we obey 
\textit{its} commands.

The case of morality is like that of the family. Many a man renounces morals, 
but with great difficulty the conception, "{}morality."{} Morality is the 
"{}idea"{} of morals, their intellectual power, their power over the 
conscience; on the other hand, morals are too material to rule the mind, and 
do not fetter an "{}intellectual"{} man, a so-called independent, a 
"{}freethinker."{}

The Protestant may put it as he will, the "{}holy\footnote{[\textit{heilig}]} 
Scripture,"{} the "{}Word of God,"{} still remains 
sacred\footnote{[\textit{heilig}]} for him. He for whom this is no longer 
"{}holy"{} has ceased to -- be a Protestant. But herewith what is 
"{}ordained"{} in it, the public authorities appointed by God, etc., also 
remain sacred for him. For him these things remain indissoluble, 
unapproachable, "{}raised above all doubt"{}; and, as \textit{doubt}, which in 
practice becomes a \textit{buffeting}, is what is most man's own, these things 
remain "{}raised"{} above himself. He who cannot \textit{get away} from them 
will -- \textit{believe}; for to believe in them is to be \textit{bound} to 
them. Through the fact that in Protestantism the \textit{faith} becomes a more 
inward faith, the \textit{servitude} has also become a more inward servitude; 
one has taken those sanctities up into himself, entwined them with all his 
thoughts and endeavors, made them a \textit{"{}matter of conscience"{}}, 
constructed out of them a \textit{"{}sacred duty"{}} for himself. Therefore 
what the Protestant's conscience cannot get away from is sacred to him, and 
\textit{conscientiousness} most clearly designates his character.

Protestantism has actually put a man in the position of a country governed by 
secret police. The spy and eavesdropper, "{}conscience,"{} watches over every 
motion of the mind, and all thought and action is for it a "{}matter of 
conscience,"{} \textit{i. e.}, police business. This tearing apart of man into 
"{}natural impulse"{} and "{}conscience"{} (inner populace and inner police) 
is what constitutes the Protestant. The reason of the Bible (in place of the 
Catholic "{}reason of the church"{}) ranks as sacred, and this feeling and 
consciousness that the word of the Bible is sacred is called -- conscience. 
With this, then, sacredness is "{}laid upon one's conscience."{} If one does 
not free himself from conscience, the consciousness of the sacred, he may act 
unconscientiously indeed, but never consciencelessly.

The Catholic finds himself satisfied when he fulfills the \textit{command}; 
the Protestant acts according to his "{}best judgment and conscience."{} For 
the Catholic is only a \textit{layman}; the Protestant is himself a 
\textit{clergyman}.\footnote{[\textit{Geistlicher}, literally "{}spiritual 
man."{}]} Just this is the progress of the Reformation period beyond the 
Middle Ages, and at the same time its curse -- that the \textit{spiritual} 
became complete.

What else was the Jesuit moral philosophy than a continuation of the sale of 
indulgences? Only that the man who was relieved of his burden of sin now 
gained also an \textit{insight} into the remission of sins, and convinced 
himself how really his sin was taken from him, since in this or that 
particular case (casuists) it was so clearly no sin at all that he committed. 
The sale of indulgences had made all sins and transgressions permissible, and 
silenced every movement of conscience. All sensuality might hold sway, if it 
was only purchased from the church. This favoring of sensuality was continued 
by the Jesuits, while the strictly moral, dark, fanatical, repentant, 
contrite, praying Protestants (as the true completers of Christianity, to be 
sure) acknowledged only the intellectual and spiritual man. Catholicism, 
especially the Jesuits, gave aid to egoism in this way, found involuntary and 
unconscious adherents within Protestantism itself, and saved us from the 
subversion and extinction of \textit{sensuality}. Nevertheless the Protestant 
spirit spreads its dominion farther and farther; and, as, beside it the 
"{}divine,"{} the Jesuit spirit represents only the "{}diabolic"{} which is 
inseparable from everything divine, the latter can never assert itself alone, 
but must look on and see how in France, \textit{e. g.}, the Philistinism of 
Protestantism wins at last, and mind is on top.

Protestantism is usually complimented on having brought the mundane into 
repute again, \textit{e. g.} marriage, the State, etc. But the mundane itself 
as mundane, the secular, is even more indifferent to it than to Catholicism, 
which lets the profane world stand, yes, and relishes its pleasures, while the 
rational, consistent Protestant sets about annihilating the mundane 
altogether, and that simply by \textit{hallowing} it. So marriage has been 
deprived of its naturalness by becoming sacred, not in the sense of the 
Catholic sacrament, where it only receives its consecration from the church 
and so is unholy at bottom, but in the sense of being something sacred in 
itself to begin with, a sacred relation. Just so the State, also. Formerly the 
pope gave consecration and his blessing to it and its princes, now the State 
is intrinsically sacred, majesty is sacred without needing the priest's 
blessing. The order of nature, or natural law, was altogether hallowed as 
"{}God's ordinance."{} Hence it is said \textit{e. g.} in the Augsburg 
Confession, Art. II: "{}So now we reasonably abide by the saying, as the 
jurisconsults have wisely and rightly said: that man and woman should be with 
each other is a natural law. Now, if it is a \textit{natural law, then it is 
God's ordinance}, therefore implanted in nature, and therefore a 
\textit{divine} law also."{} And is it anything more than Protestantism 
brought up to date, when Feuerbach pronounces moral relations sacred, not as 
God's ordinance indeed, but, instead, for the sake of the \textit{spirit} that 
dwells in them? "{}But marriage as a free alliance of love, of course -- is 
\textit{sacred of itself}, by the nature of the union that is formed here. 
\textit{That} marriage alone is a \textit{religious} one that is a 
\textit{true} one, that corresponds to the \textit{essence} of marriage, love. 
And so it is with all moral relations. They are \textit{ethical}, are 
cultivated with a moral mind, only where they rank as \textit{religious of 
themselves}. True friendship is only where the \textit{limits} of friendship 
are preserved with religious conscientiousness, with the same 
conscientiousness with which the believer guards the dignity of his God. 
Friendship is and must be \textit{sacred} for you, and property, and marriage, 
and the good of every man, but sacred \textit{in and of 
itself."{}}\footnote{"{}Essence of Christianity"{}, p. 403.}

That is a very essential consideration. In Catholicism the mundane can indeed 
be \textit{consecrated} or \textit{hallowed}, but it is not sacred without 
this priestly blessing; in Protestantism, on the contrary, mundane relations 
are sacred \textit{of themselves}, sacred by their mere existence. The Jesuit 
maxim, "{}the end hallows the means,"{} corresponds precisely to the 
consecration by which sanctity is bestowed. No means are holy or unholy in 
themselves, but their relation to the church, their use for the church, 
hallows the means. Regicide was named as such; if it was committed for the 
church's behoof, it could be certain of being hallowed by the church, even if 
the hallowing was not openly pronounced. To the Protestant, majesty ranks as 
sacred; to the Catholic only that majesty which is consecrated by the pontiff 
can rank as such; and it does rank as such to him only because the pope, even 
though it be without a special act, confers this sacredness on it once for 
all. If he retracted his consecration, the king would be left only a "{}man of 
the world or layman,"{} an "{}unconsecrated"{} man, to the Catholic.

If the Protestant seeks to discover a sacredness in the sensual itself, that 
he may then be linked only to what is holy, the Catholic strives rather to 
banish the sensual from himself into a separate domain, where it, like the 
rest of nature, keeps its value for itself. The Catholic church eliminated 
mundane marriage from its consecrated order, and withdrew those who were its 
own from the mundane family; the Protestant church declared marriage and 
family ties to be holy, and therefore not unsuitable for its clergymen.

A Jesuit may, as a good Catholic, hallow everything. He needs only, \textit{e. 
g.}, to say to himself: "{}I as a priest am necessary to the church, but serve 
it more zealously when I appease my desires properly; consequently I will 
seduce this girl, have my enemy there poisoned, etc.; my end is holy because 
it is a priest's, consequently it hallows the means."{} For in the end it is 
still done for the benefit of the church. Why should the Catholic priest 
shrink from handing Emperor Henry VII the poisoned wafer for the -- church's 
welfare?

The genuinely churchly Protestants inveighed against every "{}innocent 
pleasure,"{} because only the sacred, the spiritual, could be innocent. What 
they could not point out the holy spirit in, the Protestants had to reject -- 
dancing, the theatre, ostentation (\textit{e. g.} in the church), and the 
like.

Compared with this puritanical Calvinism, Lutheranism is again more on the 
religious, spiritual, track -- is more radical. For the former excludes at 
once a great number of things as sensual and worldly, and \textit{purifies} 
the church; Lutheranism, on the contrary, tries to bring \textit{spirit} into 
all things as far as possible, to recognize the holy spirit as an essence in 
everything, and so to \textit{hallow} everything worldly. ("{}No one can 
forbid a kiss in honor."{} The spirit of honor hallows it.) Hence it was that 
the Lutheran Hegel (he declares himself such in some passage or other: he 
"{}wants to remain a Lutheran"{}) was completely successful in carrying the 
idea through everything. In everything there is reason, \textit{i.e.} holy 
spirit, or "{}the real is rational."{} For the real is in fact everything; as 
in each thing, \textit{e. g.}, each lie, the truth can be detected: there is 
no absolute lie, no absolute evil, etc.

Great "{}works of mind"{} were created almost solely by Protestants, as they 
alone were the true disciples and consummators of \textit{mind}.

\begin{center}
--------\end{center}


How little man is able to control! He must let the sun run its course, the sea 
roll its waves, the mountains rise to heaven. Thus he stands powerless before 
the \textit{uncontrollable}. Can he keep off the impression that he is 
helpless against this gigantic world? It is a fixed \textit{law} to which he 
must submit, it determines his \textit{fate}. Now, what did pre-Christian 
humanity work toward? Toward getting rid of the irruptions of the destinies, 
not letting oneself be vexed by them. The Stoics attained this in apathy, 
declaring the attacks of nature \textit{indifferent}, and not letting 
themselves be affected by them. Horace utters the famous \textit{Nil 
admirari}, by which he likewise announces the indifference of the 
\textit{other}, the world; it is not to influence us, not to rouse our 
astonishment. And that \textit{impavidum ferient ruinae} expresses the very 
same \textit{imperturbability} as Ps. 46.3: "{}We do not fear, though the 
earth should perish."{} In all this there is room made for the Christian 
proposition that the world is empty, for the Christian \textit{contempt of the 
world}.

The \textit{imperturbable} spirit of "{}the wise man,"{} with which the old 
world worked to prepare its end, now underwent an \textit{inner perturbation} 
against which no ataraxia, no Stoic courage, was able to protect it. The 
spirit, secured against all influence of the world, insensible to its shocks 
and \textit{exalted} above its attacks, admiring nothing, not to be 
disconcerted by any downfall of the world -- foamed over irrepressibly again, 
because gases (spirits) were evolved in its own interior, and, after the 
\textit{mechanical shock} that comes from without had become ineffective, 
\textit{chemical tensions}, that agitate within, began their wonderful play.

In fact, ancient history ends with this -- that \textit{I} have struggled till 
I won my ownership of the world. "{}All things have been delivered to me by my 
Father"{} (Matt. 11. 27). It has ceased to be overpowering, unapproachable, 
sacred, divine, for me; it is \textit{undeified}, and now I treat it so 
entirely as I please that, if I cared, I could exert on it all miracle-working 
power, \textit{i. e.}, power of mind -- remove mountains, command mulberry 
trees to tear themselves up and transplant themselves into the sea (Luke 
17.6), and do everything possible, \textit{thinkable} : "{}All things are 
possible to him who believes."{}\footnote{Mark. 9. 23.} I am the \textit{lord} 
of the world, mine is the "{}glory."{}\footnote{[\textit{Herrlichkeit}, which, 
according to its derivation, means "{}lordliness."{}]} The world has become 
prosaic, for the divine has vanished from it: it is my property, which I 
dispose of as I (to wit, the mind) choose.

When I had exalted myself to be the \textit{owner of the world}, egoism had 
won its first complete victory, had vanquished the world, had become 
worldless, and put the acquisitions of a long age under lock and key.

The first property, the first "{}glory,"{} has been acquired!

But the lord of the world is not yet lord of his thoughts, his feelings, his 
will: he is not lord and owner of the spirit, for the spirit is still sacred, 
the "{}Holy Spirit,"{} and the "{}worldless"{} Christian is not able to become 
"{}godless."{} If the ancient struggle was a struggle against the 
\textit{world}, the medieval (Christian) struggle is a struggle against self, 
the mind; the former against the outer world, the latter against the inner 
world. The medieval man is the man "{}whose gaze is turned inward,"{} the 
thinking, meditative

All wisdom of the ancients is \textit{the science of the world}, all wisdom of 
the moderns is \textit{the science of God}.

The heathen (Jews included) got through with the \textit{world}; but now the 
thing was to get through with self, the spirit, too; \textit{i.e.} to become 
spiritless or godless.

For almost two thousand years we have been working at subjecting the Holy 
Spirit to ourselves, and little by little we have torn off and trodden under 
foot many bits of sacredness; but the gigantic opponent is constantly rising 
anew under a changed form and name. The spirit has not yet lost its divinity, 
its holiness, its sacredness. To be sure, it has long ceased to flutter over 
our heads as a dove; to be sure, it no longer gladdens its saints alone, but 
lets itself be caught by the laity too; but as spirit of humanity, as spirit 
of Man, it remains still an \textit{alien} spirit to me or you, still far from 
becoming our unrestricted \textit{property}, which we dispose of at our 
pleasure. However, one thing certainly happened, and visibly guided the 
progress of post-Christian history: this one thing was the endeavor to make 
the Holy Spirit \textit{more human}, and bring it nearer to men, or men to it. 
Through this it came about that at last it could be conceived as the "{}spirit 
of humanity,"{} and, under different expressions like "{}idea of humanity, 
mankind, humaneness, general philanthropy,"{} appeared more attractive, more 
familiar, and more accessible.

Would not one think that now everybody could possess the Holy Spirit, take up 
into himself the idea of humanity, bring mankind to form and existence in 
himself?

No, the spirit is not stripped of its holiness and robbed of its 
unapproachableness, is not accessible to us, not our property; for the spirit 
of humanity is not \textit{my} spirit. My \textit{ideal} it may be, and as a 
thought I call it mine; the \textit{thought} of humanity is my property, and I 
prove this sufficiently by propounding it quite according to my views, and 
shaping it today so, tomorrow otherwise; we represent it to ourselves in the 
most manifold ways. But it is at the same time an entail, which I cannot 
alienate nor get rid of.

Among many transformations, the Holy Spirit became in time the 
\textit{"{}absolute idea"{}}, which again in manifold refractions split into 
the different ideas of philanthropy, reasonableness, civic virtue, etc.

But can I call the idea my property if it is the idea of humanity, and can I 
consider the Spirit as vanquished if I am to serve it, "{}sacrifice myself"{} 
to it? Antiquity, at its close, had gained its ownership of the world only 
when it had broken the world's overpoweringness and "{}divinity,"{} recognized 
the world's powerlessness and "{}vanity."{}

The case with regard to the \textit{spirit} corresponds. When I have degraded 
it to a \textit{spook} and its control over me to a \textit{cranky notion}, 
then it is to be looked upon as having lost its sacredness, its holiness, its 
divinity, and then I \textit{use} it, as one uses \textit{nature} at pleasure 
without scruple.

The "{}nature of the case,"{} the "{}concept of the relationship,"{} is to 
guide me in dealing with the case or in contracting the relation. As if a 
concept of the case existed on its own account, and was not rather the concept 
that one forms of the case! As if a relation which we enter into was not, by 
the uniqueness of those who enter into it, itself unique! As if it depended on 
how others stamp it! But, as people separated the "{}essence of Man"{} from 
the real man, and judged the latter by the former, so they also separate his 
action from him, and appraise it by "{}human value."{} \textit{Concepts} are 
to decide everywhere, concepts to regulate life, concepts to \textit{rule}. 
This is the religious world, to which Hegel gave a systematic expression, 
bringing method into the nonsense and completing the conceptual precepts into 
a rounded, firmly-based dogmatic. Everything is sung according to concepts, 
and the real man, \textit{i.e.} I, am compelled to live according to these 
conceptual laws. Can there be a more grievous dominion of law, and did not 
Christianity confess at the very beginning that it meant only to draw 
Judaism's dominion of law tighter? ("{}Not a letter of the law shall be 
lost!"{})

Liberalism simply brought other concepts on the carpet; human instead of 
divine, political instead of ecclesiastical, "{}scientific"{} instead of 
doctrinal, or, more generally, real concepts and eternal laws instead of 
"{}crude dogmas"{} and precepts.

Now nothing but \textit{mind} rules in the world. An innumerable multitude of 
concepts buzz about in people's heads, and what are those doing who endeavor 
to get further? They are negating these concepts to put new ones in their 
place! They are saying: "{}You form a false concept of right, of the State, of 
man, of liberty, of truth, of marriage, etc.; the concept of right, etc., is 
rather that one which we now set up."{} Thus the confusion of concepts moves 
forward.

The history of the world has dealt cruelly with us, and the spirit has 
obtained an almighty power. You must have regard for my miserable shoes, which 
could protect your naked foot, my salt, by which your potatoes would become 
palatable, and my state-carriage, whose possession would relieve you of all 
need at once; you must not reach out after them. Man is to recognize the 
\textit{independence} of all these and innumerable other things: they are to 
rank in his mind as something that cannot be seized or approached, are to be 
kept away from him. He must have regard for it, respect it; woe to him if he 
stretches out his fingers desirously; we call that "{}being light-fingered!"{}

How beggarly little is left us, yes, how really nothing! Everything has been 
removed, we must not venture on anything unless it is given us; we continue to 
live only by the \textit{grace} of the giver. You must not pick up a pin, 
unless indeed you have got \textit{leave} to do so. And got it from whom? From 
\textit{respect!} Only when this lets you have it as property, only when you 
can \textit{respect} it as property, only then may you take it. And again, you 
are not to conceive a thought, speak a syllable, commit an action, that should 
have their warrant in you alone, instead of receiving it from morality or 
reason or humanity. Happy \textit{unconstraint} of the desirous man, how 
mercilessly people have tried to slay you on the altar of \textit{constraint!}

But around the altar rise the arches of a church, and its walls keep moving 
further and further out. What they enclose is \textit{sacred}. You can no 
longer get to it, no longer touch it. Shrieking with the hunger that devours 
you, you wander round about these walls in search of the little that is 
profane, and the circles of your course keep growing more and more extended. 
Soon that church will embrace the whole world, and you be driven out to the 
extreme edge; another step, and the \textit{world of the sacred} has 
conquered: you sink into the abyss. Therefore take courage while it is yet 
time, wander about no longer in the profane where now it is dry feeding, dare 
the leap, and rush in through the gates into the sanctuary itself. If you 
\textit{devour the sacred}, you have made it your \textit{own!} Digest the 
sacramental wafer, and you are rid of it!

\section[3. The Free]{\centering 3. The Free}

The ancients and the moderns having been presented above in two divisions, it 
may seem as if the free were here to be described in a third division as 
independent and distinct. This is not so. The free are only the more modern 
and most modern among the "{}moderns,"{} and are put in a separate division 
merely because they belong to the present, and what is present, above all, 
claims our attention here. I give "{}the free"{} only as a translation of 
"{}the liberals,"{} but must with regard to the concept of freedom (as in 
general with regard to so many other things whose anticipatory introduction 
cannot be avoided) refer to what comes later.

\subsection[\S{}1. Political Liberalism]{\centering \S{}1. Political Liberalism}

After the chalice of so-called absolute monarchy had been drained down to the 
dregs, in the eighteenth century people became aware that their drink did not 
taste human -- too clearly aware not to begin to crave a different cup. Since 
our fathers were "{}human beings"{} after all, they at last desired also to be 
regarded as such.

Whoever sees in us something else than human beings, in him we likewise will 
not see a human being, but an inhuman being, and will meet him as an unhuman 
being; on the other hand, whoever recognizes us as human beings and protects 
us against the danger of being treated inhumanly, him we will honor as our 
true protector and guardian.

Let us then hold together and protect the man in each other; then we find the 
necessary protection in our \textit{holding together}, and in ourselves, 
\textit{those who hold together}, a fellowship of those who know their human 
dignity and hold together as "{}human beings."{} Our holding together is the 
\textit{State}; we who hold together are the \textit{nation}.

In our being together as nation or State we are only human beings. How we 
deport ourselves in other respects as individuals, and what self-seeking 
impulses we may there succumb to, belongs solely to our \textit{private} life; 
our public or State life is a \textit{purely human} one. Everything un-human 
or "{}egoistic"{} that clings to us is degraded to a "{}private matter"{} and 
we distinguish the State definitely from "{}civil society,"{} which is the 
sphere of "{}egoism's"{} activity.

The true man is the nation, but the individual is always an egoist. Therefore 
strip off your individuality or isolation wherein dwells discord and egoistic 
inequality, and consecrate yourselves wholly to the true man -- the nation or 
the State. Then you will rank as men, and have all that is man's; the State, 
the true man, will entitle you to what belongs to it, and give you the 
"{}rights of man"{}; Man gives you his rights!

So runs the speech of the commonalty.

The commonalty\footnote{[Or "{}citizenhood."{} The word [\textit{das 
Buergertum}] means either the condition of being a citizen, or citizen-like 
principles, of the body of citizens or of the middle or business class, the 
\textit{bourgeoisie}.]} is nothing else than the thought that the State is all 
in all, the true man, and that the individual's human value consists in being 
a citizen of the State. In being a good citizen he seeks his highest honor; 
beyond that he knows nothing higher than at most the antiquated -- "{}being a 
good Christian."{}

The commonalty developed itself in the struggle against the privileged 
classes, by whom it was cavalierly treated as "{}third estate"{} and 
confounded with the \textit{canaille}. In other words, up to this time the 
State had recognized caste.\footnote{[\textit{Man hatte im Staate "{}die 
ungleiche Person angesehen,"{}} there had been "{}respect of unequal 
persons"{} in the State.]} The son of a nobleman was selected for posts to 
which the most distinguished commoners aspired in vain. The civic feeling 
revolted against this. No more distinction, no giving preference to persons, 
no difference of classes! Let all be alike! No \textit{separate interest} is 
to be pursued longer, but the \textit{general interest of all}. The State is 
to be a fellowship of free and equal men, and every one is to devote himself 
to the "{}welfare of the whole,"{} to be dissolved in the \textit{State}, to 
make the State his end and ideal. State! State! so ran the general cry, and 
thenceforth people sought for the "{}right form of State,"{} the best 
constitution, and so the State in its best conception. The thought of the 
State passed into all hearts and awakened enthusiasm; to serve it, this 
mundane god, became the new divine service and worship. The properly 
\textit{political} epoch had dawned. To serve the State or the nation became 
the highest ideal, the State's interest the highest interest, State service 
(for which one does not by any means need to be an official) the highest 
honor.

So then the separate interests and personalities had been scared away, and 
sacrifice for the State had become the shibboleth. One must give up 
\textit{himself}, and live only for the State. One must act 
"{}disinterestedly,"{} not want to benefit \textit{himself}, but the State. 
Hereby the latter has become the true person. before whom the individual 
personality vanishes; not I live, but it lives in me. Therefore, in comparison 
with the former self-seeking, this was unselfishness and 
\textit{impersonality} itself. Before this god -- State -- all egoism 
vanished, and before it all were equal; they were without any other 
distinction -- men, nothing but men.

The Revolution took fire from the inflammable material of \textit{property}. 
The government needed money. Now it must prove the proposition that it 
\textit{is absolute}, and so master of all property, sole proprietor; it must 
\textit{take} to itself \textit{its} money, which was only in the possession 
of the subjects, not their property. Instead of this, it calls States-general, 
to have this money \textit{granted} to it. The shrinking from strictly logical 
action destroyed the illusion of an \textit{absolute} government; he who must 
have something "{}granted"{} to him cannot be regarded as absolute. The 
subjects recognized that they were \textit{real proprietors}, and that it was 
\textit{their} money that was demanded. Those who had hitherto been subjects 
attained the consciousness that they were \textit{proprietors}. Bailly depicts 
this in a few words: "{}If you cannot dispose of my property without my 
assent, how much less can you of my person, of all that concerns my mental and 
social position? All this is my property, like the piece of land that I till; 
and I have a right, an interest, to make the laws myself."{} Bailly's words 
sound, certainly, as if \textit{every one} was a proprietor now. However, 
instead of the government, instead of the prince, \textit{the -- nation} now 
became proprietor and master. From this time on the ideal is spoken of as -- 
"{}popular liberty"{} -- "{}a free people,"{} etc.

As early as July 8, 1789, the declaration of the bishop of Autun and Barrere 
took away all semblance of the importance of each and every 
\textit{individual} in legislation; it showed the complete 
\textit{powerlessness} of the constituents; the \textit{majority of the 
representatives} has become \textit{master}. When on July 9 the plan for 
division of the work on the constitution is proposed, Mirabeau remarks that 
"{}the government has only power, no rights; only in the \textit{people} is 
the source of all \textit{right} to be found."{} On July 16 this same Mirabeau 
exclaims: "{}Is not the people the source of all \textit{power?"{}} The 
source, therefore, of all right, and the source of all -- 
power!\footnote{[\textit{Gewalt}, a word which is also commonly used like the 
English "{}violence,"{} denoting especially unlawful violence.]} By the way, 
here the substance of "{}right"{} becomes visible; it is -- \textit{power}. 
"{}He who has power has right."{}

The commonalty is the heir of the privileged classes. In fact, the rights of 
the barons, which were taken from them as "{}usurpations,"{} only passed over 
to the commonalty. For the commonalty was now called the "{}nation."{} "{}Into 
the hands of the nation"{} all \textit{prerogatives} were given back. Thereby 
they ceased to be "{}prerogatives"{}:\footnote{[\textit{Vorrechte}]} they 
became "{}rights."{}\footnote{[\textit{Rechte}]} From this time on the nation 
demands tithes, compulsory services; it has inherited the lord's court, the 
rights of vert and venison, the -- serfs. The night of August 4 was the 
death-night of privileges or "{}prerogatives"{} (cities, communes, boards of 
magistrates, were also privileged, furnished with prerogatives and seigniorial 
rights), and ended with the new morning of "{}right,"{} the "{}rights of the 
State,"{} the "{}rights of the nation."{}

The monarch in the person of the "{}royal master"{} had been a paltry monarch 
compared with this new monarch, the "{}sovereign nation."{} This 
\textit{monarchy} was a thousand times severer, stricter, and more consistent. 
Against the new monarch there was no longer any right, any privilege at all; 
how limited the "{}absolute king"{} of the \textit{ancien regime} looks in 
comparison! The Revolution effected the transformation of \textit{limited 
monarchy} into \textit{absolute monarchy}. From this time on every right that 
is not conferred by this monarch is an "{}assumption"{}; but every prerogative 
that he bestows, a "{}right."{} The times demanded \textit{absolute royalty}, 
absolute monarchy; therefore down fell that so-called absolute royalty which 
had so little understood how to become absolute that it remained limited by a 
thousand little lords.

What was longed for and striven for through thousands of years -- to wit, to 
find that absolute lord beside whom no other lords and lordlings any longer 
exist to clip his power -- the \textit{bourgeoisie} has brought to pass. It 
has revealed the Lord who alone confers "{}rightful titles,"{} and without 
whose warrant \textit{nothing is justified}. "{}So now we know that an idol is 
nothing in the world, and that there is no other god save the 
one."{}\footnote{1 Corinthians 8. 4.}

Against \textit{right} one can no longer, as against a right, come forward 
with the assertion that it is "{}a wrong."{} One can say now only that it is a 
piece of nonsense, an illusion. If one called it wrong, one would have to set 
up \textit{another right} in opposition to it, and measure it by this. If, on 
the contrary, one rejects right as such, right in and of itself, altogether, 
then one also rejects the concept of wrong, and dissolves the whole concept of 
right (to which the concept of wrong belongs).

What is the meaning of the doctrine that we all enjoy "{}equality of political 
rights"{}? Only this -- that the State has no regard for my person, that to it 
I, like every other, am only a man, without having another significance that 
commands its deference. I do not command its deference as an aristocrat, a 
nobleman's son, or even as heir of an official whose office belongs to me by 
inheritance (as in the Middle Ages countships, etc., and later under absolute 
royalty, where hereditary offices occur). Now the State has an innumerable 
multitude of rights to give away, \textit{e. g.} the right to lead a 
battalion, a company, etc.; the right to lecture at a university, and so 
forth; it has them to give away because they are its own, \textit{i.e.}, State 
rights or "{}political"{} rights. Withal, it makes no difference to it to whom 
it gives them, if the receiver only fulfills the duties that spring from the 
delegated rights. To it we are all of us all right, and -- \textit{equal --} 
one worth no more and no less than another. It is indifferent to me who 
receives the command of the army, says the sovereign State, provided the 
grantee understands the matter properly. "{}Equality of political rights"{} 
has, consequently, the meaning that every one may acquire every right that the 
State has to give away, if only he fulfills the conditions annexed thereto -- 
conditions which are to be sought only in the nature of the particular right, 
not in a predilection for the person (\textit{persona grata}): the nature of 
the right to become an officer brings with it, \textit{e. g.} the necessity 
that one possess sound limbs and a suitable measure of knowledge, but it does 
not have noble birth as a condition; if, on the other hand, even the most 
deserving commoner could not reach that station, then an inequality of 
political rights would exist. Among the States of today one has carried out 
that maxim of equality more, another less.

The monarchy of estates (so I will call absolute royalty, the time of the 
kings before the revolution) kept the individual in dependence on a lot of 
little monarchies. These were fellowships (societies) like the guilds, the 
nobility, the priesthood, the burgher class, cities, communes. Everywhere the 
individual must regard himself \textit{first} as a member of this little 
society, and yield unconditional obedience to its spirit, the \textit{esprit 
de corps}, as his monarch. More, \textit{e. g.} than the individual nobleman 
himself must his family, the honor of his race, be to him. Only by means of 
his \textit{corporation}, his estate, did the individual have relation to the 
greater corporation, the State -- as in Catholicism the individual deals with 
God only through the priest. To this the third estate now, showing courage to 
negate \textit{itself as an estate}, made an end. It decided no longer to be 
and be called an \textit{estate} beside other estates, but to glorify and 
generalize itself into the \textit{"{}nation."{}} Hereby it created a much 
more complete and absolute monarchy,' and the entire previously ruling 
\textit{principle of estates}, the principle of little monarchies inside the 
great, went down. Therefore it cannot be said that the Revolution was a 
revolution against the first two privileged estates. It was against the little 
monarchies of estates in general. But, if the estates and their despotism were 
broken (the king too, we know, was only a king of estates, not a 
citizen-king), the individuals freed from the inequality of estate were left. 
Were they now really to be without estate and "{}out of gear,"{} no longer 
bound by any estate, without a general bond of union? No, for the third estate 
had declared itself the nation only in order not to remain an estate 
\textit{beside} other estates, but to become the \textit{sole estate}. This 
sole \textit{estate} is the nation, the \textit{"{}State."{}} What had the 
individual now become? A political Protestant, for he had come into immediate 
connection with his God, the State. He was no longer, as an aristocrat, in the 
monarchy of the nobility; as a mechanic, in the monarchy of the guild; but he, 
like all, recognized and acknowledged only -- \textit{one lord}, the State, as 
whose servants they all received the equal title of honor, "{}citizen."{}

The \textit{bourgeoisie} is the aristocracy of DESERT; its motto, "{}Let 
desert wear its crowns."{} It fought against the "{}lazy"{} aristocracy, for 
according to it (the industrious aristocracy acquired by industry and desert) 
it is not the "{}born"{} who is free, nor yet I who am free either, but the 
"{}deserving"{} man, the honest \textit{servant} (of his king; of the State; 
of the people in constitutional States). Through \textit{service} one acquires 
freedom, \textit{i. e.}, acquires "{}deserts,"{} even if one served -- mammon. 
One must deserve well of the State, \textit{i.e.} of the principle of the 
State, of its moral spirit. He who \textit{serves} this spirit of the State is 
a good citizen, let him live to whatever honest branch of industry he will. In 
its eyes innovators practice a "{}breadless art."{} Only the "{}shopkeeper"{} 
is "{}practical,"{} and the spirit that chases after public offices is as much 
the shopkeeping spirit as is that which tries in trade to feather its nest or 
otherwise to become useful to itself and anybody else.

But, if the deserving count as the free (for what does the comfortable 
commoner, the faithful office-holder, lack of that freedom that his heart 
desires?), then the "{}servants"{} are the -- free. The obedient servant is 
the free man! What glaring nonsense! Yet this is the sense of the 
\textit{bourgeoisie}, and its poet, Goethe, as well as its philosopher, Hegel, 
succeeded in glorifying the dependence of the subject on the object, obedience 
to the objective world. He who only serves the cause, "{}devotes himself 
entirely to it,"{} has the true freedom. And among thinkers the cause was -- 
\textit{reason}, that which, like State and Church, gives -- general laws, and 
puts the individual man in irons by the \textit{thought of humanity}. It 
determines what is "{}true,"{} according to which one must then act. No more 
"{}rational"{} people than the honest servants, who primarily are called good 
citizens as servants of the State.

Be rich as Croesus or poor as Job -- the State of the commonalty leaves that 
to your option; but only have a "{}good disposition."{} This it demands of 
you, and counts it its most urgent task to establish this in all. Therefore it 
will keep you from "{}evil promptings,"{} holding the "{}ill-disposed"{} in 
check and silencing their inflammatory discourses under censors' 
canceling-marks or press-penalties and behind dungeon walls, and will, on the 
other hand, appoint people of "{}good disposition"{} as censors, and in every 
way have a \textit{moral influence} exerted on you by "{}well-disposed and 
well-meaning"{} people. If it has made you deaf to evil promptings, then it 
opens your ears again all the more diligently to good \textit{promptings}.

With the time of the \textit{bourgeoisie} begins that of \textit{liberalism}. 
People want to see what is "{}rational,"{} "{}suited to the times,"{} etc., 
established everywhere. The following definition of liberalism, which is 
supposed to be pronounced in its honor, characterizes it completely: 
"{}Liberalism is nothing else than the knowledge of reason, applied to our 
existing relations."{}\footnote{\textit{"{}Ein und zwanzig Bogen}"{}, p. 12} 
Its aim is a "{}rational order,"{} a "{}moral behavior,"{} a "{}limited 
freedom,"{} not anarchy, lawlessness, selfhood. But, if reason rules, then the 
\textit{person} succumbs. Art has for a long time not only acknowledged the 
ugly, but considered the ugly as necessary to its existence, and takes it up 
into itself; it needs the villain. In the religious domain, too, the extremest 
liberals go so far that they want to see the most religious man regarded as a 
citizen -- \textit{i. e.}, the religious villain; they want to see no more of 
trials for heresy. But against the "{}rational law"{} no one is to rebel, 
otherwise he is threatened with the severest penalty. What is wanted is not 
free movement and realization of the person or of me, but of reason -- 
\textit{i.e.} a dominion of reason, a dominion. The liberals are 
\textit{zealots}, not exactly for the faith, for God, but certainly for 
\textit{reason}, their master. They brook no lack of breeding, and therefore 
no self-development and self- determination; they \textit{play the guardian} 
as effectively as the most absolute rulers.

"{}Political liberty,"{} what are we to understand by that? Perhaps the 
individual's independence of the State and its laws? No; on the contrary, the 
individual's \textit{subjection} in the State and to the State's laws. But why 
"{}liberty"{}? Because one is no longer separated from the State by 
intermediaries, but stands in direct and immediate relation to it; because one 
is a -- citizen, not the subject of another, not even of the king as a person, 
but only in his quality as "{}supreme head of the State."{} Political liberty, 
this fundamental doctrine of liberalism, is nothing but a second phase of -- 
Protestantism, and runs quite parallel with "{}religious 
liberty."{}\footnote{Louis Blanc says (\textit{"{}Histoire des dix Ans}"{}, I, 
p. 138) of the time of the Restoration: \textit{"{}Le protestantisme devint le 
fond des id\'ees et des moeurs}."{}} Or would it perhaps be right to 
understand by the latter an independence of religion? Anything but that. 
Independence of intermediaries is all that it is intended to express, 
independence of mediating priests, the abolition of the "{}laity,"{} and so, 
direct and immediate relation to religion or to God. Only on the supposition 
that one has religion can he enjoy freedom of religion; freedom of religion 
does not mean being without religion, but inwardness of faith, unmediated 
intercourse with God. To him who is "{}religiously free"{} religion is an 
affair of the heart, it is to him his \textit{own affair}, it is to him a 
"{}sacredly serious matter."{} So, too, to the "{}politically free"{} man the 
State is a sacredly serious matter; it is his heart's affair, his chief 
affair, his own affair.

Political liberty means that the \textit{polis}, the State, is free; freedom 
of religion that religion is free, as freedom of conscience signifies that 
conscience is free; not, therefore, that I am free from the State, from 
religion, from conscience, or that I am \textit{rid} of them. It does not mean 
\textit{my} liberty, but the liberty of a power that rules and subjugates me; 
it means that one of my \textit{despots}, like State, religion, conscience, is 
free. State, religion, conscience, these despots, make me a slave, and 
\textit{their} liberty is my slavery. That in this they necessarily follow the 
principle, "{}the end hallows the means,"{} is self-evident. If the welfare of 
the State is the end, war is a hallowed means; if justice is the State's end, 
homicide is a hallowed means, and is called by its sacred name, 
"{}execution"{}; the sacred State \textit{hallows} everything that is 
serviceable to it.

"{}Individual liberty,"{} over which civic liberalism keeps jealous watch, 
does not by any means signify a completely free self-determination, by which 
actions become altogether \textit{mine}, but only independence of 
\textit{persons}. Individually free is he who is responsible to no 
\textit{man}. Taken in this sense -- and we are not allowed to understand it 
otherwise -- not only the ruler is individually free, \textit{i.e.}, 
\textit{irresponsible} toward men ("{}before God,"{} we know, he acknowledges 
himself responsible), but all who are "{}responsible only to the law."{} This 
kind of liberty was won through the revolutionary movement of the century -- 
to wit, independence of arbitrary will, or \textit{tel est notre plaisir}. 
Hence the constitutional prince must himself be stripped of all personality, 
deprived of all individual decision, that he may not as a person, as an 
\textit{individual man}, violate the "{}individual liberty"{} of others. The 
\textit{personal will of the ruler} has disappeared in the constitutional 
prince; it is with a right feeling, therefore, that absolute princes resist 
this. Nevertheless these very ones profess to be in the best sense 
"{}Christian princes."{} For this, however, they must become a \textit{purely 
spiritual} power, as the Christian is subject only to \textit{spirit} ("{}God 
is spirit"{}). The purely spiritual power is consistently represented only by 
the constitutional prince, he who, without any personal significance, stands 
there spiritualized to the degree that he can rank as a sheer, uncanny 
"{}spirit,"{} as an \textit{idea}. The constitutional king is the truly 
\textit{Christian} king, the genuine, consistent carrying-out of the Christian 
principle. In the constitutional monarchy individual dominion -- \textit{i.e.} 
a real ruler that \textit{wills} -- has found its end; here, therefore, 
\textit{individual liberty} prevails, independence of every individual 
dictator, of everyone who could dictate to me with a \textit{tel est notre 
plaisir}. It is the completed \textit{Christian} State-life, a spiritualized 
life.

The behavior of the commonalty is \textit{liberal} through and through. Every 
\textit{personal} invasion of another's sphere revolts the civic sense; if the 
citizen sees that one is dependent on the humor, the pleasure, the will of a 
man as individual (\textit{i.e.} as not as authorized by a "{}higher 
power"{}), at once he brings his liberalism to the front and shrieks about 
"{}arbitrariness."{} In fine, the citizen asserts his freedom from what is 
called \textit{orders} (\textit{ordonnance})\textit{: "{}No one has any 
business to give me -- orders!"{} Orders} carries the idea that what I am to 
do is another man's will, while law does not express a personal authority of 
another. The liberty of the commonalty is liberty or independence from the 
will of another person, so-called personal or individual liberty; for being 
personally free means being only so free that no other person can dispose of 
mine, or that what I may or may not do does not depend on the personal decree 
of another. The liberty of the press, \textit{e. g.}, is such a liberty of 
liberalism, liberalism fighting only against the coercion of the censorship as 
that of personal wilfulness, but otherwise showing itself extremely inclined 
and willing to tyrannize over the press by "{}press laws"{}; \textit{i.e.} the 
civic liberals want liberty of writing \textit{for themselves}; for, as they 
are \textit{law-abiding}, their writings will not bring them under the law. 
Only liberal matter, \textit{i.e.} only lawful matter, is to be allowed to be 
printed; otherwise the "{}press laws"{} threaten "{}press-penalties."{} If one 
sees personal liberty assured, one does not notice at all how, if a new issue 
happens to arise, the most glaring unfreedom becomes dominant. For one is rid 
of \textit{orders} indeed, and "{}no one has any business to give us 
orders,"{} but one has become so much the more submissive to the -- 
\textit{law}. One is enthralled now in due legal form.

In the citizen-State there are only "{}free people,"{} who are 
\textit{compelled} to thousands of things (\textit{e. g.} to deference, to a 
confession of faith, etc.). But what does that amount to? Why, it is only the 
-- State, the law, not any man, that compels them!

What does the commonalty mean by inveighing against every personal order, 
\textit{i.e.} every order not founded on the "{}cause,"{} on "{}reason"{}? It 
is simply fighting in the interest of the 
"{}cause"{}\footnote{[\textit{Sache}, which commonly means \textit{thing}].} 
against the dominion of "{}persons"{}! But the mind's cause is the rational, 
good, lawful, etc.; that is the "{}good cause."{} The commonalty wants an 
\textit{impersonal} ruler.

Furthermore, if the principle is this, that only the cause is to rule man -- 
to wit, the cause of morality, the cause of legality, etc., then no personal 
balking of one by the other may be authorized either (as formerly, \textit{e. 
g.} the commoner was balked of the aristocratic offices, the aristocrat of 
common mechanical trades, etc.); \textit{free competition} must exist. Only 
through the thing\footnote{[\textit{Sache}]} can one balk another (\textit{e. 
g.} the rich man balking the impecunious man by money, a thing), not as a 
person. Henceforth only one lordship, the lordship of the \textit{State}, is 
admitted; personally no one is any longer lord of another. Even at birth the 
children belong to the State, and to the parents only in the name of the 
State, which \textit{e. g.} does not allow infanticide, demands their baptism 
etc.

But all the State's children, furthermore, are of quite equal account in its 
eyes ("{}civic or political equality"{}), and they may see to it themselves 
how they get along with each other; they may \textit{compete}.

Free competition means nothing else than that every one can present himself, 
assert himself, fight, against another. Of course the feudal party set itself 
against this, as its existence depended on an absence of competition. The 
contests in the time of the Restoration in France had no other substance than 
this -- that the \textit{bourgeoisie} was struggling for free competition, and 
the feudalists were seeking to bring back the guild system.

Now, free competition has won, and against the guild system it had to win. 
(See below for the further discussion.)

If the Revolution ended in a reaction, this only showed what the Revolution 
\textit{really was}. For every effort arrives at reaction when it 
\textit{comes to discreet reflection}, and storms forward in the original 
action only so long as it is an \textit{intoxication}, an "{}indiscretion."{} 
"{}Discretion"{} will always be the cue of the reaction, because discretion 
sets limits, and liberates what was really wanted, \textit{i. e.}, the 
principle, from the initial "{}unbridledness"{} and "{}unrestrainedness."{} 
Wild young fellows, bumptious students, who set aside all considerations, are 
\textit{really} Philistines, since with them, as with the latter, 
considerations form the substance of their conduct; only that as swaggerers 
they are mutinous against considerations and in negative relations to them, 
but as Philistines, later, they give themselves up to considerations and have 
positive relations to them. In both cases all their doing and thinking turns 
upon "{}considerations,"{} but the Philistine is \textit{reactionary} in 
relation to the student; he is the wild fellow come to discreet reflection, as 
the latter is the unreflecting Philistine. Daily experience confirms the truth 
of this transformation, and shows how the swaggerers turn to Philistines in 
turning gray.

So, too, the so-called reaction in Germany gives proof that it was only the 
\textit{discreet} continuation of the warlike jubilation of liberty.

The Revolution was not directed against \textit{the established}, but against 
the \textit{establishment in question}, against a \textit{particular} 
establishment. It did away with \textit{this} ruler, not with \textit{the} 
ruler -- on the contrary, the French were ruled most inexorably; it killed the 
old vicious rulers, but wanted to confer on the virtuous ones a securely 
established position, \textit{i. e.}, it simply set virtue in the place of 
vice. (Vice and virtue, again, are on their part distinguished from each other 
only as a wild young fellow from a Philistine.) Etc.

To this day the revolutionary principle has gone no farther than to assail 
only \textit{one or another} particular establishment, \textit{i.e.} be 
\textit{reformatory}. Much as may be \textit{improved}, strongly as 
"{}discreet progress"{} may be adhered to, always there is only a \textit{new 
master} set in the old one's place, and the overturning is a -- building up. 
We are still at the distinction of the young Philistine from the old one. The 
Revolution began in \textit{bourgeois} fashion with the uprising of the third 
estate, the middle class; in \textit{bourgeois} fashion it dries away. It was 
not the \textit{individual man --} and he alone is \textit{Man} -- that became 
free, but the \textit{citizen}, the \textit{citoyen}, the \textit{political} 
man, who for that very reason is not \textit{Man} but a specimen of the human 
species, and more particularly a specimen of the species Citizen, a 
\textit{free citizen}.

In the Revolution it was not the \textit{individual} who acted so as to affect 
the world's history, but a \textit{people}; the \textit{nation}, the sovereign 
nation, wanted to effect everything. A fancied \textit{I}, an idea, \textit{e. 
g.} the nation is, appears acting; the individuals contribute themselves as 
tools of this idea, and act as "{}citizens."{}

The commonalty has its power, and at the same time its limits, in the 
\textit{fundamental law of the State}, in a charter, in a 
legitimate\footnote{[Or "{}righteous."{} German \textit{rechtlich}].} or 
"{}just"{}\footnote{[\textit{gerecht}]} prince who himself is guided, and 
rules, according to "{}rational laws,"{} in short, in \textit{legality}. The 
period of the \textit{bourgeoisie} is ruled by the British spirit of legality. 
An assembly of provincial estates, \textit{e. g.} is ever recalling that its 
authorization goes only so and so far, and that it is called at all only 
through favor and can be thrown out again through disfavor. It is always 
reminding itself of its -- \textit{vocation}. It is certainly not to be denied 
that my father begot me; but, now that I am once begotten, surely his purposes 
in begetting do not concern me a bit and, whatever he may have \textit{called} 
me to, I do what I myself will. Therefore even a called assembly of estates, 
the French assembly in the beginning of the Revolution, recognized quite 
rightly that it was independent of the caller. It \textit{existed}, and would 
have been stupid if it did not avail itself of the right of existence, but 
fancied itself dependent as on a father. The called one no longer has to ask 
"{}what did the caller want when he created me?"{} but "{}what do I want after 
I have once followed the call?"{} Not the caller, not the constituents, not 
the charter according to which their meeting was called out, nothing will be 
to him a sacred, inviolable power. He is \textit{authorized} for everything 
that is in his power; he will know no restrictive "{}authorization,"{} will 
not want to be \textit{loyal}. This, if any such thing could be expected from 
chambers at all, would give a completely \textit{egoistic} chamber, severed 
from all navel-string and without consideration. But chambers are always 
devout, and therefore one cannot be surprised if so much half-way or 
undecided,

\textit{i. e.}, hypocritical, "{}egoism"{} parades in them.

The members of the estates are to remain within the \textit{limits} that are 
traced for them by the charter, by the king's will, etc. If they will not or 
can not do that, then they are to "{}step out."{} What dutiful man could act 
otherwise, could put himself, his conviction, and his will as the 
\textit{first} thing? Who could be so immoral as to want to assert 
\textit{himself}, even if the body corporate and everything should go to ruin 
over it? People keep carefully within the limits of their 
\textit{authorization}; of course one must remain within the limits of his 
\textit{power} anyhow, because no one can do more than he can. "{}My power, 
or, if it be so, powerlessness, be my sole limit, but authorizations only 
restraining -- precepts? Should I profess this all-subversive view? No, I am a 
-- law-abiding citizen!"{}

The commonalty professes a morality which is most closely connected with its 
essence. The first demand of this morality is to the effect that one should 
carry on a solid business, an honourable trade, lead a moral life. Immoral, to 
it, is the sharper, the, demirep, the thief, robber, and murderer, the 
gamester, the penniless man without a situation, the frivolous man. The 
doughty commoner designates the feeling against these "{}immoral"{} people as 
his "{}deepest indignation."{}

All these lack settlement, the \textit{solid} quality of business, a solid, 
seemly life, a fixed income, etc.; in short, they belong, because their 
existence does not rest on a \textit{secure basis} to the dangerous 
"{}individuals or isolated persons,"{} to the dangerous \textit{proletariat}; 
they are "{}individual bawlers"{} who offer no "{}guarantee"{} and have 
"{}nothing to lose,"{} and so nothing to risk. The forming of family ties, 
\textit{e. g.}, \textit{binds} a man: he who is bound furnishes security, can 
be taken hold of; not so the street-walker. The gamester stakes everything on 
the game, ruins himself and others -- no guarantee. All who appear to the 
commoner suspicious, hostile, and dangerous might be comprised under the name 
"{}vagabonds"{}; every vagabondish way of living displeases him. For there are 
intellectual vagabonds too, to whom the hereditary dwelling-place of their 
fathers seems too cramped and oppressive for them to be willing to satisfy 
themselves with the limited space any more: instead of keeping within the 
limits of a temperate style of thinking, and taking as inviolable truth what 
furnishes comfort and tranquillity to thousands, they overlap all bounds of 
the traditional and run wild with their impudent criticism and untamed mania 
for doubt, these extravagating vagabonds. They form the class of the unstable, 
restless, changeable, \textit{i.e.} of the \textit{prol\'etariat}, and, if 
they give voice to their unsettled nature, are called "{}unruly fellows."{}

Such a broad sense has the so-called \textit{proletariat}, or pauperism. How 
much one would err if one believed the commonalty to be desirous of doing away 
with poverty (pauperism) to the best of its ability! On the contrary, the good 
citizen helps himself with the incomparably comforting conviction that "{}the 
fact is that the good things of fortune are unequally divided and will always 
remain so -- according to God's wise decree."{} The poverty which surrounds 
him in every alley does not disturb the true commoner further than that at 
most he clears his account with it by throwing an alms, or finds work and food 
for an "{}honest and serviceable"{} fellow. But so much the more does he feel 
his quiet enjoyment clouded by \textit{innovating} and \textit{discontented} 
poverty, by those poor who no longer behave quietly and endure, but begin to 
\textit{run wild} and become restless. Lock up the vagabond, thrust the 
breeder of unrest into the darkest dungeon! He wants to "{}arouse 
dissatisfaction and incite people against existing institutions"{} in the 
State -- stone him, stone him!

But from these identical discontented ones comes a reasoning somewhat as 
follows: It need not make any difference to the "{}good citizens"{} who 
protects them and their principles, whether an absolute king or a 
constitutional one, a republic, if only they are protected. And what is their 
principle, whose protector they always "{}love"{}? Not that of labor; not that 
of birth either. But, that of \textit{mediocrity}, of the golden mean: a 
little birth and a little labor, \textit{i. e.}, an \textit{interest-bearing 
possession}. Possession is here the fixed, the given, inherited (birth); 
interest-drawing is the exertion about it (labor); \textit{laboring capital}, 
therefore. Only no immoderation, no ultra, no radicalism! Right of birth 
certainly, but only hereditary possessions; labor certainly, yet little or 
none at all of one's own, but labor of capital and of the -- subject laborers.

If an age is imbued with an error, some always derive advantage from the 
error, while the rest have to suffer from it. In the Middle Ages the error was 
general among Christians that the church must have all power, or the supreme 
lordship on earth; the hierarchs believed in this "{}truth"{} not less than 
the laymen, and both were spellbound in the like error. But by it the 
hierarchs had the \textit{advantage} of power, the laymen had to 
\textit{suffer} subjection. However, as the saying goes, "{}one learns wisdom 
by suffering"{}; and so the laymen at last learned wisdom and no longer 
believed in the medieval "{}truth."{} -- A like relation exists between the 
commonalty and the laboring class. Commoner and laborer believe in the 
"{}truth"{} of \textit{money}; they who do not possess it believe in it no 
less than those who possess it: the laymen, therefore, as well as the priests.

"{}Money governs the world"{} is the keynote of the civic epoch. A destitute 
aristocrat and a destitute laborer, as "{}starvelings,"{} amount to nothing so 
far as political consideration is concerned; birth and labor do not do it, but 
\textit{money} brings \textit{consideration}.\footnote{[\textit{das Geld gibt 
Geltung}.]} The possessors rule, but the State trains up from the destitute 
its "{}servants,"{} to whom, in proportion as they are to rule (govern) in its 
name, it gives money (a salary).

I receive everything from the State. Have I anything without the 
\textit{State's assent?} What I have without this it \textit{takes} from me as 
soon as it discovers the lack of a "{}legal title."{} Do I not, therefore, 
have everything through its grace, its assent?

On this alone, on the \textit{legal title}, the commonalty rests. The commoner 
is what he is through the \textit{protection of the State}, through the 
State's grace. He would necessarily be afraid of losing everything if the 
State's power were broken.

But how is it with him who has nothing to lose, how with the proletarian? As 
he has nothing to lose, he does not need the protection of the State for his 
"{}nothing."{} He may gain, on the contrary, if that protection of the State 
is withdrawn from the \textit{prot\'eg\'e}.

Therefore the non-possessor will regard the State as a power protecting the 
possessor, which privileges the latter, but does nothing for him, the 
non-possessor, but to -- suck his blood. The State is \textit{a -- commoners' 
State}, is the estate of the commonalty. It protects man not according to his 
labor, but according to his tractableness ("{}loyalty"{}) -- to wit, according 
to whether the rights entrusted to him by the State are enjoyed and managed in 
accordance with the will, \textit{i. e.}, laws, of the State.

Under the \textit{regime} of the commonalty the laborers always fall into the 
hands of the possessors, of those who have at their disposal some bit of the 
State domains (and everything possessible in State domain, belongs to the 
State, and is only a fief of the individual), especially money and land; of 
the capitalists, therefore. The laborer cannot \textit{realize} on his labor 
to the extent of the value that it has for the consumer. "{}Labor is badly 
paid!"{} The capitalist has the greatest profit from it. -- Well paid, and 
more than well paid, are only the labors of those who heighten the splendor 
and \textit{dominion} of the State, the labors of high State 
\textit{servants}. The State pays well that its "{}good citizens,"{} the 
possessors, may be able to pay badly without danger; it secures to itself by 
good payment its servants, out of whom it forms a protecting power, a 
"{}police"{} (to the police belong soldiers, officials of all kinds, 
\textit{e. g.} those of justice, education, etc. -- in short, the whole 
"{}machinery of the State"{}) for the "{}good citizens,"{} and the "{}good 
citizens"{} gladly pay high tax-rates to it in order to pay so much lower 
rates to their laborers.

But the class of laborers, because unprotected in what they essentially are 
(for they do not enjoy the protection of the State as laborers, but as its 
subjects they have a share in the enjoyment of the police, a so-called 
protection of the law), remains a power hostile to this State, this State of 
possessors, this "{}citizen kingship."{} Its principle, labor, is not 
recognized as to its \textit{value}; it is 
exploited,\footnote{[\textit{ausgebeutet}]} a 
spoil\footnote{[\textit{Kriegsbeute}]} of the possessors, the enemy.

The laborers have the most enormous power in their hands, and, if they once 
became thoroughly conscious of it and used it, nothing would withstand them; 
they would only have to stop labor, regard the product of labor as theirs, and 
enjoy it. This is the sense of the labor disturbances which show themselves 
here and there.

The State rests on the -- \textit{slavery of labor}. If \textit{labor} becomes 
\textit{free}. the State is lost.

\subsection[\S{}2. Social Liberalism]{\centering \S{}2. Social Liberalism}

We are freeborn men, and wherever we look we see ourselves made servants of 
egoists! Are we therefore to become egoists too! Heaven forbid! We want rather 
to make egoists impossible! We want to make them all "{}ragamuffins"{}; all of 
us must have nothing, that "{}all may have."{}

So say the Socialists.

Who is this person that you call "{}All"{}? -- It is "{}society"{}! -- But is 
it corporeal, then? -- \textit{We} are its body! -- You? Why, you are not a 
body yourselves -- you, sir, are corporeal to be sure, you too, and you, but 
you all together are only bodies, not a body. Accordingly the united society 
may indeed have bodies at its service, but no one body of its own. Like the 
"{}nation of the politicians"{}, it will turn out to be nothing but a 
"{}spirit,"{} its body only semblance.

The freedom of man is, in political liberalism, freedom from \textit{persons}, 
from personal dominion, from the \textit{master}; the securing of each 
individual person against other persons, personal freedom.

No one has any orders to give; the law alone gives orders.

But, even if the persons have become \textit{equal}, yet their 
\textit{possessions} have not. And yet the poor man \textit{needs the rich}, 
the rich the poor, the former the rich man's money, the latter the poor man's 
labor. So no one needs another as a \textit{person}, but needs him as a 
\textit{giver}, and thus as one who has something to give, as holder or 
possessor. So what he \textit{has} makes the \textit{man}. And in 
\textit{having}, or in "{}possessions,"{} people are unequal.

Consequently, social liberalism concludes, \textit{no one must have}, as 
according to political liberalism \textit{no one was to give orders}; 
\textit{i.e.} as in that case the \textit{State} alone obtained the command, 
so now \textit{society} alone obtains the possessions.

For the State, protecting each one's person and property against the other, 
\textit{separates} them from one another; each one \textit{is} his special 
part and has his special part. He who is satisfied with what he is and has 
finds this state of things profitable; but he who would like to be and have 
more looks around for this "{}more,"{} and finds it in the power of other 
\textit{persons}. Here he comes upon a contradiction; as a person no one is 
inferior to another, and yet one person \textit{has} what another has not but 
would like to have. So, he concludes, the one person is more than the other, 
after all, for the former has what he needs, the latter has not; the former is 
a rich man, the latter a poor man.

He now asks himself further, are we to let what we rightly buried come to life 
again? Are we to let this circuitously restored inequality of persons pass? 
No; on the contrary, we must bring quite to an end what was only half 
accomplished. Our freedom from another's person still lacks the freedom from 
what the other's person can command, from what he has in his personal power -- 
in short, from "{}personal property."{} Let us then do away with 
\textit{personal property}. Let no one have anything any longer, let every one 
be a -- ragamuffin. Let property be \textit{impersonal}, let it belong to -- 
\textit{society}.

Before the supreme \textit{ruler}, the sole \textit{commander}, we had all 
become equal, equal persons, \textit{i. e.}, nullities.

Before the supreme \textit{proprietor} we all become equal -- ragamuffins. For 
the present, one is still in another's estimation a "{}ragamuffin,"{} a 
"{}have-nothing"{}; but then this estimation ceases. We are all ragamuffins 
together, and as the aggregate of Communistic society we might call ourselves 
a "{}ragamuffin crew."{}

When the proletarian shall really have founded his purposed "{}society"{} in 
which the interval between rich and poor is to be removed, then he 
\textit{will be} a ragamuffin, for then he will feel that it amounts to 
something to be a ragamuffin, and might lift "{}Ragamuffin"{} to be an 
honourable form of address, just as the Revolution did with the word 
"{}Citizen."{} Ragamuffin is his ideal; we are all to become ragamuffins.

This is the second robbery of the "{}personal"{} in the interest of 
"{}humanity."{} Neither command nor property is left to the individual; the 
State took the former, society the latter.

Because in society the most oppressive evils make themselves felt, therefore 
the oppressed especially, and consequently the members of the lower regions of 
society, think they found the fault in society, and make it their task to 
discover the \textit{right society}. This is only the old phenomenon -- that 
one looks for the fault first in everything but \textit{himself}, and 
consequently in the State, in the self-seeking of the rich, etc., which yet 
have precisely our fault to thank for their existence.

 The reflections and conclusions of Communism look very simple. As matters lie 
at this time -- in the present situation with regard to the State, therefore 
-- some, and they the majority, are at a disadvantage compared to others, the 
minority. In this \textit{state} of things the former are in a \textit{state 
of prosperity}, the latter in \textit{state of need}. Hence the present 
\textit{state} of things, \textit{i.e.} the State itself, must be done away 
with. And what in its place? Instead of the isolated state of prosperity -- a 
\textit{general state of prosperity}, a \textit{prosperity of all}.

Through the Revolution the \textit{bourgeoisie} became omnipotent, and all 
inequality was abolished by every one's being raised or degraded to the 
dignity of a \textit{citizen} : the common man -- raised, the aristocrat -- 
degraded; the \textit{third} estate became sole estate, \textit{viz.}, namely, 
the estate of -- \textit{citizens of the State}. Now Communism responds: Our 
dignity and our essence consist not in our being all -- the \textit{equal 
children} of our mother, the State, all born with equal claim to her love and 
her protection, but in our all existing \textit{for each other}. This is our 
equality, or herein we are \textit{equal}, in that we, I as well as you and 
you and all of you, are active or "{}labor"{} each one for the rest; in that 
each of us is a \textit{laborer}, then. The point for us is not what we are 
\textit{for the State} (citizens), not our \textit{citizenship} therefore, but 
what we are \textit{for each other}, that each of us exists only through the 
other, who, caring for my wants, at the same time sees his own satisfied by 
me. He labors \textit{e. g.} for my clothing (tailor), I for his need of 
amusement (comedy-writer, rope-dancer), he for my food (farmer), I for his 
instruction (scientist). It is \textit{labor} that constitutes our dignity and 
our -- equality.

What advantage does citizenship bring us? Burdens! And how high is our labor 
appraised? As low as possible! But labor is our sole value all the same: that 
we are \textit{laborers} is the best thing about us, this is our significance 
in the world, and therefore it must be our consideration too and must come to 
receive \textit{consideration}. What can you meet us with? Surely nothing but 
-- \textit{labor} too. Only for labor or services do we owe you a recompense, 
not for your bare existence; not for what you are \textit{for yourselves} 
either, but only for what you are \textit{for us}. By what have you claims on 
us? Perhaps by your high birth? No, only by what you do for us that is 
desirable or useful. Be it thus then: we are willing to be worth to you only 
so much as we do for you; but you are to be held likewise by us. 
\textit{Services} determine value, -- \textit{i.e.} those services that are 
worth something to us, and consequently \textit{labors for each other, labors 
for the common good}. Let each one be in the other's eyes a \textit{laborer}. 
He who accomplishes something useful is inferior to none, or -- all laborers 
(laborers, of course, in the sense of laborers "{}for the common good,"{} 
\textit{i. e.}, communistic laborers) are equal. But, as the laborer is worth 
his wages,\footnote{[In German an exact quotation of Luke 10. 7.]} let the 
wages too be equal.

As long as faith sufficed for man's honor and dignity, no labor, however 
harassing, could be objected to if it only did not hinder a man in his faith. 
Now, on the contrary, when every one is to cultivate himself into man, 
condemning a man to \textit{machine-like labor} amounts to the same thing as 
slavery. If a factory worker must tire himself to death twelve hours and more, 
he is cut off from becoming man. Every labor is to have the intent that the 
man be satisfied. Therefore he must become a \textit{master} in it too, 
\textit{i.e.} be able to perform it as a totality. He who in a pin factory 
only puts on the heads, only draws the wire, works, as it were, mechanically, 
like a machine; he remains half-trained, does not become a master: his labor 
cannot \textit{satisfy} him, it can only \textit{fatigue} him. His labor is 
nothing by itself, has no object \textit{in} \textit{itself}, is nothing 
complete in itself; he labors only into another's hands, and is \textit{used} 
(exploited) by this other. For this laborer in another's service there is no 
\textit{enjoyment of a cultivated mind}, at most, crude amusements: 
\textit{culture}, you see, is barred against him. To be a good Christian one 
needs only to \textit{believe}, and that can be done under the most oppressive 
circumstances. Hence the Christian-minded take care only of the oppressed 
laborers' piety, their patience, submission, etc. Only so long as the 
downtrodden classes were \textit{Christians} could they bear all their misery: 
for Christianity does not let their murmurings and exasperation rise. Now the 
\textit{hushing} of desires is no longer enough, but their \textit{sating} is 
demanded. The \textit{bourgeoisie} has proclaimed the gospel of the 
\textit{enjoyment of the world}, of material enjoyment, and now wonders that 
this doctrine finds adherents among us poor: it has shown that not faith and 
poverty, but culture and possessions, make a man blessed; we proletarians 
understand that too.

The commonalty freed us from the orders and arbitrariness of individuals. But 
that arbitrariness was left which springs from the conjuncture of situations, 
and may be called the fortuity of circumstances; favoring \textit{fortune}. 
and those "{}favored by fortune,"{} still remain.

When, \textit{e. g.}, a branch of industry is ruined and thousands of laborers 
become breadless, people think reasonably enough to acknowledge that it is not 
the individual who must bear the blame, but that "{}the evil lies in the 
situation."{} Let us change the situation then, but let us change it 
thoroughly, and so that its fortuity becomes powerless. and \textit{a law!} 
Let us no longer be slaves of chance! Let us create a new order that makes an 
end of \textit{fluctuations}. Let this order then be sacred!

Formerly one had to suit the \textit{lords} to come to anything; after the 
Revolution the word was "{}Grasp \textit{fortune!"{}} Luck-hunting or 
hazard-playing, civil life was absorbed in this. Then, alongside this, the 
demand that he who has obtained something shall not frivolously stake it 
again.

Strange and yet supremely natural contradiction. Competition, in which alone 
civil or political life unrolls itself, is a game of luck through and through, 
from the speculations of the exchange down to the solicitation of offices, the 
hunt for customers, looking for work, aspiring to promotion and decorations, 
the second-hand dealer's petty haggling, etc. If one succeeds in supplanting 
and outbidding his rivals, then the "{}lucky throw"{} is made; for it must be 
taken as a piece of luck to begin with that the victor sees himself equipped 
with an ability (even though it has been developed by the most careful 
industry) against which the others do not know how to rise, consequently that 
-- no abler ones are found. And now those who ply their daily lives in the 
midst of these changes of fortune without seeing any harm in it are seized 
with the most virtuous indignation when their own principle appears in naked 
form and "{}breeds misfortune"{} as -- \textit{hazard-playing}. 
Hazard-playing, you see, is too clear, too barefaced a competition, and, like 
every decided nakedness, offends honourable modesty.

The Socialists want to put a stop to this activity of chance, and to form a 
society in which men are no longer dependent on \textit{fortune}, but free.

In the most natural way in the world this endeavor first utters itself as 
hatred of the "{}unfortunate"{} against the "{}fortunate,"{} \textit{i.e.}, of 
those for whom fortune has done little or nothing, against those for whom it 
has done everything. But properly the ill- feeling is not directed against the 
fortunate, but against \textit{fortune}, this rotten spot of the commonalty.

As the Communists first declare free activity to be man's essence, they, like 
all work-day dispositions, need a Sunday; like all material endeavors, they 
need a God, an uplifting and edification alongside their witless "{}labor."{}

That the Communist sees in you the man, the brother, is only the Sunday side 
of Communism. According to the work-day side he does not by any means take you 
as man simply, but as human laborer or laboring man. The first view has in it 
the liberal principle; in the second, illiberality is concealed. If you were a 
"{}lazy-bones,"{} he would not indeed fail to recognize the man in you, but 
would endeavor to cleanse him as a "{}lazy man"{} from laziness and to convert 
you to the \textit{faith} that labor is man's "{}destiny and calling."{}

Therefore he shows a double face: with the one he takes heed that the 
spiritual man be satisfied, with the other he looks about him for means for 
the material or corporeal man. He gives man a twofold \textit{post} -- an 
office of material acquisition and one of spiritual.

The commonalty had \textit{thrown open} spiritual and material goods, and left 
it with each one to reach out for them if he liked.

Communism really procures them for each one, presses them upon him, and 
compels him to acquire them. It takes seriously the idea that, because only 
spiritual and material goods make us men, we must unquestionably acquire these 
goods in order to be man. The commonalty made acquisition free; Communism 
\textit{compels} to acquisition, and recognizes only the acquirer, him who 
practices a trade. It is not enough that the trade is free, but you must 
\textit{take it up}.

So all that is left for criticism to do is to prove that the acquisition of 
these goods does not yet by any means make us men.

With the liberal commandment that every one is to make a man of himself, or 
every one to make himself man, there was posited the necessity that every one 
must gain time for this labor of humanization, \textit{i. e.}, that it should 
become possible for every one to labor on \textit{himself}.

The commonalty thought it had brought this about if it handed over everything 
human to competition, but gave the individual a right to every human thing. 
"{}Each may strive after everything!"{}

Social liberalism finds that \textit{the} matter is not settled with the 
"{}may,"{} because may means only "{}it is forbidden to none"{} but not "{}it 
is made possible to every one."{} Hence it affirms that the commonalty is 
liberal only with the mouth and in words, supremely illiberal in act. It on 
its part wants to give all of us the \textit{means} to be able to labor on 
ourselves.

By the principle of labor that of fortune or competition is certainly outdone. 
But at the same time the laborer, in his consciousness that the essential 
thing in him is "{}the laborer,"{} holds himself aloof from egoism and 
subjects himself to the supremacy of a society of laborers, as the commoner 
clung with self-abandonment to the competition-State. The beautiful dream of a 
"{}social duty"{} still continues to be dreamed. People think again that 
society \textit{gives} what we need, and we are \textit{under obligations} to 
it on that account, owe it everything.\footnote{Proudhon (\textit{Cr\'eation 
de l'Ordre}) cries out, p. 414, "{}In industry, as in science, the publication 
of an invention is the first and \textit{most sacred of duties}!"{}} They are 
still at the point of wanting to \textit{serve} a "{}supreme giver of all 
good."{} That society is no ego at all, which could give, bestow, or grant, 
but an instrument or means, from which we may derive benefit; that we have no 
social duties, but solely interests for the pursuance of which society must 
serve us; that we owe society no sacrifice, but, if we sacrifice anything, 
sacrifice it to ourselves -- of this the Socialists do not think, because they 
-- as liberals -- are imprisoned in the religious principle, and zealously 
aspire after -- a sacred society, \textit{e. g.} the State was hitherto.

Society, from which we have everything, is a new master, a new spook, a new 
"{}supreme being,"{} which "{}takes us into its service and allegiance!"{}

The more precise appreciation of political as well as social liberalism must 
wait to find its place further on. For the present we pass this over, in order 
first to summon them before the tribunal of humane or critical liberalism.

\subsection[\S{} 3. Humane Liberalism]{\centering \S{} 3. Humane Liberalism}

As liberalism is completed in self-criticizing, "{}critical"{}\footnote{[In 
his strictures on "{}criticism"{} Stirner refers to a special movement known 
by that name in the early forties of the last century, of which Bruno Bauer 
was the principal exponent. After his official separation from the faculty of 
the university of Bonn on account of his views in regard to the Bible, Bruno 
Bauer in 1843 settled near Berlin and founded the \textit{Allgemeine 
Literatur-Zeitung}, in which he and his friends, at war with their 
surroundings, championed the "{}absolute emancipation"{} of the individual 
within the limits of "{}pure humanity"{} and fought as their foe "{}the 
mass,"{} comprehending in that term the radical aspirations of political 
liberalism and the communistic demands of the rising Socialist movement of 
that time. For a brief account of Bruno Bauer's movement of criticism, see 
John Henry Mackay, \textit{Max Stirner. Sein Leben und sein Werk}.]} 
liberalism -- in which the critic remains a liberal and does not go beyond the 
principle of liberalism, Man -- this may distinctively be named after Man and 
called the "{}humane."{}

The laborer is counted as the most material and egoistical man. He does 
nothing at all \textit{for humanity}, does everything for \textit{himself}, 
for his welfare.

The commonalty, because it proclaimed the freedom of \textit{Man} only as to 
his birth, had to leave him in the claws of the un-human man (the egoist) for 
the rest of life. Hence under the regime of political liberalism egoism has an 
immense field for free utilization.

The laborer will \textit{utilize} society for his \textit{egoistic} ends as 
the commoner does the State. You have only an egoistic end after all, your 
welfare, is the humane liberal's reproach to the Socialist; take up a 
\textit{purely human interest}, then I will be your companion. "{}But to this 
there belongs a consciousness stronger, more comprehensive, than a 
\textit{laborer-consciousness"{}}. "{}The laborer makes nothing, therefore he 
has nothing; but he makes nothing because his labor is always a labor that 
remains individual, calculated strictly for his own want, a labor day by 
day."{}\footnote{Br. Bauer, \textit{"{}Lit. Ztg}."{} V, 18} In opposition to 
this one might, \textit{e. g.}, consider the fact that Gutenberg's labor did 
not remain individual, but begot innumerable children, and still lives today; 
it was calculated for the want of humanity, and was an eternal, imperishable 
labor.

The humane consciousness despises the commoner-consciousness as well as the 
laborer-consciousness: for the commoner is "{}indignant"{} only at vagabonds 
(at all who have "{}no definite occupation"{}) and their "{}immorality"{}; the 
laborer is "{}disgusted"{} by the idler ("{}lazy-bones"{}) and his 
"{}immoral,"{} because parasitic and unsocial, principles. To this the humane 
liberal retorts: The unsettledness of many is only your product, Philistine! 
But that you, proletarian, demand the \textit{grind} of all, and want to make 
\textit{drudgery} general, is a part, still clinging to you, of your pack-mule 
life up to this time. Certainly you want to lighten drudgery itself by 
\textit{all} having to drudge equally hard, yet only for this reason, that all 
may gain \textit{leisure} to an equal extent. But what are they to do with 
their leisure? What does your "{}society"{} do, that this leisure may be 
passed \textit{humanly?} It must leave the gained leisure to egoistic 
preference again, and the very \textit{gain} that your society furthers falls 
to the egoist, as the gain of the commonalty, the \textit{masterlessness of 
man}, could not be filled with a human element by the State, and therefore was 
left to arbitrary choice.

It is assuredly necessary that man be masterless: but therefore the egoist is 
not to become master over man again either, but man over the egoist. Man must 
assuredly find leisure: but, if the egoist makes use of it, it will be lost 
for man; therefore you ought to have given leisure a human significance. But 
you laborers undertake even your labor from an egoistic impulse, because you 
want to eat, drink, live; how should you be less egoists in leisure? You labor 
only because having your time to yourselves (idling) goes well after work 
done, and what you are to while away your leisure time with is left to 
\textit{chance}.

But, if every door is to be bolted against egoism, it would be necessary to 
strive after completely "{}disinterested"{} action, \textit{total} 
disinterestedness. This alone is human, because only Man is disinterested, the 
egoist always interested.

\begin{center}
--------\end{center}


If we let disinterestedness pass unchallenged for a while, then we ask, do you 
mean not to take an interest in anything, not to be enthusiastic for anything, 
not for liberty, humanity, etc.? "{}Oh, yes, but that is not an egoistic 
interest, not \textit{interestedness}, but a human, \textit{i.e.} a -- 
\textit{theoretical} interest, to wit, an interest not for an individual or 
individuals ('all'), but for the \textit{idea}, for Man!"{}

And you do not notice that you too are enthusiastic only for \textit{your} 
idea, \textit{your} idea of liberty?

And, further, do you not notice that your disinterestedness is again, like 
religious disinterestedness, a heavenly interestedness? Certainly benefit to 
the individual leaves you cold, and abstractly you could cry \textit{fiat 
libertas, pereat mundus}. You do not take thought for the coming day either, 
and take no serious care for the individual's wants anyhow, not for your own 
comfort nor for that of the rest; but you make nothing of all this, because 
you are a -- dreamer.

Do you suppose the humane liberal will be so liberal as to aver that 
everything possible to man is \textit{human?} On the contrary! He does not, 
indeed, share the Philistine's moral prejudice about the strumpet, but "{}that 
this woman turns her body into a money-getting 
machine"{}\footnote{\textit{"{}Lit. Ztg}."{} V, 26} makes her despicable to 
him as "{}human being."{} His judgment is, the strumpet is not a human being; 
or, so far as a woman is a strumpet, so far is she unhuman, dehumanized. 
Further: The Jew, the Christian, the privileged person, the theologian, etc., 
is not a human being; so far as you are a Jew, etc., you are not a human 
being. Again the imperious postulate: Cast from you everything peculiar, 
criticize it away! Be not a Jew, not a Christian, but be a human being, 
nothing but a human being. Assert your \textit{humanity} against every 
restrictive specification; make yourself, by means of it, a human being, and 
free from those limits; make yourself a "{}free man"{} -- \textit{i.e.} 
recognize humanity as your all-determining \textit{essence}.

I say: You are indeed more than a Jew, more than a Christian, etc., but you 
are also more than a human being. Those are all ideas, but you are corporeal. 
Do you suppose, then, that you can ever become a "{}human being as such?"{} Do 
you suppose our posterity will find no prejudices and limits to clear away, 
for which our powers were not sufficient? Or do you perhaps think that in your 
fortieth or fiftieth year you have come so far that the following days have 
nothing more to dissipate in you, and that you are a human being? The men of 
the future will yet fight their way to many a liberty that we do not even 
miss. What do you need that later liberty for? If you meant to esteem yourself 
as nothing before you had become a human being, you would have to wait till 
the "{}last judgment,"{} till the day when man, or humanity, shall have 
attained perfection. But, as you will surely die before that, what becomes of 
your prize of victory?

Rather, therefore, invert the case, and say to yourself, \textit{I am a human 
being!} I do not need to begin by producing the human being in myself, for he 
belongs to me already, like all my qualities.

But, asks the critic, how can one be a Jew and a man at once? In the first 
place, I answer, one cannot be either a Jew or a man at all, if "{}one"{} and 
Jew or man are to mean the same; "{}one"{} always reaches beyond those 
specifications, and -- let Isaacs be ever so Jewish -- a Jew, nothing but a 
Jew, he cannot be, just because he is \textit{this} Jew. In the second place, 
as a Jew one assuredly cannot be a man, if being a man means being nothing 
special. But in the third place -- and this is the point -- I can, as a Jew, 
be entirely what I -- \textit{can} be. From Samuel or Moses, and others, you 
hardly expect that they should have raised themselves above Judaism, although 
you must say that they were not yet "{}men."{} They simply were what they 
could be. Is it otherwise with the Jews of today? Because you have discovered 
the idea of humanity, does it follow from this that every Jew can become a 
convert to it? If he can, he does not fail to, and, if he fails to, he -- 
cannot. What does your demand concern him? What the \textit{call} to be a man, 
which you address to him?

\begin{center}
--------\end{center}


As a universal principle, in the "{}human society"{} which the humane liberal 
promises, nothing "{}special"{} which one or another has is to find 
recognition, nothing which bears the character of "{}private"{} is to have 
value. In this way the circle of liberalism, which has its good principle in 
man and human liberty, its bad in the, egoist and everything private, its God 
in the former, its devil in the latter, rounds itself off completely; and, if 
the special or private person lost his value in the State (no personal 
prerogative), if in the "{}laborers' or ragamuffins' society"{} special 
(private) property is no longer recognized, so in "{}human society"{} 
everything special or private will be left out of account; and, when "{}pure 
criticism"{} shall have accomplished its arduous task, then it will be known 
just what we must look upon as private, and what, "{}penetrated with a sense 
of our nothingness,"{} we must -- let stand.

Because State and Society do not suffice for humane liberalism, it negates 
both, and at the same time retains them. So at one time the cry is that the 
task of the day is "{}not a political, but a social, one,"{} and then again 
the "{}free State"{} is promised for the future. In truth, "{}human society"{} 
is both -- the most general State and the most general society. Only against 
the limited State is it asserted that it makes too much stir about spiritual 
private interests (\textit{e. g.} people's religious belief), and against 
limited society that it makes too much of material private interests. Both are 
to leave private interests to private people, and, as human society, concern 
themselves solely about general human interests.

The politicians, thinking to abolish \textit{personal will}, self-will or 
arbitrariness, did not observe that through 
\textit{property}\footnote{[\textit{Eigentum}, "{}owndom"{}]} our 
\textit{self-will}\footnote{[\textit{Eigenwille} "{}own-will"{}]} gained a 
secure place of refuge.

The Socialists, taking away \textit{property} too, do not notice that this 
secures itself a continued existence in \textit{self-ownership}. Is it only 
money and goods, then, that are a property. or is every opinion something of 
mine, something of my own?

So every \textit{opinion} must be abolished or made impersonal. The person is 
entitled to no opinion, but, as self-will was transferred to the State, 
property to society, so opinion too must be transferred to something 
\textit{general}, "{}Man,"{} and thereby become a general human opinion.

If opinion persists, then I have my God (why, God exists only as "{}my God,"{} 
he is an opinion or my "{}faith"{}), and consequently \textit{my} faith, my 
religion, my thoughts, my ideals. Therefore a general human faith must come 
into existence, the \textit{"{}fanaticism of liberty."{}} For this would be a 
faith that agreed with the "{}essence of man,"{} and, because only "{}man"{} 
is reasonable (you and I might be very unreasonable!), a reasonable faith.

As self-will and property become \textit{powerless}, so must self-ownership or 
egoism in general.

In this supreme development of "{}free man"{} egoism, self-ownership, is 
combated on principle, and such subordinate ends as the social "{}welfare"{} 
of the Socialists, etc., vanish before the lofty "{}idea of humanity."{} 
Everything that is not a "{}general human"{} entity is something separate, 
satisfies only some or one; or, if it satisfies all, it does this to them only 
as individuals, not as men, and is therefore called "{}egoistic."{}

To the Socialists \textit{welfare} is still the supreme aim, as free 
\textit{rivalry} was the approved thing to the political liberals; now welfare 
is free too, and we are free to achieve welfare, just as he who wanted to 
enter into rivalry (competition) was free to do so.

But to take part in the rivalry you need only to be \textit{commoners}; to 
take part in the welfare, only to be \textit{laborers}. Neither reaches the 
point of being synonymous with "{}man."{} It is "{}truly well"{} with man only 
when he is also "{}intellectually free!"{} For man is mind: therefore all 
powers that are alien to him, the mind -- all superhuman, heavenly, unhuman 
powers -- must be overthrown and the name "{}man"{} must be above every name.

So in this end of the modern age (age of the moderns) there returns again, as 
the main point, what had been the main point at its beginning: "{}intellectual 
liberty."{}

To the Communist in particular the humane liberal says: If society prescribes 
to you your activity, then this is indeed free from the influence of the 
individual, \textit{i.e.} the egoist, but it still does not on that account 
need to be a \textit{purely human} activity, nor you to be a complete organ of 
humanity. What kind of activity society demands of you remains 
\textit{accidental}, you know; it might give you a place in building a temple 
or something of that sort, or, even if not that, you might yet on your own 
impulse be active for something foolish, therefore unhuman; yes, more yet, you 
really labor only to nourish yourself, in general to live, for dear life's 
sake, not for the glorification of humanity. Consequently free activity is not 
attained till you make yourself free from all stupidities, from everything 
non-human, \textit{i.e.}, egoistic (pertaining only to the individual, not to 
the Man in the individual), dissipate all untrue thoughts that obscure man or 
the idea of humanity: in short, when you are not merely unhampered in your 
activity, but the substance too of your activity is only what is human, and 
you live and work only for humanity. But this is not the case so long as the 
aim of your effort is only your \textit{welfare} and that of all; what you do 
for the society of ragamuffins is not yet anything done for "{}human 
society."{}

Laboring does not alone make you a man, because it is something formal and its 
object accidental; the question is who you that labor are. As far as laboring 
goes, you might do it from an egoistic (material) impulse, merely to procure 
nourishment and the like; it must be a labor furthering humanity, calculated 
for the good of humanity, serving historical (\textit{i.e.} human) evolution 
-- in short, a \textit{human} labor. This implies two things: one, that it be 
useful to humanity; next, that it be the work of a "{}man."{} The first alone 
may be the case with every labor, as even the labors of nature, \textit{e. g.} 
of animals, are utilized by humanity for the furthering of science, etc.; the 
second requires that he who labors should know the human object of his labor; 
and, as he can have this consciousness only when he \textit{knows himself as 
man}, the crucial condition is -- \textit{self-consciousness.}

Unquestionably much is already attained when you cease to be a 
"{}fragment-laborer,"{}\footnote{[Referring to minute subdivision of labor, 
whereby the single workman produces, not a whole, but a part.]} yet therewith 
you only get a view of the whole of your labor, and acquire a consciousness 
about it, which is still far removed from a self-consciousness, a 
consciousness about your true "{}self"{} or "{}essence,"{} Man. The laborer 
has still remaining the desire for a "{}higher consciousness,"{} which, 
because the activity of labor is unable to quiet it, he satisfies in a leisure 
hour. Hence leisure stands by the side of his labor, and he sees himself 
compelled to proclaim labor and idling human in one breath, yes, to attribute 
the true elevation to the idler, the leisure-enjoyer. He labors only to get 
rid of labor; he wants to make labor free, only that he may be free from 
labor.

In fine, his work has no satisfying substance, because it is only imposed by 
society, only a stint, a task, a calling; and, conversely, his society does 
not satisfy, because it gives only work.

His labor ought to satisfy him as a man; instead of that, it satisfies 
society; society ought to treat him as a man, and it treats him as -- a 
rag-tag laborer, or a laboring ragamuffin.

Labor and society are of use to him not as he needs them as a man, but only as 
he needs them as an "{}egoist."{}

Such is the attitude of criticism toward labor. It points to "{}mind,"{} wages 
the war "{}of mind with the masses,"{}\footnote{\textit{"{}Lit. Ztg}."{} V, 
34.} and pronounces communistic labor unintellectual mass-labor. Averse to 
labor as they are, the masses love to make labor easy for themselves. In 
literature, which is today furnished in mass, this aversion to labor begets 
the universally-known \textit{superficiality}, which puts from it "{}the toil 
of research."{}\footnote{\textit{"{}Lit. Ztg"{}} \textit{ibid}.}

Therefore humane liberalism says: You want labor; all right, we want it 
likewise, but we want it in the fullest measure. We want it, not that we may 
gain spare time, but that we may find all satisfaction in it itself. We want 
labor because it is our self-development.

But then the labor too must be adapted to that end! Man is honored only by 
human, self-conscious labor, only by the labor that has for its end no 
"{}egoistic"{} purpose, but Man, and is Man's self-revelation; so that the 
saying should be \textit{laboro, ergo sum}, I labor, therefore I am a man. The 
humane liberal wants that labor of the \textit{mind} which \textit{works up} 
all material; he wants the mind, that leaves no thing quiet or in its existing 
condition, that acquiesces in nothing, analyzes everything, criticises anew 
every result that has been gained. This restless mind is the true laborer, it 
obliterates prejudices, shatters limits and narrownesses, and raises man above 
everything that would like to dominate over him, while the Communist labors 
only for himself, and not even freely, but from necessity, -- in short, 
represents a man condemned to hard labor.

The laborer of such a type is not "{}egoistic,"{} because he does not labor 
for individuals, neither for himself nor for other individuals, not for 
\textit{private} men therefore, but for humanity and its progress: he does not 
ease individual pains, does not care for individual wants, but removes limits 
within which humanity is pressed, dispels prejudices which dominate an entire 
time, vanquishes hindrances that obstruct the path of all, clears away errors 
in which men entangle themselves, discovers truths which are found through him 
for all and for all time; in short -- he lives and labors for humanity.

Now, in the first place, the discoverer of a great truth doubtless knows that 
it can be useful to the rest of men, and, as a jealous withholding furnishes 
him no enjoyment, he communicates it; but, even though he has the 
consciousness that his communication is highly valuable to the rest, yet he 
has in no wise sought and found his truth for the sake of the rest, but for 
his own sake, because he himself desired it, because darkness and fancies left 
him no rest till he had procured for himself light and enlightenment to the 
best of his powers.

He labors, therefore, for his own sake and for the satisfaction of his want. 
That along with this he was also useful to others, yes, to posterity, does not 
take from his labor the \textit{egoistic} character.

In the next place, if he did labor only on his own account, like the rest, why 
should his act be human, those of the rest unhuman, \textit{i. e.}, egoistic? 
Perhaps because this book, painting, symphony, etc., is the labor of his whole 
being, because he has done his best in it, has spread himself out wholly and 
is wholly to be known from it, while the work of a handicraftsman mirrors only 
the handicraftsman, \textit{i.e.} the skill in handicraft, not "{}the man?"{} 
In his poems we have the whole Schiller; in so many hundred stoves, on the 
other hand, we have before us only the stove-maker, not "{}the man."{}

But does this mean more than "{}in the one work you see \textit{me} as 
completely as possible, in the other only my skill?"{} Is it not me again that 
the act expresses? And is it not more egoistic to offer \textit{oneself} to 
the world in a work, to work out and shape \textit{oneself}, than to remain 
concealed behind one's labor? You say, to be sure, that you are revealing Man. 
But the Man that you reveal is you; you reveal only yourself, yet with this 
distinction from the handicraftsman -- that he does not understand how to 
compress himself into one labor, but, in order to be known as himself, must be 
searched out in his other relations of life, and that your want, through whose 
satisfaction that work came into being, was a -- theoretical want.

But you will reply that you reveal quite another man, a worthier, higher, 
greater, a man that is more man than that other. I will assume that you 
accomplish all that is possible to man, that you bring to pass what no other 
succeeds in. Wherein, then, does your greatness consist? Precisely in this, 
that you are more than other men (the "{}masses"{}), more than \textit{men} 
ordinarily are, more than "{}ordinary men"{}; precisely in your elevation 
above men. You are distinguished beyond other men not by being man, but 
because you are a "{}unique"{}\footnote{[\textit{"{}einziger"{}}]} man. 
Doubtless you show what a man can do; but because you, a man, do it, this by 
no means shows that others, also men, are able to do as much; you have 
executed it only as a \textit{unique} man, and are unique therein.

It is not man that makes up your greatness, but you create it, because you are 
more than man, and mightier than other -- men.

It is believed that one cannot be more than man. Rather, one cannot be less!

It is believed further that whatever one attains is good for Man. In so far as 
I remain at all times a man -- or, like Schiller, a Swabian; like Kant, a 
Prussian; like Gustavus Adolfus, a near-sighted person -- I certainly become 
by my superior qualities a notable man, Swabian, Prussian, or near-sighted 
person. But the case is not much better with that than with Frederick the 
Great's cane, which became famous for Frederick's sake.

To "{}Give God the glory"{} corresponds the modern "{}Give Man the glory."{} 
But I mean to keep it for myself.

Criticism, issuing the summons to man to be "{}human,"{} enunciates the 
necessary condition of sociability; for only as a man among men is one 
\textit{companionable}. Herewith it makes known its \textit{social} object, 
the establishment of "{}human society."{}

Among social theories criticism is indisputably the most complete, because it 
removes and deprives of value everything that \textit{separates} man from man: 
all prerogatives, down to the prerogative of faith. In it the love-principle 
of Christianity, the true social principle, comes to the purest fulfillment, 
and the last possible experiment is tried to take away exclusiveness and 
repulsion from men: a fight against egoism in its simplest and therefore 
hardest form, in the form of singleness,\footnote{[\textit{"{}Einzigkeit"{}}]} 
exclusiveness, itself.

"{}How can you live a truly social life so long as even one exclusiveness 
still exists between you?"{}

I ask conversely, How can you be truly single so long as even one connection 
still exists between you? If you are connected, you cannot leave each other; 
if a "{}tie"{} clasps you, you are something only \textit{with another}, and 
twelve of you make a dozen, thousands of you a people, millions of you 
humanity.

"{}Only when you are human can you keep company with each other as men, just 
as you can understand each other as patriots only when you are patriotic!"{}

All right; then I answer, Only when you are single can you have intercourse 
with each other as what you are.

It is precisely the keenest critic who is hit hardest by the curse of his 
principle. Putting from him one exclusive thing after another, shaking off 
churchliness, patriotism, etc., he undoes one tie after another and separates 
himself from the churchly man, from the patriot, till at last, when all ties 
are undone, he stands -- alone. He, of all men, must exclude all that have 
anything exclusive or private; and, when you get to the bottom, what can be 
more exclusive than the exclusive, single person himself!

Or does he perhaps think that the situation would be better if \textit{all} 
became "{}man"{} and gave up exclusiveness? Why, for the very reason that 
"{}all"{} means "{}every individual"{} the most glaring contradiction is still 
maintained, for the "{}individual"{} is exclusiveness itself. If the humane 
liberal no longer concedes to the individual anything private or exclusive, 
any private thought, any private folly; if he criticises everything away from 
him before his face, since his hatred of the private is an absolute and 
fanatical hatred; if he knows no tolerance toward what is private, because 
everything private is \textit{unhuman} -- yet he cannot criticize away the 
private person himself, since the hardness of the individual person resists 
his criticism, and he must be satisfied with declaring this person a 
"{}private person"{} and really leaving everything private to him again.

What will the society that no longer cares about anything private do? Make the 
private impossible? No, but "{}subordinate it to the interests of society, 
and, \textit{e. g.}, leave it to private will to institute holidays as many as 
it chooses, if only it does not come in collision with the general 
interest."{}\footnote{Br. Bauer, \textit{"{}Judenfrage},"{} p. 66} Everything 
private is \textit{left free}; \textit{i.e.}, it has no interest for society.

"{}By their raising barriers against science the church and religiousness have 
declared that they are what they always were, only that this was hidden under 
another semblance when they were proclaimed to be the basis and necessary 
foundation of the State -- a matter of purely private concern. Even when they 
were connected with the State and made it Christian, they were only the proof 
that the State had not yet developed its general political idea, that it was 
only instituting private rights -- they were only the highest expression for 
the fact that the State was a private affair and had to do only with private 
affairs. When the State shall at last have the courage and strength to fulfil 
its general destiny and to be free; when, therefore, it is also able to give 
separate interests and private concerns their true position -- then religion 
and the church will be free as they have never been hitherto. As a matter of 
the most purely private concern, and a satisfaction of purely personal want, 
they will be left to themselves; and every individual, every congregation and 
ecclesiastical communion, will be able to care for the blessedness of their 
souls as they choose and as they think necessary. Every one will care for his 
soul's blessedness so far as it is to him a personal want, and will accept and 
pay as spiritual caretaker the one who seems to him to offer the best 
guarantee for the satisfaction of his want. Science is at last left entirely 
out of the game."{}\footnote{Br. Bauer, \textit{"{}Die gute Sache der 
Freiheit},"{} pp. 62-63.}

What is to happen, though? Is social life to have an end, and all affability, 
all fraternization, everything that is created by the love or society 
principle, to disappear?

As if one will not always seek the other because he \textit{needs} him; as if 
one must accommodate himself to the other when he \textit{needs} him. But the 
difference is this, that then the individual really \textit{unites} with the 
individual, while formerly they were \textit{bound together} by a tie; son and 
father are bound together before majority, after it they can come together 
independently; before it they \textit{belonged} together as members of the 
family, after it they unite as egoists; sonship and fatherhood remain, but son 
and father no longer pin themselves down to these.

The last privilege, in truth, is "{}Man"{}; with it all are privileged or 
invested. For, as Bruno Bauer himself says, "{}privilege remains even when it 
is extended to all."{}\footnote{Br. Bauer, \textit{"{}Judenfrage},"{} p. 60.}

Thus liberalism runs its course in the following transformations: "{}First, 
the individual is not man, therefore his individual personality is of no 
account: no personal will, no arbitrariness, no orders or mandates!

"{}Second, the individual \textit{has} nothing human, therefore no mine and 
thine, or property, is valid.

"{}Third, as the individual neither is man nor has anything human, he shall 
not exist at all: he shall, as an egoist with his egoistic belongings, be 
annihilated by criticism to make room for Man, 'Man, just discovered.'"{}

But, although the individual is not Man, Man is yet present in the individual, 
and, like every spook and everything divine, has its existence in him. Hence 
political liberalism awards to the individual everything that pertains to him 
as "{}a man by birth,"{} as a born man, among which there are counted liberty 
of conscience, the possession of goods, etc. -- in short, the "{}rights of 
man"{}; Socialism grants to the individual what pertains to him as an 
\textit{active} man, as a "{}laboring"{} man; finally. humane liberalism gives 
the individual what he has as "{}a man,"{} \textit{i. e.}, everything that 
belongs to humanity. Accordingly the single 
one\footnote{[\textit{"{}Einzige"{}}]} has nothing at all, humanity 
everything; and the necessity of the "{}regeneration"{} preached in 
Christianity is demanded unambiguously and in the completest measure. Become a 
new creature, become "{}man!"{}

One might even think himself reminded of the close of the Lord's Prayer. To 
Man belongs the \textit{lordship} (the "{}power"{} or \textit{dynamis}); 
therefore no individual may be lord, but Man is the lord of individuals; -- 
Man's is the \textit{kingdom}, \textit{i.e.} the world, consequently the 
individual is not to be proprietor, but Man, "{}all,"{} command the world as 
property -- to Man is due renown, \textit{glorification} or "{}glory"{} 
(\textit{doxa}) from all, for Man or humanity is the individual's end, for 
which he labors, thinks, lives, and for whose glorification he must become 
"{}man."{}

Hitherto men have always striven to find out a fellowship in which their 
inequalities in other respects should become "{}nonessential"{}; they strove 
for equalization, consequently for \textit{equality}, and wanted to come all 
under one hat, which means nothing less than that they were seeking for one 
lord, one tie, one faith ("{}`Tis in one God we all believe"{}). There cannot 
be for men anything more fellowly or more equal than Man himself, and in this 
fellowship the love-craving has found its contentment: it did not rest till it 
had brought on this last equalization, leveled all inequality, laid man on the 
breast of man. But under this very fellowship decay and ruin become most 
glaring. In a more limited fellowship the Frenchman still stood against the 
German, the Christian against the Mohammedan, etc. Now, on the contrary, 
\textit{man} stands against \textit{men}, or, as men are not man, man stands 
against the un-man.

The sentence "{}God has become man"{} is now followed by the other, "{}Man has 
become I."{} This is \textit{the human 1}. But we invert it and say: I was not 
able to find myself so long as I sought myself as Man. But, now that it 
appears that Man is aspiring to become I and to gain a corporeity in me, I 
note that, after all, everything depends on me, and Man is lost without me. 
But I do not care to give myself up to be the shrine of this most holy thing, 
and shall not ask henceforward whether I am man or un-man in what I set about; 
let this \textit{spirit} keep off my neck!

Humane liberalism goes to work radically. If you want to be or have anything 
especial even in one point, if you want to retain for yourself even one 
prerogative above others, to claim even one right that is not a "{}general 
right of man,"{} you are an egoist.

Very good! I do not want to have or be anything especial above others, I do 
not want to claim any prerogative against them, but -- I do not measure myself 
by others either, and do not want to have any \textit{right} whatever. I want 
to be all and have all that I can be and have. Whether others are and have 
anything \textit{similar}, what do I care? The equal, the same, they can 
neither be nor have. I cause no \textit{detriment} to them, as I cause no 
detriment to the rock by being "{}ahead of it"{} in having motion. If they 
could have it, they would have it.

To cause other men no \textit{detriment} is the point of the demand to possess 
no prerogative; to renounce all "{}being ahead,"{} the strictest theory of 
\textit{renunciation}. One is not to count himself as "{}anything especial,"{} 
\textit{e. g.} a Jew or a Christian. Well, I do not count myself as anything 
especial, but as unique.\footnote{[\textit{"{}einzig"{}}]} Doubtless I have 
\textit{similarity} with others; yet that holds good only for comparison or 
reflection; in fact I am incomparable, unique. My flesh is not their flesh, my 
mind is not their mind. If you bring them under the generalities "{}flesh, 
mind,"{} those are your \textit{thoughts}, which have nothing to do with 
\textit{my} flesh, \textit{my} mind, and can least of all issue a "{}call"{} 
to mine.

I do not want to recognize or respect in you any thing, neither the proprietor 
nor the ragamuffin, nor even the man, but to \textit{use you}. In salt I find 
that it makes food palatable to me, therefore I dissolve it; in the fish I 
recognize an aliment, therefore I eat it; in you I discover the gift of making 
my life agreeable, therefore I choose you as a companion. Or, in salt I study 
crystallization, in the fish animality, in you men, etc. But to me you are 
only what you are for me -- to wit, my object; and, because \textit{my} 
object, therefore my property.

In humane liberalism ragamuffinhood is completed. We must first come down to 
the most ragamuffin-like, most poverty-stricken condition if we want to arrive 
at \textit{ownness}, for we must strip off everything alien. But nothing seems 
more ragamuffin-like than naked -- Man.

It is more than ragamuffinhood, however, when I throw away Man too because I 
feel that he too is alien to me and that T can make no pretensions on that 
basis. This is no longer mere ragamuffinhood: because even the last rag has 
fallen off, here stands real nakedness, denudation of everything alien. The 
ragamuffin has stripped off ragamuffinhood itself, and therewith has ceased to 
be what he was, a ragamuffin.

I am no longer a ragamuffin, but have been one.

\begin{center}
--------\end{center}


Up to this time the discord could not come to an outbreak, because properly 
there is current only a contention of modern liberals with antiquated 
liberals, a contention of those who understand "{}freedom"{} in a small 
measure and those who want the "{}full measure"{} of freedom; of the 
\textit{moderate} and \textit{measureless}, therefore. Everything turns on the 
question, \textit{how free} must \textit{man} be? That man must be free, in 
this all believe; therefore all are liberal too. But the un-man\footnote{[It 
should be remembered that to be an \textit{Unmensch}["{}un-man"{}] one must be 
a man. The word means an inhuman or unhuman man, a man who is not man. A 
tiger, an avalanche, a drought, a cabbage, is not an un-man.]} who is 
somewhere in every individual, how is he blocked? How can it be arranged not 
to leave the un-man free at the same time with man?

Liberalism as a whole has a deadly enemy, an invincible opposite, as God has 
the devil: by the side of man stands always the un-man, the individual, the 
egoist. State, society, humanity, do not master this devil.

Humane liberalism has undertaken the task of showing the other liberals that 
they still do not want "{}freedom."{}

If the other liberals had before their eyes only isolated egoism and were for 
the most part blind, radical liberalism has against it egoism "{}in mass,"{} 
throws among the masses all who do not make the cause of freedom their own as 
it does, so that now man and un-man rigorously separated, stand over against 
each other as enemies, to wit, the "{}masses"{} and 
"{}criticism"{};\footnote{\textit{"{}Lit. Ztg"{}}., V, 23; as comment, V, 12ff.} 
namely, "{}free, human criticism,"{} as it is called \textit{(Judenfrage}, p. 
114), in opposition to crude, that is, religious criticism.

Criticism expresses the hope that it will be victorious over all the masses 
and "{}give them a general certificate of 
insolvency."{}\footnote{\textit{"{}Lit. Ztg"{}}, V 15.} So it means finally to 
make itself out in the right, and to represent all contention of the 
"{}faint-hearted and timorous"{} as an egoistic 
\textit{stubbornness},\footnote{[\textit{Rechthaberei}, literally the 
character of always insisting on making one's self out to be in the right.]} 
as pettiness, paltriness. All wrangling loses significance, and petty 
dissensions are given up, because in criticism a common enemy enters the 
field. "{}You are egoists altogether, one no better than another!"{} Now the 
egoists stand together against criticism. Really the egoists? No, they fight 
against criticism precisely because it accuses them of egoism; they do not 
plead guilty of egoism. Accordingly criticism and the masses stand on the same 
basis: both fight against egoism, both repudiate it for themselves and charge 
it to each other.

Criticism and the masses pursue the same goal, freedom from egoism, and 
wrangle only over which of them approaches nearest to the goal or even attains 
it.

The Jews, the Christians, the absolutists, the men of darkness and men of 
light, politicians, Communists -- all, in short -- hold the reproach of egoism 
far from them; and, as criticism brings against them this reproach in plain 
terms and in the most extended sense, all \textit{justify} themselves against 
the accusation of egoism, and combat -- egoism, the same enemy with whom 
criticism wages war.

Both, criticism and masses, are enemies of egoists, and both seek to liberate 
themselves from egoism, as well by clearing or whitewashing 
\textit{themselves} as by ascribing it to the opposite party.

The critic is the true "{}spokesman of the masses"{} who gives them the 
"{}simple concept and the phrase"{} of egoism, while the spokesmen to whom the 
triumph is denied were only bunglers. He is their prince and general in the 
war against egoism for freedom; what he fights against they fight against. But 
at the same time he is their enemy too, only not the enemy before them, but 
the friendly enemy who wields the knout behind the timorous to force courage 
into them.

Hereby the opposition of criticism and the masses is reduced to the following 
contradiction: "{}You are egoists!"{} "{}No, we are not!"{} "{}I will prove it 
to you!"{} "{}You shall have our justification!"{}

Let us then take both for what they give themselves out for, non-egoists, and 
what they take each other for, egoists. They are egoists and are not.

Properly criticism says: You must liberate your ego from all limitedness so 
entirely that it becomes a \textit{human} ego. I say: Liberate yourself as far 
as you can, and you have done your part; for it is not given to every one to 
break through all limits, or, more expressively: not to every one is that a 
limit which is a limit for the rest. Consequently, do not tire yourself with 
toiling at the limits of others; enough if you tear down yours. Who has ever 
succeeded in tearing down even one limit \textit{for all men?} Are not 
countless persons today, as at all times, running about with all the 
"{}limitations of humanity?"{} He who overturns one of \textit{his} limits may 
have shown others the way and the means; the overturning of \textit{their} 
limits remains their affair. Nobody does anything else either. To demand of 
people that they become wholly men is to call on them to cast down all human 
limits. That is impossible, because \textit{Man} has no limits. I have some 
indeed, but then it is only \textit{mine} that concern me any, and only they 
can be overcome by me. A human ego I cannot become, just because I am I and 
not merely man.

Yet let us still see whether criticism has not taught us something that we can 
lay to heart! I am not free if I am not without interests, not man if I am not 
disinterested? Well, even if it makes little difference to me to be free or 
man, yet I do not want to leave unused any occasion to realize \textit{myself} 
or make myself count. Criticism offers me this occasion by the teaching that, 
if anything plants itself firmly in me, and becomes indissoluble, I become its 
prisoner and servant, \textit{i.e.} a possessed man. An interest, be it for 
what it may, has kidnapped a slave in me if I cannot get away from it, and is 
no longer my property, but I am its. Let us therefore accept criticism's 
lesson to let no part of our property become stable, and to feel comfortable 
only in -- \textit{dissolving} it.

So, if criticism says: You are man only when you are restlessly criticizing 
and dissolving! then we say: Man I am without that, and I am I likewise; 
therefore I want only to be careful to secure my property to myself; and, in 
order to secure it, I continually take it back into myself, annihilate in it 
every movement toward independence, and swallow it before it can fix itself 
and become a "{}fixed idea"{} or a "{}mania."{}

But I do that not for the sake of my "{}human calling,"{} but because I call 
myself to it. I do not strut about dissolving everything that it is possible 
for a man to dissolve, and, \textit{e. g.}, while not yet ten years old I do 
not criticize the nonsense of the Commandments, but I am man all the same, and 
act humanly in just this -- that I still leave them uncriticized. In short, I 
have no calling, and follow none, not even that to be a man.

Do I now reject what liberalism has won in its various exertions? Far be the 
day that anything won should be lost! Only, after "{}Man"{} has become free 
through liberalism, I turn my gaze back upon myself and confess to myself 
openly: What Man seems to have gained, \textit{I} alone have gained.

Man is free when "{}Man is to man the supreme being."{} So it belongs to the 
completion of liberalism that every other supreme being be annulled, theology 
overturned by anthropology, God and his grace laughed down, "{}atheism"{} 
universal.

The egoism of property has given up the last that it had to give when even the 
"{}My God"{} has become senseless; for God exists only when he has at heart 
the individual's welfare, as the latter seeks his welfare in him.

Political liberalism abolished the inequality of masters and servants: it made 
people masterless, anarchic. The master was now removed from the individual, 
the "{}egoist,"{} to become a ghost -- the law or the State. Social liberalism 
abolishes the inequality of possession, of the poor and rich, and makes people 
\textit{possessionless} or propertyless. Property is withdrawn from the 
individual and surrendered to ghostly society. Humane liberalism makes people 
\textit{godless}, atheistic. Therefore the individual's God, "{}My God,"{} 
must be put an end to. Now masterlessness is indeed at the same time freedom 
from service, possessionlessness at the same time freedom from care, and 
godlessness at the same time freedom from prejudice: for with the master the 
servant falls away; with possession, the care about it; with the firmly-rooted 
God, prejudice. But, since the master rises again as State, the servants 
appears again as subject; since possession becomes the property of society, 
care is begotten anew as labor; and, since God as Man becomes a prejudice, 
there arises a new faith, faith in humanity or liberty. For the individual's 
God the God of all, \textit{viz}., "{}Man,"{} is now exalted; "{}for it is the 
highest thing in us all to be man."{} But, as nobody can become entirely what 
the idea "{}man"{} imports, Man remains to the individual a lofty other world, 
an unattained supreme being, a God. But at the same time this is the "{}true 
God,"{} because he is fully adequate to us -- to wit, our own 
\textit{"{}self"{}}; we ourselves, but separated from us and lifted above us.

\begin{center}
--------\end{center}


\subsection[Postscript]{\centering Postscript}

The foregoing review of "{}free human criticism"{} was written by bits 
immediately after the appearance of the books in question, as was also that 
which elsewhere refers to writings of this tendency, and I did little more 
than bring together the fragments. But criticism is restlessly pressing 
forward, and thereby makes it necessary for me to come back to it once more, 
now that my book is finished, and insert this concluding note.

I have before me the latest (eighth) number of the \textit{Allgemeine 
Literatur-Zeitung} of Bruno Bauer.

There again "{}the general interests of society"{} stand at the top. But 
criticism has reflected, and given this "{}society"{} a specification by which 
it is discriminated from a form which previously had still been confused with 
it: the "{}State,"{} in former passages still celebrated as "{}free State,"{} 
is quite given up because it can in no wise fulfil the task of "{}human 
society."{} Criticism only "{}saw itself compelled to identify for a moment 
human and political affairs"{} in 1842; but now it has found that the State, 
even as "{}free State,"{} is not human society, or, as it could likewise say, 
that the people is not "{}man."{} We saw how it got through with theology and 
showed clearly that God sinks into dust before Man; we see it now come to a 
clearance with politics in the same way, and show that before Man peoples and 
nationalities fall: so we see how it has its explanation with Church and 
State, declaring them both unhuman, and we shall see -- for it betrays this to 
us already -- how it can also give proof that before Man the "{}masses,"{} 
which it even calls a "{}spiritual being,"{} appear worthless. And how should 
the lesser "{}spiritual beings"{} be able to maintain themselves before the 
supreme spirit? "{}Man"{} casts down the false idols.

So what the critic has in view for the present is the scrutiny of the 
"{}masses,"{} which he will place before "{}Man"{} in order to combat them 
from the standpoint of Man. "{}What is now the object of criticism?"{} "{}The 
masses, a spiritual being!"{} These the critic will "{}learn to know,"{} and 
will find that they are in contradiction with Man; he will demonstrate that 
they are unhuman, and will succeed just as well in this demonstration as in 
the former ones, that the divine and the national, or the concerns of Church 
and of State, were the unhuman.

The masses are defined as "{}the most significant product of the Revolution, 
as the deceived multitude which the illusions of political Illumination, and 
in general the entire Illumination movement of the eighteenth century, have 
given over to boundless disgruntlement."{} The Revolution satisfied some by 
its result, and left others unsatisfied; the satisfied part is the commonalty 
(\textit{bourgeoisie}, etc.), the unsatisfied is the -- masses. Does not the 
critic, so placed, himself belong to the "{}masses"{}?

But the unsatisfied are still in great mistiness, and their discontent utters 
itself only in a "{}boundless disgruntlement."{} This the likewise unsatisfied 
critic now wants to master: he cannot want and attain more than to bring that 
"{}spiritual being,"{} the masses, out of its disgruntlement, and to 
"{}uplift"{} those who were only disgruntled, \textit{i.e.} to give them the 
right attitude toward those results of the Revolution which are to be 
overcome; -- he can become the head of the masses, their decided spokesman. 
Therefore he wants also to "{}abolish the deep chasm which parts him from the 
multitude."{} From those who want to "{}uplift the lower classes of the 
people"{} he is distinguished by wanting to deliver from "{}disgruntlement,"{} 
not merely these, but himself too.

But assuredly his consciousness does not deceive him either, when he takes the 
masses to be the "{}natural opponents of theory,"{} and foresees that, "{}the 
more this theory shall develop itself, so much the more will it make the 
masses compact."{} For the critic cannot enlighten or satisfy the masses with 
his \textit{presupposition}, Man. If over against the commonalty they are only 
the "{}lower classes of the people,"{} politically insignificant masses, over 
against "{}Man"{} they must still more be mere "{}masses,"{} humanly 
insignificant -- yes, unhuman -- masses, or a multitude of un-men.

The critic clears away everything human; and, starting from the presupposition 
that the human is the true, he works against himself, denying it wherever it 
had been hitherto found. He proves only that the human is to be found nowhere 
except in his head, but the unhuman everywhere. The unhuman is the real, the 
extant on all hands, and by the proof that it is "{}not human"{} the critic 
only enunciates plainly the tautological sentence that it is the unhuman.

But what if the unhuman, turning its back on itself with resolute heart, 
should at the same time turn away from the disturbing critic and leave him 
standing, untouched and unstung by his remonstrance? "{}You call me the 
unhuman,"{} it might say to him, "{}and so I really am -- for you; but I am so 
only because you bring me into opposition to the human, and I could despise 
myself only so long as I let myself be hypnotized into this opposition. I was 
contemptible because I sought my 'better self' outside me; I was the unhuman 
because I dreamed of the 'human'; I resembled the pious who hunger for their 
'true self' and always remain 'poor sinners'; I thought of myself only in 
comparison to another; enough, I was not all in all, was not -- 
\textit{unique}.\footnote{[\textit{"{}einzig"{}}]} But now I cease to appear 
to myself as the unhuman, cease to measure myself and let myself be measured 
by man, cease to recognize anything above me: consequently -- adieu, humane 
critic! I only have been the unhuman, am it now no longer, but am the unique, 
yes, to your loathing, the egoistic; yet not the egoistic as it lets itself be 
measured by the human, humane, and unselfish, but the egoistic as the -- 
unique."{}

We have to pay attention to still another sentence of the same number. 
"{}Criticism sets up no dogmas, and wants to learn to know nothing but 
\textit{things}. "{}

The critic is afraid of becoming "{}dogmatic"{} or setting up dogmas. Of 
course: why, thereby he would become the opposite of the critic -- the 
dogmatist; he would now become bad, as he is good as critic, or would become 
from an unselfish man an egoist, etc. "{}Of all things, no dogma!"{} This is 
his -- dogma. For the critic remains on one and the same ground with the 
dogmatist -- that of \textit{thoughts}. Like the latter he always starts from 
a thought, but varies in this, that he never ceases to keep the 
principle-thought in the \textit{process of thinking}, and so does not let it 
become stable. He only asserts the thought-process against the thought-faith, 
the progress of thinking against stationariness in it. From criticism no 
thought is safe, since criticism is thought or the thinking mind itself.

Therefore I repeat that the religious world -- and this is the world of 
thought -- reaches its completion in criticism, where thinking extends its 
encroachments over every thought, no one of which may "{}egoistically"{} 
establish itself. Where would the "{}purity of criticism,"{} the purity of 
thinking, be left if even one thought escaped the process of thinking? This 
explains the fact that the critic has even begun already to gibe gently here 
and there at the thought of Man, of humanity and humaneness, because he 
suspects that here a thought is approaching dogmatic fixity. But yet he cannot 
decompose this thought till he has found a -- "{}higher"{} in which it 
dissolves; for he moves only -- in thoughts. This higher thought might be 
enunciated as that of the movement or process of thinking itself, 
\textit{i.e.} as the thought of thinking or of criticism, for example.

Freedom of thinking has in fact become complete hereby, freedom of mind 
celebrates its triumph: for the individual, "{}egoistic"{} thoughts have lost 
their dogmatic truculence. There is nothing left but the -- dogma of free 
thinking or of criticism.

Against everything that belongs to the world of thought, criticism is in the 
right, \textit{i. e.}, in might: it is the victor. Criticism, and criticism 
alone, is "{}up to date."{} From the standpoint of thought there is no power 
capable of being an overmatch for criticism's, and it is a pleasure to see how 
easily and sportively this dragon swallows all other serpents of thought. Each 
serpent twists, to be sure, but criticism crushes it in all its "{}turns."{}

I am no opponent of criticism, \textit{i.e.} I am no dogmatist, and do not 
feel myself touched by the critic's tooth with which he tears the dogmatist to 
pieces. If I were a "{}dogmatist,"{} I should place at the head a dogma, 
\textit{i.e.} a thought, an idea, a principle, and should complete this as a 
"{}systematist,"{} spinning it out to a system, a structure of thought. 
Conversely, if I were a critic, \textit{viz}., an opponent of the dogmatist, I 
should carry on the fight of free thinking against the enthralling thought, I 
should defend thinking against what was thought. But I am neither the champion 
of a thought nor the champion of thinking; for "{}I,"{} from whom I start, am 
not a thought, nor do I consist in thinking. Against me, the unnameable, the 
realm of thoughts, thinking, and mind is shattered.

Criticism is the possessed man's fight against possession as such, against all 
possession: a fight which is founded in the consciousness that everywhere 
possession, or, as the critic calls it, a religious and theological attitude, 
is extant. He knows that people stand in a religious or believing attitude not 
only toward God, but toward other ideas as well, like right, the State, law; 
\textit{i.e.} he recognizes possession in all places. So he wants to break up 
thoughts by thinking; but I say, only thoughtlessness really saves me from 
thoughts. It is not thinking, but my thoughtlessness, or I the unthinkable, 
incomprehensible, that frees me from possession.

A jerk does me the service of the most anxious thinking, a stretching of the 
limbs shakes off the torment of thoughts, a leap upward hurls from my breast 
the nightmare of the religious world, a jubilant Hoopla throws off year-long 
burdens. But the monstrous significance of unthinking jubilation could not be 
recognized in the long night of thinking and believing.

"{}What clumsiness and frivolity, to want to solve the most difficult 
problems, acquit yourself of the most comprehensive tasks, by \textit{a 
breaking off}!"{}

But have you tasks if you do not set them to yourself? So long as you set 
them, you will not give them up, and I certainly do not care if you think, 
and, thinking, create a thousand thoughts. But you who have set the tasks, are 
you not to be able to upset them again? Must you be bound to these tasks, and 
must they become absolute tasks?

To cite only one thing, the government has been disparaged on account of its 
resorting to forcible means against thoughts, interfering against the press by 
means of the police power of the censorship, and making a personal fight out 
of a literary one. As if it were solely a matter of thoughts, and as if one's 
attitude toward thoughts must be unselfish, self-denying, and 
self-sacrificing! Do not those thoughts attack the governing parties 
themselves, and so call out egoism? And do the thinkers not set before the 
attacked ones the \textit{religious} demand to reverence the power of thought, 
of ideas? They are to succumb voluntarily and resignedly, because the divine 
power of thought, Minerva, fights on their enemies' side. Why, that would be 
an act of possession, a religious sacrifice. To be sure, the governing parties 
are themselves held fast in a religious bias, and follow the leading power of 
an idea or a faith; but they are at the same time unconfessed egoists, and 
right here, against the enemy, their pent-up egoism breaks loose: possessed in 
their faith, they are at the same time unpossessed by their opponents' faith, 
\textit{i.e.} they are egoists toward this. If one wants to make them a 
reproach, it could only be the converse -- to wit, that they are possessed by 
their ideas.

Against thoughts no egoistic power is to appear, no police power etc. So the 
believers in thinking believe. But thinking and its thoughts are not sacred to 
me, and I defend my \textit{skin} against them as against other things. That 
may be an unreasonable defense; but, if I am in duty bound to reason, then I, 
like Abraham, must sacrifice my dearest to it!

In the kingdom of thought, which, like that of faith, is the kingdom of 
heaven, every one is assuredly wrong who uses unthinking force, just as every 
one is wrong who in the kingdom of love behaves unlovingly, or, although he is 
a Christian and therefore lives in the kingdom of love, yet acts 
un-Christianly; in these kingdoms, to which he supposes himself to belong 
though he nevertheless throws off their laws, he is a "{}sinner"{} or 
"{}egoist."{} But it is only when he becomes a criminal against these kingdoms 
that he can throw off their dominion.

Here too the result is this, that the fight of the thinkers against the 
government is indeed in the right, namely, in might -- so far as it is carried 
on against the government's thoughts (the government is dumb, and does not 
succeed in making any literary rejoinder to speak of), but is, on the other 
hand, in the wrong, to wit, in impotence, so far as it does not succeed in 
bringing into the field anything but thoughts against a personal power (the 
egoistic power stops the mouths of the thinkers). The theoretical fight cannot 
complete the victory, and the sacred power of thought succumbs to the might of 
egoism. Only the egoistic fight, the fight of egoists on both sides, clears up 
everything.

This last now, to make thinking an affair of egoistic option, an affair of the 
single person,\footnote{[\textit{"{}des Einzigen"{}}]} a mere pastime or hobby 
as it were, and, to take from it the importance of "{}being the last decisive 
power"{}; this degradation and desecration of thinking; this equalization of 
the unthinking and thoughtful ego; this clumsy but real "{}equality"{} -- 
criticism is not able to produce, because it itself is only the priest of 
thinking, and sees nothing beyond thinking but -- the deluge.

Criticism does indeed affirm, \textit{e. g.} that free criticism may overcome 
the State, but at the same time it defends itself against the reproach which 
is laid upon it by the State government, that it is "{}self-will and 
impudence"{}; it thinks, then, that "{}self-will and impudence"{} may not 
overcome, it alone may. The truth is rather the reverse: the State can be 
really overcome only by impudent self-will.

It may now, to conclude with this, be clear that in the critic's new change of 
front he has not transformed himself, but only "{}made good an oversight,"{} 
"{}disentangled a subject,"{} and is saying too much when he speaks of 
"{}criticism criticizing itself"{}; it, or rather he, has only criticized its 
"{}oversight"{} and cleared it of its "{}inconsistencies."{} If he wanted to 
criticize criticism, he would have to look and see if there was anything in 
its presupposition.

I on my part start from a presupposition in presupposing \textit{myself}; but 
my presupposition does not struggle for its perfection like "{}Man struggling 
for his perfection,"{} but only serves me to enjoy it and consume it. I 
consume my presupposition, and nothing else, and exist only in consuming it. 
But that presupposition is therefore not a presupposition at all: for, as I am 
the Unique, I know nothing of the duality of a presupposing and a presupposed 
ego (an "{}incomplete"{} and a "{}complete"{} ego or man); but this, that I 
consume myself, means only that I am. I do not presuppose myself, because I am 
every moment just positing or creating myself, and am I only by being not 
presupposed but posited, and, again, posited only in the moment when I posit 
myself; \textit{i. e.}, I am creator and creature in one.

If the presuppositions that have hitherto been current are to melt away in a 
full dissolution, they must not be dissolved into a higher presupposition 
again -- \textit{i.e.} a thought, or thinking itself, criticism. For that 
dissolution is to be for \textit{my} good; otherwise it would belong only in 
the series of the innumerable dissolutions which, in favor of others 
(\textit{e. g.} this very Man, God, the State, pure morality, etc.), declared 
old truths to be untruths and did away with long-fostered presuppositions.

\chapter[Part Second: I]{\centering {\normalsize Part Second}\\
I}

\medskip{}

\noindent{}At the entrance of the modern time stands the "{}God-man."{} At its 
exit will only the God in the God-man evaporate? And can the God-man really 
die if only the God in him dies? They did not think of this question, and 
thought they were through when in our days they brought to a victorious end 
the work of the Illumination, the vanquishing of God: they did not notice that 
Man has killed God in order to become now -- "{}sole God on high."{} The 
\textit{other world outside us} is indeed brushed away, and the great 
undertaking of the Illuminators completed; but the \textit{other world in 
us}has become a new heaven and calls us forth to renewed heaven-storming: God 
has had to give place, yet not to us, but to -- Man. How can you believe that 
the God-man is dead before the Man in him, besides the God, is dead?

\medskip{}

\chapter[I. Ownness]{\centering I.\\
OWNNESS\footnote{[This is a literal translation of the German word 
\textit{Eigenheit}, which, with its primitive eigen, "{}own,"{} is used in 
this chapter in a way that the German dictionaries do not quite recognize. The 
author's conception being new, he had to make an innovation in the German 
language to express it. The translator is under the like necessity. In most 
passages "{}self-ownership,"{} or else "{}personality,"{} would translate the 
word, but there are some where the thought is so \textit{eigen}, \textit{i. 
e.}, so peculiar or so thoroughly the author's own, that no English word I can 
think of would express it. It will explain itself to one who has read Part 
First intelligently.]}}

"{}Does not the spirit thirst for freedom?"{} -- Alas, not my spirit alone, my 
body too thirsts for it hourly! When before the odorous castle-kitchen my nose 
tells my palate of the savory dishes that are being prepared therein, it feels 
a fearful pining at its dry bread; when my eyes tell the hardened back about 
soft down on which one may lie more delightfully than on its compressed straw, 
a suppressed rage seizes it; when -- but let us not follow the pains further. 
-- And you call that a longing for freedom? What do you want to become free 
from, then? From your hardtack and your straw bed? Then throw them away! -- 
But that seems not to serve you: you want rather to have the freedom to enjoy 
delicious foods and downy beds. Are men to give you this "{}freedom"{} -- are 
they to permit it to you? You do not hope that from their philanthropy, 
because you know they all think like you: each is the nearest to himself! How, 
therefore, do you mean to come to the enjoyment of those foods and beds? 
Evidently not otherwise than in making them your property!

If you think it over rightly, you do not want the freedom to have all these 
fine things, for with this freedom you still do not have them; you want really 
to have them, to call them \textit{yours} and possess them as \textit{your 
property}. Of what use is a freedom to you, indeed, if it brings in nothing? 
And, if you became free from everything, you would no longer have anything; 
for freedom is empty of substance. Whoso knows not how to make use of it, for 
him it has no value, this useless permission; but how I make use of it depends 
on my personality.\footnote{[\textit{Eigenheit}]}

I have no objection to freedom, but I wish more than freedom for you: you 
should not merely \textit{be rid} of what you do not want; you should not only 
be a "{}freeman,"{} you should be an "{}owner"{} too.

Free -- from what? Oh! what is there that cannot be shaken off? The yoke of 
serfdom, of sovereignty, of aristocracy and princes, the dominion of the 
desires and passions; yes, even the dominion of one's own will, of self-will, 
for the completest self-denial is nothing but freedom -- freedom, to wit, from 
self-determination, from one's own self. And the craving for freedom as for 
something absolute, worthy of every praise, deprived us of ownness: it created 
self-denial. However, the freer I become, the more compulsion piles up before 
my eyes; and the more impotent I feel myself. The unfree son of the wilderness 
does not yet feel anything of all the limits that crowd a civilized man: he 
seems to himself freer than this latter. In the measure that I conquer freedom 
for myself I create for myself new bounds and new tasks: if I have invented 
railroads, I feel myself weak again because I cannot yet sail through the 
skies like the bird; and, if I have solved a problem whose obscurity disturbed 
my mind, at once there await me innumerable others, whose perplexities impede 
my progress, dim my free gaze, make the limits of my \textit{freedom} 
painfully sensible to me. "{}Now that you have become free from sin, you have 
become servants of righteousness."{}\footnote{Rom. 6, 18.} Republicans in 
their broad freedom, do they not become servants of the law? How true 
Christian hearts at all times longed to "{}become free,"{} how they pined to 
see themselves delivered from the "{}bonds of this earth-life"{}! They looked 
out toward the land of freedom. ("{}The Jerusalem that is above is the 
freewoman; she is the mother of us all."{} Gal. 4. 26.)

Being free from anything -- means only being clear or rid. "{}He is free from 
headache"{} is equal to "{}he is rid of it."{} "{}He is free from this 
prejudice"{} is equal to "{}he has never conceived it"{} or "{}he has got rid 
of it."{} In "{}less"{} we complete the freedom recommended by Christianity, 
in sinless, godless, moralityless, etc.

 Freedom is the doctrine of Christianity. "{}Ye, dear brethren, are called to 
freedom."{}\footnote{1 Pet. 2. 16.} "{}So speak and so do, as those who are to 
be judged by the law of freedom."{}\footnote{James 2. 12.}

Must we then, because freedom betrays itself as a Christian ideal, give it up? 
No, nothing is to be lost, freedom no more than the rest; but it is to become 
our own, and in the form of freedom it cannot.

What a difference between freedom and ownness! One can get \textit{rid} of a 
great many things, one yet does not get rid of all; one becomes free from 
much, not from everything. Inwardly one may be free in spite of the condition 
of slavery, although, too, it is again only from all sorts of things, not from 
everything; but from the whip, the domineering temper, of the master, one does 
not as slave become \textit{free}. "{}Freedom lives only in the realm of 
dreams!"{} Ownness, on the contrary, is my whole being and existence, it is I 
myself. I am free from what I am \textit{rid} of, owner of what I have in my 
\textit{power} or what I \textit{control. My own} I am at all times and under 
all circumstances, if I know how to have myself and do not throw myself away 
on others. To be free is something that I cannot truly \textit{will}, because 
I cannot make it, cannot create it: I can only wish it and -- aspire toward 
it, for it remains an ideal, a spook. The fetters of reality cut the sharpest 
welts in my flesh every moment. But \textit{my own} I remain. Given up as serf 
to a master, I think only of myself and my advantage; his blows strike me 
indeed, I am not \textit{free} from them; but I endure them only for 
\textit{my benefit}, perhaps in order to deceive him and make him secure by 
the semblance of patience, or, again, not to draw worse upon myself by 
contumacy. But, as I keep my eye on myself and my selfishness, I take by the 
forelock the first good opportunity to trample the slaveholder into the dust. 
That I then become \textit{free} from him and his whip is only the consequence 
of my antecedent egoism. Here one perhaps says I was "{}free"{} even in the 
condition of slavery -- to wit, "{}intrinsically"{} or "{}inwardly."{} But 
"{}intrinsically free"{} is not "{}really free,"{} and "{}inwardly"{} is not 
"{}outwardly."{} I was own, on the other hand, my own, altogether, inwardly 
and outwardly. Under the dominion of a cruel master my body is not "{}free"{} 
from torments and lashes; but it is \textit{my} bones that moan under the 
torture, \textit{my} fibres that quiver under the blows, and \textit{I} moan 
because \textit{my} body moans. That \textit{I} sigh and shiver proves that I 
have not yet lost \textit{myself}, that I am still my own. My leg is not 
"{}free"{} from the master's stick, but it is my leg and is inseparable. Let 
him tear it off me and look and see if he still has my leg! He retains in his 
hand nothing but the -- corpse of my leg, which is as little my leg as a dead 
dog is still a dog: a dog has a pulsating heart, a so-called dead dog has none 
and is therefore no longer a dog.

If one opines that a slave may yet be inwardly free, he says in fact only the 
most indisputable and trivial thing. For who is going to assert that any man 
is \textit{wholly} without freedom? If I am an eye-servant, can I therefore 
not be free from innumerable things, \textit{e. g.} from faith in Zeus, from 
the desire for fame, etc.? Why then should not a whipped slave also be able to 
be inwardly free from un-Christian sentiments, from hatred of his enemy, etc.? 
He then has "{}Christian freedom,"{} is rid of the un-Christian; but has he 
absolute freedom, freedom from everything, \textit{e. g.} from the Christian 
delusion, or from bodily pain?

In the meantime, all this seems to be said more against names than against the 
thing. But is the name indifferent, and has not a word, a shibboleth, always 
inspired and -- fooled men? Yet between freedom and ownness there lies still a 
deeper chasm than the mere difference of the words.

All the world desires freedom, all long for its reign to come. Oh, 
enchantingly beautiful dream of a blooming "{}reign of freedom,"{} a "{}free 
human race"{}! -- who has not dreamed it? So men shall become free, entirely 
free, free from all constraint! From all constraint, really from all? Are they 
never to put constraint on themselves any more? "{}Oh yes, that, of course; 
don't you see, that is no constraint at all?"{} Well, then at any rate they -- 
are to become free from religious faith, from the strict duties of morality, 
from the inexorability of the law, from -- "{}What a fearful 
misunderstanding!"{} Well, \textit{what} are they to be free from then, and 
what not?

The lovely dream is dissipated; awakened, one rubs his half-opened eyes and 
stares at the prosaic questioner. "{}What men are to be free from?"{} -- From 
blind credulity, cries one. What's that? exclaims another, all faith is blind 
credulity; they must become free from all faith. No, no, for God's sake -- 
inveighs the first again -- do not cast all faith from you, else the power of 
brutality breaks in. We must have the republic -- a third makes himself heard, 
-- and become -- free from all commanding lords. There is no help in that, 
says a fourth: we only get a new lord then, a "{}dominant majority"{}; let us 
rather free ourselves from this dreadful inequality. -- O, hapless equality, 
already I hear your plebeian roar again! How I had dreamed so beautifully just 
now of a paradise of \textit{freedom}, and what -- impudence and 
licentiousness now raises its wild clamor! Thus the first laments, and gets on 
his feet to grasp the sword against "{}unmeasured freedom."{} Soon we no 
longer hear anything but the clashing of the swords of the disagreeing 
dreamers of freedom.

What the craving for freedom has always come to has been the desire for a 
\textit{particular} freedom, \textit{e. g.} freedom of faith; \textit{i.e.} 
the believing man wanted to be free and independent; of what? of faith 
perhaps? no! but of the inquisitors of faith. So now "{}political or civil"{} 
freedom. The citizen wants to become free not from citizenhood, but from 
bureaucracy, the arbitrariness of princes, etc. Prince Metternich once said he 
had "{}found a way that was adapted to guide men in the path of 
\textit{genuine} freedom for all the future."{} The Count of Provence ran away 
from France precisely at the time when he was preparing the "{}reign of 
freedom,"{} and said: "{}My imprisonment had become intolerable to me; I had 
only one passion, the desire for \textit{freedom}; I thought only of it."{}

The craving for a \textit{particular} freedom always includes the purpose of a 
new \textit{dominion}, as it was with the Revolution, which indeed "{}could 
give its defenders the uplifting feeling that they were fighting for 
freedom,"{} but in truth only because they were after a particular freedom, 
therefore a new \textit{dominion}, the "{}dominion of the law."{}

Freedom you all want, you want \textit{freedom}. Why then do you haggle over a 
more or less? \textit{Freedom} can only be the whole of freedom; a piece of 
freedom is not \textit{freedom}. You despair of the possibility of obtaining 
the whole of freedom, freedom from everything -- yes, you consider it insanity 
even to wish this? -- Well, then leave off chasing after the phantom, and 
spend your pains on something better than the -- \textit{unattainable}.

"{}Ah, but there is nothing better than freedom!"{}

What have you then when you have freedom, \textit{viz}., -- for I will not 
speak here of your piecemeal bits of freedom -- complete freedom? Then you are 
rid of everything that embarrasses you, everything, and there is probably 
nothing that does not once in your life embarrass you and cause you 
inconvenience. And for whose sake, then, did you want to be rid of it? 
Doubtless \textit{for your} sake, because it is in \textit{your} way! But, if 
something were not inconvenient to you; if, on the contrary, it were quite to 
your mind (\textit{e. g.} the gently but \textit{irresistibly commanding} look 
of your loved one) -- then you would not want to be rid of it and free from 
it. Why not? For \textit{your sake} again! So you take \textit{yourselves} as 
measure and judge over all. You gladly let freedom go when unfreedom, the 
"{}sweet service of love,"{} suits \textit{you}; and you take up your freedom 
again on occasion when it begins to suit \textit{you} better -- \textit{i. 
e.}, supposing, which is not the point here, that you are not afraid of such a 
Repeal of the Union for other (perhaps religious) reasons.

Why will you not take courage now to really make \textit{yourselves} the 
central point and the main thing altogether? Why grasp in the air at freedom, 
your dream? Are you your dream? Do not begin by inquiring of your dreams, your 
notions, your thoughts, for that is all "{}hollow theory."{} Ask yourselves 
and ask after yourselves -- that is \textit{practical}, and you know you want 
very much to be "{}practical."{} But there the one hearkens what his God (of 
course what he thinks of at the name God is his God) may be going to say to 
it, and another what his moral feelings, his conscience, his feeling of duty, 
may determine about it, and a third calculates what folks will think of it -- 
and, when each has thus asked his Lord God (folks are a Lord God just as good 
as, nay, even more compact than, the other-worldly and imaginary one: 
\textit{vox populi, vox dei)}, then he accommodates himself to his Lord's will 
and listens no more at all for what \textit{he himself} would like to say and 
decide.

Therefore turn to yourselves rather than to your gods or idols. Bring out from 
yourselves what is in you, bring it to the light, bring yourselves to 
revelation.

How one acts only from himself, and asks after nothing further, the Christians 
have realized in the notion "{}God."{} He acts "{}as it pleases him."{} And 
foolish man, who could do just so, is to act as it "{}pleases God"{} instead. 
-- If it is said that even God proceeds according to eternal laws, that too 
fits me, since I too cannot get out of my skin, but have my law in my whole 
nature, \textit{i.e.} in myself.

But one needs only admonish you of yourselves to bring you to despair at once. 
"{}What am I?"{} each of you asks himself. An abyss of lawless and unregulated 
impulses, desires, wishes, passions, a chaos without light or guiding star! 
How am I to obtain a correct answer, if, without regard to God's commandments 
or to the duties which morality prescribes, without regard to the voice of 
reason, which in the course of history, after bitter experiences, has exalted 
the best and most reasonable thing into law, I simply appeal to myself? My 
passion would advise me to do the most senseless thing possible. -- Thus each 
deems himself the -- devil; for, if, so far as he is unconcerned about 
religion, etc., he only deemed himself a beast, he would easily find that the 
beast, which does follow only \textit{its} impulse (as it were, its advice), 
does not advise and impel itself to do the "{}most senseless"{} things, but 
takes very correct steps. But the habit of the religious way of thinking has 
biased our mind so grievously that we are -- terrified at \textit{ourselves} 
in our nakedness and naturalness; it has degraded us so that we deem ourselves 
depraved by nature, born devils. Of course it comes into your head at once 
that your calling requires you to do the "{}good,"{} the moral, the right. 
Now, if you ask \textit{yourselves} what is to be done, how can the right 
voice sound forth from you, the voice which points the way of the good, the 
right, the true, etc.? What concord have God and Belial?

But what would you think if one answered you by saying: "{}That one is to 
listen to God, conscience, duties, laws, and so forth, is flim-flam with which 
people have stuffed your head and heart and made you crazy"{}? And if he asked 
you how it is that you know so surely that the voice of nature is a seducer? 
And if he even demanded of you to turn the thing about and actually to deem 
the voice of God and conscience to be the devil's work? There are such 
graceless men; how will you settle them? You cannot appeal to your parsons, 
parents, and good men, for precisely these are designated by them as your 
\textit{seducers}, as the true seducers and corrupters of youth, who busily 
sow broadcast the tares of self-contempt and reverence to God, who fill young 
hearts with mud and young heads with stupidity.

But now those people go on and ask: For whose sake do you care about God's and 
the other commandments? You surely do not suppose that this is done merely out 
of complaisance toward God? No, you are doing it -- \textit{for your sake} 
again. -- Here too, therefore, \textit{you} are the main thing, and each must 
say to himself, \textit{I} am everything to myself and I do everything 
\textit{on my} account. If it ever became clear to you that God, the 
commandments, etc., only harm you, that they reduce and ruin \textit{you}, to 
a certainty you would throw them from you just as the Christians once 
condemned Apollo or Minerva or heathen morality. They did indeed put in the 
place of these Christ and afterward Mary, as well as a Christian morality; but 
they did this for the sake of \textit{their} souls' welfare too, therefore out 
of egoism or ownness.

And it was by this egoism, this ownness, that they got \textit{rid} of the old 
world of gods and became \textit{free} from it. Ownness \textit{created} a new 
\textit{freedom}; for ownness is the creator of everything, as genius (a 
definite ownness), which is always originality, has for a long time already 
been looked upon as the creator of new productions that have a place in the 
history of the world.

If your efforts are ever to make "{}freedom"{} the issue, then exhaust 
freedom's demands. Who is it that is to become free? You, I, we. Free from 
what? From everything that is not you, not I, not we. I, therefore, am the 
kernel that is to be delivered from all wrappings and -- freed from all 
cramping shells. What is left when I have been freed from everything that is 
not I? Only I; nothing but I. But freedom has nothing to offer to this I 
himself. As to what is now to happen further after I have become free, freedom 
is silent -- as our governments, when the prisoner's time is up, merely let 
him go, thrusting him out into abandonment.

Now why, if freedom is striven after for love of the I after all -- why not 
choose the I himself as beginning, middle, and end? Am I not worth more than 
freedom? Is it not I that make myself free, am not I the first? Even unfree, 
even laid in a thousand fetters, I yet am; and I am not, like freedom, extant 
only in the future and in hopes, but even as the most abject of slaves I am -- 
present.

Think that over well, and decide whether you will place on your banner the 
dream of "{}freedom"{} or the resolution of "{}egoism,"{} of "{}ownness."{} 
"{}Freedom"{} awakens your \textit{rage} against everything that is not you; 
"{}egoism"{} calls you to \textit{joy} over yourselves, to self-enjoyment; 
"{}freedom"{} is and remains a \textit{longing} , a romantic plaint, a 
Christian hope for unearthliness and futurity; "{}ownness"{} is a reality, 
which \textit{of itself} removes just so much unfreedom as by barring your own 
way hinders you. What does not disturb you, you will not want to renounce; 
and, if it begins to disturb you, why, you know that "{}you must obey 
\textit{yourselves} rather than men!"{}

Freedom teaches only: Get yourselves rid, relieve yourselves, of everything 
burdensome; it does not teach you who you yourselves are. Rid, rid! So call, 
get rid even of yourselves, "{}deny yourselves."{} But ownness calls you back 
to yourselves, it says "{}Come to yourself!"{} Under the aegis of freedom you 
get rid of many kinds of things, but something new pinches you again: "{}you 
are rid of the Evil One; evil is left."{}\footnote{[See note, p. 112]} As 
\textit{own} you are \textit{really rid of everything}, and what clings to you 
\textit{you have accepted}; it is your choice and your pleasure. The 
\textit{own} man is the \textit{free-born}, the man free to begin with; the 
free man, on the contrary, is only the \textit{eleutheromaniac}, the dreamer 
and enthusiast.

The former is \textit{originally free}, because he recognizes nothing but 
himself; he does not need to free himself first, because at the start he 
rejects everything outside himself, because he prizes nothing more than 
himself, rates nothing higher, because, in short, he starts from himself and 
"{}comes to himself."{} Constrained by childish respect, he is nevertheless 
already working at "{}freeing"{} himself from this constraint. Ownness works 
in the little egoist, and procures him the desired -- freedom.

Thousands of years of civilization have obscured to you what you are, have 
made you believe you are not egoists but are \textit{called} to be idealists 
("{}good men"{}). Shake that off! Do not seek for freedom, which does 
precisely deprive you of yourselves, in "{}self-denial"{}; but seek for 
\textit{yourselves}, become egoists, become each of you an \textit{almighty 
ego}. Or, more clearly: Just recognize yourselves again, just recognize what 
you really are, and let go your hypocritical endeavors, your foolish mania to 
be something else than you are. Hypocritical I call them because you have yet 
remained egoists all these thousands of years, but sleeping, self-deceiving, 
crazy egoists, you \textit{Heautontimorumenoses}, you self- tormentors. Never 
yet has a religion been able to dispense with "{}promises,"{} whether they 
referred us to the other world or to this ("{}long life,"{} etc.); for man is 
\textit{mercenary} and does nothing "{}gratis."{} But how about that "{}doing 
the good for the good's sake"{} without prospect of reward? As if here too the 
pay was not contained in the satisfaction that it is to afford. Even religion, 
therefore, is founded on our egoism and -- exploits it; calculated for our 
\textit{desires}, it stifles many others for the sake of one. This then gives 
the phenomenon of \textit{cheated} egoism, where I satisfy, not myself, but 
one of my desires, \textit{e. g.} the impulse toward blessedness. Religion 
promises me the -- "{}supreme good"{}; to gain this I no longer regard any 
other of my desires, and do not slake them. -- All your doings are 
\textit{unconfessed} , secret, covert, and concealed egoism. But because they 
are egoism that you are unwilling to confess to yourselves, that you keep 
secret from yourselves, hence not manifest and public egoism, consequently 
unconscious egoism -- therefore they are \textit{not egoism}, but thraldom, 
service, self-renunciation; you are egoists, and you are not, since you 
renounce egoism. Where you seem most to be such, you have drawn upon the word 
"{}egoist"{} -- loathing and contempt.

I secure my freedom with regard to the world in the degree that I make the 
world my own, \textit{i.e.} "{}gain it and take possession of it"{} for 
myself, by whatever might, by that of persuasion, of petition, of categorical 
demand, yes, even by hypocrisy, cheating, etc.; for the means that I use for 
it are determined by what I am. If I am weak, I have only weak means, like the 
aforesaid, which yet are good enough for a considerable part of the world. 
Besides, cheating, hypocrisy, lying, look worse than they are. Who has not 
cheated the police, the law? Who has not quickly taken on an air of honourable 
loyalty before the sheriff's officer who meets him, in order to conceal an 
illegality that may have been committed, etc.? He who has not done it has 
simply let violence be done to him; he was a \textit{weakling} from -- 
conscience. I know that my freedom is diminished even by my not being able to 
carry out my will on another object, be this other something without will, 
like a rock, or something with will, like a government, an individual; I deny 
my ownness when -- in presence of another -- I give myself up, \textit{i.e.} 
give way, desist, submit; therefore by \textit{loyalty, submission}. For it is 
one thing when I give up my previous course because it does not lead to the 
goal, and therefore turn out of a wrong road; it is another when I yield 
myself a prisoner. I get around a rock that stands in my way, till I have 
powder enough to blast it; I get around the laws of a people, till I have 
gathered strength to overthrow them. Because I cannot grasp the moon, is it 
therefore to be "{}sacred"{} to me, an Astarte? If I only could grasp you, I 
surely would, and, if I only find a means to get up to you, you shall not 
frighten me! You inapprehensible one, you shall remain inapprehensible to me 
only till I have acquired the might for apprehension and call you my 
\textit{own}; I do not give myself up before you, but only bide my time. Even 
if for the present I put up with my inability to touch you, I yet remember it 
against you.

Vigorous men have always done so. When the "{}loyal"{} had exalted an 
unsubdued power to be their master and had adored it, when they had demanded 
adoration from all, then there came some such son of nature who would not 
loyally submit, and drove the adored power from its inaccessible Olympus. He 
cried his "{}Stand still"{} to the rolling sun, and made the earth go round; 
the loyal had to make the best of it; he laid his axe to the sacred oaks, and 
the "{}loyal"{} were astonished that no heavenly fire consumed him; he threw 
the pope off Peter's chair, and the "{}loyal"{} had no way to hinder it; he is 
tearing down the divine-right business, and the "{}loyal"{} croak in vain, and 
at last are silent.

My freedom becomes complete only when it is my -- \textit{might}; but by this 
I cease to be a merely free man, and become an own man. Why is the freedom of 
the peoples a "{}hollow word"{}? Because the peoples have no might! With a 
breath of the living ego I blow peoples over, be it the breath of a Nero, a 
Chinese emperor, or a poor writer. Why is it that the G.....\footnote{[Meaning 
"{}German"{}. Written in this form because of the censorship.]} legislatures 
pine in vain for freedom, and are lectured for it by the cabinet ministers? 
Because they are not of the "{}mighty"{}! Might is a fine thing, and useful 
for many purposes; for "{}one goes further with a handful of might than with a 
bagful of right."{} You long for freedom? You fools! If you took might, 
freedom would come of itself. See, he who has might "{}stands above the 
law."{} How does this prospect taste to you, you "{}law-abiding"{} people? But 
you have no taste!

The cry for "{}freedom"{} rings loudly all around. But is it felt and known 
what a donated or chartered freedom must mean? It is not recognized in the 
full amplitude of the word that all freedom is essentially -- self-liberation 
-- \textit{i.e.} that I can have only so much freedom as I procure for myself 
by my ownness. Of what use is it to sheep that no one abridges their freedom 
of speech? They stick to bleating. Give one who is inwardly a Mohammedan, a 
Jew, or a Christian, permission to speak what he likes: he will yet utter only 
narrow-minded stuff. If, on the contrary, certain others rob you of the 
freedom of speaking and hearing, they know quite rightly wherein lies their 
temporary advantage, as you would perhaps be able to say and hear something 
whereby those "{}certain"{} persons would lose their credit.

If they nevertheless give you freedom, they are simply knaves who give more 
than they have. For then they give you nothing of their own, but stolen wares: 
they give you your own freedom, the freedom that you must take for yourselves; 
and they \textit{give} it to you only that you may not take it and call the 
thieves and cheats to an account to boot. In their slyness they know well that 
given (chartered) freedom is no freedom, since only the freedom one 
\textit{takes} for himself, therefore the egoist's freedom, rides with full 
sails. Donated freedom strikes its sails as soon as there comes a storm -- or 
calm; it requires always a -- gentle and moderate breeze.

Here lies the difference between self-liberation and emancipation 
(manumission, setting free). Those who today "{}stand in the opposition"{} are 
thirsting and screaming to be "{}set free."{} The princes are to "{}declare 
their peoples of age,"{} \textit{i. e.}, emancipate them! Behave as if you 
were of age, and you are so without any declaration of majority; if you do not 
behave accordingly, you are not worthy of it, and would never be of age even 
by a declaration of majority. When the Greeks were of age, they drove out 
their tyrants, and, when the son is of age, he makes himself independent of 
his father. If the Greeks had waited till their tyrants graciously allowed 
them their majority, they might have waited long. A sensible father throws out 
a son who will not come of age, and keeps the house to himself; it serves the 
noodle right.

The man who is set free is nothing but a freed man, a \textit{libertinus}, a 
dog dragging a piece of chain with him: he is an unfree man in the garment of 
freedom, like the ass in the lion's skin. Emancipated Jews are nothing 
bettered in themselves, but only relieved as Jews, although he who relieves 
their condition is certainly more than a churchly Christian, as the latter 
cannot do this without inconsistency. But, emancipated or not emancipated, Jew 
remains Jew; he who is not self-freed is merely an -- emancipated man. The 
Protestant State can certainly set free (emancipate) the Catholics; but, 
because they do not make themselves free, they remain simply -- Catholics.

Selfishness and unselfishness have already been spoken of. The friends of 
freedom are exasperated against selfishness because in their religious 
striving after freedom they cannot -- free themselves from that sublime thing, 
"{}self-renunciation."{} The liberal's anger is directed against egoism, for 
the egoist, you know, never takes trouble about a thing for the sake of the 
thing, but for his sake: the thing must serve him. It is egoistic to ascribe 
to no thing a value of its own, an "{}absolute"{} value, but to seek its value 
in me. One often hears that pot-boiling study which is so common counted among 
the most repulsive traits of egoistic behavior, because it manifests the most 
shameful desecration of science; but what is science for but to be consumed? 
If one does not know how to use it for anything better than to keep the pot 
boiling, then his egoism is a petty one indeed, because this egoist's power is 
a limited power; but the egoistic element in it, and the desecration of 
science, only a possessed man can blame.

Because Christianity, incapable of letting the individual count as an 
ego,\footnote{[\textit{"{}Einzige"{}}]} thought of him only as a dependent, 
and was properly nothing but a \textit{social theory --} a doctrine of living 
together, and that of man with God as well as of man with man -- therefore in 
it everything "{}own"{} must fall into most woeful disrepute: selfishness, 
self-will, ownness, self-love, etc. The Christian way of looking at things has 
on all sides gradually re-stamped honourable words into dishonorable; why 
should they not be brought into honor again? So \textit{Schimpf} (contumely) 
is in its old sense equivalent to jest, but for Christian seriousness pastime 
became a dishonor,\footnote{[I take \textit{Entbehrung}, "{}destitution,"{} to 
be a misprint for \textit{Entehrung}.]} for that seriousness cannot take a 
joke; \textit{frech} (impudent) formerly meant only bold, brave; 
\textit{Frevel} (wanton outrage) was only daring. It is well known how askance 
the word "{}reason"{} was looked at for a long time.

Our language has settled itself pretty well to the Christian standpoint, and 
the general consciousness is still too Christian not to shrink in terror from 
everything un-Christian as from something incomplete or evil. Therefore 
"{}selfishness"{} is in a bad way too.

Selfishness,\footnote{[\textit{Eigennutz}, literally "{}own-use."{}]} in the 
Christian sense, means something like this: I look only to see whether 
anything is of use to me as a sensual man. But is sensuality then the whole of 
my ownness? Am I in my own senses when I am given up to sensuality? Do I 
follow myself, my own determination, when I follow that? I am my \textit{own} 
only when I am master of myself, instead of being mastered either by 
sensuality or by anything else (God, man, authority, law, State, Church, 
etc.); what is of use to me, this self-owned or self-appertaining one, my 
selfishness pursues.

Besides, one sees himself every moment compelled to believe in that 
constantly-blasphemed selfishness as an all-controlling power. In the session 
of February 10, 1844, Welcker argues a motion on the dependence of the judges, 
and sets forth in a detailed speech that removable, dismissable, transferable, 
and pensionable judges -- in short, such members of a court of justice as can 
by mere administrative process be damaged and endangered -- are wholly without 
reliability, yes, lose all respect and all confidence among the people. The 
whole bench, Welcker cries, is demoralized by this dependence! In blunt words 
this means nothing else than that the judges find it more to their advantage 
to give judgment as the ministers would have them than to give it as the law 
would have them. How is that to be helped? Perhaps by bringing home to the 
judges' hearts the ignominiousness of their venality, and then cherishing the 
confidence that they will repent and henceforth prize justice more highly than 
their selfishness? No, the people does not soar to this romantic confidence, 
for it feels that selfishness is mightier than any other motive. Therefore the 
same persons who have been judges hitherto may remain so, however thoroughly 
one has convinced himself that they behaved as egoists; only they must not any 
longer find their selfishness favored by the venality of justice, but must 
stand so independent of the government that by a judgment in conformity with 
the facts they do not throw into the shade their own cause, their 
"{}well-understood interest,"{} but rather secure a comfortable combination of 
a good salary with respect among the citizens.

So Welcker and the commoners of Baden consider themselves secured only when 
they can count on selfishness. What is one to think, then, of the countless 
phrases of unselfishness with which their mouths overflow at other times?

To a cause which I am pushing selfishly I have another relation than to one 
which I am serving unselfishly. The following criterion might be cited for it; 
against the one I can \textit{sin} or commit a \textit{sin}, the other I can 
only \textit{trifle away}, push from me, deprive myself of -- \textit{i.e.} 
commit an imprudence. Free trade is looked at in both ways, being regarded 
partly as a freedom which may \textit{under certain circumstances} be granted 
or withdrawn, partly as one which is to be held \textit{sacred under all 
circumstances}.

If I am not concerned about a thing in and for itself, and do not desire it 
for its own sake, then I desire it solely as a \textit{means to an end}, for 
its usefulness; for the sake of another end, \textit{e. g.}, oysters for a 
pleasant flavor. Now will not every thing whose final end he himself is, serve 
the egoist as means? And is he to protect a thing that serves him for nothing 
-- \textit{e. g.}, the proletarian to protect the State?

Ownness includes in itself everything own, and brings to honor again what 
Christian language dishonored. But ownness has not any alien standard either, 
as it is not in any sense an \textit{idea} like freedom, morality, humanity, 
etc.: it is only a description of the -- \textit{owner}.

\chapter[II. The Owner]{\centering II.\\
THE OWNER}

I -- do I come to myself and mine through liberalism? Whom does the liberal 
look upon as his equal? Man! Be only man -- and that you are anyway -- and the 
liberal calls you his brother. He asks very little about your private opinions 
and private follies, if only he can espy "{}Man"{} in you.

But, as he takes little heed of what you are \textit{privatim --} nay, in a 
strict following out of his principle sets no value at all on it -- he sees in 
you only what you are \textit{generatim}. In other words, he sees in you, not 
you, but the \textit{species;} not Tom or Jim, but Man; not the real or unique 
one,\footnote{[\textit{Einzigen}] }but your essence or your concept; not the 
bodily man, but the \textit{spirit}.

As Tom you would not be his equal, because he is Jim, therefore not Tom; as 
man you are the same that he is. And, since as Tom you virtually do not exist 
at all for him (so far, to wit, as he is a liberal and not unconsciously an 
egoist), he has really made "{}brother-love"{} very easy for himself: he loves 
in you not Tom, of whom he knows nothing and wants to know nothing, but Man.

To see in you and me nothing further than "{}men,"{} that is running the 
Christian way of looking at things, according to which one is for the other 
nothing but a \textit{concept} (\textit{e. g.} a man called to salvation, 
etc.), into the ground.

Christianity properly so called gathers us under a less utterly general 
concept: there we are "{}sons of God"{} and "{}led by the Spirit of 
God."{}\footnote{Rom 8. 14.} Yet not all can boast of being God's sons, but 
"{}the same Spirit which witnesses to our spirit that we are sons of God 
reveals also who are the sons of the devil."{}\footnote{Cf. John 3. 10. with 
Rom. 8. 16.} Consequently, to be a son of God one must not be a son of the 
devil; the sonship of God excluded certain men. To be \textit{sons of men} -- 
\textit{i. e.}, men -- on the contrary, we need nothing but to belong to the 
human \textit{species}, need only to be specimens of the same species. What I 
am as this I is no concern of yours as a good liberal, but is my 
\textit{private affair} alone; enough that we are both sons of one and the 
same mother, to wit, the human species: as "{}a son of man"{} I am your equal.

What am I now to you? Perhaps this \textit{bodily} I as I walk and stand? 
Anything but that. This bodily I, with its thoughts, decisions, and passions, 
is in your eyes a "{}private affair"{} which is no concern of yours: it is an 
"{}affair by itself."{} As an "{}affair for you"{} there exists only my 
concept, my generic concept, only \textit{the Man}, who, as he is called Tom, 
could just as well be Joe or Dick. You see in me not me, the bodily man, but 
an unreal thing, the spook, \textit{i.e.} a \textit{Man}.

In the course of the Christian centuries we declared the most various persons 
to be "{}our equals,"{} but each time in the measure of that \textit{spirit} 
which we expected from them -- \textit{e. g.} each one in whom the spirit of 
the need of redemption may be assumed, then later each one who has the spirit 
of integrity, finally each one who shows a human spirit and a human face. Thus 
the fundamental principle of "{}equality"{} varied.

 Equality being now conceived as equality of the \textit{human spirit}, there 
has certainly been discovered an equality that includes \textit{all} men; for 
who could deny that we men have a human spirit, \textit{i. e.}, no other than 
a human!

But are we on that account further on now than in the beginning of 
Christianity? Then we were to have a \textit{divine spirit}, now a 
\textit{human;} but, if the divine did not exhaust us, how should the human 
wholly express what \textit{we} are? Feuerbach \textit{e. g.} thinks, that if 
he humanizes the divine, he has found the truth. No, if God has given us pain, 
"{}Man"{} is capable of pinching us still more torturingly. The long and the 
short of it is this: that we are men is the slightest thing about us, and has 
significance only in so far as it is one of our 
\textit{qualities},\footnote{[\textit{Eigenschaften}]} \textit{i. e.} our 
property.\footnote{[\textit{Eigentum}]} I am indeed among other things a man, 
as I am \textit{e. g.} a living being, therefore an animal, or a European, a 
Berliner, etc.; but he who chose to have regard for me only as a man, or as a 
Berliner, would pay me a regard that would be very unimportant to me. And 
wherefore? Because he would have regard only for one of my \textit{qualities}, 
not for \textit{me}.

It is just so with the \textit{spirit too}. A Christian spirit, an upright 
spirit, etc. may well be my acquired quality, my property, but I am not this 
spirit: it is mine, not I its.

Hence we have in liberalism only the continuation of the old Christian 
depreciation of the I, the bodily Tom. Instead of taking me as I am, one looks 
solely at my property, my qualities, and enters into marriage bonds with me 
only for the sake of my -- possessions; one marries, as it were, what I have, 
not what I am. The Christian takes hold of my spirit, the liberal of my 
humanity.

But, if the spirit, which is not regarded as the \textit{property} of the 
bodily ego but as the proper ego itself, is a ghost, then the Man too, who is 
not recognized as my quality but as the proper I, is nothing but a spook, a 
thought, a concept.

Therefore the liberal too revolves in the same circle as the Christian. 
Because the spirit of mankind, \textit{i.e.} Man, dwells in you, you are a 
man, as when the spirit of Christ dwells in you are a Christian; but, because 
it dwells in you only as a second ego, even though it be as your proper or 
"{}better"{} ego, it remains otherworldly to you, and you have to strive to 
become wholly man. A striving just as fruitless as the Christian's to become 
wholly a blessed spirit!

One can now, after liberalism has proclaimed Man, declare openly that herewith 
was only completed the consistent carrying out of Christianity, and that in 
truth Christianity set itself no other task from the start than to realize 
"{}man,"{} the "{}true man."{} Hence, then, the illusion that Christianity 
ascribes an infinite value to the ego (as \textit{e. g.} in the doctrine of 
immortality, in the cure of souls, etc.) comes to light. No, it assigns this 
value to \textit{Man} alone. Only \textit{Man} is immortal, and only because I 
am Man am I too immortal. In fact, Christianity had to teach that no one is 
lost, just as liberalism too puts all on an equality as men; but that 
eternity, like this equality, applied only to the \textit{Man} in me, not to 
me. Only as the bearer and harborer of Man do I not die, as notoriously "{}the 
king never dies."{} Louis dies, but the king remains; I die, but my spirit, 
Man, remains. To identify me now entirely with Man the demand has been 
invented, and stated, that I must become a "{}real generic 
being."{}\footnote{Karl Marx, in the \textit{"{}Deutsch-franz\"osische 
Jahrbucher},"{} p. 197.}

The \textbf{human} \textit{religion} is only the last metamorphosis of the 
Christian religion. For liberalism is a religion because it separates my 
essence from me and sets it above me, because it exalts "{}Man"{} to the same 
extent as any other religion does its God or idol, because it makes what is 
mine into something otherworldly, because in general it makes out of what is 
mine, out of my qualities and my property, something alien -- to wit, an 
"{}essence"{}; in short, because it sets me beneath Man, and thereby creates 
for me a "{}vocation."{} But liberalism declares itself a religion in form too 
when it demands for this supreme being, Man, a zeal of faith, "{}a faith that 
some day will at last prove its fiery zeal too, a zeal that will be 
invincible."{}\footnote{Br. Bauer, \textit{"{}Judenfrage}"{}, p. 61.} But, as 
liberalism is a human religion, its professor takes a \textit{tolerant} 
attitude toward the professor of any other (Catholic, Jewish, etc.), as 
Frederick the Great did toward every one who performed his duties as a 
subject, whatever fashion of becoming blest he might be inclined toward. This 
religion is now to be raised to the rank of the generally customary one, and 
separated from the others as mere "{}private follies,"{} toward which, 
besides, one takes a highly \textit{liberal} attitude on account of their 
unessentialness.

One may call it the \textit{State-religion}, the religion of the "{}free 
State,"{} not in the sense hitherto current that it is the one favored or 
privileged by the State, but as that religion which the "{}free State"{} not 
only has the right, but is compelled, to demand from each of those who belong 
to it, let him be \textit{privatim} a Jew, a Christian, or anything else. For 
it does the same service to the State as filial piety to the family. If the 
family is to be recognized and maintained, in its existing condition, by each 
one of those who belong to it, then to him the tie of blood must be sacred, 
and his feeling for it must be that of piety, of respect for the ties of 
blood, by which every blood-relation becomes to him a consecrated person. So 
also to every member of the State-community this community must be sacred, and 
the concept which is the highest to the State must likewise be the highest to 
him.

But what concept is the highest to the State? Doubtless that of being a really 
human society, a society in which every one who is really a man, \textit{i. 
e.},\textit{not an un-man}, can obtain admission as a member. Let a State's 
tolerance go ever so far, toward an un-man and toward what is inhuman it 
ceases. And yet this "{}un-man"{} is a man, yet the "{}inhuman"{} itself is 
something human, yes, possible only to a man, not to any beast; it is, in 
fact, something "{}possible to man."{} But, although every un-man is a man, 
yet the State excludes him; \textit{i.e.} it locks him up, or transforms him 
from a fellow of the State into a fellow of the prison (fellow of the lunatic 
asylum or hospital, according to Communism).

To say in blunt words what an un-man is not particularly hard: it is a man who 
does not correspond to the \textit{concept} man, as the inhuman is something 
human which is not conformed to the concept of the human. Logic calls this a 
"{}self-contradictory judgment."{} Would it be permissible for one to 
pronounce this judgment, that one can be a man without being a man, if he did 
not admit the hypothesis that the concept of man can be separated from the 
existence, the essence from the appearance? They say, he \textit{appears} 
indeed as a man, but \textit{is} not a man.

Men have passed this "{}self-contradictory judgment"{} through a long line of 
centuries! Nay, what is still more, in this long time there were only -- 
\textit{un-men}. What individual can have corresponded to his concept? 
Christianity knows only one Man, and this one -- Christ -- is at once an 
un-man again in the reverse sense, to wit, a superhuman man, a "{}God."{} Only 
the -- un-man is a \textit{real} man.

Men that are not men, what should they be but \textit{ghosts?} Every real man, 
because he does not correspond to the concept "{}man,"{} or because he is not 
a "{}generic man,"{} is a spook. But do I still remain an un-man even if I 
bring Man (who towered above me and remained otherworldly to me only as my 
ideal, my task, my essence or concept) down to be my \textit{quality}, my own 
and inherent in me; so that Man is nothing else than my humanity, my human 
existence, and everything that I do is human precisely because \textit{I} do 
it, but not because it corresponds to the \textit{concept} "{}man"{}? 
\textit{I} am really Man and the un-man in one; for I am a man and at the same 
time more than a man; \textit{i.e.} I am the ego of this my mere quality.

It had to come to this at last, that it was no longer merely demanded of us to 
be Christians, but to become men; for, though we could never really become 
even Christians, but always remained "{}poor sinners"{} (for the Christian was 
an unattainable ideal too), yet in this the contradictoriness did not come 
before our consciousness so, and the illusion was easier than now when of us, 
who are men act humanly (yes, cannot do otherwise than be such and act so), 
the demand is made that we are to be men, "{}real men."{}

Our States of today, because they still have all sorts of things sticking to 
them, left from their churchly mother, do indeed load those who belong to them 
with various obligations (\textit{e. g.} churchly religiousness) which 
properly do not a bit concern them, the States; yet on the whole they do not 
deny their significance, since they want to be looked upon as \textit{human 
societies}, in which man as man can be a member, even if he is less privileged 
than other members; most of them admit adherence of every religious sect, and 
receive people without distinction of race or nation: Jews, Turks, Moors, 
etc., can become French citizens. In the act of reception, therefore, the 
State looks only to see whether one is a \textit{man}. The Church, as a 
society of believers, could not receive every man into her bosom; the State, 
as a society of men, can. But, when the State has carried its principle clear 
through, of presupposing in its constituents nothing but that they are men 
(even the North Americans still presuppose in theirs that they have religion, 
at least the religion of integrity, of responsibility), then it has dug its 
grave. While it will fancy that those whom it possesses are without exception 
men, these have meanwhile become without exception \textit{egoists}, each of 
whom utilizes it according to his egoistic powers and ends. Against the 
egoists "{}human society"{} is wrecked; for they no longer have to do with 
each other as \textit{men}, but appear egoistically as an \textit{I} against a 
\textit{You} altogether different from me and in opposition to me.

If the State must count on our humanity, it is the same if one says it must 
count on our \textit{morality}. Seeing Man in each other, and acting as men 
toward each other, is called moral behavior. This is every whit the 
"{}spiritual love"{} of Christianity. For, if I see Man in you, as in myself I 
see Man and nothing but Man, then I care for you as I would care for myself; 
for we represent, you see, nothing but the mathematical proposition: A = C and 
B = C, consequently A = B -- \textit{i.e.} I nothing but man and you nothing 
but man, consequently I and you the same. Morality is incompatible with 
egoism, because the former does not allow validity to \textit{me}, but only to 
the Man in me. But, if the State is a \textit{society of men}, not a union of 
egos each of whom has only himself before his eyes, then it cannot last 
without morality, and must insist on morality.

Therefore we two, the State and I, are enemies. I, the egoist, have not at 
heart the welfare of this "{}human society,"{} I sacrifice nothing to it, I 
only utilize it; but to be able to utilize it completely I transform it rather 
into my property and my creature; \textit{i. e.}, I annihilate it, and form in 
its place the \textit{Union of Egoists}.

So the State betrays its enmity to me by demanding that I be a man, which 
presupposes that I may also not be a man, but rank for it as an "{}un- man"{}; 
it imposes being a man upon me as a \textit{duty}. Further, it desires me to 
do nothing along with which \textit{it} cannot last; so \textit{its 
permanence} is to be sacred for me. Then I am not to be an egoist, but a 
"{}respectable, upright,"{} \textit{i.e.} moral, man. Enough: before it and 
its permanence I am to be impotent and respectful.

This State, not a present one indeed, but still in need of being first 
created, is the ideal of advancing liberalism. There is to come into existence 
a true "{}society of men,"{} in which every "{}man"{} finds room. Liberalism 
means to realize "{}Man,"{} \textit{i.e.} create a world for him; and this 
should be the \textit{human} world or the general (Communistic) society of 
men. It was said, "{}The Church could regard only the spirit, the State is to 
regard the whole man."{}\footnote{Hess, \textit{"{}Triarchie},"{} p. 76.} But 
is not "{}Man"{} "{}spirit"{}? The kernel of the State is simply "{}Man,"{} 
this unreality, and it itself is only a "{}society of men."{} The world which 
the believer (believing spirit) creates is called Church, the world which the 
man (human or humane spirit) creates is called State. But that is not 
\textit{my} world. I never execute anything \textit{human} in the abstract, 
but always my \textit{own} things; \textit{my} human act is diverse from every 
other human act, and only by this diversity is it a real act belonging to me. 
The human in it is an abstraction, and, as such, spirit, \textit{i.e.} 
abstracted essence.

Bruno Bauer states (\textit{e. g. Judenfrage}, p. 84) that the truth of 
criticism is the final truth, and in fact the truth sought for by Christianity 
itself --to wit, "{}Man."{} He says, "{}The history of the Christian world is 
the history of the supreme fight for truth, for in it -- and in it only! -- 
the thing at issue is the discovery of the final or the primal truth -- man 
and freedom."{}

All right, let us accept this gain, and let us take \textit{man} as the 
ultimately found result of Christian history and of the religious or ideal 
efforts of man in general. Now, who is Man? \textit{I} am! \textit{Man}, the 
end and outcome of Christianity, is, as \textit{I}, the beginning and raw 
material of the new history, a history of enjoyment after the history of 
sacrifices, a history not of man or humanity, but of -- \textit{me. Man} ranks 
as the general. Now then, I and the egoistic are the really general, since 
every one is an egoist and of paramount importance to himself. The Jewish is 
not the purely egoistic, because the Jew still devotes \textit{himself} to 
Jehovah; the Christian is not, because the Christian lives on the grace of God 
and subjects \textit{himself} to him. As Jew and as Christian alike a man 
satisfies only certain of his wants, only a certain need, not 
\textit{himself:} a half-egoism, because the egoism of a half-man, who is half 
he, half Jew, or half his own proprietor, half a slave. Therefore, too, Jew 
and Christian always half-way exclude each other; \textit{i.e.} as men they 
recognize each other, as slaves they exclude each other, because they are 
servants of two different masters. If they could be complete egoists, they 
would exclude each other \textit{wholly} and hold together so much the more 
firmly. Their ignominy is not that they exclude each other, but that this is 
done only \textit{half-way}. Bruno Bauer, on the contrary, thinks Jews and 
Christians cannot regard and treat each other as "{}men"{} till they give up 
the separate essence which parts them and obligates them to eternal 
separation, recognize the general essence of "{}Man,"{} and regard this as 
their "{}true essence."{}

According to his representation the defect of the Jews and the Christians 
alike lies in their wanting to be and have something "{}particular"{} instead 
of only being men and endeavoring after what is human -- to wit, the 
"{}general rights of man."{} He thinks their fundamental error consists in the 
belief that they are "{}privileged,"{} possess "{}prerogatives"{}; in general, 
in the belief in \textit{prerogative}.\footnote{[\textit{Vorrecht}, literally 
"{}precedent right."{}] }In opposition to this he holds up to them the general 
rights of man. The rights of man! --

\textit{Man is man in general}, and in so far every one who is a man. Now 
every one is to have the eternal rights of man, and, according to the opinion 
of Communism, enjoy them in the complete "{}democracy,"{} or, as it ought more 
correctly to be called -- anthropocracy. But it is I alone who have everything 
that I -- procure for myself; as man I have nothing. People would like to give 
every man an affluence of all good, merely because he has the title "{}man."{} 
But I put the accent on \textit{me}, not on my being \textit{man}.

Man is something only as \textit{my} quality\footnote{[\textit{Eigenschaft}]} 
(property\footnote{[\textit{Eigentum}]}), like masculinity or femininity. The 
ancients found the ideal in one's being \textit{male} in the full sense; their 
virtue is \textit{virtus} and \textit{arete} -- \textit{i.e.} manliness. What 
is one to think of a woman who should want only to be perfectly "{}woman?"{} 
That is not given to all, and many a one would therein be fixing for herself 
an unattainable goal. \textit{Feminine}, on the other hand, she is anyhow, by 
nature; femininity is her quality, and she does not need "{}true 
femininity."{} I am a man just as the earth is a star. As ridiculous as it 
would be to set the earth the task of being a "{}thorough star,"{} so 
ridiculous it is to burden me with the call to be a "{}thorough man."{}

When Fichte says, "{}The ego is all,"{} this seems to harmonize perfectly with 
my thesis. But it is not that the ego \textit{is} all, but the ego 
\textit{destroys} all, and only the self-dissolving ego, the never-being ego, 
the -- \textit{finite} ego is really I. Fichte speaks of the "{}absolute"{} 
ego, but I speak of me, the transitory ego.

How natural is the supposition that \textit{man} and \textit{ego} mean the 
same! And yet one sees, \textit{e. g.}, by Feuerbach, that the expression 
"{}man"{} is to designate the absolute ego, the \textit{species}, not the 
transitory, individual ego. Egoism and humanity (humaneness) ought to mean the 
same, but according to Feuerbach the individual can "{}only lift himself above 
the limits of his individuality, but not above the laws, the positive 
ordinances, of his species."{}\footnote{"{}Essence of Christianity,"{} 2nd 
ed., p. 401} But the species is nothing, and, if the individual lifts himself 
above the limits of his individuality, this is rather his very self as an 
individual; he exists only in raising himself, he exists only in not remaining 
what he is; otherwise he would be done, dead. Man with the great M is only an 
ideal, the species only something thought of. To be a man is not to realize 
the ideal of \textit{Man}, but to present \textit{oneself}, the individual. It 
is not how I realize the \textit{generally human} that needs to be my task, 
but how I satisfy myself. I am my species, am without norm, without law, 
without model, etc. It is possible that I can make very little out of myself; 
but this little is everything, and is better than what I allow to be made out 
of me by the might of others, by the training of custom, religion, the laws, 
the State. Better -- if the talk is to be of better at all -- better an 
unmannerly child than an old head on young shoulders, better a mulish man than 
a man compliant in everything. The unmannerly and mulish fellow is still on 
the way to form himself according to his own will; the prematurely knowing and 
compliant one is determined by the "{}species,"{} the general demands -- the 
species is law to him. He is \textit{determined}\footnote{[\textit{bestimmt}]} 
by it; for what else is the species to him but his 
"{}destiny,"{}\footnote{[\textit{Bestimmung}]} his "{}calling"{}? Whether I 
look to "{}humanity,"{} the species, in order to strive toward this ideal, or 
to God and Christ with like endeavor, where is the essential dissimilarity? At 
most the former is more washed-out than the latter. As the individual is the 
whole of nature, so he is the whole of the species too.

Everything that I do, think -- in short, my expression or manifestation -- is 
indeed \textit{conditioned} by what I am. The Jew \textit{e. g.} can will only 
thus or thus, can "{}present himself"{} only thus; the Christian can present 
and manifest himself only Christianly, etc. If it were possible that you could 
be a Jew or Christian, you would indeed bring out only what was Jewish or 
Christian; but it is not possible; in the most rigorous conduct you yet remain 
an \textit{egoist}, a sinner against that concept -- \textit{i.e.}, 
\textit{you} are not the precise equivalent of Jew. Now, because the egoistic 
always keeps peeping through, people have inquired for a more perfect concept 
which should really wholly express what you are, and which, because it is your 
true nature, should contain all the laws of your activity. The most perfect 
thing of the kind has been attained in "{}Man."{} As a Jew you are too little, 
and the Jewish is not your task; to be a Greek, a German, does not suffice. 
But be a -- man, then you have everything; look upon the human as your 
calling.

Now I know what is expected of me, and the new catechism can be written. The 
subject is again subjected to the predicate, the individual to something 
general; the dominion is again secured to an \textit{idea}, and the foundation 
laid for a new \textit{religion}. This is a \textit{step forward} in the 
domain of religion, and in particular of Christianity; not a step out beyond 
it.

To step out beyond it leads into the \textit{unspeakable}. For me paltry 
language has no word, and "{}the Word,"{} the Logos, is to me a "{}mere 
word."{}

My \textit{essence} is sought for. If not the Jew, the German, etc., then at 
any rate it is -- the man. "{}Man is my essence."{}

I am repulsive or repugnant to myself; I have a horror and loathing of myself, 
I am a horror to myself, or, I am never enough for myself and never do enough 
to satisfy myself. From such feelings springs self-dissolution or 
self-criticism. Religiousness begins with self-renunciation, ends with 
completed criticism.

I am possessed, and want to get rid of the "{}evil spirit."{} How do I set 
about it? I fearlessly commit the sin that seems to the Christian the most 
dire, the sin and blasphemy against the Holy Spirit. "{}He who blasphemes the 
Holy Spirit has no forgiveness forever, but is liable to the eternal 
judgment!"{}\footnote{Mark 3. 29.} I want no forgiveness, and am not afraid of 
the judgment.

\textit{Man} is the last evil \textit{spirit} or spook, the most deceptive or 
most intimate, the craftiest liar with honest mien, the father of lies.

The egoist, turning against the demands and concepts of the present, executes 
pitilessly the most measureless -- \textit{desecration}. Nothing is holy to 
him!

It would be foolish to assert that there is no power above mine. Only the 
attitude that I take toward it will be quite another than that of the 
religious age: I shall be the \textit{enemy} of -- every higher power, while 
religion teaches us to make it our friend and be humble toward it.

The \textit{desecrator} puts forth his strength against every \textit{fear of 
God}, for fear of God would determine him in everything that he left standing 
as sacred. Whether it is the God or the Man that exercises the hallowing power 
in the God-man -- whether, therefore, anything is held sacred for God's sake 
or for Man's (Humanity's) -- this does not change the fear of God, since Man 
is revered as "{}supreme essence,"{} as much as on the specifically religious 
standpoint God as "{}supreme essence"{} calls for our fear and reverence; both 
overawe us.

The fear of God in the proper sense was shaken long ago, and a more or less 
conscious "{}atheism,"{} externally recognizable by a wide-spread 
"{}unchurchliness,"{} has involuntarily become the mode. But what was taken 
from God has been superadded to Man, and the power of humanity grew greater in 
just the degree that of piety lost weight: "{}Man"{} is the God of today, and 
fear of Man has taken the place of the old fear of God.

But, because Man represents only another Supreme Being, nothing in fact has 
taken place but a metamorphosis in the Supreme Being, and the fear of Man is 
merely an altered form of the fear of God.

Our atheists are pious people.

If in the so-called feudal times we held everything as a fief from God, in the 
liberal period the same feudal relation exists with Man. God was the Lord, now 
Man is the Lord; God was the Mediator, now Man is; God was the Spirit, now Man 
is. In this three fold regard the feudal relation has experienced a 
transformation. For now, firstly, we hold as a fief from all-powerful Man our 
\textit{power}, which, because it comes from a higher, is not called power or 
might, but "{}right"{} -- the "{}rights of man"{}; we further hold as a fief 
from him our position in the world, for he, the mediator, mediates our 
\textit{intercourse} with others, which therefore may not be otherwise than 
"{}human"{}; finally, we hold as a fief from him ourselves -- to wit, our own 
value, or all that we are worth -- inasmuch as we are worth nothing when 
\textit{he} does not dwell in us, and when or where we are not "{}human."{} 
The power is Man's, the world is Man's, I am Man's.

But am I not still unrestrained from declaring \textit{myself} the entitler, 
the mediator, and the own self? Then it runs thus:

My power is \textit{my} property.

My power \textit{gives} me property.

My power \textit{am} I myself, and through it am I my property.

\section[1. My Power]{\centering 1. My Power}

Right\footnote{[This word has also, in German, the meaning of "{}common 
law,"{} and will sometimes be translated "{}law"{} in the following 
paragraphs.]} is the \textit{spirit of society}. If society has a 
\textit{will} this will is simply right: society exists only through right. 
But, as it endures only exercising a \textit{sovereignty} over individuals, 
right is its SOVEREIGN WILL. Aristotle says justice is the advantage of 
\textit{society}.

All existing right is -- \textit{foreign law;} some one makes me out to be in 
the right, "{}does right by me."{} But should I therefore be in the right if 
all the world made me out so? And yet what else is the right that I obtain in 
the State, in society, but a right of those \textit{foreign} to me? When a 
blockhead makes me out in the right, I grow distrustful of my rightness; I 
don't like to receive it from him. But, even when a wise man makes me out in 
the right, I nevertheless am not in the right on that account. Whether I am in 
the right is completely independent of the fool's making out and of the wise 
man's.

All the same, we have coveted this right till now. We seek for right, and turn 
to the court for that purpose. To what? To a royal, a papal, a popular court, 
etc. Can a sultanic court declare another right than that which the sultan has 
ordained to be right? Can it make me out in the right if I seek for a right 
that does not agree with the sultan's law? Can it, \textit{e. g.}, concede to 
me high treason as a right, since it is assuredly not a right according to the 
sultan's mind? Can it as a court of censorship allow me the free utterance of 
opinion as a right, since the sultan will hear nothing of this \textit{my} 
right? What am I seeking for in this court, then? I am seeking for sultanic 
right, not my right; I am seeking for -- \textit{foreign} right. As long as 
this foreign right harmonizes with mine, to be sure, I shall find in it the 
latter too.

The State does not permit pitching into each other man to man; it opposes the 
\textit{duel}. Even every ordinary appeal to blows, notwithstanding that 
neither of the fighters calls the police to it, is punished; except when it is 
not an I whacking away at a you, but, say, the \textit{head of a family} at 
the child. The \textit{family} is entitled to this, and in its name the 
father; I as Ego am not. The \textit{Vossische Zeitung} presents to us the 
"{}commonwealth of right."{} There everything is to be decided by the judge 
and a \textit{court}. It ranks the supreme court of censorship as a 
"{}court"{} where "{}right is declared."{} What sort of a right? The right of 
the censorship. To recognize the sentences of that court as right one must 
regard the censorship as right. But it is thought nevertheless that this court 
offers a protection. Yes, protection against an individual censor's error: it 
protects only the censorship-legislator against false interpretation of his 
will, at the same time making his statute, by the "{}sacred power of right,"{} 
all the firmer against writers.

Whether I am in the right or not there is no judge but myself. Others can 
judge only whether they endorse my right, and whether it exists as right for 
them too.

In the meantime let us take the matter yet another way. I am to reverence 
sultanic law in the sultanate, popular law in republics, canon law in Catholic 
communities. To these laws I am to subordinate myself; I am to regard them as 
sacred. A "{}sense of right"{} and "{}law-abiding mind"{} of such a sort is so 
firmly planted in people's heads that the most revolutionary persons of our 
days want to subject us to a new "{}sacred law,"{} the "{}law of society,"{} 
the law of mankind, the "{}right of all,"{} and the like. The right of 
"{}all"{} is to go before \textit{my} right. As a right of all it would indeed 
be my right among the rest, since I, with the rest, am included in all; but 
that it is at the same time a right of others, or even of all others, does not 
move me to its upholding. Not as a right \textit{of all} will I defend it, but 
as \textit{my} right; and then every other may see to it how he shall likewise 
maintain it for himself. The right of all (\textit{e. g.,} to eat) is a right 
of every individual. Let each keep this right unabridged for \textit{himself}, 
then all exercise it spontaneously; let him not take care for all though -- 
let him not grow zealous for it as for a right of all.

But the social reformers preach to us a \textit{"{}law of society"{}}. There 
the individual becomes society's slave, and is in the right only when society 
\textit{makes him out} in the right, \textit{i.e.} when he lives according to 
society's \textit{statutes} and so is -- \textit{loyal}. Whether I am loyal 
under a despotism or in a "{}society"{} \textit{\`ala} Weitling, it is the 
same absence of right in so far as in both cases I have not \textit{my} right 
but \textit{foreign} right.

In consideration of right the question is always asked, "{}What or who gives 
me the right to it?"{} Answer: God, love, reason, nature, humanity, etc. No, 
only \textit{your might, your} power gives you the right (your reason, 
\textit{e. g.,}, may give it to you).

Communism, which assumes that men "{}have equal rights by nature,"{} 
contradicts its own proposition till it comes to this, that men have no right 
at all by nature. For it is not willing to recognize, \textit{e. g.}, that 
parents have "{}by nature"{} rights as against their children, or the children 
as against the parents: it abolishes the family. Nature gives parents, 
brothers, etc., no right at all. Altogether, this entire revolutionary or 
Babouvist principle\footnote{Cf. \textit{"{}Die Kommunisten in der 
Schweiz},"{} committee report, p. 3.} rests on a religious, \textit{i. e.}, 
false, view of things. Who can ask after "{}right"{} if he does not occupy the 
religious standpoint himself? Is not "{}right"{} a religious concept, 
\textit{i.e.} something sacred? Why, \textit{"{}equality of rights"{}}, as the 
Revolution propounded it, is only another name for "{}Christian equality,"{} 
the "{}equality of the brethren,"{} "{}of God's children,"{} "{}of 
Christians"{}; in short, \textit{fraternit\'e}. Each and every inquiry after 
right deserves to be lashed with Schiller's words:

\begin{quotation}

\noindent{} Many a year I've used my nose\\
 To smell the onion and the rose;\\
 Is there any proof which shows\\
 That I've a right to that same nose? \end{quotation}

\noindent{}When the Revolution stamped equality as a "{}right,"{} it took 
flight into the religious domain, into the region of the sacred, of the ideal. 
Hence, since then, the fight for the "{}sacred, inalienable rights of man."{} 
Against the "{}eternal rights of man"{} the "{}well-earned rights of the 
established order"{} are quite naturally, and with equal right, brought to 
bear: right against right, where of course one is decried by the other as 
"{}wrong."{} This has been the \textit{contest of 
rights}\footnote{[\textit{Rechtsstreit}, a word which usually means 
"{}lawsuit."{}]} since the Revolution.

You want to be "{}in the right"{} as against the rest. That you cannot; as 
against them you remain forever "{}in the wrong"{}; for they surely would not 
be your opponents if they were not in "{}their right"{} too; they will always 
make you out "{}in the wrong."{} But, as against the right of the rest, yours 
is a higher, greater, \textit{more powerful} right, is it not? No such thing! 
Your right is not more powerful if you are not more powerful. Have Chinese 
subjects a right to freedom? Just bestow it on them, and then look how far you 
have gone wrong in your attempt: because they do not know how to use freedom 
they have no right to it, or, in clearer terms, because they have not freedom 
they have not the right to it. Children have no right to the condition of 
majority because they are not of age, \textit{i.e.} because they are children. 
Peoples that let themselves be kept in nonage have no rights to the condition 
of majority; if they ceased to be in nonage, then only would they have the 
right to be of age. This means nothing else than "{}What you have the 
\textit{power} to be you have the \textit{right} to."{} I derive all right and 
all warrant from \textit{me ;} I am \textit{entitled} to everything that I 
have in my power. I am entitled to overthrow Zeus, Jehovah, God, etc., if I 
\textit{can ;} if I cannot, then these gods will always remain in the right 
and in power as against me, and what I do will be to fear their right and 
their power in impotent "{}god-fearingness,"{} to keep their commandments and 
believe that I do right in everything that I do according to \textit{their} 
right, about as the Russian boundary-sentinels think themselves rightfully 
entitled to shoot dead the suspicious persons who are escaping, since they 
murder "{}by superior authority,"{} \textit{i.e.} "{}with right."{} But I am 
entitled by myself to murder if I myself do not forbid it to myself, if I 
myself do not fear murder as a "{}wrong."{} This view of things lies at the 
foundation of Chamisso's poem, "{}The Valley of Murder,"{} where the 
gray-haired Indian murderer compels reverence from the white man whose 
brethren he has murdered. The only thing I am not entitled to is what I do not 
do with a free cheer, \textit{i. e.} what I do not entitle myself to.

I decide whether it is the \textit{right thing} in me; there is no right 
\textit{outside} me. If it is right for \textit{me},\footnote{[A common German 
phrase for "{}it suits me."{}]} it is right. Possibly this may not suffice to 
make it right for the rest; \textit{i. e.}, their care, not mine: let them 
defend themselves. And if for the whole world something were not right, but it 
were right for me, \textit{i. e.}, I wanted it, then I would ask nothing about 
the whole world. So every one does who knows how to value \textit{himself}, 
every one in the degree that he is an egoist; for might goes before right, and 
that -- with perfect right.

Because I am "{}by nature"{} a man I have an equal right to the enjoyment of 
all goods, says Babeuf. Must he not also say: because I am "{}by nature"{} a 
first-born prince I have a right to the throne? The rights of man and the 
"{}well-earned rights"{} come to the same thing in the end, \textit{i.e.} to 
\textit{nature}, which \textit{gives} me a right, \textit{i. e.} to 
\textit{birth} (and, further, inheritance, etc.). "{}I am born as a man"{} is 
equal to "{}I am born as a king's son."{} The natural man has only a natural 
right (because he has only a natural power) and natural claims: he has right 
of birth and claims of birth. But \textit{nature} cannot entitle me, 
\textit{i.e.} give me capacity or might, to that to which only my act entitles 
me. That the king's child sets himself above other children, even this is his 
act, which secures to him the precedence; and that the other children approve 
and recognize this act is their act, which makes them worthy to be -- 
subjects.

Whether nature gives me a right, or whether God, the people's choice, etc., 
does so, all of \textit{i. e.}, the same \textit{foreign} right, a right that 
I do not give or take to myself.

Thus the Communists say, equal labor entitles man to equal enjoyment. Formerly 
the question was raised whether the "{}virtuous"{} man must not be "{}happy"{} 
on earth. The Jews actually drew this inference: "{}That it may go well with 
thee on earth."{} No, equal labor does not entitle you to it, but equal 
enjoyment alone entitles you to equal enjoyment. Enjoy, then you are entitled 
to enjoyment. But, if you have labored and let the enjoyment be taken from 
you, then -- "{}it serves you right."{}

If you \textit{take} the enjoyment, it is your right; if, on the contrary, you 
only pine for it without laying hands on it, it remains as before, a, 
"{}well-earned right"{} of those who are privileged for enjoyment. It is 
\textit{their} right, as by laying hands on it would become your right.

The conflict over the "{}right of property"{} wavers in vehement commotion. 
The Communists affirm\footnote{A. Becker, \textit{"{}Volksphilosophie}"{}, p. 
22f.} that "{}the earth belongs rightfully to him who tills it, and its 
products to those who bring them out."{} I think it belongs to him who knows 
how to take it, or who does not let it be taken from him, does not let himself 
be deprived of it. If he appropriates it, then not only the earth, but the 
right to it too, belongs to him. This is \textit{egoistic right}: 
\textit{i.e.} it is right for \textit{me}, therefore it is right.

Aside from this, right does have "{}a wax nose."{} The tiger that assails me 
is in the right, and I who strike him down am also in the right. I defend 
against him not my \textit{right}, but \textit{myself.}

As human right is always something given, it always in reality reduces to the 
right which men give, \textit{i.e.} "{}concede,"{} to each other. If the right 
to existence is conceded to new-born children, then they have the right; if it 
is not conceded to them, as was the case among the Spartans and ancient 
Romans, then they do not have it. For only society can give or concede it to 
them; they themselves cannot take it, or give it to themselves. It will be 
objected, the children had nevertheless "{}by nature"{} the right to exist; 
only the Spartans refused \textit{recognition} to this right. But then they 
simply had no right to this recognition -- no more than they had to 
recognition of their life by the wild beasts to which they were thrown.

People talk so much about \textit{birthright} and complain:

\begin{quotation}

\noindent{} There is alas! -- no mention of the rights\\
 That were born with us.\footnote{[Mephistopheles in "{}Faust."{}]} 
\end{quotation}

\noindent{}What sort of right, then, is there that was born with me? The right 
to receive an entailed estate, to inherit a throne, to enjoy a princely or 
noble education; or, again, because poor parents begot me, to -- get free 
schooling, be clothed out of contributions of alms, and at last earn my bread 
and my herring in the coal-mines or at the loom? Are these not birthrights, 
rights that have come down to me from my parents through \textit{birth?} You 
think -- no; you think these are only rights improperly so called, it is just 
these rights that you aim to abolish through the \textit{real birthright}. To 
give a basis for this you go back to the simplest thing and affirm that every 
one is by birth \textit{equal} to another -- to wit, a \textit{man}. I will 
grant you that every one is born as man, hence the new-born are therein 
\textit{equal} to each other. Why are they? Only because they do not yet show 
and exert themselves as anything but bare -- \textit{children of men}, naked 
little human beings. But thereby they are at once different from those who 
have already made something out of themselves, who thus are no longer bare 
"{}children of man,"{} but -- children of their own creation. The latter 
possesses more than bare birthrights: they have \textit{earned} rights. What 
an antithesis, what a field of combat! The old combat of the birthrights of 
man and well-earned rights. Go right on appealing to your birthrights; people 
will not fail to oppose to you the well-earned. Both stand on the "{}ground of 
right"{}; for each of the two has a "{}right"{} against the other, the one the 
birthright of natural right, the other the earned or "{}well-earned"{} right.

If you remain on the ground of right, you remain in -- 
\textit{Rechthaberei}.\footnote{"{}I beg you, spare my lungs! He who insists 
on proving himself right, if he but has one of those things called tongues, 
can hold his own in all the world's despite!"{} [Faust's words to 
Mephistopheles, slightly misquoted. -- For \textit{Rechthaberei}see note on p. 
185.]} The other cannot give you your right; he cannot "{}mete out right"{} to 
you. He who has might has -- right; if you have not the former, neither have 
you the latter. Is this wisdom so hard to attain? Just look at the mighty and 
their doings! We are talking here only of China and Japan, of course. Just try 
it once, you Chinese and Japanese, to make them out in the wrong, and learn by 
experience how they throw you into jail. (Only do not confuse with this the 
"{}well-meaning counsels"{} which -- in China and Japan -- are permitted, 
because they do not hinder the mighty one, but possibly \textit{help him on}.) 
For him who should want to make them out in the wrong there would stand open 
only one way thereto, that of might. If he deprives them of their 
\textit{might}, then he has \textit{really} made them out in the wrong, 
deprived them of their right; in any other case he can do nothing but clench 
his little fist in his pocket, or fall a victim as an obtrusive fool.

In short, if you Chinese or Japanese did not ask after right, and in 
particular if you did not ask after the rights "{}that were born with you,"{} 
then you would not need to ask at all after the well-earned rights either.

You start back in fright before others, because you think you see beside them 
the \textit{ghost of right}, which, as in the Homeric combats, seems to fight 
as a goddess at their side, helping them. What do you do? Do you throw the 
spear? No, you creep around to gain the spook over to yourselves, that it may 
fight on your side: you woo for the ghost's favor. Another would simply ask 
thus: Do I will what my opponent wills? "{}No!"{} Now then, there may fight 
for him a thousand devils or gods, I go at him all the same!

The "{}commonwealth of right,"{} as the \textit{Vossische Zeitung} among 
others stands for it, asks that office-holders be removable only by the 
\textit{judge}, not by the \textit{administration}. Vain illusion! If it were 
settled by law that an office-holder who is once seen drunken shall lose his 
office, then the judges would have to condemn him on the word of the 
witnesses. In short, the law-giver would only have to state precisely all the 
possible grounds which entail the loss of office, however laughable they might 
be (\textit{e. g.} he who laughs in his superiors' faces, who does not go to 
church every Sunday, who does not take the communion every four weeks, who 
runs in debt, who has disreputable associates, who shows no determination, 
etc., shall be removed. These things the law-giver might take it into his head 
to prescribe, \textit{e. g.}, for a court of honor); then the judge would 
solely have to investigate whether the accused had "{}become guilty"{} of 
those "{}offenses,"{} and, on presentation of the proof, pronounce sentence of 
removal against him "{}in the name of the law."{}

The judge is lost when he ceases to be \textit{mechanical}, when he "{}is 
forsaken by the rules of evidence."{} Then he no longer has anything but an 
opinion like everybody else; and, if he decides according to this 
\textit{opinion}, his action is \textit{no longer an official action}. As 
judge he must decide only according to the law. Commend me rather to the old 
French parliaments, which wanted to examine for themselves what was to be 
matters of right, and to register it only after their own approval. They at 
least judged according to a right of their own, and were not willing to give 
themselves up to be machines of the law-giver, although as judges they must, 
to be sure, become their own machines.

It is said that punishment is the criminal's right. But impunity is just as 
much his right. If his undertaking succeeds, it serves him right, and, if it 
does not succeed, it likewise serves him right. You make your bed and lie in 
it. If some one goes foolhardily into dangers and perishes in them, we are apt 
to say, "{}It serves him right; he would have it so."{} But, if he conquered 
the dangers, \textit{i.e.} if his \textit{might} was victorious, then he would 
be in the \textit{right} too. If a child plays with the knife and gets cut, it 
is served right; but, if it doesn't get cut, it is served right too. Hence 
right befalls the criminal, doubtless, when he suffers what he risked; why, 
what did he risk it for, since he knew the possible consequences? But the 
punishment that we decree against him is only our right, not his. Our right 
reacts against his, and he is -- "{}in the wrong at last"{} because -- we get 
the upper hand.

\begin{center}
--------\end{center}


But what is right, what is matter of right in a society, is voiced too -- in 
the law.\footnote{[\textit{Gesetz}, statute; no longer the same German word as 
"{}right"{}]}

Whatever the law may be, it must be respected by the -- loyal citizen. Thus 
the law-abiding mind of Old England is eulogized. To this that Euripidean 
sentiment (Orestes, 418) entirely corresponds: "{}We serve the gods, whatever 
the gods are."{} \textit{Law as such, God as such}, thus far we are today.

People are at pains to distinguish\textit{law} from arbitrary \textit{orders}, 
from an ordinance: the former comes from a duly entitled authority. But a law 
over human action (ethical law, State law, etc.) is always a 
\textit{declaration of will}, and so an order. Yes, even if I myself gave 
myself the law, it would yet be only my order, to which in the next moment I 
can refuse obedience. One may well enough declare what he will put up with, 
and so deprecate the opposite of the law, making known that in the contrary 
case he will treat the transgressor as his enemy; but no one has any business 
to command \textit{my} actions, to say what course I shall pursue and set up a 
code to govern it. I must put up with it that he treats me as his 
\textit{enemy}, but never that he makes free with me as his \textit{creature}, 
and that he makes \textit{his} reason, or even unreason, my plumbline.

States last only so long as there is \textit{a ruling will} and this ruling 
will is looked upon as tantamount to the own will. The lord's will is -- law. 
What do your laws amount to if no one obeys them? What your orders, if nobody 
lets himself be ordered? The State cannot forbear the claim to determine the 
individual's will, to speculate and count on this. For the State it is 
indispensable that nobody have an \textit{own will ;} if one had, the State 
would have to exclude (lock up, banish, etc.) this one; if all had, they would 
do away with the State. The State is not thinkable without lordship and 
servitude (subjection); for the State must will to be the lord of all that it 
embraces, and this will is called the "{}will of the State."{}

He who, to hold his own, must count on the absence of will in others is a 
thing made by these others, as the master is a thing made by the servant. If 
submissiveness ceased, it would be over with all lordship.

The \textit{own will} of Me is the State's destroyer; it is therefore branded 
by the State as "{}self-will."{} Own will and the State are powers in deadly 
hostility, between which no "{}eternal peace"{} is possible. As long as the 
State asserts itself, it represents own will, its ever-hostile opponent, as 
unreasonable, evil; and the latter lets itself be talked into believing this 
-- nay, it really is such, for no more reason than this, that it still lets 
itself be talked into such belief: it has not yet come to itself and to the 
consciousness of its dignity; hence it is still incomplete, still amenable to 
fine words, etc.

Every State is a \textit{despotism}, be the despot one or many, or (as one is 
likely to imagine about a republic) if all be lords, \textit{i. e.} despotize 
one over another. For this is the case when the law given at any time, the 
expressed volition of (it may be) a popular assembly, is thenceforth to be 
\textit{law} for the individual, to which \textit{obedience is due} from him 
or toward which he has the \textit{duty} of obedience. If one were even to 
conceive the case that every individual in the people had expressed the same 
will, and hereby a complete "{}collective will"{} had come into being, the 
matter would still remain the same. Would I not be bound today and henceforth 
to my will of yesterday? My will would in this case be \textit{frozen}. 
Wretched \textit{stability!} My creature -- to wit, a particular expression of 
will -- would have become my commander. But I in my will, I the creator, 
should be hindered in my flow and my dissolution. Because I was a fool 
yesterday I must remain such my life long. So in the State-life I am at best 
-- I might just as well say, at worst -- a bondman of myself. Because I was a 
willer yesterday, I am today without will: yesterday voluntary, today 
involuntary.

How change it? Only be recognizing no \textit{duty}, not \textit{binding} 
myself nor letting myself be bound. If I have no duty, then I know no law 
either.

"{}But they will bind me!"{} My will nobody can bind, and my disinclination 
remains free.

"{}Why, everything must go topsy-turvy if every one could do what he would!"{} 
Well, who says that every one can do everything? What are you there for, pray, 
you who do not need to put up with everything? Defend yourself, and no one 
will do anything to you! He who would break your will has to do with you, and 
is your \textit{enemy}. Deal with him as such. If there stand behind you for 
your protection some millions more, then you are an imposing power and will 
have an easy victory. But, even if as a power you overawe your opponent, still 
you are not on that account a hallowed authority to him, unless he be a 
simpleton. He does not owe you respect and regard, even though he will have to 
consider your might.

We are accustomed to classify States according to the different ways in which 
"{}the supreme might"{} is distributed. If an individual has it -- monarchy; 
if all have it -- democracy; etc. Supreme might then! Might against whom? 
Against the individual and his "{}self-will."{} The State practices 
"{}violence,"{} the individual must not do so. The State's behavior is 
violence, and it calls its violence "{}law"{}; that of the individual, 
"{}crime."{} Crime, then\footnote{[\textit{Verbrechen}]} -- so the 
individual's violence is called; and only by crime does he 
overcome\footnote{[\textit{brechen}]} the State's violence when he thinks that 
the State is not above him, but he is above the State.

Now, if I wanted to act ridiculously, I might, as a well-meaning person, 
admonish you not to make laws which impair my self-development, self-activity, 
self-creation. I do not give this advice. For, if you should follow it, you 
would be unwise, and I should have been cheated of my entire profit. I request 
nothing at all from you; for, whatever I might demand, you would still be 
dictatorial law-givers, and must be so, because a raven cannot sing, nor a 
robber live without robbery. Rather do I ask those who would be egoists what 
they think the more egoistic -- to let laws be given them by you, and to 
respect those that are given, or to practice \textit{refractoriness}, yes, 
complete disobedience. Good-hearted people think the laws ought to prescribe 
only what is accepted in the people's feeling as right and proper. But what 
concern is it of mine what is accepted in the nation and by the nation? The 
nation will perhaps be against the blasphemer; therefore a law against 
blasphemy. Am I not to blaspheme on that account? Is this law to be more than 
an "{}order"{} to me? I put the question.

Solely from the principle that all \textit{right} and all \textit{authority} 
belong to the \textit{collectivity of the people} do all forms of government 
arise. For none of them lacks this appeal to the collectivity, and the despot, 
as well as the president or any aristocracy, acts and commands "{}in the name 
of the State."{} They are in possession of the "{}authority of the State,"{} 
and it is perfectly indifferent whether, were this possible, the people as a 
\textit{collectivity} (all individuals) exercise this State -- 
\textit{authority}, or whether it is only the representatives of this 
collectivity, be there many of them as in aristocracies or one as in 
monarchies. Always the collectivity is above the individual, and has a power 
which is called \textit{legitimate}, \textit{i.e.} which is \textit{law}.

Over against the sacredness of the State, the individual is only a vessel of 
dishonor, in which "{}exuberance, malevolence, mania for ridicule and slander, 
frivolity,"{} etc., are left as soon as he does not deem that object of 
veneration, the State, to be worthy of recognition. The spiritual 
\textit{haughtiness} of the servants and subjects of the State has fine 
penalties against unspiritual "{}exuberance."{}

When the government designates as punishable all play of mind \textit{against} 
the State, the moderate liberals come and opine that fun, satire, wit, humor, 
must have free play anyhow, and \textit{genius} must enjoy freedom. So not the 
\textit{individual man} indeed, but still \textit{genius}, is to be free. Here 
the State, or in its name the government, says with perfect right: He who is 
not for me is against me. Fun, wit, etc. -- in short, the turning of State 
affairs into a comedy -- have undermined States from of old: they are not 
"{}innocent."{} And, further, what boundaries are to be drawn between guilty 
and innocent wit, etc.? At this question the moderates fall into great 
perplexity, and everything reduces itself to the prayer that the State 
(government) would please not be so \textit{sensitive}, so \textit{ticklish ;} 
that it would not immediately scent malevolence in "{}harmless"{} things, and 
would in general be a little "{}more tolerant."{} Exaggerated sensitiveness is 
certainly a weakness, its avoidance may be praiseworthy virtue; but in time of 
war one cannot be sparing, and what may be allowed under peaceable 
circumstances ceases to be permitted as soon as a state of siege is declared. 
Because the well-meaning liberals feel this plainly, they hasten to declare 
that, considering "{}the devotion of the people,"{} there is assuredly no 
danger to be feared. But the government will be wiser, and not let itself be 
talked into believing anything of that sort. It knows too well how people 
stuff one with fine words, and will not let itself be satisfied with the 
Barmecide dish.

But they are bound to have their play-ground, for they are children, you know, 
and cannot be so staid as old folks; boys will be boys. Only for this 
playground, only for a few hours of jolly running about, they bargain. They 
ask only that the State should not, like a splenetic papa, be too cross. It 
should permit some Processions of the Ass and plays of fools, as the church 
allowed them in the Middle Ages. But the times when it could grant this 
without danger are past. Children that now once come \textit{into the open}, 
and live through an hour without the rod of discipline, are no longer willing 
to go into the \textit{cell}. For the open is now no longer a 
\textit{supplement} to the cell, no longer a refreshing \textit{recreation}, 
but its \textit{opposite}, an \textit{aut-aut}. In short, the State must 
either no longer put up with anything, or put up with everything and perish; 
it must be either sensitive through and through, or, like a dead man, 
insensitive. Tolerance is done with. If the State but gives a finger, they 
take the whole hand at once. There can be no more "{}jesting,"{} and all jest, 
such as fun, wit, humor, becomes bitter earnest.

The clamor of the Liberals for freedom of the press runs counter to their own 
principle, their proper \textit{will}. They will what they \textit{do not 
will}, \textit{i.e.} they wish, they would like. Hence it is too that they 
fall away so easily when once so-called freedom of the press appears; then 
they would like censorship. Quite naturally. The State is sacred even to them; 
likewise morals. They behave toward it only as ill-bred brats, as tricky 
children who seek to utilize the weaknesses of their parents. Papa State is to 
permit them to say many things that do not please him, but papa has the right, 
by a stern look, to blue-pencil their impertinent gabble. If they recognize in 
him their papa, they must in his presence put up with the censorship of 
speech, like every child.

\begin{center}
--------\end{center}


If you let yourself be made out in the right by another, you must no less let 
yourself be made out in the wrong by him; if justification and reward come to 
you from him, expect also his arraignment and punishment. Alongside right goes 
wrong, alongside legality \textit{crime}. What are you? -- You are a -- 
\textit{criminal!}

"{}The criminal is in the utmost degree the State's own crime!"{} says 
Bettina.\footnote{"{}This Book Belongs to the King,"{}, p. 376.} One may let 
this sentiment pass, even if Bettina herself does not understand it exactly 
so. For in the State the unbridled I -- I, as I belong to myself alone -- 
cannot come to my fulfillment and realization. Every ego is from birth a 
criminal to begin with against the people, the State. Hence it is that it does 
really keep watch over all; it sees in each one an -- egoist, and it is afraid 
of the egoist. It presumes the worst about each one, and takes care, 
police-care, that "{}no harm happens to the State,"{} \textit{ne quid 
respublica detrimenti capiat}. The unbridled ego -- and this we originally 
are, and in our secret inward parts we remain so always -- is the 
never-ceasing criminal in the State. The man whom his boldness, his will, his 
inconsiderateness and fearlessness lead is surrounded with spies by the State, 
by the people. I say, by the people! The people (think it something wonderful, 
you good-hearted folks, what you have in the people) -- the people is full of 
police sentiments through and through. -- Only he who renounces his ego, who 
practices "{}self-renunciation,"{} is acceptable to the people.

In the book cited Bettina is throughout good-natured enough to regard the 
State as only sick, and to hope for its recovery, a recovery which she would 
bring about through the "{}demagogues"{};\footnote{P. 376.} but it is not 
sick; rather is it in its full strength, when it puts from it the demagogues 
who want to acquire something for the individuals, for "{}all."{} In its 
believers it is provided with the best demagogues (leaders of the people). 
According to Bettina, the State is to\footnote{P. 374.} "{}develop mankind's 
germ of freedom; otherwise it is a raven-mother\footnote{[An unnatural 
mother]} and caring for raven-fodder!"{} It cannot do otherwise, for in its 
very caring for "{}mankind"{} (which, besides, would have to be the 
"{}humane"{} or "{} free"{} State to begin with) the "{}individual"{} is 
raven-fodder for it. How rightly speaks the burgomaster, on the other 
hand:\footnote{P. 381.} "{}What? the State has no other duty than to be merely 
the attendant of incurable invalids? -- that isn't to the point. From of old 
the healthy State has relieved itself of the diseased matter, and not mixed 
itself with it. It does not need to be so economical with its juices. Cut off 
the robber-branches without hesitation, that the others may bloom. -- Do not 
shiver at the State's harshness; its morality, its policy and religion, point 
it to that. Accuse it of no want of feeling; its sympathy revolts against 
this, but its experience finds safety only in this severity! There are 
diseases in which only drastic remedies will help. The physician who 
recognizes the disease as such, but timidly turns to palliatives, will never 
remove the disease, but may well cause the patient to succumb after a shorter 
or longer sickness."{} Frau Rat's question, "{}If you apply death as a drastic 
remedy, how is the cure to be wrought then?"{} isn't to the point. Why, the 
State does not apply death against itself, but against an offensive member; it 
tears out an eye that offends it, etc.

"{}For the invalid State the only way of salvation is to make man flourish in 
it."{}\footnote{P. 385.} If one here, like Bettina, understand by man the 
concept "{}Man,"{} she is right; the "{}invalid"{} State will recover by the 
flourishing of "{}Man,"{} for, the more infatuated the individuals are with 
"{}Man,"{} the better it serves the State's turn. But, if one referred it to 
the individuals, to "{}all"{} (and the authoress half-does this too, because 
about "{}Man"{} she is still involved in vagueness), then it would sound 
somewhat like the following: For an invalid band of robbers the only way of 
salvation is to make the loyal citizen nourish in it! Why, thereby the band of 
robbers would simply go to ruin as a band of robbers; and, because it 
perceives this, it prefers to shoot every one who has a leaning toward 
becoming a "{}steady man."{}

In this book Bettina is a patriot, or, what is little more, a philanthropist, 
a worker for human happiness. She is discontented with the existing order in 
quite the same way as is the title-ghost of her book, along with all who would 
like to bring back the good old faith and what goes with it. Only she thinks, 
contrariwise, that the politicians, place-holders, and diplomats ruined the 
State, while those lay it at the door of the malevolent, the "{}seducers of 
the people."{}

What is the ordinary criminal but one who has committed the fatal mistake of 
endeavoring after what is the people's instead of seeking for what is his? He 
has sought despicable \textit{alien} goods, has done what believers do who 
seek after what is God's. What does the priest who admonishes the criminal do? 
He sets before him the great wrong of having desecrated by his act what was 
hallowed by the State, its property (in which, of course, must be included 
even the life of those who belong to the State); instead of this, he might 
rather hold up to him the fact that he has befouled \textit{himself} in not 
despising the alien thing, but thinking it worth stealing; he could, if he 
were not a parson. Talk with the so-called criminal as with an egoist, and he 
will be ashamed, not that he transgressed against your laws and goods, but 
that he considered your laws worth evading, your goods worth desiring; he will 
be ashamed that he did not -- despise you and yours together, that he was too 
little an egoist. But you cannot talk egoistically with him, for you are not 
so great as a criminal, you -- commit no crime! You do not know that an ego 
who is his own cannot desist from being a criminal, that crime is his life. 
And yet you should know it, since you believe that "{}we are all miserable 
sinners"{}; but you think surreptitiously to get beyond sin, you do not 
comprehend -- for you are devil-fearing -- that guilt is the value of a man. 
Oh, if you were guilty! But now you are 
"{}righteous."{}\footnote{[\textit{Gerechte}]} Well -- just put every thing 
nicely to rights\footnote{[\textit{macht Alles h\"ubsch gerecht}]} for your 
master!

When the Christian consciousness, or the Christian man, draws up a criminal 
code, what can the concept of \textit{crime} be there but simply -- 
\textit{heartlessness?} Each severing and wounding of a \textit{heart 
relation}, each \textit{heartless behavior} toward a sacred being, is crime. 
The more heartfelt the relation is supposed to be, the more scandalous is the 
deriding of it, and the more worthy of punishment the crime. Everyone who is 
subject to the lord should love him; to deny this love is a high treason 
worthy of death. Adultery is a heartlessness worthy of punishment; one has no 
heart, no enthusiasm, no pathetic feeling for the sacredness of marriage. So 
long as the heart or soul dictates laws, only the heartful or soulful man 
enjoys the protection of the laws. That the man of soul makes laws means 
properly that the \textit{moral} man makes them: what contradicts these men's 
"{}moral feeling,"{} this they penalize. How, \textit{e. g.}, should 
disloyalty, secession, breach of oaths -- in short, all \textit{radical 
breaking off}, all tearing asunder of venerable \textit{ties --} not be 
flagitious and criminal in their eyes? He who breaks with these demands of the 
soul has for enemies all the moral, all the men of soul. Only Krummacher and 
his mates are the right people to set up consistently a penal code of the 
heart, as a certain bill sufficiently proves. The consistent legislation of 
the Christian State must be placed wholly in the hands of the -- 
\textit{parsons}, and will not become pure and coherent so long as it is 
worked out only by -- the \textit{parson-ridden}, who are always only 
\textit{half-parsons}. Only then will every lack of soulfulness, every 
heartlessness, be certified as an unpardonable crime, only then will every 
agitation of the soul become condemnable, every objection of criticism and 
doubt be anathematized; only then is the own man, before the Christian 
consciousness, a convicted -- \textit{criminal} to begin with.

The men of the Revolution often talked of the people's "{}just revenge"{} as 
its "{}right."{} Revenge and right coincide here. Is this an attitude of an 
ego to an ego? The people cries that the opposite party has committed 
"{}crimes"{} against it. Can I assume that one commits a crime against me, 
without assuming that he has to act as I see fit? And this action I call the 
right, the good, etc.; the divergent action, a crime. So I think that the 
others must aim at the \textit{same} goal with me; \textit{i.e.}, I do not 
treat them as unique beings\footnote{[\textit{Einzige}]} who bear their law in 
themselves and live according to it, but as beings who are to obey some 
"{}rational"{} law. I set up what "{}Man"{} is and what acting in a "{}truly 
human"{} way is, and I demand of every one that this law become norm and ideal 
to him; otherwise he will expose himself as a "{}sinner and criminal."{} But 
upon the "{}guilty"{} falls the "{}penalty of the law"{}!

One sees here how it is "{}Man"{} again who sets on foot even the concept of 
crime, of sin, and therewith that of right. A man in whom I do not recognize 
"{}man"{} is "{}sinner, a guilty one."{}

Only against a sacred thing are there criminals; you against me can never be a 
criminal, but only an opponent. But not to hate him who injures a sacred thing 
is in itself a crime, as St. Just cries out against Danton: "{}Are you not a 
criminal and responsible for not having hated the enemies of the 
fatherland?"{} --

If, as in the Revolution, what "{}Man"{} is apprehended as "{}good citizen,"{} 
then from this concept of "{}Man"{} we have the well-known "{}political 
offenses and crimes."{}

In all this the individual, the individual man, is regarded as refuse, and on 
the other hand the general man, "{}Man,"{} is honored. Now, according to how 
this ghost is named -- as Christian, Jew, Mussulman, good citizen, loyal 
subject, freeman, patriot, etc. -- just so do those who would like to carry 
through a divergent concept of man, as well as those who want to put 
\textit{themselves} through, fall before victorious "{}Man."{}

And with what unction the butchery goes on here in the name of the law, of the 
sovereign people, of God, etc.!

Now, if the persecuted trickily conceal and protect themselves from the stern 
parsonical judges, people stigmatize them as St. Just, \textit{e. g.}, does 
those whom he accuses in the speech against Danton.\footnote{See "{}Political 
Speeches,"{} 10, p. 153} One is to be a fool, and deliver himself up to their 
Moloch.

Crimes spring from \textit{fixed ideas}. The sacredness of marriage is a fixed 
idea. From the sacredness it follows that infidelity is a \textit{crime}, and 
therefore a certain marriage law imposes upon it a shorter or longer 
\textit{penalty}. But by those who proclaim "{}freedom as sacred"{} this 
penalty must be regarded as a crime against freedom, and only in this sense 
has public opinion in fact branded the marriage law.

Society would have \textit{every one} come to his right indeed, but yet only 
to that which is sanctioned by society, to the society-right, not really to 
\textit{his} right. But I give or take to myself the right out of my own 
plenitude of power, and against every superior power I am the most impenitent 
criminal. Owner and creator of my right, I recognize no other source of right 
than -- me, neither God nor the State nor nature nor even man himself with his 
"{}eternal rights of man,"{} neither divine nor human right.

Right "{}in and for itself."{} Without relation to me, therefore! "{}Absolute 
right."{} Separated from me, therefore! A thing that exists in and for itself! 
An absolute! An eternal right, like an eternal truth!

According to the liberal way of thinking, right is to be obligatory for me 
because it is thus established by \textit{human reason}, against which 
\textit{my reason} is "{}unreason."{} Formerly people inveighed in the name of 
divine reason against weak human reason; now, in the name of strong human 
reason, against egoistic reason, which is rejected as "{}unreason."{} And yet 
none is real but this very "{}unreason."{} Neither divine nor human reason, 
but only your and my reason existing at any given time, is real, as and 
because you and I are real.

The thought of right is originally my thought; or, it has its origin in me. 
But, when it has sprung from me, when the "{}Word"{} is out, then it has 
"{}become flesh,"{} it is a \textit{fixed idea}. Now I no longer get rid of 
the thought; however I turn, it stands before me. Thus men have not become 
masters again of the thought "{}right,"{} which they themselves created; their 
creature is running away with them. This is absolute right, that which is 
absolved or unfastened from me. We, revering it as absolute, cannot devour it 
again, and it takes from us the creative power: the creature is more than the 
creator, it is "{}in and for itself."{}

Once you no longer let right run around free, once you draw it back into its 
origin, into you, it is \textit{your} right; and that is right which suits 
you.

\begin{center}
--------\end{center}


Right has had to suffer an attack within itself, \textit{i.e.} from the 
standpoint of right; war being declared on the part of liberalism against 
"{}privilege."{}\footnote{[Literally, "{}precedent right."{}]}

\textit{Privileged} and \textit{endowed with equal rights --} on these two 
concepts turns a stubborn fight. Excluded or admitted -- would mean the same. 
But where should there be a power -- be it an imaginary one like God, law, or 
a real one like I, you -- of which it should not be true that before it all 
are "{}endowed with equal rights,"{} \textit{i. e.}, no respect of persons 
holds? Every one is equally dear to God if he adores him, equally agreeable to 
the law \textit{if} only he is a law- abiding person; whether the lover of God 
and the law is humpbacked and lame, whether poor or rich, etc., that amounts 
to nothing for God and the law; just so, when you are at the point of 
drowning, you like a Negro as rescuer as well as the most excellent Caucasian 
-- yes, in this situation you esteem a dog not less than a man. But to whom 
will not every one be also, contrariwise, a preferred or disregarded person? 
God punishes the wicked with his wrath, the law chastises the lawless, you let 
one visit you every moment and show the other the door.

The "{}equality of right"{} is a phantom just because right is nothing more 
and nothing less than admission, \textit{a matter of grace}, which, be it 
said, one may also acquire by his desert; for desert and grace are not 
contradictory, since even grace wishes to be "{}deserved"{} and our gracious 
smile falls only to him who knows how to force it from us.

So people dream of "{}all citizens of the State having to stand side by side, 
with equal rights."{} As citizens of the State they are certainly all equal 
for the State. But it will divide them, and advance them or put them in the 
rear, according to its special ends, if on no other account; and still more 
must it distinguish them from one another as good and bad citizens.

Bruno Bauer disposes of the Jew question from the standpoint that 
"{}privilege"{} is not justified. Because Jew and Christian have each some 
point of advantage over the other, and in having this point of advantage are 
exclusive, therefore before the critic's gaze they crumble into nothingness. 
With them the State lies under the like blame, since it justifies their having 
advantages and stamps it as a "{}privilege."{} or prerogative, but thereby 
derogates from its calling to become a "{}free State."{}

But now every one has something of advantage over another -- \textit{viz}., 
himself or his individuality; in this everybody remains exclusive.

And, again, before a third party every one makes his peculiarity count for as 
much as possible, and (if he wants to win him at all) tries to make it appear 
attractive before him.

Now, is the third party to be insensible to the difference of the one from the 
other? Do they ask that of the free State or of humanity? Then these would 
have to be absolutely without self-interest, and incapable of taking an 
interest in any one whatever. Neither God (who divides his own from the 
wicked) nor the State (which knows how to separate good citizens from bad) was 
thought of as so indifferent. But they are looking for this very third party 
that bestows no more "{}privilege."{} Then it is called perhaps the free 
State, or humanity, or whatever else it may be.

As Christian and Jew are ranked low by Bruno Bauer on account of their 
asserting privileges, it must be that they could and should free themselves 
from their narrow standpoint by self-renunciation or unselfishness. If they 
threw off their "{}egoism,"{} the mutual wrong would cease, and with it 
Christian and Jewish religiousness in general; it would be necessary only that 
neither of them should any longer want to be anything peculiar.

But, if they gave up this exclusiveness, with that the ground on which their 
hostilities were waged would in truth not yet be forsaken. In case of need 
they would indeed find a third thing on which they could unite, a "{}general 
religion,"{} a "{}religion of humanity,"{} etc.; in short, an equalization, 
which need not be better than that which would result if all Jews became 
Christians, by this likewise the "{}privilege"{} of one over the other would 
have an end. The \textit{tension}\footnote{[\textit{Spannung}]} would indeed 
be done away, but in this consisted not the essence of the two, but only their 
neighborhood. As being distinguished from each other they must necessarily be 
mutually resistant,\footnote{[\textit{gespannt}]} and the disparity will 
always remain. Truly it is not a failing in you that you 
stiffen\footnote{[\textit{spannen}]} yourself against me and assert your 
distinctness or peculiarity: you need not give way or renounce yourself.

People conceive the significance of the opposition too \textit{formally} and 
weakly when they want only to "{}dissolve"{} it in order to make room for a 
third thing that shall "{}unite."{} The opposition deserves rather to be 
\textit{sharpened}. As Jew and Christian you are in too slight an opposition, 
and are contending only about religion, as it were about the emperor's beard, 
about a fiddlestick's end. Enemies in religion indeed, \textit{in the rest} 
you still remain good friends, and equal to each other, \textit{e. g.} as men. 
Nevertheless the rest too is unlike in each; and the time when you no longer 
merely \textit{dissemble} your opposition will be only when you entirely 
recognize it, and everybody asserts himself from top to toe as 
\textit{unique}.\footnote{[\textit{Einzig}]} Then the former opposition will 
assuredly be dissolved, but only because a stronger has taken it up into 
itself.

Our weakness consists not in this, that we are in opposition to others, but in 
this, that we are not completely so; that we are not entirely \textit{severed} 
from them, or that we seek a "{}communion,"{} a "{}bond,"{} that in communion 
we have an ideal. One faith, one God, one idea, one hat, for all! If all were 
brought under one hat, certainly no one would any longer need to take off his 
hat before another.

The last and most decided opposition, that of unique against unique, is at 
bottom beyond what is called opposition, but without having sunk back into 
"{}unity"{} and unison. As unique you have nothing in common with the other 
any longer, and therefore nothing divisive or hostile either; you are not 
seeking to be in the right against him before a \textit{third} party, and are 
standing with him neither "{}on the ground of right"{} nor on any other common 
ground. The opposition vanishes in complete -- \textit{severance} or 
singleness.\footnote{[\textit{Einzigkeit}]} This might indeed be regarded as 
the new point in common or a new parity, but here the parity consists 
precisely in the disparity, and is itself nothing but disparity, a par of 
disparity, and that only for him who institutes a "{}comparison."{}

The polemic against privilege forms a characteristic feature of liberalism, 
which fumes against "{}privilege"{} because it itself appeals to "{}right."{} 
Further than to fuming it cannot carry this; for privileges do not fall before 
right falls, as they are only forms of right. But right falls apart into its 
nothingness when it is swallowed up by might, \textit{i.e.} when one 
understands what is meant by "{}Might goes before right."{} All right explains 
itself then as privilege, and privilege itself as power, as -- 
\textit{superior power}.

But must not the mighty combat against superior power show quite another face 
than the modest combat against privilege, which is to be fought out before a 
first judge, "{}Right,"{} according to the judge's mind?

\begin{center}
--------\end{center}


Now, in conclusion, I have still to take back the half-way form of expression 
of which I was willing to make use only so long as I was still rooting among 
the entrails of right, and letting the word at least stand. But, in fact, with 
the concept the word too loses its meaning. What I called "{}my right"{} is no 
longer "{}right"{} at all, because right can be bestowed only by a spirit, be 
it the spirit of nature or that of the species, of mankind, the Spirit of God 
or that of His Holiness or His Highness, etc. What I have without an entitling 
spirit I have without right; I have it solely and alone through my power.

I do not demand any right, therefore I need not recognize any either. What I 
can get by force I get by force, and what I do not get by force I have no 
right to, nor do I give myself airs, or consolation, with my imprescriptible 
right.

With absolute right, right itself passes away; the dominion of the "{}concept 
of right"{} is canceled at the same time. For it is not to be forgotten that 
hitherto concepts, ideas, or principles ruled us, and that among these rulers 
the concept of right, or of justice, played one of the most important parts.

Entitled or unentitled -- that does not concern me, if I am only 
\textit{powerful}, I am of myself \textit{empowered}, and need no other 
empowering or entitling.

Right -- is a wheel in the head, put there by a spook; power -- that am I 
myself, I am the powerful one and owner of power. Right is above me, is 
absolute, and exists in one higher, as whose grace it flows to me: right is a 
gift of grace from the judge; power and might exist only in me the powerful 
and mighty.

\section[2. My Intercourse]{\centering 2. My Intercourse}

In society the human demand at most can be satisfied, while the egoistic must 
always come short. Because it can hardly escape anybody that the present shows 
no such living interest in any question as in the "{}social,"{} one has to 
direct his gaze especially to society. Nay, if the interest felt in it were 
less passionate and dazzled, people would not so much, in looking at society, 
lose sight of the individuals in it, and would recognize that a society cannot 
become new so long as those who form and constitute it remain the old ones. 
If, \textit{e. g.}, there was to arise in the Jewish people a society which 
should spread a new faith over the earth, these apostles could in no case 
remain Pharisees.

As you are, so you present yourself, so you behave toward men: a hypocrite as 
a hypocrite, a Christian as a Christian. Therefore the character of a society 
is determined by the character of its members: they are its creators. So much 
at least one must perceive even if one were not willing to put to the test the 
concept "{}society"{} itself.

Ever far from letting \textit{themselves} come to their full development and 
consequence, men have hitherto not been able to found their societies on 
\textit{themselves;} or rather, they have been able only to found 
"{}societies"{} and to live in societies. The societies were always persons, 
powerful persons, so-called "{}moral persons,"{} \textit{i.e.} ghosts, before 
which the individual had the appropriate wheel in his head, the fear of 
ghosts. As such ghosts they may most suitably be designated by the respective 
names "{}people"{} and "{}peoplet"{}: the people of the patriarchs, the people 
of the Hellenes, etc., at last the -- people of men, Mankind (Anacharsis 
Clootz was enthusiastic for the "{}nation"{} of mankind); then every 
subdivision of this "{}people,"{} which could and must have its special 
societies, the Spanish, French people, etc.; within it again classes, cities, 
in short all kinds of corporations; lastly, tapering to the finest point, the 
little peoplet of the --family. Hence, instead of saying that the person that 
walked as ghost in all societies hitherto has been the people, there might 
also have been named the two extremes -- to wit, either "{}mankind"{} or the 
"{}family,"{} both the most "{}natural-born units."{} We choose the word 
"{}people"{}\footnote{[\textit{Volk}; but the etymological remark following 
applies equally to the English word "{}people."{} See Liddell \& Scott's Greek 
lexicon, under \textit{pimplemi}.]} because its derivation has been brought 
into connection with the Greek \textit{polloi}, the "{}many"{} or "{}the 
masses,"{} but still more because "{}national efforts"{} are at present the 
order of the day, and because even the newest mutineers have not yet shaken 
off this deceptive person, although on the other hand the latter consideration 
must give the preference to the expression "{}mankind,"{} since on all sides 
they are going in for enthusiasm over "{}mankind."{}

The people, then -- mankind or the family -- have hitherto, as it seems, 
played history: no \textit{egoistic} interest was to come up in these 
societies, but solely general ones, national or popular interests, class 
interests, family interests, and "{}general human interests."{} But who has 
brought to their fall the peoples whose decline history relates? Who but the 
egoist, who was seeking \textit{his} satisfaction! If once an egoistic 
interest crept in, the society was "{}corrupted"{} and moved toward its 
dissolution, as Rome, \textit{e. g.} proves with its highly developed system 
of private rights, or Christianity with the incessantly-breaking-in 
"{}rational self-determination,"{} "{}self-consciousness,"{} the "{}autonomy 
of the spirit,"{} etc.

The Christian people has produced two societies whose duration will keep equal 
measure with the permanence of that people: these are the societies 
\textit{State} and \textit{Church}. Can they be called a union of egoists? Do 
we in them pursue an egoistic, personal, own interest, or do we pursue a 
popular (\textit{i.e.} an interest of the Christian \textit{people}), to wit, 
a State, and Church interest? Can I and may I be myself in them? May I think 
and act as I will, may I reveal myself, live myself out, busy myself? Must I 
not leave untouched the majesty of the State, the sanctity of the Church?

Well, I may not do so as I will. But shall I find in any society such an 
unmeasured freedom of maying? Certainly no! Accordingly we might be content? 
Not a bit! It is a different thing whether I rebound from an ego or from a 
people, a generalization. There I am my opponent's opponent, born his equal; 
here I am a despised opponent, bound and under a guardian: there I stand man 
to man; here I am a schoolboy who can accomplish nothing against his comrade 
because the latter has called father and mother to aid and has crept under the 
apron, while I am well scolded as an ill-bred brat, and I must not 
"{}argue"{}: there I fight against a bodily enemy; here against mankind, 
against a generalization, against a "{}majesty,"{} against a spook. But to me 
no majesty, nothing sacred, is a limit; nothing that I know how to overpower. 
Only that which I cannot overpower still limits my might; and I of limited 
might am temporarily a limited I, not limited by the might \textit{outside} 
me, but limited by my \textit{own} still deficient might, by my \textit{own 
impotence}. However, "{}the Guard dies, but does not surrender!"{} Above all, 
only a bodily opponent!

\begin{quotation}

\noindent{} I dare meet every foeman\\
 Whom I can see and measure with my eye,\\
 mettle fires my mettle for the fight -- etc. \end{quotation}

\noindent{}Many privileges have indeed been cancelled with time, but solely 
for the sake of the common weal, of the State and the State's weal, by no 
means for the strengthening of me. Vassalage, \textit{e. g.}, was abrogated 
only that a single liege lord, the lord of the people, the monarchical power, 
might be strengthened: vassalage under the one became yet more rigorous 
thereby. Only in favor of the monarch, be he called "{}prince"{} or "{}law,"{} 
have privileges fallen. In France the citizens are not, indeed, vassals of the 
king, but are instead vassals of the "{}law"{} (the Charter). 
\textit{Subordination} was retained, only the Christian State recognized that 
man cannot serve two masters (the lord of the manor and the prince); therefore 
one obtained all the prerogatives; now he can again \textit{place} one above 
another, he can make "{}men in high place."{}

But of what concern to me is the common weal? The common weal as such is not 
\textit{my weal}, but only the furthest extremity of \textit{self- 
renunciation}. The common weal may cheer aloud while I must 
"{}down"{};\footnote{[\textit{Kuschen}, a word whose only use is in ordering 
dogs to keep quiet.]} the State may shine while I starve. In what lies the 
folly of the political liberals but in their opposing the people to the 
government and talking of people's rights? So there is the people going to be 
of age, etc. As if one who has no mouth could be 
\textit{m\"undig}!\footnote{This is the word for "{}of age"{}; but it is 
derived from \textit{Mund}, "{}mouth,"{} and refers properly to the right of 
speaking through one's own mouth, not by a guardian.]} Only the individual is 
able to be \textit{m\"undig}. Thus the whole question of the liberty of the 
press is turned upside down when it is laid claim to as a "{}right of the 
people."{} It is only a right, or better the might, of the 
\textit{individual}. If a people has liberty of the press, then I, although in 
the midst of this people, have it not; a liberty of the people is not 
\textit{my} liberty, and the liberty of the press as a liberty of the people 
must have at its side a press law directed against \textit{me}.

This must be insisted on all around against the present-day efforts for 
liberty:

Liberty of the \textit{people} is not \textit{my} liberty!

Let us admit these categories, liberty of the people and right of the people: 
\textit{e. g.}, the right of the people that everybody may bear arms. Does one 
not forfeit such a right? One cannot forfeit his own right, but may well 
forfeit a right that belongs not to me but to the people. I may be locked up 
for the sake of the liberty of the people; I may, under sentence, incur the 
loss of the right to bear arms.

Liberalism appears as the last attempt at a creation of the liberty of the 
people, a liberty of the commune, of "{}society,"{} of the general, of 
mankind; the dream of a humanity, a people, a commune, a "{}society,"{} that 
shall be of age.

A people cannot be free otherwise than at the individual's expense; for it is 
not the individual that is the main point in this liberty, but the people. The 
freer the people, the more bound the individual; the Athenian people, 
precisely at its freest time, created ostracism, banished the atheists, 
poisoned the most honest thinker.

How they do praise Socrates for his conscientiousness, which makes him resist 
the advice to get away from the dungeon! He is a fool that he concedes to the 
Athenians a right to condemn him. Therefore it certainly serves him right; why 
then does he remain standing on an equal footing with the Athenians? Why does 
he not break with them? Had he known, and been able to know, what he was, he 
would have conceded to such judges no claim, no right. That \textit{he did not 
escape} was just his weakness, his delusion of still having something in 
common with the Athenians, or the opinion that he was a member, a mere member 
of this people. But he was rather this people itself in person, and could only 
be his own judge. There was no \textit{judge over him}, as he himself had 
really pronounced a public sentence on himself and rated himself worthy of the 
Prytaneum. He should have stuck to that, and, as he had uttered no sentence of 
death against himself, should have despised that of the Athenians too and 
escaped. But he subordinated himself and recognized in the \textit{people} his 
\textit{judge;} he seemed little to himself before the majesty of the people. 
That he subjected himself to \textit{might} (to which alone he could succumb) 
as to a "{}right"{} was treason against himself: it was \textit{virtue}. To 
Christ, who, it is alleged, refrained from using the power over his heavenly 
legions, the same scrupulousness is thereby ascribed by the narrators. Luther 
did very well and wisely to have the safety of his journey to Worms warranted 
to him in black and white, and Socrates should have known that the Athenians 
were his \textit{enemies}, he alone his judge. The self-deception of a 
"{}reign of law,"{} etc., should have given way to the perception that the 
relation was a relation of \textit{might}.

It was with pettifoggery and intrigues that Greek liberty ended. Why? Because 
the ordinary Greeks could still less attain that logical conclusion which not 
even their hero of thought, Socrates, was able to draw. What then is 
pettifoggery but a way of utilizing something established without doing away 
with it? I might add "{}for one's own advantage,"{} but, you see, that lies in 
"{}utilizing."{} Such pettifoggers are the theologians who "{}wrest"{} and 
"{}force"{} God's word; what would they have to wrest if it were not for the 
"{}established"{} Word of God? So those liberals who only shake and wrest the 
"{}established order."{} They are all perverters, like those perverters of the 
law. Socrates recognized law, right; the Greeks constantly retained the 
authority of right and law. If with this recognition they wanted nevertheless 
to assert their advantage, every one his own, then they had to seek it in 
perversion of the law, or intrigue. Alcibiades, an intriguer of genius, 
introduces the period of Athenian "{}decay"{}; the Spartan Lysander and others 
show that intrigue had become universally Greek. Greek \textit{law}, on which 
the Greek \textit{States} rested, had to be perverted and undermined by the 
egoists within these States, and the \textit{States} went down that the 
\textit{individuals} might become free, the Greek people fell because the 
individuals cared less for this people than for themselves. In general, all 
States, constitutions, churches, have sunk by the \textit{secession} of 
individuals; for the individual is the irreconcilable enemy of every 
\textit{generality}, every \textit{tie}, \textit{i.e.} every fetter. Yet 
people fancy to this day that man needs "{}sacred ties"{}: he, the deadly 
enemy of every "{}tie."{} The history of the world shows that no tie has yet 
remained unrent, shows that man tirelessly defends himself against ties of 
every sort; and yet, blinded, people think up new ties again and again, and 
think, \textit{e.g.}, that they have arrived at the right one if one puts upon 
them the tie of a so-called free constitution, a beautiful, constitutional 
tie; decoration ribbons, the ties of confidence between

"{}-- -- --,"{} do seem gradually to have become somewhat infirm, but people 
have made no further progress than from apron-strings to garters and collars.

\textit{Everything sacred is a tie, a fetter}.

Everything sacred is and must be perverted by perverters of the law; therefore 
our present time has multitudes of such perverters in all spheres. They are 
preparing the way for the break-up of law, for lawlessness.

Poor Athenians who are accused of pettifoggery and sophistry! poor Alcibiades, 
of intrigue! Why, that was just your best point, your first step in freedom. 
Your \AE{}eschylus, Herodotus, etc., only wanted to have a free Greek 
\textit{people;} you were the first to surmise something of \textit{your} 
freedom.

A people represses those who tower above \textit{its majesty}, by ostracism 
against too-powerful citizens, by the Inquisition against the heretics of the 
Church, by the -- Inquisition against traitors in the State.

For the people is concerned only with its self-assertion; it demands 
"{}patriotic self-sacrifice"{} from everybody. To it, accordingly, every one 
\textit{in himself} is indifferent, a nothing, and it cannot do, not even 
suffer, what the individual and he alone must do -- to wit, \textit{turn him 
to account}. Every people, every State, is unjust toward the \textit{egoist}.

As long as there still exists even one institution which the individual may 
not dissolve, the ownness and self-appurtenance of Me is still very remote. 
How can I, \textit{e.g.} be free when I must bind myself by oath to a 
constitution, a charter, a law, "{}vow body and soul"{} to my people? How can 
I be my own when my faculties may develop only so far as they "{}do not 
disturb the harmony of society"{} (Weitling)?

The fall of peoples and mankind will invite \textit{me} to my rise.

Listen, even as I am writing this, the bells begin to sound, that they may 
jingle in for tomorrow the festival of the thousand years' existence of our 
dear Germany. Sound, sound its knell! You do sound solemn enough, as if your 
tongue was moved by the presentiment that it is giving convoy to a corpse. The 
German people and German peoples have behind them a history of a thousand 
years: what a long life! O, go to rest, never to rise again -- that all may 
become free whom you so long have held in fetters. -- The \textit{people} is 
dead. -- Up with \textit{me}!

O thou my much-tormented German people -- what was thy torment? It was the 
torment of a thought that cannot create itself a body, the torment of a 
walking spirit that dissolves into nothing at every cock-crow and yet pines 
for deliverance and fulfillment. In me too thou hast lived long, thou dear -- 
thought, thou dear -- spook. Already I almost fancied I had found the word of 
thy deliverance, discovered flesh and bones for the wandering spirit; then I 
hear them sound, the bells that usher thee into eternal rest; then the last 
hope fades out, then the notes of the last love die away, then I depart from 
the desolate house of those who now are dead and enter at the door of the -- 
living one:

\begin{quotation}

\noindent{}For only he who is alive is in the right.\end{quotation}

\noindent{}Farewell, thou dream of so many millions; farewell, thou who hast 
tyrannized over thy children for a thousand years!

Tomorrow they carry thee to the grave; soon thy sisters, the peoples, will 
follow thee. But, when they have all followed, then -- -- mankind is buried, 
and I am my own, I am the laughing heir!

\begin{center}
--------\end{center}


The word \textit{Gesellschaft} (society) has its origin in the word 
\textit{Sal} (hall). If one hall encloses many persons, then the hall causes 
these persons to be in society. They \textit{are} in society, and at most 
constitute a parlor-society by talking in the traditional forms of parlor 
speech. When it comes to real \textit{intercourse}, this is to be regarded as 
independent of society: it may occur or be lacking, without altering the 
nature of what is named society. Those who are in the hall are a society even 
as mute persons, or when they put each other off solely with empty phrases of 
courtesy. Intercourse is mutuality, it is the action, the \textit{commercium}, 
of individuals; society is only community of the hall, and even the statues of 
a museum-hall are in society, they are "{}grouped."{} People are accustomed to 
say "{}they \textit{haben inne}\footnote{["{}Occupy"{}; literally, "{}have 
within"{}.]} this hall in common,"{} but the case is rather that the hall has 
us \textit{inne} or in it. So far the natural signification of the word 
society. In this it comes out that society is not generated by me and you, but 
by a third factor which makes associates out of us two, and that it is just 
this third factor that is the creative one, that which creates society.

Just so a prison society or prison companionship (those who 
enjoy\footnote{[The word \textit{Genosse}, "{}companion,"{} signifies 
originally a companion in enjoyment.]} the same prison). Here we already hit 
upon a third factor fuller of significance than was that merely local one, the 
hall. Prison no longer means a space only, but a space with express reference 
to its inhabitants: for it is a prison only through being destined for 
prisoners, without whom it would be a mere building. What gives a common stamp 
to those who are gathered in it? Evidently the prison, since it is only by 
means of the prison that they are prisoners. What, then, determines the 
\textit{manner} of life of the prison society? The prison! What determines 
their intercourse? The prison too, perhaps? Certainly they can enter upon 
intercourse only as prisoners, \textit{i.e.} only so far as the prison laws 
allow it; but that \textit{they themselves} hold intercourse, I with you, this 
the prison cannot bring to pass; on the contrary, it must have an eye to 
guarding against such egoistic, purely personal intercourse (and only as such 
is it really intercourse between me and you). That we \textit{jointly} execute 
a job, run a machine, effectuate anything in general -- for this a prison will 
indeed provide; but that I forget that I am a prisoner, and engage in 
intercourse with you who likewise disregard it, brings danger to the prison, 
and not only cannot be caused by it, but must not even be permitted. For this 
reason the saintly and moral-minded French chamber decides to introduce 
solitary confinement, and other saints will do the like in order to cut off 
"{}demoralizing intercourse."{} Imprisonment is the established and -- sacred 
condition, to injure which no attempt must be made. The slightest push of that 
kind is punishable, as is every uprising against a sacred thing by which man 
is to be charmed and chained.

Like the hall, the prison does form a society, a companionship, a communion 
(\textit{e. g.} communion of labor), but no \textit{intercourse}, no 
reciprocity, no \textit{union}. On the contrary, every union in the prison 
bears within it the dangerous seed of a "{}plot,"{} which under favorable 
circumstances might spring up and bear fruit.

Yet one does not usually enter the prison voluntarily, and seldom remains in 
it voluntarily either, but cherishes the egoistic desire for liberty. Here, 
therefore, it sooner becomes manifest that personal intercourse is in hostile 
relations to the prison society and tends to the dissolution of this very 
society, this joint incarceration.

Let us therefore look about for such communions as, it seems, we remain in 
gladly and voluntarily, without wanting to endanger them by our egoistic 
impulses.

As a communion of the required sort the \textit{family} offers itself in the 
first place. Parents, husbands and wife, children, brothers and sisters, 
represent a whole or form a family, for the further widening of which the 
collateral relatives also may be made to serve if taken into account. The 
family is a true communion only when the law of the family, 
piety\footnote{[This word in German does not mean religion, but, as in Latin, 
faithfulness to family ties -- as we speak of "{}filial piety."{} But the word 
elsewhere translated "{}pious"{} [\textit{fromm}] means "{}religious,"{} as 
usually in English.]} or family love, is observed by its members. A son to 
whom parents, brothers, and sisters have become indifferent \textit{has been} 
a son; for, as the sonship no longer shows itself efficacious, it has no 
greater significance than the long-past connection of mother and child by the 
navel-string. That one has once lived in this bodily juncture cannot as a fact 
be undone; and so far one remains irrevocably this mother's son and the 
brother of the rest of her children; but it would come to a lasting connection 
only by lasting piety, this spirit of the family. Individuals are members of a 
family in the full sense only when they make the \textit{persistence} of the 
family their task; only as \textit{conservative} do they keep aloof from 
doubting their basis, the family. To every member of the family one thing must 
be fixed and sacred -- \textit{viz}., the family itself, or, more 
expressively, piety. That the family is to \textit{persist} remains to its 
member, so long as he keeps himself free from that egoism which is hostile to 
the family, an unassailable truth. In a word: -- If the family is sacred, then 
nobody who belongs to it may secede from it; else he becomes a "{}criminal"{} 
against the family: he may never pursue an interest hostile to the family, 
\textit{e. g.} form a misalliance. He who does this has "{}dishonored the 
family,"{} "{}put it to shame,"{} etc.

Now, if in an individual the egoistic impulse has not force enough, he 
complies and makes a marriage which suits the claims of the family, takes a 
rank which harmonizes with its position, etc.; in short, he "{}does honor to 
the family."{}

If, on the contrary, the egoistic blood flows fierily enough in his veins, he 
prefers to become a "{}criminal"{} against the family and to throw off its 
laws.

Which of the two lies nearer my heart, the good of the family or my good? In 
innumerable cases both go peacefully together; the advantage of the family is 
at the same time mine, and \textit{vice versa}. Then it is hard to decide 
whether I am thinking \textit{selfishly} or \textit{for the common benefit}, 
and perhaps I complacently flatter myself with my unselfishness. But there 
comes the day when a necessity of choice makes me tremble, when I have it in 
mind to dishonor my family tree, to affront parents, brothers, and kindred. 
What then? Now it will appear how I am disposed at the bottom of my heart; now 
it will be revealed whether piety ever stood above egoism for me, now the 
selfish one can no longer skulk behind the semblance of unselfishness. A wish 
rises in my soul, and, growing from hour to hour, becomes a passion. To whom 
does it occur at first blush that the slightest thought which may result 
adversely to the spirit of the family (piety) bears within it a transgression 
against this? Nay, who at once, in the first moment, becomes completely 
conscious of the matter? It happens so with Juliet in "{}Romeo and Juliet."{} 
The unruly passion can at last no longer be tamed, and undermines the building 
of piety. You will say, indeed, it is from self-will that the family casts out 
of its bosom those wilful ones that grant more of a hearing to their passion 
than to piety; the good Protestants used the same excuse with much success 
against the Catholics, and believed in it themselves. But it is just a 
subterfuge to roll the fault off oneself, nothing more. The Catholics had 
regard for the common bond of the church, and thrust those heretics from them 
only because these did not have so much regard for the bond of the church as 
to sacrifice their convictions to it; the former, therefore, held the bond 
fast, because the bond, the Catholic (\textit{i.e.} common and united) church, 
was sacred to them; the latter, on the contrary, disregarded the bond. Just so 
those who lack piety. They are not thrust out, but thrust themselves out, 
prizing their passion, their wilfulness, higher than the bond of the family.

But now sometimes a wish glimmers in a less passionate and wilful heart than 
Juliet's. The pliable girl brings herself as a \textit{sacrifice} to the peace 
of the family. One might say that here too selfishness prevailed, for the 
decision came from the feeling that the pliable girl felt herself more 
satisfied by the unity of the family than by the fulfillment of her wish. That 
might be; but what if there remained a sure sign that egoism had been 
sacrificed to piety? What if, even after the wish that had been directed 
against the peace of the family was sacrificed, it remained at least as a 
recollection of a "{}sacrifice"{} brought to a sacred tie? What if the pliable 
girl were conscious of having left her self-will unsatisfied and humbly 
subjected herself to a higher power? Subjected and sacrificed, because the 
superstition of piety exercised its dominion over her!

There egoism won, here piety wins and the egoistic heart bleeds; there egoism 
was strong, here it was -- weak. But the weak, as we have long known, are the 
-- unselfish. For them, for these its weak members, the family cares, because 
they \textit{belong} to the family, do not belong to themselves and care for 
themselves. This weakness Hegel, \textit{e. g.} praises when he wants to have 
match- making left to the choice of the parents.

As a sacred communion to which, among the rest, the individual owes obedience, 
the family has the judicial function too vested in it; such a "{}family 
court"{} is described \textit{e. g.} in the \textit{Cabanis} \textit{}of 
Wilibald Alexis. There the father, in the name of the "{}family council,"{} 
puts the intractable son among the soldiers and thrusts him out of the family, 
in order to cleanse the smirched family again by means of this act of 
punishment. -- The most consistent development of family responsibility is 
contained in Chinese law, according to which the whole family has to expiate 
the individual's fault.

Today, however, the arm of family power seldom reaches far enough to take 
seriously in hand the punishment of apostates (in most cases the State 
protects even against disinheritance). The criminal against the family 
(family-criminal) flees into the domain of the State and is free, as the 
State-criminal who gets away to America is no longer reached by the 
punishments of his State. He who has shamed his family, the graceless son, is 
protected against the family's punishment because the State, this protecting 
lord, takes away from family punishment its "{}sacredness"{} and profanes it, 
decreeing that it is only --"{}revenge"{}: it restrains punishment, this 
sacred family right, because before its, the State's, "{}sacredness"{} the 
subordinate sacredness of the family always pales and loses its sanctity as 
soon as it comes in conflict with this higher sacredness. Without the 
conflict, the State lets pass the lesser sacredness of the family; but in the 
opposite case it even commands crime against the family, charging, \textit{e. 
g.}, the son to refuse obedience to his parents as soon as they want to 
beguile him to a crime against the State.

Well, the egoist has broken the ties of the family and found in the State a 
lord to shelter him against the grievously affronted spirit of the family. But 
where has he run now? Straight into a new \textit{society}, in which his 
egoism is awaited by the same snares and nets that it has just escaped. For 
the State is likewise a society, not a union; it is the broadened 
\textit{family} ("{}Father of the Country -- Mother of the Country -- children 
of the country"{}).

\begin{center}
--------\end{center}


What is called a State is a tissue and plexus of dependence and adherence; it 
is a \textit{belonging together}, a holding together, in which those who are 
placed together fit themselves to each other, or, in short, mutually depend on 
each other: it is the \textit{order} of this \textit{dependence}. Suppose the 
king, whose authority lends authority to all down to the beadle, should 
vanish: still all in whom the will for order was awake would keep order erect 
against the disorders of bestiality. If disorder were victorious, the State 
would be at an end.

But is this thought of love, to fit ourselves to each other, to adhere to each 
other and depend on each other, really capable of winning us? According to 
this the State should be \textit{love} realized, the being for each other and 
living for each other of all. Is not self-will being lost while we attend to 
the will for order? Will people not be satisfied when order is cared for by 
authority, \textit{i.e.} when authority sees to it that no one "{}gets in the 
way of"{} another; when, then, the \textit{herd} is judiciously distributed or 
ordered? Why, then everything is in "{}the best order,"{} and it is this best 
order that is called -- State!

Our societies and States \textit{are} without our \textit{making} them, are 
united without our uniting, are predestined and established, or have an 
independent standing\footnote{[It should be remembered that the words 
"{}establish"{} and "{}State"{} are both derived from the root "{}stand."{}]} 
of their own, are the indissolubly established against us egoists. The fight 
of the world today is, as it is said, directed against the "{}established."{} 
Yet people are wont to misunderstand this as if it were only that what is now 
established was to be exchanged for another, a better, established system. But 
war might rather be declared against establishment itself, the \textit{State}, 
not a particular State, not any such thing as the mere condition of the State 
at the time; it is not another State (\textit{e. g.} a "{}people's State"{}) 
that men aim at, but their \textit{union}, uniting, this ever-fluid uniting of 
everything standing. -- A State exists even without my co-operation: I am born 
in it, brought up in it, under obligations to it, and must "{}do it 
homage."{}\footnote{[\textit{huldigen}]} It takes me up into its 
"{}favor,"{}\footnote{[\textit{Huld}]} and I live by its "{}grace."{} Thus the 
independent establishment of the State founds my lack of independence; its 
condition as a "{}natural growth,"{} its organism, demands that my nature do 
not grow freely, but be cut to fit it. That \textit{it} may be able to unfold 
in natural growth, it applies to me the shears of "{}civilization"{}; it gives 
me an education and culture adapted to it, not to me, and teaches me 
\textit{e. g.} to respect the laws, to refrain from injury to State property 
(\textit{i.e.} private property), to reverence divine and earthly highness, 
etc.; in short, it teaches me to be -- \textit{unpunishable}, 
"{}sacrificing"{} my ownness to "{}sacredness"{} (everything possible is 
sacred; \textit{e. g.} property, others' life, etc.). In this consists the 
sort of civilization and culture that the State is able to give me: it brings 
me up to be a "{}serviceable instrument,"{} a "{}serviceable member of 
society."{}

This every State must do, the people's State as well as the absolute or 
constitutional one. It must do so as long as we rest in the error that it is 
an \textit{I}, as which it then applies to itself the name of a "{}moral, 
mystical, or political person."{} I, who really am I, must pull off this 
lion-skin of the I from the stalking thistle-eater. What manifold robbery have 
I not put up with in the history of the world! There I let sun, moon, and 
stars, cats and crocodiles, receive the honor of ranking as I; there Jehovah, 
Allah, and Our Father came and were invested with the I; there families, 
tribes, peoples, and at last actually mankind, came and were honored as I's; 
there the Church, the State, came with the pretension to be I -- and I gazed 
calmly on all. What wonder if then there was always a real I too that joined 
the company and affirmed in my face that it was not my \textit{you} but my 
real \textit{I}. Why, \textit{the} Son of Man \textit{par excellence} had done 
the like; why should not \textit{a} son of man do it too? So I saw my I always 
above me and outside me, and could never really come to myself.

I never believed in myself; I never believed in my present, I saw myself only 
in the future. The boy believes he will be a proper I, a proper fellow, only 
when he has become a man; the man thinks, only in the other world will he be 
something proper. And, to enter more closely upon reality at once, even the 
best are today still persuading each other that one must have received into 
himself the State, his people, mankind, and what not, in order to be a real I, 
a "{}free burgher,"{} a "{}citizen,"{} a "{}free or true man"{}; they too see 
the truth and reality of me in the reception of an alien I and devotion to it. 
And what sort of an I? An I that is neither an I nor a you, a \textit{fancied} 
I, a spook.

While in the Middle Ages the church could well brook many States living united 
in it, the States learned after the Reformation, especially after the Thirty 
Years' War, to tolerate many churches (confessions) gathering under one crown. 
But all States are religious and, as the case may be, "{}Christian States,"{} 
and make it their task to force the intractable, the "{}egoists,"{} under the 
bond of the unnatural, \textit{e. g.}, Christianize them. All arrangements of 
the Christian State have the object of \textit{Christianizing the people}. 
Thus the court has the object of forcing people to justice, the school that of 
forcing them to mental culture -- in short, the object of protecting those who 
act Christianly against those who act un-Christianly, of bringing Christian 
action to \textit{dominion}, of making it \textit{powerful}. Among these means 
of force the State counted the \textit{Church} too, it demanded a -- 
particular religion from everybody. Dupin said lately against the clergy, 
"{}Instruction and education belong to the State."{}

Certainly everything that regards the principle of morality is a State affair. 
Hence it is that the Chinese State meddles so much in family concerns, and one 
is nothing there if one is not first of all a good child to his parents. 
Family concerns are altogether State concerns with us too, only that our State 
-- puts confidence in the families without painful oversight; it holds the 
family bound by the marriage tie, and this tie cannot be broken without it.

But that the State makes me responsible for my principles, and demands certain 
ones from me, might make me ask, what concern has it with the "{}wheel in my 
head"{} (principle)? Very much, for the State is the -- \textit{ruling 
principle}. It is supposed that in divorce matters, in marriage law in 
general, the question is of the proportion of rights between Church and 
States. Rather, the question is of whether anything sacred is to rule over 
man, be it called faith or ethical law (morality). The State behaves as the 
same ruler that the Church was. The latter rests on godliness, the former on 
morality.

People talk of the tolerance, the leaving opposite tendencies free, etc., by 
which civilized States are distinguished. Certainly some are strong enough to 
look with complacency on even the most unrestrained meetings, while others 
charge their catchpolls to go hunting for tobacco-pipes. Yet for one State as 
for another the play of individuals among themselves, their buzzing to and 
fro, their daily life, is an \textit{incident} which it must be content to 
leave to themselves because it can do nothing with this. Many, indeed, still 
strain out gnats and swallow camels, while others are shrewder. Individuals 
are "{}freer"{} in the latter, because less pestered. But \textit{I} am free 
in \textit{no} State. The lauded tolerance of States is simply a tolerating of 
the "{}harmless,"{} the "{}not dangerous"{}; it is only elevation above 
pettymindedness, only a more estimable, grander, prouder -- despotism. A 
certain State seemed for a while to mean to be pretty well elevated above 
\textit{literary} combats, which might be carried on with all heat; England is 
elevated above \textit{popular turmoil} and -- tobacco-smoking. But woe to the 
literature that deals blows at the State itself, woe to the mobs that 
"{}endanger"{} the State. In that certain State they dream of a "{}free 
science,"{} in England of a "{}free popular life."{}

The State does let individuals \textit{play} as freely as possible, only they 
must not be in \textit{earnest}, must not forget \textit{it}. Man must not 
carry on intercourse with man \textit{unconcernedly}, not without "{}superior 
oversight and mediation."{} I must not execute all that I am able to, but only 
so much as the State allows; I must not turn to account \textit{my} thoughts, 
nor \textit{my} work, nor, in general, anything of mine.

The State always has the sole purpose to limit, tame, subordinate, the 
individual -- to make him subject to some \textit{generality} or other; it 
lasts only so long as the individual is not all in all, and it is only the 
clearly-marked \textit{restriction of me}, my limitation, my slavery. Never 
does a State aim to bring in the free activity of individuals, but always that 
which is bound to the \textit{purpose of the State}. Through the State nothing 
\textit{in common} comes to pass either, as little as one can call a piece of 
cloth the common work of all the individual parts of a machine; it is rather 
the work of the whole machine as a unit, \textit{machine work}. In the same 
style everything is done by the \textit{State machine} too; for it moves the 
clockwork of the individual minds, none of which follow their own impulse. The 
State seeks to hinder every free activity by its censorship, its supervision, 
its police, and holds this hindering to be its duty, because it is in truth a 
duty of self-preservation. The State wants to make something out of man, 
therefore there live in it only \textit{made} men; every one who wants to be 
his own self is its opponent and is nothing. "{}He is nothing"{} means as much 
as, the State does not make use of him, grants him no position, no office, no 
trade, etc.

Edgar Bauer,\footnote{What was said in the concluding remarks after Humane 
Liberalism holds good of the following -- to wit, that it was likewise written 
immediately after the appearance of the book cited.} in the \textit{Liberale 
Bestrebungen} (vol. II, p.50), is still dreaming of a "{}government which, 
proceeding out of the people, can never stand in opposition to it."{} He does 
indeed (p.69) himself take back the word "{}government"{}: "{}In the republic 
no government at all obtains, but only an executive authority. An authority 
which proceeds purely and alone out of the people; which has not an 
independent power, independent principles, independent officers, over against 
the people; but which has its foundation, the fountain of its power and of its 
principles, in the sole, supreme authority of the State, in the people. The 
concept government, therefore, is not at all suitable in the people's 
State."{} But the thing remains the same. That which has "{}proceeded, been 
founded, sprung from the fountain"{} becomes something "{}independent"{} and, 
like a child delivered from the womb, enters upon opposition at once. The 
government, if it were nothing independent and opposing, would be nothing at 
all.

"{}In the free State there is no government,"{} etc. (p.94). This surely means 
that the people, when it is the \textit{sovereign}, does not let itself be 
conducted by a superior authority. Is it perchance different in absolute 
monarchy? Is there \textit{there} for the \textit{sovereign}, perchance, a 
government standing over him? \textit{Over} the sovereign, be he called prince 
or people, there never stands a government: that is understood of itself. But 
over \textit{me} there will stand a government in every "{}State,"{} in the 
absolute as well as in the republican or "{}free."{} I am as badly off in one 
as in the other.

The republic is nothing whatever but -- absolute monarchy; for it makes no 
difference whether the monarch is called prince or people, both being a 
"{}majesty."{} Constitutionalism itself proves that nobody is able and willing 
to be only an instrument. The ministers domineer over their master the prince, 
the deputies over their master the people. Here, then, the \textit{parties} at 
least are already free -- \textit{videlicet}, the office-holders' party 
(so-called people's party). The prince must conform to the will of the 
ministers, the people dance to the pipe of the chambers. Constitutionalism is 
further than the republic, because it is the \textit{State} in incipient 
\textit{dissolution}.

Edgar Bauer denies (p.56) that the people is a "{}personality"{} in the 
constitutional State; \textit{per contra}, then, in the republic? Well, in the 
constitutional State the people is -- a \textit{party}, and a party is surely 
a "{}personality"{} if one is once resolved to talk of a "{}political"{} 
(p.76) moral person anyhow. The fact is that a moral person, be it called 
people's party or people or even "{}the Lord,"{} is in no wise a person, but a 
spook.

Further, Edgar Bauer goes on (p.69): "{}guardianship is the characteristic of 
a government."{} Truly, still more that of a people and "{}people's State"{}; 
it is the characteristic of all \textit{dominion}. A people's State, which 
"{}unites in itself all completeness of power,"{} the "{}absolute master,"{} 
cannot let me become powerful. And what a chimera, to be no longer willing to 
call the "{}people's officials"{} "{}servants, instruments,"{} because they 
"{}execute the free, rational law-will of the people!"{} (p.73). He thinks 
(p.74): "{}Only by all official circles subordinating themselves to the 
government's views can unity be brought into the State"{}; but his "{}people's 
State"{} is to have "{}unity"{} too; how will a lack of subordination be 
allowed there? subordination to the -- people's will.

"{}In the constitutional State it is the regent and his \textit{disposition} 
that the whole structure of government rests on in the end."{} (p. 130.) How 
would that be otherwise in the "{}people's State"{}? Shall \textit{I} not 
there be governed by the people's \textit{disposition} too, and does it make a 
difference \textit{for me} whether I see myself kept in dependence by the 
prince's disposition or by the people's disposition, so-called "{}public 
opinion"{}? If dependence means as much as "{}religious relation,"{} as Edgar 
Bauer rightly alleges, then in the people's State the people remains 
\textit{for me} the superior power, the "{}majesty"{} (for God and prince have 
their proper essence in "{}majesty"{}) to which I stand in religious 
relations. -- Like the sovereign regent, the sovereign people too would be 
reached by no \textit{law}. Edgar Bauer's whole attempt comes to a 
\textit{change of masters}. Instead of wanting to make the \textit{people} 
free, he should have had his mind on the sole realizable freedom, his own.

In the constitutional State \textit{absolutism} itself has at last come in 
conflict with itself, as it has been shattered into a duality; the government 
wants to be absolute, and the people wants to be absolute. These two absolutes 
will wear out against each other.

Edgar Bauer inveighs against the determination of the regent \textit{by 
birth}, by \textit{chance}. But, when "{}the people"{} have become "{}the sole 
power in the State"{} (p. 132), have \textit{we} not then in it a master from 
\textit{chance?} Why, what is the people? The people has always been only the 
\textit{body} of the government: it is many under one hat (a prince's hat) or 
many under one constitution. And the constitution is the -- prince. Princes 
and peoples will persist so long as both do not \textit{col}lapse, \textit{i. 
e.}, fall \textit{together}. If under one constitution there are many 
"{}peoples"{} -- as in the ancient Persian monarchy and today --then these 
"{}peoples"{} rank only as "{}provinces."{} For me the people is in any case 
an --accidental power, a force of nature, an enemy that I must overcome.

What is one to think of under the name of an "{}organized"{} people (p. 132)? 
A people "{}that no longer has a government,"{} that governs itself. In which, 
therefore, no ego stands out prominently; a people organized by ostracism. The 
banishment of egos, ostracism, makes the people autocrat.

If you speak of the people, you must speak of the prince; for the people, if 
it is to be a subject\footnote{[In the philosophical sense [a thinking and 
acting being] not in the political sense.]} and make history, must, like 
everything that acts, have a \textit{head}, its "{}supreme head."{} Weitling 
sets this forth in [\textit{Die Europ\"aische}] Triarchie, and Proudhon 
declares, \textit{"{}une soci\'et\'e, pour ainsi dire ac\'ephale, ne peut 
vivre}."{}\footnote{[\textit{"{}Cr\'eation de l'Ordre},"{} p.485.]}

The \textit{vox populi} is now always held up to us, and "{}public opinion"{} 
is to rule our princes. Certainly the \textit{vox populi} is at the same time 
\textit{vox dei;} but is either of any use, and is not the \textit{vox 
principis} also \textit{vox dei}?

At this point the "{}Nationals"{} may be brought to mind. To demand of the 
thirty-eight States of Germany that they shall act as \textit{one nation} can 
only be put alongside the senseless desire that thirty-eight swarms of bees, 
led by thirty-eight queen-bees, shall unite themselves into one swarm. 
\textit{Bees} they all remain; but it is not the bees as bees that belong 
together and can join themselves together, it is only that the 
\textit{subject} bees are connected with the \textit{ruling} queens. Bees and 
peoples are destitute of will, and the \textit{instinct} of their queens leads 
them.

If one were to point the bees to their beehood, in which at any rate they are 
all equal to each other, one would be doing the same thing that they are now 
doing so stormily in pointing the Germans to their Germanhood. Why, Germanhood 
is just like beehood in this very thing, that it bears in itself the necessity 
of cleavages and separations, yet without pushing on to the last separation, 
where, with the complete carrying through of the process of separating, its 
end appears: I mean, to the separation of man from man. Germanhood does indeed 
divide itself into different peoples and tribes, \textit{i.e.} beehives; but 
the individual who has the quality of being a German is still as powerless as 
the isolated bee. And yet only individuals can enter into union with each 
other, and all alliances and leagues of peoples are and remain mechanical 
compoundings, because those who come together, at least so far as the 
"{}peoples"{} are regarded as the ones that have come together, are 
\textit{destitute of will}. Only with the last separation does separation 
itself end and change to unification.

Now the Nationals are exerting themselves to set up the abstract, lifeless 
unity of beehood; but the self-owned are going to fight for the unity willed 
by their own will, for union. This is the token of all reactionary wishes, 
that they want to set up something \textit{general}, abstract, an empty, 
lifeless \textit{concept}, in distinction from which the self-owned aspire to 
relieve the robust, lively \textit{particular} from the trashy burden of 
generalities. The reactionaries would be glad to smite a \textit{people, a 
nation}, forth from the earth; the self-owned have before their eyes only 
themselves. In essentials the two efforts that are just now the order of the 
day - to wit, the restoration of provincial rights and of the old tribal 
divisions (Franks, Bavarians, Lusatia, etc.), and the restoration of the 
entire nationality -- coincide in one. But the Germans will come into unison, 
\textit{i.e.} unite \textit{themselves}, only when they knock over their 
beehood as well as all the beehives; in other words, when they are more than 
-- Germans: only then can they form a "{}German Union."{} They must not want 
to turn back into their nationality, into the womb, in order to be born again, 
but let every one turn in to \textit{himself}. How ridiculously sentimental 
when one German grasps another's hand and presses it with sacred awe because 
"{}he too is a German!"{} With that he is something great! But this will 
certainly still be thought touching as long as people are enthusiastic for 
"{}brotherliness,"{} \textit{i.e.} as long as they have a \textit{"{}family 
disposition"{}}. From the superstition of "{}piety,"{} from 
"{}brotherliness"{} or "{}childlikeness"{} or however else the soft-hearted 
piety-phrases run -- from the \textit{family spirit} -- the Nationals, who 
want to have a great \textit{family of Germans}, cannot liberate themselves.

Aside from this, the so-called Nationals would only have to understand 
themselves rightly in order to lift themselves out of their juncture with the 
good-natured Teutomaniacs. For the uniting for material ends and interests, 
which they demand of the Germans, comes to nothing else than a voluntary 
union. Carri\`ere, inspired, cries out,\footnote{[\textit{"{}K\"olner Dom},"{} 
p. 4.]} "{}Railroads are to the more penetrating eye the way to a \textit{life 
of the people} \textit{e. g.} has not yet anywhere appeared in such 
significance."{} Quite right, it will be a life of the people that has nowhere 
appeared, because it is not a -- life of the people. -- So Carri\`ere then 
combats himself (p. 10): "{}Pure humanity or manhood cannot be better 
represented than by a people fulfilling its mission."{} Why, by this 
nationality only is represented. "{}Washed-out generality is lower than the 
form complete in itself, which is itself a whole, and lives as a living member 
of the truly general, the organized."{} Why, the people is this very 
"{}washed-out generality,"{} and it is only a man that is the "{}form complete 
in itself."{}

The impersonality of what they call "{}people, nation,"{} is clear also from 
this: that a people which wants to bring its I into view to the best of its 
power puts at its head the ruler \textit{without will}. It finds itself in the 
alternative either to be subjected to a prince who realizes only 
\textit{himself, his individual pleasure --} then it does not recognize in the 
"{}absolute master"{} its own will, the so-called will of the people -- or to 
seat on the throne a prince who gives effect to no will of his own -- then it 
has a prince \textit{without will}, whose place some ingenious clockwork would 
perhaps fill just as well. -- Therefore insight need go only a step farther; 
then it becomes clear of itself that the I of the people is an impersonal, 
"{}spiritual"{} power, the -- law. The people's I, therefore, is a -- spook, 
not an I. I am I only by this, that I make myself; \textit{i.e.} that it is 
not another who makes me, but I must be my own work. But how is it with this I 
of the people? \textit{Chance} plays it into the people's hand, chance gives 
it this or that born lord, accidents procure it the chosen one; he is not its 
(the \textit{"{}sovereign"{}} people's) product, as I am \textit{my} product. 
Conceive of one wanting to talk you into believing that you were not your I, 
but Tom or Jack was your I! But so it is with the people, and rightly. For the 
people has an I as little as the eleven planets counted together have an I, 
though they revolve around a common \textit{center}.

Bailly's utterance is representative of the slave-disposition that folks 
manifest before the sovereign people, as before the prince. "{}I have,"{} says 
he, "{}no longer any extra reason when the general reason has pronounced 
itself. My first law was the nation's will; as soon as it had assembled I knew 
nothing beyond its sovereign will."{} He would have no "{}extra reason,"{} and 
yet this extra reason alone accomplishes everything. Just so Mirabeau inveighs 
in the words, "{}No power on earth has the \textit{right} to say to the 
nation's representatives, It is my will!"{}

As with the Greeks, there is now a wish to make man a \textit{zoon politicon}, 
a citizen of the State or political man. So he ranked for a long time as a 
"{}citizen of heaven."{} But the Greek fell into ignominy along with his 
\textit{State}, the citizen of heaven likewise falls with heaven; we, on the 
other hand, are not willing to go down along with the \textit{people}, the 
nation and nationality, not willing to be merely \textit{political} men or 
politicians. Since the Revolution they have striven to "{}make the people 
happy,"{} and in making the people happy, great, etc., they make us unhappy: 
the people's good hap is -- my mishap.

What empty talk the political liberals utter with emphatic decorum is well 
seen again in Nauwerck's "{}On Taking Part in the State"{}. There complaint is 
made of those who are indifferent and do not take part, who are not in the 
full sense citizens, and the author speaks as if one could not be man at all 
if one did not take a lively part in State affairs, \textit{i.e.} if one were 
not a politician. In this he is right; for, if the State ranks as the warder 
of everything "{}human,"{} we can have nothing human without taking part in 
it. But what does this make out against the egoist? Nothing at all, because 
the egoist is to himself the warder of the human, and has nothing to say to 
the State except "{}Get out of my sunshine."{} Only when the State comes in 
contact with his ownness does the egoist take an active interest in it. If the 
condition of the State does not bear hard on the closet-philosopher, is he to 
occupy himself with it because it is his "{}most sacred duty?"{} So long as 
the State does according to his wish, what need has he to look up from his 
studies? Let those who from an interest of their own want to have conditions 
otherwise busy themselves with them. Not now, nor evermore, will "{}sacred 
duty"{} bring folks to reflect about the State -- as little as they become 
disciples of science, artists, etc., from "{}sacred duty."{} Egoism alone can 
impel them to it, and will as soon as things have become much worse. If you 
showed folks that their egoism demanded that they busy themselves with State 
affairs, you would not have to call on them long; if, on the other hand, you 
appeal to their love of fatherland etc., you will long preach to deaf hearts 
in behalf of this "{}service of love."{} Certainly, in your sense the egoists 
will not participate in State affairs at all.

Nauwerck utters a genuine liberal phrase on p. 16: "{}Man completely fulfills 
his calling only in feeling and knowing himself as a member of humanity, and 
being active as such. The individual cannot realize the idea of 
\textit{manhood} if he does not stay himself upon all humanity, if he does not 
draw his powers from it like Antaeus."{}

In the same place it is said: "{}Man's relation to the \textit{res publica} is 
degraded to a purely private matter by the theological view; is, accordingly, 
made away with by denial."{} As if the political view did otherwise with 
religion! There religion is a "{}private matter."{}

If, instead of "{}sacred duty,"{} "{}man's destiny,"{} the "{}calling to full 
manhood,"{} and similar commandments, it were held up to people that their 
\textit{self-interest} was infringed on when they let everything in the State 
go as it goes, then, without declamations, they would be addressed as one will 
have to address them at the decisive moment if he wants to attain his end. 
Instead of this, the theology-hating author says, "{}If there has ever been a 
time when the \textit{State} laid claim to all that are hers, such a time is 
ours. -- The thinking man sees in participation in the theory and practice of 
the State a \textit{duty}, one of the most sacred duties that rest upon him"{} 
-- and then takes under closer consideration the "{}unconditional necessity 
that everybody participate in the State."{}

He in whose head or heart or both the \textit{State} is seated, he who is 
possessed by the State, or the \textit{believer in the State}, is a 
politician, and remains such to all eternity.

"{}The State is the most necessary means for the complete development of 
mankind."{} It assuredly has been so as long as we wanted to develop mankind; 
but, if we want to develop ourselves, it can be to us only a means of 
hindrance.

Can State and people still be reformed and bettered now? As little as the 
nobility, the clergy, the church, etc.: they can be abrogated, annihilated, 
done away with, not reformed. Can I change a piece of nonsense into sense by 
reforming it, or must I drop it outright?

Henceforth what is to be done is no longer about the \textit{State} (the form 
of the State, etc.), but about me. With this all questions about the prince's 
power, the constitution, etc., sink into their true abyss and their true 
nothingness. I, this nothing, shall put forth my \textit{creations} from 
myself.

\begin{center}
--------\end{center}


To the chapter of society belongs also "{}the party,"{} whose praise has of 
late been sung.

In the State the \textit{party} is current. "{}Party, party, who should not 
join one!"{} But the individual is 
\textit{unique},\footnote{[\textit{einzig}]} not a member of the party. He 
unites freely, and separates freely again. The party is nothing but a State in 
the State, and in this smaller bee- State "{}peace"{} is also to rule just as 
in the greater. The very people who cry loudest that there must be an 
\textit{opposition} in the State inveigh against every discord in the party. A 
proof that they too want only a --State. All parties are shattered not against 
the State, but against the ego.\footnote{[\textit{am Einzigen}]}

One hears nothing oftener now than the admonition to remain true to his party; 
party men despise nothing so much as a mugwump. One must run with his party 
through thick and thin, and unconditionally approve and represent its chief 
principles. It does not indeed go quite so badly here as with closed 
societies, because these bind their members to fixed laws or statutes 
(\textit{e. g.} the orders, the Society of Jesus, etc.). But yet the party 
ceases to be a union at the same moment at which it makes certain principles 
\textit{binding} and wants to have them assured against attacks; but this 
moment is the very birth-act of the party. As party it is already a 
\textit{born society}, a dead union, an idea that has become fixed. As party 
of absolutism it cannot will that its members should doubt the irrefragable 
truth of this principle; they could cherish this doubt only if they were 
egoistic enough to want still to be something outside their party, 
\textit{i.e.} non-partisans. Non-partisans they cannot be as party-men, but 
only as egoists. If you are a Protestant and belong to that party, you must 
only justify Protestantism, at most "{}purge"{} it, not reject it; if you are 
a Christian and belong among men to the Christian party, you cannot be beyond 
this as a member of this party, but only when your egoism, \textit{i.e.} 
non-partisanship, impels you to it. What exertions the Christians, down to 
Hegel and the Communists, have put forth to make their party strong! They 
stuck to it that Christianity must contain the eternal truth, and that one 
needs only to get at it, make sure of it, and justify it.

In short, the party cannot bear non-partisanship, and it is in this that 
egoism appears. What matters the party to me? I shall find enough anyhow who 
\textit{unite} with me without swearing allegiance to my flag.

He who passes over from one party to another is at once abused as a 
"{}turncoat."{} Certainly \textit{morality} demands that one stand by his 
party, and to become apostate from it is to spot oneself with the stain of 
"{}faithlessness"{}; but ownness knows no commandment of "{}faithlessness"{}; 
adhesion, etc., ownness permits everything, even apostasy, defection. 
Unconsciously even the moral themselves let themselves be led by this 
principle when they have to judge one who passes over to \textit{their} party 
-- nay, they are likely to be making proselytes; they should only at the same 
time acquire a consciousness of the fact that one must commit \textit{immoral} 
actions in order to commit his own -- \textit{i.e.} here, that one must break 
faith, yes, even his oath, in order to determine himself instead of being 
determined by moral considerations. In the eyes of people of strict moral 
judgment an apostate always shimmers in equivocal colors, and will not easily 
obtain their confidence; for there sticks to him the taint of 
"{}faithlessness,"{} \textit{i.e.} of an immorality. In the lower man this 
view is found almost generally; advanced thinkers fall here too, as always, 
into an uncertainty and bewilderment, and the contradiction necessarily 
founded in the principle of morality does not, on account of the confusion of 
their concepts, come clearly to their consciousness. They do not venture to 
call the apostate downright immoral, because they themselves entice to 
apostasy, to defection from one religion to another, etc.; still, they cannot 
give up the standpoint of morality either. And yet here the occasion was to be 
seized to step outside of morality.

Are the Own or Unique\footnote{[\textit{Einzigen}]} perchance a party? How 
could they be \textit{own} if they were \textit{e. g.} \textit{belonged} to a 
party?

Or is one to hold with no party? In the very act of joining them and entering 
their circle one forms a union with them that lasts as long as party and I 
pursue one and the same goal. But today I still share the party's tendency, as 
by tomorrow I can do so no longer and I become "{}untrue"{} to it. The party 
has nothing binding (obligatory) for me, and I do not have respect for it; if 
it no longer pleases me, I become its foe.

In every party that cares for itself and its persistence, the members are 
unfree (or better, unown) in that degree, they lack egoism in that degree, in 
which they serve this desire of the party. The independence of the party 
conditions the lack of independence in the party- members.

A party, of whatever kind it may be, can never do without a \textit{confession 
of faith}. For those who belong to the party must \textit{believe} in its 
principle, it must not be brought in doubt or put in question by them, it must 
be the certain, indubitable thing for the party-member. That is: One must 
belong to a party body and soul, else one is not truly a party-man, but more 
or less -- an egoist. Harbor a doubt of Christianity, and you are already no 
longer a true Christian, you have lifted yourself to the "{}effrontery"{} of 
putting a question beyond it and haling Christianity before your egoistic 
judgment-seat. You have -- \textit{sinned} against Christianity, this party 
cause (for it is surely not \textit{e. g.} a cause for the Jews, another 
party.) But well for you if you do not let yourself be affrighted: your 
effrontery helps you to ownness.

So then an egoist could never embrace a party or take up with a party? Oh, 
yes, only he cannot let himself be embraced and taken up by the party. For him 
the party remains all the time nothing but a gathering: he is one of the 
party, he takes part.

\begin{center}
--------\end{center}


The best State will clearly be that which has the most loyal citizens, and the 
more the devoted mind for \textit{legality} is lost, so much the more will the 
State, this system of morality, this moral life itself, be diminished in force 
and quality. With the "{}good citizens"{} the good State too perishes and 
dissolves into anarchy and lawlessness. "{}Respect for the law!"{} By this 
cement the total of the State is held together. "{}The law is \textit{sacred}, 
and he who affronts it a \textit{criminal"{}}. Without crime no State: the 
moral world -- and this the State is -- is crammed full of scamps, cheats, 
liars, thieves, etc. Since the State is the "{}lordship of law,"{} its 
hierarchy, it follows that the egoist, in all cases where \textit{his} 
advantage runs against the State's, can satisfy himself only by crime.

The State cannot give up the claim that its \textit{laws} and ordinances are 
\textit{sacred}.\footnote{[\textit{heilig}]} At this the individual ranks as 
the \textit{unholy}\footnote{[\textit{unheilig}]} (barbarian, natural man, 
"{}egoist"{}) over against the State, exactly as he was once regarded by the 
Church; before the individual the State takes on the nimbus of a 
saint.\footnote{[\textit{Heiliger}]} Thus it issues a law against dueling. Two 
men who are both at one in this, that they are willing to stake their life for 
a cause (no matter what), are not to be allowed this, because the State will 
not have it: it imposes a penalty on it. Where is the liberty of 
self-determination then? It is at once quite another situation if, as 
\textit{e. g.} in North America, society determines to let the duelists bear 
certain evil \textit{consequences} of their act, \textit{e. g.} withdrawal of 
the credit hitherto enjoyed. To refuse credit is everybody's affair, and, if a 
society wants to withdraw it for this or that reason, the man who is hit 
cannot therefore complain of encroachment on his liberty: the society is 
simply availing itself of its own liberty. That is no penalty for sin, no 
penalty for a \textit{crime}. The duel is no crime there, but only an act 
against which the society adopts counter-measures, resolves on a 
\textit{defense}. The State, on the contrary, stamps the duel as a crime, 
\textit{i.e.} as an injury to its sacred law: it makes it a \textit{criminal 
case}. The society leaves it to the individual's decision whether he will draw 
upon himself evil consequences and inconveniences by his mode of action, and 
hereby recognizes his free decision; the State behaves in exactly the reverse 
way, denying all right to the individual's decision and, instead, ascribing 
the sole right to its own decision, the law of the State, so that he who 
transgresses the State's commandment is looked upon as if he were acting 
against God's commandment -- a view which likewise was once maintained by the 
Church. Here God is the Holy in and of himself, and the commandments of the 
Church, as of the State, are the commandments of this Holy One, which he 
transmits to the world through his anointed and Lords-by-the-Grace-of-God. If 
the Church had \textit{deadly sins}, the State has \textit{capital crimes;} if 
the one had \textit{heretics}, the other has \textit{traitors;} the one 
\textit{ecclesiastical penalties}, the other \textit{criminal penalties;} the 
one \textit{inquisitorial} processes, the other \textit{fiscal;} in short, 
there sins, here crimes, there inquisition and here -- inquisition. Will the 
sanctity of the State not fall like the Church's? The awe of its laws, the 
reverence for its highness, the humility of its "{}subjects,"{} will this 
remain? Will the "{}saint's"{} face not be stripped of its adornment?

What a folly, to ask of the State's authority that it should enter into an 
honourable fight with the individual, and, as they express themselves in the 
matter of freedom of the press, share sun and wind equally! If the State, this 
thought, is to be a \textit{de facto} power, it simply must be a superior 
power against the individual. The State is "{}sacred"{} and must not expose 
itself to the "{}impudent attacks"{} of individuals. If the State is 
\textit{sacred}, there must be censorship. The political liberals admit the 
former and dispute the inference. But in any case they concede repressive 
measures to it, for -- they stick to this, that State is \textit{more} than 
the individual and exercises a justified revenge, called punishment.

\textit{Punishment} has a meaning only when it is to afford expiation for the 
injuring of a\textit{sacred} thing. If something is sacred to any one, he 
certainly deserves punishment when he acts as its enemy. A man who lets a 
man's life continue in existence \textit{because} to him it is sacred and he 
has a \textit{dread} of touching it is simply a -- \textit{religious} man.

Weitling lays crime at the door of "{}social disorder,"{} and lives in the 
expectation that under Communistic arrangements crimes will become impossible, 
because the temptations to them, \textit{e. g.} money, fall away. As, however, 
his organized society is also exalted into a sacred and inviolable one, he 
miscalculates in that good-hearted opinion. \textit{e. g.} with their mouth 
professed allegiance to the Communistic society, but worked underhand for its 
ruin, would not be lacking. Besides, Weitling has to keep on with "{}curative 
means against the natural remainder of human diseases and weaknesses,"{} and 
"{}curative means"{} always announce to begin with that individuals will be 
looked upon as "{}called"{} to a particular "{}salvation"{} and hence treated 
according to the requirements of this "{}human calling."{} \textit{Curative 
means} or \textit{healing} is only the reverse side of \textit{punishment}, 
the \textit{theory of cure} runs parallel with the \textit{theory of 
punishment;} if the latter sees in an action a sin against right, the former 
takes it for a sin of the man \textit{against himself}, as a decadence from 
his health. But the correct thing is that I regard it either as an action that 
\textit{suits me} or as one that \textit{does not suit me}, as hostile or 
friendly to \textit{me}, \textit{i.e.} that I treat it as my 
\textit{property}, which I cherish or demolish. "{}Crime"{} or "{}disease"{} 
are not either of them an \textit{egoistic} view of the matter, \textit{i.e.} 
a judgment \textit{starting from me}, but starting from \textit{another --} to 
wit, whether it injures \textit{right}, general right, or the \textit{health} 
partly of the individual (the sick one), partly of the generality 
(\textit{society}). "{}Crime"{} is treated inexorably, "{}disease"{} with 
"{}loving gentleness, compassion,"{} etc.

Punishment follows crime. If crime falls because the sacred vanishes, 
punishment must not less be drawn into its fall; for it too has significance 
only over against something sacred. Ecclesiastical punishments have been 
abolished. Why? Because how one behaves toward the "{}holy God"{} is his own 
affair. But, as this one punishment, \textit{ecclesiastical punishment}, has 
fallen, so all \textit{punishments} must fall. As sin against the so-called 
God is a man's own affair, so is that against every kind of the so-called 
sacred. According to our theories of penal law, with whose "{}improvement in 
conformity to the times"{} people are tormenting themselves in vain, they want 
to \textit{punish} men for this or that "{}inhumanity"{}; and therein they 
make the silliness of these theories especially plain by their consistency, 
hanging the little thieves and letting the big ones run. For injury to 
property they have the house of correction, and for "{}violence to thought,"{} 
suppression of "{}natural rights of man,"{} only --representations and 
petitions.

The criminal code has continued existence only through the sacred, and 
perishes of itself if punishment is given up. Now they want to create 
everywhere a new penal law, without indulging in a misgiving about punishment 
itself. But it is exactly punishment that must make room for satisfaction, 
which, again, cannot aim at satisfying right or justice, but at procuring 
\textit{us} a satisfactory outcome. If one does to us what we \textit{will not 
put up with}, we break his power and bring our own to bear: we satisfy 
\textit{ourselves} on him, and do not fall into the folly of wanting to 
satisfy right (the spook). It is not the \textit{sacred} that is to defend 
itself against man, but man against man; as \textit{God} too, you know, no 
longer defends himself against man, God to whom formerly (and in part, indeed, 
even now) all the "{}servants of God"{} offered their hands to punish the 
blasphemer, as they still at this very day lend their hands to the sacred. 
This devotion to the sacred brings it to pass also that, without lively 
participation of one's own, one only delivers misdoers into the hands of the 
police and courts: a non-participating making over to the authorities, "{}who, 
of course, will best administer sacred matters."{} The people is quite crazy 
for hounding the police on against everything that seems to it to be immoral, 
often only unseemly, and this popular rage for the moral protects the police 
institution more than the government could in any way protect it.

In crime the egoist has hitherto asserted himself and mocked at the sacred; 
the break with the sacred, or rather of the sacred, may become general. A 
revolution never returns, but a mighty, reckless, shameless, conscienceless. 
proud --\textit{crime}, does it not rumble in distant thunders, and do you not 
see how the sky grows presciently silent and gloomy?

\begin{center}
--------\end{center}


He who refuses to spend his powers for such limited societies as family, 
party, nation, is still always longing for a worthier society, and thinks he 
has found the true object of love, perhaps, in "{}human society"{} or 
"{}mankind,"{} to sacrifice himself to which constitutes his honor; from now 
on he "{}lives for and serves \textit{mankind}."{}

\textit{People} is the name of the body, \textit{State} of the spirit, of that 
\textit{ruling person} that has hitherto suppressed me. Some have wanted to 
transfigure peoples and States by broadening them out to "{}mankind"{} and 
"{}general reason"{}; but servitude would only become still more intense with 
this widening, and philanthropists and humanitarians are as absolute masters 
as politicians and diplomats.

Modern critics inveigh against religion because it sets God, the divine, 
moral, etc., \textit{outside} of man, or makes them something objective, in 
opposition to which the critics rather transfer these very subjects 
\textit{into} man. But those critics none the less fall into the proper error 
of religion, to give man a "{}destiny,"{} in that they too want to have him 
divine, human, and the like: morality, freedom and humanity, etc., are his 
essence. And, like religion politics too wanted to \textit{"{}educate"{}} man, 
to bring him to the realization of his "{}essence,"{} his "{}destiny,"{} to 
\textit{make} something out of him -- to wit, a "{}true man,"{} the one in the 
form of the "{}true believer,"{} the other in that of the "{}true citizen or 
subject."{} In fact, it comes to the same whether one calls the destiny the 
divine or human.

Under religion and politics man finds himself at the standpoint of 
\textit{should: he should} become this and that, should be so and so. With 
this postulate, this commandment, every one steps not only in front of another 
but also in front of himself. Those critics say: You should be a whole, free 
man. Thus they too stand in the temptation to proclaim a new 
\textit{religion}, to set up a new absolute, an ideal -- to wit, freedom. Men 
\textit{should} be free. Then there might even arise \textit{missionaries} of 
freedom, as Christianity, in the conviction that all were properly destined to 
become Christians, sent out missionaries of the faith. Freedom would then (as 
have hitherto faith as Church, morality as State) constitute itself as a new 
\textit{community} and carry on a like "{}propaganda"{} therefrom. Certainly 
no objection can be raised against a getting together; but so much the more 
must one oppose every renewal of the old \textit{care} for us, of culture 
directed toward an end -- in short, the principle of \textit{making something} 
out of us, no matter whether Christians, subjects, or freemen and men.

One may well say with Feuerbach and others that religion has displaced the 
human from man, and has transferred it so into another world that, 
unattainable, it went on with its own existence there as something personal in 
itself, as a "{}God"{}: but the error of religion is by no means exhausted 
with this. One might very well let fall the personality of the displaced 
human, might transform God into the divine, and still remain religious. For 
the religious consists in discontent with the \textit{present} men, in the 
setting up of a "{}perfection"{} to be striven for, in "{}man wrestling for 
his completion."{}\footnote{B. Bauer, \textit{"{}Lit. Ztg}."{} 8,22.} ("{}Ye 
therefore \textit{should} be perfect as your father in heaven is perfect."{} 
Matt. 5, 48): it consists in the fixation of an ideal, an absolute. Perfection 
is the "{}supreme good,"{} the \textit{finis bonorum;} every one's ideal is 
the perfect man, the true, the free man, etc.

The efforts of modern times aim to set up the ideal of the "{}free man."{} If 
one could find it, there would be a new -- religion, because a new ideal; 
there would be a new longing, a new torment, a new devotion, a new deity, a 
new contrition.

With the ideal of "{}absolute liberty,"{} the same turmoil is made as with 
everything absolute, and according to Hess, \textit{e. g.}, it is said to 
"{}be realizable in absolute human society."{}\footnote{\textit{"{}E. u. Z. 
B.},"{} p. 89ff.} Nay, this realization is immediately afterward styled a 
"{}vocation"{}; just so he then defines liberty as "{}morality"{}: the kingdom 
of "{}justice"{} (equality) and "{}morality"{} (\textit{i.e.} liberty) is to 
begin, etc.

Ridiculous is he who, while fellows of his tribe, family, nation, rank high, 
is -- nothing but "{}puffed up"{} over the merit of his fellows; but blinded 
too is he who wants only to be "{}man."{} Neither of them puts his worth in 
\textit{exclusiveness}, but in \textit{connectedness}, or in the "{}tie"{} 
that conjoins him with others, in the ties of blood, of nationality, of 
humanity.

Through the "{}Nationals"{} of today the conflict has again been stirred up 
between those who think themselves to have merely human blood and human ties 
of blood, and the others who brag of their special blood and the special ties 
of blood.

If we disregard the fact that pride may mean conceit, and take it for 
consciousness alone, there is found to be a vast difference between pride in 
"{}belonging to"{} a nation and therefore being its property, and that in 
calling a nationality one's property. Nationality is my quality, but the 
nation my owner and mistress. If you have bodily strength, you can apply it at 
a suitable place and have a self-consciousness or pride of it; if, on the 
contrary, your strong body has you, then it pricks you everywhere, and at the 
most unsuitable place, to show its strength: you can give nobody your hand 
without squeezing his.

The perception that one is more than a member of the family, more than a 
fellow of the tribe, more than an individual of the people, has finally led to 
saying, one is more than all this because one is man, or, the man is more than 
the Jew, German, etc. "{}Therefore be every one wholly and solely -- man."{} 
Could one not rather say: Because we are more than what has been stated, 
therefore we will be this, as well as that "{}more"{} also? Man and Germans, 
then, man and Guelph, etc.? The Nationals are in the right; one cannot deny 
his nationality: and the humanitarians are in the right; one must not remain 
in the narrowness of the national. In 
\textit{uniqueness}\footnote{[\textit{Einzigkeit}]} the contradiction is 
solved; the national is my quality. But I am not swallowed up in my quality -- 
as the human too is my quality, but I give to man his existence first through 
my uniqueness.

History seeks for Man: but he is I, you, we. Sought as a mysterious 
\textit{essence}, as the divine, first as \textit{God}, then as Man (humanity, 
humaneness, and mankind), he is found as the individual, the finite, the 
unique one.

I am owner of humanity, am humanity, and do nothing for the good of another 
humanity. Fool, you who are a unique humanity, that you make a merit of 
wanting to live for another than you are.

The hitherto-considered relation of me to the \textit{world of men} offers 
such a wealth of phenomena that it will have to be taken up again and again on 
other occasions, but here, where it was only to have its chief outlines made 
clear to the eye, it must be broken off to make place for an apprehension of 
two other sides toward which it radiates. For, as I find myself in relation 
not merely to men so far as they present in themselves the concept "{}man"{} 
or are children of men (children of \textit{Man}, as children of God are 
spoken of), but also to that which they have of man and call their own, and as 
therefore I relate myself not only to that which they \textit{are} through 
man, but also to their human \textit{possessions:} so, besides the world of 
men, the world of the senses and of ideas will have to be included in our 
survey, and somewhat said of what men call their own of sensuous goods, and of 
spiritual as well.

According as one had developed and clearly grasped the concept of man, he gave 
it to us to respect as this or that \textit{person of respect}, and from the 
broadest understanding of this concept there proceeded at last the command 
"{}to respect Man in every one."{} But if I respect Man, my respect must 
likewise extend to the human, or what is Man's.

Men have somewhat of their \textit{own}, and \textit{I} am to recognize this 
own and hold it sacred. Their own consists partly in outward, partly in inward 
\textit{possessions}. The former are things, the latter spiritualities, 
thoughts, convictions, noble feelings, etc. But I am always to respect only 
\textit{rightful} or \textit{human} possessions: the wrongful and unhuman I 
need not spare, for only \textit{Man's} own is men's real own. An inward 
possession of this sort is, \textit{e. g.}, religion; because 
\textit{religion} is free, \textit{i. e.} is Man's, \textit{I} must not strike 
at it. Just so \textit{honor} is an inward possession; it is free and must not 
be struck at my me. (Action for insult, caricatures, etc.) Religion and honor 
are "{}spiritual property."{} In tangible property the person stands foremost: 
my person is my first property. Hence freedom of the person; but only the 
\textit{rightful} or human person is free, the other is locked up. Your life 
is your property; but it is sacred for men only if it is not that of an 
inhuman monster.

What a man as such cannot defend of bodily goods, we may take from him: this 
is the meaning of competition, of freedom of occupation. What he cannot defend 
of spiritual goods falls a prey to us likewise: so far goes the liberty of 
discussion, of science, of criticism.

But \textit{consecrated} goods are inviolable. Consecrated and guarantied by 
whom? Proximately by the State, society, but properly by man or the 
"{}concept,"{} the "{}concept of the thing"{}; for the concept of consecrated 
goods is this, that they are truly human, or rather that the holder possesses 
them as man and not as un-man.\footnote{[See note on p. 184.]}

On the spiritual side man's faith is such goods, his honor, his moral feeling 
-- yes, his feeling of decency, modesty, etc. Actions (speeches, writings) 
that touch honor are punishable; attacks on "{}the foundations of all 
religion"{}; attacks on political faith; in short, attacks on everything that 
a man "{}rightly"{} has.

How far critical liberalism would extend the sanctity of goods -- on this 
point it has not yet made any pronouncement, and doubtless fancies itself to 
be ill-disposed toward all sanctity; but, as it combats egoism, it must set 
limits to it, and must not let the un-man pounce on the human. To its 
theoretical contempt for the "{}masses"{} there must correspond a practical 
snub if it should get into power.

What extension the concept "{}man"{} receives, and what comes to the 
individual man through it -- what, therefore, man and the human are -- on this 
point the various grades of liberalism differ, and the political, the social, 
the humane man are each always claiming more than the other for "{}man."{} He 
who has best grasped this concept knows best what is "{}man's."{} The State 
still grasps this concept in political restriction, society in social; 
mankind, so it is said, is the first to comprehend it entirely, or "{}the 
history of mankind develops it."{} But, if "{}man is discovered,"{} then we 
know also what pertains to man as his own, man's property, the human.

But let the individual man lay claim to ever so many rights because Man or the 
concept man "{}entitles"{} him to them, because his being man does it: what do 
I care for his right and his claim? If he has his right only from Man and does 
not have it from \textit{me}, then for \textit{me} he has no right. His life, 
\textit{e. g.}, counts to \textit{me} only for what it is \textit{worth} to 
\textit{me}. I respect neither a so-called right of property (or his claim to 
tangible goods) nor yet his right to the "{}sanctuary of his inner nature"{} 
(or his right to have the spiritual goods and divinities, his gods, remain 
un-aggrieved). His goods, the sensuous as well as the spiritual, are 
\textit{mine}, and I dispose of them as proprietor, in the measure of my -- 
might.

In the \textit{property question} lies a broader meaning than the limited 
statement of the question allows to be brought out. Referred solely to what 
men call our possessions, it is capable of no solution; the decision is to be 
found in him "{}from whom we have everything."{} Property depends on the 
\textit{owner}.

The Revolution directed its weapons against everything which came "{}from the 
grace of God,"{} \textit{e. g.}, against divine right, in whose place the 
human was confirmed. To that which is granted by the grace of God, there is 
opposed that which is derived "{}from the essence of man."{}

Now, as men's relation to each other, in opposition to the religious dogma 
which commands a "{}Love one another for God's sake,"{} had to receive its 
human position by a "{}Love each other for man's sake,"{} so the revolutionary 
teaching could not do otherwise than, first, as to what concerns the relation 
of men to the things of this world, settle it that the world, which hitherto 
was arranged according to God's ordinance, henceforth belongs to "{}Man."{}

The world belongs to "{}Man,"{} and is to be respected by me as his property.

Property is what is mine!

Property in the civic sense means \textit{sacred} property, such that I must 
\textit{respect} your property. "{}Respect for property!"{} Hence the 
politicians would like to have every one possess his little bit of property, 
and they have in part brought about an incredible parcellation by this effort. 
Each must have his bone on which he may find something to bite.

The position of affairs is different in the egoistic sense. I do not step 
shyly back from your property, but look upon it always as my property, in 
which I need to "{}respect"{} nothing. Pray do the like with what you call my 
property!

With this view we shall most easily come to an understanding with each other.

The political liberals are anxious that, if possible, all servitudes be 
dissolved, and every one be free lord on his ground, even if this ground has 
only so much area as can have its requirements adequately filled by the manure 
of one person. (The farmer in the story married even in his old age "{}that he 
might profit by his wife's dung."{}) Be it ever so little, if one only has 
somewhat of his own -- to wit, a \textit{respected} property! The more such 
owners, such cotters,\footnote{[The words "{}cot"{} and "{}dung"{} are alike 
in German.]} the more "{}free people and good patriots"{} has the State.

Political liberalism, like everything religious, counts on \textit{respect}, 
humaneness, the virtues of love. Therefore does it live in incessant vexation. 
For in practice people respect nothing, and every day the small possessions 
are bought up again by greater proprietors, and the "{}free people"{} change 
into day- laborers.

If, on the contrary, the "{}small proprietors"{} had reflected that the great 
property was also theirs, they would not have respectfully shut themselves out 
from it, and would not have been shut out.

Property as the civic liberals understand it deserves the attacks of the 
Communists and Proudhon: it is untenable, because the civic proprietor is in 
truth nothing but a property-less man, one who is everywhere \textit{shut 
out}. Instead of owning the world, as he might, he does not own even the 
paltry point on which he turns around.

Proudhon wants not the \textit{propri\'etaire} but the \textit{possesseur} or 
\textit{usufruitier}.\footnote{\textit{e. g.}, \textit{"{}Qu'est-ce que la 
Propri\'et\'e?}"{} p. 83} What does that mean? He wants no one to own the 
land; but the benefit of it -- even though one were allowed only the hundredth 
part of this benefit, this fruit -- is at any rate one's property, which he 
can dispose of at will. He who has only the benefit of a field is assuredly 
not the proprietor of it; still less he who, as Proudhon would have it, must 
give up so much of this benefit as is not required for his wants; but he is 
the proprietor of the share that is left him. Proudhon, therefore, denies only 
such and such property, not \textit{property} itself. If we want no longer to 
leave the land to the landed proprietors, but to appropriate it to 
\textit{ourselves}, we unite ourselves to this end, form a union, a 
\textit{soci\'et\'e}, that makes \textit{itself} proprietor; if we have good 
luck in this, then those persons cease to be landed proprietors. And, as from 
the land, so we can drive them out of many another property yet, in order to 
make it \textit{our} property, the property of the -- \textit{conquerors}. The 
conquerors form a society which one may imagine so great that it by degrees 
embraces all humanity; but so-called humanity too is as such only a thought 
(spook); the individuals are its reality. And these individuals as a 
collective (mass will treat land and earth not less arbitrarily than an 
isolated individual or so-called \textit{propri\'etaire}. Even so, therefore, 
\textit{property} remains standing, and that as exclusive too, in that 
\textit{humanity}, this great society, excludes the \textit{individual} from 
its property (perhaps only leases to him, gives his as a fief, a piece of it) 
as it besides excludes everything that is not humanity, \textit{e. g.} does 
not allow animals to have property. -- So too it will remain, and will grow to 
be. That in which \textit{all} want to have a \textit{share} will be withdrawn 
from that individual who wants to have it for himself alone: it is made a 
\textit{common estate}. As a \textit{common estate} every one has his 
\textit{share} in it, and this share is his \textit{property}. Why, so in our 
old relations a house which belongs to five heirs is their common estate; but 
the fifth part of the revenue is, each one's property. Proudhon might spare 
his prolix pathos if he said: "{}There are some things that belong only to a 
few, and to which we others will from now on lay claim or -- siege. Let us 
take them, because one comes to property by taking, and the property of which 
for the present we are still deprived came to the proprietors likewise only by 
taking. It can be utilized better if it is in the hands of us \textit{all} 
than if the few control it. Let us therefore associate ourselves for the 
purpose of this robbery (\textit{vol})."{} -- Instead of this, he tries to get 
us to believe that society is the original possessor and the sole proprietor, 
of imprescriptible right; against it the so-called proprietors have become 
thieves (\textit{La propri\'et\'e c'est le vol}); if it now deprives of his 
property the present proprietor, it robs him of nothing, as it is only 
availing itself of its imprescriptible right. -- So far one comes with the 
spook of society as a \textit{moral person}. On the contrary, what man can 
obtain belongs to him: the world belongs to \textit{me}. Do you say anything 
else by your opposite proposition? "{}The world belongs to \textit{all"{}}? 
All are I and again I, etc. But you make out of the "{}all"{} a spook, and 
make it sacred, so that then the "{}all"{} become the individual's fearful 
\textit{master}. Then the ghost of "{}right"{} places itself on their side.

Proudhon, like the Communists, fights against \textit{egoism}. Therefore they 
are continuations and consistent carryings-out of the Christian principle, the 
principle of love, of sacrifice for something general, something alien. They 
complete in property, \textit{e. g.,} only what has long been extant as a 
matter of fact -- to wit, the propertylessness of the individual. When the 
laws says, \textit{Ad reges potestas omnium pertinet, ad singulos proprietas; 
omnia rex imperio possidet, singuli dominio}, this means: The king is 
proprietor, for he alone can control and dispose of "{}everything,"{} he has 
\textit{potestas} and \textit{imperium} over it. The Communists make this 
clearer, transferring that \textit{imperium} to the "{}society of all."{} 
Therefore: Because enemies of egoism, they are on that account -- Christians, 
or, more generally speaking, religious men, believers in ghosts, dependents, 
servants of some generality (God, society, etc.). In this too Proudhon is like 
the Christians, that he ascribes to God that which he denies to men. He names 
him (\textit{e. g.} page 90) the Propri\'etaire of the earth. Herewith he 
proves that he cannot think away the \textit{proprietor as such;} he comes to 
a proprietor at last, but removes him to the other world.

Neither God nor Man ("{}human society"{}) is proprietor, but the individual.

\begin{center}
--------\end{center}


Proudhon (Weitling too) thinks he is telling the worst about property when he 
calls it theft (\textit{vol}). Passing quite over the embarrassing question, 
what well-founded objection could be made against theft, we only ask: Is the 
concept "{}theft"{} at all possible unless one allows validity to the concept 
"{}property"{}? How can one steal if property is not already extant? What 
belongs to no one cannot be \textit{stolen;} the water that one draws out of 
the sea he does \textit{not steal}. Accordingly property is not theft, but a 
theft becomes possible only through property. Weitling has to come to this 
too, as he does regard everything as the \textit{property of all:} if 
something is "{}the property of all,"{} then indeed the individual who 
appropriates it to himself steals.

Private property lives by grace of the \textit{law}. Only in the law has it 
its warrant -- for possession is not yet property, it becomes "{}mine"{} only 
by assent of the law; it is not a fact, not \textit{un fait} as Proudhon 
thinks, but a fiction, a thought. This is legal property, legitimate property, 
guarantied property. It is mine not through \textit{me} but through the -- 
\textit{law}.

Nevertheless, property is the expression for \textit{unlimited dominion} over 
somewhat (thing, beast, man) which "{}I can judge and dispose of as seems good 
to me."{} According to Roman law, indeed, \textit{jus utendi et abutendi re 
sua, quatenus juris ratio patitur}, an \textit{exclusive} and 
\textit{unlimited right;} but property is conditioned by might. What I have in 
my power, that is my own. So long as I assert myself as holder, I am the 
proprietor of the thing; if it gets away from me again, no matter by what 
power, \textit{e. g.} through my recognition of a title of others to the thing 
-- then the property is extinct. Thus property and possession coincide. It is 
not a right lying outside my might that legitimizes me, but solely my might: 
if I no longer have this, the thing vanishes away from me. When the Romans no 
longer had any might against the Germans, the world-empire of Rome 
\textit{belonged} to the latter, and it would sound ridiculous to insist that 
the Romans had nevertheless remained properly the proprietors. Whoever knows 
how to take and to defend the thing, to him it belongs till it is again taken 
from him, as liberty belongs to him who \textit{takes} it.--

Only might decides about property, and, as the State (no matter whether State 
or well-to-do citizens or of ragamuffins or of men in the absolute) is the 
sole mighty one, it alone is proprietor; I, the 
unique,\footnote{[\textit{Einzige}]} have nothing, and am only enfeoffed, am 
vassal and as such, servitor. Under the dominion of the State there is no 
property of \textit{mine}.

I want to raise the value of myself, the value of ownness, and should I 
cheapen property? No, as I was not respected hitherto because people, mankind, 
and a thousand other generalities were put higher, so property too has to this 
day not yet been recognized in its full value. Property too was only the 
property of a ghost, \textit{e. g.} the people's property; my whole existence 
"{}belonged to the fatherland"{}; \textit{I} belonged to the fatherland, the 
people, the State, and therefore also everything that I called \textit{my 
own}. It is demanded of States that they make away with pauperism. It seems to 
me this is asking that the State should cut off its own head and lay it at its 
feet; for so long as the State is the ego the individual ego must remain a 
poor devil, a non-ego. The State has an interest only in being itself rich; 
whether Michael is rich and Peter poor is alike to it; Peter might also be 
rich and Michael poor. It looks on indifferently as one grows poor and the 
other rich, unruffled by this alternation. As \textit{individuals} they are 
really equal before its face; in this it is just: before it both of them are 
-- nothing, as we "{}are altogether sinners before God"{}; on the other hand, 
it has a very great interest in this, that those individuals who make it their 
ego should have a part in \textit{its} wealth; it makes them partakers in 
\textit{its property}. Through property, with which it rewards the 
individuals, it tames them; but this remains \textit{its} property, and every 
one has the usufruct of it only so long as he bears in himself the ego of the 
State, or is a "{}loyal member of society"{}; in the opposite case the 
property is confiscated, or made to melt away by vexatious lawsuits. The 
property, then, is and remains \textit{State property}, not property of the 
ego. That the State does not arbitrarily deprive the individual of what he has 
from the State means simply that the State does not rob itself. He who is 
State-ego, \textit{i.e.} a good citizen or subject, holds his fief undisturbed 
as \textit{such an ego}, not as being an ego of his own. According to the 
code, property is what I call mine "{}by virtue of God and law."{} But it is 
mine by virtue of God and law only so long as -- the State has nothing against 
it.

In expropriations, disarmaments, etc. (as, when the exchequer confiscates 
inheritances if the heirs do not put in an appearance early enough) how 
plainly the else-veiled principle that only the \textit{people}, "{}the 
State,"{} is proprietor, while the individual is feoffee, strikes the eye!

The State, I mean to say, cannot intend that anybody should \textit{for his 
own sake} have property or actually be rich, nay, even well-to-do; it can 
acknowledge nothing, yield nothing, grant nothing to me as me. The State 
cannot check pauperism, because the poverty of possession is a poverty of me. 
He who \textit{is} nothing but what chance or another -- to wit, the State -- 
makes out of him also \textit{has} quite rightly nothing but what another 
gives him. And this other will \textit{give} him only what he 
\textit{deserves}, \textit{i.e.} what he is worth by \textit{service}. It is 
not he that realizes a value from himself; the State realizes a value from 
him.

National economy busies itself much with this subject. It lies far out beyond 
the "{}national,"{} however, and goes beyond the concepts and horizon of the 
State, which knows only State property and can distribute nothing else. For 
this reason it binds the possessions of property to \textit{conditions --} as 
it binds everything to them, \textit{e. g.} marriage, allowing validity only 
to the marriage sanctioned by it, and wresting this out of my power. But 
property is my property only when I hold it \textit{unconditionally} : only I, 
an \textit{unconditional} ego, have property, enter a relation of love, carry 
on free trade.

The State has no anxiety about me and mine, but about itself and its: I count 
for something to it only as its \textit{child}, as "{}a son of the country"{}; 
as \textit{ego} I am nothing at all for it. For the State's understanding, 
what befalls me as ego is something accidental, my wealth as well as my 
impoverishment. But, if I with all that is mine am an accident in the State's 
eyes, this proves that it cannot comprehend \textit{me: I} go beyond its 
concepts, or, its understanding is too limited to comprehend me. Therefore it 
cannot do anything for me either.

Pauperism is the \textit{valuelessness of me}, the phenomenon that I cannot 
realize value from myself. For this reason State and pauperism are one and the 
same. The State does not let me come to my value, and continues in existence 
only through my valuelessness: it is forever intent on \textit{getting 
benefit} from me, \textit{i.e.} exploiting me, turning me to account, using me 
up, even if the use it gets from me consists only in my supplying a 
\textit{proles} (proletariat); it wants me to be "{}its creature."{}

Pauperism can be removed only when I as ego \textit{realize value} from 
myself, when I give my own self value, and make my price myself. I must rise 
in revolt to rise in the world.

What I produce, flour, linen, or iron and coal, which I toilsomely win from 
the earth, is my work that I want to realize value from. But then I may long 
complain that I am not paid for my work according to its value: the payer will 
not listen to me, and the State likewise will maintain an apathetic attitude 
so long as it does not think it must "{}appease"{} me that \textit{I} may not 
break out with my dreaded might. But this "{}appeasing"{} will be all, and, if 
it comes into my head to ask for more, the State turns against me with all the 
force of its lion-paws and eagle-claws: for it is the king of beasts, it is 
lion and eagle. If I refuse to be content with the price that it fixes for my 
ware and labor, if I rather aspire to determine the price of my ware myself, 
\textit{e. g.}, "{}to pay myself,"{} in the first place I come into a conflict 
with the buyers of the ware. If this were stilled by a mutual understanding, 
the State would not readily make objections; for how individuals get along 
with each other troubles it little, so long as therein they do not get in its 
way. Its damage and its danger begin only when they do not agree, but, in the 
absence of a settlement, take each other by the hair. The State cannot endure 
that man stand in a direct relation to man; it must step between as 
--\textit{mediator}, must \textit{-- intervene}. What Christ was, what the 
saints, the Church were, the State has become -- to wit, "{}mediator."{} It 
tears man from man to put itself between them as "{}spirit."{} The laborers 
who ask for higher pay are treated as criminals as soon as they want to 
\textit{compel} it. What are they to do? Without compulsion they don't get it, 
and in compulsion the State sees a self-help, a determination of price by the 
ego, a genuine, free realization of value from his property, which it cannot 
admit of. What then are the laborers to do? Look to themselves and ask nothing 
about the State? -- --

But, as is the situation with regard to my material work, so it is with my 
intellectual too. The State allows me to realize value from all my thoughts 
and to find customers for them (I do realize value from them, \textit{e. g.} 
in the very fact that they bring me honor from the listeners, etc.); but only 
so long as \textit{my} thoughts are --\textit{its} thoughts. If, on the other 
hand, I harbor thoughts that it cannot approve (\textit{i.e.} make its own), 
then it does not allow me at all to realize value from them, to bring them 
into \textit{exchange} into \textit{commerce. My} thoughts are free only if 
they are granted to me by the State's \textit{grace}, \textit{i.e.} if they 
are the State's thoughts. It lets me philosophize freely only so far as I 
approve myself a "{}philosopher of State"{}; \textit{against} the State I must 
not philosophize, gladly as it tolerates my helping it out of its 
"{}deficiencies,"{} "{}furthering"{} it. -- Therefore, as I may behave only as 
an ego most graciously permitted by the State, provided with its testimonial 
of legitimacy and police pass, so too it is not granted me to realize value 
from what is mine, unless this proves to be its, which I hold as fief from it. 
My ways must be its ways, else it distrains me; my thoughts its thoughts, else 
it stops my mouth.

The State has nothing to be more afraid of than the value of me, and nothing 
must it more carefully guard against than every occasion that offers itself to 
me for \textit{realizing value} from myself. \textit{I} am the deadly enemy of 
the State, which always hovers between the alternatives, it or I. Therefore it 
strictly insists not only on not letting \textit{me} have a standing, but also 
on keeping down what is \textit{mine}. In the State there is no property, 
\textit{i.e.} no property of the individual, but only State property. Only 
through the State have I what I have, as I am only through it what I am. My 
private property is only that which the State leaves to me of \textit{its, 
cutting off} others from it (depriving them, making it private); it is State 
property.

But, in opposition to the State, I feel more and more clearly that there is 
still left me a great might, the might over myself, \textit{i.e.} over 
everything that pertains only to me and that \textit{exists} only in being my 
own.

What do I do if my ways are no longer its ways, my thoughts no longer its 
thoughts? I look to myself, and ask nothing about it! In \textit{my} thoughts, 
which I get sanctioned by no assent, grant, or grace, I have my real property, 
a property with which I can trade. For as mine they are my \textit{creatures}, 
and I am in a position to give them away in return for \textit{other} 
thoughts: I give them up and take in exchange for them others, which then are 
my new purchased property.

What then is \textit{my} property? Nothing but what is in my \textit{power!} 
To what property am I entitled? To every property to which I -- 
\textit{empower} myself.\footnote{[A German idiom for "{}take upon myself,"{} 
"{}assume."{}]} I give myself the right of property in taking property to 
myself, or giving myself the proprietor's \textit{power}, full power, 
empowerment.

Everything over which I have might that cannot be torn from me remains my 
property; well, then let might decide about property, and I will expect 
everything from my might! Alien might, might that I leave to another, makes me 
an owned slave: then let my own might make me an owner. Let me then withdraw 
the might that I have conceded to others out of ignorance regarding the 
strength of my \textit{own} might! Let me say to myself, what my might reaches 
to is my property; and let me claim as property everything that I feel myself 
strong enough to attain, and let me extend my actual property as far as 
\textit{I} entitle, \textit{i. e.} -- empower, myself to take.

Here egoism, selfishness, must decide; not the principle of \textit{love}, not 
love-motives like mercy, gentleness, good-nature, or even justice and equity 
(for \textit{justitia} too is a phenomenon of -- love, a product of love): 
love knows only \textit{sacrifices} and demands "{}self-sacrifice."{}

Egoism does not think of sacrificing anything, giving away anything that it 
wants; it simply decides, what I want I must have and will procure.

All attempts to enact rational laws about property have put out from the bay 
of \textit{love} into a desolate sea of regulations. Even Socialism and 
Communism cannot be excepted from this. Every one is to be provided with 
adequate means, for which it is little to the point whether one 
socialistically finds them still in a personal property, or communistically 
draws them from the community of goods. The individual's mind in this remains 
the same; it remains a mind of dependence. The distributing \textit{board of 
equity} lets me have only what the sense of equity, its \textit{loving} care 
for all, prescribes. For me, the individual, there lies no less of a check in 
\textit{collective wealth} than in that of \textit{individual others;} neither 
that is mine, nor this: whether the wealth belongs to the collectivity, which 
confers part of it on me, or to individual possessors, is for me the same 
constraint, as I cannot decide about either of the two. On the contrary, 
Communism, by the abolition of all personal property, only presses me back 
still more into dependence on another, \textit{viz}., on the generality or 
collectivity; and, loudly as it always attacks the "{}State,"{} what it 
intends is itself again a State, a \textit{status}, a condition hindering my 
free movement, a sovereign power over me. Communism rightly revolts against 
the pressure that I experience from individual proprietors; but still more 
horrible is the might that it puts in the hands of the collectivity.

Egoism takes another way to root out the non-possessing rabble. It does not 
say: Wait for what the board of equity will -- bestow on you in the name of 
the collectivity (for such bestowal took place in "{}States"{} from the most 
ancient times, each receiving "{}according to his desert,"{} and therefore 
according to the measure in which each was able to \textit{deserve} it, to 
acquire it by \textit{service}), but: Take hold, and take what you require! 
With this the war of all against all is declared. I alone decide what I will 
have.

"{}Now, that is truly no new wisdom, for self-seekers have acted so at all 
times!"{} Not at all necessary either that the thing be new, if only 
\textit{consciousness} of it is present. But this latter will not be able to 
claim great age, unless perhaps one counts in the Egyptian and Spartan law; 
for how little current it is appears even from the stricture above, which 
speaks with contempt of "{}self-seekers."{} One is to know just this, that the 
procedure of taking hold is not contemptible, but manifests the pure deed of 
the egoist at one with himself.

Only when I expect neither from individuals nor from a collectivity what I can 
give to myself, only then do I slip out of the snares of --love; the rabble 
ceases to be rabble only when it \textit{takes hold}. Only the dread of taking 
hold, and the corresponding punishment thereof, makes it a rabble. Only that 
taking hold is \textit{sin}, crime -- only this dogma creates a rabble. For 
the fact that the rabble remains what it is, it (because it allows validity to 
that dogma) is to blame as well as, more especially, those who 
"{}self-seekingly"{} (to give them back their favorite word) demand that the 
dogma be respected. In short, the lack of \textit{consciousness} of that 
"{}new wisdom,"{} the old consciousness of sin, alone bears the blame.

If men reach the point of losing respect for property, every one will have 
property, as all slaves become free men as soon as they no longer respect the 
master as master. \textit{Unions} will then, in this matter too, multiply the 
individual's means and secure his assailed property.

According to the Communists' opinion the commune should be proprietor. On the 
contrary, \textit{I} am proprietor, and I only come to an understanding with 
others about my property. If the commune does not do what suits me, I rise 
against it and defend my property. I am proprietor, but property is 
\textit{not sacred}. I should be merely possessor? No, hitherto one was only 
possessor, secured in the possession of a parcel by leaving others also in 
possession of a parcel; but now \textit{everything} belongs to me, I am 
proprietor of \textit{everything that I require} and can get possession of. If 
it is said socialistically, society gives me what I require -- then the egoist 
says, I take what I require. If the Communists conduct themselves as 
ragamuffins, the egoist behaves as proprietor.

All swan-fraternities,\footnote{[Apparently some benevolent scheme of the day; 
compare note on p. 343.]} and attempts at making the rabble happy, that spring 
from the principle of love, must miscarry. Only from egoism can the rabble get 
help, and this help it must give to itself and -- will give to itself. If it 
does not let itself be coerced into fear, it is a power. "{}People would lose 
all respect if one did not coerce them into fear,"{} says bugbear Law in 
\textit{Der gestiefelte Kater}.

Property, therefore, should not and cannot be abolished; it must rather be 
torn from ghostly hands and become \textit{my} property; then the erroneous 
consciousness, that I cannot entitle myself to as much as I require, will 
vanish. --

"{}But what cannot man require!"{} Well, whoever requires much, and 
understands how to get it, has at all times helped himself to it, as Napoleon 
did with the Continent and France with Algiers. Hence the exact point is that 
the respectful "{}rabble"{} should learn at last to help itself to what it 
requires. If it reaches out too far for you, why, then defend yourselves. You 
have no need at all to good-heartedly -- bestow anything on it; and, when it 
learns to know itself, it -- or rather: whoever of the rabble learns to know 
himself, he -- casts off the rabble-quality in refusing your alms with thanks. 
But it remains ridiculous that you declare the rabble "{}sinful and 
criminal"{} if it is not pleased to live from your favors because it can do 
something in its own favor. Your bestowals cheat it and put it off. Defend 
your property, then you will be strong; if, on the other hand, you want to 
retain your ability to bestow, and perhaps actually have the more political 
rights the more alms (poor-rates) you can give, this will work just as long as 
the recipients let you work it.\footnote{In a registration bill for Ireland 
the government made the proposal to let those be electors who pay \pounds{}5 
sterling of poor-rates. He who gives alms, therefore, acquires political 
rights, or elsewhere becomes a swan-knight. [See p. 342.]}

In short, the property question cannot be solved so amicably as the 
Socialists, yes, even the Communists, dream. It is solved only by the war of 
all against all. The poor become free and proprietors only when they -- 
\textit{rise}. Bestow ever so much on them, they will still always want more; 
for they want nothing less than that at last -- nothing more be bestowed.

It will be asked, but how then will it be when the have- nots take heart? Of 
what sort is the settlement to be? One might as well ask that I cast a child's 
nativity. What a slave will do as soon as he has broken his fetters, one must 
--await.

In Kaiser's pamphlet, worthless for lack of form as well as substance 
(\textit{"{}Die Pers\"onlichkeit des Eigent\"umers in Bezug auf den 
Socialismus und Communismus},"{} etc.), he hopes from the \textit{State} that 
it will bring about a leveling of property. Always the State! Herr Papa! As 
the Church was proclaimed and looked upon as the "{}mother"{} of believers, so 
the State has altogether the face of the provident father.

\begin{center}
--------\end{center}


\textit{Competition} shows itself most strictly connected with the principle 
of civism. Is it anything else than \textit{equality} (\textit{\'egalit\'e})? 
And is not equality a product of that same Revolution which was brought on by 
the commonalty, the middle classes? As no one is barred from competing with 
all in the State (except the prince, because he represents the State itself) 
and working himself up to their height, yes, overthrowing or exploiting them 
for his own advantage, soaring above them and by stronger exertion depriving 
them of their favorable circumstances -- this serves as a clear proof that 
before the State's judgment-seat every one has only the value of a "{}simple 
individual"{} and may not count on any favoritism. Outrun and outbid each 
other as much as you like and can; that shall not trouble me, the State! Among 
yourselves you are free in competing, you are competitors; that is your 
\textit{social} position. But before me, the State, you are nothing but 
"{}simple individuals"{}!\footnote{Minister Stein used this expression about 
Count von Reisach, when he cold-bloodedly left the latter at the mercy of the 
Bavarian government because to him, as he said, "{}a government like Bavaria 
must be worth more than a simple individual."{} Reisach had written against 
Montgelas at Stein's bidding, and Stein later agreed to the giving up of 
Reisach, which was demanded by Montgelas on account of this very book. See 
Hinrichs, \textit{"{}Politische Vorlesungen},"{} I, 280.}

What in the form of principle or theory was propounded as the equality of all 
has found here in competition its realization and practical carrying out; for 
\textit{\'egalit\'e} is -- free competition. All are, before the State 
--simple individuals; in society, or in relation to each other -- competitors.

I need be nothing further than a simple individual to be able to compete with 
all others aside from the prince and his family: a freedom which formerly was 
made impossible by the fact that only by means of one's corporation, and 
within it, did one enjoy any freedom of effort.

In the guild and feudality the State is in an intolerant and fastidious 
attitude, granting \textit{privileges;} in competition and liberalism it is in 
a tolerant and indulgent attitude, granting only \textit{patents} (letters 
assuring the applicant that the business stands open (patent) to him) or 
"{}concessions."{} Now, as the State has thus left everything to the 
\textit{applicants}, it must come in conflict with all, because each and all 
are entitled to make application. It will be "{}stormed,"{} and will go down 
in this storm.

Is "{}free competition"{} then really "{}free?"{} nay, is it really a 
"{}competition"{} -- to wit, one of \textit{persons --} as it gives itself out 
to be because on this title it bases its right? It originated, you know, in 
persons becoming free of all personal rule. Is a competition "{}free"{} which 
the State, this ruler in the civic principle, hems in by a thousand barriers? 
There is a rich manufacturer doing a brilliant business, and I should like to 
compete with him. "{}Go ahead,"{} says the State, "{}I have no objection to 
make to your \textit{person} as competitor."{} Yes, I reply, but for that I 
need a space for buildings, I need money! "{}That's bad; but, if you have no 
money, you cannot compete. You must not take anything from anybody, for I 
protect property and grant it privileges."{} Free competition is not 
"{}free,"{} because I lack the THINGS for competition. Against my 
\textit{person} no objection can be made, but because I have not the things my 
person too must step to the rear. And who has the necessary things? Perhaps 
that manufacturer? Why, from him I could take them away! No, the State has 
them as property, the manufacturer only as fief, as possession.

But, since it is no use trying it with the manufacturer, I will compete with 
that professor of jurisprudence; the man is a booby, and I, who know a hundred 
times more than he, shall make his class-room empty. "{}Have you studied and 
graduated, friend?"{} No, but what of that? I understand abundantly what is 
necessary for instruction in that department. "{}Sorry, but competition is not 
'free' here. Against your person there is nothing to be said, but the 
\textit{thing}, the doctor's diploma, is lacking. And this diploma I, the 
State, demand. Ask me for it respectfully first; then we will see what is to 
be done."{}

This, therefore, is the "{}freedom"{} of competition. The State, \textit{my 
lord}, first qualifies me to compete.

But do \textit{persons} really compete? No, again \textit{things} only! Moneys 
in the first place, etc.

In the rivalry one will always be left behind another (\textit{e. g.} a 
poetaster behind a poet). But it makes a difference whether the means that the 
unlucky competitor lacks are personal or material, and likewise whether the 
material means can be won by \textit{personal energy} or are to be obtained 
only by \textit{grace}, only as a present; as when \textit{e. g.} the poorer 
man must leave, \textit{i. e.} present, to the rich man his riches. But, if I 
must all along wait for the State's \textit{approval} to obtain or to use 
(\textit{e. g.} in the case of graduation) the means, I have the means by the 
\textit{grace of the State}.\footnote{In colleges and universities poor men 
compete with rich. But they are able to do in most eases only through 
scholarships, which -- a significant point -- almost all come down to us from 
a time when free competition was still far from being a controlling principle. 
The principle of competition founds no scholarship, but says, Help yourself; 
provide yourself the means. What the State gives for such purposes it pays out 
from interested motives, to educate "{}servants"{} for itself.}

Free competition, therefore, has only the following meaning: To the State all 
rank as its equal children, and every one can scud and run to earn the 
\textit{State's goods and largesse}. Therefore all do chase after havings, 
holdings, possessions (be it of money or offices, titles of honor, etc.), 
after the \textit{things}.

In the mind of the commonalty every one is possessor or "{}owner."{} Now, 
whence comes it that the most have in fact next to nothing? From this, that 
the most are already joyful over being possessors at all, even though it be of 
some rags, as children are joyful in their first trousers or even the first 
penny that is presented to them. More precisely, however, the matter is to be 
taken as follows. Liberalism came forward at once with the declaration that it 
belonged to man's essence not to be property, but proprietor. As the 
consideration here was about "{}man,"{} not about the individual, the how-much 
(which formed exactly the point of the individual's special interest) was left 
to him. Hence the individual's egoism retained room for the freest play in 
this how- much, and carried on an indefatigable competition.

However, the lucky egoism had to become a snag in the way of the less 
fortunate, and the latter, still keeping its feet planted on the principle of 
humanity, put forward the question as to how-much of possession, and answered 
it to the effect that "{}man must have as much as he requires."{}

Will it be possible for \textit{my} egoism to let itself be satisfied with 
that? What "{}man"{} requires furnishes by no means a scale for measuring me 
and my needs; for I may have use for less or more. I must rather have so much 
as I am competent to appropriate.

Competition suffers from the unfavorable circumstance that the \textit{means} 
for competing are not at every one's command, because they are not taken from 
personality, but from accident. Most are \textit{without means}, and for this 
reason \textit{without goods}.

Hence the Socialists demand the \textit{means} for all, and aim at a society 
that shall offer means. Your money value, say they, we no longer recognize as 
your "{}competence"{}; you must show another competence -- to wit, your 
\textit{working force}. In the possession of a property, or as 
"{}possessor,"{} man does certainly show himself as man; it was for this 
reason that we let the possessor, whom we called "{}proprietor,"{} keep his 
standing so long. Yet you possess the things only so long as you are not 
"{}put out of this property."{}

The possessor is competent, but only so far as the others are incompetent. 
Since your ware forms your competence only so long as you are competent to 
defend it (\textit{i.e.} as \textit{we} are not competent to do anything with 
it), look about you for another competence; for we now, by our might, surpass 
your alleged competence.

It was an extraordinarily large gain made, when the point of being regarded as 
possessors was put through. Therein bondservice was abolished, and every one 
who till then had been bound to the lord's service, and more or less had been 
his property, now became a "{}lord."{} But henceforth your having, and what 
you have, are no longer adequate and no longer recognized; \textit{per 
contra}, your working and your work rise in value. We now respect your 
\textit{subduing} things, as we formerly did your possessing them. Your work 
is your competence! You are lord or possessor only of what comes by 
\textit{work}, not by \textit{inheritance}. But as at the time everything has 
come by inheritance, and every copper that you possess bears not a labor-stamp 
but an inheritance-stamp, everything must be melted over.

But is my work then really, as the Communists suppose, my sole competence? or 
does not this consist rather in everything that I am competent for? And does 
not the workers' society itself have to concede this, \textit{e. g.,} in 
supporting also the sick, children, old men -- in short, those who are 
incapable of work? These are still competent for a good deal, \textit{e. g.} 
for instance, to preserve their life instead of taking it. If they are 
competent to cause you to desire their continued existence, they have a power 
over you. To him who exercised utterly no power over you, you would vouchsafe 
nothing; he might perish.

Therefore, what you are \textit{competent} for is your \textit{competence!} If 
you are competent to furnish pleasure to thousands, then thousands will pay 
you an honorarium for it; for it would stand in your power to forbear doing 
it, hence they must purchase your deed. If you are not competent to 
\textit{captivate} any one, you may simply starve.

Now am I, who am competent for much, perchance to have no advantage over the 
less competent?

We are all in the midst of abundance; now shall I not help myself as well as I 
can, but only wait and see how much is left me in an equal division?

Against competition there rises up the principle of ragamuffin society -- 
\textit{partition}.

To be looked upon as a mere \textit{part}, part of society, the individual 
cannot bear -- because he is \textit{more;} his uniqueness puts from it this 
limited conception.

Hence he does not await his competence from the sharing of others, and even in 
the workers' society there arises the misgiving that in an equal partition the 
strong will be exploited by the weak; he awaits his competence rather from 
himself, and says now, what I am competent to have, that is my competence.

What competence does not the child possess in its smiling, its playing, its 
screaming! in short, in its mere existence! Are you capable of resisting its 
desire? Or do you not hold out to it, as mother, your breast; as father, as 
much of your possessions as it needs? It compels you, therefore it possesses 
what you call yours.

If your person is of consequence to me, you pay me with your very existence; 
if I am concerned only with one of your qualities, then your compliance, 
perhaps, or your aid, has a value (a money value) for me, and I 
\textit{purchase} it.

If you do not know how to give yourself any other than a money value in my 
estimation, there may arise the case of which history tells us, that Germans, 
sons of the fatherland, were sold to America. Should those who let themselves 
to be traded in be worth more to the seller? He preferred the cash to this 
living ware that did not understand how to make itself precious to him. That 
he discovered nothing more valuable in it was assuredly a defect of his 
competence; but it takes a rogue to give more than he has. How should he show 
respect when he did not have it, nay, hardly could have it for such a pack!

You behave egoistically when you respect each other neither as possessors nor 
as ragamuffins or workers, but as a part of your competence, as 
\textit{"{}useful bodies"{}}. Then you will neither give anything to the 
possessor ("{}proprietor"{}) for his possessions, nor to him who works, but 
only to him whom you \textit{require}. The North Americans ask themselves, Do 
we require a king? and answer, Not a farthing are he and his work worth to us.

If it is said that competition throws every thing open to all, the expression 
is not accurate, and it is better put thus: competition makes everything 
purchasable. In \textit{abandoning}\footnote{[\textit{preisgeben}]} it to 
them, competition leaves it to their appraisal\footnote{[\textit{Preis}]} or 
their estimation, and demands a price\footnote{[\textit{Preis}]} for it.

But the would-be buyers mostly lack the means to make themselves buyers: they 
have no money. For money, then, the purchasable things are indeed to be had 
("{}For money everything is to be had!"{}), but it is exactly money that is 
lacking. Where is one to get money, this current or circulating property? Know 
then, you have as much money\footnote{[\textit{Geld}]} as you have -- might; 
for you count\footnote{[\textit{gelten}]} for as much as you make yourself 
count for.

One pays not with money, of which there may come a lack, but with his 
competence, by which alone we are "{}competent"{};\footnote{[Equivalent in 
ordinary German use to our "{}possessed of a competence."{}]} for one is 
proprietor only so far as the arm of our power reaches.

Weitling has thought out a new means of payment -- work. But the true means of 
payment remains, as always, \textit{competence}. With what you have "{}within 
your competence"{} you pay. Therefore think on the enlargement of your 
competence.

This being admitted, they are nevertheless right on hand again with the motto, 
"{}To each according to his competence!"{} Who is to \textit{give} to me 
according to my competence? Society? Then I should have to put up with its 
estimation. Rather, I shall \textit{take} according to my competence.

"{}All belongs to all!"{} This proposition springs from the same unsubstantial 
theory. To each belongs only what he is competent for. If I say, The world 
belongs to me, properly that too is empty talk, which has a meaning only in so 
far as I respect no alien property. But to me belongs only as much as I am 
competent for, or have within my competence.

One is not worthy to have what one, through weakness, lets be taken from him; 
one is not worthy of it because one is not capable of it.

They raise a mighty uproar over the "{}wrong of a thousand years"{} which is 
being committed by the rich against the poor. As if the rich were to blame for 
poverty, and the poor were not in like manner responsible for riches! Is there 
another difference between the two than that of competence and incompetence, 
of the competent and incompetent? Wherein, pray, does the crime of the rich 
consist? "{}In their hardheartedness."{} But who then have maintained the 
poor? Who have cared for their nourishment? Who have given alms, those alms 
that have even their name from mercy (\textit{eleemosyne})? Have not the rich 
been "{}merciful"{} at all times? Are they not to this day 
"{}tender-hearted,"{} as poor-taxes, hospitals, foundations of all sorts, 
etc., prove?

But all this does not satisfy you! Doubtless, then, they are to \textit{share} 
with the poor? Now you are demanding that they shall abolish poverty. Aside 
from the point that there might be hardly one among you who would act so, and 
that this one would be a fool for it, do ask yourselves: why should the rich 
let go their fleeces and give up \textit{themselves}, thereby pursuing the 
advantage of the poor rather than their own? You, who have your thaler daily, 
are rich above thousands who live on four groschen. Is it for your interest to 
share with the thousands, or is it not rather for theirs? --

With competition is connected less the intention to do the thing \textit{best} 
than the intention to make it as \textit{profitable}, as productive, as 
possible. Hence people study to get into the civil service (pot-boiling 
study), study cringing and flattery, routine and "{}acquaintance with 
business,"{} work "{}for appearance."{} Hence, while it is apparently a matter 
of doing "{}good service,"{} in truth only a "{}good business"{} and earning 
of money are looked out for. The job is done only ostensibly for the job's 
sake, but in fact on account of the gain that it yields. One would indeed 
prefer not to be censor, but one wants to be -- advanced; one would like to 
judge, administer, etc., according to his best convictions, but one is afraid 
of transference or even dismissal; one must, above all things -- live.

Thus these goings-on are a fight for \textit{dear life}, and, in gradation 
upward, for more or less of a "{}good living."{}

And yet, withal, their whole round of toil and care brings in for most only 
"{}bitter life"{} and "{}bitter poverty."{} All the bitter painstaking for 
this!

Restless acquisition does not let us take breath, take a calm 
\textit{enjoyment:} we do not get the comfort of our possessions.

But the organization of labor touches only such labors as others can do for 
us, slaughtering, tillage, etc.; the rest remain egoistic, because no one can 
in your stead elaborate your musical compositions, carry out your projects of 
painting, etc.; nobody can replace Raphael's labors. The latter are labors of 
a unique person,\footnote{[\textit{Einzige}]} which only he is competent to 
achieve, while the former deserved to be called "{}human,"{} since what is 
anybody's \textit{own} in them is of slight account, and almost "{}any man"{} 
can be trained to it.

Now, as society can regard only labors for the common benefit, \textit{human} 
labors, he who does anything \textit{unique} remains without its care; nay, he 
may find himself disturbed by its intervention. The unique person will work 
himself forth out of society all right, but society brings forth no unique 
person.

Hence it is at any rate helpful that we come to an agreement about 
\textit{human} labors, that they may not, as under competition, claim all our 
time and toil. So far Communism will bear its fruits. For before the dominion 
of the commonalty even that for which all men are qualified, or can be 
qualified, was tied up to a few and withheld from the rest: it was a 
privilege. To the commonalty it looked equitable to leave free all that seemed 
to exist for every "{}man."{} But, because left\footnote{[Literally, 
"{}given."{}]} free, it was yet given to no one, but rather left to each to be 
got hold of by his \textit{human} power. By this the mind was turned to the 
acquisition of the human, which henceforth beckoned to every one; and there 
arose a movement which one hears so loudly bemoaned under the name of 
"{}materialism."{}

Communism seeks to check its course, spreading the belief that the human is 
not worth so much discomfort, and, with sensible arrangements, could be gained 
without the great expense of time and powers which has hitherto seemed 
requisite.

But for whom is time to be gained? For what does man require more time than is 
necessary to refresh his wearied powers of labor? Here Communism is silent.

For what? To take comfort in himself as the unique, after he has done his part 
as man!

In the first joy over being allowed to stretch out their hands toward 
everything human, people forgot to want anything else; and they competed away 
vigorously, as if the possession of the human were the goal of all our wishes.

But they have run themselves tired, and are gradually noticing that 
"{}possession does not give happiness."{} Therefore they are thinking of 
obtaining the necessary by an easier bargain, and spending on it only so much 
time and toil as its indispensableness exacts. Riches fall in price, and 
contented poverty, the care-free ragamuffin, becomes the seductive ideal.

Should such human activities, that every one is confident of his capacity for, 
be highly salaried, and sought for with toil and expenditure of all 
life-forces? Even in the everyday form of speech, "{}If I were minister, or 
even the., then it should go quite otherwise,"{} that confidence expresses 
itself -- that one holds himself capable of playing the part of such a 
dignitary; one does get a perception that to things of this sort there belongs 
not uniqueness, but only a culture which is attainable, even if not exactly by 
all, at any rate by many; \textit{i.e.} that for such a thing one need only be 
an ordinary man.

If we assume that, as \textit{order} belongs to the essence of the State, so 
\textit{subordination} too is founded in its nature, then we see that the 
subordinates, or those who have received preferment, disproportionately 
\textit{overcharge} and \textit{overreach} those who are put in the lower 
ranks. But the latter take heart (first from the Socialist standpoint, but 
certainly with egoistic consciousness later, of which we will therefore at 
once give their speech some coloring) for the question, By what then is your 
property secure, you creatures of preferment? -- and give themselves the 
answer, By our refraining from interference! And so by \textit{our} 
protection! And what do you give us for it? Kicks and disdain you give to the 
"{}common people"{}; police supervision, and a catechism with the chief 
sentence "{}Respect what is \textit{not yours}, what belongs to 
\textit{others!} respect others, and especially your superiors!"{} But we 
reply, "{}If you want our respect, \textit{buy} it for a price agreeable to 
us. We will leave you your property, if you give a due equivalent for this 
leaving."{} Really, what equivalent does the general in time of peace give for 
the many thousands of his yearly income.? -- another for the sheer 
hundred-thousands and millions yearly? What equivalent do you give for our 
chewing potatoes and looking calmly on while you swallow oysters? Only buy the 
oysters of us as dear as we have to buy the potatoes of you, then you may go 
on eating them. Or do you suppose the oysters do not belong to us as much as 
to you? You will make an outcry over \textit{violence} if we reach out our 
hands and help consume them, and you are right. Without violence we do not get 
them, as you no less have them by doing violence to us.

But take the oysters and have done with it, and let us consider our nearer 
property, labor; for the other is only possession. We distress ourselves 
twelve hours in the sweat of our face, and you offer us a few groschen for it. 
Then take the like for your labor too. Are you not willing? You fancy that our 
labor is richly repaid with that wage, while yours on the other hands is worth 
a wage of many thousands. But, if you did not rate yours so high, and gave us 
a better chance to realize value from ours, then we might well, if the case 
demanded it, bring to pass still more important things than you do for the 
many thousand thalers; and, if you got only such wages as we, you would soon 
grow more industrious in order to receive more. But, if you render any service 
that seems to us worth ten and a hundred times more than our own labor, why, 
then you shall get a hundred times more for it too; we, on the other hand, 
think also to produce for you things for which you will requite us more highly 
than with the ordinary day's wages. We shall be willing to get along with each 
other all right, if only we have first agreed on this -- that neither any 
longer needs to -- \textit{present} anything to the other. Then we may perhaps 
actually go so far as to pay even the cripples and sick and old an appropriate 
price for not parting from us by hunger and want; for, if we want them to 
live, it is fitting also that we -- purchase the fulfillment of our will. I 
say "{}purchase,"{} and therefore do not mean a wretched "{}alms."{} For their 
life is the property even of those who cannot work; if we (no matter for what 
reason) want them not to withdraw this life from us, we can mean to bring this 
to pass only by purchase; nay, we shall perhaps (maybe because we like to have 
friendly faces about us) even want a life of comfort for them. In short, we 
want nothing presented by you, but neither will we present you with anything. 
For centuries we have handed alms to you from goodhearted -- stupidity, have 
doled out the mite of the poor and given to the masters the things that are -- 
not the masters'; now just open your wallet, for henceforth our ware rises in 
price quite enormously. We do not want to take from you anything, anything at 
all, only you are to pay better for what you want to have. What then have you? 
"{}I have an estate of a thousand acres."{} And I am your plowman, and will 
henceforth attend to your fields only for one thaler a day wages. "{}Then I'll 
take another."{} You won't find any, for we plowmen are no longer doing 
otherwise, and, if one puts in an appearance who takes less, then let him 
beware of us. There is the housemaid, she too is now demanding as much, and 
you will no longer find one below this price. "{}Why, then it is all over with 
me."{} Not so fast! You will doubtless take in as much as we; and, if it 
should not be so, we will take off so much that you shall have wherewith to 
live like us. "{}But I am accustomed to live better."{} We have nothing 
against that, but it is not our look-out; if you can clear more, go ahead. Are 
we to hire out under rates, that you may have a good living?

The rich man always puts off the poor with the words, "{}What does your want 
concern me? See to it how you make your way through the world; that is 
\textit{your affair}, not mine."{} Well, let us let it be our affair, then, 
and let us not let the means that we have to realize value from ourselves be 
pilfered from us by the rich. "{}But you uncultured people really do not need 
so much."{} Well, we are taking somewhat more in order that for it we may 
procure the culture that we perhaps need. "{}But, if you thus bring down the 
rich, who is then to support the arts and sciences hereafter?"{} Oh, well, we 
must make it up by numbers; we club together, that gives a nice little sum -- 
besides, you rich men now buy only the most tasteless books and the most 
lamentable Madonnas or a pair of lively dancer's legs. "{}O ill-starred 
equality!"{} No, my good old sir, nothing of equality. We only want to count 
for what we are worth, and, if you are worth more, you shall count for more 
right along. We only want to be \textit{worth our price}, and think to show 
ourselves worth the price that you will pay.

Is the State likely to be able to awaken so secure a temper and so forceful a 
self-consciousness in the menial? Can it make man feel himself? Nay, may it 
even do so much as set this goal for itself? Can it want the individual to 
recognize his value and realize this value from himself? Let us keep the parts 
of the double question separate, and see first whether the State can bring 
about such a thing. As the unanimity of the plowmen is required, only this 
unanimity can bring it to pass, and a State law would be evaded in a thousand 
ways by competition and in secret.

But can the State bear with it? The State cannot possibly bear with people's 
suffering coercion from another than it; it could not, therefore, admit the 
self-help of the unanimous plowmen against those who want to engage for lower 
wages. Suppose, however, that the State made the law, and all the plowmen were 
in accord with it: could the State bear with it then?

In the isolated case -- yes; but the isolated case is more than that, it is a 
case of \textit{principle}. The question therein is of the whole range of the 
\textit{ego's self-realization of value from himself}, and therefore also of 
his self-consciousness \textit{against} the State. So far the Communists keep 
company; but, as self-realization of value from self necessarily directs 
itself against the State, so it does against \textit{society} too, and 
therewith reaches out beyond the commune and the communistic -- out of egoism.

Communism makes the maxim of the commonalty, that every one is a possessor 
("{}proprietor"{}), into an irrefragable truth, into a reality, since the 
anxiety about \textit{obtaining} now ceases and every one \textit{has} from 
the start what he requires. In his labor-force he \textit{has} his competence, 
and, if he makes no use of it, that is his fault. The grasping and hounding is 
at an end, and no competition is left (as so often now) without fruit, because 
with every stroke of labor an adequate supply of the needful is brought into 
the house. Now for the first time one is a \textit{real possessor}, because 
what one has in his labor-force can no longer escape from him as it was 
continually threatening to do under the system of competition. One is a 
\textit{care-free} and assured possessor. And one is this precisely by seeking 
his competence no longer in a ware, but in his own labor, his competence for 
labor; and therefore by being a \textit{ragamuffin}, a man of only ideal 
wealth. \textit{I}, however, cannot content myself with the little that I 
scrape up by my competence for labor, because my competence does not consist 
merely in my labor.

By labor I can perform the official functions of a president, a minister, 
etc.; these offices demand only a general culture -- to wit, such a culture as 
is generally attainable (for general culture is not merely that which every 
one has attained, but broadly that which every one can attain, and therefore 
every special culture, \textit{e. g.} medical, military, philological, of 
which no "{}cultivated man"{} believes that they surpass his powers), or, 
broadly, only a skill possible to all.

But, even if these offices may vest in every one, yet it is only the 
individual's unique force, peculiar to him alone. that gives them, so to 
speak, life and significance. That he does not manage his office like an 
"{}ordinary man."{} but puts in the competence of his uniqueness, this he is 
not yet paid for when he is paid only in general as an official or a minister. 
If he has done it so as to earn your thanks, and you wish to retain this 
thank-worthy force of the unique one, you must not pay him like a mere man who 
performed only what was human, but as one who accomplishes what is unique. Do 
the like with your labor, do!

There cannot be a general schedule-price fixed for my uniqueness as there can 
for what I do as man. Only for the latter can a schedule-price be set.

Go right on, then, setting up a general appraisal for human labors, but do not 
deprive your uniqueness of its desert.

\textit{Human} or \textit{general} needs can be satisfied through society; for 
satisfaction of \textit{unique} needs you must do some seeking. A friend and a 
friendly service, or even an individual's service, society cannot procure you. 
And yet you will every moment be in need of such a service, and on the 
slightest occasions require somebody who is helpful to you. Therefore do not 
rely on society, but see to it that you have the wherewithal to -- purchase 
the fulfillment of your wishes.

Whether money is to be retained among egoists? To the old stamp an inherited 
possession adheres. If you no longer let yourselves be paid with it, it is 
ruined: if you do nothing for this money, it loses all power. Cancel the 
\textit{inheritance}, and you have broken off the executor's court-seal. For 
now everything is an inheritance, whether it be already inherited or await its 
heir. If it is yours, wherefore do you let it be sealed up from you? Why do 
you respect the seal?

But why should you not create a new money? Do you then annihilate the ware in 
taking from it the hereditary stamp? Now, money is a ware, and an essential 
\textit{means} or competence. For it protects against the ossification of 
resources, keeps them in flux and brings to pass their exchange. If you know a 
better medium of exchange, go ahead; yet it will be a "{}money"{} again. It is 
not the money that does you damage, but your incompetence to take it. Let your 
competence take effect, collect yourselves, and there will be no lack of money 
-- of your money, the money of \textit{your} stamp. But working I do not call 
"{}letting your competence take effect."{} Those who are only "{}looking for 
work"{} and "{}willing to work hard"{} are preparing for their own selves the 
infallible upshot -- to be out of work.

Good and bad luck depend on money. It is a power in the \textit{bourgeois} 
period for this reason, that it is only wooed on all hands like a girl, 
indissolubly wedded by nobody. All the romance and chivalry of \textit{wooing} 
for a dear object come to life again in competition. Money, an object of 
longing, is carried off by the bold "{}knights of industry."{}\footnote{[A 
German phrase for sharpers.]}

He who has luck takes home the bride. The ragamuffin has luck; he takes her 
into his household, "{}society,"{} and destroys the virgin. In his house she 
is no longer bride, but wife; and with her virginity her family name is also 
lost. As housewife the maiden Money is called "{}Labor,"{} for "{}Labor"{} is 
her husband's name. She is a possession of her husband's.

To bring this figure to an end, the child of Labor and Money is again a girl, 
an unwedded one and therefore Money but with the certain descent from Labor, 
her father. The form of the face, the "{}effigy,"{} bears another stamp.

Finally, as regards competition once more, it has a continued existence by 
this very means, that all do not attend to \textit{their affair} and come to 
an \textit{understanding} with each other about it. Bread \textit{e. g.} is a 
need of all the inhabitants of a city; therefore they might easily agree on 
setting up a public bakery. Instead of this, they leave the furnishing of the 
needful to the competing bakers. Just so meat to the butchers, wine to 
wine-dealers, etc.

Abolishing competition is not equivalent to favoring the guild. The difference 
is this: In the \textit{guild} baking, etc., is the affair of the 
guild-brothers; in \textit{competition}, the affair of chance competitors; in 
the \textit{union}, of those who require baked goods, and therefore my affair, 
yours, the affair of neither the guildic nor the concessionary baker, but the 
affair of the \textit{united}.

If \textit{I} do not trouble myself about my affair, I must be 
\textit{content} with what it pleases others to vouchsafe me. To have bread is 
my affair, my wish and desire, and yet people leave that to the bakers and 
hope at most to obtain through their wrangling, their getting ahead of each 
other, their rivalry --in short, their competition -- an advantage which one 
could not count on in the case of the guild-brothers who were lodged 
\textit{entirely} and \textit{alone} in the proprietorship of the baking 
franchise. -- What every one requires, every one should also take a hand in 
procuring and producing; it is \textit{his} affair, his property, not the 
property of the guildic or concessionary master.

Let us look back once more. The world belongs to the children of this world, 
the children of men; it is no longer God's world, but man's. As much as every 
man can procure of it, let him call his; only the true man, the State, human 
society or mankind, will look to it that each shall make nothing else his own 
than what he appropriates as man, \textit{i.e.} in human fashion. Unhuman 
appropriation is that which is not consented to by man, \textit{i.e.}, it is a 
"{}criminal"{} appropriation, as the human, \textit{vice versa}, is a 
"{}rightful"{} one, one acquired in the "{}way of law."{}

So they talk since the Revolution.

But my property is not a thing, since this has an existence independent of me; 
only my might is my own. Not this tree, but my might or control over it, is 
what is mine.

Now, how is this might perversely expressed? They say I have a \textit{right} 
to this tree, or it is my \textit{rightful} property. So I have 
\textit{earned} it by might. That the might must last in order that the tree 
may also be \textit{held --} or better, that the might is not a thing existing 
of itself, but has existence solely in the \textit{mighty ego}, in me the 
mighty -- is forgotten. Might, like other of my \textit{qualities} (\textit{e. 
g.} humanity, majesty, etc.), is exalted to something existing of itself, so 
that it still exists long after it has ceased to be \textit{my} might. Thus 
transformed into a ghost, might is -- \textit{right}. This 
\textit{eternalized} might is not extinguished even with my death, but is 
transferred to "{}bequeathed."{}

Things now really belong not to me, but to right.

On the other side, this is nothing but a hallucination of vision. For the 
individual's might becomes permanent and a right only by others joining their 
might with his. The delusion consists in their believing that they cannot 
withdraw their might. The same phenomenon over again; might is separated from 
me. I cannot take back the might that I gave to the possessor. One has 
"{}granted power of attorney,"{} has given away his power, has renounced 
coming to a better mind.

The proprietor can give up his might and his right to a thing by giving the 
thing away, squandering it, etc. And \textit{we} should not be able likewise 
to let go the might that we lend to him?

The rightful man, the \textit{just}, desires to call nothing his own that he 
does not have "{}rightly"{} or have the right to, and therefore only 
\textit{legitimate property}.

Now, who is to be judge, and adjudge his right to him? At last, surely, Man, 
who imparts to him the rights of man: then he can say, in an infinitely 
broader sense than Terence, \textit{humani nihil a me alienum puto}, 
\textit{e. g.}, \textit{the human is my property}. However he may go about it, 
so long as he occupies this standpoint he cannot get clear of a judge; and in 
our time the multifarious judges that had been selected have set themselves 
against each other in two persons at deadly enmity -- to wit, in God and Man. 
The one party appeal to divine right, the other to human right or the rights 
of man.

So much is clear, that in neither case does the individual do the entitling 
himself.

Just pick me out an action today that would not be a violation of right! Every 
moment the rights of man are trampled under foot by one side, while their 
opponents cannot open their mouth without uttering a blasphemy against divine 
right. Give an alms, you mock at a right of man, because the relation of 
beggar and benefactor is an inhuman relation; utter a doubt, you sin against a 
divine right. Eat dry bread with contentment, you violate the right of man by 
your equanimity; eat it with discontent, you revile divine right by your 
repining. There is not one among you who does not commit a crime at every 
moment; your speeches are crimes, and every hindrance to your freedom of 
speech is no less a crime. Ye are criminals altogether!

Yet you are so only in that you all stand on the \textit{ground of right}, 
\textit{i.e.} in that you do not even know, and understand how to value, the 
fact that you are criminals.

Inviolable or \textit{sacred} property has grown on this very ground: it is a 
\textit{juridical concept}.

A dog sees the bone in another's power, -- and stands off only if it feels 
itself too weak. But man respects the other's \textit{right} to his bone. The 
latter action, therefore, ranks as \textit{human}, the former as 
\textit{brutal} or "{}egoistic."{}

And as here, so in general, it is called \textit{"{}human"{}} when one sees in 
everything something \textit{spiritual} (here right), \textit{i.e.} makes 
everything a ghost and takes his attitude toward it as toward a ghost, which 
one can indeed scare away at its appearance, but cannot kill. It is human to 
look at what is individual not as individual, but as a generality.

In nature as such I no longer respect anything, but know myself to be entitled 
to everything against it; in the tree in that garden, on the other hand, I 
must respect \textit{alienness} (they say in one-sided fashion 
"{}property"{}), I must keep my hand off it. This comes to an end only when I 
can indeed leave that tree to another as I leave my stick. etc., to another, 
but do not in advance regard it as alien to me, \textit{i.e.} sacred. Rather, 
I make to myself no \textit{crime} of felling it if I will, and it remains my 
property, however long as I resign it to others: it is and remains 
\textit{mine}. In the banker's fortune I as little see anything alien as 
Napoleon did in the territories of kings: we have no \textit{dread} of 
\textit{"{}conquering"{}} it, and we look about us also for the means thereto. 
We strip off from it, therefore, the \textit{spirit} of \textit{alienness}, of 
which we had been afraid.

Therefore it is necessary that I do not lay claim to, anything more \textit{as 
man}, but to everything as I, this I; and accordingly to nothing human, but to 
mine; \textit{i. e.}, nothing that pertains to me as man, but -- what I will 
and because I will it.

Rightful, or legitimate, property of another will be only that which 
\textit{you} are content to recognize as such. If your content ceases, then 
this property has lost legitimacy for you, and you will laugh at absolute 
right to it.

Besides the hitherto discussed property in the limited sense, there is held up 
to our reverent heart another property against which we are far less "{}to 
sin."{} This property consists in spiritual goods, in the "{}sanctuary of the 
inner nature."{} What a man holds sacred, no other is to gibe at; because, 
untrue as it may be, and zealously as one may "{}in loving and modest wise"{} 
seek to convince of a true sanctity the man who adheres to it and believes in 
it, yet \textit{the sacred} itself is always to be honored in it: the mistaken 
man does believe in the sacred, even though in an incorrect essence of it, and 
so his belief in the sacred must at least be respected.

In ruder times than ours it was customary to demand a particular faith, and 
devotion to a particular sacred essence, and they did not take the gentlest 
way with those who believed otherwise; since, however, "{}freedom of belief"{} 
spread itself more and more abroad, the "{}jealous God and sole Lord"{} 
gradually melted into a pretty general "{}supreme being,"{} and it satisfied 
humane tolerance if only every one revered "{}something sacred."{}

Reduced to the most human expression, this sacred essence is "{}man himself"{} 
and "{}the human."{} With the deceptive semblance as if the human were 
altogether our own, and free from all the otherworldliness with which the 
divine is tainted -- yes, as if Man were as much as I or you -- there may 
arise even the proud fancy that the talk is no longer of a "{}sacred 
essence"{} and that we now feel ourselves everywhere at home and no longer in 
the uncanny,\footnote{[Literally, "{}unhomely."{}]} \textit{i.e.} in the 
sacred and in sacred awe: in the ecstasy over "{}Man discovered at last"{} the 
egoistic cry of pain passes unheard, and the spook that has become so intimate 
is taken for our true ego.

But "{}Humanus is the saint's name"{} (see Goethe), and the humane is only the 
most clarified sanctity.

The egoist makes the reverse declaration. For this precise reason, because you 
hold something sacred, I gibe at you; and, even if I respected everything in 
you, your sanctuary is precisely what I should not respect.

With these opposed views there must also be assumed a contradictory relation 
to spiritual goods: the egoist insults them, the religious man (\textit{i.e.} 
every one who puts his "{}essence"{} above himself) must consistently -- 
protect them. But what kind of spiritual goods are to be protected, and what 
left unprotected, depends entirely on the concept that one forms of the 
"{}supreme being"{}; and he who fears God, \textit{e. g.}, has more to shelter 
than he (the liberal) who fears Man.

In spiritual goods we are (in distinction from the sensuous) injured in a 
spiritual way, and the sin against them consists in a direct 
\textit{desecration}, while against the sensuous a purloining or alienation 
takes place; the goods themselves are robbed of value and of consecration, not 
merely taken away; the sacred is immediately compromised. With the word 
"{}irreverence"{} or "{}flippancy"{} is designated everything that can be 
committed as \textit{crime} against spiritual goods, \textit{i.e.} against 
everything that is sacred for us; and scoffing, reviling, contempt, doubt, 
etc., are only different shades of \textit{criminal flippancy}.

That desecration can be practiced in the most manifold way is here to be 
passed over, and only that desecration is to be preferentially mentioned which 
threatens the sacred with danger through an \textit{unrestricted press}.

As long as respect is demanded even for one spiritual essence, speech and the 
press must be enthralled in the name of this essence; for just so long the 
egoist might "{}trespass"{} against it by his \textit{utterances}, from which 
thing he must be hindered by "{}due punishment"{} at least, if one does not 
prefer to take up the more correct means against it, the preventive use of 
police authority, \textit{e. g.} censorship.

What a sighing for liberty of the press! What then is the press to be 
liberated from? Surely from a dependence, a belonging, and a liability to 
service!

But to liberate himself from that is every one's affair, and it may with 
safety be assumed that, when you have delivered yourself from liability to 
service, that which you compose and write will also belong to you as your 
\textit{own} instead of having been thought and indicted \textit{in} the 
service of some power. What can a believer in Christ say and have printed, 
that should be freer from that belief in Christ than he himself is? If I 
cannot or may not write something, perhaps the primary fault lies with 
\textit{me}. Little as this seems to hit the point, so near is the application 
nevertheless to be found. By a press-law I draw a boundary for my 
publications, or let one be drawn, beyond which wrong and its 
\textit{punishment} follows. I myself \textit{limit} myself.

If the press was to be free, nothing would be so important as precisely its 
liberation from every coercion that could be put on it in the \textit{name of 
a law}. And, that it might come to that, I my own self should have to have 
absolved myself from obedience to the law.

Certainly, the absolute liberty of the press is like every absolute liberty, a 
nonentity. The press can become free from full many a thing, but always only 
from what I too am free from. If we make ourselves free from the sacred, if we 
have become \textit{graceless} and \textit{lawless}, our words too will become 
so.

As little as \textit{we} can be declared clear of every coercion in the world, 
so little can our writing be withdrawn from it. But as free as we are, so free 
we can make it too.

It must therefore become our \textit{own}, instead of, as hitherto, serving a 
spook.

People do not yet know what they mean by their cry for liberty of the press. 
What they ostensibly ask is that the State shall set the press free; but what 
they are really after, without knowing it themselves, is that the press become 
free from the State, or clear of the State. The former is a \textit{petition 
to} the State, the latter an \textit{insurrection against} the State. As a 
"{}petition for right,"{} even as a serious demanding of the right of liberty 
of the press, it presupposes the State as the giver, and can hope only for a 
\textit{present}, a permission, a chartering. Possible, no doubt, that a State 
acts so senselessly as to grant the demanded present; but you may bet 
everything that those who receive the present will not know how to use it so 
long as they regard the State as a truth: they will not trespass against this 
"{}sacred thing,"{} and will call for a penal press-law against every one who 
would be willing to dare this.

In a word, the press does not become free from what I am not free from.

Do I perhaps hereby show myself an opponent of the liberty of the press? On 
the contrary, I only assert that one will never get it if one wants only it, 
the liberty of the press, \textit{i.e.} if one sets out only for an 
unrestricted permission. Only beg right along for this permission: you may 
wait forever for it, for there is no one in the world who could give it to 
you. As long as you want to have yourselves "{}entitled"{} to the use of the 
press by a permission, \textit{i.e.} liberty of the press, you live in vain 
hope and complaint.

"{}Nonsense! Why, you yourself, who harbor such thoughts as stand in your 
book, can unfortunately bring them to publicity only through a lucky chance or 
by stealth; nevertheless you will inveigh against one's pressing and 
importuning his own State till it gives the refused permission to print?"{} 
But an author thus addressed would perhaps -- for the impudence of such people 
goes far -- give the following reply: "{}Consider well what you say! What then 
do I do to procure myself liberty of the press for my book? Do I ask for 
permission, or do I not rather, without any question of legality, seek a 
favorable occasion and grasp it in complete recklessness of the State and its 
wishes? I -- the terrifying word must be uttered -- I cheat the State. You 
unconsciously do the same. From your tribunes you talk it into the idea that 
it must give up its sanctity and inviolability, it must lay itself bare to the 
attacks of writers, without needing on that account to fear danger. But you 
are imposing on it; for its existence is done for as soon as it loses its 
unapproachableness. To \textit{you} indeed it might well accord liberty of 
writing, as England has done; you are \textit{believers in the State} and 
incapable of writing against the State, however much you would like to reform 
it and 'remedy its defects.' But what if opponents of the State availed 
themselves of free utterance, and stormed out against Church, State, morals, 
and everything 'sacred' with inexorable reasons? You would then be the first, 
in terrible agonies, to call into life the \textit{September laws}. Too late 
would you then rue the stupidity that earlier made you so ready to fool and 
palaver into compliance the State, or the government of the State. -- But, I 
prove by my act only two things. This for one, that the liberty of the press 
is always bound to 'favorable opportunities,' and accordingly will never be an 
absolute liberty; but secondly this, that he who would enjoy it must seek out 
and, if possible, create the favorable opportunity, availing himself of his 
\textit{own advantage} against the State; and counting himself and his will 
more than the State and every 'superior' power. Not in the State, but only 
against it, can the liberty of the press be carried through; if it is to be 
established, it is to be obtained not as the consequence of a 
\textit{petition} but as the work of an \textit{insurrection}. Every petition 
and every motion for liberty of the press is already an insurrection, be it 
conscious or unconscious: a thing which Philistine halfness alone will not and 
cannot confess to itself until, with a shrinking shudder, it shall see it 
clearly and irrefutably by the outcome. For the requested liberty of the press 
has indeed a friendly and well-meaning face at the beginning, as it is not in 
the least minded ever to let the 'insolence of the press' come into vogue; but 
little by little its heart grows more hardened, and the inference flatters its 
way in that really a liberty is not a liberty if it stands in the 
\textit{service} of the State, of morals, or of the law. A liberty indeed from 
the coercion of censorship, it is yet not a liberty from the coercion of law. 
The press, once seized by the lust for liberty, always wants to grow freer, 
till at last the writer says to himself, really I am not wholly free till I 
ask about nothing; and writing is free only when it is my \textit{own}, 
dictated to me by no power or authority, by no faith, no dread; the press must 
not be free -- that is too little -- it must be \textit{mine: -- ownness of 
the press} or \textit{property in the press}, that is what I will take.

``Why, liberty of the press is only \textit{permission of the press}, and the 
State never will or can voluntarily permit me to grind it to nothingness by 
the press."{}

Let us now, in conclusion, bettering the above language, which is still vague, 
owing to the phrase 'liberty of the press,' rather put it thus: 
\textit{"{}liberty of the press}, the liberals' loud demand, is assuredly 
possible in the State; yes, it is possible only \textit{in} the State, because 
it is a \textit{permission}, and consequently the permitter (the State) must 
not be lacking. But as permission it has its limit in this very State, which 
surely should not in reason permit more than is compatible with itself and its 
welfare: the State fixes for it this limit as the \textit{law} of its 
existence and of its extension. That one State brooks more than another is 
only a quantitative distinction, which alone, nevertheless, lies at the heart 
of the political liberals: they want in Germany, \textit{i. e.}, only a 
'\textit{more extended, broader} accordance of free utterance.' The liberty of 
the press which is sought for is an affair of the \textit{people's}, and 
before the people (the State) possesses it I may make no use of it. From the 
standpoint of property in the press, the situation is different. Let my 
people, if they will, go without liberty of free press, I will manage to print 
by force or ruse; I get my permission to print only from -- \textit{myself} 
and my strength.

If the press is \textit{my own}, I as little need a permission of the State 
for employing it as I seek that permission in order to blow my nose. The press 
is my \textit{property} from the moment when nothing is more to me than 
myself; for from this moment State, Church, people, society, etc., cease, 
because they have to thank for their existence only the disrespect that I have 
for myself, and with the vanishing of this undervaluation they themselves are 
extinguished: they exist only when they exist \textit{above me}, exist only as 
\textit{powers} and \textit{power-holders}. Or can you imagine a State whose 
citizens one and all think nothing of it? It would be as certainly a dream, an 
existence in seeming, as 'united Germany.'

The press is my own as soon as I myself am my own, a self- owned man: to the 
egoist belongs the world, because he belongs to no power of the world.

With this my press might still be very \textit{unfree}, as \textit{e. g.} at 
this moment. But the world is large, and one helps himself as well as he can. 
If I were willing to abate from the \textit{property} of my press, I could 
easily attain the point where I might everywhere have as much printed as my 
fingers produced. But, as I want to assert my property, I must necessarily 
swindle my enemies. 'Would you not accept their permission if it were given 
you?' Certainly, with joy; for their permission would be to me a proof that I 
had fooled them and started them on the road to ruin. I am not concerned for 
their permission, but so much the more for their folly and their overthrow. I 
do not sue for their permission as if I flattered myself (like the political 
liberals) that we both, they and I, could make out peaceably alongside and 
with each other, yes, probably raise and prop each other; but I sue for it in 
order to make them bleed to death by it, that the permitters themselves may 
cease at last. I act as a conscious enemy, overreaching them and 
\textit{utilizing} their heedlessness.

The press is \textit{mine} when I recognize outside myself no \textit{judge} 
whatever over its utilization, \textit{i.e.} when my writing is no longer 
determined by morality or religion or respect for the State laws or the like, 
but by me and my egoism!"{}

Now, what have you to reply to him who gives you so impudent an answer? -- We 
shall perhaps put the question most strikingly by phrasing it as follows: 
Whose is the press, the people's (State's) or mine? The politicals on their 
side intend nothing further than to liberate the press from personal and 
arbitrary interferences of the possessors of power, without thinking of the 
point that to be really open for everybody it would also have to be free from 
the laws, from the people's (State's) will. They want to make a "{}people's 
affair"{} of it.

But, having become the people's property, it is still far from being mine; 
rather, it retains for me the subordinate significance of a 
\textit{permission}. The people plays judge over my thoughts; it has the right 
of calling me to account for them, or, I am responsible to it for them. 
Jurors, when their fixed ideas are attacked, have just as hard heads as the 
stiffest despots and their servile officials.

In the \textit{"{}Liberale Bestrebungen}"{}\footnote{II, p. 91ff. (See my note 
above.)} Edgar Bauer asserts that liberty of the press is impossible in the 
absolutist and the constitutional State, whereas in the "{}free State"{} it 
finds its place. "{}Here,"{} the statement is, "{}it is recognized that the 
individual, because he is no longer an individual but a member of a true and 
rational generality, has the right to utter his mind."{} So not the 
individual, but the "{}member,"{} has liberty of the press. But, if for the 
purpose of liberty of the press the individual must first give proof of 
himself regarding his belief in the generality, the people; if he does not 
have this liberty \textit{through might of his own --} then it is a 
\textit{people's liberty}, a liberty that he is invested with for the sake of 
his faith, his "{}membership."{} The reverse is the case: it is precisely as 
an individual that every one has open to him the liberty to utter his mind. 
But he has not the "{}right"{}: that liberty is assuredly not his "{}sacred 
right."{} He has only the \textit{might;} but the might alone makes him owner. 
I need no concession for the liberty of the press, do not need the people's 
consent to it, do not need the "{}right"{} to it, nor any "{}justification."{} 
The liberty of the press too, like every liberty, I must "{}take"{}; the 
people, "{}as being the sole judge,"{} cannot \textit{give} it to me. It can 
put up with me the liberty that I take, or defend itself against it; give, 
bestow, grant it cannot. I exercise it \textit{despite} the people, purely as 
an individual; \textit{i.e.} I get it by fighting the people, my -- enemy, and 
obtain it only when I really get it by such fighting, \textit{i. e. take} it. 
But I take it because it is my property.

Sander, against whom E. Bauer writes, lays claim (page 99) to the liberty of 
the press "{}as the right and the liberty of the \textit{citizens in the 
State"{}}. What else does Edgar Bauer do? To him also it is only a right of 
the free \textit{citizen}.

The liberty of the press is also demanded under the name of a "{}general human 
right."{} Against this the objection was well-founded that not every man knew 
how to use it rightly, for not every individual was truly man. Never did a 
government refuse it to \textit{Man} as such; but \textit{Man} writes nothing, 
for the reason that he is a ghost. It always refused it to 
\textit{individuals} only, and gave it to others, \textit{e. g.} its organs. 
If then one would have it for all, one must assert outright that it is due to 
the individual, me, not to man or to the individual so far as he is man. 
Besides, another than a man (a beast) can make no use of it. The French 
government, \textit{e. g.}, does not dispute the liberty of the press as a 
right of man, but demands from the individual a security for his really being 
man; for it assigns liberty of the press not to the individual, but to man.

Under the exact pretense that it was \textit{not human}, what was mine was 
taken from me! What was human was left to me undiminished.

Liberty of the press can bring about only a \textit{responsible} press; the 
\textit{irresponsible} proceeds solely from property in the press.

\begin{center}
--------\end{center}


For intercourse with men an express law (conformity to which one may venture 
at times sinfully to forget, but the absolute value of which one at no time 
ventures to deny) is placed foremost among all who live religiously: this is 
the law -- of \textit{love}, to which not even those who seem to fight against 
its principle, and who hate its name, have as yet become untrue; for they also 
still have love, yes, they love with a deeper and more sublimated love, they 
love "{}man and mankind."{}

If we formulate the sense of this law, it will be about as follows: Every man 
must have a something that is more to him than himself. You are to put your 
"{}private interest"{} in the background when it is a question of the welfare 
of others, the weal of the fatherland, of society, the common weal, the weal 
of mankind, the good cause, etc.! Fatherland, society, mankind, must be more 
to you than yourself, and as against their interest your "{}private 
interest"{} must stand back; for you must not be an --egoist.

Love is a far-reaching religious demand, which is not, as might be supposed, 
limited to love to God and man, but stands foremost in every regard. Whatever 
we do, think, will, the ground of it is always to be love. Thus we may indeed 
judge, but only "{}with love."{} The Bible may assuredly be criticized, and 
that very thoroughly, but the critic must before all things \textit{love} it 
and see in it the sacred book. Is this anything else than to say he must not 
criticize it to death, he must leave it standing, and that as a sacred thing 
that cannot be upset? -- In our criticism on men too, love must remain the 
unchanged key-note. Certainly judgments that hatred inspires are not at all 
our \textit{own} judgments, but judgments of the hatred that rules us, 
"{}rancorous judgments."{} But are judgments that love inspires in us any more 
our \textit{own}? They are judgments of the love that rules us, they are 
"{}loving, lenient"{} judgments, they are not our \textit{own}, and 
accordingly not real judgments at all. He who burns with love for justice 
cries out, \textit{fiat justitia, pereat mundus!} He can doubtless ask and 
investigate what justice properly is or demands, and \textit{in what} it 
consists, but not \textit{whether} it is anything.

It is very true, "{}He who abides in love abides in God, and God in him."{} (1 
John 4. 16.) God abides in him, he does not get rid of God, does not become 
godless; and he abides in God, does not come to himself and into his own home, 
abides in love to God and does not become loveless.

"{}God is love! All times and all races recognize in this word the central 
point of Christianity."{} God, who is love, is an officious God: he cannot 
leave the world in peace, but wants to make it \textit{blest}. "{}God became 
man to make men divine."{}\footnote{Athanasius.} He has his hand in the game 
everywhere, and nothing happens without it; everywhere he has his "{}best 
purposes,"{} his "{}incomprehensible plans and decrees."{} Reason, which he 
himself is, is to be forwarded and realized in the whole world. His fatherly 
care deprives us of all independence. We can do nothing sensible without its 
being said, God did that, and can bring upon ourselves no misfortune without 
hearing, God ordained that; we have nothing that we have not from him, he 
"{}gave"{} everything. But, as God does, so does Man. God wants perforce to 
make the world \textit{blest}, and Man wants to make it \textit{happy}, to 
make all men happy. Hence every "{}man"{} wants to awaken in all men the 
reason which he supposes his own self to have: everything is to be rational 
throughout. God torments himself with the devil, and the philosopher does it 
with unreason and the accidental. God lets no being go \textit{its own} gait, 
and Man likewise wants to make us walk only in human wise.

But whoso is full of sacred (religious, moral, humane) love loves only the 
spook, the "{}true man,"{} and persecutes with dull mercilessness the 
individual, the real man, under the phlegmatic legal title of measures against 
the "{}un- man."{} He finds it praiseworthy and indispensable to exercise 
pitilessness in the harshest measure; for love to the spook or generality 
commands him to hate him who is not ghostly, \textit{i.e.} the egoist or 
individual; such is the meaning of the renowned love-phenomenon that is called 
"{}justice."{}

The criminally arraigned man can expect no forbearance, and no one spreads a 
friendly veil over his unhappy nakedness. Without emotion the stern judge 
tears the last rags of excuse from the body of the poor accused; without 
compassion the jailer drags him into his damp abode; without placability, when 
the time of punishment has expired, he thrusts the branded man again among 
men, his good, Christian, loyal brethren, who contemptuously spit on him. Yes, 
without grace a criminal "{}deserving of death"{} is led to the scaffold, and 
before the eyes of a jubilating crowd the appeased moral law celebrates its 
sublime -- revenge. For only one can live, the moral law or the criminal. 
Where criminals live unpunished, the moral law has fallen; and, where this 
prevails, those must go down. Their enmity is indestructible.

The Christian age is precisely that of \textit{mercy, love}, solicitude to 
have men receive what is due them, yes, to bring them to fulfil their human 
(divine) calling. Therefore the principle has been put foremost for 
intercourse, that this and that is man's essence and consequently his calling, 
to which either God has called him or (according to the concepts of today) his 
being man (the species) calls him. Hence the zeal for conversion. That the 
Communists and the humane expect from man more than the Christians do does not 
change the standpoint in the least. Man shall get what is human! If it was 
enough for the pious that what was divine became his part, the humane demand 
that he be not curtailed of what is human. Both set themselves against what is 
egoistic. Of course; for what is egoistic cannot be accorded to him or vested 
in him (a fief); he must procure it for himself. Love imparts the former, the 
latter can be given to me by myself alone.

Intercourse hitherto has rested on love, \textit{regardful} behavior, doing 
for each other. As one owed it to himself to make himself blessed, or owed 
himself the bliss of taking up into himself the supreme essence and bringing 
it to a \textit{v\'erit\'e} (a truth and reality), so one owed it to 
\textit{others} to help them realize their essence and their calling: in both 
cases one owed it to the essence of man to contribute to its realization.

But one owes it neither to himself to make anything out of himself, nor to 
others to make anything out of them; for one owes nothing to his essence and 
that of others. Intercourse resting on essence is an intercourse with the 
spook, not with anything real. If I hold intercourse with the supreme essence, 
I am not holding intercourse with myself, and, if I hold intercourse with the 
essence of man, I am not holding intercourse with men.

The natural man's love becomes through culture a \textit{commandment}. But as 
commandment it belongs to \textit{Man} as such. not to me; it is my 
\textit{essence},\footnote{[\textit{Wesen}]} about which much 
ado\footnote{[\textit{Wesen}]} is made. not my property. \textit{Man}, 
\textit{i.e.} humanity, presents that demand to me; love \textit{is demanded}, 
it is my \textit{duty}. Instead, therefore, of being really won for 
\textit{me}, it has been won for the generality, \textit{Man}, as his property 
or peculiarity: "{}it becomes man, every man, to love; love is the duty and 
calling of man,"{} etc.

Consequently I must again vindicate love for \textit{myself}, and deliver it 
out of the power of Man with the great M.

What was originally \textit{mine}, but \textit{accidentally} mine, 
instinctively mine, I was invested with as the property of Man; I became 
feoffee in loving, I became the retainer of mankind, only a specimen of this 
species, and acted, loving, not as \textit{I}, but as \textit{man}, as a 
specimen of man, the humanly. The whole condition of civilization is the 
\textit{feudal system}, the property being Man's or mankind's, not 
\textit{mine}. A monstrous feudal State was founded, the individual robbed of 
everything, everything left to "{}man."{} The individual had to appear at last 
as a "{}sinner through and through."{}

Am I perchance to have no lively interest in the person of another, are 
\textit{his} joy and \textit{his} weal not to lie at my heart, is the 
enjoyment that I furnish him not to be more to me than other enjoyments of my 
own? On the contrary, I can with joy sacrifice to him numberless enjoyments, I 
can deny myself numberless things for the enhancement of \textit{his} 
pleasure, and I can hazard for him what without him was the dearest to me, my 
life, my welfare, my freedom. Why, it constitutes my pleasure and my happiness 
to refresh myself with his happiness and his pleasure. But \textit{myself, my 
own self}, I do not sacrifice to him, but remain an egoist and -- enjoy him. 
If I sacrifice to him everything that but for my love to him I should keep, 
that is very simple, and even more usual in life than it seems to be; but it 
proves nothing further than that this one passion is more powerful in me than 
all the rest. Christianity too teaches us to sacrifice all other passions to 
this. But, if to one passion I sacrifice others, I do not on that account go 
so far as to sacrifice \textit{myself}, nor sacrifice anything of that whereby 
I truly am myself; I do not sacrifice my peculiar value, my \textit{ownness}. 
Where this bad case occurs, love cuts no better figure than any other passion 
that I obey blindly. The ambitious man, who is carried away by ambition and 
remains deaf to every warning that a calm moment begets in him, has let this 
passion grow up into a despot against whom he abandons all power of 
dissolution: he has given up himself, because he cannot \textit{dissolve} 
himself, and consequently cannot absolve himself from the passion: he is 
possessed.

I love men too -- not merely individuals, but every one. But I love them with 
the consciousness of egoism; I love them because love makes \textit{me} happy, 
I love because loving is natural to me, because it pleases me. I know no 
"{}commandment of love."{} I have a \textit{fellow-feeling} with every feeling 
being, and their torment torments, their refreshment refreshes me too; I can 
kill them, not torture them. \textit{Per contra}, the high-souled, virtuous 
Philistine prince Rudolph in \textit{The Mysteries of Paris}, because the 
wicked provoke his "{}indignation,"{} plans their torture. That fellow-feeling 
proves only that the feeling of those who feel is mine too, my property; in 
opposition to which the pitiless dealing of the "{}righteous"{} man 
(\textit{e. g.} against notary Ferrand) is like the unfeelingness of that 
robber [Procrustes] who cut \textit{off} or stretched his prisoners' legs to 
the measure of his bedstead: Rudolph's bedstead, which he cuts men to fit, is 
the concept of the "{}good."{} The for right, virtue, etc., makes people 
hard-hearted and intolerant. Rudolph does not feel like the notary, but the 
reverse; he feels that "{}it serves the rascal right"{}; that is no 
fellow-feeling.

You love man, therefore you torture the individual man, the egoist; your 
philanthropy (love of men) is the tormenting of men.

If I see the loved one suffer, I suffer with him, and I know no rest till I 
have tried everything to comfort and cheer him; if I see him glad, I too 
become glad over his joy. From this it does not follow that suffering or joy 
is caused in me by the same thing that brings out this effect in him, as is 
sufficiently proved by every bodily pain which I do not feel as he does; his 
tooth pains him, but his pain pains me.

But, because I cannot bear the troubled crease on the beloved forehead, for 
that reason, and therefore for my sake, I kiss it away. If I did not love this 
person, he might go right on making creases, they would not trouble me; I am 
only driving away \textit{my} trouble.

How now, has anybody or anything, whom and which I do not love, a 
\textit{right} to be loved by me? Is my love first, or is his right first? 
Parents, kinsfolk, fatherland, nation, native town, etc., finally fellowmen in 
general ("{}brothers, fraternity"{}), assert that they have a right to my 
love, and lay claim to it without further ceremony. They look upon it as 
\textit{their property}, and upon me, if I do not respect this, as a robber 
who takes from them what pertains to them and is theirs. I \textit{should} 
love. If love is a commandment and law, then I must be educated into it, 
cultivated up to it, and, if I trespass against it, punished. Hence people 
will exercise as strong a "{}moral influence"{} as possible on me to bring me 
to love. And there is no doubt that one can work up and seduce men to love as 
one can to other passions -- if you like, to hate. Hate runs through whole 
races merely because the ancestors of the one belonged to the Guelphs, those 
of the other to the Ghibellines.

But love is not a commandment, but, like each of my feelings, \textit{my 
property. Acquire}, \textit{i.e.} purchase, my property, and then I will make 
it over to you. A church, a nation, a fatherland, a family, etc., that does 
not know how to acquire my love, I need not love; and I fix the purchase price 
of my love quite at my pleasure.

Selfish love is far distant from unselfish, mystical, or romantic love. One 
can love everything possible, not merely men, but an "{}object"{} in general 
(wine, one's fatherland, etc.). Love becomes blind and crazy by a 
\textit{must} taking it out of my power (infatuation), romantic by a 
\textit{should} entering into it, \textit{i.e.} by the "{}objects"{} becoming 
sacred for me, or my becoming bound to it by duty, conscience, oath. Now the 
object no longer exists for me, but I for it.

Love is a possessedness, not as my feeling -- as such I rather keep it in my 
possession as property -- but through the alienness of the object. For 
religious love consists in the commandment to love in the beloved a "{}holy 
one,"{} or to adhere to a holy one; for unselfish love there are objects 
\textit{absolutely lovable} for which my heart is to beat, \textit{e. g.} 
fellow-men, or my wedded mate, kinsfolk, etc. Holy Love loves the holy in the 
beloved, and therefore exerts itself also to make of the beloved more and more 
a holy one (a "{}man"{}).

The beloved is an object that \textit{should} be loved by me. He is not an 
object of my love on account of, because of, or by, my loving him, but is an 
object of love in and of himself. Not I make him an object of love, but he is 
such to begin with; for it is here irrelevant that he has become so by my 
choice, if so it be (as with a \textit{fianc\'ee}, a spouse, etc.), since even 
so he has in any case, as the person once chosen, obtained a "{}right of his 
own to my love,"{} and I, because I have loved him, am under obligation to 
love him forever. He is therefore not an object of \textit{my} love, but of 
love in general: an object that \textit{should} be loved. Love appertains to 
him, is due to him, or is his \textit{right}, while I am under 
\textit{obligation} to love him. My love, \textit{i.e.} the toll of love that 
I pay him, is in truth \textit{his} love, which he only collects from me as 
toll.

Every love to which there clings but the smallest speck of obligation is an 
unselfish love, and, so far as this speck reaches, a possessedness. He who 
believes that he \textit{owes} the object of his love anything loves 
romantically or religiously.

Family love, \textit{e. g.} as it is usually understood as "{}piety,"{} is a 
religious love; love of fatherland, preached as "{}patriotism,"{} likewise. 
All our romantic loves move in the same pattern: everywhere the hypocrisy, or 
rather self-deception, of an "{}unselfish love,"{} an interest in the object 
for the object's sake, not for my sake and mine alone.

Religious or romantic love is distinguished from sensual love by the 
difference of the object indeed, but not by the dependence of the relation to 
it. In the latter regard both are possessedness; but in the former the one 
object is profane, the other sacred. The dominion of the object over me is the 
same in both cases, only that it is one time a sensuous one, the other time a 
spiritual (ghostly) one. My love is my own only when it consists altogether in 
a selfish and egoistic interest, and when consequently the object of my love 
is really \textit{my} object or my property. I owe my property nothing, and 
have no duty to it, as little as I might have a duty to my eye; if 
nevertheless I guard it with the greatest care, I do so on my account.

Antiquity lacked love as little as do Christian times; the god of love is 
older than the God of Love. But the mystical possessedness belongs to the 
moderns.

The possessedness of love lies in the alienation of the object, or in my 
powerlessness as against its alienness and superior power. To the egoist 
nothing is high enough for him to humble himself before it, nothing so 
independent that he would live for love of it, nothing so sacred that he would 
sacrifice himself to it. The egoist's love rises in selfishness, flows in the 
bed of selfishness, and empties into selfishness again.

Whether this can still be called love? If you know another word for it, go 
ahead and choose it; then the sweet word love may wither with the departed 
world; for the present I at least find none in our \textit{Christian} 
language, and hence stick to the old sound and "{}love"{} \textit{my} object, 
my -- property.

Only as one of my feelings do I harbor love; but as a power above me, as a 
divine power, as Feuerbach says, as a passion that I am not to cast off, as a 
religious and moral duty, I -- scorn it. As my feeling it is \textit{mine;} as 
a principle to which I consecrate and "{}vow"{} my soul it is a dominator and 
\textit{divine}, just as hatred as a principle \textit{is diabolical;} one not 
better than the other. In short, egoistic love, \textit{i.e.} my love, is 
neither holy nor unholy, neither divine nor diabolical.

"{}A love that is limited by faith is an untrue love. The sole limitation that 
does not contradict the essence of love is the self-limitation of love by 
reason, intelligence. Love that scorns the rigor, the law, of intelligence, is 
theoretically a false love, practically a ruinous one."{}\footnote{Feuerbach, 
"{}Essence of Chr.,"{} 394.} So love is in its essence \textit{rational!} So 
thinks Feuerbach; the believer, on the contrary, thinks, Love is in its 
essence \textit{believing}. The one inveighs against \textit{irrational}, the 
other against \textit{unbelieving}, love. To both it can at most rank as a 
\textit{splendidum vitium}. Do not both leave love standing, even in the form 
of unreason and unbelief? They do not dare to say, irrational or unbelieving 
love is nonsense, is not love; as little as they are willing to say, 
irrational or unbelieving tears are not tears. But, if even irrational love, 
etc., must count as love, and if they are nevertheless to be unworthy of man, 
there follows simply this: love is not the highest thing, but reason or faith; 
even the unreasonable and the unbelieving can love; but love has value only 
when it is that of a rational or believing person. It is an illusion when 
Feuerbach calls the rationality of love its "{}self-limitation"{}; the 
believer might with the same right call belief its "{}self-limitation."{} 
Irrational love is neither "{}false"{} nor "{}ruinous"{}; its does its service 
as love.

Toward the world, especially toward men, I am to \textit{assume a particular 
feeling}, and "{}meet them with love,"{} with the feeling of love, from the 
beginning. Certainly, in this there is revealed far more free-will and 
self-determination than when I let myself be stormed, by way of the world, by 
all possible feelings, and remain exposed to the most checkered, most 
accidental impressions. I go to the world rather with a preconceived feeling, 
as if it were a prejudice and a preconceived opinion; I have prescribed to 
myself in advance my behavior toward it, and, despite all its temptations, 
feel and think about it only as I have once determined to. Against the 
dominion of the world I secure myself by the principle of love; for, whatever 
may come, I -- love. The ugly -- \textit{e. g.} --makes a repulsive impression 
on me; but, determined to love, I master this impression as I do every 
antipathy.

But the feeling to which I have determined and -- condemned myself from the 
start is a \textit{narrow} feeling, because it is a predestined one, of which 
I myself am not able to get clear or to declare myself clear. Because 
preconceived, it is a \textit{prejudice. I} no longer show myself in face of 
the world, but my love shows itself. The \textit{world} indeed does not rule 
me, but so much the more inevitably does the spirit of \textit{love} rule this 
spirit.

If I first said, I love the world, I now add likewise: I do not love it, for I 
\textit{annihilate} it as I annihilate myself; I \textit{dissolve it}. I do 
not limit myself to one feeling for men, but give free play to all that I am 
capable of. Why should I not dare speak it out in all its glaringness? Yes, I 
\textit{utilize} the world and men! With this I can keep myself open to every 
impression without being torn away from myself by one of them. I can love, 
love with a full heart, and let the most consuming glow of passion burn in my 
heart, without taking the beloved one for anything else than the 
\textit{nourishment} of my passion, on which it ever refreshes itself anew. 
All my care for him applies only to the \textit{object of my love}, only to 
him whom my love \textit{requires}, only to him, the "{}warmly loved."{} How 
indifferent would he be to me without this -- my love! I feed only my love 
with him, I \textit{utilize} him for this only: I \textit{enjoy} him.

Let us choose another convenient example. I see how men are fretted in dark 
superstition by a swarm of ghosts. If to the extent of my powers I let a bit 
of daylight fall in on the nocturnal spookery, is it perchance because love to 
you inspires this in me? Do I write out of love to men? No, I write because I 
want to procure for \textit{my} thoughts an existence in the world; and, even 
if I foresaw that these thoughts would deprive you of your rest and your 
peace, even if I saw the bloodiest wars and the fall of many generations 
springing up from this seed of thought -- I would nevertheless scatter it. Do 
with it what you will and can, that is your affair and does not trouble me. 
You will perhaps have only trouble, combat, and death from it, very few will 
draw joy from it. If your weal lay at my heart, I should act as the church did 
in withholding the Bible from the laity, or Christian governments, which make 
it a sacred duty for themselves to "{}protect the common people from bad 
books."{}

But not only not for your sake, not even for truth's sake either do I speak 
out what I think. No --

\begin{quotation}

\noindent{}I sing as the bird sings\\
 That on the bough alights;\\
 The song that from me springs\\
 Is pay that well requites.\end{quotation}

\noindent{}I sing because -- I am a singer. But I 
\textit{use}\footnote{[\textit{gebrauche}]} you for it because I -- 
need\footnote{[\textit{brauche}]} ears.

Where the world comes in my way -- and it comes in my way everywhere -- I 
consume it to quiet the hunger of my egoism. For me you are nothing but --my 
food, even as I too am fed upon and turned to use by you. We have only one 
relation to each other, that of \textit{usableness}, of utility, of use. We 
owe \textit{each other} nothing, for what I seem to owe you I owe at most to 
myself. If I show you a cheery air in order to cheer you likewise, then your 
cheeriness is of consequence to \textit{me}, and my air serves \textit{my} 
wish; to a thousand others, whom I do not aim to cheer, I do not show it.

\begin{center}
--------\end{center}


One has to be educated up to that love which founds itself on the "{}essence 
of man"{} or, in the ecclesiastical and moral period, lies upon us as a 
"{}commandment."{} In what fashion moral influence, the chief ingredient of 
our education, seeks to regulate the intercourse of men shall here be looked 
at with egoistic eyes in one example at least.

Those who educate us make it their concern early to break us of lying and to 
inculcate the principle that one must always tell the truth. If selfishness 
were made the basis for this rule, every one would easily understand how by 
lying he fools away that confidence in him which he hopes to awaken in others, 
and how correct the maxim proves, Nobody believes a liar even when he tells 
the truth. Yet, at the same time, he would also feel that he had to meet with 
truth only him whom \textit{he} authorized to hear the truth. If a spy walks 
in disguise through the hostile camp, and is asked who he is, the askers are 
assuredly entitled to inquire after his name, but the disguised man does not 
give them the right to learn the truth from him; he tells them what he likes, 
only not the fact. And yet morality demands, "{}Thou shalt not lie!"{} By 
morality those persons are vested with the right to expect the truth; but by 
me they are not vested with that right, and I recognize only the right that 
\textit{I} impart. In a gathering of revolutionists the police force their way 
in and ask the orator for his name; everybody knows that the police have the 
right to do so, but they do not have it from the \textit{revolutionist}, since 
he is their enemy; he tells them a false name and --cheats them with a lie. 
The police do not act so foolishly either as to count on their enemies' love 
of truth; on the contrary, they do not believe without further ceremony, but 
have the questioned individual "{}identified"{} if they can. Nay, the State -- 
everywhere proceeds incredulously with individuals, because in their egoism it 
recognizes its natural enemy; it invariably demands a "{}voucher,"{} and he 
who cannot show vouchers falls a prey to its investigating inquisition. The 
State does not believe nor trust the individual, and so of itself places 
itself with him in the \textit{convention of lying}; it trusts me only when it 
has \textit{convinced} itself of the truth of my statement, for which there 
often remains to it no other means than the oath. How clearly, too, this (the 
oath) proves that the State does not count on our credibility and love of 
truth, but on our \textit{interest}, our selfishness: it relies on our not 
wanting to fall foul of God by a perjury.

Now, let one imagine a French revolutionist in the year 1788, who among 
friends let fall the now well-known phrase, "{}the world will have no rest 
till the last king is hanged with the guts of the last priest."{} The king 
then still had all power, and, when the utterance is betrayed by an accident, 
yet without its being possible to produce witnesses, confession is demanded 
from the accused. Is he to confess or not?

If he denies, he lies and -- remains unpunished; if he confesses, he is candid 
and -- is beheaded. If truth is more than everything else to him, all right, 
let him die. Only a paltry poet could try to make a tragedy out of the end of 
his life; for what interest is there in seeing how a man succumbs from 
cowardice? But, if he had the courage not to be a slave of truth and 
sincerity, he would ask somewhat thus: Why need the judges know what I have 
spoken among friends? If I had \textit{wished} them to know, I should have 
said it to them as I said it to my friends. I will not have them know it. They 
force themselves into my confidence without my having called them to it and 
made them my confidants; they \textit{will} learn what I \textit{will} keep 
secret. Come on then, you who wish to break my will by your will, and try your 
arts. You can torture me by the rack, you can threaten me with hell and 
eternal damnation, you can make me so nerveless that I swear a false oath, but 
the truth you shall not press out of me, for I \textit{will} lie to you 
because I have given you no claim and no right to my sincerity. Let God, 
"{}who is truth,"{} look down ever so threateningly on me, let lying come ever 
so hard to me, I have nevertheless the courage of a lie; and, even if I were 
weary of my life, even if nothing appeared to me more welcome than your 
executioner's sword, you nevertheless should not have the joy of finding in me 
a slave of truth, whom by your priestly arts you make a traitor to his 
\textit{will}. When I spoke those treasonable words, I would not have had you 
know anything of them; I now retain the same will, and do not let myself be 
frightened by the curse of the lie.

Sigismund is not a miserable caitiff because he broke his princely word, but 
he broke the word because he was a caitiff; he might have kept his word and 
would still have been a caitiff, a priest-ridden man. Luther, driven by a 
higher power, became unfaithful to his monastic vow: he became so for God's 
sake. Both broke their oath as possessed persons: Sigismund, because he wanted 
to appear as a \textit{sincere} professor of the divine \textit{truth}, 
\textit{i. e.}, of the true, genuinely Catholic faith; Luther, in order to 
give testimony for the gospel \textit{sincerely} and with entire truth. with 
body and soul; both became perjured in order to be sincere toward the 
"{}higher truth."{} Only, the priests absolved the one, the other absolved 
himself. What else did both observe than what is contained in those apostolic 
words, "{}Thou hast not lied to men, but to God?"{} They lied to men, broke 
their oath before the world's eyes, in order not to lie to God, but to serve 
him. Thus they show us a way to deal with truth before men. For God's glory, 
and for God's sake, a -- breach of oath, a lie, a prince's word broken!

How would it be, now, if we changed the thing a little and wrote, A perjury 
and lie for -- \textit{my sake?} Would not that be pleading for every 
baseness? It seems so, assuredly, only in this it is altogether like the 
"{}for God's sake."{} For was not every baseness committed for God's sake, 
were not all the scaffolds filled for his sake and all the 
\textit{autos-da-f\'e} held for his sake, was not all stupefaction introduced 
for his sake? And do they not today still for God's sake fetter the mind in 
tender children by religious education? Were not sacred vows broken for his 
sake, and do not missionaries and priests still go around every day to bring 
Jews, heathen, Protestants or Catholics, to treason against the faith of their 
fathers -- for his sake? And that should be worse with the \textit{for my 
sake?} What then does \textit{on my account} mean? There people immediately 
think of \textit{"{}filthy lucre"{}}. But he who acts from love of filthy 
lucre does it on his own account indeed, as there is nothing anyhow that one 
does not do for his own sake -- among other things, everything that is done 
for God's glory; yet he, for whom he seeks the lucre, is a slave of lucre, not 
raised above lucre; he is one who belongs to lucre, the money-bag, not to 
himself; he is not his own. Must not a man whom the passion of avarice rules 
follow the commands of this \textit{master?} And, if a weak goodnaturedness 
once beguiles him, does this not appear as simply an exceptional case of 
precisely the same sort as when pious believers are sometimes forsaken by 
their Lord's guidance and ensnared by the arts of the "{}devil?"{} So an 
avaricious man is not a self-owned man, but a servant; and he can do nothing 
for his own sake without at the same time doing it for his lord's sake -- 
precisely like the godly man.

Famous is the breach of oath which Francis I committed against Emperor Charles 
V. Not later, when he ripely weighed his promise, but at once, when he swore 
the oath, King Francis took it back in thought as well as by a secret 
protestation documentarily subscribed before his councillors; he uttered a 
perjury aforethought. Francis did not show himself disinclined to buy his 
release, but the price that Charles put on it seemed to him too high and 
unreasonable. Even though Charles behaved himself in a sordid fashion when he 
sought to extort as much as possible, it was yet shabby of Francis to want to 
purchase his freedom for a lower ransom; and his later dealings, among which 
there occurs yet a second breach of his word, prove sufficiently how the 
huckster spirit held him enthralled and made him a shabby swindler. However, 
what shall we say to the reproach of perjury against him? In the first place, 
surely, this again: that not the perjury, but his sordidness, shamed him; that 
he did not deserve contempt for his perjury, but made himself guilty of 
perjury because he was a contemptible man. But Francis's perjury, regarded in 
itself, demands another judgment. One might say Francis did not respond to the 
confidence that Charles put in him in setting him free. But, if Charles had 
really favored him with confidence, he would have named to him the price that 
he considered the release worth, and would then have set him at liberty and 
expected Francis to pay the redemption-sum. Charles harbored no such trust, 
but only believed in Francis's impotence and credulity, which would not allow 
him to act against his oath; but Francis deceived only this -- credulous 
calculation. When Charles believed he was assuring himself of his enemy by an 
oath, right there he was freeing him from every obligation. Charles had given 
the king credit for a piece of stupidity, a narrow conscience, and, without 
confidence in Francis, counted only on Francis's stupidity, \textit{e. g.}, 
conscientiousness: he let him go from the Madrid prison only to hold him the 
more securely in the prison of conscientiousness, the great jail built about 
the mind of man by religion: he sent him back to France locked fast in 
invisible chains, what wonder if Francis sought to escape and sawed the chains 
apart? No man would have taken it amiss of him if he had secretly fled from 
Madrid, for he was in an enemy's power; but every good Christian cries out 
upon him, that he wanted to loose himself from God's bonds too. (It was only 
later that the pope absolved him from his oath.)

It is despicable to deceive a confidence that we voluntarily call forth; but 
it is no shame to egoism to let every one who wants to get us into his power 
by an oath bleed to death by the failure of his untrustful craft. If you have 
wanted to bind me, then learn that I know how to burst your bonds.

The point is whether I give the confider the right to confidence. If the 
pursuer of my friend asks me where he has fled to, I shall surely put him on a 
false trail. Why does he ask precisely me, the pursued man's friend? In order 
not to be a false, traitorous friend, I prefer to be false to the enemy. I 
might certainly in courageous conscientiousness, answer, "{}I will not tell"{} 
(so Fichte decides the case); by that I should salve my love of truth and do 
for my friend as much as -- nothing, for, if I do not mislead the enemy, he 
may accidentally take the right street, and my love of truth would have given 
up my friend as a prey, because it hindered me from the --courage for a lie. 
He who has in the truth an idol, a sacred thing, must \textit{humble} himself 
before it, must not defy its demands, not resist courageously; in short, he 
must renounce the \textit{heroism of the lie}. For to the lie belongs not less 
courage than to the truth: a courage that young men are most apt to be 
defective in, who would rather confess the truth and mount the scaffold for it 
than confound the enemy's power by the impudence of a lie. To them the truth 
is "{}sacred,"{} and the sacred at all times demands blind reverence, 
submission, and self-sacrifice. If you are not impudent, not mockers of the 
sacred, you are tame and its servants. Let one but lay a grain of truth in the 
trap for you, you peck at it to a certainty, and the fool is caught. You will 
not lie? Well, then, fall as sacrifices to the truth and become -- martyrs! 
Martyrs! -- for what? For yourselves, for self-ownership? No, for your goddess 
-- the truth. You know only two \textit{services}, only two kinds of servants: 
servants of the truth and servants of the lie. Then in God's name serve the 
truth!

Others, again, serve the truth also; but they serve it "{}in moderation,"{} 
and make, \textit{e. g.} a great distinction between a simple lie and a lie 
sworn to. And yet the whole chapter of the oath coincides with that of the 
lie, since an oath, everybody knows, is only a strongly assured statement. You 
consider yourselves entitled to lie, if only you do not swear to it besides? 
One who is particular about it must judge and condemn a lie as sharply as a 
false oath. But now there has been kept up in morality an ancient point of 
controversy, which is customarily treated of under the name of the "{}lie of 
necessity."{} No one who dares plead for this can consistently put from him an 
"{}oath of necessity."{} If I justify my lie as a lie of necessity, I should 
not be so pusillanimous as to rob the justified lie of the strongest 
corroboration. Whatever I do, why should I not do it entirely and without 
reservations (\textit{reservatio mentalis})? If I once lie, why then not lie 
completely, with entire consciousness and all my might? As a spy I should have 
to swear to each of my false statements at the enemy's demand; determined to 
lie to him, should I suddenly become cowardly and undecided in face of an 
oath? Then I should have been ruined in advance for a liar and spy; for, you 
see, I should be voluntarily putting into the enemy's hands a means to catch 
me. -- The State too fears the oath of necessity, and for this reason does not 
give the accused a chance to swear. But you do not justify the State's fear; 
you lie, but do not swear falsely. If, \textit{e. g.} you show some one a 
kindness, and he is not to know it, but he guesses it and tells you so to your 
face, you deny; if he insists, you say, "{}honestly, no!"{} If it came to 
swearing, then you would refuse; for, from fear of the sacred, you always stop 
half way. \textit{Against} the sacred you have no \textit{will of your own}. 
You lie in -- moderation, as you are free "{}in moderation,"{} religious "{}in 
moderation"{} (the clergy are not to "{}encroach"{}; over this point the most 
rapid of controversies is now being carried on, on the part of the university 
against the church), monarchically disposed "{}in moderation"{} (you want a 
monarch limited by the constitution, by a fundamental law of the State), 
everything nicely \textit{tempered}, lukewarm, half God's, half the devil's.

There was a university where the usage was that every word of honor that must 
be given to the university judge was looked upon by the students as null and 
void. For the students saw in the demanding of it nothing but a snare, which 
they could not escape otherwise than by taking away all its significance. He 
who at that same university broke his word of honor to one of the fellows was 
infamous; he who gave it to the university judge derided, in union with these 
very fellows, the dupe who fancied that a word had the same value among 
friends and among foes. It was less a correct theory than the constraint of 
practice that had there taught the students to act so, as, without that means 
of getting out, they would have been pitilessly driven to treachery against 
their comrades. But, as the means approved itself in practice, so it has its 
theoretical probation too. A word of honor, an oath, is one only for him whom 
I entitle to receive it; he who forces me to it obtains only a forced, 
\textit{i.e.} a \textit{hostile} word, the word of a foe, whom one has no 
right to trust; for the foe does not give us the right.

Aside from this, the courts of the State do not even recognize the 
inviolability of an oath. For, if I had sworn to one who comes under 
examination that I would not declare anything against him, the court would 
demand my declaration in spite of the fact that an oath binds me, and, in case 
of refusal, would lock me up till I decided to become -- an oath-breaker. The 
court "{}absolves me from my oath"{}; -- how magnanimous! If any power can 
absolve me from the oath, I myself am surely the very first power that has a 
claim to.

As a curiosity, and to remind us of customary oaths of all sorts, let place be 
given here to that which Emperor Paul commanded the captured Poles 
(Kosciuszko, Potocki, Niemcewicz, and others) to take when he released them: 
"{}We not merely swear fidelity and obedience to the emperor, but also further 
promise to pour out our blood for his glory; we obligate ourselves to discover 
everything threatening to his person or his empire that we ever learn; we 
declare finally that, in whatever part of the earth we may be, a single word 
of the emperor shall suffice to make us leave everything and repair to him at 
once."{}

\begin{center}
--------\end{center}


In one domain the principle of love seems to have been long outsoared by 
egoism, and to be still in need only of sure consciousness, as it were of 
victory with a good conscience. This domain is speculation, in its double 
manifestation as thinking and as trade. One thinks with a will, whatever may 
come of it; one speculates, however many may suffer under our speculative 
undertakings. But, when it finally becomes serious, when even the last remnant 
of religiousness, romance, or "{}humanity"{} is to be done away, then the 
pulse of religious conscience beats, and one at least \textit{professes} 
humanity. The avaricious speculator throws some coppers into the poor-box and 
"{}does good,"{} the bold thinker consoles himself with the fact that he is 
working for the advancement of the human race and that his devastation 
"{}turns to the good"{} of mankind, or, in another case, that he is "{}serving 
the idea"{}; mankind, the idea, is to him that something of which he must say, 
It is more to me than myself.

To this day thinking and trading have been done for -- God's sake. Those who 
for six days were trampling down everything by their selfish aims sacrificed 
on the seventh to the Lord; and those who destroyed a hundred "{}good 
causes"{} by their reckless thinking still did this in the service of another 
"{}good cause,"{} and had yet to think of another -- besides themselves -- to 
whose good their self-indulgence should turn; of the people, mankind, etc. But 
this other thing is a being above them, a higher or supreme being; and 
therefore I say, they are toiling for God's sake.

Hence I can also say that the ultimate basis of their actions is -- love. Not 
a voluntary love however, not their own, but a tributary love, or the higher 
being's own (God's, who himself is love); in short, not the egoistic, but the 
religious; a love that springs from their fancy that they \textit{must} 
discharge a tribute of love, \textit{i.e.} that they must not be 
"{}egoists."{}

If \textit{we} want to deliver the world from many kinds of unfreedom, we want 
this not on its account but on ours; for, as we are not world-liberators by 
profession and out of "{}love,"{} we only want to win it away from others. We 
want to make it \textit{our} own; it is not to be any longer \textit{owned as 
serf} by God (the church) nor by the law (State), but to be \textit{our own}; 
therefore we seek to "{}win"{} it, to "{}captivate"{} it, and, by meeting it 
halfway and "{}devoting"{} ourselves to it as to ourselves as soon as it 
belongs to us, to complete and make superfluous the force that it turns 
against us. If the world is ours, it no longer attempts any force 
\textit{against} us, but only \textit{with us}. My selfishness has an interest 
in the liberation of the world, that it may become -- my property.

Not isolation or being alone, but society, is man's original state. Our 
existence begins with the most intimate conjunction, as we are already living 
with our mother before we breathe; when we see the light of the world, we at 
once lie on a human being's breast again, her love cradles us in the lap, 
leads us in the go-cart, and chains us to her person with a thousand ties. 
Society is our \textit{state of nature}. And this is why, the more we learn to 
feel ourselves, the connection that was formerly most intimate becomes ever 
looser and the dissolution of the original society more unmistakable. To have 
once again for herself the child that once lay under her heart, the mother 
must fetch it from the street and from the midst of its playmates. The child 
prefers the \textit{intercourse} that it enters into with \textit{its fellows} 
to the \textit{society} that it has not entered into, but only been born in.

But the dissolution of \textit{society} is \textit{intercourse} or 
\textit{union}. A society does assuredly arise by union too, but only as a 
fixed idea arises by a thought -- to wit, by the vanishing of the energy of 
the thought (the thinking itself, this restless taking back all thoughts that 
make themselves fast) from the thought. If a union\footnote{[\textit{Verein}]} 
has crystallized into a society, it has ceased to be a 
coalition;\footnote{[\textit{Vereinigung}]} for coalition is an incessant 
self-uniting; it has become a unitedness, come to a standstill, degenerated 
into a fixity; it is -- \textit{dead} as a union, it is the corpse of the 
union or the coalition, \textit{i.e.} it is --society, community. A striking 
example of this kind is furnished by the \textit{party}.

That a society (\textit{e. g.} the society of the State) diminishes my 
\textit{liberty} offends me little. Why, I have to let my liberty be limited 
by all sorts of powers and by every one who is stronger; nay, by every 
fellow-man; and, were I the autocrat of all the R......, I yet should not 
enjoy absolute liberty. But \textit{ownness} I will not have taken from me. 
And ownness is precisely what every society has designs on, precisely what is 
to succumb to its power.

A society which I join does indeed take from me many liberties, but in return 
it affords me other liberties; neither does it matter if I myself deprive 
myself of this and that liberty (\textit{e. g.} by any contract). On the other 
hand, I want to hold jealously to my ownness. Every community has the 
propensity, stronger or weaker according to the fullness of its power, to 
become an \textit{authority} to its members and to set \textit{limits} for 
them: it asks, and must ask, for a "{}subject's limited understanding"{}; it 
asks that those who belong to it be subjected to it, be its "{}subjects"{}; it 
exists only by \textit{subjection}. In this a certain tolerance need by no 
means be excluded; on the contrary, the society will welcome improvements, 
corrections, and blame, so far as such are calculated for its gain: but the 
blame must be "{}well-meaning,"{} it may not be "{}insolent and 
disrespectful"{} -- in other words, one must leave uninjured, and hold sacred, 
the substance of the society. The society demands that those who belong to it 
shall not \textit{go beyond it} and exalt themselves, but remain "{}within the 
bounds of legality,"{} \textit{e. g.}, allow themselves only so much as the 
society and its law allow them.

There is a difference whether my liberty or my ownness is limited by a 
society. If the former only is the case, it is a coalition, an agreement, a 
union; but, if ruin is threatened to ownness, it is \textit{a power of 
itself}, a power \textit{above me}, a thing unattainable by me, which I can 
indeed admire, adore, reverence, respect, but cannot subdue and consume, and 
that for the reason that I \textit{am resigned}. It exists by my 
\textit{resignation}, my \textit{self-renunciation}, my 
spiritlessness,\footnote{[\textit{Muthl\"osigkeit}]} called --

HUMILITY.\footnote{[\textit{Demuth}]} My humility makes its 
courage,\footnote{[\textit{Muth}]} my submissiveness gives it its dominion.

 But in reference to \textit{liberty}, State and union are subject to no 
essential difference. The latter can just as little come into existence, or 
continue in existence, without liberty's being limited in all sorts of ways, 
as the State is compatible with unmeasured liberty. Limitation of liberty is 
inevitable everywhere, for one cannot get \textit{rid} of everything; one 
cannot fly like a bird merely because one would like to fly so, for one does 
not get free from his own weight; one cannot live under water as long as he 
likes, like a fish, because one cannot do without air and cannot get free from 
this indispensable necessity; etc. As religion, and most decidedly 
Christianity, tormented man with the demand to realize the unnatural and self- 
contradictory, so it is to be looked upon only as the true logical outcome of 
that religious over-straining and overwroughtness that finally \textit{liberty 
itself, absolute liberty}, was exalted into an ideal, and thus the nonsense of 
the impossible to come glaringly to the light. -- The union will assuredly 
offer a greater measure of liberty, as well as (and especially because by it 
one escapes all the coercion peculiar to State and society life) admit of 
being considered as "{}a new liberty"{}; but nevertheless it will still 
contain enough of unfreedom and involuntariness. For its object is not this -- 
liberty (which on the contrary it sacrifices to ownness), but only 
\textit{ownness}. Referred to this, the difference between State and union is 
great enough. The former is an enemy and murderer of \textit{ownness}, the 
latter a son and co-worker of it; the former a spirit that would be adored in 
spirit and in truth, the latter my work, my product ; the State is the lord of 
my spirit, who demands faith and prescribes to me articles of faith, the creed 
of legality; it exerts moral influence, dominates my spirit, drives away my 
ego to put itself in its place as "{}my true ego"{} -- in short, the State is 
sacred, and as against me, the individual man, it is the true man, the spirit, 
the ghost; but the union is my own creation, my creature, not sacred, not a 
spiritual power above my spirit, as little as any association of whatever 
sort. As I am not willing to be a slave of my maxims, but lay them bare to my 
continual criticism without \textit{any warrant}, and admit no bail at all for 
their persistence, so still less do I obligate myself to the union for my 
future and pledge my soul to it, as is said to be done with the devil, and is 
really the case with the State and all spiritual authority; but I am and 
remain \textit{more} to myself than State, Church, God, etc.; consequently 
infinitely more than the union too.

That society which Communism wants to found seems to stand nearest to 
\textit{coalition}. For it is to aim at the "{}welfare of all,"{} oh, yes, of 
all, cries Weitling innumerable times, of all! That does really look as if in 
it no one needed to take a back seat. But what then will this welfare be? Have 
all one and the same welfare, are all equally well off with one and the same 
thing? If that be so, the question is of the "{}true welfare."{} Do we not 
with this come right to the point where religion begins its dominion of 
violence? Christianity says, Look not on earthly toys, but seek your true 
welfare, become -- pious Christians; being Christians is the true welfare. It 
is the true welfare of "{}all,"{} because it is the welfare of Man as such 
(this spook). Now, the welfare of all is surely to be \textit{your} and 
\textit{my} welfare too? But, if you and I do not look upon that welfare as 
\textit{our} welfare, will care then be taken for that in which \textit{we} 
feel well? On the contrary, society has decreed a welfare as the "{}true 
welfare,"{} if this welfare were called \textit{e. g.} "{}enjoyment honestly 
worked for"{}; but if you preferred enjoyable laziness, enjoyment without 
work, then society, which cares for the "{}welfare of all,"{} would wisely 
avoid caring for that in which you are well off. Communism, in proclaiming the 
welfare of all, annuls outright the well-being of those who hitherto lived on 
their income from investments and apparently felt better in that than in the 
prospect of Weitling's strict hours of labor. Hence the latter asserts that 
with the welfare of thousands the welfare of millions cannot exist, and the 
former must give up \textit{their} special welfare "{}for the sake of the 
general welfare."{} No, let people not be summoned to sacrifice their special 
welfare for the general, for this Christian admonition will not carry you 
through; they will better understand the opposite admonition, not to let their 
\textit{own} welfare be snatched from them by anybody, but to put it on a 
permanent foundation. Then they are of themselves led to the point that they 
care best for their welfare if they \textit{unite} with others for this 
purpose, \textit{e. g.}, "{}sacrifice a part of their liberty,"{} yet not to 
the welfare of others, but to their own. An appeal to men's self-sacrificing 
disposition end self- renouncing love ought at least to have lost its 
seductive plausibility when, after an activity of thousands of years, it has 
left nothing behind but the -- \textit{mis\`ere} of today. Why then still 
fruitlessly expect self-sacrifice to bring us better time? Why not rather hope 
for them from \textit{usurpation?} Salvation comes no longer from the giver, 
the bestower, the loving one, but from the \textit{taker}, the appropriator 
(usurper), the owner. Communism, and, consciously, egoism-reviling humanism, 
still count on \textit{love}.

If community is once a need of man, and he finds himself furthered by it in 
his aims, then very soon, because it has become his principle, it prescribes 
to him its laws too, the laws of -- society. The principle of men exalts 
itself into a sovereign power over them, becomes their supreme essence, their 
God, and, as such -- law-giver. Communism gives this principle the strictest 
effect, and Christianity is the religion of society, for, as Feuerbach rightly 
says, although he does not mean it rightly, love is the essence of man; 
\textit{e. g.}, the essence of society or of societary (Communistic) man. All 
religion is a cult of society, this principle by which societary (cultivated) 
man is dominated; neither is any god an ego's exclusive god, but always a 
society's or community's, be it of the society, "{}family"{} (Lar, Penates) or 
of a "{}people"{} ("{}national god"{}) or of "{}all men"{} ("{}he is a Father 
of all men"{}).

Consequently one has a prospect of extirpating religion down to the ground 
only when one antiquates \textit{society} and everything that flows from this 
principle. But it is precisely in Communism that this principle seeks to 
culminate, as in it everything is to become \textit{common} for the 
establishment of -- "{}equality."{} If this "{}equality"{} is won, 
"{}liberty"{} too is not lacking. But whose liberty? \textit{Society's}! 
Society is then all in all, and men are only "{}for each other."{} It would be 
the glory of the -- love-State.

But I would rather be referred to men's selfishness than to their 
"{}kindnesses,"{}\footnote{[Literally, "{}love-services."{}]} their mercy, 
pity, etc. The former demands \textit{reciprocity} (as thou to me, so I to 
thee), does nothing "{}gratis,"{} and may be won and -- \textit{bought}. But 
with what shall I obtain the kindness? It is a matter of chance whether I am 
at the time having to do with a "{}loving"{} person. The affectionate one's 
service can be had only by -- \textit{begging}, be it by my lamentable 
appearance, by my need of help, my misery, my -- \textit{suffering}. What can 
I offer him for his assistance? Nothing! I must accept it as a --present. Love 
is \textit{unpayable}, or rather, love can assuredly be paid for, but only by 
counter-love ("{}One good turn deserves another"{}). What paltriness and 
beggarliness does it not take to accept gifts year in and year out without 
service in return, as they are regularly collected \textit{e. g.} from the 
poor day-laborer? What can the receiver do for him and his donated pennies, in 
which his wealth consists? The day- laborer would really have more enjoyment 
if the receiver with his laws, his institutions, etc., all of which the 
day-laborer has to pay for though, did not exist at all. And yet, with it all, 
the poor wight \textit{loves} his master.

No, community, as the "{}goal"{} of history hitherto, is impossible. Let us 
rather renounce every hypocrisy of community, and recognize that, if we are 
equal as men, we are not equal for the very reason that we are not men. We are 
equal \textit{only in thoughts}, only when "{}we"{} are \textit{thought}, not 
as we really and bodily are. I am ego, and you are ego: but I am not this 
thought-of ego; this ego in which we are all equal is only my 
\textit{thought}. I am man, and you are man: but "{}man"{} is only a thought, 
a generality; neither I nor you are speakable, we are \textit{unutterable}, 
because only \textit{thoughts} are speakable and consist in speaking.

Let us therefore not aspire to community, but to \textit{one-sidedness}. Let 
us not seek the most comprehensive commune, "{}human society,"{} but let us 
seek in others only means and organs which we may use as our property! As we 
do not see our equals in the tree, the beast, so the presupposition that 
others are \textit{our equals} springs from a hypocrisy. No one is \textit{my 
equal}, but I regard him, equally with all other beings, as my property. In 
opposition to this I am told that I should be a man among "{}fellow-men"{} 
(\textit{Judenfrage}, p. 60); I should "{}respect"{} the fellow-man in them. 
For me no one is a person to be respected, not even the fellow-man, but 
solely, like other beings, an \textit{object} in which I take an interest or 
else do not, an interesting or uninteresting object, a usable or unusable 
person.

And, if I can use him, I doubtless come to an understanding and make myself at 
one with him, in order, by the agreement, to strengthen \textit{my power}, and 
by combined force to accomplish more than individual force could effect. In 
this combination I see nothing whatever but a multiplication of my force, and 
I retain it only so long as it is my multiplied force. But thus it is a -- 
union.

Neither a natural ligature nor a spiritual one holds the union together, and 
it is not a natural, not a spiritual league. It is not brought about by one 
\textit{blood}, not by one \textit{faith} (spirit). In a natural league -- 
like a family, a tribe, a nation, yes, mankind -- the individuals have only 
the value of \textit{specimens} of the same species or genus; in a spiritual 
league -- like a commune, a church -- the individual signifies only a 
\textit{member} of the same spirit; what you are in both cases as a unique 
person must be -- suppressed. Only in the union can you assert yourself as 
unique, because the union does not possess you, but you possess it or make it 
of use to you.

Property is recognized in the union, and only in the union, because one no 
longer holds what is his as a fief from any being. The Communists are only 
consistently carrying further what had already been long present during 
religious evolution, and especially in the State; to wit, propertylessness, 
the feudal system.

The State exerts itself to tame the desirous man; in other words, it seeks to 
direct his desire to it alone, and to \textit{content} that desire with what 
it offers. To sate the desire for the desirous man's sake does not come into 
the mind: on the contrary, it stigmatizes as an "{}egoistic man"{} the man who 
breathes out unbridled desire, and the "{}egoistic man"{} is its enemy. He is 
this for it because the capacity to agree with him is wanting to the State; 
the egoist is precisely what it cannot "{}comprehend."{} Since the State (as 
nothing else is possible) has to do only for itself, it does not take care for 
my needs, but takes care only of how it make away with me, \textit{i.e.} make 
out of me another ego, a good citizen. It takes measures for the 
"{}improvement of morals."{} -- And with what does it win individuals for 
itself? With itself, \textit{i.e.} with what is the State's, with 
\textit{State property}. It will be unremittingly active in making all 
participants in its "{}goods,"{} providing all with the "{}good things of 
culture"{}; it presents them its education, opens to them the access to its 
institutions of culture, capacitates them to come to property (\textit{i.e.} 
to a fief) in the way of industry, etc. For all these \textit{fiefs} it 
demands only the just rent of continual \textit{thanks}. But the 
"{}unthankful"{} forget to pay these thanks. -- Now, neither can "{}society"{} 
do essentially otherwise than the State.

You bring into a union your whole power, your competence, and \textit{make 
yourself count}; in a society you are \textit{employed}, with your working 
power; in the former you live egoistically, in the latter humanly, 
\textit{i.e.} religiously, as a "{}member in the body of this Lord"{}; to a 
society you owe what you have, and are in duty bound to it, are -- possessed 
by "{}social duties"{}; a union you utilize, and give it up undutifully and 
unfaithfully when you see no way to use it further. If a society is more than 
you, then it is more to you than yourself; a union is only your instrument, or 
the sword with which you sharpen and increase your natural force; the union 
exists for you and through you, the society conversely lays claim to you for 
itself and exists even without you, in short, the society is \textit{sacred}, 
the union your \textit{own}; consumes \textit{you, you} consume the union.

Nevertheless people will not be backward with the objection that the agreement 
which has been concluded may again become burdensome to us and limit our 
freedom; they will say, we too would at last come to this, that "{}every one 
must sacrifice a part of his freedom for the sake of the generality."{} But 
the sacrifice would not be made for the "{}generality's"{} sake a bit, as 
little as I concluded the agreement for the "{}generality's"{} or even for any 
other man's sake; rather I came into it only for the sake of my own benefit, 
from \textit{selfishness}.\footnote{[Literally, "{}own-benefit."{}]} But, as 
regards the sacrificing, surely I "{}sacrifice"{} only that which does not 
stand in my power, \textit{i.e.}, I "{}sacrifice"{} nothing at all.

To come back to property, the lord is proprietor. Choose then whether you want 
to be lord, or whether society shall be! On this depends whether you are to be 
an \textit{owner} or a \textit{ragamuffin}! The egoist is owner, the Socialist 
a ragamuffin. But ragamuffinism or propertylessness is the sense of feudalism, 
of the feudal system which since the last century has only changed its 
overlord, putting "{}Man"{} in the place of God, and accepting as a fief from 
Man what had before been a fief from the grace of God. That the ragamuffinism 
of Communism is carried out by the humane principle into the absolute or most 
ragamuffinly ragamuffinism has been shown above; but at the same time also, 
how ragamuffinism can only thus swing around into ownness. The \textit{old} 
feudal system was so thoroughly trampled into the ground in the Revolution 
that since then all reactionary craft has remained fruitless, and will always 
remain fruitless, because the dead is -- dead; but the resurrection too had to 
prove itself a truth in Christian history, and has so proved itself: for in 
another world feudalism is risen again with a glorified body, the \textit{new} 
feudalism under the suzerainty of "{}Man."{}

Christianity is not annihilated, but the faithful are right in having hitherto 
trustfully assumed of every combat against it that this could serve only for 
the purgation and confirmation of Christianity; for it has really only been 
glorified, and "{}Christianity exposed"{} is the -- \textit{human 
Christianity}. We are still living entirely in the Christian age, and the very 
ones who feel worst about it are the most zealously contributing to 
"{}complete"{} it. The more human, the dearer has feudalism become to us; for 
we the less believe that it still is feudalism, we take it the more 
confidently for ownness and think we have found what is "{}most absolutely our 
own"{} when we discover "{}the human."{}

Liberalism wants to give me what is mine, but it thinks to procure it for me 
not under the title of mine, but under that of the "{}human."{} As if it were 
attainable under this mask! The rights of man, the precious work of the 
Revolution, have the meaning that the Man in me 
\textit{entitles}\footnote{[Literally, furnishes me with a \textit{right}.]} 
me to this and that; I as individual, \textit{i.e.} as this man, am not 
entitled, but Man has the right and entitles me. Hence as man I may well be 
entitled; but, as I am more than man, to wit, a \textit{special} man, it may 
be refused to this very me, the special one. If on the other hand you insist 
on the \textit{value} of your gifts, keep up their price, do not let 
yourselves be forced to sell out below price, do not let yourselves be talked 
into the idea that your ware is not worth its price. do not make yourself 
ridiculous by a "{}ridiculous price,"{} but imitate the brave man who says, I 
will \textit{sell} my life (property) dear, the enemy shall not have it at a 
cheap \textit{bargain}; then you have recognized the reverse of Communism as 
the correct thing, and the word then is not "{}Give up your property!"{} but 
\textit{"{}Get the value out of} your property!"{}

Over the portal of our time stands not that "{}Know thyself"{} of Apollo, but 
a \textit{"{}Get the value out of thyself!"{}}

Proudhon calls property "{}robbery"{} (\textit{le vol}). But alien property -- 
and he is talking of this alone -- is not less existent by renunciation, 
cession, and humility; it is a \textit{present}. Why so sentimentally call for 
compassion as a poor victim of robbery, when one is just a foolish, cowardly 
giver of presents? Why here again put the fault on others as if they were 
robbing us, while we ourselves do bear the fault in leaving the others 
unrobbed? The poor are to blame for there being rich men.

Universally, no one grows indignant at \textit{his}, but at \textit{alien} 
property. They do not in truth attack property, but the alienation of 
property. They want to be able to call \textit{more}, not less, 
\textit{theirs}; they want to call everything \textit{theirs}. They are 
fighting, therefore, against \textit{alienness}, or, to form a word similar to 
property, against alienty. And how do they help themselves therein? Instead of 
transforming the alien into own, they play impartial and ask only that all 
property be left to a third party, \textit{e. g.} human society. They 
revindicate the alien not in their own name but in a third party's. Now the 
"{}egoistic"{} coloring is wiped off, and everything is so clean and -- human!

Propertylessness or ragamuffinism, this then is the "{}essence of 
Christianity,"{} as it is essence of all religiousness (\textit{i.e.} 
godliness, morality, humanity), and only announced itself most clearly, and, 
as glad tidings, became a gospel capable of development, in the "{}absolute 
religion."{} We have before us the most striking development in the present 
fight against property, a fight which is to bring "{}Man"{} to victory and 
make propertylessness complete: victorious humanity is the victory of 
--Christianity. But the "{}Christianity exposed"{} thus is feudalism 
completed. the most all-embracing feudal system, \textit{i.e.} perfect 
ragamuffinism.

Once more then, doubtless, a "{}revolution"{} against the feudal system? --

Revolution and insurrection must not be looked upon as synonymous. The former 
consists in an overturning of conditions, of the established condition or 
status, the State or society, and is accordingly a \textit{political} or 
\textit{social} act; the latter has indeed for its unavoidable consequence a 
transformation of circumstances, yet does not start from it but from men's 
discontent with themselves, is not an armed rising, but a rising of 
individuals, a getting up, without regard to the arrangements that spring from 
it. The Revolution aimed at new \textit{arrangements}; insurrection leads us 
no longer to \textit{let} ourselves be arranged, but to arrange ourselves, and 
sets no glittering hopes on "{}institutions."{} It is not a fight against the 
established, since, if it prospers, the established collapses of itself; it is 
only a working forth of me out of the established. If I leave the established, 
it is dead and passes into decay. Now, as my object is not the overthrow of an 
established order but my elevation above it, my purpose and deed are not a 
political or social but (as directed toward myself and my ownness alone) an 
\textit{egoistic} purpose and deed.

The revolution commands one to make \textit{arrangements}, the 
insurrection\footnote{[\textit{Emp\"orung}]} demands that he \textit{rise or 
exalt himself}.\footnote{[\textit{sich auf-oder emp\"orzurichten}]} What 
\textit{constitution} was to be chosen, this question busied the revolutionary 
heads, and the whole political period foams with constitutional fights and 
constitutional questions, as the social talents too were uncommonly inventive 
in societary arrangements (phalansteries etc.). The insurgent\footnote{To 
secure myself against a criminal charge I superfluously make the express 
remark that I choose the word "{}insurrection"{} on account of its 
etymological sense, and therefore am not using it in the limited sense which 
is disallowed by the penal code.} strives to become constitutionless.

While, to get greater clearness, I am thinking up a comparison, the founding 
of Christianity comes unexpectedly into my mind. On the liberal side it is 
noted as a bad point in the first Christians that they preached obedience to 
the established heathen civil order, enjoined recognition of the heathen 
authorities, and confidently delivered a command, "{}Give to the emperor that 
which is the emperor's."{} Yet how much disturbance arose at the same time 
against the Roman supremacy, how mutinous did the Jews and even the Romans 
show themselves against their own temporal government! In short, how popular 
was "{}political discontent!"{} Those Christians would hear nothing of it; 
would not side with the "{}liberal tendencies."{} The time was politically so 
agitated that, as is said in the gospels, people thought they could not accuse 
the founder of Christianity more successfully than if they arraigned him for 
"{}political intrigue,"{} and yet the same gospels report that he was 
precisely the one who took least part in these political doings. But why was 
he not a revolutionist, not a demagogue, as the Jews would gladly have seen 
him? Why was he not a liberal? Because he expected no salvation from a change 
of \textit{conditions}, and this whole business was indifferent to him. He was 
not a revolutionist, like \textit{e. g.} Caesar, but an insurgent; not a 
State-overturner, but one who straightened \textit{himself} up. That was why 
it was for him only a matter of "{}Be ye wise as serpents,"{} which expresses 
the same sense as, in the special case, that "{}Give to the emperor that which 
is the emperor's"{}; for he was not carrying on any liberal or political fight 
against the established authorities, but wanted to walk his \textit{own} way, 
untroubled about, and undisturbed by, these authorities. Not less indifferent 
to him than the government were its enemies, for neither understood what he 
wanted, and he had only to keep them off from him with the wisdom of the 
serpent. But, even though not a ringleader of popular mutiny, not a demagogue 
or revolutionist, he (and every one of the ancient Christians) was so much the 
more an \textit{insurgent}, who lifted himself above everything that seemed 
sublime to the government and its opponents, and absolved himself from 
everything that they remained bound to, and who at the same time cut off the 
sources of life of the whole heathen world, with which the established State 
must wither away as a matter of course; precisely because he put from him the 
upsetting of the established, he was its deadly enemy and real annihilator; 
for he walled it in, confidently and recklessly carrying up the building of 
\textit{his} temple over it, without heeding the pains of the immured.

Now, as it happened to the heathen order of the world, will the Christian 
order fare likewise? A revolution certainly does not bring on the end if an 
insurrection is not consummated first!

My intercourse with the world, what does it aim at? I want to have the 
enjoyment of it, therefore it must be my property, and therefore I want to win 
it. I do not want the liberty of men, nor their equality; I want only 
\textit{my} power over them, I want to make them my property, \textit{i.e. 
material for enjoyment}. And, if I do not succeed in that, well, then I call 
even the power over life and death, which Church and State reserved to 
themselves -- mine. Brand that officer's widow who, in the flight in Russia, 
after her leg has been shot away, takes the garter from it, strangles her 
child therewith, and then bleeds to death alongside the corpse -- brand the 
memory of the -- infanticide. Who knows, if this child had remained alive, how 
much it might have "{}been of use to the world!"{} The mother murdered it 
because she wanted to die \textit{satisfied} and at rest. Perhaps this case 
still appeals to your sentimentality, and you do not know how to read out of 
it anything further. Be it so; I on my part use it as an example for this, 
that \textit{my} satisfaction decides about my relation to men, and that I do 
not renounce, from any access of humility, even the power over life and death.

As regards "{}social duties"{} in general, another does not give me my 
position toward others, therefore neither God nor humanity prescribes to me my 
relation to men, but I give myself this position. This is more strikingly said 
thus: I have no \textit{duty} to others, as I have a duty even to myself 
(\textit{e. g.} that of self-preservation, and therefore not suicide) only so 
long as I distinguish myself from myself (my immortal soul from my earthly 
existence, etc.).

I no longer \textit{humble} myself before any power, and I recognize that all 
powers are only my power, which I have to subject at once when they threaten 
to become a power \textit{against} or \textit{above} me; each of them must be 
only one of \textit{my means} to carry my point, as a hound is our power 
against game, but is killed by us if it should fall upon us ourselves. All 
powers that dominate me I then reduce to serving me. The idols exist through 
me; I need only refrain from creating them anew, then they exist no longer: 
"{}higher powers"{} exist only through my exalting them and abasing myself.

Consequently my relation to the world is this: I no longer do anything for it 
"{}for God's sake,"{} I do nothing "{}for man's sake,"{} but what I do I do 
"{}for my sake."{} Thus alone does the world satisfy me, while it is 
characteristic of the religious standpoint, in which I include the moral and 
humane also, that from it everything remains a pious wish (\textit{pium 
desiderium}), \textit{i.e.} an other-world matter, something unattained. Thus 
the general salvation of men, the moral world of a general love, eternal 
peace, the cessation of egoism, etc. "{}Nothing in this world is perfect."{} 
With this miserable phrase the good part from it, and take flight into their 
closet to God, or into their proud "{}self-consciousness."{} But we remain in 
this "{}imperfect"{} world, because even so we can use it for our -- 
self-enjoyment.

My intercourse with the world consists in my enjoying it, and so consuming it 
for my self-enjoyment. \textit{Intercourse} is the \textit{enjoyment of the 
world}, and belongs to my -- self-enjoyment.

\section[3. My Self-Enjoyment]{\centering 3. My Self-Enjoyment}

We stand at the boundary of a period. The world hitherto took thought for 
nothing but the gain of life, took care for -- \textit{life}. For whether all 
activity is put on the stretch for the life of this world or of the other, for 
the temporal or for the eternal, whether one hankers for "{}daily bread"{} 
("{}Give us our daily bread"{}) or for "{}holy bread"{} ("{}the true bread 
from heaven"{} "{}the bread of God, that comes from heaven and \textit{gives 
life} to the world"{}; "{}the bread of life,"{} John 6), whether one takes 
care for "{}dear life"{} or for "{}life to eternity"{} -- this does not change 
the object of the strain and care, which in the one case as in the other shows 
itself to be \textit{life}. Do the modern tendencies announce themselves 
otherwise? People now want nobody to be embarrassed for the most indispensable 
necessaries of life, but want every one to feel secure as to these; and on the 
other hand they teach that man has this life to attend to and the real world 
to adapt himself to, without vain care for another.

Let us take up the same thing from another side. When one is anxious only to 
live, he easily, in this solicitude, forgets the enjoyment of life. If his 
only concern is for life, and he thinks "{}if I only have my dear life,"{} he 
does not apply his full strength to using, \textit{i. e.}, enjoying, life. But 
how does one use life? In using it up, like the candle, which one uses in 
burning it up. One uses life, and consequently himself the living one, in 
\textit{consuming} it and himself. \textit{Enjoyment of life} is using life 
up.

Now -- we are in search of the \textit{enjoyment} of life! And what did the 
religious world do? It went in search of life. Wherein consists the true life, 
the blessed life; etc.? How is it to be attained? What must man do and become 
in order to become a truly living man? How does he fulfil this calling? These 
and similar questions indicate that the askers were still seeking for 
\textit{themselves --} to wit, themselves in the true sense, in the sense of 
true living. "{}What I am is foam and shadow; what I shall be is my true 
self."{} To chase after this self, to produce it, to realize it, constitutes 
the hard task of mortals, who die only to \textit{rise again}, live only to 
die, live only to find the true life.

Not till I am certain of myself, and no longer seeking for myself, am I really 
my property; I have myself, therefore I use and enjoy myself. On the other 
hand, I can never take comfort in myself as long as I think that I have still 
to find my true self and that it must come to this, that not I but Christ or 
some other spiritual, \textit{i.e.} ghostly, self (\textit{e. g.} the true 
man, the essence of man, etc.) lives in me.

A vast interval separates the two views. In the old I go toward myself, in the 
new I start from myself; in the former I long for myself, in the latter I have 
myself and do with myself as one does with any other property -- I enjoy 
myself at my pleasure. I am no longer afraid for my life, but "{}squander"{} 
it.

Henceforth, the question runs, not how one can acquire life, but how one can 
squander, enjoy it; or, not how one is to produce the true self in himself, 
but how one is to dissolve himself, to live himself out.

What else should the ideal be but the sought-for ever-distant self? One seeks 
for himself, consequently one doth not yet have himself; one aspires toward 
what one \textit{ought} to be, consequently one \textit{is} not it. One lives 
in \textit{longing} and has lived thousands of years in it, in \textit{hope}. 
Living is quite another thing in -- \textit{enjoyment!}

Does this perchance apply only to the so-called pious? No, it applies to all 
who belong to the departing period of history, even to its men of pleasure. 
For them too the work-days were followed by a Sunday, and the rush of the 
world by the dream of a better world, of a general happiness of humanity; in 
short by an ideal. But philosophers especially are contrasted with the pious. 
Now, have they been thinking of anything else than the ideal, been planning 
for anything else than the absolute self? Longing and hope everywhere, and 
nothing but these. For me, call it romanticism.

If the \textit{enjoyment of life} is to triumph over the \textit{longing for 
life} or hope of life, it must vanquish this in its double significance which 
Schiller introduces in his "{}Ideal and Life"{}; it must crush spiritual and 
secular poverty, exterminate the ideal and -- the want of daily bread. He who 
must expend his life to prolong life cannot enjoy it, and he who is still 
seeking for his life does not have it and can as little enjoy it: both are 
poor, but "{}blessed are the poor."{}

Those who are hungering for the true life have no power over their present 
life, but must apply it for the purpose of thereby gaining that true life, and 
must sacrifice it entirely to this aspiration and this task. If in the case of 
those devotees who hope for a life in the other world, and look upon that in 
this world as merely a preparation for it, the tributariness of their earthly 
existence, which they put solely into the service of the hoped-for heavenly 
existence, is pretty distinctly apparent; one would yet go far wrong if one 
wanted to consider the most rationalistic and enlightened as less 
self-sacrificing. Oh, there is to be found in the "{}true life"{} a much more 
comprehensive significance than the "{}heavenly"{} is competent to express. 
Now, is not -- to introduce the liberal concept of it at once -- the 
"{}human"{} and "{}truly human"{} life the true one? And is every one already 
leading this truly human life from the start, or must he first raise himself 
to it with hard toil? Does he already have it as his present life, or must he 
struggle for it as his future life, which will become his part only when he 
"{}is no longer tainted with any egoism"{}? In this view life exists only to 
gain life, and one lives only to make the essence of man alive in oneself, one 
lives for the sake of this essence. One has his life only in order to procure 
by means of it the "{}true"{} life cleansed of all egoism. Hence one is afraid 
to make any use he likes of his life: it is to serve only for the "{}right 
use."{}

In short, one has a \textit{calling in life}, a task in life; one has 
something to realize and produce by his life, a something for which our life 
is only means and implement, a something that is worth more than this life, a 
something to which one \textit{owes} his life. One has a God who asks a 
\textit{living sacrifice}. Only the rudeness of human sacrifice has been lost 
with time; human sacrifice itself has remained unabated, and criminals hourly 
fall sacrifices to justice, and we "{}poor sinners"{} slay our own selves as 
sacrifices for "{}the human essence,"{} the "{}idea of mankind,"{} 
"{}humanity,"{} and whatever the idols or gods are called besides.

But, because we owe our life to that something, therefore --this is the next 
point -- we have no right to take it from us.

The conservative tendency of Christianity does not permit thinking of death 
otherwise than with the purpose to take its sting from it and -- live on and 
preserve oneself nicely. The Christian lets everything happen and come upon 
him if he -- the arch-Jew -- can only haggle and smuggle himself into heaven; 
he must not kill himself, he must only -- preserve himself and work at the 
"{}preparation of a future abode."{} Conservatism or "{}conquest of death"{} 
lies at his heart; "{}the last enemy that is abolished is death."{}\footnote{1 
Cor. 15. 26.} "{}Christ has taken the power from death and brought life and 
\textit{imperishable} being to light by the gospel."{}\footnote{2 Tim. 1. 10.} 
"{}Imperishableness,"{} stability.

The moral man wants the good, the right; and, if he takes to the means that 
lead to this goal, really lead to it, then these means are not \textit{his} 
means, but those of the good, right, etc., itself. These means are never 
immoral, because the good end itself mediates itself through them: the end 
sanctifies the means. They call this maxim jesuitical, but it is "{}moral"{} 
through and through. The moral man acts \textit{in the service} of an end or 
an idea: he makes himself the \textit{tool} of the idea of the good, as the 
pious man counts it his glory to be a tool or instrument of God. To await 
death is what the moral commandment postulates as the good; to give it to 
oneself is immoral and bad: \textit{suicide} finds no excuse before the 
judgment-seat of morality. If the religious man forbids it because "{}you have 
not given yourself life, but God, who alone can also take it from you again"{} 
(as if, even taking in this conception, God did not take it from me just as 
much when I kill myself as when a tile from the roof, or a hostile bullet, 
fells me; for he would have aroused the resolution of death in me too!), the 
moral man forbids it because I owe my life to the fatherland, etc., "{}because 
I do not know whether I may not yet accomplish good by my life."{} Of course, 
for in me good loses a tool, as God does an instrument. If I am immoral, the 
good is served in my \textit{amendment}; if I am "{}ungodly,"{} God has joy in 
my \textit{penitence}. Suicide, therefore, is ungodly as well as nefarious. If 
one whose standpoint is religiousness takes his own life, he acts in 
forgetfulness of God; but, if the suicide's standpoint is morality, he acts in 
forgetfulness of duty, immorally. People worried themselves much with the 
question whether Emilia Galotti's death can be justified before morality (they 
take it as if it were suicide, which it is too in substance). That she is so 
infatuated with chastity, this moral good, as to yield up even her life for it 
is certainly moral; but, again, that she fears the weakness of her flesh is 
immoral.\footnote{[See the next to the last scene of the tragedy:

ODOARDO: Under the pretext of a judicial investigation he tears you out of our 
arms and takes you to Grimaldi. ...

EMILIA: Give me that dagger, father, me! ...

ODOARDO: No, no! Reflect -- You too have only one life to lose.

EMILIA: And only one innocence!

ODOARDO: Which is above the reach of any violence. --

EMILIA: But not above the reach of any seduction. -- Violence! violence! Who 
cannot defy violence? What is called violence is nothing; seduction is the 
true violence. -- I have blood, father; blood as youthful and warm as 
anybody's. My senses are senses. -- I can warrant nothing. I am sure of 
nothing. I know Grimaldi's house. It is the house of pleasure. An hour there, 
under my mother's eyes -- and there arose in my soul so much tumult as the 
strictest exercises of religion could hardly quiet in weeks. -- Religion! And 
what religion? -- To escape nothing worse, thousands sprang into the water and 
are saints. -- Give me that dagger, father, give it to me. ...

EMILIA: Once indeed there was a father who. to save his daughter from shame, 
drove into her heart whatever steel he could quickest find -- gave life to her 
for the second time. But all such deeds are of the past! Of such fathers there 
are no more.

ODOARDO: Yes, daughter, yes! (\textit{Stabs her}.)]

}

Such contradictions form the tragic conflict universally in the moral drama; 
and one must think and feel morally to be able to take an interest in it.

What holds good of piety and morality will necessarily apply to humanity also, 
because one owes his life likewise to man, mankind or the species. Only when I 
am under obligation to no being is the maintaining of life -- my affair. "{}A 
leap from this bridge makes me free!"{}

But, if we owe the maintaining of our life to that being that we are to make 
alive in ourselves, it is not less our duty not to lead this life according to 
\textit{our} pleasure, but to shape it in conformity to that being. All my 
feeling, thinking, and willing, all my doing and designing, belongs to -- him.

What is in conformity to that being is to be inferred from his concept; and 
how differently has this concept been conceived! or how differently has that 
being been imagined! What demands the Supreme Being makes on the Mohammedan; 
what different ones the Christian, again, thinks he hears from him; how 
divergent, therefore, must the shaping of the lives of the two turn out! Only 
this do all hold fast, that the Supreme Being is to 
\textit{judge}\footnote{[Or, "{}regulate"{} (\textit{richten})]} our life.

But the pious who have their judge in God, and in his word a book of 
directions for their life, I everywhere pass by only reminiscently, because 
they belong to a period of development that has been lived through, and as 
petrifactions they may remain in their fixed place right along; in our time it 
is no longer the pious, but the liberals, who have the floor, and piety itself 
cannot keep from reddening its pale face with liberal coloring. But the 
liberals do not adore their judge in God, and do not unfold their life by the 
directions of the divine word, but regulate\footnote{[\textit{richten}]} 
themselves by man: they want to be not "{}divine"{} but "{}human,"{} and to 
live so.

Man is the liberal's supreme being, man the \textit{judge} of his life, 
humanity his \textit{directions}, or catechism. God is spirit, but man is the 
"{}most perfect spirit,"{} the final result of the long chase after the spirit 
or of the "{}searching in the depths of the Godhead,"{} \textit{i.e.} in the 
depths of the spirit.

Every one of your traits is to be human; you yourself are to be so from top to 
toe, in the inward as in the outward; for humanity is your calling.

Calling -- destiny -- task! --

What one can become he does become. A born poet may well be hindered by the 
disfavor of circumstances from standing on the high level of his time, and, 
after the great studies that are indispensable for this, producing 
\textit{consummate} works of art; but he will make poetry, be he a plowman or 
so lucky as to live at the court of Weimar. A born musician will make music, 
no matter whether on all instruments or only on an oaten pipe. A born 
philosophical head can give proof of itself as university philosopher or as 
village philosopher. Finally, a born dolt, who, as is very well compatible 
with this, may at the same time be a sly-boots, will (as probably every one 
who has visited schools is in a position to exemplify to himself by many 
instances of fellow-scholars) always remain a blockhead, let him have been 
drilled and trained into the chief of a bureau, or let him serve that same 
chief as bootblack. Nay, the born shallow-pates indisputably form the most 
numerous class of men. And why. indeed, should not the same distinctions show 
themselves in the human species that are unmistakable in every species of 
beasts? The more gifted and the less gifted are to be found everywhere.

Only a few, however, are so imbecile that one could not get ideas into them. 
Hence, people usually consider all men capable of having religion. In a 
certain degree they may be trained to other ideas too, \textit{e. g.} to some 
musical intelligence, even some philosophy. At this point then the priesthood 
of religion, of morality, of culture, of science, etc., takes its start, and 
the Communists, \textit{e. g.} want to make everything accessible to all by 
their "{}public school."{} There is heard a common assertion that this 
"{}great mass"{} cannot get along without religion; the Communists broaden it 
into the proposition that not only the "{}great mass,"{} but absolutely all, 
are called to everything.

Not enough that the great mass has been trained to religion, now it is 
actually to have to occupy itself with "{}everything human."{} Training is 
growing ever more general and more comprehensive.

You poor beings who could live so happily if you might skip according to your 
mind, you are to dance to the pipe of schoolmasters and bear-leaders, in order 
to perform tricks that you yourselves would never use yourselves for. And you 
do not even kick out of the traces at last against being always taken 
otherwise than you want to give yourselves. No, you mechanically recite to 
yourselves the question that is recited to you: "{}What am I called to? What 
\textit{ought} I to do?"{} You need only ask thus, to have yourselves 
\textit{told} what you ought to do and \textit{ordered} to do it, to have your 
\textit{calling} marked out for you, or else to order yourselves and impose it 
on yourselves according to the spirit's prescription. Then in reference to the 
will the word is, I will to do what I \textit{ought}.

A man is "{}called"{} to nothing, and has no "{}calling,"{} no "{}destiny,"{} 
as little as a plant or a beast has a "{}calling."{} The flower does not 
follow the calling to complete itself, but it spends all its forces to enjoy 
and consume the world as well as it can -- \textit{i.e.} it sucks in as much 
of the juices of the earth, as much air of the ether, as much light of the 
sun, as it can get and lodge. The bird lives up to no calling, but it uses its 
forces as much as is practicable; it catches beetles and sings to its heart's 
delight. But the forces of the flower and the bird are slight in comparison to 
those of a man, and a man who applies his forces will affect the world much 
more powerfully than flower and beast. A calling he has not, but he has forces 
that manifest themselves where they are because their being consists solely in 
their manifestation, and are as little able to abide inactive as life, which, 
if it "{}stood still"{} only a second, would no longer be life. Now, one might 
call out to the man, "{}use your force."{} Yet to this imperative would be 
given the meaning that it was man's task to use his force. It is not so. 
Rather, each one really uses his force without first looking upon this as his 
calling: at all times every one uses as much force as he possesses. One does 
say of a beaten man that he ought to have exerted his force more; but one 
forgets that, if in the moment of succumbing he had the force to exert his 
forces (\textit{e. g.} bodily forces), he would not have failed to do it: even 
if it was only the discouragement of a minute, this was yet a --destitution of 
force, a minute long. Forces may assuredly be sharpened and redoubled, 
especially by hostile resistance or friendly assistance; but where one misses 
their application one may be sure of their absence too. One can strike fire 
out of a stone, but without the blow none comes out; in like manner a man too 
needs "{}impact."{}

Now, for this reason that forces always of themselves show themselves 
operative, the command to use them would be superfluous and senseless. To use 
his forces is not man's \textit{calling} and task, but is his \textit{act}, 
real and extant at all times. Force is only a simpler word for manifestation 
of force.

Now, as this rose is a true rose to begin with, this nightingale always a true 
nightingale, so I am not for the first time a true man when I fulfil my 
calling, live up to my destiny, but I am a "{}true man"{} from the start. My 
first babble is the token of the life of a "{}true man,"{} the struggles of my 
life are the outpourings of his force, my last breath is the last exhalation 
of the force of the "{}man."{}

The true man does not lie in the future, an object of longing, but lies, 
existent and real, in the present. Whatever and whoever I may be, joyous and 
suffering, a child or a graybeard, in confidence or doubt, in sleep or in 
waking, I am it, I am the true man.

But, if I am Man, and have really found in myself him whom religious humanity 
designated as the distant goal, then everything "{}truly human"{} is also 
\textit{my} own. What was ascribed to the idea of humanity belongs to me. That 
freedom of trade,

\textit{e. g.}, which humanity has yet to attain -- and which, like an 
enchanting dream, people remove to humanity's golden future -- I take by 
anticipation as my property, and carry it on for the time in the form of 
smuggling. There may indeed be but few smugglers who have sufficient 
understanding to thus account to themselves for their doings, but the instinct 
of egoism replaces their consciousness. Above I have shown the same thing 
about freedom of the press.

Everything is my own, therefore I bring back to myself what wants to withdraw 
from me; but above all I always bring myself back when I have slipped away 
from myself to any tributariness. But this too is not my calling, but my 
natural act.

Enough, there is a mighty difference whether I make myself the starting-point 
or the goal. As the latter I do not have myself, am consequently still alien 
to myself, am my \textit{essence}, my "{}true essence,"{} and this "{}true 
essence,"{} alien to me, will mock me as a spook of a thousand different 
names. Because I am not yet I, another (like God, the true man, the truly 
pious man, the rational man, the freeman, etc.) is I, my ego.

Still far from myself, I separate myself into two halves, of which one, the 
one unattained and to be fulfilled, is the true one. The one, the untrue, must 
be brought as a sacrifice; to wit, the unspiritual one. The other, the true, 
is to be the whole man; to wit, the spirit. Then it is said, "{}The spirit is 
man's proper essence,"{} or, "{}man exists as man only spiritually."{} Now, 
there is a greedy rush to catch the spirit, as if one would then have bagged 
\textit{himself}; and so, in chasing after himself, one loses sight of 
himself, whom he is.

And, as one stormily pursues his own self, the never-attained, so one also 
despises shrewd people's rule to take men as they are, and prefers to take 
them as they should be; and, for this reason, hounds every one on after his 
should-be self and "{}endeavors to make all into equally entitled, equally 
respectable, equally moral or rational men."{}\footnote{\textit{"{}Der 
Kommunismus in der Schweiz}"{}, p. 24.}

Yes, "{}if men were what they \textit{should} be, \textit{could} be, if all 
men were rational, all loved each other as brothers,"{} then it would be a 
paradisiacal life.\footnote{\textit{Ibid}, p. 63} -- All right, men are as 
they should be, can be. What should they be? Surely not more than they can be! 
And what can they be? Not more, again, than they -- can, than they have the 
competence, the force, to be. But this they really are, because what they are 
not they are \textit{incapable} of being; for to be capable means -- really to 
be. One is not capable for anything that one really is not; one is not capable 
of anything that one does not really do. Could a man blinded by cataracts see? 
Oh, yes, if he had his cataracts successfully removed. But now he cannot see 
because he does not see. Possibility and reality always coincide. One can do 
nothing that one does not, as one does nothing that one cannot.

The singularity of this assertion vanishes when one reflects that the words 
"{}it is possible that."{} almost never contain another meaning than "{}I can 
imagine that. . .,"{} \textit{e. g.}, It is possible for all men to live 
rationally; \textit{e. g.}, I can imagine that all, etc. Now -- since my 
thinking cannot, and accordingly does not, cause all men to live rationally, 
but this must still be left to the men themselves -- general reason is for me 
only thinkable, a thinkableness, but as such in fact a \textit{reality} that 
is called a possibility only in reference to what I \textit{can} not bring to 
pass, to wit, the rationality of others. So far as depends on you, all men 
might be rational, for you have nothing against it; nay, so far as your 
thinking reaches, you perhaps cannot discover any hindrance either, and 
accordingly nothing does stand in the way of the thing in your thinking; it is 
thinkable to you.

As men are not all rational, though, it is probable that they -- cannot be so.

If something which one imagines to be easily possible is not, or does not 
happen, then one may be assured that something stands in the way of the thing, 
and that it is -- impossible. Our time has its art, science, etc.; the art may 
be bad in all conscience; but may one say that we deserved to have a better, 
and "{}could"{} have it if we only would? We have just as much art as we can 
have. Our art of today is the \textit{only art possible}, and therefore real, 
at the time.

Even in the sense to which one might at last still reduce the word 
"{}possible,"{} that it should mean "{}future,"{} it retains the full force of 
the "{}real."{} If one says, \textit{e. g.}, "{}It is possible that the sun 
will rise tomorrow"{} -- this means only, "{}for today tomorrow is the real 
future"{}; for I suppose there is hardly need of the suggestion that a future 
is real "{}future"{} only when it has not yet appeared.

Yet wherefore this dignifying of a word? If the most prolific misunderstanding 
of thousands of years were not in ambush behind it, if this single concept of 
the little word "{}possible"{} were not haunted by all the spooks of possessed 
men, its contemplation should trouble us little here.

The thought, it was just now shown, rules the possessed world. Well, then, 
possibility is nothing but thinkableness, and innumerable sacrifices have 
hitherto been made to hideous \textit{thinkableness}. It was 
\textit{thinkable} that men might become rational; thinkable, that they might 
know Christ; thinkable, that they might become moral and enthusiastic for the 
good; thinkable, that they might all take refuge in the Church's lap; 
thinkable, that they might meditate, speak, and do, nothing dangerous to the 
State; thinkable, that they \textit{might} be obedient subjects; but, because 
it was thinkable, it was -- so ran the inference -- possible, and further, 
because it was possible to men (right here lies the deceptive point; because 
it is thinkable to me, it is possible to \textit{men}), therefore they ought 
to be so, it was their \textit{calling}; and finally -- one is to take men 
only according to this calling, only as \textit{called} men, "{}not as they 
are, but as they ought to be."{}

And the further inference? Man is not the individual, but man is a 
\textit{thought}, an \textit{ideal}, to which the individual is related not 
even as the child to the man, but as a chalk point to a point thought of, or 
as a -- finite creature to the eternal Creator, or, according to modern views, 
as the specimen to the species. Here then comes to light the glorification of 
"{}humanity,"{} the "{}eternal, immortal,"{} for whose glory (\textit{in 
majorem humanitatis gloriam}) the individual must devote himself and find his 
"{}immortal renown"{} in having done something for the "{}spirit of 
humanity."{}

Thus the \textit{thinkers} rule in the world as long as the age of priests or 
of schoolmasters lasts, and what they think of is possible, but what is 
possible must be realized. They \textit{think} an ideal of man, which for the 
time is real only in their thoughts; but they also think the possibility of 
carrying it out, and there is no chance for dispute, the carrying out is 
really -- thinkable, it is an -- idea.

But you and I, we may indeed be people of whom a Krummacher can \textit{think} 
that we might yet become good Christians; if, however, he wanted to "{}labor 
with"{} us, we should soon make it palpable to him that our Christianity is 
only \textit{thinkable}, but in other respects \textit{impossible}; if he 
grinned on and on at us with his obtrusive \textit{thoughts}, his "{}good 
belief,"{} he would have to learn that we do not at all \textit{need} to 
become what we do not like to become.

And so it goes on, far beyond the most pious of the pious. "{}If all men were 
rational, if all did right, if all were guided by philanthropy, etc."{}! 
Reason, right, philanthropy, are put before the eyes of men as their calling, 
as the goal of their aspiration. And what does being rational mean? Giving 
oneself a hearing?\footnote{[Cf. note p. 81]} No, reason is a book full of 
laws, which are all enacted against egoism.

History hitherto is the history of the \textit{intellectual} man. After the 
period of sensuality, history proper begins; \textit{i.e.} the period of 
intellectuality,\footnote{[\textit{Geistigkeit}]} 
spirituality,\footnote{[\textit{Geistlichkeit}]} non-sensuality, 
supersensuality, nonsensicality. Man now begins to want to be and become 
\textit{something}. What? Good, beautiful, true; more precisely, moral, pious, 
agreeable, etc. He wants to make of himself a "{}proper man,"{} "{}something 
proper."{} \textit{Man} is his goal, his ought, his destiny, calling, task, 
his -- \textit{ideal}; he is to himself a future, otherworldly he. And 
\textit{what} makes a "{}proper fellow"{} of him? Being true, being good, 
being moral, etc. Now he looks askance at every one who does not recognize the 
same "{}what,"{} seek the same morality, have the same faith, he chases out 
"{}separatists, heretics, sects,"{} etc.

No sheep, no dog, exerts itself to become a "{}proper sheep, a proper dog"{}; 
no beast has its essence appear to it as a task, \textit{i.e.} as a concept 
that it has to realize. It realizes itself in living itself out, in dissolving 
itself, passing away. It does not ask to be or to become anything 
\textit{other} than it is.

Do I mean to advise you to be like the beasts? That you ought to become beasts 
is an exhortation which I certainly cannot give you, as that would again be a 
task, an ideal ("{}How doth the little busy bee improve each shining hour. In 
works of labor or of skill I would be busy too, for Satan finds some mischief 
still for idle hands to do"{}). It would be the same, too, as if one wished 
for the beasts that they should become human beings. Your nature is, once for 
all, a human one; you are human natures, human beings. But, just because you 
already are so, you do not still need to become so. Beasts too are 
"{}trained,"{} and a trained beast executes many unnatural things. But a 
trained dog is no better for itself than a natural one, and has no profit from 
it, even if it is more companionable for us.

Exertions to "{}form"{} all men into moral, rational, pious, human, 
"{}beings"{} (\textit{i.e.} training) were in vogue from of yore. They are 
wrecked against the indomitable quality of I, against own nature, against 
egoism. Those who are trained never attain their ideal, and only profess with 
their \textit{mouth} the sublime principles, or make a \textit{profession}, a 
profession of faith. In face of this profession they must in \textit{life} 
"{}acknowledge themselves sinners altogether,"{} and they fall short of their 
ideal, are "{}weak men,"{} and bear with them the consciousness of "{}human 
weakness."{}

It is different if you do not chase after an \textit{ideal} as your 
"{}destiny,"{} but dissolve yourself as time dissolves everything. The 
dissolution is not your "{}destiny,"{} because it is present time.

Yet the \textit{culture}, the religiousness, of men has assuredly made them 
free, but only free from one lord, to lead them to another. I have learned by 
religion to tame my appetite, I break the world's resistance by the cunning 
that is put in my hand by \textit{science}; I even serve no man; "{}I am no 
man's lackey."{} But then it comes. You must obey God more than man. Just so I 
am indeed free from irrational determination by my impulses. but obedient to 
the master \textit{Reason}. I have gained "{}spiritual freedom,"{} "{}freedom 
of the spirit."{} But with that I have then become subject to that very 
\textit{spirit}. The spirit gives me orders, reason guides me, they are my 
leaders and commanders. The "{}rational,"{} the "{}servants of the spirit,"{} 
rule. But, if \textit{I} am not flesh, I am in truth not spirit either. 
Freedom of the spirit is servitude of me, because I am more than spirit or 
flesh.

Without doubt culture has made me \textit{powerful}. It has given me power 
over all \textit{motives}, over the impulses of my nature as well as over the 
exactions and violences of the world. I know, and have gained the force for it 
by culture, that I need not let myself be coerced by any of my appetites, 
pleasures, emotions, etc.; I am their -- \textit{master}; in like manner I 
become, through the sciences and arts, the \textit{master} of the refractory 
world, whom sea and earth obey, and to whom even the stars must give an 
account of themselves. The spirit has made me \textit{master. --} But I have 
no power over the spirit itself. From religion (culture) I do learn the means 
for the "{}vanquishing of the world,"{} but not how I am to subdue 
\textit{God} too and become master of him; for God "{}is the spirit."{} And 
this same spirit, of which I am unable to become master, may have the most 
manifold shapes; he may be called God or National Spirit, State, Family, 
Reason, also -- Liberty, Humanity, Man.

\textit{I} receive with thanks what the centuries of culture have acquired for 
me; I am not willing to throw away and give up anything of it: I have not 
lived in vain. The experience that I have \textit{power} over my nature, and 
need not be the slave of my appetites, shall not be lost to me; the experience 
that I can subdue the world by culture's means is too dear- bought for me to 
be able to forget it. But I want still more.

People ask, what can man do? What can he accomplish? What goods procure, and 
put down the highest of everything as a calling. As if everything were 
possible to \textit{me!}

If one sees somebody going to ruin in a mania, a passion, etc. (\textit{e. g.} 
in the huckster-spirit, in jealousy), the desire is stirred to deliver him out 
of this possession and to help him to "{}self-conquest."{} "{}We want to make 
a man of him!"{} That would be very fine if another possession were not 
immediately put in the place of the earlier one. But one frees from the love 
of money him who is a thrall to it, only to deliver him over to piety, 
humanity, or some principle else, and to transfer him to a \textit{fixed 
standpoint} anew.

This transference from a narrow standpoint to a sublime one is declared in the 
words that the sense must not be directed to the perishable, but to the 
imperishable alone: not to the temporal, but to the eternal, absolute, divine, 
purely human, etc. -- to the spiritual.

People very soon discerned that it was not indifferent what one set his 
affections on, or what one occupied himself with; they recognized the 
importance of the \textit{object}. An object exalted above the individuality 
of things is the \textit{essence} of things; yes, the essence is alone the 
thinkable in them. it is for the \textit{thinking} man. Therefore direct no 
longer your \textit{sense} to the \textit{things}, but your \textit{thoughts} 
to the \textit{essence}. "{}Blessed are they who see not, and yet believe"{}; 
\textit{i. e.}, blessed are the \textit{thinkers}, for they have to do with 
the invisible and believe in it. Yet even an object of thought, that 
constituted an essential point of contention centuries long, comes at last to 
the point of being "{}No longer worth speaking of."{} This was discerned, but 
nevertheless people always kept before their eyes again a self-valid 
importance of the object, an absolute value of it, as if the doll were not the 
most important thing to the child, the Koran to the Turk. As long as I am not 
the sole important thing to myself, it is indifferent of what object I "{}make 
much,"{} and only my greater or lesser \textit{delinquency} against it is of 
value. The degree of my attachment and devotion marks the standpoint of my 
liability to service, the degree of my sinning shows the measure of my 
ownness.

But finally, and in general, one must know how to "{}put everything out of his 
mind,"{} if only so as to be able to -- go to sleep. Nothing may occupy us 
with which \textit{we} do not occupy ourselves: the victim of ambition cannot 
run away from his ambitious plans, nor the God-fearing man from the thought of 
God; infatuation and possessedness coincide.

To want to realize his essence or live comfortably to his concept (which with 
believers in God signifies as much as to be "{}pious,"{} and with believers in 
humanity means living "{}humanly"{}) is what only the sensual and sinful man 
can propose to himself, the man so long as he has the anxious choice between 
happiness of sense and peace of soul, so long as he is a "{}poor sinner."{} 
The Christian is nothing but a sensual man who, knowing of the sacred and 
being conscious that he violates it, sees in himself a poor sinner: 
sensualness, recognized as "{}sinfulness,"{} is Christian consciousness, is 
the Christian himself. And if "{}sin"{} and "{}sinfulness"{} are now no longer 
taken into the mouths of moderns, but, instead of that, "{}egoism,"{} 
"{}self-seeking,"{} "{}selfishness,"{} etc., engage them; if the devil has 
been translated into the "{}un-man"{} or "{}egoistic man"{} -- is the 
Christian less present then than before? Is not the old discord between good 
and evil -- is not a judge over us, man -- is not a calling, the calling to 
make oneself man -- left? If they no longer name it calling, but "{}task"{} 
or, very likely, "{}duty,"{} the change of name is quite correct, because 
"{}man"{} is not, like God, a personal being that can "{}call"{}; but outside 
the name the thing remains as of old.

\begin{center}
--------\end{center}


Every one has a relation to objects, and more, every one is differently 
related to them. Let us choose as an example that book to which millions of 
men had a relation for two thousand years, the Bible. What is it, what was it, 
to each? Absolutely, only what he \textit{made out of it!} For him who makes 
to himself nothing at all out of it, it is nothing at all; for him who uses it 
as an amulet, it has solely the value, the significance, of a means of 
sorcery; for him who, like children, plays with it, it is nothing but a 
plaything, etc.

Now, Christianity asks that it shall \textit{be the same for all}: say the 
sacred book or the "{}sacred Scriptures."{} This means as much as that the 
Christian's view shall also be that of other men, and that no one may be 
otherwise related to that object. And with this the ownness of the relation is 
destroyed, and one mind, one disposition, is fixed as the "{}true"{}, the 
"{}only true"{} one. In the limitation of the freedom to make of the Bible 
what I will, the freedom of making in general is limited; and the coercion of 
a view or a judgment is put in its place. He who should pass the judgment that 
the Bible was a long error of mankind would judge -- \textit{criminally}.

In fact, the child who tears it to pieces or plays with it, the Inca Atahualpa 
who lays his ear to it and throws it away contemptuously when it remains dumb, 
judges just as correctly about the Bible as the priest who praises in it the 
"{}Word of God,"{} or the critic who calls it a job of men's hands. For how we 
toss things about is the affair of our \textit{option}, our \textit{free 
will}: we use them according to our \textit{heart's pleasure}, or, more 
clearly, we use them just as we \textit{can}. Why, what do the parsons scream 
about when they see how Hegel and the speculative theologians make speculative 
thoughts out of the contents of the Bible? Precisely this, that they deal with 
it according to their heart's pleasure, or "{}proceed arbitrarily with it."{}

But, because we all show ourselves arbitrary in the handling of objects, 
\textit{i.e.} do with them as we \textit{like} best, at our \textit{liking} 
(the philosopher likes nothing so well as when he can trace out an "{}idea"{} 
in everything, as the God-fearing man likes to make God his friend by 
everything, and so, \textit{e. g.}, by keeping the Bible sacred), therefore we 
nowhere meet such grievous arbitrariness, such a frightful tendency to 
violence, such stupid coercion, as in this very domain of our -- \textit{own 
free will}. If \textit{we} proceed arbitrarily in taking the sacred objects 
thus or so, how is it then that we want to take it ill of the parson-spirits 
if they take us just as arbitrarily, \textit{in their fashion}, and esteem us 
worthy of the heretic's fire or of another punishment, perhaps of the -- 
censorship?

What a man is, he makes out of things; "{}as you look at the world, so it 
looks at you again."{} Then the wise advice makes itself heard again at once, 
You must only look at it "{}rightly, unbiasedly,"{} etc. As if the child did 
not look at the Bible "{}rightly and unbiasedly"{} when it makes it a 
plaything. That shrewd precept is given us, \textit{e. g.} by Feuerbach. One 
does look at things rightly when one makes of them what one \textit{will} (by 
things objects in general are here understood, \textit{e. g.} God, our 
fellowmen, a sweetheart, a book, a beast, etc.). And therefore the things and 
the looking at them are not first, but I am, my will is. One \textit{will} 
brings thoughts out of the things, \textit{will} discover reason in the world, 
\textit{will} have sacredness in it: therefore one shall find them. "{}Seek 
and ye shall find."{} \textit{What} I will seek, I determine: I want, 
\textit{e. g.}, to get edification from the Bible; it is to be found; I want 
to read and test the Bible thoroughly; my outcome will be a thorough 
instruction and criticism -- to the extent of my powers. I elect for myself 
what I have a fancy for, and in electing I show myself -- arbitrary.

Connected with this is the discernment that every judgment which I pass upon 
an object is the \textit{creature} of my will; and that discernment again 
leads me to not losing myself in the \textit{creature}, the judgment, but 
remaining the \textit{creator}, the judge, who is ever creating anew. All 
predicates of objects are my statements, my judgments, my -- creatures. If 
they want to tear themselves loose from me and be something for themselves, or 
actually overawe me, then I have nothing more pressing to do than to take them 
back into their nothing, into me the creator. God, Christ, Trinity, morality, 
the good, etc., are such creatures, of which I must not merely allow myself to 
say that they are truths, but also that they are deceptions. As I once willed 
and decreed their existence, so I want to have license to will their non- 
existence too; I must not let them grow over my head, must not have the 
weakness to let them become something "{}absolute,"{} whereby they would be 
eternalized and withdrawn from my power and decision. With that I should fall 
a prey to the \textit{principle of stability}, the proper life-principle of 
religion, which concerns itself with creating "{}sanctuaries that must not be 
touched,"{} "{}eternal truths"{} -- in short, that which shall be "{}sacred"{} 
-- and depriving you of what is \textit{yours}.

The object makes us into possessed men in its sacred form just as in its 
profane, as a supersensuous object, just as it does as a sensuous one. The 
appetite or mania refers to both, and avarice and longing for heaven stand on 
a level. When the rationalists wanted to win people for the sensuous world, 
Lavater preached the longing for the invisible. The one party wanted to call 
forth \textit{emotion}, the other \textit{motion}, activity.

 The conception of objects is altogether diverse, even as God, Christ, the 
world, were and are conceived of in the most manifold wise. In this every one 
is a "{}dissenter,"{} and after bloody combats so much has at last been 
attained, that opposite views about one and the same object are no longer 
condemned as heresies worthy of death. The "{}dissenters"{} reconcile 
themselves to each other. But why should I only dissent (think otherwise) 
about a thing? Why not push the thinking otherwise to its last extremity, that 
of no longer having any regard at all for the thing, and therefore thinking 
its nothingness, crushing it? Then the \textit{conception} itself has an end, 
because there is no longer anything to conceive of. Why am I to say, let us 
suppose, "{}God is not Allah, not Brahma, not Jehovah, but -- God"{}; but not, 
"{}God is nothing but a deception"{}? Why do people brand me if I am an 
"{}atheist"{}? Because they put the creature above the creator ("{}They honor 
and serve the creature more than the Creator"{}\footnote{Rom. 1. 25.}) and 
require a \textit{ruling object}, that the subject may be right 
\textit{submissive}. I am to bend \textit{beneath} the absolute, I 
\textit{ought} to.

By the "{}realm of thoughts"{} Christianity has completed itself; the thought 
is that inwardness in which all the world's lights go out, all existence 
becomes existenceless, the inward. man (the heart, the head) is all in all. 
This realm of thoughts awaits its deliverance, awaits, like the Sphinx, 
Oedipus's key- word to the riddle, that it may enter in at last to its death. 
I am the annihilator of its continuance, for in the creator's realm it no 
longer forms a realm of its own, not a State in the State, but a creature of 
my creative -- thoughtlessness. Only together and at the same time with the 
benumbed \textit{thinking} world can the world of Christians, Christianity and 
religion itself, come to its downfall; only when thoughts run out are there no 
more believers. To the thinker his thinking is a "{}sublime labor, a sacred 
activity,"{} and it rests on a firm \textit{faith}, the faith in truth. At 
first praying is a sacred activity, then this sacred "{}devotion"{} passes 
over into a rational and reasoning "{}thinking,"{} which, however, likewise 
retains in the "{}sacred truth"{} its underangeable basis of faith, and is 
only a marvelous machine that the spirit of truth winds up for its service. 
Free thinking and free science busy \textit{me} -- for it is not I that am 
free, not \textit{I} that busy myself, but thinking is free and busies me -- 
with heaven and the heavenly or "{}divine"{}; \textit{e. g.}, properly, with 
the world and the worldly, not this world but "{}another"{} world; it is only 
the reversing and deranging of the world, a busying with the \textit{essence} 
of the world, therefore a \textit{derangement}. The thinker is blind to the 
immediateness of things, and incapable of mastering them: he does not eat, 
does not drink, does not enjoy; for the eater and drinker is never the 
thinker, nay, the latter forgets eating and drinking, his getting on in life, 
the cares of nourishment, etc., over his thinking; he forgets it as the 
praying man too forgets it. This is why he appears to the forceful son of 
nature as a queer Dick, a \textit{fool --} even if he does look upon him as 
holy, just as lunatics appeared so to the ancients. Free thinking is lunacy, 
because it is \textit{pure movement of the inwardness}, of the merely 
\textit{inward man}, which guides and regulates the rest of the man. The 
shaman and the speculative philosopher mark the bottom and top rounds on the 
ladder of the \textit{inward} man, the -- Mongol. Shaman and philosopher fight 
with ghosts, demons, \textit{spirits}, gods.

Totally different from this \textit{free} thinking is \textit{own} thinking, 
\textit{my} thinking, a thinking which does not guide me, but is guided, 
continued, or broken off, by me at my pleasure. The distinction of this own 
thinking from free thinking is similar to that of own sensuality, which I 
satisfy at pleasure, from free, unruly sensuality to which I succumb.

Feuerbach, in the \textit{Principles of the Philosophy of the Future}, is 
always harping upon \textit{being}. In this he too, with all his antagonism to 
Hegel and the absolute philosophy, is stuck fast in abstraction; for 
"{}being"{} is abstraction, as is even "{}the I."{} Only \textit{I am} not 
abstraction alone: \textit{I am} all in all, consequently even abstraction or 
nothing; I am all and nothing; I am not a mere thought, but at the same time I 
am full of thoughts, a thought-world. Hegel condemns the own, 
mine,\footnote{[\textit{das Meinige}]} -- "{}opinion."{}\footnote{[\textit{die 
--"{}Meinung"{}}]} "{}Absolute thinking"{} is that which forgets that it is 
\textit{my} thinking, that \textit{I} think, and that it exists only through 
\textit{me}. But I, as I, swallow up again what is mine, am its master; it is 
only my \textit{opinion}, which I can at any moment \textit{change}, 
\textit{i.e.} annihilate, take back into myself, and consume. Feuerbach wants 
to smite Hegel's "{}absolute thinking"{} with \textit{unconquered being}. But 
in me being is as much conquered as thinking is. It is \textit{my} being, as 
the other is \textit{my} thinking.

With this, of course, Feuerbach does not get further than to the proof, 
trivial in itself, that I require the \textit{senses} for everything, or that 
I cannot entirely do without these organs. Certainly I cannot think if I do 
not exist sensuously. But for thinking as well as for feeling, and so for the 
abstract as well as for the sensuous, I need above all things \textit{myself}, 
this quite particular myself, this \textit{unique} myself. If I were not this 
one, \textit{e. g.} Hegel, I should not look at the world as I do look at it, 
I should not pick out of it that philosophical system which just I as Hegel 
do, etc. I should indeed have senses, as do other people too, but I should not 
utilize them as I do.

Thus the reproach is brought up against Hegel by Feuerbach\footnote{P. 47ff.} 
that he misuses language, understanding by many words something else than what 
natural consciousness takes them for; and yet he too commits the same fault 
when he gives the "{}sensuous"{} a sense of unusual eminence. Thus it is said, 
p. 69, "{}the sensuous is not the profane, the destitute of thought, the 
obvious, that which is understood of itself."{} But, if it is the sacred, the 
full of thought, the recondite, that which can be understood only through 
mediation -- well, then it is no longer what people call the sensuous. The 
sensuous is only that which exists for \textit{the senses}; what, on the other 
hand, is enjoyable only to those who enjoy with \textit{more} than the senses, 
who go beyond sense-enjoyment or sense-reception, is at most mediated or 
introduced by the senses, \textit{i. e.}, the senses constitute a 
\textit{condition} for obtaining it, but it is no longer anything sensuous. 
The sensuous, whatever it may be, when taken up into me becomes something 
non-sensuous, which, however, may again have sensuous effects, \textit{e. g.} 
as by the stirring of my emotions and my blood.

It is well that Feuerbach brings sensuousness to honor, but the only thing he 
is able to do with it is to clothe the materialism of his "{}new philosophy"{} 
with what had hitherto been the property of idealism, the "{}absolute 
philosophy."{} As little as people let it be talked into them that one can 
live on the "{}spiritual"{} alone without bread, so little will they believe 
his word that as a sensuous being one is already everything, and so spiritual, 
full of thoughts, etc.

Nothing at all is justified by \textit{being}. What is thought of \textit{is} 
as well as what is not thought of; the stone in the street \textit{is}, and my 
notion of it \textit{is} too. Both are only in different \textit{spaces}, the 
former in airy space, the latter in my head, in \textit{me}; for I am space 
like the street.

The professionals, the privileged, brook no freedom of thought, \textit{i.e.} 
no thoughts that do not come from the "{}Giver of all good,"{} be he called 
God, pope, church, or whatever else. If anybody has such illegitimate 
thoughts, he must whisper them into his confessor's ear, and have himself 
chastised by him till the slave-whip becomes unendurable to the free thoughts. 
In other ways too the professional spirit takes care that free thoughts shall 
not come at all: first and foremost, by a wise education. He on whom the 
principles of morality have been duly inculcated never becomes free again from 
moralizing thoughts, and robbery, perjury, overreaching, etc., remain to him 
fixed ideas against which no freedom of thought protects him. He has his 
thoughts "{}from above,"{} and gets no further.

It is different with the holders of concessions or patents. Every one must be 
able to have and form thoughts as he will. If he has the patent, or the 
concession, of a capacity to think, he needs no special \textit{privilege}. 
But, as "{}all men are rational,"{} it is free to every one to put into his 
head any thoughts whatever, and, to the extent of the patent of his natural 
endowment, to have a greater or less wealth of thoughts. Now one hears the 
admonitions that one "{}is to honor all opinions and convictions,"{} that 
"{}every conviction is authorized,"{} that one must be "{}tolerant to the 
views of others,"{} etc.

But "{}your thoughts are not my thoughts, and your ways are not my ways."{} Or 
rather, I mean the reverse: Your thoughts are \textit{my} thoughts, which I 
dispose of as I will, and which I strike down unmercifully; they are my 
property, which I annihilate as I list. I do not wait for authorization from 
you first, to decompose and blow away your thoughts. It does not matter to me 
that you call these thoughts yours too, they remain mine nevertheless, and how 
I will proceed with them is \textit{my affair}, not a usurpation. It may 
please me to leave you in your thoughts; then I keep still. Do you believe 
thoughts fly around free like birds, so that every one may get himself some 
which he may then make good against me as his inviolable property? What is 
flying around is all -- \textit{mine}.

Do you believe you have your thoughts for yourselves and need answer to no one 
for them, or as you do also say, you have to give an account of them to God 
only? No, your great and small thoughts belong to me, and I handle them at my 
pleasure.

The thought is my \textit{own} only when I have no misgiving about bringing it 
in danger of death every moment, when I do not have to fear its loss as a 
\textit{loss for me}, a loss of me. The thought is my own only when I can 
indeed subjugate it, but it never can subjugate me, never fanaticizes me, 
makes me the tool of its realization.

So freedom of thought exists when I can have all possible thoughts; but the 
thoughts become property only by not being able to become masters. In the time 
of freedom of thought, thoughts (ideas) \textit{rule}; but, if I attain to 
property in thought, they stand as my creatures.

If the hierarchy had not so penetrated men to the innermost as to take from 
them all courage to pursue free thoughts, \textit{e. g.}, thoughts perhaps 
displeasing to God, one would have to consider freedom of thought just as 
empty a word as, say, a freedom of digestion.

According to the professionals' opinion, the thought is \textit{given} to me; 
according to the freethinkers', I \textit{seek} the thought. There the 
\textit{truth} is already found and extant, only I must -- receive it from its 
Giver by grace; here the truth is to be sought and is my goal, lying in the 
future, toward which I have to run.

In both cases the truth (the true thought) lies outside me, and I aspire to 
\textit{get} it, be it by presentation (grace), be it by earning (merit of my 
own). Therefore, (1) The truth is a \textit{privilege}; (2) No, the way to it 
is patent to all, and neither the Bible nor the holy fathers nor the church 
nor any one else is in possession of the truth; but one can come into 
possession of it by -- speculating.

Both, one sees, are \textit{property-less} in relation to the truth: they have 
it either as a \textit{fief} (for the "{}holy father,"{} \textit{e. g.} is not 
a unique person; as unique he is this Sixtus, Clement, but he does not have 
the truth as Sixtus, Clement, but as "{}holy father,"{} \textit{i.e.} as a 
spirit) or as an \textit{ideal}. As a fief, it is only for a few (the 
privileged); as an ideal, for \textit{all} (the patentees).

Freedom of thought, then, has the meaning that we do indeed all walk in the 
dark and in the paths of error, but every one can on this path approach 
\textit{the truth} and is accordingly on the right path ("{}All roads lead to 
Rome, to the world's end, etc."{}). Hence freedom of thought means this much, 
that the true thought is not my \textit{own}; for, if it were this, how should 
people want to shut me off from it?

Thinking has become entirely free, and has laid down a lot of truths which I 
must accommodate myself to. It seeks to complete itself into a \textit{system} 
and to bring itself to an absolute "{}constitution."{} In the State \textit{e. 
g.} it seeks for the idea, say, till it has brought out the "{}rational 
State,"{} in which I am then obliged to be suited; in man (anthropology), till 
it "{}has found man."{}

The thinker is distinguished from the believer only by believing much more 
than the latter, who on his part thinks of much less as signified by his faith 
(creed). The thinker has a thousand tenets of faith where the believer gets 
along with few; but the former brings \textit{coherence} into his tenets, and 
takes the coherence in turn for the scale to estimate their worth by. If one 
or the other does not fit into his budget, he throws it out.

The thinkers run parallel to the believers in their pronouncements. Instead of 
"{}If it is from God you will not root it out,"{} the word is "{}If it is from 
the \textit{truth}, is true, etc."{}; instead of "{}Give God the glory"{} -- 
"{}Give truth the glory."{} But it is very much the same to me whether God or 
the truth wins; first and foremost I want to win.

Aside from this, how is an "{}unlimited freedom"{} to be thinkable inside of 
the State or society? The State may well protect one against another, but yet 
it must not let itself be endangered by an unmeasured freedom, a so-called 
unbridledness. Thus in "{}freedom of instruction"{} the \textit{State} 
declares only this -- that it is suited with every one who instructs as the 
State (or, speaking more comprehensibly, the political power) would have it. 
The point for the competitors is this "{}as the State would have it."{} If the 
clergy, \textit{e. g.}, does not will as the State does, then it itself 
excludes itself from \textit{competition} (\textit{vid.} France). The limit 
that is necessarily drawn in the State for any and all competition is called 
"{}the oversight and superintendence of the State."{} In bidding freedom of 
instruction keep within the due bounds, the State at the same time fixes the 
scope of freedom of thought; because, as a rule, people do not think farther 
than their teachers have thought.

Hear Minister Guizot: "{}The great difficulty of today is the \textit{guiding 
and dominating of the mind}. Formerly the church fulfilled this mission; now 
it is not adequate to it. It is from the university that this great service 
must be expected, and the university will not fail to perform it. We, the 
\textit{government}, have the duty of supporting it therein. The charter calls 
for the freedom of thought and that of conscience."{}\footnote{Chamber of 
peers, Apr. 25, 1844.} So, in favor of freedom of thought and conscience, the 
minister demands "{}the guiding and dominating of the mind."{}

Catholicism haled the examinee before the forum of ecclesiasticism, 
Protestantism before that of biblical Christianity. It would be but little 
bettered if one haled him before that of reason, as Ruge, \textit{e. g.}, 
wants to.\footnote{\textit{"{}Anekdota},"{} 1, 120.} Whether the church, the 
Bible, or reason (to which, moreover, Luther and Huss already appealed) is the 
\textit{sacred authority} makes no difference in essentials.

The "{}question of our time"{} does not become soluble even when one puts it 
thus: Is anything general authorized, or only the individual? Is the 
generality (\textit{e. g.} State, law, custom, morality, etc.) authorized, or 
individuality? It becomes soluble for the first time when one no longer asks 
after an "{}authorization"{} at all, and does not carry on a mere fight 
against "{}privileges."{} -- A "{}rational"{} freedom of teaching, which 
recognizes only the conscience of reason, \footnote{\textit{``Anekdota''}, 
1, 127. }does not bring us to the goal; we require an \textit{egoistic} 
freedom of teaching rather, a freedom of teaching for all ownness, wherein 
\textit{I} become audible and can announce myself unchecked. That I make 
myself \textit{``audible''},\footnote{[\textit{vernehmbar}]} this alone is 
``reason,''\footnote{[\textit{Vernunft}]} be I ever so irrational; in my 
making myself heard, and so hearing myself, others as well as I myself enjoy 
me, and at the same time consume me.

What would be gained if, as formerly the orthodox I, the loyal I, the moral I, 
etc., was free, now the rational I should become free? Would this be the 
freedom of me?

If I am free as "{}rational I,"{} then the rational in me, or reason, is free; 
and this freedom of reason, or freedom of the thought, was the ideal of the 
Christian world from of old. They wanted to make thinking -- and, as 
aforesaid, faith is also thinking, as thinking is faith -- free; the thinkers, 
\textit{i.e.} the believers as well as the rational, were to be free; for the 
rest freedom was impossible. But the freedom of thinkers is the "{}freedom of 
the children of God,"{} and at the same time the most merciless --hierarchy or 
dominion of the thought; for \textit{I} succumb to the thought. If thoughts 
are free, I am their slave; I have no power over them, and am dominated by 
them. But I want to have the thought, want to be full of thoughts, but at the 
same time I want to be thoughtless, and, instead of freedom of thought, I 
preserve for myself thoughtlessness.

If the point is to have myself understood and to make communications, then 
assuredly I can make use only of \textit{human} means, which are at my command 
because I am at the same time man. And really I have thoughts only as 
\textit{man}; as I, I am at the same time 
\textit{thoughtless.}\footnote{[Literally, "{}thought-rid."{}]} He who cannot 
get rid of a thought is so far \textit{only} man, is a thrall of 
\textit{language}, this human institution, this treasury of \textit{human} 
thoughts. Language or "{}the word"{} tyrannizes hardest over us, because it 
brings up against us a whole army of \textit{fixed ideas}. Just observe 
yourself in the act of reflection, right now, and you will find how you make 
progress only by becoming thoughtless and speechless every moment. You are not 
thoughtless and speechless merely in (say) sleep, but even in the deepest 
reflection; yes, precisely then most so. And only by this thoughtlessness, 
this unrecognized "{}freedom of thought"{} or freedom from the thought, are 
you your own. Only from it do you arrive at putting language to use as your 
\textit{property}.

If thinking is not \textit{my} thinking, it is merely a spun-out thought; it 
is slave work, or the work of a "{}servant obeying at the word."{} For not a 
thought, but I, am the beginning for my thinking, and therefore I am its goal 
too, even as its whole course is only a course of my self-enjoyment; for 
absolute or free thinking, on the other hand, thinking itself is the 
beginning, and it plagues itself with propounding this beginning as the 
extremest "{}abstraction"{} (\textit{e. g.} as being). This very abstraction, 
or this thought, is then spun out further.

Absolute thinking is the affair of the human spirit, and this is a holy 
spirit. Hence this thinking is an affair of the parsons, who have "{}a sense 
for it,"{} a sense for the "{}highest interests of mankind,"{} for "{}the 
spirit."{}

To the believer, truths are a \textit{settled} thing, a fact; to the 
freethinker, a thing that is still to be \textit{settled}. Be absolute 
thinking ever so unbelieving, its incredulity has its limits, and there does 
remain a belief in the truth, in the spirit, in the idea and its final 
victory: this thinking does not sin against the holy spirit. But all thinking 
that does not sin against the holy spirit is belief in spirits or ghosts.

I can as little renounce thinking as feeling, the spirit's activity as little 
as the activity of the senses. As feeling is our sense for things, so thinking 
is our sense for essences (thoughts). Essences have their existence in 
everything sensuous, especially in the word. The power of words follows that 
of things: first one is coerced by the rod, afterward by conviction. The might 
of things overcomes our courage, our spirit; against the power of a 
conviction, and so of the word, even the rack and the sword lose their 
overpoweringness and force. The men of conviction are the priestly men, who 
resist every enticement of Satan.

Christianity took away from the things of this world only their 
irresistibleness, made us independent of them. In like manner I raise myself 
above truths and their power: as I am supersensual, so I am supertrue. 
\textit{Before me} truths are as common and as indifferent as things; they do 
not carry me away, and do not inspire me with enthusiasm. There exists not 
even one truth, not right, not freedom, humanity, etc., that has stability 
before me, and to which I subject myself. They are \textit{words}, nothing but 
words, as to the Christian nothing but "{}vain things."{} In words and truths 
(every word is a truth, as Hegel asserts that one cannot \textit{tell} a lie) 
there is no salvation for me, as little as there is for the Christian in 
things and vanities. As the riches of this world do not make me happy, so 
neither do its truths. It is now no longer Satan, but the spirit, that plays 
the story of the temptation; and he does not seduce by the things of this 
world, but by its thoughts, by the "{}glitter of the idea."{}

Along with worldly goods, all sacred goods too must be put away as no longer 
valuable.

Truths are phrases, ways of speaking, words (l\'ogos); brought into 
connection, or into an articulate series, they form logic, science, 
philosophy.

For thinking and speaking I need truths and words, as I do foods for eating; 
without them I cannot think nor speak. Truths are men's thoughts, set down in 
words and therefore just as extant as other things, although extant only for 
the mind or for thinking. They are human institutions and human creatures, 
and, even if they are given out for divine revelations, there still remains in 
them the quality of alienness for me; yes, as my own creatures they are 
already alienated from me after the act of creation.

The Christian man is the man with faith in thinking, who believes in the 
supreme dominion of thoughts and wants to bring thoughts, so-called 
"{}principles,"{} to dominion. Many a one does indeed test the thoughts, and 
chooses none of them for his master without criticism, but in this he is like 
the dog who sniffs at people to smell out "{}his master"{}; he is always 
aiming at the \textit{ruling} thought. The Christian may reform and revolt an 
infinite deal, may demolish the ruling concepts of centuries; he will always 
aspire to a new "{}principle"{} or new master again, always set up a higher or 
"{}deeper"{} truth again, always call forth a cult again, always proclaim a 
spirit called to dominion, lay down a \textit{law} for all.

If there is even one truth only to which man has to devote his life and his 
powers because he is man, then he is subjected to a rule, dominion, law; he is 
a servingman. It is supposed that, \textit{e. g.} man, humanity, liberty, 
etc., are such truths.

On the other hand, one can say thus: Whether you will further occupy yourself 
with thinking depends on you; only know that, \textit{if} in your thinking you 
would like to make out anything worthy of notice, many hard problems are to be 
solved, without vanquishing which you cannot get far. There exists, therefore, 
no duty and no calling for you to meddle with thoughts (ideas, truths); but, 
if you will do so, you will do well to utilize what the forces of others have 
already achieved toward clearing up these difficult subjects.

Thus, therefore, he who will think does assuredly have a task, which 
\textit{he} consciously or unconsciously sets for himself in willing that; but 
no one has the task of thinking or of believing. In the former case it may be 
said, "{}You do not go far enough, you have a narrow and biased interest, you 
do not go to the bottom of the thing; in short, you do not completely subdue 
it. But, on the other hand, however far you may come at any time, you are 
still always at the end, you have no call to step farther, and you can have it 
as you will or as you are able. It stands with this as with any other piece of 
work, which you can give up when the humor for it wears off. Just so, if you 
can no longer \textit{believe} a thing, you do not have to force yourself into 
faith or to busy yourself lastingly as if with a sacred truth of the faith, as 
theologians or philosophers do, but you can tranquilly draw back your interest 
from it and let it run. Priestly spirits will indeed expound this your lack of 
interest as `laziness, thoughtlessness, obduracy, self-deception,' etc. 
But do you just let the trumpery lie, notwithstanding. No 
thing,\footnote{[\textit{Sache}]} no so-called `highest interest of 
mankind,' no `sacred cause,'\footnote{[\textit{Sache}]} is worth your 
serving it, and occupying yourself with it for \textit{its sake}; you may seek 
its worth in this alone, whether it is worth anything to \textit{you} for your 
sake. Become like children, the biblical saying admonishes us. But children 
have no sacred interest and know nothing of a `good cause.' They know all 
the more accurately what they have a fancy for; and they think over, to the 
best of their powers, how they are to arrive at it."{}

Thinking will as little cease as feeling. But the power of thoughts and ideas, 
the dominion of theories and principles, the sovereignty of the spirit, in 
short the -- \textit{hierarchy}, lasts as long as the parsons, \textit{i.e.}, 
theologians, philosophers, statesmen, philistines, liberals, schoolmasters, 
servants, parents, children, married couples, Proudhon, George Sand, 
Bluntschli, etc., etc., have the floor; the hierarchy will endure as long as 
people believe in, think of, or even criticize, principles; for even the most 
inexorable criticism, which undermines all current principles, still does 
finally \textit{believe in the principle}.

Every one criticises, but the criterion is different. People run after the 
"{}right"{} criterion. The right criterion is the first presupposition. The 
critic starts from a proposition, a truth, a belief. This is not a creation of 
the critic, but of the dogmatist; nay, commonly it is actually taken up out of 
the culture of the time without further ceremony, like \textit{e. g.} 
"{}liberty,"{} "{}humanity,"{} etc. The critic has not "{}discovered man,"{} 
but this truth has been established as "{}man"{} by the dogmatist, and the 
critic (who, besides, may be the same person with him) believes in this truth, 
this article of faith. In this faith, and possessed by this faith, he 
criticises.

The secret of criticism is some "{}truth"{} or other: this remains its 
energizing mystery.

But I distinguish between \textit{servile} and \textit{own} criticism. If I 
criticize under the presupposition of a supreme being, my criticism 
\textit{serves} the being and is carried on for its sake: if \textit{e. g.} I 
am possessed by the belief in a "{}free State,"{} then everything that has a 
bearing on it I criticize from the standpoint of whether it is suitable to 
this State, for I \textit{love} this State; if I criticize as a pious man, 
then for me everything falls into the classes of divine and diabolical, and 
before my criticism nature consists of traces of God or traces of the devil 
(hence names like Godsgift, Godmount, the Devil's Pulpit), men of believers 
and unbelievers; if I criticize while believing in man as the "{}true 
essence,"{} then for me everything falls primarily into the classes of man and 
the un-man, etc.

Criticism has to this day remained a work of love: for at all times we 
exercised it for the love of some being. All servile criticism is a product of 
love, a possessedness, and proceeds according to that New Testament precept, 
"{}Test everything and hold fast the \textit{good."{}}\footnote{1 Thess. 5. 
21.} "{}The good"{} is the touchstone, the criterion. The good, returning 
under a thousand names and forms, remained always the presupposition, remained 
the dogmatic fixed point for this criticism, remained the -- fixed idea.

The critic, in setting to work, impartially presupposes the "{}truth,"{} and 
seeks for the truth in the belief that it is to be found. He wants to 
ascertain the true, and has in it that very "{}good."{}

Presuppose means nothing else than put a \textit{thought} in front, or think 
something before everything else and think the rest from the starting-point of 
this that has \textit{been thought}, \textit{i.e.} measure and criticize it by 
this. In other words, this is as much as to say that thinking is to begin with 
something already thought. If thinking began at all, instead of being begun, 
if thinking were a subject, an acting personality of its own, as even the 
plant is such, then indeed there would be no abandoning the principle that 
thinking must begin with itself. But it is just the personification of 
thinking that brings to pass those innumerable errors. In the Hegelian system 
they always talk as if thinking or "{}the thinking spirit"{} (\textit{i.e.} 
personified thinking, thinking as a ghost) thought and acted; in critical 
liberalism it is always said that "{}criticism"{} does this and that, or else 
that "{}self- consciousness"{} finds this and that. But, if thinking ranks as 
the personal actor, thinking itself must be presupposed; if criticism ranks as 
such, a thought must likewise stand in front. Thinking and criticism could be 
active only starting from themselves, would have to be themselves the 
presupposition of their activity, as without being they could not be active. 
But thinking, as a thing presupposed, is a fixed thought, a \textit{dogma}; 
thinking and criticism, therefore, can start only from a \textit{dogma, i. e.} 
from a thought, a fixed idea, a presupposition.

With this we come back again to what was enunciated above, that Christianity 
consists in the development of a world of thoughts, or that it is the proper 
"{}freedom of thought,"{} the "{}free thought,"{} the "{}free spirit."{} The 
"{}true"{} criticism, which I called "{}servile,"{} is therefore just as much 
"{}free"{} criticism, for it is not \textit{my own}.

The case stands otherwise when what is yours is not made into something that 
is of itself, not personified, not made independent as a "{}spirit"{} to 
itself. \textit{Your} thinking has for a presupposition not "{}thinking,"{} 
but \textit{you}. But thus you do presuppose yourself after all? Yes, but not 
for myself, but for my thinking. Before my thinking, there is -- I. From this 
it follows that my thinking is not preceded by a \textit{thought}, or that my 
thinking is without a "{}presupposition."{} For the presupposition which I am 
for my thinking is not one \textit{made by thinking}, not one \textit{thought 
of}, but it is \textit{posited} thinking \textit{itself}, it is the 
\textit{owner} of the thought, and proves only that thinking is nothing more 
than -- \textit{property}, \textit{i. e.} that an "{}independent"{} thinking, 
a "{}thinking spirit,"{} does not exist at all.

 This reversal of the usual way of regarding things might so resemble an empty 
playing with abstractions that even those against whom it is directed would 
acquiesce in the harmless aspect I give it, if practical consequences were not 
connected with it.

To bring these into a concise expression, the assertion now made is that man 
is not the measure of all things, but I am this measure. The servile critic 
has before his eyes another being, an idea, which he means to serve; therefore 
he only slays the false idols for his God. What is done for the love of this 
being, what else should it be but a -- work of love? But I, when I criticize, 
do not even have myself before my eyes, but am only doing myself a pleasure, 
amusing myself according to my taste; according to my several needs I chew the 
thing up or only inhale its odor.

The distinction between the two attitudes will come out still more strikingly 
if one reflects that the servile critic, because love guides him, supposes he 
is serving the thing (cause) itself.

\textit{The} truth, or "{}truth in general,"{} people are bound not to give 
up, but to seek for. What else is it but the \textit{\^Etre supr\^eme}, the 
highest essence? Even "{}true criticism"{} would have to despair if it lost 
faith in the truth. And yet the truth is only a -- \textit{thought}; but it is 
not merely "{}a"{} thought, but the thought that is above all thoughts, the 
irrefragable thought; it is \textit{the} thought itself, which gives the first 
hallowing to all others; it is the consecration of thoughts, the 
"{}absolute,"{} the "{}sacred"{} thought. The truth wears longer than all the 
gods; for it is only in the truth's service, and for love of it, that people 
have overthrown the gods and at last God himself. "{}The truth"{} outlasts the 
downfall of the world of gods, for it is the immortal soul of this transitory 
world of gods, it is Deity itself.

I will answer Pilate's question, What is truth? Truth is the free thought, the 
free idea, the free spirit; truth is what is free from you, what is not your 
own, what is not in your power. But truth is also the completely 
unindependent, impersonal, unreal, and incorporeal; truth cannot step forward 
as you do, cannot move, change, develop; truth awaits and receives everything 
from you, and itself is only through you; for it exists only -- in your head. 
You concede that the truth is a thought, but say that not every thought is a 
true one, or, as you are also likely to express it, not every thought is truly 
and really a thought. And by what do you measure and recognize the thought? By 
\textit{your impotence}, to wit, by your being no longer able to make any 
successful assault on it! When it overpowers you, inspires you, and carries 
you away, then you hold it to be the true one. Its dominion over you certifies 
to you its truth; and, when it possesses you, and you are possessed by it, 
then you feel well with it, for then you have found your -- \textit{lord and 
master}. When you were seeking the truth, what did your heart then long for? 
For your master! You did not aspire to \textit{your} might, but to a Mighty 
One, and wanted to exalt a Mighty One ("{}Exalt ye the Lord our God!"{}). The 
truth, my dear Pilate, is -- the Lord, and all who seek the truth are seeking 
and praising the Lord. Where does the Lord exist? Where else but in your head? 
He is only spirit, and, wherever you believe you really see him, there he is a 
-- ghost; for the Lord is merely something that is thought of, and it was only 
the Christian pains and agony to make the invisible visible, the spiritual 
corporeal, that generated the ghost and was the frightful misery of the belief 
in ghosts.

As long as you believe in the truth, you do not believe in yourself, and you 
are a -- \textit{servant}, a -- \textit{religious man}. You alone are the 
truth, or rather, you are more than the truth, which is nothing at all before 
you. You too do assuredly ask about the truth, you too do assuredly 
"{}criticize,"{} but you do not ask about a "{}higher truth"{} -- to wit, one 
that should be higher than you -- nor criticize according to the criterion of 
such a truth. You address yourself to thoughts and notions, as you do to the 
appearances of things, only for the purpose of making them palatable to you, 
enjoyable to you, and your own: you want only to subdue them and become their 
\textit{owner}, you want to orient yourself and feel at home in them, and you 
find them true, or see them in their true light, when they can no longer slip 
away from you, no longer have any unseized or uncomprehended place, or when 
they are \textit{right for you}, when they are your \textit{property}. If 
afterward they become heavier again, if they wriggle themselves out of your 
power again, then that is just their untruth -- to wit, your impotence. Your 
impotence is their power, your humility their exaltation. Their truth, 
therefore, is you, or is the nothing which you are for them and in which they 
dissolve: their truth is their \textit{nothingness}.

Only as the property of me do the spirits, the truths, get to rest; and they 
then for the first time really are, when they have been deprived of their 
sorry existence and made a property of mine, when it is no longer said "{}the 
truth develops itself, rules, asserts itself; history (also a concept) wins 
the victory,"{} etc. The truth never has won a victory, but was always my 
\textit{means} to the victory, like the sword ("{}the sword of truth"{}). The 
truth is dead, a letter, a word, a material that I can use up. All truth by 
itself is dead, a corpse; it is alive only in the same way as my lungs are 
alive -- to wit, in the measure of my own vitality. Truths are material, like 
vegetables and weeds; as to whether vegetable or weed, the decision lies in 
me.

Objects are to me only material that I use up. Wherever I put my hand I grasp 
a truth, which I trim for myself. The truth is certain to me, and I do not 
need to long after it. To do the truth a service is in no case my intent; it 
is to me only a nourishment for my thinking head, as potatoes are for my 
digesting stomach, or as a friend is for my social heart. As long as I have 
the humor and force for thinking, every truth serves me only for me to work it 
up according to my powers. As reality or worldliness is "{}vain and a thing of 
naught"{} for Christians, so is the truth for me. It exists, exactly as much 
as the things of this world go on existing although the Christian has proved 
their nothingness; but it is vain, because it has its \textit{value} not 
\textit{in itself} but \textit{in me. Of itself} it is \textit{valueless}. The 
truth is a -- \textit{creature}.

As you produce innumerable things by your activity, yes, shape the earth's 
surface anew and set up works of men everywhere, so too you may still 
ascertain numberless truths by your thinking, and we will gladly take delight 
in them. Nevertheless, as I do not please to hand myself over to serve your 
newly discovered machines mechanically, but only help to set them running for 
my benefit, so too I will only use your truths, without letting myself be used 
for their demands.

All truths \textit{beneath} me are to my liking; a truth \textit{above} me, a 
truth that I should have to \textit{direct} myself by, I am not acquainted 
with. For me there is no truth, for nothing is more than I! Not even my 
essence, not even the essence of man, is more than I! than I, this "{}drop in 
the bucket,"{} this "{}insignificant man"{}!

You believe that you have done the utmost when you boldly assert that, because 
every time has its own truth, there is no "{}absolute truth."{} Why, with this 
you nevertheless still leave to each time its truth, and thus you quite 
genuinely create an "{}absolute truth,"{} a truth that no time lacks, because 
every time, however its truth may be, still has a "{}truth."{}

Is it meant only that people have been thinking in every time, and so have had 
thoughts or truths, and that in the subsequent time these were other than they 
were in the earlier? No, the word is to be that every time had its "{}truth of 
faith"{}; and in fact none has yet appeared in which a "{}higher truth"{} has 
not been recognized, a truth that people believed they must subject themselves 
to as "{}highness and majesty."{}

Every truth of a time is its fixed idea, and, if people later found another 
truth, this always happened only because they sought for another; they only 
reformed the folly and put a modern dress on it. For they did want -- who 
would dare doubt their justification for this? -- they wanted to be 
"{}inspired by an idea."{} They wanted to be dominated -- possessed, by a 
\textit{thought}! The most modern ruler of this kind is "{}our essence,"{} or 
"{}man."{}

For all free criticism a thought was the criterion; for own criticism I am, I 
the unspeakable, and so not the merely thought-of; for what is merely thought 
of is always speakable, because word and thought coincide. That is true which 
is mine, untrue that whose own I am; true, \textit{e. g.} the union; untrue, 
the State and society. "{}Free and true"{} criticism takes care for the 
consistent dominion of a thought, an idea, a spirit; "{}own"{} criticism, for 
nothing but my \textit{self-enjoyment}. But in this the latter is in fact -- 
and we will not spare it this "{}ignominy"{}! -- like the bestial criticism of 
instinct. I, like the criticizing beast, am concerned only for 
\textit{myself}, not "{}for the cause."{} I am the criterion of truth, but I 
am not an idea, but more than idea, \textit{e. g.}, unutterable. \textit{My 
criticism} is not a "{}free"{} criticism, not free from me, and not 
"{}servile,"{} not in the service of an idea, but an \textit{own} criticism.

True or human criticism makes out only whether something is \textit{suitable} 
to man, to the true man; but by own criticism you ascertain whether it is 
suitable to \textit{you}.

Free criticism busies itself with \textit{ideas}, and therefore is always 
theoretical. However it may rage against ideas, it still does not get clear of 
them. It pitches into the ghosts, but it can do this only as it holds them to 
be ghosts. The ideas it has to do with do not fully disappear; the morning 
breeze of a new day does not scare them away.

The critic may indeed come to ataraxia before ideas, but he never gets 
\textit{rid} of them; \textit{i.e.} he will never comprehend that above the 
\textit{bodily man} there does not exist something higher -- to wit, liberty, 
his humanity, etc. He always has a "{}calling"{} of man still left, 
"{}humanity."{} And this idea of humanity remains unrealized, just because it 
is an "{}idea"{} and is to remain such.

If, on the other hand, I grasp the idea as \textit{my} idea, then it is 
already realized, because I am its reality; its reality consists in the fact 
that I, the bodily, have it.

They say, the idea of liberty realizes itself in the history of the world. The 
reverse is the case; this idea is real as a man thinks it, and it is real in 
the measure in which it is idea, \textit{i. e.} in which I think it or 
\textit{have} it. It is not the idea of liberty that develops itself, but men 
develop themselves, and, of course, in this self-development develop their 
thinking too.

In short, the critic is not yet \textit{owner}, because he still fights with 
ideas as with powerful aliens -- as the Christian is not owner of his "{}bad 
desires"{} so long as he has to combat them; for him who contends against 
vice, vice \textit{exists}.

Criticism remains stuck fast in the "{}freedom of knowing,"{} the freedom of 
the spirit, and the spirit gains its proper freedom when it fills itself with 
the pure, true idea; this is the freedom of thinking, which cannot be without 
thoughts.

Criticism smites one idea only by another, \textit{e. g.} that of privilege by 
that of manhood, or that of egoism by that of unselfishness.

In general, the beginning of Christianity comes on the stage again in its 
critical end, egoism being combated here as there. I am not to make myself 
(the individual) count, but the idea, the general.

Why, warfare of the priesthood with \textit{egoism}, of the spiritually minded 
with the worldly-minded, constitutes the substance of all Christian history. 
In the newest criticism this war only becomes all-embracing, fanaticism 
complete. Indeed, neither can it pass away till it passes thus, after it has 
had its life and its rage out.

\begin{center}
--------\end{center}


Whether what I think and do is Christian, what do I care? Whether it is human, 
liberal, humane, whether unhuman, illiberal, inhuman, what do I ask about 
that? If only it accomplishes what I want, if only I satisfy myself in it, 
then overlay it with predicates as you will; it is all alike to me.

Perhaps I too, in the very next moment, defend myself against my former 
thoughts; I too am likely to change suddenly my mode of action; but not on 
account of its not corresponding to Christianity, not on account of its 
running counter to the eternal rights of man, not on account of its affronting 
the idea of mankind, humanity, and humanitarianism, but -- because I am no 
longer all in it, because it no longer furnishes me any full enjoyment, 
because I doubt the earlier thought or no longer please myself in the mode of 
action just now practiced. As the world as property has become a 
\textit{material} with which I undertake what I will, so the spirit too as 
property must sink down into a \textit{material} before which I no longer 
entertain any sacred dread. Then, firstly, I shall shudder no more before a 
thought, let it appear as presumptuous and "{}devilish"{} as it will, because, 
if it threatens to become too inconvenient and unsatisfactory for \textit{me}, 
its end lies in my power; but neither shall I recoil from any deed because 
there dwells in it a spirit of godlessness, immorality, wrongfulness. as 
little as St. Boniface pleased to desist, through religious scrupulousness, 
from cutting down the sacred oak of the heathens. If the \textit{things} of 
the world have once become vain, the thoughts of the spirit must also become 
vain.

No thought is sacred, for let no thought rank as 
"{}devotions"{};\footnote{[\textit{Andacht}, a compound form of the word 
"{}thought"{}.]} no feeling is sacred (no sacred feeling of friendship, 
mother's feelings, etc.), no belief is sacred. They are all 
\textit{alienable}, my alienable property, and are annihilated, as they are 
created, by \textit{me}.

The Christian can lose all \textit{things} or objects, the most loved persons, 
these "{}objects"{} of his love, without giving up himself (\textit{i.e.}, in 
the Christian sense, his spirit, his soul! as lost. The owner can cast from 
him all the \textit{thoughts} that were dear to his heart and kindled his 
zeal, and will likewise "{}gain a thousandfold again,"{} because he, their 
creator, remains.

Unconsciously and involuntarily we all strive toward ownness, and there will 
hardly be one among us who has not given up a sacred feeling, a sacred 
thought, a sacred belief; nay, we probably meet no one who could not still 
deliver himself from one or another of his sacred thoughts. All our contention 
against convictions starts from the opinion that maybe we are capable of 
driving our opponent out of his entrenchments of thought. But what I do 
unconsciously I half-do, and therefore after every victory over a faith I 
become again the \textit{prisoner} (possessed) of a faith which then takes my 
whole self anew into its \textit{service}, and makes me an enthusiast for 
reason after I have ceased to be enthusiastic for the Bible, or an enthusiast 
for the idea of humanity after I have fought long enough for that of 
Christianity.

Doubtless, as owner of thoughts, I shall cover my property with my shield, 
just as I do not, as owner of things, willingly let everybody help himself to 
them; but at the same time I shall look forward smilingly to the outcome of 
the battle, smilingly lay the shield on the corpses of my thoughts and my 
faith, smilingly triumph when I am beaten. That is the very humor of the 
thing. Every one who has "{}sublimer feelings"{} is able to vent his humor on 
the pettiness of men; but to let it play with all "{}great thoughts, sublime 
feelings, noble inspiration, and sacred faith"{} presupposes that I am the 
owner of all.

If religion has set up the proposition that we are sinners altogether, I set 
over against it the other: we are perfect altogether! For we are, every 
moment, all that we can be; and we never need be more. Since no defect cleaves 
to us, sin has no meaning either. Show me a sinner in the world still, if no 
one any longer needs to do what suits a superior! If I only need do what suits 
myself, I am no sinner if I do not do what suits myself, as I do not injure in 
myself a "{}holy one"{}; if, on the other hand, I am to be pious, then I must 
do what suits God; if I am to act humanly, I must do what suits the essence of 
man, the idea of mankind, etc. What religion calls the "{}sinner,"{} 
humanitarianism calls the "{}egoist."{} But, once more: if I need not do what 
suits any other, is the "{}egoist,"{} in whom humanitarianism has borne to 
itself a new-fangled devil, anything more than a piece of nonsense? The 
egoist, before whom the humane shudder, is a spook as much as the devil is: he 
exists only as a bogie and phantasm in their brain. If they were not 
unsophisticatedly drifting back and forth in the antediluvian opposition of 
good and evil, to which they have given the modern names of "{}human"{} and 
"{}egoistic,"{} they would not have freshened up the hoary "{}sinner"{} into 
an "{}egoist"{} either, and put a new patch on an old garment. But they could 
not do otherwise, for they hold it for their task to be "{}men."{} They are 
rid of the Good One; good is left!\footnote{[See note on p. 112.]}

We are perfect altogether, and on the whole earth there is not one man who is 
a sinner! There are crazy people who imagine that they are God the Father, God 
the Son, or the man in the moon, and so too the world swarms with fools who 
seem to themselves to be sinners; but, as the former are not the man in the 
moon, so the latter are -- not sinners. Their sin is imaginary, yet, it is 
insidiously objected, their craziness or their possessedness is at least their 
sin. Their possessedness is nothing but what they -- could achieve, the result 
of their development, just as Luther's faith in the Bible was all that he was 
-- competent to make out. The one brings himself into the madhouse with his 
development, the other brings himself therewith into the Pantheon and to the 
loss of -- Valhalla.

There is no sinner and no sinful egoism!

Get away from me with your "{}philanthropy"{}! Creep in, you philanthropist, 
into the "{}dens of vice,"{} linger awhile in the throng of the great city: 
will you not everywhere find sin, and sin, and again sin? Will you not wail 
over corrupt humanity, not lament at the monstrous egoism? Will you see a rich 
man without finding him pitiless and "{}egoistic?"{} Perhaps you already call 
yourself an atheist, but you remain true to the Christian feeling that a camel 
will sooner go through a needle's eye than a rich man not be an "{}un-man."{} 
How many do you see anyhow that you would not throw into the "{}egoistic 
mass"{}? What, therefore, has your philanthropy [love of man] found? Nothing 
but unlovable men! And where do they all come from? From you, from your 
philanthropy! You brought the sinner with you in your head, therefore you 
found him, therefore you inserted him everywhere. Do not call men sinners, and 
they are not: you alone are the creator of sinners; you, who fancy that you 
love men, are the very one to throw them into the mire of sin, the very one to 
divide them into vicious and virtuous, into men and un-men, the very one to 
befoul them with the slaver of your possessedness; for you love not 
\textit{men}, but \textit{man}. But I tell you, you have never seen a sinner, 
you have only -- dreamed of him.

Self-enjoyment is embittered to me by my thinking I must serve another, by my 
fancying myself under obligation to him, by my holding myself called to 
"{}self-sacrifice,"{} "{}resignation,"{} "{}enthusiasm."{} All right: if I no 
longer serve any idea, any "{}higher essence,"{} then it is clear of itself 
that I no longer serve any man either, but -- under all circumstances -- 
\textit{myself}. But thus I am not merely in fact or in being, but also for my 
consciousness, the -- unique.\footnote{[\textit{Einzige}]}

There pertains to \textit{you} more than the divine, the human, etc.; 
\textit{yours} pertains to you.

Look upon yourself as more powerful than they give you out for, and you have 
more power; look upon yourself as more, and you have more.

You are then not merely \textit{called} to everything divine, 
\textit{entitled} to everything human, but \textit{owner} of what is yours, 
\textit{i.e.} of all that you possess the force to make your 
own;\footnote{[\textit{Eigen}]} \textit{i.e.} you are 
\textit{appropriate}\footnote{[\textit{geeignet}]} and capacitated for 
everything that is yours.

People have always supposed that they must give me a destiny lying outside 
myself, so that at last they demanded that I should lay claim to the human 
because I am -- man. This is the Christian magic circle. Fichte's ego too is 
the same essence outside me, for every one is ego; and, if only this ego has 
rights, then it is "{}the ego,"{} it is not I. But I am not an ego along with 
other egos, but the sole ego: I am unique. Hence my wants too are unique, and 
my deeds; in short, everything about me is unique. And it is only as this 
unique I that I take everything for my own, as I set myself to work, and 
develop myself, only as this. I do not develop men, nor as man, but, as I, I 
develop -- myself.

This is the meaning of the -- \textit{unique one}.

\chapter[III. The Unique One]{\centering III.\\
THE UNIQUE ONE}

Pre-Christian and Christian times pursue opposite goals; the former wants to 
idealize the real, the latter to realize the ideal; the former seeks the 
"{}holy spirit,"{} the latter the "{}glorified body."{} Hence the former 
closes with insensitivity to the real, with "{}contempt for the world"{}; the 
latter will end with the casting off of the ideal, with "{}contempt for the 
spirit."{}

The opposition of the real and the ideal is an irreconcilable one, and the one 
can never become the other: if the ideal became the real, it would no longer 
be the ideal; and, if the real became the ideal, the ideal alone would be, but 
not at all the real. The opposition of the two is not to be vanquished 
otherwise than if some one annihilates both. Only in this \textit{"{}some 
one},"{} the third party, does the opposition find its end; otherwise idea and 
reality will ever fail to coincide. The idea cannot be so realized as to 
remain idea, but is realized only when it dies as idea; and it is the same 
with the real.

But now we have before us in the ancients adherents of the idea, in the 
moderns adherents of reality. Neither can get clear of the opposition, and 
both pine only, the one party for the spirit, and, when this craving of the 
ancient world seemed to be satisfied and this spirit to have come, the others 
immediately for the secularization of this spirit again, which must forever 
remain a "{}pious wish."{}

The pious wish of the ancients was \textit{sanctity}, the pious wish of the 
moderns is \textit{corporeity}. But, as antiquity had to go down if its 
longing was to be satisfied (for it consisted only in the longing), so too 
corporeity can never be attained within the ring of Christianness. As the 
trait of sanctification or purification goes through the old world (the 
washings, etc.), so that of incorporation goes through the Christian world: 
God plunges down into this world, becomes flesh, and wants to redeem it, 
\textit{e. g.}, fill it with himself; but, since he is "{}the idea"{} or 
"{}the spirit,"{} people (\textit{e. g.} Hegel) in the end introduce the idea 
into everything, into the world, and prove "{}that the idea is, that reason 
is, in everything."{} "{}Man"{} corresponds in the culture of today to what 
the heathen Stoics set up as "{}the wise man"{}; the latter, like the former, 
a -- \textit{fleshless} being. The unreal "{}wise man,"{} this bodiless 
"{}holy one"{} of the Stoics, became a real person, a bodily "{}Holy One,"{} 
in God \textit{made flesh;} the unreal "{}man,"{} the bodiless ego, will 
become real in the \textit{corporeal ego}, in \textit{me}.

There winds its way through Christianity the question about the "{}existence 
of God,"{} which, taken up ever and ever again, gives testimony that the 
craving for existence, corporeity, personality, reality, was incessantly 
busying the heart because it never found a satisfying solution. At last the 
question about the existence of God fell, but only to rise up again in the 
proposition that the "{}divine"{} had existence (Feuerbach). But this too has 
no existence, and neither will the last refuge, that the "{}purely human"{} is 
realizable, afford shelter much longer. No idea has existence, for none is 
capable of corporeity. The scholastic contention of realism and nominalism has 
the same content; in short, this spins itself out through all Christian 
history, and cannot end \textit{in} it.

The world of Christians is working at \textit{realizing ideas} in the 
individual relations of life, the institutions and laws of the Church and the 
State; but they make resistance, and always keep back something unembodied 
(unrealizable). Nevertheless this embodiment is restlessly rushed after, no 
matter in what degree \textit{corporeity} constantly fails to result.

For realities matter little to the realizer, but it matters everything that 
they be realizations of the idea. Hence he is ever examining anew whether the 
realized does in truth have the idea, its kernel, dwelling in it; and in 
testing the real he at the same time tests the idea, whether it is realizable 
as he thinks it, or is only thought by him incorrectly, and for that reason 
unfeasibly.

The Christian is no longer to care for family, State, etc., as 
\textit{existences;} Christians are not to sacrifice themselves for these 
"{}divine things"{} like the ancients, but these are only to be utilized to 
make the \textit{spirit alive} in them. The \textit{real} family has become 
indifferent, and there is to arise out of it an \textit{ideal} one which would 
then be the "{}truly real,"{} a sacred family, blessed by God, or, according 
to the liberal way of thinking, a "{}rational"{} family. With the ancients, 
family, State, fatherland, is divine as a thing \textit{extant;} with the 
moderns it is still awaiting divinity, as extant it is only sinful, earthly, 
and has still to be "{}redeemed,"{} \textit{i. e.}, to become truly real. This 
has the following meaning: The family, etc., is not the extant and real, but 
the divine, the idea, is extant and real; whether \textit{this} family will 
make itself real by taking up the truly real, the idea, is still unsettled. It 
is not the individual's task to serve the family as the divine, but, 
reversely, to serve the divine and to bring to it the still undivine family, 
to subject everything in the idea's name, to set up the idea's banner 
everywhere, to bring the idea to real efficacy.

But, since the concern of Christianity, as of antiquity, is for the 
\textit{divine}, they always come out at this again on their opposite ways. At 
the end of heathenism the divine becomes the \textit{extramundane}, at the end 
of Christianity the \textit{intramundane}. Antiquity does not succeed in 
putting it entirely outside the world, and, when Christianity accomplishes 
this task, the divine instantly longs to get back into the world and wants to 
"{}redeem"{} the world. But within Christianity it does not and cannot come to 
this, that the divine as \textit{intramundane} should really become the 
\textit{mundane itself:} there is enough left that does and must maintain 
itself unpenetrated as the "{}bad,"{} irrational, accidental, "{}egoistic,"{} 
the "{}mundane"{} in the bad sense. Christianity begins with God's becoming 
man, and carries on its work of conversion and redemption through all time in 
order to prepare for God a reception in all men and in everything human, and 
to penetrate everything with the spirit: it sticks to preparing a place for 
the "{}spirit."{}

When the accent was at last laid on Man or mankind, it was again the idea that 
they \textit{"{}pronounced eternal}. "{} "{}Man does not die!"{} They thought 
they had now found the reality of the idea: \textit{Man is} the I of history, 
of the world's history; it is he, this \textit{ideal}, that really develops, 
\textit{i.e. realizes}, himself. He is the really real and corporeal one, for 
history is his body, in which individuals are only members. Christ is the I of 
the world's history, even of the pre-Christian; in modern apprehension it is 
man, the figure of Christ has developed into the \textit{figure of man:} man 
as such, man absolutely, is the "{}central point"{} of history. In "{}man"{} 
the imaginary beginning returns again; for "{}man"{} is as imaginary as Christ 
is. "{}Man,"{} as the I of the world's history, closes the cycle of Christian 
apprehensions.

Christianity's magic circle would be broken if the strained relation between 
existence and calling, \textit{e. g.}, between me as I am and me as I should 
be, ceased; it persists only as the longing of the idea for its bodiliness, 
and vanishes with the relaxing separation of the two: only when the idea 
remains -- idea, as man or mankind is indeed a bodiless idea, is Christianity 
still extant. The corporeal idea, the corporeal or "{}completed"{} spirit, 
floats before the Christian as "{}the end of the days"{} or as the "{}goal of 
history"{}; it is not present time to him.

The individual can only have a part in the founding of the Kingdom of God, or, 
according to the modern notion of the same thing, in the development and 
history of humanity; and only so far as he has a part in it does a Christian, 
or according to the modern expression human, value pertain to him; for the 
rest he is dust and a worm-bag. That the individual is of himself a world's 
history, and possesses his property in the rest of the world's history, goes 
beyond what is Christian. To the Christian the world's history is the higher 
thing, because it is the history of Christ or "{}man"{}; to the egoist only 
\textit{his} history has value, because he wants to develop only 
\textit{himself} not the mankind-idea, not God's plan, not the purposes of 
Providence, not liberty, etc. He does not look upon himself as a tool of the 
idea or a vessel of God, he recognizes no calling, he does not fancy that he 
exists for the further development of mankind and that he must contribute his 
mite to it, but he lives himself out, careless of how well or ill humanity may 
fare thereby. If it were not open to confusion with the idea that a state of 
nature is to be praised, one might recall Lenau's \textit{"{}Three 
Gypsies."{}}- What, am I in the world to realize ideas? To do my part by my 
citizenship, say, toward the realization of the idea "{}State,"{} or by 
marriage, as husband and father, to bring the idea of the family into an 
existence? What does such a calling concern me! I live after a calling as 
little as the flower grows and gives fragrance after a calling.

The ideal "{}Man"{} is \textit{realized} when the Christian apprehension turns 
about and becomes the proposition, "{}I, this unique one, am man."{} The 
conceptual question, "{}what is man?"{} -- has then changed into the personal 
question, "{}who is man?"{} With "{}what"{} the concept was sought for, in 
order to realize it; with "{}who"{} it is no longer any question at all, but 
the answer is personally on hand at once in the asker: the question answers 
itself.

They say of God, "{}Names name thee not."{} That holds good of me: no 
\textit{concept} expresses me, nothing that is designated as my essence 
exhausts me; they are only names. Likewise they say of God that he is perfect 
and has no calling to strive after perfection. That too holds good of me 
alone.

I am \textit{owner} of my might, and I am so when I know myself as 
\textit{unique}. In the \textit{unique one} the owner himself returns into his 
creative nothing, of which he is born. Every higher essence above me, be it 
God, be it man, weakens the feeling of my uniqueness, and pales only before 
the sun of this consciousness. If I concern myself for 
myself,\footnote{[\textit{Stell' Ich auf Mich meine Sache.} Literally, "{}if I 
set my affair on myself."{}]} the unique one, then my concern rests on its 
transitory, mortal creator, who consumes himself, and I may say:

All things are nothing to me.\footnote{[\textit{"{}Ich hab' Mein' Sach' auf 
Nichts gestellt}."{} Literally, "{}I have set my affair on nothing."{} See 
note on p. 8.]}

\begin{center}
THE END\end{center}


\end{document}

