
\chapter[Introduction]{\centering INTRODUCTION}

Fifty years sooner or later can make little difference in the; case of a book 
so revolutionary as this. It saw the light when a so-called revolutionary 
movement was preparing in men's minds which agitation was, however, only a 
disturbance due to desires to participate in government, and to govern and to 
be governed, in a manner different to that which prevails. The 
``revolutionists'' of 1848 were bewitched with an idea. They were not at all 
the masters of ideas. Most of those who since that time have prided themselves 
upon being revolutionists have been and are likewise but the bondmen of an 
idea, ---that of the different lodgment of authority.

The temptation is, of course, present to attempt an explanation of the central 
thought of this work; but such an effort appears to be unnecessary to one who 
has the volume in his hand. The author's care in illustrating his meaning 
shows that he realized how prone the possessed man is to misunderstand 
whatever is not moulded according to the fashions in thinking. The author's 
learning was considerable, his command of words and ideas may never be 
excelled by another, and he judged it needful to develop his argument in 
manifold ways. So those who enter into the spirit of it will scarcely hope to 
impress others with the same conclusion in a more summary manner. Or, if one 
might deem that possible after reading Stirner, still one cannot think that it 
could be done so surely. The author has made certain work of it, even though 
he has to wait for his public; but still, the reception of the book by its 
critics amply proves the truth of the saying that one can give another 
arguments, but not understanding. The system-makers and system-believers thus 
far cannot get it out of their heads that any discourse about the nature of an 
ego must turn upon the common characteristics of egos, to make a systematic 
scheme of what they share as a generality. The critics inquire what kind of 
man the author is talking about. They repeat the question: What does he 
believe in? They fail to grasp the purport of the recorded answer: ``I 
believe in myself''; which is attributed to a common soldier long before the 
time of Stirner. They ask, what is the principle of the self-conscious egoist, 
the Einzige? To this perplexity Stirner says: Change the question; put 
``who?'' instead of ``what?'' and an answer can then be given by naming 
him!

This, of course, is too simple for persons governed by ideas, and for persons 
in quest of new governing ideas. They wish to classify the man. Now, that in 
me which you can classify is not my distinguishing self. ``Man'' is the 
horizon or zero of my existence as an individual. Over that I rise as I can. 
At least I am something more than ``man in general.'' Pre-existing worship 
of ideals and disrespect for self had made of the ego at the very most a 
Somebody, oftener an empty vessel to be filled with the grace or the leavings 
of a tyrannous doctrine; thus a Nobody. Stirner dispels the morbid subjection, 
and recognizes each one who knows and feels himself as his own property to be 
neither humble Nobody nor befogged Somebody, but henceforth flat-footed and 
level-headed Mr. Thisbody, who has a character and good pleasure of his own, 
just as he has a name of his own. The critics who attacked this work and were 
answered in the author's minor writings, rescued from oblivion by John Henry 
Mackay, nearly all display the most astonishing triviality and impotent 
malice.

We owe to Dr. Eduard von Hartmann the unquestionable service which he rendered 
by directing attention to this book in his ``Philosophie des 
Unbewu\ss{}ten,'' the first edition of which was published in 1869, and in 
other writings. I do not begrudge Dr. von Hartmann the liberty of criticism 
which he used; and I think the admirers of Stirner's teaching must quite 
appreciate one thing which Von Hartmann did at a much later date. In ``Der 
Eigene'' of August 10, 1896, there appeared a letter written by him and 
giving, among other things, certain data from which to judge that, when 
Friedrich Nietzsche wrote his later essays, Nietzsche was not ignorant of 
Stirner's book.

Von Hartmann wishes that Stirner had gone on and developed his principle. Von 
Hartmann suggests that you and I are really the same spirit, looking out 
through two pairs of eyes. Then, one may reply, I need not concern myself 
about you, for in myself I have---us; and at that rate Von Hartmann is merely 
accusing himself of inconsistency: for, when Stirner wrote this book, Von 
Hartmann's spirit was writing it; and it is just the pity that Von Hartmann in 
his present form does not indorse what he said in the form of Stirner, ---that 
Stirner was different from any other man; that his ego was not Fichte's 
transcendental generality, but ``this transitory ego of flesh and blood.'' 
It is not as a generality that you and I differ, but as a couple of facts 
which are not to be reasoned into one. ``I'' is somewise Hartmann, and thus 
Hartmann is ``I''; but I am not Hartmann, and Hartmann is not---I. Neither 
am I the ``I'' of Stirner; only Stirner himself was Stirner's ``I.'' Note 
how comparatively indifferent a matter it is with Stirner that one is an ego, 
but how all-important it is that one be a self-conscious ego, ---a 
self-conscious, self-willed person.

Those not self-conscious and self-willed are constantly acting from 
self-interested motives, but clothing these in various garbs. Watch those 
people closely in the light of Stirner's teaching, and they seem to be 
hypocrites, they have so many good moral and religious plans of which 
self-interest is at the end and bottom; but they, we may believe, do not know 
that this is more than a coincidence.

In Stirner we have the philosophical foundation for political liberty. His 
interest in the practical development of egoism to the dissolution of the 
State and the union of free men is clear and pronounced, and harmonizes 
perfectly with the economic philosophy of Josiah Warren. Allowing for 
difference of temperament and language, there is a substantial agreement 
between Stirner and Proudhon. Each would be free, and sees in every increase 
of the number of free people and their intelligence an auxiliary force against 
the oppressor. But, on the other hand, will any one for a moment seriously 
contend that Nietzsche and Proudhon march together in general aim and 
tendency, ---that they have anything in common except the daring to profane 
the shrine and sepulchre of superstition?

Nietzsche has been much spoken of as a disciple of Stirner, and, owing to 
favorable cullings from Nietzsche's writings, it has occurred that one of his 
books has been supposed to contain more sense than it really does---so long 
as one had read only the extracts.

Nietzsche cites scores or hundreds of authors. Had he read everything, and not 
read Stirner?

But Nietzsche is as unlike Stirner as a tight-rope performance is unlike an 
algebraic equation.

Stirner loved liberty for himself, and loved to see any and all men and women 
taking liberty, and he had no lust of power. Democracy to him was sham 
liberty, egoism the genuine liberty.

Nietzsche, on the contrary, pours out his contempt upon democracy because it 
is not aristocratic. He is predatory to the point of demanding that those who 
must succumb to feline rapacity shall be taught to submit with resignation. 
When he speaks of ``Anarchistic dogs'' scouring the streets of great 
civilized cities; it is true, the context shows that he means the Communists; 
but his worship of Napoleon, his bathos of anxiety for the rise of an 
aristocracy that shall rule Europe for thousands of years, his idea of 
treating women in the oriental fashion, show that Nietzsche has struck out in 
a very old path---doing the apotheosis of tyranny. We individual egoistic 
Anarchists, however, may say to the Nietzsche school, so as not to be 
misunderstood: We do not ask of the Napoleons to have pity, nor of the 
predatory barons to do justice. They will find it convenient for their own 
welfare to make terms with men who have learned of Stirner what a man can be 
who worships nothing, bears allegiance to nothing. To Nietzsche's rhodomontade 
of eagles in baronial form, born to prey on industrial lambs, we rather 
tauntingly oppose the ironical question: Where are your claws? What if the 
``eagles'' are found to be plain barn-yard fowls on which more silly fowls 
have fastened steel spurs to hack the victims, who, however, have the power to 
disarm the sham ``eagles'' between two suns? Stirner shows that men make 
their tyrants as they make their gods, and his purpose is to unmake tyrants.

Nietzsche dearly loves a tyrant.

In style Stirner's work offers the greatest possible contrast to the puerile, 
padded phraseology of Nietzsche's ``Zarathustra'' and its false imagery. Who 
ever imagined such an unnatural conjuncture as an eagle ``toting'' a serpent 
in friendship? which performance is told of in bare words, but nothing comes 
of it. In Stirner we are treated to an enlivening and earnest discussion 
addressed to serious minds, and every reader feels that the word is to him, 
for his instruction and benefit, so far as he has mental independence and 
courage to take it and use it. The startling intrepidity of this book is 
infused with a whole-hearted love for all mankind, as evidenced by the fact 
that the author shows not one iota of prejudice or any idea of division of men 
into ranks. He would lay aside government, but would establish any regulation 
deemed convenient, and for this only our convenience in consulted. Thus there 
will be general liberty only when the disposition toward tyranny is met by 
intelligent opposition that will no longer submit to such a rule. Beyond this 
the manly sympathy and philosophical bent of Stirner are such that rulership 
appears by contrast a vanity, an infatuation of perverted pride. We know not 
whether we more admire our author or more love him.

Stirner's attitude toward woman is not special. She is an individual if she 
can be, not handicapped by anything he says, feels, thinks, or plans. This was 
more fully exemplified in his life than even in this book; but there is not a 
line in the book to put or keep woman in an inferior position to man, neither 
is there anything of caste or aristocracy in the book. Likewise there is 
nothing of obscurantism or affected mysticism about it. Everything in it is 
made as plain as the author could make it. He who does not so is not Stirner's 
disciple nor successor nor co-worker. Some one may ask: How does plumb-line 
Anarchism train with the unbridled egoism proclaimed by Stirner? The 
plumb-line is not a fetish, but an intellectual conviction, and egoism is a 
universal fact of animal life. Nothing could seem clearer to my mind than that 
the reality of egoism must first come into the consciousness of men, before we 
can have the unbiased Einzige in place of the prejudiced biped who lends 
himself to the support of tyrannies a million times stronger over me than the 
natural self-interest of any individual. When plumb-line doctrine is 
misconceived as duty between unequal-minded men, ---as a religion of humanity, ---it is indeed the confusion of trying to read without knowing the alphabet 
and of putting philanthropy in place of contract. But, if the plumb-line be 
scientific, it is or can be my possession, my property, and I choose it for 
its use---when circumstances admit of its use. I do not feel bound to use it 
because it is scientific, in building my house; but, as my will, to be 
intelligent, is not to be merely wilful, the adoption of the plumb-line 
follows the discarding of incantations. There is no plumb-line without the 
unvarying lead at the end of the line; not a fluttering bird or a clawing cat.

On the practical side of the question of egoism versus self-surrender and for 
a trial of egoism in politics, this may be said: the belief that men not moved 
by a sense of duty will be unkind or unjust to others is but an indirect 
confession that those who hold that belief are greatly interested in having 
others live for them rather than for themselves. But I do not ask or expect so 
much.

I am content if others individually live for themselves, and thus cease in so 
many ways to act in opposition to my living for myself, ---to our living for 
ourselves.

If Christianity has failed to turn the world from evil, it is not to be 
dreamed that rationalism of a pious moral stamp will succeed in the same task. 
Christianity, or all philanthropic love, is tested in non-resistance. It is a 
dream that example will change the hearts of rulers, tyrants, mobs. If the 
extremest self-surrender fails, how can a mixture of Christian love and 
worldly caution succeed? This at least must be given up. The policy of Christ 
and Tolstoi can soon be tested, but Tolstoi's belief is not satisfied with a 
present test and failure. He has the infatuation of one who persists because 
this ought to be. The egoist who thinks ``I should like this to be'' still 
has the sense to perceive that it is not accomplished by the fact of some 
believing and submitting, inasmuch as others are alert to prey upon the 
unresisting. The Pharaohs we have ever with us.

Several passages in this most remarkable book show the author as a man full of 
sympathy. When we reflect upon his deliberately expressed opinions and 
sentiments, ---his spurning of the sense of moral obligation as the last form 
of superstition, ---may we not be warranted in thinking that the total 
disappearance of the sentimental supposition of duty liberates a quantity of 
nervous energy for the purest generosity and clarifies the intellect for the 
more discriminating choice of objects of merit?

\begin{flushright}
J. L. WALKER.\end{flushright}
