\chapter[II. Men Of The Old And The New]{\centering II.\\
MEN OF THE OLD TIME AND THE NEW}

\medskip{}

\noindent{}How each of us developed himself, what he strove for, attained, or 
missed, what objects he formerly pursued and what plans and wishes his heart 
is now set on, what transformation his views have experienced, what 
perturbations his principles -- in short, how he has today become what 
yesterday or years ago he was not -- this he brings out again from his memory 
with more or less ease, and he feels with especial vividness what changes have 
taken place in himself when he has before his eyes the unrolling of another's 
life.

Let us therefore look into the activities our forefathers busied themselves 
with.

\medskip{}

\section[1. The Ancients]{\centering 1. THE ANCIENTS}

Custom having once given the name of ``the ancients'' to our pre-Christian 
ancestors, we will not throw it up against them that, in comparison with us 
experienced people, they ought properly to be called children, but will rather 
continue to honor them as our good old fathers. But how have they come to be 
antiquated, and who could displace them through his pretended newness?

We know, of course, the revolutionary innovator and disrespectful heir, who 
even took away the sanctity of the fathers' sabbath to hallow his Sunday, and 
interrupted the course of time to begin at himself with a new chronology; we 
know him, and know that it is -- the Christian. But does he remain forever 
young, and is he today still the new man, or will he too be superseded, as he 
has superseded the ``ancients''?

The fathers must doubtless have themselves begotten the young one who entombed 
them. Let us then peep at this act of generation.

``To the ancients the world was a truth,'' says Feuerbach, but he forgets to 
make the important addition, ``a truth whose untruth they tried to get back 
of, and at last really did.'' What is meant by those words of Feuerbach will 
be easily recognized if they are put alongside the Christian thesis of the 
``vanity and transitoriness of the world.'' For, as the Christian can never 
convince himself of the vanity of the divine word, but believes in its eternal 
and unshakable truth, which, the more its depths are searched, must all the 
more brilliantly come to light and triumph, so the ancients on their side 
lived in the feeling that the world and mundane relations (\textit{e.g.} the 
natural ties of blood) were the truth before which their powerless ``I'' 
must bow. The very thing on which the ancients set the highest value is 
spurned by Christians as the valueless, and what they recognized as truth 
these brand as idle lies; the high significance of the fatherland disappears, 
and the Christian must regard himself as ``a stranger on 
earth'';\footnote{Heb. 11. 13.} the sanctity of funeral rites, from which 
sprang a work of art like the Antigone of Sophocles, is designated as a paltry 
thing (``Let the dead bury their dead''); the infrangible truth of family 
ties is represented as an untruth which one cannot promptly enough get clear 
of;\footnote{Mark 10. 29.} and so in everything.

If we now see that to the two sides opposite things appear as truth, to one 
the natural, to the other the intellectual, to one earthly things and 
relations, to the other heavenly (the heavenly fatherland, ``Jerusalem that 
is above,'' etc.), it still remains to be considered how the new time and 
that undeniable reversal could come out of antiquity. But the ancients 
themselves worked toward making their truth a lie.

Let us plunge at once into the midst of the most brilliant years of the 
ancients, into the Periclean century. Then the Sophistic culture was 
spreading, and Greece made a pastime of what had hitherto been to her a 
monstrously serious matter.

The fathers had been enslaved by the undisturbed power of existing things too 
long for the posterity not to have to learn by bitter experience to 
\textit{feel themselves}. Therefore the Sophists, with courageous sauciness, 
pronounce the reassuring words, ``Don't be bluffed!'' and diffuse the 
rationalistic doctrine, ``Use your understanding, your wit, your mind, 
against everything; it is by having a good and well-drilled understanding that 
one gets through the world best, provides for himself the best lot, the most 
pleasant \textit{life.''} Thus they recognize in \textit{mind} man's true 
weapon against the world. This is why they lay such stress on dialectic skill, 
command of language, the art of disputation, etc. They announce that mind is 
to be used against everything; but they are still far removed from the 
holiness of the Spirit, for to them it is a \textit{means}, a weapon, as 
trickery and defiance serve children for the same purpose; their mind is the 
unbribable \textit{understanding}.

Today we should call that a one-sided culture of the understanding, and add 
the warning, ``Cultivate not only your understanding, but also, and 
especially, your heart.'' Socrates did the same. For, if the heart did not 
become free from its natural impulses, but remained filled with the most 
fortuitous contents and, as an uncriticized \textit{avidity}, altogether in 
the power of things, \textit{i.e.} nothing but a vessel of the most various 
\textit{appetites} -- then it was unavoidable that the free understanding must 
serve the ``bad heart'' and was ready to justify everything that the wicked 
heart desired.

Therefore Socrates says that it is not enough for one to use his understanding 
in all things, but it is a question of what \textit{cause} one exerts it for. 
We should now say, one must serve the ``good cause.'' But serving the good 
cause is -- being moral. Hence Socrates is the founder of ethics.

Certainly the principle of the Sophistic doctrine must lead to the possibility 
that the blindest and most dependent slave of his desires might yet be an 
excellent sophist, and, with keen understanding, trim and expound everything 
in favor of his coarse heart. What could there be for which a ``good 
reason'' might not be found, or which might not be defended through thick and 
thin?

Therefore Socrates says: ``You must be 'pure-hearted' if your shrewdness is 
to be valued.'' At this point begins the second period of Greek liberation of 
the mind, the period of \textit{purity of heart}. For the first was brought to 
a close by the Sophists in their proclaiming the omnipotence of the 
understanding. But the heart remained \textit{worldly-minded}, remained a 
servant of the world, always affected by worldly wishes. This coarse heart was 
to be cultivated from now on -- the era of \textit{culture of the heart}. But 
how is the heart to be cultivated? What the understanding; this one side of 
the mind, has reached -- to wit, the capability of playing freely with and 
over every concern -- awaits the heart also; everything \textit{worldly} must 
come to grief before it, so that at last family, commonwealth, fatherland, 
etc., are given up for the sake of the heart, \textit{i.e.}, \textit{of 
blessedness}, the heart's blessedness.

Daily experience confirms the truth that the understanding may have renounced 
a thing many years before the heart has ceased to beat for it. So the 
Sophistic understanding too had so far become master over the dominant, 
ancient powers that they now needed only to be driven out of the heart, in 
which they dwelt unmolested, to have at last no part at all left in man. This 
war is opened by Socrates, and not till the dying day of the old world does it 
end in peace.

The examination of the heart takes its start with Socrates, and all the 
contents of the heart are sifted. In their last and extremest struggles the 
ancients threw all contents out of the heart and let it no longer beat for 
anything; this was the deed of the Skeptics. The same purgation of the heart 
was now achieved in the Skeptical age, as the understanding had succeeded in 
establishing in the Sophistic age.

The Sophistic culture has brought it to pass that one's understanding no 
longer \textit{stands still} before anything, and the Skeptical, that his 
heart is no longer \textit{moved} by anything.

So long as man is entangled in the movements of the world and embarrassed by 
relations to the world -- and he is so till the end of antiquity, because his 
heart still has to struggle for independence from the worldly -- so long he is 
not yet spirit; for spirit is without body, and has no relations to the world 
and corporeality; for it the world does not exist, nor natural bonds, but only 
the spiritual, and spiritual bonds. Therefore man must first become so 
completely unconcerned and reckless, so altogether without relations, as the 
Skeptical culture presents him -- so altogether indifferent to the world that 
even its falling in ruins would not move him -- before he could feel himself 
as worldless; \textit{i.e.}, as spirit. And this is the result of the 
gigantic work of the ancients: that man knows himself as a being without 
relations and without a world, as \textit{spirit}.

Only now, after all worldly care has left him, is he all in all to himself, is 
he only for himself, \textit{i.e.} he is he spirit for the spirit, or, in 
plainer language, he cares only for the spiritual.

In the Christian wisdom of serpents and innocence of doves the two sides -- 
understanding and heart -- of the ancient liberation of mind are so completed 
that they appear young and new again, and neither the one nor the other lets 
itself be bluffed any longer by the worldly and natural.

Thus the ancients mounted to \textit{spirit}, and strove to become 
\textit{spiritual}. But a man who wishes to be active as spirit is drawn to 
quite other tasks than he was able to set himself formerly: to tasks which 
really give something to do to the spirit and not to mere sense or 
acuteness,\footnote{Italicized in the original for the sake of its etymology, 
\textit{Scharfsinn} -- ``sharp-sense''. Compare next paragraph.} which 
exerts itself only to become master \textit{of things}. The spirit busies 
itself solely about the spiritual, and seeks out the ``traces of mind'' in 
everything; to the \textit{believing} spirit ``everything comes from God,'' 
and interests him only to the extent that it reveals this origin; to the 
\textit{philosophic} spirit everything appears with the stamp of reason, and 
interests him only so far as he is able to discover in it reason, \textit{i.e.},
spiritual content.

Not the spirit, then, which has to do with absolutely nothing unspiritual, 
with no \textit{thing}, but only with the essence which exists behind and 
above things, with \textit{thoughts --} not that did the ancients exert, for 
they did not yet have it; no, they had only reached the point of struggling 
and longing for it, and therefore sharpened it against their too-powerful foe, 
the world of sense (but what would not have been sensuous for them, since 
Jehovah or the gods of the heathen were yet far removed from the conception 
``God is \textit{spirit},'' since the ``heavenly fatherland'' had not yet 
stepped into the place of the sensuous, etc.?) -- they sharpened against the 
world of sense their \textit{sense}, their acuteness. To this day the Jews, 
those precocious children of antiquity, have got no farther; and with all the 
subtlety and strength of their prudence and understanding, which easily 
becomes master of things and forces them to obey it, they cannot discover 
\textit{spirit}, which \textit{takes no account whatever of things}.

The Christian has spiritual interests, because he allows himself to be a 
\textit{spiritual} man; the Jew does not even understand these interests in 
their purity, because he does not allow himself to assign \textit{no value} to 
things. He does not arrive at pure \textit{spirituality}, a spirituality 
\textit{e.g.} is religiously expressed, \textit{e.g.}, in the \textit{faith} 
of Christians, which alone (\textit{i.e.} without works) justifies. Their 
\textit{unspirituality} sets Jews forever apart from Christians; for the 
spiritual man is incomprehensible to the unspiritual, as the unspiritual is 
contemptible to the spiritual. But the Jews have only ``the spirit of this 
world.''

The ancient acuteness and profundity lies as far from the spirit and the 
spirituality of the Christian world as earth from heaven.

He who feels himself as free spirit is not oppressed and made anxious by the 
things of this world, because he does not care for them; if one is still to 
feel their burden, he must be narrow enough to attach \textit{weight} to them 
-- as is evidently the case, \textit{e.g.}, when one is still concerned for 
his ``dear life.'' He to whom everything centers in knowing and conducting 
himself as a free spirit gives little heed to how scantily he is supplied 
meanwhile, and does not reflect at all on how he must make his arrangements to 
have a thoroughly inconveniences of the life that depends on things, because 
he lives only spiritually and on spiritual food, while aside from this he only 
gulps things down like a beast, hardly knowing it, and dies bodily, to be 
sure, when his fodder gives out, but knows himself immortal as spirit, and 
closes his eyes with an adoration or a thought. His life is occupation with 
the spiritual, is -- thinking; the rest does not bother him; let him busy 
himself with the spiritual in any way that he can and chooses -- in devotion, 
in contemplation, or in philosophic cognition -- his doing is always thinking; 
and therefore Descartes, to whom this had at last become quite clear, could 
lay down the proposition: ``I think, that is -- I am.'' This means, my 
thinking is my being or my life; only when I live spiritually do I live; only 
as spirit am I really, or -- I am spirit through and through and nothing but 
spirit. Unlucky Peter Schlemihl, who has lost his shadow, is the portrait of 
this man become a spirit; for the spirit's body is shadowless. -- Over against 
this, how different among the ancients! Stoutly and manfully as they might 
bear themselves against the might of things, they must yet acknowledge the 
might itself, and got no farther than to protect their \textit{life} against 
it as well as possible. Only at a late hour did they recognize that their 
``true life'' was not that which they led in the fight against the things of 
the world, but the ``spiritual life,'' ``turned away'' from these things; 
and, when they saw this, they became Christians, \textit{i.e.} the moderns, 
and innovators upon the ancients. But the life turned away from things, the 
spiritual life, no longer draws any nourishment from nature, but ``lives only 
on thoughts,'' and therefore is no longer ``life,'' but -- 
\textit{thinking}.

Yet it must not be supposed now that the ancients were \textit{without 
thoughts}, just as the most spiritual man is not to be conceived of as if he 
could be without life. Rather, they had their thoughts about everything, about 
the world, man, the gods, etc., and showed themselves keenly active in 
bringing all this to their consciousness. But they did not know 
\textit{thought}, even though they thought of all sorts of things and 
``worried themselves with their thoughts.'' Compare with their position the 
Christian saying, ``My thoughts are not your thoughts; as the heaven is 
higher than the earth, so are my thoughts higher than your thoughts,'' and 
remember what was said above about our child-thoughts.

What is antiquity seeking, then? The true \textit{enjoyment of life!} You will 
find that at bottom it is all the same as ``the true life.''

The Greek poet Simonides sings: ``Health is the noblest good for mortal man, 
the next to this is beauty, the third riches acquired without guile, the 
fourth the enjoyment of social pleasures in the company of young friends.'' 
These are all \textit{good things of life}, pleasures of life. What else was 
Diogenes of Sinope seeking for than the true enjoyment of life, which he 
discovered in having the least possible wants? What else Aristippus, who found 
it in a cheery temper under all circumstances? They are seeking for cheery, 
unclouded \textit{life-courage}, for \textit{cheeriness}; they are seeking to 
``be of good \textit{cheer}.''

The Stoics want to realize the \textit{wise man}, the man with 
\textit{practical philosophy}, the man who \textit{knows how to live --} a 
wise life, therefore; they find him in contempt for the world, in a life 
without development, without spreading out, without friendly relations with 
the world, thus in the \textit{isolated life}, in life as life, not in life 
with others; only the Stoic \textit{lives}, all else is dead for him. The 
Epicureans, on the contrary, demand a moving life.

The ancients, as they want to be of good cheer, desire \textit{good living} 
(the Jews especially a long life, blessed with children and goods), 
\textit{eudaemonia}, well-being in the most various forms. Democritus, 
\textit{e.g.}, praises as such the ``calm of the soul'' in which one 
\textit{``lives} smoothly, without fear and without excitement.''

So what he thinks is that with this he gets on best, provides for himself the 
best lot, and gets through the world best. But as he cannot get rid of the 
world -- and in fact cannot for the very reason that his whole activity is 
taken up in the effort to get rid of it, \textit{i.e.}, in \textit{repelling 
the world} (for which it is yet necessary that what can be and is repelled 
should remain existing, otherwise there would be no longer anything to repel) 
-- he reaches at most an extreme degree of liberation, and is distinguishable 
only in degree from the less liberated. If he even got as far as the deadening 
of the earthly sense, which at last admits only the monotonous whisper of the 
word ``Brahm,'' he nevertheless would not be essentially distinguishable 
from the \textit{sensual} man.

Even the stoic attitude and manly virtue amounts only to this -- that one must 
maintain and assert himself against the world; and the ethics of the Stoics 
(their only science, since they could tell nothing about the spirit but how it 
should behave toward the world, and of nature (physics) only this, that the 
wise man must assert himself against it) is not a doctrine of the spirit, but 
only a doctrine of the repelling of the world and of self-assertion against 
the world. And this consists in ``imperturbability and equanimity of life,'' 
and so in the most explicit Roman virtue.

The Romans too (Horace, Cicero, etc.) went no further than this 
\textit{practical philosophy}.

The \textit{comfort} (\textit{hedone}) of the Epicureans is the same 
\textit{practical philosophy} the Stoics teach, only trickier, more deceitful. 
They teach only another \textit{behavior} toward the world, exhort us only to 
take a shrewd attitude toward the world; the world must be deceived, for it is 
my enemy.

The break with the world is completely carried through by the Skeptics. My 
entire relation to the world is ``worthless and truthless.'' Timon says, 
``The feelings and thoughts which we draw from the world contain no truth.'' 
``What is truth?'' cries Pilate. According to Pyrrho's doctrine the world is 
neither good nor bad, neither beautiful nor ugly, etc., but these are 
predicates which I give it. Timon says that ``in itself nothing is either 
good or bad, but man only \textit{thinks} of it thus or thus''; to face the 
world only \textit{ataraxia} (unmovedness) and \textit{aphasia} 
(speechlessness -- or, in other words, isolated \textit{inwardness)} are left. 
There is ``no longer any truth to be recognized'' in the world; things 
contradict themselves; thoughts about things are without distinction (good and 
bad are all the same, so that what one calls good another finds bad); here the 
recognition of ``truth'' is at an end, and only the man \textit{without 
power of recognition}, the \textit{man} who finds in the world nothing to 
recognize, is left, and this man just leaves the truth-vacant world where it 
is and takes no account of it.

So antiquity gets through with the \textit{world of things}, the order of the 
world, the world as a whole; but to the order of the world, or the things of 
this world, belong not only nature, but all relations in which man sees 
himself placed by nature, \textit{e.g.} the family, the community -- in 
short, the so-called ``natural bonds.'' With the \textit{world of the 
spirit} Christianity then begins. The man who still faces the world 
\textit{armed} is the ancient, the -- \textit{heathen} (to which class the 
Jew, too, as non-Christian, belongs); the man who has come to be led by 
nothing but his ``heart's pleasure,'' the interest he takes, his 
fellow-feeling, his --\textit{spirit}, is the modern, the -- Christian.

As the ancients worked toward the \textit{conquest of the world} and strove to 
release man from the heavy trammels of connection with \textit{other things}, 
at last they came also to the dissolution of the State and giving preference 
to everything private. Of course community, family, etc., as \textit{natural} 
relations, are burdensome hindrances which diminish my \textit{spiritual 
freedom.}

\medskip{}

\section[2. The Moderns]{\centering 2. THE MODERNS}

``If any man be in Christ, he is a \textit{new creature}; the old is passed 
away, behold, all is become new.''\footnote{2 Cor. 5. 17. [The words 
``new'' and ``modern'' are the same in German.]}

As it was said above, ``To the ancients the world was a truth,'' we must say 
here, ``To the moderns the spirit was a truth''; but here, as there, we must 
not omit the supplement, ``a truth whose untruth they tried to get back of, 
and at last they really do.''

A course similar to that which antiquity took may be demonstrated in 
Christianity also, in that the \textit{understanding} was held a prisoner 
under the dominion of the Christian dogmas up to the time preparatory to the 
Reformation, but in the pre-Reformation century asserted itself 
\textit{sophistically} and played heretical pranks with all tenets of the 
faith. And the talk then was, especially in Italy and at the Roman court, 
``If only the heart remains Christian-minded, the understanding may go right 
on taking its pleasure.''

Long before the Reformation, people were so thoroughly accustomed to fine-spun 
``wranglings'' that the pope, and most others, looked on Luther's appearance 
too as a mere ``wrangling of monks'' at first. Humanism corresponds to 
Sophisticism, and, as in the time of the Sophists Greek life stood in its 
fullest bloom (the Periclean age), so the most brilliant things happened in 
the time of Humanism, or, as one might perhaps also say, of Machiavellianism 
(printing, the New World, etc.). At this time the heart was still far from 
wanting to relieve itself of its Christian contents.

But finally the Reformation, like Socrates, took hold seriously of the 
\textit{heart} itself, and since then hearts have kept growing visibly -- more 
unchristian. As with Luther people began to take the matter to heart, the 
outcome of this step of the Reformation must be that the heart also gets 
lightened of the heavy burden of Christian faith. The heart, from day to day 
more unchristian, loses the contents with which it had busied itself, till at 
last nothing but empty \textit{warmheartedness} is left it, the quite general 
love of men, the love of \textit{Man}, the consciousness of freedom, 
``self-consciousness.''

Only so is Christianity complete, because it has become bald, withered, and 
void of contents. There are now no contents whatever against which the heart 
does not mutiny, unless indeed the heart unconsciously or without ``self- 
consciousness'' lets them slip in. The heart \textit{criticises} to death 
with \textit{hard-hearted} mercilessness everything that wants to make its way 
in, and is capable (except, as before, unconsciously or taken by surprise) of 
no friendship, no love. What could there be in men to love, since they are all 
alike ``egoists,'' none of them man as such, \textit{i.e.} none 
\textit{spirit only}? The Christian loves only the spirit; but where could one 
be found who should be really nothing but spirit?

To have a liking for the corporeal man with hide and hair -- why, that would 
no longer be a ``spiritual'' warmheartedness, it would be treason against 
``pure'' warmheartedness, the ``theoretical regard.'' For pure 
warmheartedness is by no means to be conceived as like that kindliness that 
gives everybody a friendly hand-shake; on the contrary, pure warmheartedness 
is warm-hearted toward nobody, it is only a theoretical interest, concern for 
man as man, not as a person. The person is repulsive to it because of being 
``egoistic,'' because of not being that abstraction, Man. But it is only for 
the abstraction that one can have a theoretical regard. To pure 
warmheartedness or pure theory men exist only to be criticized, scoffed at, 
and thoroughly despised; to it, no less than to the fanatical parson, they are 
only ``filth'' and other such nice things.

Pushed to this extremity of disinterested warmheartedness, we must finally 
become conscious that the spirit, which alone the Christian loves, is nothing; 
in other words, that the spirit is -- a lie.

What has here been set down roughly, summarily, and doubtless as yet 
incomprehensibly, will, it is to be hoped, become clear as we go on.

Let us take up the inheritance left by the ancients, and, as active workmen, 
do with it as much as -- can be done with it! The world lies despised at our 
feet, far beneath us and our heaven, into which its mighty arms are no longer 
thrust and its stupefying breath does not come. Seductively as it may pose, it 
can delude nothing but our \textit{sense}; it cannot lead astray the spirit -- 
and spirit alone, after all, we really are. Having once got \textit{back of} 
things, the spirit has also got \textit{above} them, and become free from 
their bonds, emancipated, supernal, free. So speaks ``spiritual freedom.''

To the spirit which, after long toil, has got rid of the world, the worldless 
spirit, nothing is left after the loss of the world and the worldly but -- the 
spirit and the spiritual.

Yet, as it has only moved away from the world and made of itself a being 
\textit{free from the world}, without being able really to annihilate the 
world, this remains to it a stumbling-block that cannot be cleared away, a 
discredited existence; and, as, on the other hand, it knows and recognizes 
nothing but the spirit and the spiritual, it must perpetually carry about with 
it the longing to spiritualize the world, \textit{i.e.} to redeem it from the 
``black list.'' Therefore, like a youth, it goes about with plans for the 
redemption or improvement of the world.

The ancients, we saw, served the natural, the worldly, the natural order of 
the world, but they incessantly asked themselves of this service; and, when 
they had tired themselves to death in ever-renewed attempts at revolt, then, 
among their last sighs, was born to them the \textit{God}, the ``conqueror of 
the world.'' All their doing had been nothing but \textit{wisdom of the 
world}, an effort to get back of the world and above it. And what is the 
wisdom of the many following centuries? What did the moderns try to get back 
of? No longer to get back of the world, for the ancients had accomplished 
that; but back of the God whom the ancients bequeathed to them, back of the 
God who ``is spirit,'' back of everything that is the spirit's, the 
spiritual. But the activity of the spirit, which ``searches even the depths 
of the Godhead,'' is \textit{theology}. If the ancients have nothing to show 
but wisdom of the world, the moderns never did nor do make their way further 
than to theology. We shall see later that even the newest revolts against God 
are nothing but the extremest efforts of ``theology,'' \textit{i.e.}, 
theological insurrections.

\subsection[\S{}1. The Spirit]{\centering \S{}1. The Spirit}

The realm of spirits is monstrously great, there is an infinite deal of the 
spiritual; yet let us look and see what the spirit, this bequest of the 
ancients, properly is.

Out of their birth-pangs it came forth, but they themselves could not utter 
themselves as spirit; they could give birth to it, it itself must speak. The 
``born God, the Son of Man,'' is the first to utter the word that the 
spirit, \textit{i.e.} he, God, has to do with nothing earthly and no earthly 
relationship, but solely, with the spirit and spiritual relationships.

Is my courage, indestructible under all the world's blows, my inflexibility 
and my obduracy, perchance already spirit in the full sense, because the world 
cannot touch it? Why, then it would not yet be at enmity with the world, and 
all its action would consist merely in not succumbing to the world! No, so 
long as it does not busy itself with itself alone, so long as it does not have 
to do with \textit{its} world, the spiritual, alone, it is not \textit{free} 
spirit, but only the ``spirit of this world,'' the spirit fettered to it. 
The spirit is free spirit, \textit{i.e.}, really spirit, only in a world of 
\textit{its own}; in ``this,'' the earthly world, it is a stranger. Only 
through a spiritual world is the spirit really spirit, for ``this'' world 
does not understand it and does not know how to keep ``the maiden from a 
foreign land''\footnote{[Title of a poem by Schiller]} from departing.

But where is it to get this spiritual world? Where but out of itself? It must 
reveal itself; and the words that it speaks, the revelations in which it 
unveils itself, these are \textit{its} world. As a visionary lives and has his 
world only in the visionary pictures that he himself creates, as a crazy man 
generates for himself his own dream-world, without which he could not be 
crazy, so the spirit must create for itself its spirit world, and is not 
spirit till it creates it.

Thus its creations make it spirit, and by its creatures we know it, the 
creator; in them it lives, they are its world.

Now, what is the spirit? It is the creator of a spiritual world! Even in you 
and me people do not recognize spirit till they see that we have appropriated 
to ourselves something spiritual, -- \textit{i.e.} though thoughts may have 
been set before us, we have at least brought them to live in ourselves; for, 
as long as we were children, the most edifying thoughts might have been laid 
before us without our wishing, or being able, to reproduce them in ourselves. 
So the spirit also exists only when it creates something spiritual; it is real 
only together with the spiritual, its creature.

As, then, we know it by its works, the question is what these works are. But 
the works or children of the spirit are nothing else but -- spirits.

If I had before me Jews, Jews of the true metal, I should have to stop here 
and leave them standing before this mystery as for almost two thousand years 
they have remained standing before it, unbelieving and without knowledge. But, 
as you, my dear reader, are at least not a full-blooded Jew -- for such a one 
will not go astray as far as this -- we will still go along a bit of road 
together, till perhaps you too turn your back on me because I laugh in your 
face.

If somebody told you were altogether spirit, you would take hold of your body 
and not believe him, but answer: ``I \textit{have} a spirit, no doubt, but do 
not exist only as spirit, but as a man with a body.'' You would still 
distinguish \textit{yourself} from ``your spirit.'' ``But,'' replies he, 
``it is your destiny, even though now you are yet going about in the fetters 
of the body, to be one day a 'blessed spirit,' and, however you may conceive 
of the future aspect of your spirit, so much is yet certain, that in death you 
will put off this body and yet keep yourself, \textit{i.e.} your spirit, for 
all eternity; accordingly your spirit is the eternal and true in you, the body 
only a dwelling here below, which you may leave and perhaps exchange for 
another.''

Now you believe him! For the present, indeed, you are not spirit only; but, 
when you emigrate from the mortal body, as one day you must, then you will 
have to help yourself without the body, and therefore it is needful that you 
be prudent and care in time for your proper self. ``What should it profit a 
man if he gained the whole world and yet suffered damage in his soul?''

But, even granted that doubts, raised in the course of time against the tenets 
of the Christian faith, have long since robbed you of faith in the immortality 
of your spirit, you have nevertheless left one tenet undisturbed, and still 
ingenuously adhere to the one truth, that the spirit is your better part, and 
that the spiritual has greater claims on you than anything else. Despite all 
your atheism, in zeal against \textit{egoism} you concur with the believers in 
immortality.

But whom do you think of under the name of egoist? A man who, instead of 
living to an idea, \textit{i.e.}, a spiritual thing, and sacrificing to it 
his personal advantage, serves the latter. A good patriot brings his sacrifice 
to the altar of the fatherland; but it cannot be disputed that the fatherland 
is an idea, since for beasts incapable of mind,\footnote{[The reader will 
remember (it is to be hoped has never forgotten) that ``mind'' and 
``spirit'' are one and the same word in German. For several pages back the 
connection of the discourse has seemed to require the almost exclusive use of 
the translation ``spirit,'' but to complete the sense it has often been 
necessary that the reader recall the thought of its identity with ``mind,'' 
as stated in a previous note.]} or children as yet without mind, there is no 
fatherland and no patriotism. Now, if any one does not approve himself as a 
good patriot, he betrays his egoism with reference to the fatherland. And so 
the matter stands in innumerable other cases: he who in human society takes 
the benefit of a prerogative sins egoistically against the idea of equality; 
he who exercises dominion is blamed as an egoist against the idea of liberty, 
-- etc.

You despise the egoist because he puts the spiritual in the background as 
compared with the personal, and has his eyes on himself where you would like 
to see him act to favor an idea. The distinction between you is that he makes 
himself the central point, but you the spirit; or that you cut your identity 
in two and exalt your ``proper self,'' the spirit, to be ruler of the 
paltrier remainder, while he will hear nothing of this cutting in two, and 
pursues spiritual and material interests just \textit{as he pleases}. You 
think, to be sure, that you are falling foul of those only who enter into no 
spiritual interest at all, but in fact you curse at everybody who does not 
look on the spiritual interest as his ``true and highest'' interest. You 
carry your knightly service for this beauty so far that you affirm her to be 
the only beauty of the world. You live not to \textit{yourself}, but to your 
\textit{spirit} and to what is the spirit's, \textit{i.e.} ideas.

As the spirit exists only in its creating of the spiritual, let us take a look 
about us for its first creation. If only it has accomplished this, there 
follows thenceforth a natural propagation of creations, as according to the 
myth only the first human beings needed to be created, the rest of the race 
propagating of itself. The first creation, on the other hand, must come forth 
``out of nothing'' -- \textit{i.e.} the spirit has toward its realization 
nothing but itself, or rather it has not yet even itself, but must create 
itself; hence its first creation is itself, \textit{the spirit}. Mystical as 
this sounds, we yet go through it as an every-day experience. Are you a 
thinking being before you think? In creating the first thought you create 
yourself, the thinking one; for you do not think before you think a thought, 
\textit{i.e.} have a thought. Is it not your singing that first makes you a 
singer, your talking that makes you a talker? Now, so too it is the production 
of the spiritual that first makes you a spirit.

Meantime, as you distinguish \textit{yourself} from the thinker, singer, and 
talker, so you no less distinguish yourself from the spirit, and feel very 
clearly that you are something beside spirit. But, as in the thinking ego 
hearing and sight easily vanish in the enthusiasm of thought, so you also have 
been seized by the spirit-enthusiasm, and you now long with all your might to 
become wholly spirit and to be dissolved in spirit. The spirit is your 
\textit{ideal}, the unattained, the other-worldly; spirit is the name of your 
-- god, ``God is spirit.''

Against all that is not spirit you are a zealot, and therefore you play the 
zealot against \textit{yourself} who cannot get rid of a remainder of the 
non-spiritual. Instead of saying, ``I am \textit{more} than spirit,'' you 
say with contrition, ``I am less than spirit; and spirit, pure spirit, or the 
spirit that is nothing but spirit, I can only think of, but am not; and, since 
I am not it, it is another, exists as another, whom I call 'God'.''

It lies in the nature of the case that the spirit that is to exist as pure 
spirit must be an otherworldly one, for, since I am not it, it follows that it 
can only be \textit{outside} me; since in any case a human being is not fully 
comprehended in the concept ``spirit,'' it follows that the pure spirit, the 
spirit as such, can only be outside of men, beyond the human world -- not 
earthly, but heavenly.

Only from this disunion in which I and the spirit lie; only because ``I'' 
and ``spirit'' are not names for one and the same thing, but different names 
for completely different things; only because I am not spirit and spirit not I 
-- only from this do we get a quite tautological explanation of the necessity 
that the spirit dwells in the other world, \textit{i.e.} is God.

But from this it also appears how thoroughly theological is the liberation 
that Feuerbach\footnote{``Essence of Christianity''} is laboring to give us. 
What he says is that we had only mistaken our own essence, and therefore 
looked for it in the other world, but that now, when we see that God was only 
our human essence, we must recognize it again as ours and move it back out of 
the other world into this. To God, who is spirit, Feuerbach gives the name 
``Our Essence.'' Can we put up with this, that ``Our Essence'' is brought 
into opposition to \textit{us} -- that we are split into an essential and an 
unessential self? Do we not therewith go back into the dreary misery of seeing 
ourselves banished out of ourselves?

What have we gained, then, when for a variation we have transferred into 
ourselves the divine outside us? \textit{Are we} that which is in us? As 
little as we are that which is outside us. I am as little my heart as I am my 
sweetheart, this ``other self'' of mine. Just because we are not the spirit 
that dwells in us, just for that reason we had to take it and set it outside 
us; it was not we, did not coincide with us, and therefore we could, not think 
of it as existing otherwise than outside us, on the other side from us, in the 
other world.

With the strength of \textit{despair} Feuerbach clutches at the total 
substance of Christianity, not to throw it away, no, to drag it to himself, to 
draw it, the long-yearned-for, ever-distant, out of its heaven with a last 
effort, and keep it by him forever. Is not that a clutch of the uttermost 
despair, a clutch for life or death, and is it not at the same time the 
Christian yearning and hungering for the other world? The hero wants not to go 
into the other world, but to draw the other world to him, and compel it to 
become this world! And since then has not all the world, with more or less 
consciousness, been crying that ``this world'' is the vital point, and 
heaven must come down on earth and be experienced even here?

Let us, in brief, set Feuerbach's theological view and our contradiction over 
against each other! ``The essence of man is man's supreme 
being;\footnote{[Or, ''highest essence.`` The word \textit{Wesen}, which 
means both ''essence`` and ''being,`` will be translated now one way and 
now the other in the following pages. The reader must bear in mind that these 
two words are identical in German; and so are ''supreme`` and 
''highest.``]} now by religion, to be sure, the \textit{supreme being is} 
called \textit{God} and regarded as an objective essence, but in truth it is 
only man's own essence; and therefore the turning point of the world's history 
is that henceforth no longer \textit{God}, but man, is to appear to man as 
God.''\footnote{Cf. \textit{e.g.} ``Essence of Christianity'', p. 402.}

To this we reply: The supreme being is indeed the essence of man, but, just 
because it is his \textit{essence} and not he himself, it remains quite 
immaterial whether we see it outside him and view it as ``God,'' or find it 
in him and call it ``Essence of Man'' or ``Man.'' I am neither God nor 
Man,\footnote{[That is, the abstract conception of man, as in the preceding 
sentence.]} neither the supreme essence nor my essence, and therefore it is 
all one in the main whether I think of the essence as in me or outside me. 
Nay, we really do always think of the supreme being as in both kinds of 
otherworldliness, the inward and outward, at once; for the ``Spirit of God'' 
is, according to the Christian view, also ``our spirit,'' and ``dwells in 
us.''\footnote{\textit{E.g.}Rom. 8. 9, 1 Cor. 3. 16, John 20. 22 and 
innumerable other passages.} It dwells in heaven and dwells in us; we poor 
things are just its ``dwelling,'' and, if Feuerbach goes on to destroy its 
heavenly dwelling and force it to move to us bag and baggage, then we, its 
earthly apartments, will be badly overcrowded.

But after this digression (which, if we were at all proposing to work by line 
and level, we should have had to save for later pages in order to avoid 
repetition) we return to the spirit's first creation, the spirit itself.

The spirit is something other than myself. But this other, what is it?

\medskip{}

\subsection[\S{}2. The Possessed]{\centering \S{}2. The Possessed.}

Have you ever seen a spirit? ``No, not I, but my grandmother.'' Now, you 
see, it's just so with me too; I myself haven't seen any, but my grandmother 
had them running between her feet all sorts of ways, and out of confidence in 
our grandmothers' honesty we believe in the existence of spirits.

But had we no grandfathers then, and did they not shrug their shoulders every 
time our grandmothers told about their ghosts? Yes, those were unbelieving men 
who have harmed our good religion much, those rationalists! We shall feel 
that! What else lies at the bottom of this warm faith in ghosts, if not the 
faith in ``the existence of spiritual beings in general,'' and is not this 
latter itself disastrously unsettled if saucy men of the understanding may 
disturb the former? The Romanticists were quite conscious what a blow the very 
belief in God suffered by the laying aside of the belief in spirits or ghosts, 
and they tried to help us out of the baleful consequences not only by their 
reawakened fairy world, but at last, and especially, by the ``intrusion of a 
higher world,'' by their somnambulists of Prevorst, etc. The good believers 
and fathers of the church did not suspect that with the belief in ghosts the 
foundation of religion was withdrawn, and that since then it had been floating 
in the air. He who no longer believes in any ghost needs only to travel on 
consistently in his unbelief to see that there is no separate being at all 
concealed behind things, no ghost or -- what is naively reckoned as synonymous 
even in our use of words -- no \textit{``spirit.''}

``Spirits exist!'' Look about in the world, and say for yourself whether a 
spirit does not gaze upon you out of everything. Out of the lovely little 
flower there speaks to you the spirit of the Creator, who has shaped it so 
wonderfully; the stars proclaim the spirit that established their order; from 
the mountain-tops a spirit of sublimity breathes down; out of the waters a 
spirit of yearning murmurs up; and -- out of men millions of spirits speak. 
The mountains may sink, the flowers fade, the world of stars fall in ruins, 
the men die -- what matters the wreck of these visible bodies? The spirit, the 
``invisible spirit,'' abides eternally!

Yes, the whole world is haunted! Only is haunted? Nay, it itself ``walks,'' 
it is uncanny through and through, it is the wandering seeming-body of a 
spirit, it is a spook. What else should a ghost be, then, than an apparent 
body, but real spirit? Well, the world is ``empty,'' is ``naught,'' is 
only glamorous ``semblance''; its truth is the spirit alone; it is the 
seeming-body of a spirit.

Look out near or far, a \textit{ghostly} world surrounds you everywhere; you 
are always having ``apparitions'' or visions. Everything that appears to you 
is only the phantasm of an indwelling spirit, is a ghostly ``apparition''; 
the world is to you only a ``world of appearances,'' behind which the spirit 
walks. You ``see spirits.''

Are you perchance thinking of comparing yourself with the ancients, who saw 
gods everywhere? Gods, my dear modern, are not spirits; gods do not degrade 
the world to a semblance, and do not spiritualize it.

But to you the whole world is spiritualized, and has become an enigmatical 
ghost; therefore do not wonder if you likewise find in yourself nothing but a 
spook. Is not your body haunted by your spirit, and is not the latter alone 
the true and real, the former only the ``transitory, naught'' or a 
``semblance''? Are we not all ghosts, uncanny beings that wait for 
``deliverance'' -- to wit, ``spirits''?

Since the spirit appeared in the world, since ``the Word became flesh,'' 
since then the world has been spiritualized, enchanted, a spook.

You have spirit, for you have thoughts. What are your thoughts? ``Spiritual 
entities.'' Not things, then? ``No, but the spirit of things, the main point 
in all things, the inmost in them, their -- idea.'' Consequently what you 
think is not only your thought?

``On the contrary, it is that in the world which is most real, that which is 
properly to be called true; it is the truth itself; if I only think truly, I 
think the truth. I may, to be sure, err with regard to the truth, and 
\textit{fail to recognize} it; but, if I \textit{recognize} truly, the object 
of my cognition is the truth.'' So, I suppose, you strive at all times to 
recognize the truth? ``To me the truth is sacred. It may well happen that I 
find a truth incomplete and replace it with a better, but \textit{the} truth I 
cannot abrogate. I \textit{believe} in the truth, therefore I search in it; 
nothing transcends it, it is eternal.''

Sacred, eternal is the truth; it is the Sacred, the Eternal. But you, who let 
yourself be filled and led by this sacred thing, are yourself hallowed. 
Further, the sacred is not for your senses -- and you never as a sensual man 
discover its trace -- but for your faith, or, more definitely still, for your 
\textit{spirit}; for it itself, you know, is a spiritual thing, a spirit -- is 
spirit for the spirit.

The sacred is by no means so easily to be set aside as many at present affirm, 
who no longer take this ``unsuitable'' word into their mouths. If even in a 
single respect I am still \textit{upbraided} as an ``egoist,'' there is left 
the thought of something else which I should serve more than myself, and which 
must be to me more important than everything; in short, somewhat in which I 
should have to seek my true welfare,\footnote{[Heil]} something -- 
``sacred.''\footnote{[heiling]} However human this sacred thing may look, 
though it be the Human itself, that does not take away its sacredness, but at 
most changes it from an unearthly to an earthly sacred thing, from a divine 
one to a human.

Sacred things exist only for the egoist who does not acknowledge himself, the 
\textit{involuntary egoist}, for him who is always looking after his own and 
yet does not count himself as the highest being, who serves only himself and 
at the same time always thinks he is serving a higher being, who knows nothing 
higher than himself and yet is infatuated about something higher; in short, 
for the egoist who would like not to be an egoist, and abases himself 
(\textit{i.e.} combats his egoism), but at the same time abases himself only 
for the sake of ``being exalted,'' and therefore of gratifying his egoism. 
Because he would like to cease to be an egoist, he looks about in heaven and 
earth for higher beings to serve and sacrifice himself to; but, however much 
he shakes and disciplines himself, in the end he does all for his own sake, 
and the disreputable egoism will not come off him. On this account I call him 
the involuntary egoist.

His toil and care to get away from himself is nothing but the misunderstood 
impulse to self-dissolution. If you are bound to your past hour, if you must 
babble today because you babbled yesterday,\footnote{[How the priests tinkle! 
how important they\\
 Would make it out, that men should come their way\\
 And babble, just as yesterday, today!

Oh, blame them not! They know man's need, I say!\\
 For he takes all his happiness this way,\\
 To babble just tomorrow as today.

Translated from Goethe's ``Venetian Epigrams.'']

} if you cannot transform yourself each instant, you feel yourself fettered in 
slavery and benumbed. Therefore over each minute of your existence a fresh 
minute of the future beckons to you, and, developing yourself, you get away 
``from yourself,'' \textit{i.e.}, from the self that was at that moment. As 
you are at each instant, you are your own creature, and in this very 
``creature'' you do not wish to lose yourself, the creator. You are yourself 
a higher being than you are, and surpass yourself. But that you are the one 
who is higher than you, \textit{i.e.}, that you are not only creature, but 
likewise your creator -- just this, as an involuntary egoist, you fail to 
recognize; and therefore the ``higher essence'' is to you -- an 
alien\footnote{[\textit{fremd}]} essence. Every higher essence, \textit{e.g.} 
truth, mankind, etc., is an essence \textit{over} us.

Alienness is a criterion of the ``sacred.'' In everything sacred there lies 
something ``uncanny,'' \textit{i.e.} strange,\footnote{[\textit{fremd}]} 
\textit{e.g.} we are not quite familiar and at home in. What is sacred to me 
is \textit{not my own}; and if, \textit{e.g.,}, the property of others was 
not sacred to me, I should look on it as \textit{mine}, which I should take to 
myself when occasion offered. Or, on the other side, if I regard the face of 
the Chinese emperor as sacred, it remains strange to my eye, which I close at 
its appearance.

Why is an incontrovertible mathematical truth, which might even be called 
eternal according to the common understanding of words, not -- sacred? Because 
it is not revealed, or not the revelation of, a higher being. If by revealed 
we understand only the so-called religious truths, we go far astray, and 
entirely fail to recognize the breadth of the concept ``higher being.'' 
Atheists keep up their scoffing at the higher being, which was also honored 
under the name of the ``highest'' or \textit{\^Etre supr\^eme}, and trample 
in the dust one ``proof of his existence'' after another, without noticing 
that they themselves, out of need for a higher being, only annihilate the old 
to make room for a new. Is ``Man'' perchance not a higher essence than an 
individual man, and must not the truths, rights, and ideas which result from 
the concept of him be honored and --counted sacred, as revelations of this 
very concept? For, even though we should abrogate again many a truth that 
seemed to be made manifest by this concept, yet this would only evince a 
misunderstanding on our part, without in the least degree harming the sacred 
concept itself or taking their sacredness from those truths that must 
``rightly'' be looked upon as its revelations. \textit{Man} reaches beyond 
every individual man, and yet -- though he be ``his essence'' -- is not in 
fact \textit{his} essence (which rather would be as 
single\footnote{[\textit{einzig}]} as he the individual himself), but a 
general and ``higher,'' yes, for atheists ``the highest 
essence.''\footnote{[\textit{``the supreme being}.'']} And, as the divine 
revelations were not written down by God with his own hand, but made public 
through ``the Lord's instruments,'' so also the new highest essence does not 
write out its revelations itself, but lets them come to our knowledge through 
``true men.'' Only the new essence betrays, in fact, a more spiritual style 
of conception than the old God, because the latter was still represented in a 
sort of embodiedness or form, while the undimmed spirituality of the new is 
retained, and no special material body is fancied for it. And withal it does 
not lack corporeity, which even takes on a yet more seductive appearance 
because it looks more natural and mundane and consists in nothing less than in 
every bodily man -- yes, or outright in ``humanity'' or ``all men.'' 
Thereby the spectralness of the spirit in a seeming body has once again become 
really solid and popular.

Sacred, then, is the highest essence and everything in which this highest 
essence reveals or will reveal itself; but hallowed are they who recognize 
this highest essence together with its own, \textit{i.e.} together with its 
revelations. The sacred hallows in turn its reverer, who by his worship 
becomes himself a saint, as Likewise what he does is saintly, a saintly walk, 
saintly thoughts and actions, imaginations and aspirations.

It is easily understood that the conflict over what is revered as the highest 
essence can be significant only so long as even the most embittered opponents 
concede to each other the main point -- that there is a highest essence to 
which worship or service is due. If one should smile compassionately at the 
whole struggle over a highest essence, as a Christian might at the war of 
words between a Shiite and a Sunnite or between a Brahman and a Buddhist, then 
the hypothesis of a highest essence would be null in his eyes, and the 
conflict on this basis an idle play. Whether then the one God or the three in 
one. whether the Lutheran God or the \textit{\^Etre supr\^eme} or not God at 
all, but ``Man,'' may represent the highest essence, that makes no 
difference at all for him who denies the highest essence itself, for in his 
eyes those servants of a highest essence are one and all-pious people, the 
most raging atheist not less than the most faith-filled Christian.

In the foremost place of the sacred,\footnote{[\textit{heilig}]} then, stands 
the highest essence and the faith in this essence, our 
``holy\footnote{[\textit{heilig}]} faith.''

\medskip{}

\subsection[The Spook]{\centering The Spook}

With ghosts we arrive in the spirit-realm, in the realm of \textit{essences}.

What haunts the universe, and has its occult, ``incomprehensible'' being 
there, is precisely the mysterious spook that we call highest essence. And to 
get to the bottom of this \textit{spook}, to comprehend it, to discover 
\textit{reality} in it (to prove ``the existence of God'') -- this task men 
set to themselves for thousands of years; with the horrible impossibility, the 
endless Danaid-labor, of transforming the spook into a non-spook, the unreal 
into something real, the \textit{spirit} into an entire and \textit{corporeal} 
person -- with this they tormented themselves to death. Behind the existing 
world they sought the ``thing in itself,'' the essence; behind the 
\textit{thing} they sought the \textit{un-thing}.

When one looks to the \textit{bottom} of anything, \textit{i.e.} searches out 
its \textit{essence}, one often discovers something quite other than what it 
\textit{seems} to be; honeyed speech and a lying heart, pompous words and 
beggarly thoughts, etc. By bringing the essence into prominence one degrades 
the hitherto misapprehended appearance to a bare \textit{semblance}, a 
deception. The essence of the world, so attractive and splendid, is for him 
who looks to the bottom of it -- emptiness; emptiness is = world's essence 
(world's doings). Now, he who is religious does not occupy himself with the 
deceitful semblance, with the empty appearances, but looks upon the essence, 
and in the essence has -- the truth.

The essences which are deduced from some appearances are the evil essences, 
and conversely from others the good. The essence of human feeling, \textit{e. 
g.}, is love; the essence of human will is the good; that of one's thinking, 
the true, etc.

What at first passed for existence, \textit{e.g.} the world and its like, 
appears now as bare semblance, and the \textit{truly existent} is much rather 
the essence, whose realm is filled with gods, spirits, demons, with good or 
bad essences. Only this inverted world, the world of essences, truly exists 
now. The human heart may be loveless, but its essence exists, God, ``who is 
love''; human thought may wander in error, but its essence, truth, exists; 
``God is truth,'' and the like.

To know and acknowledge essences alone and nothing but essences, that is 
religion; its realm is a realm of essences, spooks, and ghosts.

The longing to make the spook comprehensible, or to realize 
\textit{non-sense}, has brought about a \textit{corporeal ghost}, a ghost or 
spirit with a real body, an embodied ghost. How the strongest and most 
talented Christians have tortured themselves to get a conception of this 
ghostly apparition! But there always remained the contradiction of two 
natures, the divine and human, \textit{i.e.,} the ghostly and sensual; there 
remained the most wondrous spook, a thing that was not a thing. Never yet was 
a ghost more soul torturing, and no shaman, who pricks himself to raving fury 
and nerve-lacerating cramps to conjure a ghost, can endure such soul-torment 
as Christians suffered from that most incomprehensible ghost.

But through Christ the truth of the matter had at the same time come to light, 
that the veritable spirit or ghost is -- man. The \textit{corporeal} or 
embodied spirit is just man; he himself is the ghostly being and at the same 
time the being's appearance and existence. Henceforth man no longer, in 
typical cases, shudders at ghosts \textit{outside} him, but at himself; he is 
terrified at himself. In the depth of his breast dwells the \textit{spirit of 
sin}; even the faintest thought (and this is itself a spirit, you know) may be 
a \textit{devil}, etc. -- The ghost has put on a body, God has become man, but 
now man is himself the gruesome spook which he seeks to get back of, to 
exorcise, to fathom, to bring to reality and to speech; man is -- 
\textit{spirit}. What matter if the body wither, if only the spirit is saved? 
Everything rests on the spirit, and the spirit's or ``soul's'' welfare 
becomes the exclusive goal. Man has become to himself a ghost, an uncanny 
spook, to which there is even assigned a distinct seat in the body (dispute 
over the seat of the soul, whether in the head, etc.).

You are not to me, and I am not to you, a higher essence. Nevertheless a 
higher essence may be hidden in each of us, and call forth a mutual reverence. 
To take at once the most general, Man lives in you and me. If I did not see 
Man in you, what occasion should I have to respect you? To be sure, you are 
not Man and his true and adequate form, but only a mortal veil of his, from 
which he can withdraw without himself ceasing; but yet for the present this 
general and higher essence is housed in you, and you present before me 
(because an imperishable spirit has in you assumed a perishable body, so that 
really your form is only an ``assumed'' one) a spirit that appears, appears 
in you, without being bound to your body and to this particular mode of 
appearance -- therefore a spook. Hence I do not regard you as a higher essence 
but only respect that higher essence which ``walks'' in you; I ``respect 
Man in you.'' The ancients did not observe anything of this sort in their 
slaves, and the higher essence ``Man'' found as yet little response. To make 
up for this, they saw in each other ghosts of another sort. The People is a 
higher essence than an individual, and, like Man or the Spirit of Man, a 
spirit haunting the individual -- the Spirit of the People. For this reason 
they revered this spirit, and only so far as he served this or else a spirit 
related to it (\textit{e.g.} the Spirit of the Family) could the individual 
appear significant; only for the sake of the higher essence, the People, was 
consideration allowed to the ``member of the people.'' As you are hallowed 
to us by ``Man'' who haunts you, so at every time men have been hallowed by 
some higher essence or other, like People, Family, and such. Only for the sake 
of a higher essence has any one been honored from of old, only as a ghost has 
he been regarded in the light of a hallowed, \textit{i.e.}, protected and 
recognized person. If I cherish you because I hold you dear, because in you my 
heart finds nourishment, my need satisfaction, then it is not done for the 
sake of a higher essence, whose hallowed body you are, not on account of my 
beholding in you a ghost, \textit{i.e.} an appearing spirit, but from egoistic 
pleasure; you yourself with \textit{your} essence are valuable to me, for your 
essence is not a higher one, is not higher and more general than you, is 
unique\footnote{[\textit{einzig}]} like you yourself, because it is you.

But it is not only man that ``haunts''; so does everything. The higher 
essence, the spirit, that walks in everything, is at the same time bound to 
nothing, and only -- ``appears'' in it. Ghosts in every corner!

Here would be the place to pass the haunting spirits in review, if they were 
not to come before us again further on in order to vanish before egoism. Hence 
let only a few of them be particularized by way of example, in order to bring 
us at once to our attitude toward them.

Sacred above all, \textit{e.g.}, is the ``holy Spirit,'' sacred the truth, 
sacred are right, law, a good cause, majesty, marriage, the common good, 
order, the fatherland, etc.

\subsection[Wheels In The Head]{\centering Wheels In The Head}

Man, your head is haunted; you have wheels in your head! You imagine great 
things, and depict to yourself a whole world of gods that has an existence for 
you, a spirit-realm to which you suppose yourself to be called, an ideal that 
beckons to you. You have a fixed idea!

Do not think that I am jesting or speaking figuratively when I regard those 
persons who cling to the Higher, and (because the vast majority belongs under 
this head) almost the whole world of men, as veritable fools, fools in a 
madhouse. What is it, then, that is called a ``fixed idea''? An idea that 
has subjected the man to itself. When you recognize, with regard to such a 
fixed idea, that it is a folly, you shut its slave up in an asylum. And is the 
truth of the faith, say, which we are not to doubt; the majesty of (\textit{e. 
g.}) the people, which we are not to strike at (he who does is guilty of -- 
lese-majesty); virtue, against which the censor is not to let a word pass, 
that morality may be kept pure; -- are these not ``fixed ideas''? Is not all 
the stupid chatter of (\textit{e.g.}) most of our newspapers the babble of 
fools who suffer from the fixed idea of morality, legality, Christianity, 
etc., and only seem to go about free because the madhouse in which they walk 
takes in so broad a space? Touch the fixed idea of such a fool, and you will 
at once have to guard your back against the lunatic's stealthy malice. For 
these great lunatics are like the little so-called lunatics in this point too 
-- that they assail by stealth him who touches their fixed idea. They first 
steal his weapon, steal free speech from him, and then they fall upon him with 
their nails. Every day now lays bare the cowardice and vindictiveness of these 
maniacs, and the stupid populace hurrahs for their crazy measures. One must 
read the journals of this period, and must hear the Philistines talk, to get 
the horrible conviction that one is shut up in a house with fools. ``Thou 
shalt not call thy brother a fool; if thou dost -- etc.'' But I do not fear 
the curse, and I say, my brothers are arch-fools. Whether a poor fool of the 
insane asylum is possessed by the fancy that he is God the Father, Emperor of 
Japan, the Holy Spirit, etc., or whether a citizen in comfortable 
circumstances conceives that it is his mission to be a good Christian, a 
faithful Protestant, a loyal citizen, a virtuous man -- both these are one and 
the same ``fixed idea.'' He who has never tried and dared not to be a good 
Christian, a faithful Protestant, a virtuous man, etc., is \textit{possessed} 
and prepossessed\footnote{[\textit{gefangen und befangen}, literally 
``imprisoned and prepossessed.'']} by faith, virtuousness, etc. Just as the 
schoolmen philosophized only \textit{inside} the belief of the church; as Pope 
Benedict XIV wrote fat books \textit{inside} the papist superstition, without 
ever throwing a doubt upon this belief; as authors fill whole folios on the 
State without calling in question the fixed idea of the State itself; as our 
newspapers are crammed with politics because they are conjured into the fancy 
that man was created to be a \textit{zoon politicon} -- so also subjects 
vegetate in subjection, virtuous people in virtue, liberals in humanity, 
without ever putting to these fixed ideas of theirs the searching knife of 
criticism. Undislodgeable, like a madman's delusion, those thoughts stand on a 
firm footing, and he who doubts them -- lays hands on the \textit{sacred!} 
Yes, the ``fixed idea,'' that is the truly sacred!

Is it perchance only people possessed by the devil that meet us, or do we as 
often come upon people \textit{possessed} in the contrary way -- possessed by 
``the good,'' by virtue, morality, the law, or some ``principle'' or 
other? Possessions of the devil are not the only ones. God works on us, and 
the devil does; the former ``workings of grace,'' the latter ``workings of 
the devil.'' Possessed\footnote{[\textit{besessene}]} people are 
set\footnote{[\textit{versessen}]} in their opinions.

If the word ``possession'' displeases you, then call it prepossession; yes, 
since the spirit possesses you, and all ``inspirations'' come from it, call 
it -- inspiration and enthusiasm. I add that complete enthusiasm -- for we 
cannot stop with the sluggish, half- way kind -- is called fanaticism.

It is precisely among cultured people that \textit{fanaticism} is at home; for 
man is cultured so far as he takes an interest in spiritual things, and 
interest in spiritual things, when it is alive, is and must be 
\textit{fanaticism}; it is a fanatical interest in the sacred 
\textit{(fanum)}. Observe our liberals, look into the \textit{S\"achsischen 
Vaterlandsbl\"atter}, hear what Schlosser 
says:\footnote{\textit{``Achtzehntes Jahrhundert}'', II, 519.} ``Holbach's 
company constituted a regular plot against the traditional doctrine and the 
existing system, and its members were as fanatical on behalf of their unbelief 
as monks and priests, Jesuits and Pietists, Methodists, missionary and Bible 
societies, commonly are for mechanical worship and orthodoxy.''

Take notice how a ``moral man'' behaves, who today often thinks he is 
through with God and throws off Christianity as a bygone thing. If you ask him 
whether he has ever doubted that the copulation of brother and sister is 
incest, that monogamy is the truth of marriage, that filial piety is a sacred 
duty, then a moral shudder will come over him at the conception of one's being 
allowed to touch his sister as wife also, etc. And whence this shudder? 
Because he \textit{believes} in those moral commandments. This moral 
\textit{faith} is deeply rooted in his breast. Much as he rages against the 
\textit{pious} Christians, he himself has nevertheless as thoroughly remained 
a Christian -- to wit, a \textit{moral} Christian. In the form of morality 
Christianity holds him a prisoner, and a prisoner under \textit{faith}. 
Monogamy is to be something sacred, and he who may live in bigamy is punished 
as a \textit{criminal}; he who commits incest suffers as a \textit{criminal}. 
Those who are always crying that religion is not to be regarded in the State, 
and the Jew is to be a citizen equally with the Christian, show themselves in 
accord with this. Is not this of incest and monogamy a \textit{dogma of 
faith?} Touch it, and you will learn by experience how this moral man is a 
\textit{hero of faith} too, not less than Krummacher, not less than Philip II. 
These fight for the faith of the Church, he for the faith of the State, or the 
moral laws of the State; for articles of faith, both condemn him who acts 
otherwise than \textit{their faith will} allow. The brand of ``crime'' is 
stamped upon him, and he may languish in reformatories, in jails. Moral faith 
is as fanatical as religious faith! They call that ``liberty of faith'' 
then, when brother and sister, on account of a relation that they should have 
settled with their ``conscience,'' are thrown into prison. ``But they set a 
pernicious example.'' Yes, indeed: others might have taken the notion that 
the State had no business to meddle with their relation, and thereupon 
``purity of morals'' would go to ruin. So then the religious heroes of faith 
are zealous for the ``sacred God,'' the moral ones for the ``sacred 
good.''

Those who are zealous for something sacred often look very little like each 
other. How the strictly orthodox or old-style believers differ from the 
fighters for ``truth, light, and justice,'' from the Philalethes, the 
Friends of Light, the Rationalists, and others. And yet, how utterly 
unessential is this difference! If one buffets single traditional truths 
(\textit{i.e.} miracles, unlimited power of princes), then the Rationalists 
buffet them too, and only the old-style believers wail. But, if one buffets 
truth itself, he immediately has both, as \textit{believers}, for opponents. 
So with moralities; the strict believers are relentless, the clearer heads are 
more tolerant. But he who attacks morality itself gets both to deal with. 
``Truth, morality, justice, light, etc.,'' are to be and remain 
``sacred.'' What any one finds to censure in Christianity is simply supposed 
to be ``unchristian'' according to the view of these rationalists, but 
Christianity must remain a ``fixture,'' to buffet it is outrageous, ``an 
outrage.'' To be sure, the heretic against pure faith no longer exposes 
himself to the earlier fury of persecution, but so much the more does it now 
fall upon the heretic against pure morals.

\myhrule


Piety has for a century received so many blows, and had to hear its superhuman 
essence reviled as an ``inhuman'' one so often, that one cannot feel tempted 
to draw the sword against it again. And yet it has almost always been only 
moral opponents that have appeared in the arena, to assail the supreme essence 
in favor of -- another supreme essence. So Proudhon, unabashed, 
says:\footnote{\textit{``De la Cr\'eation de l'Ordre}'' etc., p. 36.} ``Man 
is destined to live without religion, but the moral law is eternal and 
absolute. Who would dare today to attack morality?'' Moral people skimmed off 
the best fat from religion, ate it themselves, and are now having a tough job 
to get rid of the resulting scrofula. If, therefore, we point out that 
religion has not by any means been hurt in its inmost part so long as people 
reproach it only with its superhuman essence, and that it takes its final 
appeal to the ``spirit'' alone (for God is spirit), then we have 
sufficiently indicated its final accord with morality, and can leave its 
stubborn conflict with the latter lying behind us. It is a question of a 
supreme essence with both, and whether this is a superhuman or a human one can 
make (since it is in any case an essence over me, a super-mine one, so to 
speak) but little difference to me. In the end the relation to the human 
essence, or to ``Man,'' as soon as ever it has shed the snake-skin of the 
old religion, will yet wear a religious snake-skin again.

So Feuerbach instructs us that, ``if one only \textit{inverts} speculative 
philosophy, \textit{i.e.} always makes the predicate the subject, and so makes 
the subject the object and principle, one has the undraped truth, pure and 
clean.''\footnote{\textit{``Anekdota''}, II, 64.} Herewith, to be sure, we 
lose the narrow religious standpoint, lost the \textit{God}, who from this 
standpoint is subject; but we take in exchange for it the other side of the 
religious standpoint, the \textit{moral} standpoint. Thus we no longer say 
``God is love,'' but ``Love is divine.'' If we further put in place of the 
predicate ``divine'' the equivalent ``sacred,'' then, as far as concerns 
the sense, all the old comes back-again. According to this, love is to be the 
\textit{good} in man, his divineness, that which does him honor, his true 
\textit{humanity} (it ``makes him Man for the first time,'' makes for the 
first time a man out of him). So then it would be more accurately worded thus: 
Love is what is \textit{human} in man, and what is inhuman is the loveless 
egoist. But precisely all that which Christianity and with it speculative 
philosophy (\textit{i.e.}, theology) offers as the good, the absolute, is to 
self-ownership simply not the good (or, what means the same, it is 
\textit{only the good)}. Consequently, by the transformation of the predicate 
into the subject, the Christian \textit{essence} (and it is the predicate that 
contains the essence, you know) would only be fixed yet more oppressively. God 
and the divine would entwine themselves all the more inextricably with me. To 
expel God from his heaven and to rob him of his \textit{``transcendence''} 
cannot yet support a claim of complete victory, if therein he is only chased 
into the human breast and gifted with indelible \textit{immanence}. Now they 
say, ``The divine is the truly human!''

The same people who oppose Christianity as the basis of the State, 
\textit{i.e.} oppose the so-called Christian State, do not tire of repeating 
that morality is ``the fundamental pillar of social life and of the State.'' 
As if the dominion of morality were not a complete dominion of the sacred, a 
``hierarchy.''

So we may here mention by the way that rationalist movement which, after 
theologians had long insisted that only faith was capable of grasping 
religious truths, that only to believers did God reveal himself, and that 
therefore only the heart, the feelings, the believing fancy was religious, 
broke out with the assertion that the ``natural understanding,'' human 
reason, was also capable of discerning God. What does that mean but that the 
reason laid claim to be the same visionary as the 
fancy?\footnote{[\textit{dieselbe Phantastin wie die Phantasie.}]} In this 
sense Reimarus wrote his \textit{Most Notable Truths of Natural Religion}. It 
had to come to this -- that the \textit{whole} man with all his faculties was 
found to be \textit{religious}; heart and affections, understanding and 
reason, feeling, knowledge, and will -- in short, everything in man -- 
appeared religious. Hegel has shown that even philosophy is religious. And 
what is not called religion today? The ``religion of love,'' the ``religion 
of freedom,'' ``political religion'' -- in short, every enthusiasm. So it 
is, too, in fact.

To this day we use the Romance word ``religion,'' which expresses the 
concept of a condition of being \textit{bound}. To be sure, \textit{we} remain 
bound, so far as religion takes possession of our inward parts; but is the 
mind also bound? On the contrary, that is free, is sole lord, is not our mind, 
but absolute. Therefore the correct affirmative translation of the word 
religion would be \textit{``freedom of mind''}! In whomsoever the mind is 
free, he is religious in just the same way as he in whom the senses have free 
course is called a sensual man. The mind binds the former, the desires the 
latter. Religion, therefore, is boundness or \textit{religion} with reference 
to me -- I am bound; it is freedom with reference to the mind -- the mind is 
free, or has freedom of mind. Many know from experience how hard it is on 
\textit{us} when the desires run away with us, free and unbridled; but that 
the free mind, splendid intellectuality, enthusiasm for intellectual 
interests, or however this jewel may in the most various phrase be named, 
brings \textit{us} into yet more grievous straits than even the wildest 
impropriety, people will not perceive; nor can they perceive it without being 
consciously egoists.

Reimarus, and all who have shown that our reason, our heart, etc., also lead 
to God, have therewithal shown that we are possessed through and through. To 
be sure, they vexed the theologians, from whom they took away the prerogative 
of religious exaltation; but for religion, for freedom of mind, they thereby 
conquered yet more ground. For, when the mind is no longer limited to feeling 
or faith, but also, as understanding, reason, and thought in general, belongs 
to itself the mind -- when therefore, it may take part in the 
spiritual\footnote{[The same word as ``intellectual'', as ``mind'' and 
``spirit'' are the same.]} and heavenly truths in the form of understanding, 
as well as in its other forms -- then the whole mind is occupied only with 
spiritual things, \textit{i.e.}, with itself, and is therefore free. Now we 
are so through-and-through religious that ``jurors,'' \textit{i.e.} ``sworn 
men,'' condemn us to death, and every policeman, as a good Christian, takes 
us to the lock-up by virtue of an ``oath of office.''

Morality could not come into opposition with piety till after the time when in 
general the boisterous hate of everything that looked like an ``order'' 
(decrees, commandments, etc.) spoke out in revolt, and the personal 
``absolute lord'' was scoffed at and persecuted; consequently it could 
arrive at independence only through liberalism, whose first form acquired 
significance in the world's history as ``citizenship,'' and weakened the 
specifically religious powers (see ``Liberalism'' below). For, when morality 
not merely goes alongside of piety, but stands on feet of its own, then its 
principle lies no longer in the divine commandments, but in the law of reason, 
from which the commandments, so far as they are still to remain valid, must 
first await justification for their validity. In the law of reason man 
determines himself out of himself, for ``Man'' is rational, and out of the 
``essence of Man'' those laws follow of necessity. Piety and morality part 
company in this -- that the former makes God the law-giver, the latter Man.

From a certain standpoint of morality people reason about as follows: Either 
man is led by his sensuality, and is, following it, \textit{immoral}, or he is 
led by the good, which, taken up into the will, is called moral sentiment 
(sentiment and prepossession in favor of the good); then he shows himself 
\textit{moral}. From this point of view how, \textit{e.g.}, can Sand's act 
against Kotzebue be called immoral? What is commonly understood by unselfish 
it certainly was, in the same measure as (among other things) St. Crispin's 
thieveries in favor of the poor. ``He should not have murdered, for it stands 
written, Thou shalt not murder!'' Then to serve the good, the welfare of the 
people, as Sand at least intended, or the welfare of the poor, like Crispin -- 
is moral; but murder and theft are immoral; the purpose moral, the means 
immoral. Why? ``Because murder, assassination, is something absolutely 
bad.'' When the Guerrillas enticed the enemies of the country into ravines 
and shot them down unseen from the bushes, do you suppose that was 
assassination? According to the principle of morality, which commands us to 
serve the good, you could really ask only whether murder could never in any 
case be a realization of the good, and would have to endorse that murder which 
realized the good. You cannot condemn Sand's deed at all; it was moral, 
because in the service of the good, because unselfish; it was an act of 
punishment, which the individual inflicted, an -- \textit{execution} inflicted 
at the risk of the executioner's life. What else had his scheme been, after 
all, but that he wanted to suppress writings by brute force? Are you not 
acquainted with the same procedure as a ``legal'' and sanctioned one? And 
what can be objected against it from your principle of morality? -- ``But it 
was an illegal execution.'' So the immoral thing in it was the illegality, 
the disobedience to law? Then you admit that the good is nothing else than -- 
law, morality nothing else than \textit{loyalty}. And to this externality of 
``loyalty'' your morality must sink, to this righteousness of works in the 
fulfillment of the law, only that the latter is at once more tyrannical and 
more revolting than the old-time righteousness of works. For in the latter 
only the \textit{act} is needed, but you require the \textit{disposition} too; 
one must carry \textit{in himself} the law, the statute; and he who is most 
legally disposed is the most moral. Even the last vestige of cheerfulness in 
Catholic life must perish in this Protestant legality. Here at last the 
domination of the law is for the first time complete. ``Not I live, but the 
law lives in me.'' Thus I have really come so far to be only the ``vessel of 
its glory.'' ``Every Prussian carries his \textit{gendarme} in his 
breast,'' says a high Prussian officer.

Why do certain \textit{opposition parties} fail to flourish? Solely for the 
reason that they refuse to forsake the path of morality or legality. Hence the 
measureless hypocrisy of devotion, love, etc., from whose repulsiveness one 
may daily get the most thorough nausea at this rotten and hypocritical 
relation of a ``lawful opposition.'' -- In the \textit{moral} relation of 
love and fidelity a divided or opposed will cannot have place; the beautiful 
relation is disturbed if the one wills this and the other the reverse. But 
now, according to the practice hitherto and the old prejudice of the 
opposition, the moral relation is to be preserved above all. What is then left 
to the opposition? Perhaps the will to have a liberty, if the beloved one sees 
fit to deny it? Not a bit! It may not \textit{will} to have the freedom, it 
can only \textit{wish} for it, ``petition'' for it, lisp a ``Please, 
please!'' What would come of it, if the opposition really \textit{willed}, 
willed with the full energy of the will? No, it must renounce will in order to 
live to \textit{love}, renounce liberty -- for love of morality. It may never 
``claim as a right'' what it is permitted only to ``beg as a favor.'' 
Love, devotion. etc., demand with undeviating definiteness that there be only 
one will to which the others devote themselves, which they serve, follow, 
love. Whether this will is regarded as reasonable or as unreasonable, in both 
cases one acts morally when one follows it, and immorally when one breaks away 
from it. The will that commands the censorship seems to many unreasonable; but 
he who in a land of censorship evades the censoring of his book acts 
immorally, and he who submits it to the censorship acts morally. If some one 
let his moral judgment go, and set up \textit{e.g.} a secret press, one would 
have to call him immoral, and imprudent in the bargain if he let himself be 
caught; but will such a man lay claim to a value in the eyes of the 
``moral''? Perhaps! -- That is, if he fancied he was serving a ``higher 
morality.''

The web of the hypocrisy of today hangs on the frontiers of two domains, 
between which our time swings back and forth, attaching its fine threads of 
deception and self-deception. No longer vigorous enough to serve 
\textit{morality} without doubt or weakening, not yet reckless enough to live 
wholly to egoism, it trembles now toward the one and now toward the other in 
the spider-web of hypocrisy, and, crippled by the curse of \textit{halfness}, 
catches only miserable, stupid flies. If one has once dared to make a 
``free'' motion, immediately one waters it again with assurances of love, 
and -- \textit{shams resignation}; if, on the other side, they have had the 
face to reject the free motion with \textit{moral} appeals to confidence, 
immediately the moral courage also sinks, and they assure one how they hear 
the free words with special pleasure, etc.; they -- \textit{sham approval}. In 
short, people would like to have the one, but not go without the other; they 
would like to have a \textit{free will}, but not for their lives lack the 
\textit{moral will}. Just come in contact with a servile loyalist, you 
Liberals. You will sweeten every word of freedom with a look of the most loyal 
confidence, and he will clothe his servilism in the most flattering phrases of 
freedom. Then you go apart, and he, like you, thinks ``I know you, fox!'' He 
scents the devil in you as much as you do the dark old Lord God in him.

A Nero is a ``bad'' man only in the eyes of the ``good''; in mine he is 
nothing but a \textit{possessed} man, as are the good too. The good see in him 
an arch-villain, and relegate him to hell. Why did nothing hinder him in his 
arbitrary course? Why did people put up with so much? Do you suppose the tame 
Romans, who let all their will be bound by such a tyrant, were a hair the 
better? In old Rome they would have put him to death instantly, would never 
have been his slaves. But the contemporary ``good'' among the Romans opposed 
to him only moral demands, not their \textit{will}; they sighed that their 
emperor did not do homage to morality, like them; they themselves remained 
``moral subjects,'' till at last one found courage to give up ``moral, 
obedient subjection.'' And then the same ``good Romans'' who, as 
``obedient subjects,'' had borne all the ignominy of having no will, 
hurrahed over the nefarious, immoral act of the rebel. Where then in the 
``good'' was the courage for the \textit{revolution}, that courage which 
they now praised, after another had mustered it up? The good could not have 
this courage, for a revolution, and an insurrection into the bargain, is 
always something ``immoral,'' which one can resolve upon only when one 
ceases to be ``good'' and becomes either ``bad'' or -- neither of the two. 
Nero was no viler than his time, in which one could only be one of the two, 
good or bad. The judgment of his time on him had to be that he was bad, and 
this in the highest degree: not a milksop, but an arch-scoundrel. All moral 
people can pronounce only this judgment on him. Rascals \textit{e.g.} he was 
are still living here and there today (see \textit{e.g.} the \textit{Memoirs} 
of Ritter von Lang) in the midst of the moral. It is not convenient to live 
among them certainly, as one is not sure of his life for a moment; but can you 
say that it is more convenient to live among the moral? One is just as little 
sure of his life there, only that one is hanged ``in the way of justice,'' 
but least of all is one sure of his honor, and the national cockade is gone 
before you can say Jack Robinson. The hard fist of morality treats the noble 
nature of egoism altogether without compassion.

``But surely one cannot put a rascal and an honest man on the same level!'' 
Now, no human being does that oftener than you judges of morals; yes, still 
more than that, you imprison as a criminal an honest man who speaks openly 
against the existing constitution, against the hallowed institutions, and you 
entrust portfolios and still more important things to a crafty rascal. So 
\textit{in praxi} you have nothing to reproach me with. ``But in theory!'' 
Now there I do put both on the same level, as two opposite poles -- to wit, 
both on the level of the moral law. Both have meaning only in the ``moral 
world'', just as in the pre-Christian time a Jew who kept the law and one who 
broke it had meaning and significance only in respect to the Jewish law; 
before Jesus Christ, on the contrary, the Pharisee was no more than the 
``sinner and publican.'' So before self-ownership the moral Pharisee amounts 
to as much as the immoral sinner.

Nero became very inconvenient by his possessedness. But a self-owning man 
would not sillily oppose to him the ``sacred,'' and whine if the tyrant does 
not regard the sacred; he would oppose to him his will. How often the 
sacredness of the inalienable rights of man has been held up to their foes, 
and some liberty or other shown and demonstrated to be a ``sacred right of 
man!'' Those who do that deserve to be laughed out of court -- as they 
actually are -- were it not that in truth they do, even though unconsciously, 
take the road that leads to the goal. They have a presentiment that, if only 
the majority is once won for that liberty, it will also will the liberty, and 
will then take what it \textit{will} have. The sacredness of the liberty, and 
all possible proofs of this sacredness, will never procure it; lamenting and 
petitioning only shows beggars.

The moral man is necessarily narrow in that he knows no other enemy than the 
``immoral'' man. ``He who is not moral is immoral!'' and accordingly 
reprobate, despicable, etc. Therefore the moral man can never comprehend the 
egoist. Is not unwedded cohabitation an immorality? The moral man may turn as 
he pleases, he will have to stand by this verdict; Emilia Galotti gave up her 
life for this moral truth. And it is true, it is an immorality. A virtuous 
girl may become an old maid; a virtuous man may pass the time in fighting his 
natural impulses till he has perhaps dulled them, he may castrate himself for 
the sake of virtue as St. Origen did for the sake of heaven: he thereby honors 
sacred wedlock, sacred chastity, as inviolable; he is -- moral. Unchastity can 
never become a moral act. However indulgently the moral man may judge and 
excuse him who committed it, it remains a transgression, a sin against a moral 
commandment; there clings to it an indelible stain. As chastity once belonged 
to the monastic vow, so it does to moral conduct. Chastity is a -- good. -- 
For the egoist, on the contrary, even chastity is not a good without which he 
could not get along; he cares nothing at all about it. What now follows from 
this for the judgment of the moral man? This: that he throws the egoist into 
the only class of men that he knows besides moral men, into that of the -- 
immoral. He cannot do otherwise; he must find the egoist immoral in everything 
in which the egoist disregards morality. If he did not find him so, then he 
would already have become an apostate from morality without confessing it to 
himself, he would already no longer be a truly moral man. One should not let 
himself be led astray by such phenomena, which at the present day are 
certainly no longer to be classed as rare, but should reflect that he who 
yields any point of morality can as little be counted among the truly moral as 
Lessing was a pious Christian when, in the well-known parable, he compared the 
Christian religion, as well as the Mohammedan and Jewish, to a ``counterfeit 
ring.'' Often people are already further than they venture to confess to 
themselves. For Socrates, because in culture he stood on the level of 
morality, it would have been an immorality if he had been willing to follow 
Crito's seductive incitement and escape from the dungeon; to remain was the 
only moral thing. But it was solely because Socrates was -- a moral man. The 
``unprincipled, sacrilegious'' men of the Revolution, on the contrary, had 
sworn fidelity to Louis XVI, and decreed his deposition, yes, his death; but 
the act was an immoral one, at which moral persons will be horrified to all 
eternity.

Yet all this applies, more or less, only to ``civic morality,'' on which the 
freer look down with contempt. For it (like civism, its native ground, in 
general) is still too little removed and free from the religious heaven not to 
transplant the latter's laws without criticism or further consideration to its 
domain instead of producing independent doctrines of its own. Morality cuts a 
quite different figure when it arrives at the consciousness of its dignity, 
and raises its principle, the essence of man, or ``Man,'' to be the only 
regulative power. Those who have worked their way through to such a decided 
consciousness break entirely with religion, whose God no longer finds any 
place alongside their ``Man,'' and, as they (see below) themselves scuttle 
the ship of State, so too they crumble away that ``morality'' which 
flourishes only in the State, and logically have no right to use even its name 
any further. For what this ``critical'' party calls morality is very 
positively distinguished from the so-called ``civic or political morality,'' 
and must appear to the citizen like an ``insensate and unbridled liberty.'' 
But at bottom it has only the advantage of the ``purity of the principle,'' 
which, freed from its defilement with the religious, has now reached universal 
power in its clarified definiteness as ``humanity.''

Therefore one should not wonder that the name ``morality'' is retained along 
with others, like freedom, benevolence, self-consciousness, and is only 
garnished now and then with the addition, a ``free'' morality -- just as, 
though the civic State is abused, yet the State is to arise again as a ``free 
State,'' or, if not even so, yet as a ``free society.''

Because this morality completed into humanity has fully settled its accounts 
with the religion out of which it historically came forth, nothing hinders it 
from becoming a religion on its own account. For a distinction prevails 
between religion and morality only so long as our dealings with the world of 
men are regulated and hallowed by our relation to a superhuman being, or so 
long as our doing is a doing ``for God's sake.'' If, on the other hand, it 
comes to the point that ``man is to man the supreme being,'' then that 
distinction vanishes, and morality, being removed from its subordinate 
position, is completed into -- religion. For then the higher being who had 
hitherto been subordinated to the highest, Man, has ascended to absolute 
height, and we are related to him as one is related to the highest being, 
\textit{i.e.} religiously. Morality and piety are now as synonymous as in the 
beginning of Christianity, and it is only because the supreme being has come 
to be a different one that a holy walk is no longer called a ``holy'' one, 
but a ``human'' one. If morality has conquered, then a complete -- 
\textit{change of masters} has taken place.

After the annihilation of faith Feuerbach thinks to put in to the supposedly 
safe harbor of \textit{love}. ``The first and highest law must be the love of 
man to man. \textit{Homo homini Deus est --} this is the supreme practical 
maxim, this is the turning point of the world's 
history.''\footnote{``Essence of Christianity,'' second edition, p. 402.} 
But, properly speaking, only the god is changed -- the \textit{deus}; love has 
remained: there love to the superhuman God, here love to the human God, to 
\textit{homo as Deus}. Therefore man is to me -- sacred. And everything 
``truly human'' is to me -- sacred! ``Marriage is sacred of itself. And so 
it is with all moral relations. Friendship is and must be \textit{sacred} for 
you, and property, and marriage, and the good of every man, but sacred 
\textit{in and of itself}.\footnote{P. 403.} '' Haven't we the priest again 
there? Who is his God? Man with a great M! What is the divine? The human! Then 
the predicate has indeed only been changed into the subject, and, instead of 
the sentence ``God is love,'' they say ``love is divine''; instead of 
``God has become man,'' ``Man has become God,'' etc. It is nothing more or 
less than a new -- \textit{religion}. ``All moral relations are ethical, are 
cultivated with a moral mind, only where of themselves (without religious 
consecration by the priest's blessing) they are counted \textit{religious}. 
'' Feuerbach's proposition, ``Theology is anthropology,'' means only 
``religion must be ethics, ethics alone is religion.''

Altogether Feuerbach accomplishes only a transposition of subject and 
predicate, a giving of preference to the latter. But, since he himself says, 
``Love is not (and has never been considered by men) sacred through being a 
predicate of God, but it is a predicate of God because it is divine in and of 
itself,'' he might judge that the fight against the predicates themselves, 
against love and all sanctities, must be commenced. How could he hope to turn 
men away from God when he left them the divine? And if, as Feuerbach says, God 
himself has never been the main thing to them, but only his predicates, then 
he might have gone on leaving them the tinsel longer yet, since the doll, the 
real kernel, was left at any rate. He recognizes, too, that with him it is 
``only a matter of annihilating an illusion'';\footnote{P. 408.} he thinks, 
however, that the effect of the illusion on men is ``downright ruinous, since 
even love, in itself the truest, most inward sentiment, becomes an obscure, 
illusory one through religiousness, since religious love loves 
man\footnote{[Literally ``the man.'']} only for God's sake, therefore loves 
man only apparently, but in truth God only.'' Is this different with moral 
love? Does it love the man, \textit{this} man for \textit{this} man's sake, or 
for morality's sake, and so -- for \textit{homo homini Deus --} for God's 
sake?

\myhrule


The wheels in the head have a number of other formal aspects, some of which it 
may be useful to indicate here.

Thus \textit{self-renunciation is} common to the holy with the unholy, to the 
pure and the impure. The impure man \textit{renounces} all ``better 
feelings,'' all shame, even natural timidity, and follows only the appetite 
that rules him. The pure man renounces his natural relation to the world 
(``renounces the world'') and follows only the ``desire'' which rules him. 
Driven by the thirst for money, the avaricious man renounces all admonitions 
of conscience, all feeling of honor, all gentleness and all compassion; he 
puts all considerations out of sight; the appetite drags him along. The holy 
man behaves similarly. He makes himself the ``laughing-stock of the world,'' 
is hard-hearted and ``strictly just''; for the desire drags him along. As 
the unholy man renounces \textit{himself} before Mammon, so the holy man 
renounces \textit{himself} before God and the divine laws. We are now living 
in a time when the \textit{shamelessness} of the holy is every day more and 
more felt and uncovered, whereby it is at the same time compelled to unveil 
itself, and lay itself bare, more and more every day. Have not the 
shamelessness and stupidity of the reasons with which men antagonize the 
``progress of the age'' long surpassed all measure and all expectation? But 
it must be so. The self-renouncers must, as holy men, take the same course 
that they do so as unholy men; as the latter little by little sink to the 
fullest measure of self-renouncing vulgarity and \textit{lowness}, so the 
former must ascend to the most dishonorable \textit{exaltation}. The mammon of 
the earth and the \textit{God} of heaven both demand exactly the same degree 
of -- self-renunciation. The low man, like the exalted one, reaches out for a 
``good'' -- the former for the material good, the latter for the ideal, the 
so-called ``supreme good''; and at last both complete each other again too, 
as the ``materially-minded'' man sacrifices everything to an ideal phantasm, 
his \textit{vanity}, and the ``spiritually-minded'' man to a material 
gratification, the \textit{life of enjoyment}.

Those who exhort men to 
``unselfishness''\footnote{[\textit{uneigenn\"utzigkeit}, literally 
``un-self-benefitingness.'']} think they are saying an uncommon deal. What 
do they understand by it? Probably something like what they understand by 
``self-renunciation.'' But who is this self that is to be renounced and to 
have no benefit? It seems that you yourself are supposed to be it. And for 
whose benefit is unselfish self-renunciation recommended to you? Again for 
\textit{your} benefit and behoof, only that through unselfishness you are 
procuring your ``true benefit.''

You are to benefit \textit{yourself}, and yet you are not to seek 
\textit{your} benefit.

People regard as unselfish the \textit{benefactor} of men, a Francke who 
founded the orphan asylum, an O'Connell who works tirelessly for his Irish 
people; but also the \textit{fanatic} who, like St. Boniface, hazards his life 
for the conversion of the heathen, or, like Robespierre, ``sacrifices 
everything to virtue'' -- like K\"orner, dies for God, king, and fatherland. 
Hence, among others, O'Connell's opponents try to trump up against him some 
selfishness or mercenariness, for which the O'Connell fund seemed to give them 
a foundation; for, if they were successful in casting suspicion on his 
``unselfishness,'' they would easily separate him from his adherents.

Yet what could they show further than that O'Connell was working for another 
\textit{end} than the ostensible one? But, whether he may aim at making money 
or at liberating the people, it still remains certain, in one case as in the 
other, that he is striving for an end, and that \textit{his} end; selfishness 
here as there, only that his national self-interest would be beneficial to 
\textit{others too}, and so would be for the \textit{common} interest.

Now, do you suppose unselfishness is unreal and nowhere extant? On the 
contrary, nothing is more ordinary! One may even call it an article of fashion 
in the civilized world, which is considered so indispensable that, if it costs 
too much in solid material, people at least adorn themselves with its tinsel 
counterfeit and feign it. Where does unselfishness begin? Right where an end 
ceases to be \textit{our} end and our \textit{property}, which we, as owners, 
can dispose of at pleasure; where it becomes a fixed end or a -- fixed idea; 
where it begins to inspire, enthuse, fantasize us; in short, where it passes 
into our \textit{stubbornness} and becomes our -- master. One is not unselfish 
so long as he retains the end in his power; one becomes so only at that 
``Here I stand, I cannot do otherwise,'' the fundamental maxim of all the 
possessed; one becomes so in the case of a \textit{sacred} end, through the 
corresponding sacred zeal.

I am not unselfish so long as the end remains my own, and I, instead of giving 
myself up to be the blind means of its fulfillment, leave it always an open 
question. My zeal need not on that account be slacker than the most fanatical, 
but at the same time I remain toward it frostily cold, unbelieving, and its 
most irreconcilable enemy; I remain its \textit{judge}, because I am its 
owner.

Unselfishness grows rank as far as possessedness reaches, as much on 
possessions of the devil as on those of a good spirit; there vice, folly, 
etc.; here humility, devotion, etc.

Where could one look without meeting victims of self-renunciation? There sits 
a girl opposite me, who perhaps has been making bloody sacrifices to her soul 
for ten years already. Over the buxom form droops a deathly-tired head, and 
pale cheeks betray the slow bleeding away of her youth. Poor child, how often 
the passions may have beaten at your heart, and the rich powers of youth have 
demanded their right! When your head rolled in the soft pillow, how awakening 
nature quivered through your limbs, the blood swelled your veins, and fiery 
fancies poured the gleam of voluptuousness into your eyes! Then appeared the 
ghost of the soul and its eternal bliss. You were terrified, your hands folded 
themselves, your tormented eyes turned their look upward, you -- prayed. The 
storms of nature were hushed, a calm glided over the ocean of your appetites. 
Slowly the weary eyelids sank over the life extinguished under them, the 
tension crept out unperceived from the rounded limbs, the boisterous waves 
dried up in the heart, the folded hands themselves rested a powerless weight 
on the unresisting bosom, one last faint ``Oh dear!'' moaned itself away, 
and -- \textit{the soul was at rest}. You fell asleep, to awake in the morning 
to a new combat and a new -- prayer. Now the habit of renunciation cools the 
heat of your desire, and the roses of your youth are growing pale in the -- 
chlorosis of your heavenliness. The soul is saved, the body may perish! O 
Lais, O Ninon, how well you did to scorn this pale virtue! One free 
\textit{grisette} against a thousand virgins grown gray in virtue!

The fixed idea may also be perceived as ``maxim,'' ``principle,'' 
``standpoint,'' etc. Archimedes, to move the earth, asked for a standpoint 
\textit{outside} it. Men sought continually for this standpoint, and every one 
seized upon it as well as he was able. This foreign standpoint is the 
\textit{world of mind}, of ideas, thoughts, concepts, essences; it is 
\textit{heaven}. Heaven is the ``standpoint'' from which the earth is moved, 
earthly doings surveyed and -- despised. To assure to themselves heaven, to 
occupy the heavenly standpoint firmly and for ever -- how painfully and 
tirelessly humanity struggled for this!

Christianity has aimed to deliver us from a life determined by nature, from 
the appetites as actuating us, and so has meant that man should not let 
himself be determined by his appetites. This does not involve the idea that 
\textit{he} was not to have appetites, but that the appetites were not to have 
him, that they were not to become \textit{fixed}, uncontrollable, 
indissoluble. Now, could not what Christianity (religion) contrived against 
the appetites be applied by us to its own precept that \textit{mind} (thought, 
conceptions, ideas, faith) must determine us; could we not ask that neither 
should mind, or the conception, the idea, be allowed to determine us, to 
become fixed and inviolable or ``sacred''? Then it would end in the 
\textit{dissolution of mind}, the dissolution of all thoughts, of all 
conceptions. As we there had to say, ``We are indeed to have appetites, but 
the appetites are not to have us,'' so we should now say, ``We are indeed to 
have \textit{mind}, but mind is not to have us.'' If the latter seems lacking 
in sense, think \textit{e.g.} of the fact that with so many a man a thought 
becomes a ``maxim,'' whereby he himself is made prisoner to it, so that it 
is not he that has the maxim, but rather it that has him. And with the maxim 
he has a ``permanent standpoint'' again. The doctrines of the catechism 
become our \textit{principles} before we find it out, and no longer brook 
rejection. Their thought, or -- mind, has the sole power, and no protest of 
the ``flesh'' is further listened to. Nevertheless it is only through the 
``flesh'' that I can break tyranny of mind; for it is only when a man hears 
his flesh along with the rest of him that he hears himself wholly, and it is 
only when he wholly hears \textit{himself} that he is a hearing or 
rational\footnote{[\textit{vern\"unftig}, derived from \textit{vernehmen}, to 
hear.]} being. The Christian does not hear the agony of his enthralled nature, 
but lives in ``humility''; therefore he does not grumble at the wrong which 
befalls his \textit{person}; he thinks himself satisfied with the ``freedom 
of the spirit.'' But, if the flesh once takes the floor, and its tone is 
``passionate,'' ``indecorous,'' ``not well-disposed,'' ``spiteful'' 
(as it cannot be otherwise), then he thinks he hears voices of devils, voices 
\textit{against the spirit} (for decorum, passionlessness, kindly disposition, 
and the like, is -- spirit), and is justly zealous against them. He could not 
be a Christian if he were willing to endure them. He listens only to morality, 
and slaps unmorality in the mouth; he listens only to legality, and gags the 
lawless word. The \textit{spirit} of morality and legality holds him a 
prisoner; a rigid, unbending \textit{master}. They call that the ``mastery of 
the spirit'' -- it is at the same time the \textit{standpoint} of the spirit.

And now whom do the ordinary liberal gentlemen mean to make free? Whose 
freedom is it that they cry out and thirst for? The \textit{spirit's!} That of 
the spirit of morality, legality, piety, the fear of God. That is what the 
anti-liberal gentlemen also want, and the whole contention between the two 
turns on a matter of advantage -- whether the latter are to be the only 
speakers, or the former are to receive a ``share in the enjoyment of the same 
advantage.'' The \textit{spirit} remains the absolute \textit{lord} for both, 
and their only quarrel is over who shall occupy the hierarchical throne that 
pertains to the ``Viceregent of the Lord.'' The best of it is that one can 
calmly look upon the stir with the certainty that the wild beasts of history 
will tear each other to pieces just like those of nature; their putrefying 
corpses fertilize the ground for -- our crops.

We shall come back later to many another wheel in the head -- \textit{e.g.}, 
those of vocation, truthfulness, love, etc.

\myhrule


When one's own is contrasted with what is \textit{imparted} to him, there is 
no use in objecting that we cannot have anything isolated, but receive 
everything as a part of the universal order, and therefore through the 
impression of what is around us, and that consequently we have it as something 
``imparted''; for there is a great difference between the feelings and 
thoughts which are \textit{aroused} in me by other things and those which are 
\textit{given} to me. God, immortality, freedom, humanity, etc. are drilled 
into us from childhood as thoughts and feelings which move our inner being 
more or less strongly, either ruling us without our knowing it, or sometimes 
in richer natures manifesting themselves in systems and works of art; but are 
always not aroused, but imparted, feelings, because we must believe in them 
and cling to them. That an Absolute existed, and that it must be taken in, 
felt, and thought by us, was settled as a faith in the minds of those who 
spent all the strength of their mind on recognizing it and setting it forth. 
The \textit{feeling} for the Absolute exists there as an imparted one, and 
thenceforth results only in the most manifold revelations of its own self. So 
in Klopstock the religious feeling was an imparted one, which in the 
\textit{Messiad} simply found artistic expression. If, on the other hand, the 
religion with which he was confronted had been for him only an incitation to 
feeling and thought, and if he had known how to take an attitude completely 
\textit{his own} toward it, then there would have resulted, instead of 
religious inspiration, a dissolution and consumption of the religion itself. 
Instead of that, he only continued in mature years his childish feelings 
received in childhood, and squandered the powers of his manhood in decking out 
his childish trifles.

The difference is, then, whether feelings are imparted to me or only aroused. 
Those which are aroused are my own, egoistic, because they are not \textit{as 
feelings} drilled into me, dictated to me, and pressed upon me; but those 
which are imparted to me I receive, with open arms -- I cherish them in me as 
a heritage, cultivate them, and am \textit{possessed} by them. Who is there 
that has never, more or less consciously, noticed that our whole education is 
calculated to produce \textit{feelings} in us, \textit{i.e.} impart them to 
us, instead of leaving their production to ourselves however they may turn 
out? If we hear the name of God, we are to feel veneration; if we hear that of 
the prince's majesty, it is to be received with reverence, deference, 
submission; if we hear that of morality, we are to think that we hear 
something inviolable; if we hear of the Evil One or evil ones, we are to 
shudder. The intention is directed to these \textit{feelings}, and he who 
\textit{e.g.} should hear with pleasure the deeds of the ``bad'' would have 
to be ``taught what's what'' with the rod of discipline. Thus stuffed with 
\textit{imparted feelings}, we appear before the bar of majority and are 
``pronounced of age.'' Our equipment consists of ``elevating feelings, 
lofty thoughts, inspiring maxims, eternal principles,'' etc. The young are of 
age when they twitter like the old; they are driven through school to learn 
the old song, and, when they have this by heart, they are declared of age.

We \textit{must not} feel at every thing and every name that comes before us 
what we could and would like to feel thereat; \textit{e.g.} at the name of 
God we must think of nothing laughable, feel nothing disrespectful, it being 
prescribed and imparted to us what and how we are to feel and think at mention 
of that name. That is the meaning of the \textit{care of souls --} that my 
soul or my mind be tuned as others think right, not as I myself would like it. 
How much trouble does it not cost one, finally to secure to oneself a feeling 
of one's own at the mention of at least this or that name, and to laugh in the 
face of many who expect from us a holy face and a composed expression at their 
speeches. What is imparted is \textit{alien} to us, is not our own, and 
therefore is ``sacred,'' and it is hard work to lay aside the ``sacred 
dread of it.''

Today one again hears ``seriousness'' praised, ``seriousness in the 
presence of highly important subjects and discussions,'' ``German 
seriousness,'' etc. This sort of seriousness proclaims clearly how old and 
grave lunacy and possession have already become. For there is nothing more 
serious than a lunatic when he comes to the central point of his lunacy; then 
his great earnestness incapacitates him for taking a joke. (See madhouses.)

\medskip{}

\subsection[\S{}3. The Hierarchy]{\centering \S{}3. The Hierarchy}

The historical reflections on our Mongolism which I propose to insert 
episodically at this place are not given with the claim of thoroughness, or 
even of approved soundness, but solely because it seems to me that they may 
contribute toward making the rest clear.

The history of the world, whose shaping properly belongs altogether to the 
Caucasian race, seems till now to have run through two Caucasian ages, in the 
first of which we had to work out and work off our innate \textit{negroidity}; 
this was followed in the second by \textit{Mongoloidity} (Chineseness), which 
must likewise be terribly made an end of. Negroidity represents 
\textit{antiquity}, the time of dependence on \textit{things} (on cocks' 
eating, birds' flight, on sneezing, on thunder and lightning, on the rustling 
of sacred trees, etc.); Mongoloidity the time of dependence on thoughts, the 
\textit{Christian} time. Reserved for the future are the words, ``I am the 
owner of the world of things, I am the owner of the world of mind.''

In the negroid age fall the campaigns of Sesostris and the importance of Egypt 
and of northern Africa in general. To the Mongoloid age belong the invasions 
of the Huns and Mongols, up to the Russians.

The value of \textit{me} cannot possibly be rated high so long as the hard 
diamond of the \textit{not-me} bears so enormous a price as was the case both 
with God and with the world. The not-me is still too stony and indomitable to 
be consumed and absorbed by me; rather, men only creep about with 
extraordinary \textit{bustle} on this \textit{immovable} entity, on this 
\textit{substance}, like parasitic animals on a body from whose juices they 
draw nourishment, yet without consuming it. It is the bustle of vermin, the 
assiduity of Mongolians. Among the Chinese, we know, everything remains as it 
used to be, and nothing ``essential'' or ``substantial'' suffers a change; 
all the more actively do they work away \textit{at} that which remains, which 
bears the name of the ``old,'' ``ancestors,'' etc.

Accordingly, in our Mongolian age all change has been only reformatory or 
ameliorative, not destructive or consuming and annihilating. The substance, 
the object, \textit{remains}. All our assiduity was only the activity of ants 
and the hopping of fleas, jugglers' tricks on the immovable tight-rope of the 
objective, \textit{corv\'ee} -service under the leadership of the unchangeable 
or ``eternal.'' The Chinese are doubtless the most \textit{positive} nation, 
because totally buried in precepts; but neither has the Christian age come out 
from the \textit{positive}, \textit{i.e.} from ``limited freedom,'' freedom 
``within certain limits.'' In the most advanced stage of civilization this 
activity earns the name of \textit{scientific} activity, of working on a 
motionless presupposition, a \textit{hypothesis} that is not to be upset.

In its first and most unintelligible form morality shows itself as 
\textit{habit}. To act according to the habit and usage \textit{(mores)} of 
one's country -- is to be moral there. Therefore pure moral action, clear, 
unadulterated morality, is most straightforwardly practiced in China; they 
keep to the old habit and usage, and hate each innovation as a crime worthy of 
death. For \textit{innovation} is the deadly enemy of \textit{habit}, of the 
\textit{old}, of \textit{permanence}. In fact, too, it admits of no doubt that 
through habit man secures himself against the obtrusiveness of things, of the 
world, and founds a world of his own in which alone he is and feels at home, 
builds himself a \textit{heaven}. Why, heaven has no other meaning than that 
it is man's proper home, in which nothing alien regulates and rules him any 
longer, no influence of the earthly any longer makes him himself alien; in 
short, in which the dross of the earthly is thrown off, and the combat against 
the world has found an end -- in which, therefore, nothing is any longer 
\textit{denied} him. Heaven is the end of \textit{abnegation}, it is 
\textit{free enjoyment}. There man no longer denies himself anything, because 
nothing is any longer alien and hostile to him. But now habit is a ``second 
nature,'' which detaches and frees man from his first and original natural 
condition, in securing him against every casualty of it. The fully elaborated 
habit of the Chinese has provided for all emergencies, and everything is 
``looked out for''; whatever may come, the Chinaman always knows how he has 
to behave, and does not need to decide first according to the circumstances; 
no unforeseen case throws him down from the heaven of his rest. The morally 
habituated and inured Chinaman is not surprised and taken off his guard; he 
behaves with equanimity (\textit{i.e.}, with equal spirit or temper) toward 
everything, because his temper, protected by the precaution of his traditional 
usage, does not lose its balance. Hence, on the ladder of culture or 
civilization humanity mounts the first round through habit; and, as it 
conceives that, in climbing to culture, it is at the same time climbing to 
heaven, the realm of culture or second nature, it really mounts the first 
round of the -- ladder to heaven.

If Mongoldom has settled the existence of spiritual beings -- if it has 
created a world of spirits, a heaven -- the Caucasians have wrestled for 
thousands of years with these spiritual beings, to get to the bottom of them. 
What were they doing, then, but building on Mongolian ground? They have not 
built on sand, but in the air; they have wrestled with Mongolism, stormed the 
Mongolian heaven, Tien. When will they at last annihilate this heaven? When 
will they at last become \textit{really Caucasians}, and find themselves? When 
will the ``immortality of the soul,'' which in these latter days thought it 
was giving itself still more security if it presented itself as ``immortality 
of mind,'' at last change to the \textit{mortality of mind?}

It was when, in the industrious struggle of the Mongolian race, men had 
\textit{built a heaven}, that those of the Caucasian race, since in their 
Mongolian complexion they have to do with heaven, took upon themselves the 
opposite task, the task of storming that heaven of custom, 
\textit{heaven-storming}\footnote{[A German idiom for destructive 
radicalism.]} activity. To dig under all human ordinance, in order to set up a 
new and -- better one on the cleared site, to wreck all customs in order to 
put new and -- better customs in their place -- their act is limited to this. 
But is it thus already purely and really what it aspires to be, and does it 
reach its final aim? No, in this creation of \textit{a ``better''} it is 
tainted with Mongolism. It storms heaven only to make a heaven again, it 
overthrows an old power only to legitimate a new power, it only -- 
\textit{improves}. Nevertheless the point aimed at, often as it may vanish 
from the eyes at every new attempt, is the real, complete downfall of heaven, 
customs, etc. -- in short, of man secured only against the world, of the 
\textit{isolation} or \textit{inwardness} of man. Through the heaven of 
culture man seeks to isolate himself from the world, to break its hostile 
power. But this isolation of heaven must likewise be broken, and the true end 
of heaven-storming is the -- downfall of heaven, the annihilation of heaven. 
\textit{Improving} and \textit{reforming} is the Mongolism of the Caucasian, 
because thereby he is always getting up again what already existed -- to wit, 
a \textit{precept}, a generality, a heaven. He harbors the most irreconcilable 
enmity to heaven, and yet builds new heavens daily; piling heaven on heaven, 
he only crushes one by another; the Jews' heaven destroys the Greeks', the 
Christians' the Jews', the Protestants' the Catholics', etc. -- If the 
\textit{heaven-storming} men of Caucasian blood throw off their Mongolian 
skin, they will bury the emotional man under the ruins of the monstrous world 
of emotion, the isolated man under his isolated world, the paradisiacal man 
under his heaven. And heaven is the \textit{realm of spirits}, the realm 
\textit{of freedom of the spirit}.

The realm of heaven, the realm of spirits and ghosts, has found its right 
standing in the speculative philosophy. Here it was stated as the realm of 
thoughts, concepts, and ideas; heaven is peopled with thoughts and ideas, and 
this ``realm of spirits'' is then the true reality.

To want to win freedom for the \textit{spirit} is Mongolism; freedom of the 
spirit is Mongolian freedom, freedom of feeling, moral freedom, etc.

We may find the word ``morality'' taken as synonymous with spontaneity, 
self-determination. But that is not involved in it; rather has the Caucasian 
shown himself spontaneous only \textit{in spite} of his Mongolian morality. 
The Mongolian heaven, or morals,\footnote{[The same word that has been 
translated ``custom'' several times in this section.]} remained the strong 
castle, and only by storming incessantly at this castle did the Caucasian show 
himself moral; if he had not had to do with morals at all any longer, if he 
had not had therein his indomitable, continual enemy, the relation to morals 
would cease, and consequently morality would cease. That his spontaneity is 
still a moral spontaneity, therefore, is just the Mongoloidity of it -- is a 
sign that in it he has not arrived at himself. ``Moral spontaneity'' 
corresponds entirely with ``religious and orthodox philosophy,'' 
``constitutional monarchy,'' ``the Christian State,'' ``freedom within 
certain limits,'' ``the limited freedom of the press,'' or, in a figure, to 
the hero fettered to a sick-bed.

Man has not really vanquished Shamanism and its spooks till he possesses the 
strength to lay aside not only the belief in ghosts or in spirits, but also 
the belief in the spirit.

He who believes in a spook no more assumes the ``introduction of a higher 
world'' than he who believes in the spirit, and both seek behind the sensual 
world a supersensual one; in short, they produce and believe \textit{another} 
world, and this other \textit{world, the product of their mind}, is a 
spiritual world; for their senses grasp and know nothing of another, a 
non-sensual world, only their spirit lives in it. Going on from this Mongolian 
belief in the \textit{existence of spiritual beings} to the point that the 
\textit{proper being} of man too is his \textit{spirit}, and that all care 
must be directed to this alone, to the ``welfare of his soul,'' is not hard. 
Influence on the spirit, so-called ``moral influence,'' is hereby assured.

Hence it is manifest that Mongolism represents utter absence of any rights of 
the sensuous, represents non-sensuousness and unnature, and that sin and the 
consciousness of sin was our Mongolian torment that lasted thousands of years.

But who, then, will dissolve the spirit into its \textit{nothing?} He who by 
means of the spirit set forth nature as the \textit{null}, finite, transitory, 
he alone can bring down the spirit too to like nullity. I can; each one among 
you can, who does his will as an absolute I; in a word, the \textit{egoist} 
can.

\myhrule


Before the sacred, people lose all sense of power and all confidence; they 
occupy a \textit{powerless} and \textit{humble} attitude toward it. And yet no 
thing is sacred of itself, but by my \textit{declaring it sacred}, by my 
declaration, my judgment, my bending the knee; in short, by my -- conscience.

Sacred is everything which for the egoist is to be unapproachable, not to be 
touched, outside his \textit{power --} \textit{i.e.} above \textit{him}; 
sacred, in a word, is every \textit{matter of conscience}, for ``this is a 
matter of conscience to me'' means simply, ``I hold this sacred.''

For little children, just as for animals, nothing sacred exists, because, in 
order to make room for this conception, one must already have progressed so 
far in understanding that he can make distinctions like ``good and bad,'' 
``warranted and unwarranted''; only at such a level of reflection or 
intelligence -- the proper standpoint of religion -- can unnatural (\textit{i.e.}, 
brought into existence by thinking) \textit{reverence}, ``sacred 
dread,'' step into the place of natural fear. To this sacred dread belongs 
holding something outside oneself for mightier, greater, better warranted, 
better, etc.; \textit{i.e.} the attitude in which one acknowledges the might 
of something alien -- not merely feels it, then, but expressly acknowledges 
it, \textit{i.e.} admits it, yields, surrenders, lets himself be tied 
(devotion, humility, servility, submission). Here walks the whole ghostly 
troop of the ``Christian virtues.''

Everything toward which you cherish any respect or reverence deserves the name 
of sacred; you yourselves, too, say that you would feel a \textit{``sacred 
dread''} of laying hands on it. And you give this tinge even to the unholy 
(gallows, crime, etc.). You have a horror of touching it. There lies in it 
something uncanny, that is, unfamiliar or not \textit{your own}.

 ``If something or other did not rank as sacred in a man's mind, why, then 
all bars would be let down to self-will, to unlimited subjectivity!'' Fear 
makes the beginning, and one can make himself fearful to the coarsest man; 
already, therefore, a barrier against his insolence. But in fear there always 
remains the attempt to liberate oneself from what is feared, by guile, 
deception, tricks, etc. In reverence,\footnote{[\textit{Ehrfurcht}]} on the 
contrary, it is quite otherwise. Here something is not only 
feared,\footnote{[\textit{gef\"urchtet}]} but also 
honored\footnote{[\textit{geehrt}]}: what is feared has become an inward power 
which I can no longer get clear of; I honor it, am captivated by it and 
devoted to it, belong to it; by the honor which I pay it I am completely in 
its power, and do not even attempt liberation any longer. Now I am attached to 
it with all the strength of faith; I \textit{believe}. I and what I fear are 
one; ``not I live, but the respected lives in me!'' Because the spirit, the 
infinite, does not allow of coming to any end, therefore it is stationary; it 
fears \textit{dying}, it cannot let go its dear Jesus, the greatness of 
finiteness is no longer recognized by its blinded eye; the object of fear, now 
raised to veneration, may no longer be handled; reverence is made eternal, the 
respected is deified. The man is now no longer employed in creating, but in 
\textit{learning} (knowing, investigating, etc.), \textit{i.e.} occupied with 
a fixed \textit{object}, losing himself in its depths, without return to 
himself. The relation to this object is that of knowing, fathoming, basing, 
not that of \textit{dissolution} (abrogation, etc.). ``Man is to be 
religious,'' that is settled; therefore people busy themselves only with the 
question how this is to be attained, what is the right meaning of 
religiousness, etc. Quite otherwise when one makes the axiom itself doubtful 
and calls it in question, even though it should go to smash. Morality too is 
such a sacred conception; one must be moral, and must look only for the right 
``how,'' the right way to be so. One dares not go at morality itself with 
the question whether it is not itself an illusion; it remains exalted above 
all doubt, unchangeable. And so we go on with the sacred, grade after grade, 
from the ``holy'' to the ``holy of holies.''

\myhrule


Men are sometimes divided into two classes: \textit{cultured} and 
\textit{uncultured}. The former, so far as they were worthy of their name, 
occupied themselves with thoughts, with mind, and (because in the time since 
Christ, of which the very principle is thought, they were the ruling ones) 
demanded a servile respect for the thoughts recognized by them. State, 
emperor, church, God, morality, order, are such thoughts or spirits, that 
exist only for the mind. A merely living being, an animal, cares as little for 
them as a child. But the uncultured are really nothing but children, and he 
who attends only to the necessities of his life is indifferent to those 
spirits; but, because he is also weak before them, he succumbs to their power, 
and is ruled by -- thoughts. This is the meaning of hierarchy.

\textit{Hierarchy is dominion of thoughts, dominion of mind!}

We are hierarchic to this day, kept down by those who are supported by 
thoughts. Thoughts are the sacred.

But the two are always clashing, now one and now the other giving the offence; 
and this clash occurs, not only in the collision of two men, but in one and 
the same man. For no cultured man is so cultured as not to find enjoyment in 
things too, and so be uncultured; and no uncultured man is totally without 
thoughts. In Hegel it comes to light at last what a longing for things even 
the most cultured man has, and what a horror of every ``hollow theory'' he 
harbors. With him reality, the world of things, is altogether to correspond to 
the thought, and no concept is to be without reality. This caused Hegel's 
system to be known as the most objective, as if in it thought and thing 
celebrated their union. But this was simply the extremest case of violence on 
the part of thought, its highest pitch of despotism and sole dominion, the 
triumph of mind, and with it the triumph of \textit{philosophy}. Philosophy 
cannot hereafter achieve anything higher, for its highest is the 
\textit{omnipotence of mind}, the almightiness of mind.\footnote{[Rousseau, 
the Philanthropists; and others were hostile to culture and intelligence, but 
they overlooked the fact that this is present in \textit{all}men of the 
Christian type, and assailed only learned and refined culture.]}

Spiritual men have \textit{taken into their head} something that is to be 
realized. They have \textit{concepts} of love, goodness, etc., which they 
would like to see \textit{realized}; therefore they want to set up a kingdom 
of love on earth, in which no one any longer acts from selfishness, but each 
one ``from love.'' Love is to \textit{rule}. What they have taken into their 
head, what shall we call it but -- \textit{fixed idea?} Why, ``their head is 
\textit{haunted.''} The most oppressive spook is \textit{Man}. Think of the 
proverb, ``The road to ruin is paved with good intentions.'' The intention 
to realize humanity altogether in oneself, to become altogether man, is of 
such ruinous kind; here belong the intentions to become good, noble, loving, 
etc.

In the sixth part of the \textit{``Denkw\"urdigkeiten,''} p. 7, Bruno Bauer 
says: ``That middle class, which was to receive such a terrible importance 
for modern history, is capable of no self-sacrificing action, no enthusiasm 
for an idea, no exaltation; it devotes itself to nothing but the interests of 
its mediocrity; \textit{i.e.} it remains always limited to itself, and 
conquers at last only through its bulk, with which it has succeeded in tiring 
out the efforts of passion, enthusiasm, consistency -- through its surface, 
into which it absorbs a part of the new ideas.'' And (p. 6) ``It has turned 
the revolutionary ideas, for which not it, but unselfish or impassioned men 
sacrificed themselves, solely to its own profit, has turned spirit into money. 
-- That is, to be sure, after it had taken away from those ideas their point, 
their consistency, their destructive seriousness, fanatical against all 
egoism.'' These people, then, are not self-sacrificing, not enthusiastic, not 
idealistic, not consistent, not zealots; they are egoists in the usual sense, 
selfish people, looking out for their advantage, sober, calculating, etc.

Who, then, is ``self-sacrificing?''\footnote{[\textit{Literally, 
``sacrificing''; the German word has not the prefix ``self.''}]} In the 
full sense, surely, he who ventures everything else for \textit{one thing}, 
one object, one will, one passion. Is not the lover self-sacrificing who 
forsakes father and mother, endures all dangers and privations, to reach his 
goal? Or the ambitious man, who offers up all his desires, wishes, and 
satisfactions to the single passion, or the avaricious man who denies himself 
everything to gather treasures, or the pleasure-seeker, etc.? He is ruled by a 
passion to which he brings the rest as sacrifices.

And are these self-sacrificing people perchance not selfish, not egoist? As 
they have only one ruling passion, so they provide for only one satisfaction, 
but for this the more strenuously, they are wholly absorbed in it. Their 
entire activity is egoistic, but it is a one-sided, unopened, narrow egoism; 
it is possessedness.

``Why, those are petty passions, by which, on the contrary, man must not let 
himself be enthralled. Man must make sacrifices for a great idea, a great 
cause!'' A ``great idea,'' a ``good cause,'' is, it may be, the honor of 
God, for which innumerable people have met death; Christianity, which has 
found its willing martyrs; the Holy Catholic Church, which has greedily 
demanded sacrifices of heretics; liberty and equality, which were waited on by 
bloody guillotines.

He who lives for a great idea, a good cause, a doctrine, a system, a lofty 
calling, may not let any worldly lusts, any self-seeking interest, spring up 
in him. Here we have the concept of \textit{clericalism}, or, as it may also 
be called in its pedagogic activity, school-masterliness; for the idealists 
play the schoolmaster over us. The clergyman is especially called to live to 
the idea and to work for the idea, the truly good cause. Therefore the people 
feel how little it befits him to show worldly haughtiness, to desire good 
living, to join in such pleasures as dancing and gaming -- in short, to have 
any other than a ``sacred interest.'' Hence, too, doubtless, is derived the 
scanty salary of teachers, who are to feel themselves repaid by the sacredness 
of their calling alone, and to ``renounce'' other enjoyments.

Even a directory of the sacred ideas, one or more of which man is to look upon 
as his calling, is not lacking. Family, fatherland, science, etc., may find in 
me a servant faithful to his calling.

Here we come upon the old, old craze of the world, which has not yet learned 
to do without clericalism -- that to live and work \textit{for an idea} is 
man's calling, and according to the faithfulness of its fulfillment his 
\textit{human} worth is measured.

This is the dominion of the idea; in other words, it is clericalism. Thus 
Robespierre and St. Just were priests through and through, inspired by the 
idea, enthusiasts, consistent instruments of this idea, idealistic men. So St. 
Just exclaims in a speech, ``There is something terrible in the sacred love 
of country; it is so exclusive that it sacrifices everything to the public 
interest without mercy, without fear, without human consideration. It hurls 
Manlius down the precipice; it sacrifices its private inclinations; it leads 
Regulus to Carthage, throws a Roman into the chasm, and sets Marat, as a 
victim of his devotion, in the Pantheon.''

Now, over against these representatives of ideal or sacred interests stands a 
world of innumerable ``personal'' profane interests. No idea, no system, no 
sacred cause is so great as never to be outrivaled and modified by these 
personal interests. Even if they are silent momentarily, and in times of rage, 
and fanaticism, yet they soon come uppermost again through ``the sound sense 
of the people.'' Those ideas do not completely conquer till they are no 
longer hostile to personal interests, till they satisfy egoism.

The man who is just now crying herrings in front of my window has a personal 
interest in good sales, and, if his wife or anybody else wishes him the like, 
this remains a personal interest all the same. If, on the other hand, a thief 
deprived him of his basket, then there would at once arise an interest of 
many, of the whole city, of the whole country, or, in a word, of all who abhor 
theft; an interest in which the herring-seller's person would become 
indifferent, and in its place the category of the ``robbed man'' would come 
into the foreground. But even here all might yet resolve itself into a 
personal interest, each of the partakers reflecting that he must concur in the 
punishment of the thief because unpunished stealing might otherwise become 
general and cause him too to lose his own. Such a calculation, however, can 
hardly be assumed on the part of many, and we shall rather hear the cry that 
the thief is a ``criminal.'' Here we have before us a judgment, the thief's 
action receiving its expression in the concept ``crime.'' Now the matter 
stands thus: even if a crime did not cause the slightest damage either to me 
or to any of those in whom I take an interest, I should nevertheless denounce 
it. Why? Because I am enthusiastic for \textit{morality}, filled with the 
\textit{idea} of morality; what is hostile to it I everywhere assail. Because 
in his mind theft ranks as abominable without any question, Proudhon, 
\textit{e.g.}, thinks that with the sentence ``Property is theft'' he has 
at once put a brand on property. In the sense of the priestly, theft is always 
a \textit{crime}, or at least a misdeed.

Here the personal interest is at an end. This particular person who has stolen 
the basket is perfectly indifferent to my person; it is only the thief, this 
concept of which that person presents a specimen, that I take an interest in. 
The thief and man are in my mind irreconcilable opposites; for one is not 
truly man when one is a thief; one degrades \textit{Man} or ``humanity'' in 
himself when one steals. Dropping out of personal concern, one gets into 
\textit{philanthropy}, friendliness to man, which is usually misunderstood as 
if it was a love to men, to each individual, while it is nothing but a love of 
\textit{Man}, the unreal concept, the spook. It is not \textit{tous 
anthropous,} men, but \textit{ton anthropon}, Man, that the philanthropist 
carries in his heart. To be sure, he cares for each individual, but only 
because he wants to see his beloved ideal realized everywhere.

So there is nothing said here of care for me, you, us; that would be personal 
interest, and belongs under the head of ``worldly love.'' Philanthropy is a 
heavenly, spiritual, a -- priestly love. \textit{Man} must be restored in us, 
even if thereby we poor devils should come to grief. It is the same priestly 
principle as that famous \textit{fiat justitia, pereat mundus}; man and 
justice are ideas, ghosts, for love of which everything is sacrificed; 
therefore, the priestly spirits are the ``self-sacrificing'' ones.

He who is infatuated with \textit{Man} leaves persons out of account so far as 
that infatuation extends, and floats in an ideal, sacred interest. 
\textit{Man}, you see, is not a person, but an ideal, a spook.

Now, things as different as possible can belong to \textit{Man} and be so 
regarded. If one finds Man's chief requirement in piety, there arises 
religious clericalism; if one sees it in morality, then moral clericalism 
raises its head. On this account the priestly spirits of our day want to make 
a ``religion'' of everything, a ``religion of liberty,'' ``religion of 
equality,'' etc., and for them every idea becomes a ``sacred cause,'' 
\textit{e.g.} even citizenship, politics, publicity, freedom of the press, 
trial by jury, etc.

Now, what does ``unselfishness'' mean in this sense? Having only an ideal 
interest, before which no respect of persons avails!

The stiff head of the worldly man opposes this, but for centuries has always 
been worsted at least so far as to have to bend the unruly neck and ``honor 
the higher power''; clericalism pressed it down. When the worldly egoist had 
shaken off a higher power (\textit{e.g.} the Old Testament law, the Roman 
pope, etc.), then at once a seven times higher one was over him again, 
\textit{e.g.} faith in the place of the law, the transformation of all laymen 
into divines in place of the limited body of clergy, etc. His experience was 
like that of the possessed man into whom seven devils passed when he thought 
he had freed himself from one.

In the passage quoted above, all ideality is denied to the middle class. It 
certainly schemed against the ideal consistency with which Robespierre wanted 
to carry out the principle. The instinct of its interest told it that this 
consistency harmonized too little with what its mind was set on, and that it 
would be acting against itself if it were willing to further the enthusiasm 
for principle. Was it to behave so unselfishly as to abandon all its aims in 
order to bring a harsh theory to its triumph? It suits the priests admirably, 
to be sure, when people listen to their summons, ``Cast away everything and 
follow me,'' or ``Sell all that thou hast and give to the poor, and thou 
shalt have treasure in heaven; and come, follow me.'' Some decided idealists 
obey this call; but most act like Ananias and Sapphira, maintaining a behavior 
half clerical or religious and half worldly, serving God and Mammon.

I do not blame the middle class for not wanting to let its aims be frustrated 
by Robespierre, \textit{i.e.} for inquiring of its egoism how far it might 
give the revolutionary idea a chance. But one might blame (if blame were in 
place here anyhow) those who let their own interests be frustrated by the 
interests of the middle class. However, will not they likewise sooner or later 
learn to understand what is to their advantage? August Becker 
says:\footnote{\textit{``Die Volksphilosophie unserer Tage}'', p. 22.} ``To 
win the producers (proletarians) a negation of the traditional conception of 
right is by no means enough. Folks unfortunately care little for the 
theoretical victory of the idea. One must demonstrate to them \textit{ad 
oculos} how this victory can be practically utilized in life.'' And (p.32): 
``You must get hold of folks by their real interests if you want to work upon 
them.'' Immediately after this he shows how a fine looseness of morals is 
already spreading among our peasants, because they prefer to follow their real 
interests rather than the commands of morality.

Because the revolutionary priests or schoolmasters served \textit{Man}, they 
cut off the heads of \textit{men}. The revolutionary laymen, those outside the 
sacred circle, did not feel any greater horror of cutting off heads, but were 
less anxious about the rights of Man than about their own.

How comes it, though, that the egoism of those who affirm personal interest, 
and always inquire of it, is nevertheless forever succumbing to a priestly or 
schoolmasterly (\textit{i.e.} an ideal) interest? Their person seems to them 
too small, too insignificant -- and is so in fact -- to lay claim to 
everything and be able to put itself completely in force. There is a sure sign 
of this in their dividing themselves into two persons, an eternal and a 
temporal, and always caring either only for the one or only for the other, on 
Sunday for the eternal, on the work-day for the temporal, in prayer for the 
former, in work for the latter. They have the priest in themselves, therefore 
they do not get rid of him, but hear themselves lectured inwardly every 
Sunday.

How men have struggled and calculated to get at a solution regarding these 
dualistic essences! Idea followed upon idea, principle upon principle, system 
upon system, and none knew how to keep down permanently the contradiction of 
the ``worldly'' man, the so-called ``egoist.'' Does not this prove that 
all those ideas were too feeble to take up my whole will into themselves and 
satisfy it? They were and remained hostile to me, even if the hostility lay 
concealed for a considerable time. Will it be the same with 
\textit{self-ownership?} Is it too only an attempt at mediation? Whatever 
principle I turned to, it might be to that of \textit{reason}, I always had to 
turn away from it again. Or can I always be rational, arrange my life 
according to reason in everything? I can, no doubt, \textit{strive} after 
rationality, I can \textit{love} it, just as I can also love God and every 
other idea. I can be a philosopher, a lover of wisdom, as I love God. But what 
I love, what I strive for, is only in my idea, my conception, my thoughts; it 
is in my heart, my head, it is in me like the heart, but it is not I, I am not 
it.

To the activity of priestly minds belongs especially what one often hears 
called \textit{``moral influence.''}

Moral influence takes its start where \textit{humiliation} begins; yes, it is 
nothing else than this humiliation itself, the breaking and bending of the 
temper\footnote{[\textit{Muth}]} down to humility.\footnote{[\textit{Demuth}]} 
If I call to some one to run away when a rock is to be blasted, I exert no 
moral influence by this demand; if I say to a child ``You will go hungry if 
you will not eat what is put on the table,'' this is not moral influence. 
But, if I say to it, ``You will pray, honor your parents, respect the 
crucifix, speak the truth, for this belongs to man and is man's calling,'' or 
even ``this is God's will,'' then moral influence is complete; then a man is 
to bend before the \textit{calling} of man, be tractable, become humble, give 
up his will for an alien one which is set up as rule and law; he is to 
\textit{abase} himself before something \textit{higher}: self-abasement. ``He 
that abaseth himself shall be exalted.'' Yes, yes, children must early be 
\textit{made} to practice piety, godliness, and propriety; a person of good 
breeding is one into whom ``good maxims'' have been \textit{instilled} and 
\textit{impressed}, poured in through a funnel, thrashed in and preached in.

If one shrugs his shoulders at this, at once the good wring their hands 
despairingly, and cry: ``But, for heaven's sake, if one is to give children 
no good instruction, why, then they will run straight into the jaws of sin, 
and become good-for-nothing hoodlums!'' Gently, you prophets of evil. 
Good-for-nothing in your sense they certainly will become; but your sense 
happens to be a very good-for-nothing sense. The impudent lads will no longer 
let anything be whined and chattered into them by you, and will have no 
sympathy for all the follies for which you have been raving and driveling 
since the memory of man began; they will abolish the law of inheritance; they 
will not be willing to \textit{inherit} your stupidities as you inherited them 
from your fathers; they destroy \textit{inherited sin}.\footnote{[Called in 
English theology ``original sin.'']} If you command them, ``Bend before the 
Most High,'' they will answer: ``If he wants to bend us, let him come 
himself and do it; we, at least, will not bend of our own accord.'' And, if 
you threaten them with his wrath and his punishment, they will take it like 
being threatened with the bogie-man. If you are no more successful in making 
them afraid of ghosts, then the dominion of ghosts is at an end, and nurses' 
tales find no -- \textit{faith}.

And is it not precisely the liberals again that press for good education and 
improvement of the educational system? For how could their liberalism, their 
``liberty within the bounds of law,'' come about without discipline? Even if 
they do not exactly educate to the fear of God, yet they demand the 
\textit{fear of Man} all the more strictly, and awaken ``enthusiasm for the 
truly human calling'' by discipline.

\myhrule


A long time passed away, in which people were satisfied with the fancy that 
they had the \textit{truth}, without thinking seriously whether perhaps they 
themselves must be true to possess the truth. This time was the \textit{Middle 
Ages}. With the common consciousness -- \textit{i.e.} the consciousness which 
deals with things, that consciousness which has receptivity only for things, 
or for what is sensuous and sense-moving -- they thought to grasp what did not 
deal with things and was not perceptible by the senses. As one does indeed 
also exert his eye to see the remote, or laboriously exercise his hand till 
its fingers have become dexterous enough to press the keys correctly, so they 
chastened themselves in the most manifold ways, in order to become capable of 
receiving the supersensual wholly into themselves. But what they chastened 
was, after all, only the sensual man, the common consciousness, so-called 
finite or objective thought. Yet as this thought, this understanding, which 
Luther decries under the name of reason, is incapable of comprehending the 
divine, its chastening contributed just as much to the understanding of the 
truth as if one exercised the feet year in and year out in dancing, and hoped 
that in this way they would finally learn to play the flute. Luther, with whom 
the so-called Middle Ages end, was the first who understood that the man 
himself must become other than he was if he wanted to comprehend truth -- must 
become as true as truth itself. Only he who already has truth in his belief, 
only he who \textit{believes} in it, can become a partaker of it; 
\textit{i.e.} only the believer finds it accessible and sounds its depths. 
Only that organ of man which is able to blow can attain the further capacity 
of flute-playing, and only that man can become a partaker of truth who has the 
right organ for it. He who is capable of thinking only what is sensuous, 
objective, pertaining to things, figures to himself in truth only what 
pertains to things. But truth is spirit, stuff altogether inappreciable by the 
senses, and therefore only for the ``higher consciousness,'' not for that 
which is ``earthly-minded.''

With Luther, accordingly, dawns the perception that truth, because it is a 
\textit{thought}, is only for the \textit{thinking} man. And this is to say 
that man must henceforth take an utterly different standpoint, to wit, the 
heavenly, believing, scientific standpoint, or that of \textit{thought} in 
relation to its object, the -- \textit{thought} -- that of mind in relation to 
mind. Consequently: only the like apprehend the like. ``You are like the 
spirit that you understand.''\footnote{[Goethe, ``Faust''.]}

Because Protestantism broke the medieval hierarchy, the opinion could take 
root that hierarchy in general had been shattered by it, and it could be 
wholly overlooked that it was precisely a ``reformation,'' and so a 
reinvigoration of the antiquated hierarchy. That medieval hierarchy had been 
only a weakly one, as it had to let all possible barbarism of unsanctified 
things run on uncoerced beside it, and it was the Reformation that first 
steeled the power of hierarchy. If Bruno Bauer 
thinks:\footnote{\textit{``Anekdota''}, II, 152.} ``As the Reformation was 
mainly the abstract rending of the religious principle from art, State, and 
science, and so its liberation from those powers with which it had joined 
itself in the antiquity of the church and in the hierarchy of the Middle Ages, 
so too the theological and ecclesiastical movements which proceeded from the 
Reformation are only the consistent carrying out of this abstraction of the 
religious principle from the other powers of humanity,'' I regard precisely 
the opposite as correct, and think that the dominion of spirits, or freedom of 
mind (which comes to the same thing), was never before so all-embracing and 
all-powerful, because the present one, instead of rending the religious 
principle from art, State, and science, lifted the latter altogether out of 
secularity into the ``realm of spirit'' and made them religious.

Luther and Descartes have been appropriately put side by side in their ``He 
who believes in God'' and ``I think, therefore I am'' (\textit{cogito, ergo 
sum}). Man's heaven is thought -- mind. Everything can be wrested from him, 
except thought, except faith. \textit{Particular} faith, like faith of Zeus, 
Astarte, Jehovah, Allah, may be destroyed, but faith itself is indestructible. 
In thought is freedom. What I need and what I hunger for is no longer granted 
to me by any \textit{grace}, by the Virgin Mary. by intercession of the 
saints, or by the binding and loosing church, but I procure it for myself. In 
short, my being (the \textit{sum}) is a living in the heaven of thought, of 
mind, a \textit{cogitare}. But I myself am nothing else than mind, thinking 
mind (according to Descartes), believing mind (according to Luther). My body I 
am not; my flesh may \textit{suffer} from appetites or pains. I am not my 
flesh, but I am \textit{mind}, only mind.

This thought runs through the history of the Reformation till today.

Only by the more modern philosophy since Descartes has a serious effort been 
made to bring Christianity to complete efficacy, by exalting the ``scientific 
consciousness.'' to be the only true and valid one. Hence it begins with 
absolute \textit{doubt, dubitare}, with grinding common consciousness to 
atoms, with turning away from everything that ``mind,'' ``thought,'' does 
not legitimate. To it \textit{Nature} counts for nothing; the opinion of men, 
their ``human precepts,'' for nothing: and it does not rest till it has 
brought reason into everything, and can say ``The real is the rational, and 
only the rational is the real.'' Thus it has at last brought mind, reason, to 
victory; and everything is mind, because everything is rational, because all 
nature, as well as even the most perverse opinions of men, contains reason; 
for ``all must serve for the best,'' \textit{i.e.}, lead to the victory of 
reason.

Descartes's \textit{dubitare} contains the decided statement that only 
\textit{cogitare}, thought, mind -- \textit{is}. A complete break with 
``common'' consciousness, which ascribes reality to \textit{irrational} 
things! Only the rational is, only mind is! This is the principle of modern 
philosophy, the genuine Christian principle. Descartes in his own time 
discriminated the body sharply from the mind, and ``the spirit 'tis that 
builds itself the body,'' says Goethe.

But this philosophy itself, Christian philosophy, still does not get rid of 
the rational, and therefore inveighs against the ``merely subjective,'' 
against ``fancies, fortuities, arbitrariness,'' etc. What it wants is that 
the \textit{divine} should become visible in everything, and all consciousness 
become a knowing of the divine, and man behold God everywhere; but God never 
is, without the \textit{devil}. For this very reason the name of philosopher 
is not to be given to him who has indeed open eyes for the things of the 
world, a clear and undazzled gaze, a correct judgment about the world, but who 
sees in the world just the world, in objects only objects, and, in short, 
everything prosaically as it is; but he alone is a philosopher who sees, and 
points out or demonstrates, heaven in the world, the supernal in the earthly, 
the -- \textit{divine} in the mundane. The former may be ever so wise, there 
is no getting away from this:

\begin{quotation}

\noindent{}What wise men see not by their wisdom's art\\
 Is practiced simply by a childlike heart.\footnote{[Schiller, \textit{``Die 
Worte des Glaubens}''.]}\end{quotation}

\noindent{}It takes this childlike heart, this eye for the divine, to make a 
philosopher. The first-named man has only a ``common'' consciousness, but he 
who knows the divine, and knows how to tell it, has a ``scientific'' one. On 
this ground Bacon was turned out of the realm of philosophers. And certainly 
what is called English philosophy seems to have got no further than to the 
discoveries of so-called ``clear heads,'' \textit{e.g.} Bacon and Hume. The 
English did not know how to exalt the simplicity of the childlike heart to 
philosophic significance, did not know how to make -- philosophers out of 
childlike hearts. This is as much as to say, their philosophy was not able to 
become \textit{theological} or \textit{theology}, and yet it is only as 
theology that it can really \textit{live itself} out, complete itself. The 
field of its battle to the death is in theology. Bacon did not trouble himself 
about theological questions and cardinal points.

Cognition has its object in life. German thought seeks, more than that of 
others, to reach the beginnings and fountain-heads of life, and sees no life 
till it sees it in cognition itself. Descartes's \textit{cogito, ergo sum} has 
the meaning ``One lives only when one thinks.'' Thinking life is called 
``intellectual life''! Only mind lives, its life is the true life. Then, 
just so in nature only the ``eternal laws,'' the mind or the reason of 
nature, are its true life. In man, as in nature, only the thought lives; 
everything else is dead! To this abstraction, to the life of generalities or 
of that which is \textit{lifeless}, the history of mind had to come. God, who 
is spirit, alone lives. Nothing lives but the ghost.

How can one try to assert of modern philosophy or modern times that they have 
reached freedom, since they have not freed us from the power of objectivity? 
Or am I perhaps free from a despot when I am not afraid of the personal 
potentate, to be sure, but of every infraction of the loving reverence which I 
fancy I owe him? The case is the same with modern times. They only changed the 
\textit{existing} objects, the real ruler, into \textit{conceived} objects, 
\textit{i.e.} into \textit{ideas}, before which the old respect not only was 
not lost, but increased in intensity. Even if people snapped their fingers at 
God and the devil in their former crass reality, people devoted only the 
greater attention to their ideas. ``They are rid of the Evil One; evil is 
left.''\footnote{[Parodied from the words of Mephistopheles in the witch's 
kitchen in ``Faust''.]} The decision having once been made not to let 
oneself be imposed on any longer by the extant and palpable, little scruple 
was felt about revolting against the existing State or overturning the 
existing laws; but to sin against the \textit{idea} of the State, not to 
submit to the \textit{idea} of law, who would have dared that? So one remained 
a ``citizen'' and a ``law-respecting,'' loyal man; yes, one seemed to 
himself to be only so much more law-respecting, the more rationalistically one 
abrogated the former defective law in order to do homage to the ``spirit of 
the law.'' In all this the objects had only suffered a change of form; they 
had remained in their preponderance and pre-eminence; in short, one was still 
involved in obedience and possessedness, lived in reflection, and had an 
object on which one reflected, which one respected, and before which one felt 
reverence and fear. One had done nothing but transform the \textit{things} 
into \textit{conceptions} of the things, into thoughts and ideas, whereby 
one's \textit{dependence} became all the more intimate and indissoluble. So, 
\textit{e.g.}, it is not hard to emancipate oneself from the commands of 
parents, or to set aside the admonitions of uncle and aunt, the entreaties of 
brother and sister; but the renounced obedience easily gets into one's 
conscience, and the less one does give way to the individual demands, because 
he rationalistically, by his own reason, recognizes them to be unreasonable, 
so much the more conscientiously does he hold fast to filial piety and family 
love, and so much the harder is it for him to forgive himself a trespass 
against the \textit{conception} which he has formed of family love and of 
filial duty. Released from dependence as regards the existing family, one 
falls into the more binding dependence on the idea of the family; one is ruled 
by the spirit of the family. The family consisting of John, Maggie, etc., 
whose dominion has become powerless, is only internalized, being left as 
``family'' in general, to which one just applies the old saying, ``We must 
obey God rather than man,'' whose significance here is this: ``I cannot, to 
be sure, accommodate myself to your senseless requirements, but, as my 
'family,' you still remain the object of my love and care''; for ``the 
family'' is a sacred idea, which the individual must never offend against. -- 
And this family internalized and desensualized into a thought, a conception, 
now ranks as the ``sacred,'' whose despotism is tenfold more grievous 
because it makes a racket in my conscience. This despotism is broken when the 
conception, family, also becomes a \textit{nothing} to me The Christian dicta, 
``Woman, what have I to do with thee?''\footnote{Matt. 10. 35.} ``I am come 
to stir up a man against his father, and a daughter against her 
mother,''\footnote{John 2. 4.} and others, are accompanied by something that 
refers us to the heavenly or true family, and mean no more than the State's 
demand, in case of a collision between it and the family, that we obey 
\textit{its} commands.

The case of morality is like that of the family. Many a man renounces morals, 
but with great difficulty the conception, ``morality.'' Morality is the 
``idea'' of morals, their intellectual power, their power over the 
conscience; on the other hand, morals are too material to rule the mind, and 
do not fetter an ``intellectual'' man, a so-called independent, a 
``freethinker.''

The Protestant may put it as he will, the ``holy\footnote{[\textit{heilig}]} 
Scripture,'' the ``Word of God,'' still remains 
sacred\footnote{[\textit{heilig}]} for him. He for whom this is no longer 
``holy'' has ceased to -- be a Protestant. But herewith what is 
``ordained'' in it, the public authorities appointed by God, etc., also 
remain sacred for him. For him these things remain indissoluble, 
unapproachable, ``raised above all doubt''; and, as \textit{doubt}, which in 
practice becomes a \textit{buffeting}, is what is most man's own, these things 
remain ``raised'' above himself. He who cannot \textit{get away} from them 
will -- \textit{believe}; for to believe in them is to be \textit{bound} to 
them. Through the fact that in Protestantism the \textit{faith} becomes a more 
inward faith, the \textit{servitude} has also become a more inward servitude; 
one has taken those sanctities up into himself, entwined them with all his 
thoughts and endeavors, made them a \textit{``matter of conscience''}, 
constructed out of them a \textit{``sacred duty''} for himself. Therefore 
what the Protestant's conscience cannot get away from is sacred to him, and 
\textit{conscientiousness} most clearly designates his character.

Protestantism has actually put a man in the position of a country governed by 
secret police. The spy and eavesdropper, ``conscience,'' watches over every 
motion of the mind, and all thought and action is for it a ``matter of 
conscience,'' \textit{i.e.}, police business. This tearing apart of man into 
``natural impulse'' and ``conscience'' (inner populace and inner police) 
is what constitutes the Protestant. The reason of the Bible (in place of the 
Catholic ``reason of the church'') ranks as sacred, and this feeling and 
consciousness that the word of the Bible is sacred is called -- conscience. 
With this, then, sacredness is ``laid upon one's conscience.'' If one does 
not free himself from conscience, the consciousness of the sacred, he may act 
unconscientiously indeed, but never consciencelessly.

The Catholic finds himself satisfied when he fulfills the \textit{command}; 
the Protestant acts according to his ``best judgment and conscience.'' For 
the Catholic is only a \textit{layman}; the Protestant is himself a 
\textit{clergyman}.\footnote{[\textit{Geistlicher}, literally ``spiritual 
man.'']} Just this is the progress of the Reformation period beyond the 
Middle Ages, and at the same time its curse -- that the \textit{spiritual} 
became complete.

What else was the Jesuit moral philosophy than a continuation of the sale of 
indulgences? Only that the man who was relieved of his burden of sin now 
gained also an \textit{insight} into the remission of sins, and convinced 
himself how really his sin was taken from him, since in this or that 
particular case (casuists) it was so clearly no sin at all that he committed. 
The sale of indulgences had made all sins and transgressions permissible, and 
silenced every movement of conscience. All sensuality might hold sway, if it 
was only purchased from the church. This favoring of sensuality was continued 
by the Jesuits, while the strictly moral, dark, fanatical, repentant, 
contrite, praying Protestants (as the true completers of Christianity, to be 
sure) acknowledged only the intellectual and spiritual man. Catholicism, 
especially the Jesuits, gave aid to egoism in this way, found involuntary and 
unconscious adherents within Protestantism itself, and saved us from the 
subversion and extinction of \textit{sensuality}. Nevertheless the Protestant 
spirit spreads its dominion farther and farther; and, as, beside it the 
``divine,'' the Jesuit spirit represents only the ``diabolic'' which is 
inseparable from everything divine, the latter can never assert itself alone, 
but must look on and see how in France, \textit{e.g.}, the Philistinism of 
Protestantism wins at last, and mind is on top.

Protestantism is usually complimented on having brought the mundane into 
repute again, \textit{e.g.} marriage, the State, etc. But the mundane itself 
as mundane, the secular, is even more indifferent to it than to Catholicism, 
which lets the profane world stand, yes, and relishes its pleasures, while the 
rational, consistent Protestant sets about annihilating the mundane 
altogether, and that simply by \textit{hallowing} it. So marriage has been 
deprived of its naturalness by becoming sacred, not in the sense of the 
Catholic sacrament, where it only receives its consecration from the church 
and so is unholy at bottom, but in the sense of being something sacred in 
itself to begin with, a sacred relation. Just so the State, also. Formerly the 
pope gave consecration and his blessing to it and its princes, now the State 
is intrinsically sacred, majesty is sacred without needing the priest's 
blessing. The order of nature, or natural law, was altogether hallowed as 
``God's ordinance.'' Hence it is said \textit{e.g.} in the Augsburg 
Confession, Art. II: ``So now we reasonably abide by the saying, as the 
jurisconsults have wisely and rightly said: that man and woman should be with 
each other is a natural law. Now, if it is a \textit{natural law, then it is 
God's ordinance}, therefore implanted in nature, and therefore a 
\textit{divine} law also.'' And is it anything more than Protestantism 
brought up to date, when Feuerbach pronounces moral relations sacred, not as 
God's ordinance indeed, but, instead, for the sake of the \textit{spirit} that 
dwells in them? ``But marriage as a free alliance of love, of course -- is 
\textit{sacred of itself}, by the nature of the union that is formed here. 
\textit{That} marriage alone is a \textit{religious} one that is a 
\textit{true} one, that corresponds to the \textit{essence} of marriage, love. 
And so it is with all moral relations. They are \textit{ethical}, are 
cultivated with a moral mind, only where they rank as \textit{religious of 
themselves}. True friendship is only where the \textit{limits} of friendship 
are preserved with religious conscientiousness, with the same 
conscientiousness with which the believer guards the dignity of his God. 
Friendship is and must be \textit{sacred} for you, and property, and marriage, 
and the good of every man, but sacred \textit{in and of 
itself.''}\footnote{``Essence of Christianity'', p. 403.}

That is a very essential consideration. In Catholicism the mundane can indeed 
be \textit{consecrated} or \textit{hallowed}, but it is not sacred without 
this priestly blessing; in Protestantism, on the contrary, mundane relations 
are sacred \textit{of themselves}, sacred by their mere existence. The Jesuit 
maxim, ``the end hallows the means,'' corresponds precisely to the 
consecration by which sanctity is bestowed. No means are holy or unholy in 
themselves, but their relation to the church, their use for the church, 
hallows the means. Regicide was named as such; if it was committed for the 
church's behoof, it could be certain of being hallowed by the church, even if 
the hallowing was not openly pronounced. To the Protestant, majesty ranks as 
sacred; to the Catholic only that majesty which is consecrated by the pontiff 
can rank as such; and it does rank as such to him only because the pope, even 
though it be without a special act, confers this sacredness on it once for 
all. If he retracted his consecration, the king would be left only a ``man of 
the world or layman,'' an ``unconsecrated'' man, to the Catholic.

If the Protestant seeks to discover a sacredness in the sensual itself, that 
he may then be linked only to what is holy, the Catholic strives rather to 
banish the sensual from himself into a separate domain, where it, like the 
rest of nature, keeps its value for itself. The Catholic church eliminated 
mundane marriage from its consecrated order, and withdrew those who were its 
own from the mundane family; the Protestant church declared marriage and 
family ties to be holy, and therefore not unsuitable for its clergymen.

A Jesuit may, as a good Catholic, hallow everything. He needs only, \textit{e. 
g.}, to say to himself: ``I as a priest am necessary to the church, but serve 
it more zealously when I appease my desires properly; consequently I will 
seduce this girl, have my enemy there poisoned, etc.; my end is holy because 
it is a priest's, consequently it hallows the means.'' For in the end it is 
still done for the benefit of the church. Why should the Catholic priest 
shrink from handing Emperor Henry VII the poisoned wafer for the -- church's 
welfare?

The genuinely churchly Protestants inveighed against every ``innocent 
pleasure,'' because only the sacred, the spiritual, could be innocent. What 
they could not point out the holy spirit in, the Protestants had to reject -- 
dancing, the theatre, ostentation (\textit{e.g.} in the church), and the 
like.

Compared with this puritanical Calvinism, Lutheranism is again more on the 
religious, spiritual, track -- is more radical. For the former excludes at 
once a great number of things as sensual and worldly, and \textit{purifies} 
the church; Lutheranism, on the contrary, tries to bring \textit{spirit} into 
all things as far as possible, to recognize the holy spirit as an essence in 
everything, and so to \textit{hallow} everything worldly. (``No one can 
forbid a kiss in honor.'' The spirit of honor hallows it.) Hence it was that 
the Lutheran Hegel (he declares himself such in some passage or other: he 
``wants to remain a Lutheran'') was completely successful in carrying the 
idea through everything. In everything there is reason, \textit{i.e.} holy 
spirit, or ``the real is rational.'' For the real is in fact everything; as 
in each thing, \textit{e.g.}, each lie, the truth can be detected: there is 
no absolute lie, no absolute evil, etc.

Great ``works of mind'' were created almost solely by Protestants, as they 
alone were the true disciples and consummators of \textit{mind}.

\myhrule


How little man is able to control! He must let the sun run its course, the sea 
roll its waves, the mountains rise to heaven. Thus he stands powerless before 
the \textit{uncontrollable}. Can he keep off the impression that he is 
helpless against this gigantic world? It is a fixed \textit{law} to which he 
must submit, it determines his \textit{fate}. Now, what did pre-Christian 
humanity work toward? Toward getting rid of the irruptions of the destinies, 
not letting oneself be vexed by them. The Stoics attained this in apathy, 
declaring the attacks of nature \textit{indifferent}, and not letting 
themselves be affected by them. Horace utters the famous \textit{Nil 
admirari}, by which he likewise announces the indifference of the 
\textit{other}, the world; it is not to influence us, not to rouse our 
astonishment. And that \textit{impavidum ferient ruinae} expresses the very 
same \textit{imperturbability} as Ps. 46.3: ``We do not fear, though the 
earth should perish.'' In all this there is room made for the Christian 
proposition that the world is empty, for the Christian \textit{contempt of the 
world}.

The \textit{imperturbable} spirit of ``the wise man,'' with which the old 
world worked to prepare its end, now underwent an \textit{inner perturbation} 
against which no ataraxia, no Stoic courage, was able to protect it. The 
spirit, secured against all influence of the world, insensible to its shocks 
and \textit{exalted} above its attacks, admiring nothing, not to be 
disconcerted by any downfall of the world -- foamed over irrepressibly again, 
because gases (spirits) were evolved in its own interior, and, after the 
\textit{mechanical shock} that comes from without had become ineffective, 
\textit{chemical tensions}, that agitate within, began their wonderful play.

In fact, ancient history ends with this -- that \textit{I} have struggled till 
I won my ownership of the world. ``All things have been delivered to me by my 
Father'' (Matt. 11. 27). It has ceased to be overpowering, unapproachable, 
sacred, divine, for me; it is \textit{undeified}, and now I treat it so 
entirely as I please that, if I cared, I could exert on it all miracle-working 
power, \textit{i.e.}, power of mind -- remove mountains, command mulberry 
trees to tear themselves up and transplant themselves into the sea (Luke 
17.6), and do everything possible, \textit{thinkable} : ``All things are 
possible to him who believes.''\footnote{Mark. 9. 23.} I am the \textit{lord} 
of the world, mine is the ``glory.''\footnote{[\textit{Herrlichkeit}, which, 
according to its derivation, means ``lordliness.'']} The world has become 
prosaic, for the divine has vanished from it: it is my property, which I 
dispose of as I (to wit, the mind) choose.

When I had exalted myself to be the \textit{owner of the world}, egoism had 
won its first complete victory, had vanquished the world, had become 
worldless, and put the acquisitions of a long age under lock and key.

The first property, the first ``glory,'' has been acquired!

But the lord of the world is not yet lord of his thoughts, his feelings, his 
will: he is not lord and owner of the spirit, for the spirit is still sacred, 
the ``Holy Spirit,'' and the ``worldless'' Christian is not able to become 
``godless.'' If the ancient struggle was a struggle against the 
\textit{world}, the medieval (Christian) struggle is a struggle against self, 
the mind; the former against the outer world, the latter against the inner 
world. The medieval man is the man ``whose gaze is turned inward,'' the 
thinking, meditative

All wisdom of the ancients is \textit{the science of the world}, all wisdom of 
the moderns is \textit{the science of God}.

The heathen (Jews included) got through with the \textit{world}; but now the 
thing was to get through with self, the spirit, too; \textit{i.e.} to become 
spiritless or godless.

For almost two thousand years we have been working at subjecting the Holy 
Spirit to ourselves, and little by little we have torn off and trodden under 
foot many bits of sacredness; but the gigantic opponent is constantly rising 
anew under a changed form and name. The spirit has not yet lost its divinity, 
its holiness, its sacredness. To be sure, it has long ceased to flutter over 
our heads as a dove; to be sure, it no longer gladdens its saints alone, but 
lets itself be caught by the laity too; but as spirit of humanity, as spirit 
of Man, it remains still an \textit{alien} spirit to me or you, still far from 
becoming our unrestricted \textit{property}, which we dispose of at our 
pleasure. However, one thing certainly happened, and visibly guided the 
progress of post-Christian history: this one thing was the endeavor to make 
the Holy Spirit \textit{more human}, and bring it nearer to men, or men to it. 
Through this it came about that at last it could be conceived as the ``spirit 
of humanity,'' and, under different expressions like ``idea of humanity, 
mankind, humaneness, general philanthropy,'' appeared more attractive, more 
familiar, and more accessible.

Would not one think that now everybody could possess the Holy Spirit, take up 
into himself the idea of humanity, bring mankind to form and existence in 
himself?

No, the spirit is not stripped of its holiness and robbed of its 
unapproachableness, is not accessible to us, not our property; for the spirit 
of humanity is not \textit{my} spirit. My \textit{ideal} it may be, and as a 
thought I call it mine; the \textit{thought} of humanity is my property, and I 
prove this sufficiently by propounding it quite according to my views, and 
shaping it today so, tomorrow otherwise; we represent it to ourselves in the 
most manifold ways. But it is at the same time an entail, which I cannot 
alienate nor get rid of.

Among many transformations, the Holy Spirit became in time the 
\textit{``absolute idea''}, which again in manifold refractions split into 
the different ideas of philanthropy, reasonableness, civic virtue, etc.

But can I call the idea my property if it is the idea of humanity, and can I 
consider the Spirit as vanquished if I am to serve it, ``sacrifice myself'' 
to it? Antiquity, at its close, had gained its ownership of the world only 
when it had broken the world's overpoweringness and ``divinity,'' recognized 
the world's powerlessness and ``vanity.''

The case with regard to the \textit{spirit} corresponds. When I have degraded 
it to a \textit{spook} and its control over me to a \textit{cranky notion}, 
then it is to be looked upon as having lost its sacredness, its holiness, its 
divinity, and then I \textit{use} it, as one uses \textit{nature} at pleasure 
without scruple.

The ``nature of the case,'' the ``concept of the relationship,'' is to 
guide me in dealing with the case or in contracting the relation. As if a 
concept of the case existed on its own account, and was not rather the concept 
that one forms of the case! As if a relation which we enter into was not, by 
the uniqueness of those who enter into it, itself unique! As if it depended on 
how others stamp it! But, as people separated the ``essence of Man'' from 
the real man, and judged the latter by the former, so they also separate his 
action from him, and appraise it by ``human value.'' \textit{Concepts} are 
to decide everywhere, concepts to regulate life, concepts to \textit{rule}. 
This is the religious world, to which Hegel gave a systematic expression, 
bringing method into the nonsense and completing the conceptual precepts into 
a rounded, firmly-based dogmatic. Everything is sung according to concepts, 
and the real man, \textit{i.e.} I, am compelled to live according to these 
conceptual laws. Can there be a more grievous dominion of law, and did not 
Christianity confess at the very beginning that it meant only to draw 
Judaism's dominion of law tighter? (``Not a letter of the law shall be 
lost!'')

Liberalism simply brought other concepts on the carpet; human instead of 
divine, political instead of ecclesiastical, ``scientific'' instead of 
doctrinal, or, more generally, real concepts and eternal laws instead of 
``crude dogmas'' and precepts.

Now nothing but \textit{mind} rules in the world. An innumerable multitude of 
concepts buzz about in people's heads, and what are those doing who endeavor 
to get further? They are negating these concepts to put new ones in their 
place! They are saying: ``You form a false concept of right, of the State, of 
man, of liberty, of truth, of marriage, etc.; the concept of right, etc., is 
rather that one which we now set up.'' Thus the confusion of concepts moves 
forward.

The history of the world has dealt cruelly with us, and the spirit has 
obtained an almighty power. You must have regard for my miserable shoes, which 
could protect your naked foot, my salt, by which your potatoes would become 
palatable, and my state-carriage, whose possession would relieve you of all 
need at once; you must not reach out after them. Man is to recognize the 
\textit{independence} of all these and innumerable other things: they are to 
rank in his mind as something that cannot be seized or approached, are to be 
kept away from him. He must have regard for it, respect it; woe to him if he 
stretches out his fingers desirously; we call that ``being light-fingered!''

How beggarly little is left us, yes, how really nothing! Everything has been 
removed, we must not venture on anything unless it is given us; we continue to 
live only by the \textit{grace} of the giver. You must not pick up a pin, 
unless indeed you have got \textit{leave} to do so. And got it from whom? From 
\textit{respect!} Only when this lets you have it as property, only when you 
can \textit{respect} it as property, only then may you take it. And again, you 
are not to conceive a thought, speak a syllable, commit an action, that should 
have their warrant in you alone, instead of receiving it from morality or 
reason or humanity. Happy \textit{unconstraint} of the desirous man, how 
mercilessly people have tried to slay you on the altar of \textit{constraint!}

But around the altar rise the arches of a church, and its walls keep moving 
further and further out. What they enclose is \textit{sacred}. You can no 
longer get to it, no longer touch it. Shrieking with the hunger that devours 
you, you wander round about these walls in search of the little that is 
profane, and the circles of your course keep growing more and more extended. 
Soon that church will embrace the whole world, and you be driven out to the 
extreme edge; another step, and the \textit{world of the sacred} has 
conquered: you sink into the abyss. Therefore take courage while it is yet 
time, wander about no longer in the profane where now it is dry feeding, dare 
the leap, and rush in through the gates into the sanctuary itself. If you 
\textit{devour the sacred}, you have made it your \textit{own!} Digest the 
sacramental wafer, and you are rid of it!

\section[3. The Free]{\centering 3. The Free}

The ancients and the moderns having been presented above in two divisions, it 
may seem as if the free were here to be described in a third division as 
independent and distinct. This is not so. The free are only the more modern 
and most modern among the ``moderns,'' and are put in a separate division 
merely because they belong to the present, and what is present, above all, 
claims our attention here. I give ``the free'' only as a translation of 
``the liberals,'' but must with regard to the concept of freedom (as in 
general with regard to so many other things whose anticipatory introduction 
cannot be avoided) refer to what comes later.

\subsection[\S{}1. Political Liberalism]{\centering \S{}1. Political Liberalism}

After the chalice of so-called absolute monarchy had been drained down to the 
dregs, in the eighteenth century people became aware that their drink did not 
taste human -- too clearly aware not to begin to crave a different cup. Since 
our fathers were ``human beings'' after all, they at last desired also to be 
regarded as such.

Whoever sees in us something else than human beings, in him we likewise will 
not see a human being, but an inhuman being, and will meet him as an unhuman 
being; on the other hand, whoever recognizes us as human beings and protects 
us against the danger of being treated inhumanly, him we will honor as our 
true protector and guardian.

Let us then hold together and protect the man in each other; then we find the 
necessary protection in our \textit{holding together}, and in ourselves, 
\textit{those who hold together}, a fellowship of those who know their human 
dignity and hold together as ``human beings.'' Our holding together is the 
\textit{State}; we who hold together are the \textit{nation}.

In our being together as nation or State we are only human beings. How we 
deport ourselves in other respects as individuals, and what self-seeking 
impulses we may there succumb to, belongs solely to our \textit{private} life; 
our public or State life is a \textit{purely human} one. Everything un-human 
or ``egoistic'' that clings to us is degraded to a ``private matter'' and 
we distinguish the State definitely from ``civil society,'' which is the 
sphere of ``egoism's'' activity.

The true man is the nation, but the individual is always an egoist. Therefore 
strip off your individuality or isolation wherein dwells discord and egoistic 
inequality, and consecrate yourselves wholly to the true man -- the nation or 
the State. Then you will rank as men, and have all that is man's; the State, 
the true man, will entitle you to what belongs to it, and give you the 
``rights of man''; Man gives you his rights!

So runs the speech of the commonalty.

The commonalty\footnote{[Or ``citizenhood.'' The word [\textit{das 
Buergertum}] means either the condition of being a citizen, or citizen-like 
principles, of the body of citizens or of the middle or business class, the 
\textit{bourgeoisie}.]} is nothing else than the thought that the State is all 
in all, the true man, and that the individual's human value consists in being 
a citizen of the State. In being a good citizen he seeks his highest honor; 
beyond that he knows nothing higher than at most the antiquated -- ``being a 
good Christian.''

The commonalty developed itself in the struggle against the privileged 
classes, by whom it was cavalierly treated as ``third estate'' and 
confounded with the \textit{canaille}. In other words, up to this time the 
State had recognized caste.\footnote{[\textit{Man hatte im Staate ``die 
ungleiche Person angesehen,''} there had been ``respect of unequal 
persons'' in the State.]} The son of a nobleman was selected for posts to 
which the most distinguished commoners aspired in vain. The civic feeling 
revolted against this. No more distinction, no giving preference to persons, 
no difference of classes! Let all be alike! No \textit{separate interest} is 
to be pursued longer, but the \textit{general interest of all}. The State is 
to be a fellowship of free and equal men, and every one is to devote himself 
to the ``welfare of the whole,'' to be dissolved in the \textit{State}, to 
make the State his end and ideal. State! State! so ran the general cry, and 
thenceforth people sought for the ``right form of State,'' the best 
constitution, and so the State in its best conception. The thought of the 
State passed into all hearts and awakened enthusiasm; to serve it, this 
mundane god, became the new divine service and worship. The properly 
\textit{political} epoch had dawned. To serve the State or the nation became 
the highest ideal, the State's interest the highest interest, State service 
(for which one does not by any means need to be an official) the highest 
honor.

So then the separate interests and personalities had been scared away, and 
sacrifice for the State had become the shibboleth. One must give up 
\textit{himself}, and live only for the State. One must act 
``disinterestedly,'' not want to benefit \textit{himself}, but the State. 
Hereby the latter has become the true person. before whom the individual 
personality vanishes; not I live, but it lives in me. Therefore, in comparison 
with the former self-seeking, this was unselfishness and 
\textit{impersonality} itself. Before this god -- State -- all egoism 
vanished, and before it all were equal; they were without any other 
distinction -- men, nothing but men.

The Revolution took fire from the inflammable material of \textit{property}. 
The government needed money. Now it must prove the proposition that it 
\textit{is absolute}, and so master of all property, sole proprietor; it must 
\textit{take} to itself \textit{its} money, which was only in the possession 
of the subjects, not their property. Instead of this, it calls States-general, 
to have this money \textit{granted} to it. The shrinking from strictly logical 
action destroyed the illusion of an \textit{absolute} government; he who must 
have something ``granted'' to him cannot be regarded as absolute. The 
subjects recognized that they were \textit{real proprietors}, and that it was 
\textit{their} money that was demanded. Those who had hitherto been subjects 
attained the consciousness that they were \textit{proprietors}. Bailly depicts 
this in a few words: ``If you cannot dispose of my property without my 
assent, how much less can you of my person, of all that concerns my mental and 
social position? All this is my property, like the piece of land that I till; 
and I have a right, an interest, to make the laws myself.'' Bailly's words 
sound, certainly, as if \textit{every one} was a proprietor now. However, 
instead of the government, instead of the prince, \textit{the -- nation} now 
became proprietor and master. From this time on the ideal is spoken of as -- 
``popular liberty'' -- ``a free people,'' etc.

As early as July 8, 1789, the declaration of the bishop of Autun and Barrere 
took away all semblance of the importance of each and every 
\textit{individual} in legislation; it showed the complete 
\textit{powerlessness} of the constituents; the \textit{majority of the 
representatives} has become \textit{master}. When on July 9 the plan for 
division of the work on the constitution is proposed, Mirabeau remarks that 
``the government has only power, no rights; only in the \textit{people} is 
the source of all \textit{right} to be found.'' On July 16 this same Mirabeau 
exclaims: ``Is not the people the source of all \textit{power?''} The 
source, therefore, of all right, and the source of all -- 
power!\footnote{[\textit{Gewalt}, a word which is also commonly used like the 
English ``violence,'' denoting especially unlawful violence.]} By the way, 
here the substance of ``right'' becomes visible; it is -- \textit{power}. 
``He who has power has right.''

The commonalty is the heir of the privileged classes. In fact, the rights of 
the barons, which were taken from them as ``usurpations,'' only passed over 
to the commonalty. For the commonalty was now called the ``nation.'' ``Into 
the hands of the nation'' all \textit{prerogatives} were given back. Thereby 
they ceased to be ``prerogatives'':\footnote{[\textit{Vorrechte}]} they 
became ``rights.''\footnote{[\textit{Rechte}]} From this time on the nation 
demands tithes, compulsory services; it has inherited the lord's court, the 
rights of vert and venison, the -- serfs. The night of August 4 was the 
death-night of privileges or ``prerogatives'' (cities, communes, boards of 
magistrates, were also privileged, furnished with prerogatives and seigniorial 
rights), and ended with the new morning of ``right,'' the ``rights of the 
State,'' the ``rights of the nation.''

The monarch in the person of the ``royal master'' had been a paltry monarch 
compared with this new monarch, the ``sovereign nation.'' This 
\textit{monarchy} was a thousand times severer, stricter, and more consistent. 
Against the new monarch there was no longer any right, any privilege at all; 
how limited the ``absolute king'' of the \textit{ancien regime} looks in 
comparison! The Revolution effected the transformation of \textit{limited 
monarchy} into \textit{absolute monarchy}. From this time on every right that 
is not conferred by this monarch is an ``assumption''; but every prerogative 
that he bestows, a ``right.'' The times demanded \textit{absolute royalty}, 
absolute monarchy; therefore down fell that so-called absolute royalty which 
had so little understood how to become absolute that it remained limited by a 
thousand little lords.

What was longed for and striven for through thousands of years -- to wit, to 
find that absolute lord beside whom no other lords and lordlings any longer 
exist to clip his power -- the \textit{bourgeoisie} has brought to pass. It 
has revealed the Lord who alone confers ``rightful titles,'' and without 
whose warrant \textit{nothing is justified}. ``So now we know that an idol is 
nothing in the world, and that there is no other god save the 
one.''\footnote{1 Corinthians 8. 4.}

Against \textit{right} one can no longer, as against a right, come forward 
with the assertion that it is ``a wrong.'' One can say now only that it is a 
piece of nonsense, an illusion. If one called it wrong, one would have to set 
up \textit{another right} in opposition to it, and measure it by this. If, on 
the contrary, one rejects right as such, right in and of itself, altogether, 
then one also rejects the concept of wrong, and dissolves the whole concept of 
right (to which the concept of wrong belongs).

What is the meaning of the doctrine that we all enjoy ``equality of political 
rights''? Only this -- that the State has no regard for my person, that to it 
I, like every other, am only a man, without having another significance that 
commands its deference. I do not command its deference as an aristocrat, a 
nobleman's son, or even as heir of an official whose office belongs to me by 
inheritance (as in the Middle Ages countships, etc., and later under absolute 
royalty, where hereditary offices occur). Now the State has an innumerable 
multitude of rights to give away, \textit{e.g.} the right to lead a 
battalion, a company, etc.; the right to lecture at a university, and so 
forth; it has them to give away because they are its own, \textit{i.e.}, State 
rights or ``political'' rights. Withal, it makes no difference to it to whom 
it gives them, if the receiver only fulfills the duties that spring from the 
delegated rights. To it we are all of us all right, and -- \textit{equal --} 
one worth no more and no less than another. It is indifferent to me who 
receives the command of the army, says the sovereign State, provided the 
grantee understands the matter properly. ``Equality of political rights'' 
has, consequently, the meaning that every one may acquire every right that the 
State has to give away, if only he fulfills the conditions annexed thereto -- 
conditions which are to be sought only in the nature of the particular right, 
not in a predilection for the person (\textit{persona grata}): the nature of 
the right to become an officer brings with it, \textit{e.g.} the necessity 
that one possess sound limbs and a suitable measure of knowledge, but it does 
not have noble birth as a condition; if, on the other hand, even the most 
deserving commoner could not reach that station, then an inequality of 
political rights would exist. Among the States of today one has carried out 
that maxim of equality more, another less.

The monarchy of estates (so I will call absolute royalty, the time of the 
kings before the revolution) kept the individual in dependence on a lot of 
little monarchies. These were fellowships (societies) like the guilds, the 
nobility, the priesthood, the burgher class, cities, communes. Everywhere the 
individual must regard himself \textit{first} as a member of this little 
society, and yield unconditional obedience to its spirit, the \textit{esprit 
de corps}, as his monarch. More, \textit{e.g.} than the individual nobleman 
himself must his family, the honor of his race, be to him. Only by means of 
his \textit{corporation}, his estate, did the individual have relation to the 
greater corporation, the State -- as in Catholicism the individual deals with 
God only through the priest. To this the third estate now, showing courage to 
negate \textit{itself as an estate}, made an end. It decided no longer to be 
and be called an \textit{estate} beside other estates, but to glorify and 
generalize itself into the \textit{``nation.''} Hereby it created a much 
more complete and absolute monarchy,' and the entire previously ruling 
\textit{principle of estates}, the principle of little monarchies inside the 
great, went down. Therefore it cannot be said that the Revolution was a 
revolution against the first two privileged estates. It was against the little 
monarchies of estates in general. But, if the estates and their despotism were 
broken (the king too, we know, was only a king of estates, not a 
citizen-king), the individuals freed from the inequality of estate were left. 
Were they now really to be without estate and ``out of gear,'' no longer 
bound by any estate, without a general bond of union? No, for the third estate 
had declared itself the nation only in order not to remain an estate 
\textit{beside} other estates, but to become the \textit{sole estate}. This 
sole \textit{estate} is the nation, the \textit{``State.''} What had the 
individual now become? A political Protestant, for he had come into immediate 
connection with his God, the State. He was no longer, as an aristocrat, in the 
monarchy of the nobility; as a mechanic, in the monarchy of the guild; but he, 
like all, recognized and acknowledged only -- \textit{one lord}, the State, as 
whose servants they all received the equal title of honor, ``citizen.''

The \textit{bourgeoisie} is the aristocracy of DESERT; its motto, ``Let 
desert wear its crowns.'' It fought against the ``lazy'' aristocracy, for 
according to it (the industrious aristocracy acquired by industry and desert) 
it is not the ``born'' who is free, nor yet I who am free either, but the 
``deserving'' man, the honest \textit{servant} (of his king; of the State; 
of the people in constitutional States). Through \textit{service} one acquires 
freedom, \textit{i.e.}, acquires ``deserts,'' even if one served -- mammon. 
One must deserve well of the State, \textit{i.e.} of the principle of the 
State, of its moral spirit. He who \textit{serves} this spirit of the State is 
a good citizen, let him live to whatever honest branch of industry he will. In 
its eyes innovators practice a ``breadless art.'' Only the ``shopkeeper'' 
is ``practical,'' and the spirit that chases after public offices is as much 
the shopkeeping spirit as is that which tries in trade to feather its nest or 
otherwise to become useful to itself and anybody else.

But, if the deserving count as the free (for what does the comfortable 
commoner, the faithful office-holder, lack of that freedom that his heart 
desires?), then the ``servants'' are the -- free. The obedient servant is 
the free man! What glaring nonsense! Yet this is the sense of the 
\textit{bourgeoisie}, and its poet, Goethe, as well as its philosopher, Hegel, 
succeeded in glorifying the dependence of the subject on the object, obedience 
to the objective world. He who only serves the cause, ``devotes himself 
entirely to it,'' has the true freedom. And among thinkers the cause was -- 
\textit{reason}, that which, like State and Church, gives -- general laws, and 
puts the individual man in irons by the \textit{thought of humanity}. It 
determines what is ``true,'' according to which one must then act. No more 
``rational'' people than the honest servants, who primarily are called good 
citizens as servants of the State.

Be rich as Croesus or poor as Job -- the State of the commonalty leaves that 
to your option; but only have a ``good disposition.'' This it demands of 
you, and counts it its most urgent task to establish this in all. Therefore it 
will keep you from ``evil promptings,'' holding the ``ill-disposed'' in 
check and silencing their inflammatory discourses under censors' 
canceling-marks or press-penalties and behind dungeon walls, and will, on the 
other hand, appoint people of ``good disposition'' as censors, and in every 
way have a \textit{moral influence} exerted on you by ``well-disposed and 
well-meaning'' people. If it has made you deaf to evil promptings, then it 
opens your ears again all the more diligently to good \textit{promptings}.

With the time of the \textit{bourgeoisie} begins that of \textit{liberalism}. 
People want to see what is ``rational,'' ``suited to the times,'' etc., 
established everywhere. The following definition of liberalism, which is 
supposed to be pronounced in its honor, characterizes it completely: 
``Liberalism is nothing else than the knowledge of reason, applied to our 
existing relations.''\footnote{\textit{``Ein und zwanzig Bogen}'', p. 12} 
Its aim is a ``rational order,'' a ``moral behavior,'' a ``limited 
freedom,'' not anarchy, lawlessness, selfhood. But, if reason rules, then the 
\textit{person} succumbs. Art has for a long time not only acknowledged the 
ugly, but considered the ugly as necessary to its existence, and takes it up 
into itself; it needs the villain. In the religious domain, too, the extremest 
liberals go so far that they want to see the most religious man regarded as a 
citizen -- \textit{i.e.}, the religious villain; they want to see no more of 
trials for heresy. But against the ``rational law'' no one is to rebel, 
otherwise he is threatened with the severest penalty. What is wanted is not 
free movement and realization of the person or of me, but of reason -- 
\textit{i.e.} a dominion of reason, a dominion. The liberals are 
\textit{zealots}, not exactly for the faith, for God, but certainly for 
\textit{reason}, their master. They brook no lack of breeding, and therefore 
no self-development and self- determination; they \textit{play the guardian} 
as effectively as the most absolute rulers.

``Political liberty,'' what are we to understand by that? Perhaps the 
individual's independence of the State and its laws? No; on the contrary, the 
individual's \textit{subjection} in the State and to the State's laws. But why 
``liberty''? Because one is no longer separated from the State by 
intermediaries, but stands in direct and immediate relation to it; because one 
is a -- citizen, not the subject of another, not even of the king as a person, 
but only in his quality as ``supreme head of the State.'' Political liberty, 
this fundamental doctrine of liberalism, is nothing but a second phase of -- 
Protestantism, and runs quite parallel with ``religious 
liberty.''\footnote{Louis Blanc says (\textit{``Histoire des dix Ans}'', I, 
p. 138) of the time of the Restoration: \textit{``Le protestantisme devint le 
fond des id\'ees et des moeurs}.''} Or would it perhaps be right to 
understand by the latter an independence of religion? Anything but that. 
Independence of intermediaries is all that it is intended to express, 
independence of mediating priests, the abolition of the ``laity,'' and so, 
direct and immediate relation to religion or to God. Only on the supposition 
that one has religion can he enjoy freedom of religion; freedom of religion 
does not mean being without religion, but inwardness of faith, unmediated 
intercourse with God. To him who is ``religiously free'' religion is an 
affair of the heart, it is to him his \textit{own affair}, it is to him a 
``sacredly serious matter.'' So, too, to the ``politically free'' man the 
State is a sacredly serious matter; it is his heart's affair, his chief 
affair, his own affair.

Political liberty means that the \textit{polis}, the State, is free; freedom 
of religion that religion is free, as freedom of conscience signifies that 
conscience is free; not, therefore, that I am free from the State, from 
religion, from conscience, or that I am \textit{rid} of them. It does not mean 
\textit{my} liberty, but the liberty of a power that rules and subjugates me; 
it means that one of my \textit{despots}, like State, religion, conscience, is 
free. State, religion, conscience, these despots, make me a slave, and 
\textit{their} liberty is my slavery. That in this they necessarily follow the 
principle, ``the end hallows the means,'' is self-evident. If the welfare of 
the State is the end, war is a hallowed means; if justice is the State's end, 
homicide is a hallowed means, and is called by its sacred name, 
``execution''; the sacred State \textit{hallows} everything that is 
serviceable to it.

``Individual liberty,'' over which civic liberalism keeps jealous watch, 
does not by any means signify a completely free self-determination, by which 
actions become altogether \textit{mine}, but only independence of 
\textit{persons}. Individually free is he who is responsible to no 
\textit{man}. Taken in this sense -- and we are not allowed to understand it 
otherwise -- not only the ruler is individually free, \textit{i.e.}, 
\textit{irresponsible} toward men (``before God,'' we know, he acknowledges 
himself responsible), but all who are ``responsible only to the law.'' This 
kind of liberty was won through the revolutionary movement of the century -- 
to wit, independence of arbitrary will, or \textit{tel est notre plaisir}. 
Hence the constitutional prince must himself be stripped of all personality, 
deprived of all individual decision, that he may not as a person, as an 
\textit{individual man}, violate the ``individual liberty'' of others. The 
\textit{personal will of the ruler} has disappeared in the constitutional 
prince; it is with a right feeling, therefore, that absolute princes resist 
this. Nevertheless these very ones profess to be in the best sense 
``Christian princes.'' For this, however, they must become a \textit{purely 
spiritual} power, as the Christian is subject only to \textit{spirit} (``God 
is spirit''). The purely spiritual power is consistently represented only by 
the constitutional prince, he who, without any personal significance, stands 
there spiritualized to the degree that he can rank as a sheer, uncanny 
``spirit,'' as an \textit{idea}. The constitutional king is the truly 
\textit{Christian} king, the genuine, consistent carrying-out of the Christian 
principle. In the constitutional monarchy individual dominion -- \textit{i.e.} 
a real ruler that \textit{wills} -- has found its end; here, therefore, 
\textit{individual liberty} prevails, independence of every individual 
dictator, of everyone who could dictate to me with a \textit{tel est notre 
plaisir}. It is the completed \textit{Christian} State-life, a spiritualized 
life.

The behavior of the commonalty is \textit{liberal} through and through. Every 
\textit{personal} invasion of another's sphere revolts the civic sense; if the 
citizen sees that one is dependent on the humor, the pleasure, the will of a 
man as individual (\textit{i.e.} as not as authorized by a ``higher 
power''), at once he brings his liberalism to the front and shrieks about 
``arbitrariness.'' In fine, the citizen asserts his freedom from what is 
called \textit{orders} (\textit{ordonnance})\textit{: ``No one has any 
business to give me -- orders!'' Orders} carries the idea that what I am to 
do is another man's will, while law does not express a personal authority of 
another. The liberty of the commonalty is liberty or independence from the 
will of another person, so-called personal or individual liberty; for being 
personally free means being only so free that no other person can dispose of 
mine, or that what I may or may not do does not depend on the personal decree 
of another. The liberty of the press, \textit{e.g.}, is such a liberty of 
liberalism, liberalism fighting only against the coercion of the censorship as 
that of personal wilfulness, but otherwise showing itself extremely inclined 
and willing to tyrannize over the press by ``press laws''; \textit{i.e.} the 
civic liberals want liberty of writing \textit{for themselves}; for, as they 
are \textit{law-abiding}, their writings will not bring them under the law. 
Only liberal matter, \textit{i.e.} only lawful matter, is to be allowed to be 
printed; otherwise the ``press laws'' threaten ``press-penalties.'' If one 
sees personal liberty assured, one does not notice at all how, if a new issue 
happens to arise, the most glaring unfreedom becomes dominant. For one is rid 
of \textit{orders} indeed, and ``no one has any business to give us 
orders,'' but one has become so much the more submissive to the -- 
\textit{law}. One is enthralled now in due legal form.

In the citizen-State there are only ``free people,'' who are 
\textit{compelled} to thousands of things (\textit{e.g.} to deference, to a 
confession of faith, etc.). But what does that amount to? Why, it is only the 
-- State, the law, not any man, that compels them!

What does the commonalty mean by inveighing against every personal order, 
\textit{i.e.} every order not founded on the ``cause,'' on ``reason''? It 
is simply fighting in the interest of the 
``cause''\footnote{[\textit{Sache}, which commonly means \textit{thing}].} 
against the dominion of ``persons''! But the mind's cause is the rational, 
good, lawful, etc.; that is the ``good cause.'' The commonalty wants an 
\textit{impersonal} ruler.

Furthermore, if the principle is this, that only the cause is to rule man -- 
to wit, the cause of morality, the cause of legality, etc., then no personal 
balking of one by the other may be authorized either (as formerly, \textit{e. 
g.} the commoner was balked of the aristocratic offices, the aristocrat of 
common mechanical trades, etc.); \textit{free competition} must exist. Only 
through the thing\footnote{[\textit{Sache}]} can one balk another (\textit{e. 
g.} the rich man balking the impecunious man by money, a thing), not as a 
person. Henceforth only one lordship, the lordship of the \textit{State}, is 
admitted; personally no one is any longer lord of another. Even at birth the 
children belong to the State, and to the parents only in the name of the 
State, which \textit{e.g.} does not allow infanticide, demands their baptism 
etc.

But all the State's children, furthermore, are of quite equal account in its 
eyes (``civic or political equality''), and they may see to it themselves 
how they get along with each other; they may \textit{compete}.

Free competition means nothing else than that every one can present himself, 
assert himself, fight, against another. Of course the feudal party set itself 
against this, as its existence depended on an absence of competition. The 
contests in the time of the Restoration in France had no other substance than 
this -- that the \textit{bourgeoisie} was struggling for free competition, and 
the feudalists were seeking to bring back the guild system.

Now, free competition has won, and against the guild system it had to win. 
(See below for the further discussion.)

If the Revolution ended in a reaction, this only showed what the Revolution 
\textit{really was}. For every effort arrives at reaction when it 
\textit{comes to discreet reflection}, and storms forward in the original 
action only so long as it is an \textit{intoxication}, an ``indiscretion.'' 
``Discretion'' will always be the cue of the reaction, because discretion 
sets limits, and liberates what was really wanted, \textit{i.e.}, the 
principle, from the initial ``unbridledness'' and ``unrestrainedness.'' 
Wild young fellows, bumptious students, who set aside all considerations, are 
\textit{really} Philistines, since with them, as with the latter, 
considerations form the substance of their conduct; only that as swaggerers 
they are mutinous against considerations and in negative relations to them, 
but as Philistines, later, they give themselves up to considerations and have 
positive relations to them. In both cases all their doing and thinking turns 
upon ``considerations,'' but the Philistine is \textit{reactionary} in 
relation to the student; he is the wild fellow come to discreet reflection, as 
the latter is the unreflecting Philistine. Daily experience confirms the truth 
of this transformation, and shows how the swaggerers turn to Philistines in 
turning gray.

So, too, the so-called reaction in Germany gives proof that it was only the 
\textit{discreet} continuation of the warlike jubilation of liberty.

The Revolution was not directed against \textit{the established}, but against 
the \textit{establishment in question}, against a \textit{particular} 
establishment. It did away with \textit{this} ruler, not with \textit{the} 
ruler -- on the contrary, the French were ruled most inexorably; it killed the 
old vicious rulers, but wanted to confer on the virtuous ones a securely 
established position, \textit{i.e.}, it simply set virtue in the place of 
vice. (Vice and virtue, again, are on their part distinguished from each other 
only as a wild young fellow from a Philistine.) Etc.

To this day the revolutionary principle has gone no farther than to assail 
only \textit{one or another} particular establishment, \textit{i.e.} be 
\textit{reformatory}. Much as may be \textit{improved}, strongly as 
``discreet progress'' may be adhered to, always there is only a \textit{new 
master} set in the old one's place, and the overturning is a -- building up. 
We are still at the distinction of the young Philistine from the old one. The 
Revolution began in \textit{bourgeois} fashion with the uprising of the third 
estate, the middle class; in \textit{bourgeois} fashion it dries away. It was 
not the \textit{individual man --} and he alone is \textit{Man} -- that became 
free, but the \textit{citizen}, the \textit{citoyen}, the \textit{political} 
man, who for that very reason is not \textit{Man} but a specimen of the human 
species, and more particularly a specimen of the species Citizen, a 
\textit{free citizen}.

In the Revolution it was not the \textit{individual} who acted so as to affect 
the world's history, but a \textit{people}; the \textit{nation}, the sovereign 
nation, wanted to effect everything. A fancied \textit{I}, an idea, \textit{e. 
g.} the nation is, appears acting; the individuals contribute themselves as 
tools of this idea, and act as ``citizens.''

The commonalty has its power, and at the same time its limits, in the 
\textit{fundamental law of the State}, in a charter, in a 
legitimate\footnote{[Or ``righteous.'' German \textit{rechtlich}].} or 
``just''\footnote{[\textit{gerecht}]} prince who himself is guided, and 
rules, according to ``rational laws,'' in short, in \textit{legality}. The 
period of the \textit{bourgeoisie} is ruled by the British spirit of legality. 
An assembly of provincial estates, \textit{e.g.} is ever recalling that its 
authorization goes only so and so far, and that it is called at all only 
through favor and can be thrown out again through disfavor. It is always 
reminding itself of its -- \textit{vocation}. It is certainly not to be denied 
that my father begot me; but, now that I am once begotten, surely his purposes 
in begetting do not concern me a bit and, whatever he may have \textit{called} 
me to, I do what I myself will. Therefore even a called assembly of estates, 
the French assembly in the beginning of the Revolution, recognized quite 
rightly that it was independent of the caller. It \textit{existed}, and would 
have been stupid if it did not avail itself of the right of existence, but 
fancied itself dependent as on a father. The called one no longer has to ask 
``what did the caller want when he created me?'' but ``what do I want after 
I have once followed the call?'' Not the caller, not the constituents, not 
the charter according to which their meeting was called out, nothing will be 
to him a sacred, inviolable power. He is \textit{authorized} for everything 
that is in his power; he will know no restrictive ``authorization,'' will 
not want to be \textit{loyal}. This, if any such thing could be expected from 
chambers at all, would give a completely \textit{egoistic} chamber, severed 
from all navel-string and without consideration. But chambers are always 
devout, and therefore one cannot be surprised if so much half-way or 
undecided, \textit{i.e.}, hypocritical, ``egoism'' parades in them.

The members of the estates are to remain within the \textit{limits} that are 
traced for them by the charter, by the king's will, etc. If they will not or 
can not do that, then they are to ``step out.'' What dutiful man could act 
otherwise, could put himself, his conviction, and his will as the 
\textit{first} thing? Who could be so immoral as to want to assert 
\textit{himself}, even if the body corporate and everything should go to ruin 
over it? People keep carefully within the limits of their 
\textit{authorization}; of course one must remain within the limits of his 
\textit{power} anyhow, because no one can do more than he can. ``My power, 
or, if it be so, powerlessness, be my sole limit, but authorizations only 
restraining -- precepts? Should I profess this all-subversive view? No, I am a 
-- law-abiding citizen!''

The commonalty professes a morality which is most closely connected with its 
essence. The first demand of this morality is to the effect that one should 
carry on a solid business, an honourable trade, lead a moral life. Immoral, to 
it, is the sharper, the, demirep, the thief, robber, and murderer, the 
gamester, the penniless man without a situation, the frivolous man. The 
doughty commoner designates the feeling against these ``immoral'' people as 
his ``deepest indignation.''

All these lack settlement, the \textit{solid} quality of business, a solid, 
seemly life, a fixed income, etc.; in short, they belong, because their 
existence does not rest on a \textit{secure basis} to the dangerous 
``individuals or isolated persons,'' to the dangerous \textit{proletariat}; 
they are ``individual bawlers'' who offer no ``guarantee'' and have 
``nothing to lose,'' and so nothing to risk. The forming of family ties, 
\textit{e.g.}, \textit{binds} a man: he who is bound furnishes security, can 
be taken hold of; not so the street-walker. The gamester stakes everything on 
the game, ruins himself and others -- no guarantee. All who appear to the 
commoner suspicious, hostile, and dangerous might be comprised under the name 
``vagabonds''; every vagabondish way of living displeases him. For there are 
intellectual vagabonds too, to whom the hereditary dwelling-place of their 
fathers seems too cramped and oppressive for them to be willing to satisfy 
themselves with the limited space any more: instead of keeping within the 
limits of a temperate style of thinking, and taking as inviolable truth what 
furnishes comfort and tranquillity to thousands, they overlap all bounds of 
the traditional and run wild with their impudent criticism and untamed mania 
for doubt, these extravagating vagabonds. They form the class of the unstable, 
restless, changeable, \textit{i.e.} of the \textit{prol\'etariat}, and, if 
they give voice to their unsettled nature, are called ``unruly fellows.''

Such a broad sense has the so-called \textit{proletariat}, or pauperism. How 
much one would err if one believed the commonalty to be desirous of doing away 
with poverty (pauperism) to the best of its ability! On the contrary, the good 
citizen helps himself with the incomparably comforting conviction that ``the 
fact is that the good things of fortune are unequally divided and will always 
remain so -- according to God's wise decree.'' The poverty which surrounds 
him in every alley does not disturb the true commoner further than that at 
most he clears his account with it by throwing an alms, or finds work and food 
for an ``honest and serviceable'' fellow. But so much the more does he feel 
his quiet enjoyment clouded by \textit{innovating} and \textit{discontented} 
poverty, by those poor who no longer behave quietly and endure, but begin to 
\textit{run wild} and become restless. Lock up the vagabond, thrust the 
breeder of unrest into the darkest dungeon! He wants to ``arouse 
dissatisfaction and incite people against existing institutions'' in the 
State -- stone him, stone him!

But from these identical discontented ones comes a reasoning somewhat as 
follows: It need not make any difference to the ``good citizens'' who 
protects them and their principles, whether an absolute king or a 
constitutional one, a republic, if only they are protected. And what is their 
principle, whose protector they always ``love''? Not that of labor; not that 
of birth either. But, that of \textit{mediocrity}, of the golden mean: a 
little birth and a little labor, \textit{i.e.}, an \textit{interest-bearing 
possession}. Possession is here the fixed, the given, inherited (birth); 
interest-drawing is the exertion about it (labor); \textit{laboring capital}, 
therefore. Only no immoderation, no ultra, no radicalism! Right of birth 
certainly, but only hereditary possessions; labor certainly, yet little or 
none at all of one's own, but labor of capital and of the -- subject laborers.

If an age is imbued with an error, some always derive advantage from the 
error, while the rest have to suffer from it. In the Middle Ages the error was 
general among Christians that the church must have all power, or the supreme 
lordship on earth; the hierarchs believed in this ``truth'' not less than 
the laymen, and both were spellbound in the like error. But by it the 
hierarchs had the \textit{advantage} of power, the laymen had to 
\textit{suffer} subjection. However, as the saying goes, ``one learns wisdom 
by suffering''; and so the laymen at last learned wisdom and no longer 
believed in the medieval ``truth.'' -- A like relation exists between the 
commonalty and the laboring class. Commoner and laborer believe in the 
``truth'' of \textit{money}; they who do not possess it believe in it no 
less than those who possess it: the laymen, therefore, as well as the priests.

``Money governs the world'' is the keynote of the civic epoch. A destitute 
aristocrat and a destitute laborer, as ``starvelings,'' amount to nothing so 
far as political consideration is concerned; birth and labor do not do it, but 
\textit{money} brings \textit{consideration}.\footnote{[\textit{das Geld gibt 
Geltung}.]} The possessors rule, but the State trains up from the destitute 
its ``servants,'' to whom, in proportion as they are to rule (govern) in its 
name, it gives money (a salary).

I receive everything from the State. Have I anything without the 
\textit{State's assent?} What I have without this it \textit{takes} from me as 
soon as it discovers the lack of a ``legal title.'' Do I not, therefore, 
have everything through its grace, its assent?

On this alone, on the \textit{legal title}, the commonalty rests. The commoner 
is what he is through the \textit{protection of the State}, through the 
State's grace. He would necessarily be afraid of losing everything if the 
State's power were broken.

But how is it with him who has nothing to lose, how with the proletarian? As 
he has nothing to lose, he does not need the protection of the State for his 
``nothing.'' He may gain, on the contrary, if that protection of the State 
is withdrawn from the \textit{prot\'eg\'e}.

Therefore the non-possessor will regard the State as a power protecting the 
possessor, which privileges the latter, but does nothing for him, the 
non-possessor, but to -- suck his blood. The State is \textit{a -- commoners' 
State}, is the estate of the commonalty. It protects man not according to his 
labor, but according to his tractableness (``loyalty'') -- to wit, according 
to whether the rights entrusted to him by the State are enjoyed and managed in 
accordance with the will, \textit{i.e.}, laws, of the State.

Under the \textit{regime} of the commonalty the laborers always fall into the 
hands of the possessors, of those who have at their disposal some bit of the 
State domains (and everything possessible in State domain, belongs to the 
State, and is only a fief of the individual), especially money and land; of 
the capitalists, therefore. The laborer cannot \textit{realize} on his labor 
to the extent of the value that it has for the consumer. ``Labor is badly 
paid!'' The capitalist has the greatest profit from it. -- Well paid, and 
more than well paid, are only the labors of those who heighten the splendor 
and \textit{dominion} of the State, the labors of high State 
\textit{servants}. The State pays well that its ``good citizens,'' the 
possessors, may be able to pay badly without danger; it secures to itself by 
good payment its servants, out of whom it forms a protecting power, a 
``police'' (to the police belong soldiers, officials of all kinds, 
\textit{e.g.} those of justice, education, etc. -- in short, the whole 
``machinery of the State'') for the ``good citizens,'' and the ``good 
citizens'' gladly pay high tax-rates to it in order to pay so much lower 
rates to their laborers.

But the class of laborers, because unprotected in what they essentially are 
(for they do not enjoy the protection of the State as laborers, but as its 
subjects they have a share in the enjoyment of the police, a so-called 
protection of the law), remains a power hostile to this State, this State of 
possessors, this ``citizen kingship.'' Its principle, labor, is not 
recognized as to its \textit{value}; it is 
exploited,\footnote{[\textit{ausgebeutet}]} a 
spoil\footnote{[\textit{Kriegsbeute}]} of the possessors, the enemy.

The laborers have the most enormous power in their hands, and, if they once 
became thoroughly conscious of it and used it, nothing would withstand them; 
they would only have to stop labor, regard the product of labor as theirs, and 
enjoy it. This is the sense of the labor disturbances which show themselves 
here and there.

The State rests on the -- \textit{slavery of labor}. If \textit{labor} becomes 
\textit{free}. the State is lost.

\subsection[\S{}2. Social Liberalism]{\centering \S{}2. Social Liberalism}

We are freeborn men, and wherever we look we see ourselves made servants of 
egoists! Are we therefore to become egoists too! Heaven forbid! We want rather 
to make egoists impossible! We want to make them all ``ragamuffins''; all of 
us must have nothing, that ``all may have.''

So say the Socialists.

Who is this person that you call ``All''? -- It is ``society''! -- But is 
it corporeal, then? -- \textit{We} are its body! -- You? Why, you are not a 
body yourselves -- you, sir, are corporeal to be sure, you too, and you, but 
you all together are only bodies, not a body. Accordingly the united society 
may indeed have bodies at its service, but no one body of its own. Like the 
``nation of the politicians'', it will turn out to be nothing but a 
``spirit,'' its body only semblance.

The freedom of man is, in political liberalism, freedom from \textit{persons}, 
from personal dominion, from the \textit{master}; the securing of each 
individual person against other persons, personal freedom.

No one has any orders to give; the law alone gives orders.

But, even if the persons have become \textit{equal}, yet their 
\textit{possessions} have not. And yet the poor man \textit{needs the rich}, 
the rich the poor, the former the rich man's money, the latter the poor man's 
labor. So no one needs another as a \textit{person}, but needs him as a 
\textit{giver}, and thus as one who has something to give, as holder or 
possessor. So what he \textit{has} makes the \textit{man}. And in 
\textit{having}, or in ``possessions,'' people are unequal.

Consequently, social liberalism concludes, \textit{no one must have}, as 
according to political liberalism \textit{no one was to give orders}; 
\textit{i.e.} as in that case the \textit{State} alone obtained the command, 
so now \textit{society} alone obtains the possessions.

For the State, protecting each one's person and property against the other, 
\textit{separates} them from one another; each one \textit{is} his special 
part and has his special part. He who is satisfied with what he is and has 
finds this state of things profitable; but he who would like to be and have 
more looks around for this ``more,'' and finds it in the power of other 
\textit{persons}. Here he comes upon a contradiction; as a person no one is 
inferior to another, and yet one person \textit{has} what another has not but 
would like to have. So, he concludes, the one person is more than the other, 
after all, for the former has what he needs, the latter has not; the former is 
a rich man, the latter a poor man.

He now asks himself further, are we to let what we rightly buried come to life 
again? Are we to let this circuitously restored inequality of persons pass? 
No; on the contrary, we must bring quite to an end what was only half 
accomplished. Our freedom from another's person still lacks the freedom from 
what the other's person can command, from what he has in his personal power -- 
in short, from ``personal property.'' Let us then do away with 
\textit{personal property}. Let no one have anything any longer, let every one 
be a -- ragamuffin. Let property be \textit{impersonal}, let it belong to -- 
\textit{society}.

Before the supreme \textit{ruler}, the sole \textit{commander}, we had all 
become equal, equal persons, \textit{i.e.}, nullities.

Before the supreme \textit{proprietor} we all become equal -- ragamuffins. For 
the present, one is still in another's estimation a ``ragamuffin,'' a 
``have-nothing''; but then this estimation ceases. We are all ragamuffins 
together, and as the aggregate of Communistic society we might call ourselves 
a ``ragamuffin crew.''

When the proletarian shall really have founded his purposed ``society'' in 
which the interval between rich and poor is to be removed, then he 
\textit{will be} a ragamuffin, for then he will feel that it amounts to 
something to be a ragamuffin, and might lift ``Ragamuffin'' to be an 
honourable form of address, just as the Revolution did with the word 
``Citizen.'' Ragamuffin is his ideal; we are all to become ragamuffins.

This is the second robbery of the ``personal'' in the interest of 
``humanity.'' Neither command nor property is left to the individual; the 
State took the former, society the latter.

Because in society the most oppressive evils make themselves felt, therefore 
the oppressed especially, and consequently the members of the lower regions of 
society, think they found the fault in society, and make it their task to 
discover the \textit{right society}. This is only the old phenomenon -- that 
one looks for the fault first in everything but \textit{himself}, and 
consequently in the State, in the self-seeking of the rich, etc., which yet 
have precisely our fault to thank for their existence.

 The reflections and conclusions of Communism look very simple. As matters lie 
at this time -- in the present situation with regard to the State, therefore 
-- some, and they the majority, are at a disadvantage compared to others, the 
minority. In this \textit{state} of things the former are in a \textit{state 
of prosperity}, the latter in \textit{state of need}. Hence the present 
\textit{state} of things, \textit{i.e.} the State itself, must be done away 
with. And what in its place? Instead of the isolated state of prosperity -- a 
\textit{general state of prosperity}, a \textit{prosperity of all}.

Through the Revolution the \textit{bourgeoisie} became omnipotent, and all 
inequality was abolished by every one's being raised or degraded to the 
dignity of a \textit{citizen} : the common man -- raised, the aristocrat -- 
degraded; the \textit{third} estate became sole estate, \textit{viz.}, namely, 
the estate of -- \textit{citizens of the State}. Now Communism responds: Our 
dignity and our essence consist not in our being all -- the \textit{equal 
children} of our mother, the State, all born with equal claim to her love and 
her protection, but in our all existing \textit{for each other}. This is our 
equality, or herein we are \textit{equal}, in that we, I as well as you and 
you and all of you, are active or ``labor'' each one for the rest; in that 
each of us is a \textit{laborer}, then. The point for us is not what we are 
\textit{for the State} (citizens), not our \textit{citizenship} therefore, but 
what we are \textit{for each other}, that each of us exists only through the 
other, who, caring for my wants, at the same time sees his own satisfied by 
me. He labors \textit{e.g.} for my clothing (tailor), I for his need of 
amusement (comedy-writer, rope-dancer), he for my food (farmer), I for his 
instruction (scientist). It is \textit{labor} that constitutes our dignity and 
our -- equality.

What advantage does citizenship bring us? Burdens! And how high is our labor 
appraised? As low as possible! But labor is our sole value all the same: that 
we are \textit{laborers} is the best thing about us, this is our significance 
in the world, and therefore it must be our consideration too and must come to 
receive \textit{consideration}. What can you meet us with? Surely nothing but 
-- \textit{labor} too. Only for labor or services do we owe you a recompense, 
not for your bare existence; not for what you are \textit{for yourselves} 
either, but only for what you are \textit{for us}. By what have you claims on 
us? Perhaps by your high birth? No, only by what you do for us that is 
desirable or useful. Be it thus then: we are willing to be worth to you only 
so much as we do for you; but you are to be held likewise by us. 
\textit{Services} determine value, -- \textit{i.e.} those services that are 
worth something to us, and consequently \textit{labors for each other, labors 
for the common good}. Let each one be in the other's eyes a \textit{laborer}. 
He who accomplishes something useful is inferior to none, or -- all laborers 
(laborers, of course, in the sense of laborers ``for the common good,'' 
\textit{i.e.}, communistic laborers) are equal. But, as the laborer is worth 
his wages,\footnote{[In German an exact quotation of Luke 10. 7.]} let the 
wages too be equal.

As long as faith sufficed for man's honor and dignity, no labor, however 
harassing, could be objected to if it only did not hinder a man in his faith. 
Now, on the contrary, when every one is to cultivate himself into man, 
condemning a man to \textit{machine-like labor} amounts to the same thing as 
slavery. If a factory worker must tire himself to death twelve hours and more, 
he is cut off from becoming man. Every labor is to have the intent that the 
man be satisfied. Therefore he must become a \textit{master} in it too, 
\textit{i.e.} be able to perform it as a totality. He who in a pin factory 
only puts on the heads, only draws the wire, works, as it were, mechanically, 
like a machine; he remains half-trained, does not become a master: his labor 
cannot \textit{satisfy} him, it can only \textit{fatigue} him. His labor is 
nothing by itself, has no object \textit{in} \textit{itself}, is nothing 
complete in itself; he labors only into another's hands, and is \textit{used} 
(exploited) by this other. For this laborer in another's service there is no 
\textit{enjoyment of a cultivated mind}, at most, crude amusements: 
\textit{culture}, you see, is barred against him. To be a good Christian one 
needs only to \textit{believe}, and that can be done under the most oppressive 
circumstances. Hence the Christian-minded take care only of the oppressed 
laborers' piety, their patience, submission, etc. Only so long as the 
downtrodden classes were \textit{Christians} could they bear all their misery: 
for Christianity does not let their murmurings and exasperation rise. Now the 
\textit{hushing} of desires is no longer enough, but their \textit{sating} is 
demanded. The \textit{bourgeoisie} has proclaimed the gospel of the 
\textit{enjoyment of the world}, of material enjoyment, and now wonders that 
this doctrine finds adherents among us poor: it has shown that not faith and 
poverty, but culture and possessions, make a man blessed; we proletarians 
understand that too.

The commonalty freed us from the orders and arbitrariness of individuals. But 
that arbitrariness was left which springs from the conjuncture of situations, 
and may be called the fortuity of circumstances; favoring \textit{fortune}. 
and those ``favored by fortune,'' still remain.

When, \textit{e.g.}, a branch of industry is ruined and thousands of laborers 
become breadless, people think reasonably enough to acknowledge that it is not 
the individual who must bear the blame, but that ``the evil lies in the 
situation.'' Let us change the situation then, but let us change it 
thoroughly, and so that its fortuity becomes powerless. and \textit{a law!} 
Let us no longer be slaves of chance! Let us create a new order that makes an 
end of \textit{fluctuations}. Let this order then be sacred!

Formerly one had to suit the \textit{lords} to come to anything; after the 
Revolution the word was ``Grasp \textit{fortune!''} Luck-hunting or 
hazard-playing, civil life was absorbed in this. Then, alongside this, the 
demand that he who has obtained something shall not frivolously stake it 
again.

Strange and yet supremely natural contradiction. Competition, in which alone 
civil or political life unrolls itself, is a game of luck through and through, 
from the speculations of the exchange down to the solicitation of offices, the 
hunt for customers, looking for work, aspiring to promotion and decorations, 
the second-hand dealer's petty haggling, etc. If one succeeds in supplanting 
and outbidding his rivals, then the ``lucky throw'' is made; for it must be 
taken as a piece of luck to begin with that the victor sees himself equipped 
with an ability (even though it has been developed by the most careful 
industry) against which the others do not know how to rise, consequently that 
-- no abler ones are found. And now those who ply their daily lives in the 
midst of these changes of fortune without seeing any harm in it are seized 
with the most virtuous indignation when their own principle appears in naked 
form and ``breeds misfortune'' as -- \textit{hazard-playing}. 
Hazard-playing, you see, is too clear, too barefaced a competition, and, like 
every decided nakedness, offends honourable modesty.

The Socialists want to put a stop to this activity of chance, and to form a 
society in which men are no longer dependent on \textit{fortune}, but free.

In the most natural way in the world this endeavor first utters itself as 
hatred of the ``unfortunate'' against the ``fortunate,'' \textit{i.e.}, of 
those for whom fortune has done little or nothing, against those for whom it 
has done everything. But properly the ill- feeling is not directed against the 
fortunate, but against \textit{fortune}, this rotten spot of the commonalty.

As the Communists first declare free activity to be man's essence, they, like 
all work-day dispositions, need a Sunday; like all material endeavors, they 
need a God, an uplifting and edification alongside their witless ``labor.''

That the Communist sees in you the man, the brother, is only the Sunday side 
of Communism. According to the work-day side he does not by any means take you 
as man simply, but as human laborer or laboring man. The first view has in it 
the liberal principle; in the second, illiberality is concealed. If you were a 
``lazy-bones,'' he would not indeed fail to recognize the man in you, but 
would endeavor to cleanse him as a ``lazy man'' from laziness and to convert 
you to the \textit{faith} that labor is man's ``destiny and calling.''

Therefore he shows a double face: with the one he takes heed that the 
spiritual man be satisfied, with the other he looks about him for means for 
the material or corporeal man. He gives man a twofold \textit{post} -- an 
office of material acquisition and one of spiritual.

The commonalty had \textit{thrown open} spiritual and material goods, and left 
it with each one to reach out for them if he liked.

Communism really procures them for each one, presses them upon him, and 
compels him to acquire them. It takes seriously the idea that, because only 
spiritual and material goods make us men, we must unquestionably acquire these 
goods in order to be man. The commonalty made acquisition free; Communism 
\textit{compels} to acquisition, and recognizes only the acquirer, him who 
practices a trade. It is not enough that the trade is free, but you must 
\textit{take it up}.

So all that is left for criticism to do is to prove that the acquisition of 
these goods does not yet by any means make us men.

With the liberal commandment that every one is to make a man of himself, or 
every one to make himself man, there was posited the necessity that every one 
must gain time for this labor of humanization, \textit{i.e.}, that it should 
become possible for every one to labor on \textit{himself}.

The commonalty thought it had brought this about if it handed over everything 
human to competition, but gave the individual a right to every human thing. 
``Each may strive after everything!''

Social liberalism finds that \textit{the} matter is not settled with the 
``may,'' because may means only ``it is forbidden to none'' but not ``it 
is made possible to every one.'' Hence it affirms that the commonalty is 
liberal only with the mouth and in words, supremely illiberal in act. It on 
its part wants to give all of us the \textit{means} to be able to labor on 
ourselves.

By the principle of labor that of fortune or competition is certainly outdone. 
But at the same time the laborer, in his consciousness that the essential 
thing in him is ``the laborer,'' holds himself aloof from egoism and 
subjects himself to the supremacy of a society of laborers, as the commoner 
clung with self-abandonment to the competition-State. The beautiful dream of a 
``social duty'' still continues to be dreamed. People think again that 
society \textit{gives} what we need, and we are \textit{under obligations} to 
it on that account, owe it everything.\footnote{Proudhon (\textit{Cr\'eation 
de l'Ordre}) cries out, p. 414, ``In industry, as in science, the publication 
of an invention is the first and \textit{most sacred of duties}!''} They are 
still at the point of wanting to \textit{serve} a ``supreme giver of all 
good.'' That society is no ego at all, which could give, bestow, or grant, 
but an instrument or means, from which we may derive benefit; that we have no 
social duties, but solely interests for the pursuance of which society must 
serve us; that we owe society no sacrifice, but, if we sacrifice anything, 
sacrifice it to ourselves -- of this the Socialists do not think, because they 
-- as liberals -- are imprisoned in the religious principle, and zealously 
aspire after -- a sacred society, \textit{e.g.} the State was hitherto.

Society, from which we have everything, is a new master, a new spook, a new 
``supreme being,'' which ``takes us into its service and allegiance!''

The more precise appreciation of political as well as social liberalism must 
wait to find its place further on. For the present we pass this over, in order 
first to summon them before the tribunal of humane or critical liberalism.

\subsection[\S{} 3. Humane Liberalism]{\centering \S{} 3. Humane Liberalism}

As liberalism is completed in self-criticizing, ``critical''\footnote{[In 
his strictures on ``criticism'' Stirner refers to a special movement known 
by that name in the early forties of the last century, of which Bruno Bauer 
was the principal exponent. After his official separation from the faculty of 
the university of Bonn on account of his views in regard to the Bible, Bruno 
Bauer in 1843 settled near Berlin and founded the \textit{Allgemeine 
Literatur-Zeitung}, in which he and his friends, at war with their 
surroundings, championed the ``absolute emancipation'' of the individual 
within the limits of ``pure humanity'' and fought as their foe ``the 
mass,'' comprehending in that term the radical aspirations of political 
liberalism and the communistic demands of the rising Socialist movement of 
that time. For a brief account of Bruno Bauer's movement of criticism, see 
John Henry Mackay, \textit{Max Stirner. Sein Leben und sein Werk}.]} 
liberalism -- in which the critic remains a liberal and does not go beyond the 
principle of liberalism, Man -- this may distinctively be named after Man and 
called the ``humane.''

The laborer is counted as the most material and egoistical man. He does 
nothing at all \textit{for humanity}, does everything for \textit{himself}, 
for his welfare.

The commonalty, because it proclaimed the freedom of \textit{Man} only as to 
his birth, had to leave him in the claws of the un-human man (the egoist) for 
the rest of life. Hence under the regime of political liberalism egoism has an 
immense field for free utilization.

The laborer will \textit{utilize} society for his \textit{egoistic} ends as 
the commoner does the State. You have only an egoistic end after all, your 
welfare, is the humane liberal's reproach to the Socialist; take up a 
\textit{purely human interest}, then I will be your companion. ``But to this 
there belongs a consciousness stronger, more comprehensive, than a 
\textit{laborer-consciousness''}. ``The laborer makes nothing, therefore he 
has nothing; but he makes nothing because his labor is always a labor that 
remains individual, calculated strictly for his own want, a labor day by 
day.''\footnote{Br. Bauer, \textit{``Lit. Ztg}.'' V, 18} In opposition to 
this one might, \textit{e.g.}, consider the fact that Gutenberg's labor did 
not remain individual, but begot innumerable children, and still lives today; 
it was calculated for the want of humanity, and was an eternal, imperishable 
labor.

The humane consciousness despises the commoner-consciousness as well as the 
laborer-consciousness: for the commoner is ``indignant'' only at vagabonds 
(at all who have ``no definite occupation'') and their ``immorality''; the 
laborer is ``disgusted'' by the idler (``lazy-bones'') and his 
``immoral,'' because parasitic and unsocial, principles. To this the humane 
liberal retorts: The unsettledness of many is only your product, Philistine! 
But that you, proletarian, demand the \textit{grind} of all, and want to make 
\textit{drudgery} general, is a part, still clinging to you, of your pack-mule 
life up to this time. Certainly you want to lighten drudgery itself by 
\textit{all} having to drudge equally hard, yet only for this reason, that all 
may gain \textit{leisure} to an equal extent. But what are they to do with 
their leisure? What does your ``society'' do, that this leisure may be 
passed \textit{humanly?} It must leave the gained leisure to egoistic 
preference again, and the very \textit{gain} that your society furthers falls 
to the egoist, as the gain of the commonalty, the \textit{masterlessness of 
man}, could not be filled with a human element by the State, and therefore was 
left to arbitrary choice.

It is assuredly necessary that man be masterless: but therefore the egoist is 
not to become master over man again either, but man over the egoist. Man must 
assuredly find leisure: but, if the egoist makes use of it, it will be lost 
for man; therefore you ought to have given leisure a human significance. But 
you laborers undertake even your labor from an egoistic impulse, because you 
want to eat, drink, live; how should you be less egoists in leisure? You labor 
only because having your time to yourselves (idling) goes well after work 
done, and what you are to while away your leisure time with is left to 
\textit{chance}.

But, if every door is to be bolted against egoism, it would be necessary to 
strive after completely ``disinterested'' action, \textit{total} 
disinterestedness. This alone is human, because only Man is disinterested, the 
egoist always interested.

\myhrule


If we let disinterestedness pass unchallenged for a while, then we ask, do you 
mean not to take an interest in anything, not to be enthusiastic for anything, 
not for liberty, humanity, etc.? ``Oh, yes, but that is not an egoistic 
interest, not \textit{interestedness}, but a human, \textit{i.e.} a -- 
\textit{theoretical} interest, to wit, an interest not for an individual or 
individuals ('all'), but for the \textit{idea}, for Man!''

And you do not notice that you too are enthusiastic only for \textit{your} 
idea, \textit{your} idea of liberty?

And, further, do you not notice that your disinterestedness is again, like 
religious disinterestedness, a heavenly interestedness? Certainly benefit to 
the individual leaves you cold, and abstractly you could cry \textit{fiat 
libertas, pereat mundus}. You do not take thought for the coming day either, 
and take no serious care for the individual's wants anyhow, not for your own 
comfort nor for that of the rest; but you make nothing of all this, because 
you are a -- dreamer.

Do you suppose the humane liberal will be so liberal as to aver that 
everything possible to man is \textit{human?} On the contrary! He does not, 
indeed, share the Philistine's moral prejudice about the strumpet, but ``that 
this woman turns her body into a money-getting 
machine''\footnote{\textit{``Lit. Ztg}.'' V, 26} makes her despicable to 
him as ``human being.'' His judgment is, the strumpet is not a human being; 
or, so far as a woman is a strumpet, so far is she unhuman, dehumanized. 
Further: The Jew, the Christian, the privileged person, the theologian, etc., 
is not a human being; so far as you are a Jew, etc., you are not a human 
being. Again the imperious postulate: Cast from you everything peculiar, 
criticize it away! Be not a Jew, not a Christian, but be a human being, 
nothing but a human being. Assert your \textit{humanity} against every 
restrictive specification; make yourself, by means of it, a human being, and 
free from those limits; make yourself a ``free man'' -- \textit{i.e.} 
recognize humanity as your all-determining \textit{essence}.

I say: You are indeed more than a Jew, more than a Christian, etc., but you 
are also more than a human being. Those are all ideas, but you are corporeal. 
Do you suppose, then, that you can ever become a ``human being as such?'' Do 
you suppose our posterity will find no prejudices and limits to clear away, 
for which our powers were not sufficient? Or do you perhaps think that in your 
fortieth or fiftieth year you have come so far that the following days have 
nothing more to dissipate in you, and that you are a human being? The men of 
the future will yet fight their way to many a liberty that we do not even 
miss. What do you need that later liberty for? If you meant to esteem yourself 
as nothing before you had become a human being, you would have to wait till 
the ``last judgment,'' till the day when man, or humanity, shall have 
attained perfection. But, as you will surely die before that, what becomes of 
your prize of victory?

Rather, therefore, invert the case, and say to yourself, \textit{I am a human 
being!} I do not need to begin by producing the human being in myself, for he 
belongs to me already, like all my qualities.

But, asks the critic, how can one be a Jew and a man at once? In the first 
place, I answer, one cannot be either a Jew or a man at all, if ``one'' and 
Jew or man are to mean the same; ``one'' always reaches beyond those 
specifications, and -- let Isaacs be ever so Jewish -- a Jew, nothing but a 
Jew, he cannot be, just because he is \textit{this} Jew. In the second place, 
as a Jew one assuredly cannot be a man, if being a man means being nothing 
special. But in the third place -- and this is the point -- I can, as a Jew, 
be entirely what I -- \textit{can} be. From Samuel or Moses, and others, you 
hardly expect that they should have raised themselves above Judaism, although 
you must say that they were not yet ``men.'' They simply were what they 
could be. Is it otherwise with the Jews of today? Because you have discovered 
the idea of humanity, does it follow from this that every Jew can become a 
convert to it? If he can, he does not fail to, and, if he fails to, he -- 
cannot. What does your demand concern him? What the \textit{call} to be a man, 
which you address to him?

\myhrule


As a universal principle, in the ``human society'' which the humane liberal 
promises, nothing ``special'' which one or another has is to find 
recognition, nothing which bears the character of ``private'' is to have 
value. In this way the circle of liberalism, which has its good principle in 
man and human liberty, its bad in the, egoist and everything private, its God 
in the former, its devil in the latter, rounds itself off completely; and, if 
the special or private person lost his value in the State (no personal 
prerogative), if in the ``laborers' or ragamuffins' society'' special 
(private) property is no longer recognized, so in ``human society'' 
everything special or private will be left out of account; and, when ``pure 
criticism'' shall have accomplished its arduous task, then it will be known 
just what we must look upon as private, and what, ``penetrated with a sense 
of our nothingness,'' we must -- let stand.

Because State and Society do not suffice for humane liberalism, it negates 
both, and at the same time retains them. So at one time the cry is that the 
task of the day is ``not a political, but a social, one,'' and then again 
the ``free State'' is promised for the future. In truth, ``human society'' 
is both -- the most general State and the most general society. Only against 
the limited State is it asserted that it makes too much stir about spiritual 
private interests (\textit{e.g.} people's religious belief), and against 
limited society that it makes too much of material private interests. Both are 
to leave private interests to private people, and, as human society, concern 
themselves solely about general human interests.

The politicians, thinking to abolish \textit{personal will}, self-will or 
arbitrariness, did not observe that through 
\textit{property}\footnote{[\textit{Eigentum}, ``owndom'']} our 
\textit{self-will}\footnote{[\textit{Eigenwille} ``own-will'']} gained a 
secure place of refuge.

The Socialists, taking away \textit{property} too, do not notice that this 
secures itself a continued existence in \textit{self-ownership}. Is it only 
money and goods, then, that are a property. or is every opinion something of 
mine, something of my own?

So every \textit{opinion} must be abolished or made impersonal. The person is 
entitled to no opinion, but, as self-will was transferred to the State, 
property to society, so opinion too must be transferred to something 
\textit{general}, ``Man,'' and thereby become a general human opinion.

If opinion persists, then I have my God (why, God exists only as ``my God,'' 
he is an opinion or my ``faith''), and consequently \textit{my} faith, my 
religion, my thoughts, my ideals. Therefore a general human faith must come 
into existence, the \textit{``fanaticism of liberty.''} For this would be a 
faith that agreed with the ``essence of man,'' and, because only ``man'' 
is reasonable (you and I might be very unreasonable!), a reasonable faith.

As self-will and property become \textit{powerless}, so must self-ownership or 
egoism in general.

In this supreme development of ``free man'' egoism, self-ownership, is 
combated on principle, and such subordinate ends as the social ``welfare'' 
of the Socialists, etc., vanish before the lofty ``idea of humanity.'' 
Everything that is not a ``general human'' entity is something separate, 
satisfies only some or one; or, if it satisfies all, it does this to them only 
as individuals, not as men, and is therefore called ``egoistic.''

To the Socialists \textit{welfare} is still the supreme aim, as free 
\textit{rivalry} was the approved thing to the political liberals; now welfare 
is free too, and we are free to achieve welfare, just as he who wanted to 
enter into rivalry (competition) was free to do so.

But to take part in the rivalry you need only to be \textit{commoners}; to 
take part in the welfare, only to be \textit{laborers}. Neither reaches the 
point of being synonymous with ``man.'' It is ``truly well'' with man only 
when he is also ``intellectually free!'' For man is mind: therefore all 
powers that are alien to him, the mind -- all superhuman, heavenly, unhuman 
powers -- must be overthrown and the name ``man'' must be above every name.

So in this end of the modern age (age of the moderns) there returns again, as 
the main point, what had been the main point at its beginning: ``intellectual 
liberty.''

To the Communist in particular the humane liberal says: If society prescribes 
to you your activity, then this is indeed free from the influence of the 
individual, \textit{i.e.} the egoist, but it still does not on that account 
need to be a \textit{purely human} activity, nor you to be a complete organ of 
humanity. What kind of activity society demands of you remains 
\textit{accidental}, you know; it might give you a place in building a temple 
or something of that sort, or, even if not that, you might yet on your own 
impulse be active for something foolish, therefore unhuman; yes, more yet, you 
really labor only to nourish yourself, in general to live, for dear life's 
sake, not for the glorification of humanity. Consequently free activity is not 
attained till you make yourself free from all stupidities, from everything 
non-human, \textit{i.e.}, egoistic (pertaining only to the individual, not to 
the Man in the individual), dissipate all untrue thoughts that obscure man or 
the idea of humanity: in short, when you are not merely unhampered in your 
activity, but the substance too of your activity is only what is human, and 
you live and work only for humanity. But this is not the case so long as the 
aim of your effort is only your \textit{welfare} and that of all; what you do 
for the society of ragamuffins is not yet anything done for ``human 
society.''

Laboring does not alone make you a man, because it is something formal and its 
object accidental; the question is who you that labor are. As far as laboring 
goes, you might do it from an egoistic (material) impulse, merely to procure 
nourishment and the like; it must be a labor furthering humanity, calculated 
for the good of humanity, serving historical (\textit{i.e.} human) evolution 
-- in short, a \textit{human} labor. This implies two things: one, that it be 
useful to humanity; next, that it be the work of a ``man.'' The first alone 
may be the case with every labor, as even the labors of nature, \textit{e.g.} 
of animals, are utilized by humanity for the furthering of science, etc.; the 
second requires that he who labors should know the human object of his labor; 
and, as he can have this consciousness only when he \textit{knows himself as 
man}, the crucial condition is -- \textit{self-consciousness.}

Unquestionably much is already attained when you cease to be a 
``fragment-laborer,''\footnote{[Referring to minute subdivision of labor, 
whereby the single workman produces, not a whole, but a part.]} yet therewith 
you only get a view of the whole of your labor, and acquire a consciousness 
about it, which is still far removed from a self-consciousness, a 
consciousness about your true ``self'' or ``essence,'' Man. The laborer 
has still remaining the desire for a ``higher consciousness,'' which, 
because the activity of labor is unable to quiet it, he satisfies in a leisure 
hour. Hence leisure stands by the side of his labor, and he sees himself 
compelled to proclaim labor and idling human in one breath, yes, to attribute 
the true elevation to the idler, the leisure-enjoyer. He labors only to get 
rid of labor; he wants to make labor free, only that he may be free from 
labor.

In fine, his work has no satisfying substance, because it is only imposed by 
society, only a stint, a task, a calling; and, conversely, his society does 
not satisfy, because it gives only work.

His labor ought to satisfy him as a man; instead of that, it satisfies 
society; society ought to treat him as a man, and it treats him as -- a 
rag-tag laborer, or a laboring ragamuffin.

Labor and society are of use to him not as he needs them as a man, but only as 
he needs them as an ``egoist.''

Such is the attitude of criticism toward labor. It points to ``mind,'' wages 
the war ``of mind with the masses,''\footnote{\textit{``Lit. Ztg}.'' V, 
34.} and pronounces communistic labor unintellectual mass-labor. Averse to 
labor as they are, the masses love to make labor easy for themselves. In 
literature, which is today furnished in mass, this aversion to labor begets 
the universally-known \textit{superficiality}, which puts from it ``the toil 
of research.''\footnote{\textit{``Lit. Ztg''} \textit{ibid}.}

Therefore humane liberalism says: You want labor; all right, we want it 
likewise, but we want it in the fullest measure. We want it, not that we may 
gain spare time, but that we may find all satisfaction in it itself. We want 
labor because it is our self-development.

But then the labor too must be adapted to that end! Man is honored only by 
human, self-conscious labor, only by the labor that has for its end no 
``egoistic'' purpose, but Man, and is Man's self-revelation; so that the 
saying should be \textit{laboro, ergo sum}, I labor, therefore I am a man. The 
humane liberal wants that labor of the \textit{mind} which \textit{works up} 
all material; he wants the mind, that leaves no thing quiet or in its existing 
condition, that acquiesces in nothing, analyzes everything, criticises anew 
every result that has been gained. This restless mind is the true laborer, it 
obliterates prejudices, shatters limits and narrownesses, and raises man above 
everything that would like to dominate over him, while the Communist labors 
only for himself, and not even freely, but from necessity, -- in short, 
represents a man condemned to hard labor.

The laborer of such a type is not ``egoistic,'' because he does not labor 
for individuals, neither for himself nor for other individuals, not for 
\textit{private} men therefore, but for humanity and its progress: he does not 
ease individual pains, does not care for individual wants, but removes limits 
within which humanity is pressed, dispels prejudices which dominate an entire 
time, vanquishes hindrances that obstruct the path of all, clears away errors 
in which men entangle themselves, discovers truths which are found through him 
for all and for all time; in short -- he lives and labors for humanity.

Now, in the first place, the discoverer of a great truth doubtless knows that 
it can be useful to the rest of men, and, as a jealous withholding furnishes 
him no enjoyment, he communicates it; but, even though he has the 
consciousness that his communication is highly valuable to the rest, yet he 
has in no wise sought and found his truth for the sake of the rest, but for 
his own sake, because he himself desired it, because darkness and fancies left 
him no rest till he had procured for himself light and enlightenment to the 
best of his powers.

He labors, therefore, for his own sake and for the satisfaction of his want. 
That along with this he was also useful to others, yes, to posterity, does not 
take from his labor the \textit{egoistic} character.

In the next place, if he did labor only on his own account, like the rest, why 
should his act be human, those of the rest unhuman, \textit{i.e.}, egoistic? 
Perhaps because this book, painting, symphony, etc., is the labor of his whole 
being, because he has done his best in it, has spread himself out wholly and 
is wholly to be known from it, while the work of a handicraftsman mirrors only 
the handicraftsman, \textit{i.e.} the skill in handicraft, not ``the man?'' 
In his poems we have the whole Schiller; in so many hundred stoves, on the 
other hand, we have before us only the stove-maker, not ``the man.''

But does this mean more than ``in the one work you see \textit{me} as 
completely as possible, in the other only my skill?'' Is it not me again that 
the act expresses? And is it not more egoistic to offer \textit{oneself} to 
the world in a work, to work out and shape \textit{oneself}, than to remain 
concealed behind one's labor? You say, to be sure, that you are revealing Man. 
But the Man that you reveal is you; you reveal only yourself, yet with this 
distinction from the handicraftsman -- that he does not understand how to 
compress himself into one labor, but, in order to be known as himself, must be 
searched out in his other relations of life, and that your want, through whose 
satisfaction that work came into being, was a -- theoretical want.

But you will reply that you reveal quite another man, a worthier, higher, 
greater, a man that is more man than that other. I will assume that you 
accomplish all that is possible to man, that you bring to pass what no other 
succeeds in. Wherein, then, does your greatness consist? Precisely in this, 
that you are more than other men (the ``masses''), more than \textit{men} 
ordinarily are, more than ``ordinary men''; precisely in your elevation 
above men. You are distinguished beyond other men not by being man, but 
because you are a ``unique''\footnote{[\textit{``einziger''}]} man. 
Doubtless you show what a man can do; but because you, a man, do it, this by 
no means shows that others, also men, are able to do as much; you have 
executed it only as a \textit{unique} man, and are unique therein.

It is not man that makes up your greatness, but you create it, because you are 
more than man, and mightier than other -- men.

It is believed that one cannot be more than man. Rather, one cannot be less!

It is believed further that whatever one attains is good for Man. In so far as 
I remain at all times a man -- or, like Schiller, a Swabian; like Kant, a 
Prussian; like Gustavus Adolfus, a near-sighted person -- I certainly become 
by my superior qualities a notable man, Swabian, Prussian, or near-sighted 
person. But the case is not much better with that than with Frederick the 
Great's cane, which became famous for Frederick's sake.

To ``Give God the glory'' corresponds the modern ``Give Man the glory.'' 
But I mean to keep it for myself.

Criticism, issuing the summons to man to be ``human,'' enunciates the 
necessary condition of sociability; for only as a man among men is one 
\textit{companionable}. Herewith it makes known its \textit{social} object, 
the establishment of ``human society.''

Among social theories criticism is indisputably the most complete, because it 
removes and deprives of value everything that \textit{separates} man from man: 
all prerogatives, down to the prerogative of faith. In it the love-principle 
of Christianity, the true social principle, comes to the purest fulfillment, 
and the last possible experiment is tried to take away exclusiveness and 
repulsion from men: a fight against egoism in its simplest and therefore 
hardest form, in the form of singleness,\footnote{[\textit{``Einzigkeit''}]} 
exclusiveness, itself.

``How can you live a truly social life so long as even one exclusiveness 
still exists between you?''

I ask conversely, How can you be truly single so long as even one connection 
still exists between you? If you are connected, you cannot leave each other; 
if a ``tie'' clasps you, you are something only \textit{with another}, and 
twelve of you make a dozen, thousands of you a people, millions of you 
humanity.

``Only when you are human can you keep company with each other as men, just 
as you can understand each other as patriots only when you are patriotic!''

All right; then I answer, Only when you are single can you have intercourse 
with each other as what you are.

It is precisely the keenest critic who is hit hardest by the curse of his 
principle. Putting from him one exclusive thing after another, shaking off 
churchliness, patriotism, etc., he undoes one tie after another and separates 
himself from the churchly man, from the patriot, till at last, when all ties 
are undone, he stands -- alone. He, of all men, must exclude all that have 
anything exclusive or private; and, when you get to the bottom, what can be 
more exclusive than the exclusive, single person himself!

Or does he perhaps think that the situation would be better if \textit{all} 
became ``man'' and gave up exclusiveness? Why, for the very reason that 
``all'' means ``every individual'' the most glaring contradiction is still 
maintained, for the ``individual'' is exclusiveness itself. If the humane 
liberal no longer concedes to the individual anything private or exclusive, 
any private thought, any private folly; if he criticises everything away from 
him before his face, since his hatred of the private is an absolute and 
fanatical hatred; if he knows no tolerance toward what is private, because 
everything private is \textit{unhuman} -- yet he cannot criticize away the 
private person himself, since the hardness of the individual person resists 
his criticism, and he must be satisfied with declaring this person a 
``private person'' and really leaving everything private to him again.

What will the society that no longer cares about anything private do? Make the 
private impossible? No, but ``subordinate it to the interests of society, 
and, \textit{e.g.}, leave it to private will to institute holidays as many as 
it chooses, if only it does not come in collision with the general 
interest.''\footnote{Br. Bauer, \textit{``Judenfrage},'' p. 66} Everything 
private is \textit{left free}; \textit{i.e.}, it has no interest for society.

``By their raising barriers against science the church and religiousness have 
declared that they are what they always were, only that this was hidden under 
another semblance when they were proclaimed to be the basis and necessary 
foundation of the State -- a matter of purely private concern. Even when they 
were connected with the State and made it Christian, they were only the proof 
that the State had not yet developed its general political idea, that it was 
only instituting private rights -- they were only the highest expression for 
the fact that the State was a private affair and had to do only with private 
affairs. When the State shall at last have the courage and strength to fulfil 
its general destiny and to be free; when, therefore, it is also able to give 
separate interests and private concerns their true position -- then religion 
and the church will be free as they have never been hitherto. As a matter of 
the most purely private concern, and a satisfaction of purely personal want, 
they will be left to themselves; and every individual, every congregation and 
ecclesiastical communion, will be able to care for the blessedness of their 
souls as they choose and as they think necessary. Every one will care for his 
soul's blessedness so far as it is to him a personal want, and will accept and 
pay as spiritual caretaker the one who seems to him to offer the best 
guarantee for the satisfaction of his want. Science is at last left entirely 
out of the game.''\footnote{Br. Bauer, \textit{``Die gute Sache der 
Freiheit},'' pp. 62-63.}

What is to happen, though? Is social life to have an end, and all affability, 
all fraternization, everything that is created by the love or society 
principle, to disappear?

As if one will not always seek the other because he \textit{needs} him; as if 
one must accommodate himself to the other when he \textit{needs} him. But the 
difference is this, that then the individual really \textit{unites} with the 
individual, while formerly they were \textit{bound together} by a tie; son and 
father are bound together before majority, after it they can come together 
independently; before it they \textit{belonged} together as members of the 
family, after it they unite as egoists; sonship and fatherhood remain, but son 
and father no longer pin themselves down to these.

The last privilege, in truth, is ``Man''; with it all are privileged or 
invested. For, as Bruno Bauer himself says, ``privilege remains even when it 
is extended to all.''\footnote{Br. Bauer, \textit{``Judenfrage},'' p. 60.}

Thus liberalism runs its course in the following transformations: ``First, 
the individual is not man, therefore his individual personality is of no 
account: no personal will, no arbitrariness, no orders or mandates!

``Second, the individual \textit{has} nothing human, therefore no mine and 
thine, or property, is valid.

``Third, as the individual neither is man nor has anything human, he shall 
not exist at all: he shall, as an egoist with his egoistic belongings, be 
annihilated by criticism to make room for Man, `Man, just discovered.'{}''

But, although the individual is not Man, Man is yet present in the individual, 
and, like every spook and everything divine, has its existence in him. Hence 
political liberalism awards to the individual everything that pertains to him 
as ``a man by birth,'' as a born man, among which there are counted liberty 
of conscience, the possession of goods, etc. -- in short, the ``rights of 
man''; Socialism grants to the individual what pertains to him as an 
\textit{active} man, as a ``laboring'' man; finally. humane liberalism gives 
the individual what he has as ``a man,'' \textit{i.e.}, everything that 
belongs to humanity. Accordingly the single 
one\footnote{[\textit{``Einzige''}]} has nothing at all, humanity 
everything; and the necessity of the ``regeneration'' preached in 
Christianity is demanded unambiguously and in the completest measure. Become a 
new creature, become ``man!''

One might even think himself reminded of the close of the Lord's Prayer. To 
Man belongs the \textit{lordship} (the ``power'' or \textit{dynamis}); 
therefore no individual may be lord, but Man is the lord of individuals; -- 
Man's is the \textit{kingdom}, \textit{i.e.} the world, consequently the 
individual is not to be proprietor, but Man, ``all,'' command the world as 
property -- to Man is due renown, \textit{glorification} or ``glory'' 
(\textit{doxa}) from all, for Man or humanity is the individual's end, for 
which he labors, thinks, lives, and for whose glorification he must become 
``man.''

Hitherto men have always striven to find out a fellowship in which their 
inequalities in other respects should become ``nonessential''; they strove 
for equalization, consequently for \textit{equality}, and wanted to come all 
under one hat, which means nothing less than that they were seeking for one 
lord, one tie, one faith (```Tis in one God we all believe''). There cannot 
be for men anything more fellowly or more equal than Man himself, and in this 
fellowship the love-craving has found its contentment: it did not rest till it 
had brought on this last equalization, leveled all inequality, laid man on the 
breast of man. But under this very fellowship decay and ruin become most 
glaring. In a more limited fellowship the Frenchman still stood against the 
German, the Christian against the Mohammedan, etc. Now, on the contrary, 
\textit{man} stands against \textit{men}, or, as men are not man, man stands 
against the un-man.

The sentence ``God has become man'' is now followed by the other, ``Man has 
become I.'' This is \textit{the human 1}. But we invert it and say: I was not 
able to find myself so long as I sought myself as Man. But, now that it 
appears that Man is aspiring to become I and to gain a corporeity in me, I 
note that, after all, everything depends on me, and Man is lost without me. 
But I do not care to give myself up to be the shrine of this most holy thing, 
and shall not ask henceforward whether I am man or un-man in what I set about; 
let this \textit{spirit} keep off my neck!

Humane liberalism goes to work radically. If you want to be or have anything 
especial even in one point, if you want to retain for yourself even one 
prerogative above others, to claim even one right that is not a ``general 
right of man,'' you are an egoist.

Very good! I do not want to have or be anything especial above others, I do 
not want to claim any prerogative against them, but -- I do not measure myself 
by others either, and do not want to have any \textit{right} whatever. I want 
to be all and have all that I can be and have. Whether others are and have 
anything \textit{similar}, what do I care? The equal, the same, they can 
neither be nor have. I cause no \textit{detriment} to them, as I cause no 
detriment to the rock by being ``ahead of it'' in having motion. If they 
could have it, they would have it.

To cause other men no \textit{detriment} is the point of the demand to possess 
no prerogative; to renounce all ``being ahead,'' the strictest theory of 
\textit{renunciation}. One is not to count himself as ``anything especial,'' 
\textit{e.g.} a Jew or a Christian. Well, I do not count myself as anything 
especial, but as unique.\footnote{[\textit{``einzig''}]} Doubtless I have 
\textit{similarity} with others; yet that holds good only for comparison or 
reflection; in fact I am incomparable, unique. My flesh is not their flesh, my 
mind is not their mind. If you bring them under the generalities ``flesh, 
mind,'' those are your \textit{thoughts}, which have nothing to do with 
\textit{my} flesh, \textit{my} mind, and can least of all issue a ``call'' 
to mine.

I do not want to recognize or respect in you any thing, neither the proprietor 
nor the ragamuffin, nor even the man, but to \textit{use you}. In salt I find 
that it makes food palatable to me, therefore I dissolve it; in the fish I 
recognize an aliment, therefore I eat it; in you I discover the gift of making 
my life agreeable, therefore I choose you as a companion. Or, in salt I study 
crystallization, in the fish animality, in you men, etc. But to me you are 
only what you are for me -- to wit, my object; and, because \textit{my} 
object, therefore my property.

In humane liberalism ragamuffinhood is completed. We must first come down to 
the most ragamuffin-like, most poverty-stricken condition if we want to arrive 
at \textit{ownness}, for we must strip off everything alien. But nothing seems 
more ragamuffin-like than naked -- Man.

It is more than ragamuffinhood, however, when I throw away Man too because I 
feel that he too is alien to me and that T can make no pretensions on that 
basis. This is no longer mere ragamuffinhood: because even the last rag has 
fallen off, here stands real nakedness, denudation of everything alien. The 
ragamuffin has stripped off ragamuffinhood itself, and therewith has ceased to 
be what he was, a ragamuffin.

I am no longer a ragamuffin, but have been one.

\myhrule


Up to this time the discord could not come to an outbreak, because properly 
there is current only a contention of modern liberals with antiquated 
liberals, a contention of those who understand ``freedom'' in a small 
measure and those who want the ``full measure'' of freedom; of the 
\textit{moderate} and \textit{measureless}, therefore. Everything turns on the 
question, \textit{how free} must \textit{man} be? That man must be free, in 
this all believe; therefore all are liberal too. But the un-man\footnote{[It 
should be remembered that to be an \textit{Unmensch}[``un-man''] one must be 
a man. The word means an inhuman or unhuman man, a man who is not man. A 
tiger, an avalanche, a drought, a cabbage, is not an un-man.]} who is 
somewhere in every individual, how is he blocked? How can it be arranged not 
to leave the un-man free at the same time with man?

Liberalism as a whole has a deadly enemy, an invincible opposite, as God has 
the devil: by the side of man stands always the un-man, the individual, the 
egoist. State, society, humanity, do not master this devil.

Humane liberalism has undertaken the task of showing the other liberals that 
they still do not want ``freedom.''

If the other liberals had before their eyes only isolated egoism and were for 
the most part blind, radical liberalism has against it egoism ``in mass,'' 
throws among the masses all who do not make the cause of freedom their own as 
it does, so that now man and un-man rigorously separated, stand over against 
each other as enemies, to wit, the ``masses'' and 
``criticism'';\footnote{\textit{``Lit. Ztg''}., V, 23; as comment, V, 12ff.} 
namely, ``free, human criticism,'' as it is called \textit{(Judenfrage}, p. 
114), in opposition to crude, that is, religious criticism.

Criticism expresses the hope that it will be victorious over all the masses 
and ``give them a general certificate of 
insolvency.''\footnote{\textit{``Lit. Ztg''}, V 15.} So it means finally to 
make itself out in the right, and to represent all contention of the 
``faint-hearted and timorous'' as an egoistic 
\textit{stubbornness},\footnote{[\textit{Rechthaberei}, literally the 
character of always insisting on making one's self out to be in the right.]} 
as pettiness, paltriness. All wrangling loses significance, and petty 
dissensions are given up, because in criticism a common enemy enters the 
field. ``You are egoists altogether, one no better than another!'' Now the 
egoists stand together against criticism. Really the egoists? No, they fight 
against criticism precisely because it accuses them of egoism; they do not 
plead guilty of egoism. Accordingly criticism and the masses stand on the same 
basis: both fight against egoism, both repudiate it for themselves and charge 
it to each other.

Criticism and the masses pursue the same goal, freedom from egoism, and 
wrangle only over which of them approaches nearest to the goal or even attains 
it.

The Jews, the Christians, the absolutists, the men of darkness and men of 
light, politicians, Communists -- all, in short -- hold the reproach of egoism 
far from them; and, as criticism brings against them this reproach in plain 
terms and in the most extended sense, all \textit{justify} themselves against 
the accusation of egoism, and combat -- egoism, the same enemy with whom 
criticism wages war.

Both, criticism and masses, are enemies of egoists, and both seek to liberate 
themselves from egoism, as well by clearing or whitewashing 
\textit{themselves} as by ascribing it to the opposite party.

The critic is the true ``spokesman of the masses'' who gives them the 
``simple concept and the phrase'' of egoism, while the spokesmen to whom the 
triumph is denied were only bunglers. He is their prince and general in the 
war against egoism for freedom; what he fights against they fight against. But 
at the same time he is their enemy too, only not the enemy before them, but 
the friendly enemy who wields the knout behind the timorous to force courage 
into them.

Hereby the opposition of criticism and the masses is reduced to the following 
contradiction: ``You are egoists!'' ``No, we are not!'' ``I will prove it 
to you!'' ``You shall have our justification!''

Let us then take both for what they give themselves out for, non-egoists, and 
what they take each other for, egoists. They are egoists and are not.

Properly criticism says: You must liberate your ego from all limitedness so 
entirely that it becomes a \textit{human} ego. I say: Liberate yourself as far 
as you can, and you have done your part; for it is not given to every one to 
break through all limits, or, more expressively: not to every one is that a 
limit which is a limit for the rest. Consequently, do not tire yourself with 
toiling at the limits of others; enough if you tear down yours. Who has ever 
succeeded in tearing down even one limit \textit{for all men?} Are not 
countless persons today, as at all times, running about with all the 
``limitations of humanity?'' He who overturns one of \textit{his} limits may 
have shown others the way and the means; the overturning of \textit{their} 
limits remains their affair. Nobody does anything else either. To demand of 
people that they become wholly men is to call on them to cast down all human 
limits. That is impossible, because \textit{Man} has no limits. I have some 
indeed, but then it is only \textit{mine} that concern me any, and only they 
can be overcome by me. A human ego I cannot become, just because I am I and 
not merely man.

Yet let us still see whether criticism has not taught us something that we can 
lay to heart! I am not free if I am not without interests, not man if I am not 
disinterested? Well, even if it makes little difference to me to be free or 
man, yet I do not want to leave unused any occasion to realize \textit{myself} 
or make myself count. Criticism offers me this occasion by the teaching that, 
if anything plants itself firmly in me, and becomes indissoluble, I become its 
prisoner and servant, \textit{i.e.} a possessed man. An interest, be it for 
what it may, has kidnapped a slave in me if I cannot get away from it, and is 
no longer my property, but I am its. Let us therefore accept criticism's 
lesson to let no part of our property become stable, and to feel comfortable 
only in -- \textit{dissolving} it.

So, if criticism says: You are man only when you are restlessly criticizing 
and dissolving! then we say: Man I am without that, and I am I likewise; 
therefore I want only to be careful to secure my property to myself; and, in 
order to secure it, I continually take it back into myself, annihilate in it 
every movement toward independence, and swallow it before it can fix itself 
and become a ``fixed idea'' or a ``mania.''

But I do that not for the sake of my ``human calling,'' but because I call 
myself to it. I do not strut about dissolving everything that it is possible 
for a man to dissolve, and, \textit{e.g.}, while not yet ten years old I do 
not criticize the nonsense of the Commandments, but I am man all the same, and 
act humanly in just this -- that I still leave them uncriticized. In short, I 
have no calling, and follow none, not even that to be a man.

Do I now reject what liberalism has won in its various exertions? Far be the 
day that anything won should be lost! Only, after ``Man'' has become free 
through liberalism, I turn my gaze back upon myself and confess to myself 
openly: What Man seems to have gained, \textit{I} alone have gained.

Man is free when ``Man is to man the supreme being.'' So it belongs to the 
completion of liberalism that every other supreme being be annulled, theology 
overturned by anthropology, God and his grace laughed down, ``atheism'' 
universal.

The egoism of property has given up the last that it had to give when even the 
``My God'' has become senseless; for God exists only when he has at heart 
the individual's welfare, as the latter seeks his welfare in him.

Political liberalism abolished the inequality of masters and servants: it made 
people masterless, anarchic. The master was now removed from the individual, 
the ``egoist,'' to become a ghost -- the law or the State. Social liberalism 
abolishes the inequality of possession, of the poor and rich, and makes people 
\textit{possessionless} or propertyless. Property is withdrawn from the 
individual and surrendered to ghostly society. Humane liberalism makes people 
\textit{godless}, atheistic. Therefore the individual's God, ``My God,'' 
must be put an end to. Now masterlessness is indeed at the same time freedom 
from service, possessionlessness at the same time freedom from care, and 
godlessness at the same time freedom from prejudice: for with the master the 
servant falls away; with possession, the care about it; with the firmly-rooted 
God, prejudice. But, since the master rises again as State, the servants 
appears again as subject; since possession becomes the property of society, 
care is begotten anew as labor; and, since God as Man becomes a prejudice, 
there arises a new faith, faith in humanity or liberty. For the individual's 
God the God of all, \textit{viz}., ``Man,'' is now exalted; ``for it is the 
highest thing in us all to be man.'' But, as nobody can become entirely what 
the idea ``man'' imports, Man remains to the individual a lofty other world, 
an unattained supreme being, a God. But at the same time this is the ``true 
God,'' because he is fully adequate to us -- to wit, our own 
\textit{``self''}; we ourselves, but separated from us and lifted above us.

\myhrule


\subsection[Postscript]{\centering Postscript}

The foregoing review of ``free human criticism'' was written by bits 
immediately after the appearance of the books in question, as was also that 
which elsewhere refers to writings of this tendency, and I did little more 
than bring together the fragments. But criticism is restlessly pressing 
forward, and thereby makes it necessary for me to come back to it once more, 
now that my book is finished, and insert this concluding note.

I have before me the latest (eighth) number of the \textit{Allgemeine 
Literatur-Zeitung} of Bruno Bauer.

There again ``the general interests of society'' stand at the top. But 
criticism has reflected, and given this ``society'' a specification by which 
it is discriminated from a form which previously had still been confused with 
it: the ``State,'' in former passages still celebrated as ``free State,'' 
is quite given up because it can in no wise fulfil the task of ``human 
society.'' Criticism only ``saw itself compelled to identify for a moment 
human and political affairs'' in 1842; but now it has found that the State, 
even as ``free State,'' is not human society, or, as it could likewise say, 
that the people is not ``man.'' We saw how it got through with theology and 
showed clearly that God sinks into dust before Man; we see it now come to a 
clearance with politics in the same way, and show that before Man peoples and 
nationalities fall: so we see how it has its explanation with Church and 
State, declaring them both unhuman, and we shall see -- for it betrays this to 
us already -- how it can also give proof that before Man the ``masses,'' 
which it even calls a ``spiritual being,'' appear worthless. And how should 
the lesser ``spiritual beings'' be able to maintain themselves before the 
supreme spirit? ``Man'' casts down the false idols.

So what the critic has in view for the present is the scrutiny of the 
``masses,'' which he will place before ``Man'' in order to combat them 
from the standpoint of Man. ``What is now the object of criticism?'' ``The 
masses, a spiritual being!'' These the critic will ``learn to know,'' and 
will find that they are in contradiction with Man; he will demonstrate that 
they are unhuman, and will succeed just as well in this demonstration as in 
the former ones, that the divine and the national, or the concerns of Church 
and of State, were the unhuman.

The masses are defined as ``the most significant product of the Revolution, 
as the deceived multitude which the illusions of political Illumination, and 
in general the entire Illumination movement of the eighteenth century, have 
given over to boundless disgruntlement.'' The Revolution satisfied some by 
its result, and left others unsatisfied; the satisfied part is the commonalty 
(\textit{bourgeoisie}, etc.), the unsatisfied is the -- masses. Does not the 
critic, so placed, himself belong to the ``masses''?

But the unsatisfied are still in great mistiness, and their discontent utters 
itself only in a ``boundless disgruntlement.'' This the likewise unsatisfied 
critic now wants to master: he cannot want and attain more than to bring that 
``spiritual being,'' the masses, out of its disgruntlement, and to 
``uplift'' those who were only disgruntled, \textit{i.e.} to give them the 
right attitude toward those results of the Revolution which are to be 
overcome; -- he can become the head of the masses, their decided spokesman. 
Therefore he wants also to ``abolish the deep chasm which parts him from the 
multitude.'' From those who want to ``uplift the lower classes of the 
people'' he is distinguished by wanting to deliver from ``disgruntlement,'' 
not merely these, but himself too.

But assuredly his consciousness does not deceive him either, when he takes the 
masses to be the ``natural opponents of theory,'' and foresees that, ``the 
more this theory shall develop itself, so much the more will it make the 
masses compact.'' For the critic cannot enlighten or satisfy the masses with 
his \textit{presupposition}, Man. If over against the commonalty they are only 
the ``lower classes of the people,'' politically insignificant masses, over 
against ``Man'' they must still more be mere ``masses,'' humanly 
insignificant -- yes, unhuman -- masses, or a multitude of un-men.

The critic clears away everything human; and, starting from the presupposition 
that the human is the true, he works against himself, denying it wherever it 
had been hitherto found. He proves only that the human is to be found nowhere 
except in his head, but the unhuman everywhere. The unhuman is the real, the 
extant on all hands, and by the proof that it is ``not human'' the critic 
only enunciates plainly the tautological sentence that it is the unhuman.

But what if the unhuman, turning its back on itself with resolute heart, 
should at the same time turn away from the disturbing critic and leave him 
standing, untouched and unstung by his remonstrance? ``You call me the 
unhuman,'' it might say to him, ``and so I really am -- for you; but I am so 
only because you bring me into opposition to the human, and I could despise 
myself only so long as I let myself be hypnotized into this opposition. I was 
contemptible because I sought my 'better self' outside me; I was the unhuman 
because I dreamed of the 'human'; I resembled the pious who hunger for their 
'true self' and always remain 'poor sinners'; I thought of myself only in 
comparison to another; enough, I was not all in all, was not -- 
\textit{unique}.\footnote{[\textit{''einzig``}]} But now I cease to appear 
to myself as the unhuman, cease to measure myself and let myself be measured 
by man, cease to recognize anything above me: consequently -- adieu, humane 
critic! I only have been the unhuman, am it now no longer, but am the unique, 
yes, to your loathing, the egoistic; yet not the egoistic as it lets itself be 
measured by the human, humane, and unselfish, but the egoistic as the -- 
unique.''

We have to pay attention to still another sentence of the same number. 
``Criticism sets up no dogmas, and wants to learn to know nothing but 
\textit{things}. ''

The critic is afraid of becoming ``dogmatic'' or setting up dogmas. Of 
course: why, thereby he would become the opposite of the critic -- the 
dogmatist; he would now become bad, as he is good as critic, or would become 
from an unselfish man an egoist, etc. ``Of all things, no dogma!'' This is 
his -- dogma. For the critic remains on one and the same ground with the 
dogmatist -- that of \textit{thoughts}. Like the latter he always starts from 
a thought, but varies in this, that he never ceases to keep the 
principle-thought in the \textit{process of thinking}, and so does not let it 
become stable. He only asserts the thought-process against the thought-faith, 
the progress of thinking against stationariness in it. From criticism no 
thought is safe, since criticism is thought or the thinking mind itself.

Therefore I repeat that the religious world -- and this is the world of 
thought -- reaches its completion in criticism, where thinking extends its 
encroachments over every thought, no one of which may ``egoistically'' 
establish itself. Where would the ``purity of criticism,'' the purity of 
thinking, be left if even one thought escaped the process of thinking? This 
explains the fact that the critic has even begun already to gibe gently here 
and there at the thought of Man, of humanity and humaneness, because he 
suspects that here a thought is approaching dogmatic fixity. But yet he cannot 
decompose this thought till he has found a -- ``higher'' in which it 
dissolves; for he moves only -- in thoughts. This higher thought might be 
enunciated as that of the movement or process of thinking itself, 
\textit{i.e.} as the thought of thinking or of criticism, for example.

Freedom of thinking has in fact become complete hereby, freedom of mind 
celebrates its triumph: for the individual, ``egoistic'' thoughts have lost 
their dogmatic truculence. There is nothing left but the -- dogma of free 
thinking or of criticism.

Against everything that belongs to the world of thought, criticism is in the 
right, \textit{i.e.}, in might: it is the victor. Criticism, and criticism 
alone, is ``up to date.'' From the standpoint of thought there is no power 
capable of being an overmatch for criticism's, and it is a pleasure to see how 
easily and sportively this dragon swallows all other serpents of thought. Each 
serpent twists, to be sure, but criticism crushes it in all its ``turns.''

I am no opponent of criticism, \textit{i.e.} I am no dogmatist, and do not 
feel myself touched by the critic's tooth with which he tears the dogmatist to 
pieces. If I were a ``dogmatist,'' I should place at the head a dogma, 
\textit{i.e.} a thought, an idea, a principle, and should complete this as a 
``systematist,'' spinning it out to a system, a structure of thought. 
Conversely, if I were a critic, \textit{viz}., an opponent of the dogmatist, I 
should carry on the fight of free thinking against the enthralling thought, I 
should defend thinking against what was thought. But I am neither the champion 
of a thought nor the champion of thinking; for ``I,'' from whom I start, am 
not a thought, nor do I consist in thinking. Against me, the unnameable, the 
realm of thoughts, thinking, and mind is shattered.

Criticism is the possessed man's fight against possession as such, against all 
possession: a fight which is founded in the consciousness that everywhere 
possession, or, as the critic calls it, a religious and theological attitude, 
is extant. He knows that people stand in a religious or believing attitude not 
only toward God, but toward other ideas as well, like right, the State, law; 
\textit{i.e.} he recognizes possession in all places. So he wants to break up 
thoughts by thinking; but I say, only thoughtlessness really saves me from 
thoughts. It is not thinking, but my thoughtlessness, or I the unthinkable, 
incomprehensible, that frees me from possession.

A jerk does me the service of the most anxious thinking, a stretching of the 
limbs shakes off the torment of thoughts, a leap upward hurls from my breast 
the nightmare of the religious world, a jubilant Hoopla throws off year-long 
burdens. But the monstrous significance of unthinking jubilation could not be 
recognized in the long night of thinking and believing.

``What clumsiness and frivolity, to want to solve the most difficult 
problems, acquit yourself of the most comprehensive tasks, by \textit{a 
breaking off}!''

But have you tasks if you do not set them to yourself? So long as you set 
them, you will not give them up, and I certainly do not care if you think, 
and, thinking, create a thousand thoughts. But you who have set the tasks, are 
you not to be able to upset them again? Must you be bound to these tasks, and 
must they become absolute tasks?

To cite only one thing, the government has been disparaged on account of its 
resorting to forcible means against thoughts, interfering against the press by 
means of the police power of the censorship, and making a personal fight out 
of a literary one. As if it were solely a matter of thoughts, and as if one's 
attitude toward thoughts must be unselfish, self-denying, and 
self-sacrificing! Do not those thoughts attack the governing parties 
themselves, and so call out egoism? And do the thinkers not set before the 
attacked ones the \textit{religious} demand to reverence the power of thought, 
of ideas? They are to succumb voluntarily and resignedly, because the divine 
power of thought, Minerva, fights on their enemies' side. Why, that would be 
an act of possession, a religious sacrifice. To be sure, the governing parties 
are themselves held fast in a religious bias, and follow the leading power of 
an idea or a faith; but they are at the same time unconfessed egoists, and 
right here, against the enemy, their pent-up egoism breaks loose: possessed in 
their faith, they are at the same time unpossessed by their opponents' faith, 
\textit{i.e.} they are egoists toward this. If one wants to make them a 
reproach, it could only be the converse -- to wit, that they are possessed by 
their ideas.

Against thoughts no egoistic power is to appear, no police power etc. So the 
believers in thinking believe. But thinking and its thoughts are not sacred to 
me, and I defend my \textit{skin} against them as against other things. That 
may be an unreasonable defense; but, if I am in duty bound to reason, then I, 
like Abraham, must sacrifice my dearest to it!

In the kingdom of thought, which, like that of faith, is the kingdom of 
heaven, every one is assuredly wrong who uses unthinking force, just as every 
one is wrong who in the kingdom of love behaves unlovingly, or, although he is 
a Christian and therefore lives in the kingdom of love, yet acts 
un-Christianly; in these kingdoms, to which he supposes himself to belong 
though he nevertheless throws off their laws, he is a ``sinner'' or 
``egoist.'' But it is only when he becomes a criminal against these kingdoms 
that he can throw off their dominion.

Here too the result is this, that the fight of the thinkers against the 
government is indeed in the right, namely, in might -- so far as it is carried 
on against the government's thoughts (the government is dumb, and does not 
succeed in making any literary rejoinder to speak of), but is, on the other 
hand, in the wrong, to wit, in impotence, so far as it does not succeed in 
bringing into the field anything but thoughts against a personal power (the 
egoistic power stops the mouths of the thinkers). The theoretical fight cannot 
complete the victory, and the sacred power of thought succumbs to the might of 
egoism. Only the egoistic fight, the fight of egoists on both sides, clears up 
everything.

This last now, to make thinking an affair of egoistic option, an affair of the 
single person,\footnote{[\textit{``des Einzigen''}]} a mere pastime or hobby 
as it were, and, to take from it the importance of ``being the last decisive 
power''; this degradation and desecration of thinking; this equalization of 
the unthinking and thoughtful ego; this clumsy but real ``equality'' -- 
criticism is not able to produce, because it itself is only the priest of 
thinking, and sees nothing beyond thinking but -- the deluge.

Criticism does indeed affirm, \textit{e.g.} that free criticism may overcome 
the State, but at the same time it defends itself against the reproach which 
is laid upon it by the State government, that it is ``self-will and 
impudence''; it thinks, then, that ``self-will and impudence'' may not 
overcome, it alone may. The truth is rather the reverse: the State can be 
really overcome only by impudent self-will.

It may now, to conclude with this, be clear that in the critic's new change of 
front he has not transformed himself, but only ``made good an oversight,'' 
``disentangled a subject,'' and is saying too much when he speaks of 
``criticism criticizing itself''; it, or rather he, has only criticized its 
``oversight'' and cleared it of its ``inconsistencies.'' If he wanted to 
criticize criticism, he would have to look and see if there was anything in 
its presupposition.

I on my part start from a presupposition in presupposing \textit{myself}; but 
my presupposition does not struggle for its perfection like ``Man struggling 
for his perfection,'' but only serves me to enjoy it and consume it. I 
consume my presupposition, and nothing else, and exist only in consuming it. 
But that presupposition is therefore not a presupposition at all: for, as I am 
the Unique, I know nothing of the duality of a presupposing and a presupposed 
ego (an ``incomplete'' and a ``complete'' ego or man); but this, that I 
consume myself, means only that I am. I do not presuppose myself, because I am 
every moment just positing or creating myself, and am I only by being not 
presupposed but posited, and, again, posited only in the moment when I posit 
myself; \textit{i.e.}, I am creator and creature in one.

If the presuppositions that have hitherto been current are to melt away in a 
full dissolution, they must not be dissolved into a higher presupposition 
again -- \textit{i.e.} a thought, or thinking itself, criticism. For that 
dissolution is to be for \textit{my} good; otherwise it would belong only in 
the series of the innumerable dissolutions which, in favor of others 
(\textit{e.g.} this very Man, God, the State, pure morality, etc.), declared 
old truths to be untruths and did away with long-fostered presuppositions.
