
\chapter[I. Ownness]{\centering I.\\
OWNNESS\footnote{[This is a literal translation of the German word 
\textit{Eigenheit}, which, with its primitive eigen, ``own,'' is used in 
this chapter in a way that the German dictionaries do not quite recognize. The 
author's conception being new, he had to make an innovation in the German 
language to express it. The translator is under the like necessity. In most 
passages ``self-ownership,'' or else ``personality,'' would translate the 
word, but there are some where the thought is so \textit{eigen}, \textit{i. 
e.}, so peculiar or so thoroughly the author's own, that no English word I can 
think of would express it. It will explain itself to one who has read Part 
First intelligently.]}}

``Does not the spirit thirst for freedom?'' -- Alas, not my spirit alone, my 
body too thirsts for it hourly! When before the odorous castle-kitchen my nose 
tells my palate of the savory dishes that are being prepared therein, it feels 
a fearful pining at its dry bread; when my eyes tell the hardened back about 
soft down on which one may lie more delightfully than on its compressed straw, 
a suppressed rage seizes it; when -- but let us not follow the pains further. 
-- And you call that a longing for freedom? What do you want to become free 
from, then? From your hardtack and your straw bed? Then throw them away! -- 
But that seems not to serve you: you want rather to have the freedom to enjoy 
delicious foods and downy beds. Are men to give you this ``freedom'' -- are 
they to permit it to you? You do not hope that from their philanthropy, 
because you know they all think like you: each is the nearest to himself! How, 
therefore, do you mean to come to the enjoyment of those foods and beds? 
Evidently not otherwise than in making them your property!

If you think it over rightly, you do not want the freedom to have all these 
fine things, for with this freedom you still do not have them; you want really 
to have them, to call them \textit{yours} and possess them as \textit{your 
property}. Of what use is a freedom to you, indeed, if it brings in nothing? 
And, if you became free from everything, you would no longer have anything; 
for freedom is empty of substance. Whoso knows not how to make use of it, for 
him it has no value, this useless permission; but how I make use of it depends 
on my personality.\footnote{[\textit{Eigenheit}]}

I have no objection to freedom, but I wish more than freedom for you: you 
should not merely \textit{be rid} of what you do not want; you should not only 
be a ``freeman,'' you should be an ``owner'' too.

Free -- from what? Oh! what is there that cannot be shaken off? The yoke of 
serfdom, of sovereignty, of aristocracy and princes, the dominion of the 
desires and passions; yes, even the dominion of one's own will, of self-will, 
for the completest self-denial is nothing but freedom -- freedom, to wit, from 
self-determination, from one's own self. And the craving for freedom as for 
something absolute, worthy of every praise, deprived us of ownness: it created 
self-denial. However, the freer I become, the more compulsion piles up before 
my eyes; and the more impotent I feel myself. The unfree son of the wilderness 
does not yet feel anything of all the limits that crowd a civilized man: he 
seems to himself freer than this latter. In the measure that I conquer freedom 
for myself I create for myself new bounds and new tasks: if I have invented 
railroads, I feel myself weak again because I cannot yet sail through the 
skies like the bird; and, if I have solved a problem whose obscurity disturbed 
my mind, at once there await me innumerable others, whose perplexities impede 
my progress, dim my free gaze, make the limits of my \textit{freedom} 
painfully sensible to me. ``Now that you have become free from sin, you have 
become servants of righteousness.''\footnote{Rom. 6, 18.} Republicans in 
their broad freedom, do they not become servants of the law? How true 
Christian hearts at all times longed to ``become free,'' how they pined to 
see themselves delivered from the ``bonds of this earth-life''! They looked 
out toward the land of freedom. (``The Jerusalem that is above is the 
freewoman; she is the mother of us all.'' Gal. 4. 26.)

Being free from anything -- means only being clear or rid. ``He is free from 
headache'' is equal to ``he is rid of it.'' ``He is free from this 
prejudice'' is equal to ``he has never conceived it'' or ``he has got rid 
of it.'' In ``less'' we complete the freedom recommended by Christianity, 
in sinless, godless, moralityless, etc.

 Freedom is the doctrine of Christianity. ``Ye, dear brethren, are called to 
freedom.''\footnote{1 Pet. 2. 16.} ``So speak and so do, as those who are to 
be judged by the law of freedom.''\footnote{James 2. 12.}

Must we then, because freedom betrays itself as a Christian ideal, give it up? 
No, nothing is to be lost, freedom no more than the rest; but it is to become 
our own, and in the form of freedom it cannot.

What a difference between freedom and ownness! One can get \textit{rid} of a 
great many things, one yet does not get rid of all; one becomes free from 
much, not from everything. Inwardly one may be free in spite of the condition 
of slavery, although, too, it is again only from all sorts of things, not from 
everything; but from the whip, the domineering temper, of the master, one does 
not as slave become \textit{free}. ``Freedom lives only in the realm of 
dreams!'' Ownness, on the contrary, is my whole being and existence, it is I 
myself. I am free from what I am \textit{rid} of, owner of what I have in my 
\textit{power} or what I \textit{control. My own} I am at all times and under 
all circumstances, if I know how to have myself and do not throw myself away 
on others. To be free is something that I cannot truly \textit{will}, because 
I cannot make it, cannot create it: I can only wish it and -- aspire toward 
it, for it remains an ideal, a spook. The fetters of reality cut the sharpest 
welts in my flesh every moment. But \textit{my own} I remain. Given up as serf 
to a master, I think only of myself and my advantage; his blows strike me 
indeed, I am not \textit{free} from them; but I endure them only for 
\textit{my benefit}, perhaps in order to deceive him and make him secure by 
the semblance of patience, or, again, not to draw worse upon myself by 
contumacy. But, as I keep my eye on myself and my selfishness, I take by the 
forelock the first good opportunity to trample the slaveholder into the dust. 
That I then become \textit{free} from him and his whip is only the consequence 
of my antecedent egoism. Here one perhaps says I was ``free'' even in the 
condition of slavery -- to wit, ``intrinsically'' or ``inwardly.'' But 
``intrinsically free'' is not ``really free,'' and ``inwardly'' is not 
``outwardly.'' I was own, on the other hand, my own, altogether, inwardly 
and outwardly. Under the dominion of a cruel master my body is not ``free'' 
from torments and lashes; but it is \textit{my} bones that moan under the 
torture, \textit{my} fibres that quiver under the blows, and \textit{I} moan 
because \textit{my} body moans. That \textit{I} sigh and shiver proves that I 
have not yet lost \textit{myself}, that I am still my own. My leg is not 
``free'' from the master's stick, but it is my leg and is inseparable. Let 
him tear it off me and look and see if he still has my leg! He retains in his 
hand nothing but the -- corpse of my leg, which is as little my leg as a dead 
dog is still a dog: a dog has a pulsating heart, a so-called dead dog has none 
and is therefore no longer a dog.

If one opines that a slave may yet be inwardly free, he says in fact only the 
most indisputable and trivial thing. For who is going to assert that any man 
is \textit{wholly} without freedom? If I am an eye-servant, can I therefore 
not be free from innumerable things, \textit{e. g.} from faith in Zeus, from 
the desire for fame, etc.? Why then should not a whipped slave also be able to 
be inwardly free from un-Christian sentiments, from hatred of his enemy, etc.? 
He then has ``Christian freedom,'' is rid of the un-Christian; but has he 
absolute freedom, freedom from everything, \textit{e. g.} from the Christian 
delusion, or from bodily pain?

In the meantime, all this seems to be said more against names than against the 
thing. But is the name indifferent, and has not a word, a shibboleth, always 
inspired and -- fooled men? Yet between freedom and ownness there lies still a 
deeper chasm than the mere difference of the words.

All the world desires freedom, all long for its reign to come. Oh, 
enchantingly beautiful dream of a blooming ``reign of freedom,'' a ``free 
human race''! -- who has not dreamed it? So men shall become free, entirely 
free, free from all constraint! From all constraint, really from all? Are they 
never to put constraint on themselves any more? ``Oh yes, that, of course; 
don't you see, that is no constraint at all?'' Well, then at any rate they -- 
are to become free from religious faith, from the strict duties of morality, 
from the inexorability of the law, from -- ``What a fearful 
misunderstanding!'' Well, \textit{what} are they to be free from then, and 
what not?

The lovely dream is dissipated; awakened, one rubs his half-opened eyes and 
stares at the prosaic questioner. ``What men are to be free from?'' -- From 
blind credulity, cries one. What's that? exclaims another, all faith is blind 
credulity; they must become free from all faith. No, no, for God's sake -- 
inveighs the first again -- do not cast all faith from you, else the power of 
brutality breaks in. We must have the republic -- a third makes himself heard, 
-- and become -- free from all commanding lords. There is no help in that, 
says a fourth: we only get a new lord then, a ``dominant majority''; let us 
rather free ourselves from this dreadful inequality. -- O, hapless equality, 
already I hear your plebeian roar again! How I had dreamed so beautifully just 
now of a paradise of \textit{freedom}, and what -- impudence and 
licentiousness now raises its wild clamor! Thus the first laments, and gets on 
his feet to grasp the sword against ``unmeasured freedom.'' Soon we no 
longer hear anything but the clashing of the swords of the disagreeing 
dreamers of freedom.

What the craving for freedom has always come to has been the desire for a 
\textit{particular} freedom, \textit{e. g.} freedom of faith; \textit{i.e.} 
the believing man wanted to be free and independent; of what? of faith 
perhaps? no! but of the inquisitors of faith. So now ``political or civil'' 
freedom. The citizen wants to become free not from citizenhood, but from 
bureaucracy, the arbitrariness of princes, etc. Prince Metternich once said he 
had ``found a way that was adapted to guide men in the path of 
\textit{genuine} freedom for all the future.'' The Count of Provence ran away 
from France precisely at the time when he was preparing the ``reign of 
freedom,'' and said: ``My imprisonment had become intolerable to me; I had 
only one passion, the desire for \textit{freedom}; I thought only of it.''

The craving for a \textit{particular} freedom always includes the purpose of a 
new \textit{dominion}, as it was with the Revolution, which indeed ``could 
give its defenders the uplifting feeling that they were fighting for 
freedom,'' but in truth only because they were after a particular freedom, 
therefore a new \textit{dominion}, the ``dominion of the law.''

Freedom you all want, you want \textit{freedom}. Why then do you haggle over a 
more or less? \textit{Freedom} can only be the whole of freedom; a piece of 
freedom is not \textit{freedom}. You despair of the possibility of obtaining 
the whole of freedom, freedom from everything -- yes, you consider it insanity 
even to wish this? -- Well, then leave off chasing after the phantom, and 
spend your pains on something better than the -- \textit{unattainable}.

``Ah, but there is nothing better than freedom!''

What have you then when you have freedom, \textit{viz}., -- for I will not 
speak here of your piecemeal bits of freedom -- complete freedom? Then you are 
rid of everything that embarrasses you, everything, and there is probably 
nothing that does not once in your life embarrass you and cause you 
inconvenience. And for whose sake, then, did you want to be rid of it? 
Doubtless \textit{for your} sake, because it is in \textit{your} way! But, if 
something were not inconvenient to you; if, on the contrary, it were quite to 
your mind (\textit{e. g.} the gently but \textit{irresistibly commanding} look 
of your loved one) -- then you would not want to be rid of it and free from 
it. Why not? For \textit{your sake} again! So you take \textit{yourselves} as 
measure and judge over all. You gladly let freedom go when unfreedom, the 
``sweet service of love,'' suits \textit{you}; and you take up your freedom 
again on occasion when it begins to suit \textit{you} better -- \textit{i. 
e.}, supposing, which is not the point here, that you are not afraid of such a 
Repeal of the Union for other (perhaps religious) reasons.

Why will you not take courage now to really make \textit{yourselves} the 
central point and the main thing altogether? Why grasp in the air at freedom, 
your dream? Are you your dream? Do not begin by inquiring of your dreams, your 
notions, your thoughts, for that is all ``hollow theory.'' Ask yourselves 
and ask after yourselves -- that is \textit{practical}, and you know you want 
very much to be ``practical.'' But there the one hearkens what his God (of 
course what he thinks of at the name God is his God) may be going to say to 
it, and another what his moral feelings, his conscience, his feeling of duty, 
may determine about it, and a third calculates what folks will think of it -- 
and, when each has thus asked his Lord God (folks are a Lord God just as good 
as, nay, even more compact than, the other-worldly and imaginary one: 
\textit{vox populi, vox dei)}, then he accommodates himself to his Lord's will 
and listens no more at all for what \textit{he himself} would like to say and 
decide.

Therefore turn to yourselves rather than to your gods or idols. Bring out from 
yourselves what is in you, bring it to the light, bring yourselves to 
revelation.

How one acts only from himself, and asks after nothing further, the Christians 
have realized in the notion ``God.'' He acts ``as it pleases him.'' And 
foolish man, who could do just so, is to act as it ``pleases God'' instead. 
-- If it is said that even God proceeds according to eternal laws, that too 
fits me, since I too cannot get out of my skin, but have my law in my whole 
nature, \textit{i.e.} in myself.

But one needs only admonish you of yourselves to bring you to despair at once. 
``What am I?'' each of you asks himself. An abyss of lawless and unregulated 
impulses, desires, wishes, passions, a chaos without light or guiding star! 
How am I to obtain a correct answer, if, without regard to God's commandments 
or to the duties which morality prescribes, without regard to the voice of 
reason, which in the course of history, after bitter experiences, has exalted 
the best and most reasonable thing into law, I simply appeal to myself? My 
passion would advise me to do the most senseless thing possible. -- Thus each 
deems himself the -- devil; for, if, so far as he is unconcerned about 
religion, etc., he only deemed himself a beast, he would easily find that the 
beast, which does follow only \textit{its} impulse (as it were, its advice), 
does not advise and impel itself to do the ``most senseless'' things, but 
takes very correct steps. But the habit of the religious way of thinking has 
biased our mind so grievously that we are -- terrified at \textit{ourselves} 
in our nakedness and naturalness; it has degraded us so that we deem ourselves 
depraved by nature, born devils. Of course it comes into your head at once 
that your calling requires you to do the ``good,'' the moral, the right. 
Now, if you ask \textit{yourselves} what is to be done, how can the right 
voice sound forth from you, the voice which points the way of the good, the 
right, the true, etc.? What concord have God and Belial?

But what would you think if one answered you by saying: ``That one is to 
listen to God, conscience, duties, laws, and so forth, is flim-flam with which 
people have stuffed your head and heart and made you crazy''? And if he asked 
you how it is that you know so surely that the voice of nature is a seducer? 
And if he even demanded of you to turn the thing about and actually to deem 
the voice of God and conscience to be the devil's work? There are such 
graceless men; how will you settle them? You cannot appeal to your parsons, 
parents, and good men, for precisely these are designated by them as your 
\textit{seducers}, as the true seducers and corrupters of youth, who busily 
sow broadcast the tares of self-contempt and reverence to God, who fill young 
hearts with mud and young heads with stupidity.

But now those people go on and ask: For whose sake do you care about God's and 
the other commandments? You surely do not suppose that this is done merely out 
of complaisance toward God? No, you are doing it -- \textit{for your sake} 
again. -- Here too, therefore, \textit{you} are the main thing, and each must 
say to himself, \textit{I} am everything to myself and I do everything 
\textit{on my} account. If it ever became clear to you that God, the 
commandments, etc., only harm you, that they reduce and ruin \textit{you}, to 
a certainty you would throw them from you just as the Christians once 
condemned Apollo or Minerva or heathen morality. They did indeed put in the 
place of these Christ and afterward Mary, as well as a Christian morality; but 
they did this for the sake of \textit{their} souls' welfare too, therefore out 
of egoism or ownness.

And it was by this egoism, this ownness, that they got \textit{rid} of the old 
world of gods and became \textit{free} from it. Ownness \textit{created} a new 
\textit{freedom}; for ownness is the creator of everything, as genius (a 
definite ownness), which is always originality, has for a long time already 
been looked upon as the creator of new productions that have a place in the 
history of the world.

If your efforts are ever to make ``freedom'' the issue, then exhaust 
freedom's demands. Who is it that is to become free? You, I, we. Free from 
what? From everything that is not you, not I, not we. I, therefore, am the 
kernel that is to be delivered from all wrappings and -- freed from all 
cramping shells. What is left when I have been freed from everything that is 
not I? Only I; nothing but I. But freedom has nothing to offer to this I 
himself. As to what is now to happen further after I have become free, freedom 
is silent -- as our governments, when the prisoner's time is up, merely let 
him go, thrusting him out into abandonment.

Now why, if freedom is striven after for love of the I after all -- why not 
choose the I himself as beginning, middle, and end? Am I not worth more than 
freedom? Is it not I that make myself free, am not I the first? Even unfree, 
even laid in a thousand fetters, I yet am; and I am not, like freedom, extant 
only in the future and in hopes, but even as the most abject of slaves I am -- 
present.

Think that over well, and decide whether you will place on your banner the 
dream of ``freedom'' or the resolution of ``egoism,'' of ``ownness.'' 
``Freedom'' awakens your \textit{rage} against everything that is not you; 
``egoism'' calls you to \textit{joy} over yourselves, to self-enjoyment; 
``freedom'' is and remains a \textit{longing} , a romantic plaint, a 
Christian hope for unearthliness and futurity; ``ownness'' is a reality, 
which \textit{of itself} removes just so much unfreedom as by barring your own 
way hinders you. What does not disturb you, you will not want to renounce; 
and, if it begins to disturb you, why, you know that ``you must obey 
\textit{yourselves} rather than men!''

Freedom teaches only: Get yourselves rid, relieve yourselves, of everything 
burdensome; it does not teach you who you yourselves are. Rid, rid! So call, 
get rid even of yourselves, ``deny yourselves.'' But ownness calls you back 
to yourselves, it says ``Come to yourself!'' Under the aegis of freedom you 
get rid of many kinds of things, but something new pinches you again: ``you 
are rid of the Evil One; evil is left.''\footnote{[See note, p. 112]} As 
\textit{own} you are \textit{really rid of everything}, and what clings to you 
\textit{you have accepted}; it is your choice and your pleasure. The 
\textit{own} man is the \textit{free-born}, the man free to begin with; the 
free man, on the contrary, is only the \textit{eleutheromaniac}, the dreamer 
and enthusiast.

The former is \textit{originally free}, because he recognizes nothing but 
himself; he does not need to free himself first, because at the start he 
rejects everything outside himself, because he prizes nothing more than 
himself, rates nothing higher, because, in short, he starts from himself and 
``comes to himself.'' Constrained by childish respect, he is nevertheless 
already working at ``freeing'' himself from this constraint. Ownness works 
in the little egoist, and procures him the desired -- freedom.

Thousands of years of civilization have obscured to you what you are, have 
made you believe you are not egoists but are \textit{called} to be idealists 
(``good men''). Shake that off! Do not seek for freedom, which does 
precisely deprive you of yourselves, in ``self-denial''; but seek for 
\textit{yourselves}, become egoists, become each of you an \textit{almighty 
ego}. Or, more clearly: Just recognize yourselves again, just recognize what 
you really are, and let go your hypocritical endeavors, your foolish mania to 
be something else than you are. Hypocritical I call them because you have yet 
remained egoists all these thousands of years, but sleeping, self-deceiving, 
crazy egoists, you \textit{Heautontimorumenoses}, you self- tormentors. Never 
yet has a religion been able to dispense with ``promises,'' whether they 
referred us to the other world or to this (``long life,'' etc.); for man is 
\textit{mercenary} and does nothing ``gratis.'' But how about that ``doing 
the good for the good's sake'' without prospect of reward? As if here too the 
pay was not contained in the satisfaction that it is to afford. Even religion, 
therefore, is founded on our egoism and -- exploits it; calculated for our 
\textit{desires}, it stifles many others for the sake of one. This then gives 
the phenomenon of \textit{cheated} egoism, where I satisfy, not myself, but 
one of my desires, \textit{e. g.} the impulse toward blessedness. Religion 
promises me the -- ``supreme good''; to gain this I no longer regard any 
other of my desires, and do not slake them. -- All your doings are 
\textit{unconfessed} , secret, covert, and concealed egoism. But because they 
are egoism that you are unwilling to confess to yourselves, that you keep 
secret from yourselves, hence not manifest and public egoism, consequently 
unconscious egoism -- therefore they are \textit{not egoism}, but thraldom, 
service, self-renunciation; you are egoists, and you are not, since you 
renounce egoism. Where you seem most to be such, you have drawn upon the word 
``egoist'' -- loathing and contempt.

I secure my freedom with regard to the world in the degree that I make the 
world my own, \textit{i.e.} ``gain it and take possession of it'' for 
myself, by whatever might, by that of persuasion, of petition, of categorical 
demand, yes, even by hypocrisy, cheating, etc.; for the means that I use for 
it are determined by what I am. If I am weak, I have only weak means, like the 
aforesaid, which yet are good enough for a considerable part of the world. 
Besides, cheating, hypocrisy, lying, look worse than they are. Who has not 
cheated the police, the law? Who has not quickly taken on an air of honourable 
loyalty before the sheriff's officer who meets him, in order to conceal an 
illegality that may have been committed, etc.? He who has not done it has 
simply let violence be done to him; he was a \textit{weakling} from -- 
conscience. I know that my freedom is diminished even by my not being able to 
carry out my will on another object, be this other something without will, 
like a rock, or something with will, like a government, an individual; I deny 
my ownness when -- in presence of another -- I give myself up, \textit{i.e.} 
give way, desist, submit; therefore by \textit{loyalty, submission}. For it is 
one thing when I give up my previous course because it does not lead to the 
goal, and therefore turn out of a wrong road; it is another when I yield 
myself a prisoner. I get around a rock that stands in my way, till I have 
powder enough to blast it; I get around the laws of a people, till I have 
gathered strength to overthrow them. Because I cannot grasp the moon, is it 
therefore to be ``sacred'' to me, an Astarte? If I only could grasp you, I 
surely would, and, if I only find a means to get up to you, you shall not 
frighten me! You inapprehensible one, you shall remain inapprehensible to me 
only till I have acquired the might for apprehension and call you my 
\textit{own}; I do not give myself up before you, but only bide my time. Even 
if for the present I put up with my inability to touch you, I yet remember it 
against you.

Vigorous men have always done so. When the ``loyal'' had exalted an 
unsubdued power to be their master and had adored it, when they had demanded 
adoration from all, then there came some such son of nature who would not 
loyally submit, and drove the adored power from its inaccessible Olympus. He 
cried his ``Stand still'' to the rolling sun, and made the earth go round; 
the loyal had to make the best of it; he laid his axe to the sacred oaks, and 
the ``loyal'' were astonished that no heavenly fire consumed him; he threw 
the pope off Peter's chair, and the ``loyal'' had no way to hinder it; he is 
tearing down the divine-right business, and the ``loyal'' croak in vain, and 
at last are silent.

My freedom becomes complete only when it is my -- \textit{might}; but by this 
I cease to be a merely free man, and become an own man. Why is the freedom of 
the peoples a ``hollow word''? Because the peoples have no might! With a 
breath of the living ego I blow peoples over, be it the breath of a Nero, a 
Chinese emperor, or a poor writer. Why is it that the G.....\footnote{[Meaning 
``German''. Written in this form because of the censorship.]} legislatures 
pine in vain for freedom, and are lectured for it by the cabinet ministers? 
Because they are not of the ``mighty''! Might is a fine thing, and useful 
for many purposes; for ``one goes further with a handful of might than with a 
bagful of right.'' You long for freedom? You fools! If you took might, 
freedom would come of itself. See, he who has might ``stands above the 
law.'' How does this prospect taste to you, you ``law-abiding'' people? But 
you have no taste!

The cry for ``freedom'' rings loudly all around. But is it felt and known 
what a donated or chartered freedom must mean? It is not recognized in the 
full amplitude of the word that all freedom is essentially -- self-liberation 
-- \textit{i.e.} that I can have only so much freedom as I procure for myself 
by my ownness. Of what use is it to sheep that no one abridges their freedom 
of speech? They stick to bleating. Give one who is inwardly a Mohammedan, a 
Jew, or a Christian, permission to speak what he likes: he will yet utter only 
narrow-minded stuff. If, on the contrary, certain others rob you of the 
freedom of speaking and hearing, they know quite rightly wherein lies their 
temporary advantage, as you would perhaps be able to say and hear something 
whereby those ``certain'' persons would lose their credit.

If they nevertheless give you freedom, they are simply knaves who give more 
than they have. For then they give you nothing of their own, but stolen wares: 
they give you your own freedom, the freedom that you must take for yourselves; 
and they \textit{give} it to you only that you may not take it and call the 
thieves and cheats to an account to boot. In their slyness they know well that 
given (chartered) freedom is no freedom, since only the freedom one 
\textit{takes} for himself, therefore the egoist's freedom, rides with full 
sails. Donated freedom strikes its sails as soon as there comes a storm -- or 
calm; it requires always a -- gentle and moderate breeze.

Here lies the difference between self-liberation and emancipation 
(manumission, setting free). Those who today ``stand in the opposition'' are 
thirsting and screaming to be ``set free.'' The princes are to ``declare 
their peoples of age,'' \textit{i. e.}, emancipate them! Behave as if you 
were of age, and you are so without any declaration of majority; if you do not 
behave accordingly, you are not worthy of it, and would never be of age even 
by a declaration of majority. When the Greeks were of age, they drove out 
their tyrants, and, when the son is of age, he makes himself independent of 
his father. If the Greeks had waited till their tyrants graciously allowed 
them their majority, they might have waited long. A sensible father throws out 
a son who will not come of age, and keeps the house to himself; it serves the 
noodle right.

The man who is set free is nothing but a freed man, a \textit{libertinus}, a 
dog dragging a piece of chain with him: he is an unfree man in the garment of 
freedom, like the ass in the lion's skin. Emancipated Jews are nothing 
bettered in themselves, but only relieved as Jews, although he who relieves 
their condition is certainly more than a churchly Christian, as the latter 
cannot do this without inconsistency. But, emancipated or not emancipated, Jew 
remains Jew; he who is not self-freed is merely an -- emancipated man. The 
Protestant State can certainly set free (emancipate) the Catholics; but, 
because they do not make themselves free, they remain simply -- Catholics.

Selfishness and unselfishness have already been spoken of. The friends of 
freedom are exasperated against selfishness because in their religious 
striving after freedom they cannot -- free themselves from that sublime thing, 
``self-renunciation.'' The liberal's anger is directed against egoism, for 
the egoist, you know, never takes trouble about a thing for the sake of the 
thing, but for his sake: the thing must serve him. It is egoistic to ascribe 
to no thing a value of its own, an ``absolute'' value, but to seek its value 
in me. One often hears that pot-boiling study which is so common counted among 
the most repulsive traits of egoistic behavior, because it manifests the most 
shameful desecration of science; but what is science for but to be consumed? 
If one does not know how to use it for anything better than to keep the pot 
boiling, then his egoism is a petty one indeed, because this egoist's power is 
a limited power; but the egoistic element in it, and the desecration of 
science, only a possessed man can blame.

Because Christianity, incapable of letting the individual count as an 
ego,\footnote{[\textit{``Einzige''}]} thought of him only as a dependent, 
and was properly nothing but a \textit{social theory --} a doctrine of living 
together, and that of man with God as well as of man with man -- therefore in 
it everything ``own'' must fall into most woeful disrepute: selfishness, 
self-will, ownness, self-love, etc. The Christian way of looking at things has 
on all sides gradually re-stamped honourable words into dishonorable; why 
should they not be brought into honor again? So \textit{Schimpf} (contumely) 
is in its old sense equivalent to jest, but for Christian seriousness pastime 
became a dishonor,\footnote{[I take \textit{Entbehrung}, ``destitution,'' to 
be a misprint for \textit{Entehrung}.]} for that seriousness cannot take a 
joke; \textit{frech} (impudent) formerly meant only bold, brave; 
\textit{Frevel} (wanton outrage) was only daring. It is well known how askance 
the word ``reason'' was looked at for a long time.

Our language has settled itself pretty well to the Christian standpoint, and 
the general consciousness is still too Christian not to shrink in terror from 
everything un-Christian as from something incomplete or evil. Therefore 
``selfishness'' is in a bad way too.

Selfishness,\footnote{[\textit{Eigennutz}, literally ``own-use.'']} in the 
Christian sense, means something like this: I look only to see whether 
anything is of use to me as a sensual man. But is sensuality then the whole of 
my ownness? Am I in my own senses when I am given up to sensuality? Do I 
follow myself, my own determination, when I follow that? I am my \textit{own} 
only when I am master of myself, instead of being mastered either by 
sensuality or by anything else (God, man, authority, law, State, Church, 
etc.); what is of use to me, this self-owned or self-appertaining one, my 
selfishness pursues.

Besides, one sees himself every moment compelled to believe in that 
constantly-blasphemed selfishness as an all-controlling power. In the session 
of February 10, 1844, Welcker argues a motion on the dependence of the judges, 
and sets forth in a detailed speech that removable, dismissable, transferable, 
and pensionable judges -- in short, such members of a court of justice as can 
by mere administrative process be damaged and endangered -- are wholly without 
reliability, yes, lose all respect and all confidence among the people. The 
whole bench, Welcker cries, is demoralized by this dependence! In blunt words 
this means nothing else than that the judges find it more to their advantage 
to give judgment as the ministers would have them than to give it as the law 
would have them. How is that to be helped? Perhaps by bringing home to the 
judges' hearts the ignominiousness of their venality, and then cherishing the 
confidence that they will repent and henceforth prize justice more highly than 
their selfishness? No, the people does not soar to this romantic confidence, 
for it feels that selfishness is mightier than any other motive. Therefore the 
same persons who have been judges hitherto may remain so, however thoroughly 
one has convinced himself that they behaved as egoists; only they must not any 
longer find their selfishness favored by the venality of justice, but must 
stand so independent of the government that by a judgment in conformity with 
the facts they do not throw into the shade their own cause, their 
``well-understood interest,'' but rather secure a comfortable combination of 
a good salary with respect among the citizens.

So Welcker and the commoners of Baden consider themselves secured only when 
they can count on selfishness. What is one to think, then, of the countless 
phrases of unselfishness with which their mouths overflow at other times?

To a cause which I am pushing selfishly I have another relation than to one 
which I am serving unselfishly. The following criterion might be cited for it; 
against the one I can \textit{sin} or commit a \textit{sin}, the other I can 
only \textit{trifle away}, push from me, deprive myself of -- \textit{i.e.} 
commit an imprudence. Free trade is looked at in both ways, being regarded 
partly as a freedom which may \textit{under certain circumstances} be granted 
or withdrawn, partly as one which is to be held \textit{sacred under all 
circumstances}.

If I am not concerned about a thing in and for itself, and do not desire it 
for its own sake, then I desire it solely as a \textit{means to an end}, for 
its usefulness; for the sake of another end, \textit{e. g.}, oysters for a 
pleasant flavor. Now will not every thing whose final end he himself is, serve 
the egoist as means? And is he to protect a thing that serves him for nothing 
-- \textit{e. g.}, the proletarian to protect the State?

Ownness includes in itself everything own, and brings to honor again what 
Christian language dishonored. But ownness has not any alien standard either, 
as it is not in any sense an \textit{idea} like freedom, morality, humanity, 
etc.: it is only a description of the -- \textit{owner}.
