\chapter[Publisher's Preface]{\centering PUBLISHER'S PREFACE}

For more than twenty years I have entertained the design of publishing an 
English translation of \textit{``Der Einzige und Sein Eigentum}.'' When I 
formed this design, the number of English-speaking persons who had ever heard 
of the book was very limited. The memory of Max Stirner had been virtually 
extinct for an entire generation. But in the last two decades there has been a 
remarkable revival of interest both in the book and in its author. It began in 
this country with a discussion in the pages of the Anarchist periodical, 
``Liberty,'' in which Stirner's thought was clearly expounded and vigorously 
championed by Dr. James L. Walker, who adopted for this discussion the 
pseudonym ``Tak Kak.'' At that time Dr. Walker was the chief editorial 
writer for the Galveston ``News.'' Some years later he became a practicing 
physician in Mexico, where he died in 1904. A series of essays which he began 
in an Anarchist periodical, ``Egoism,'' and which he lived to complete, was 
published after his death in a small volume, ``The Philosophy of Egoism.'' 
It is a very able and convincing exposition of Stirner's teachings, and almost 
the only one that exists in the English language. But the chief instrument in 
the revival of Stirnerism was and is the German poet, John Henry Mackay. Very 
early in his career he met Stirner's name in Lange's ``History of 
Materialism,'' and was moved thereby to read his book. The work made such an 
impression on him that he resolved to devote a portion of his life to the 
rediscovery and rehabilitation of the lost and forgotten genius. Through years 
of toil and correspondence and travel, and triumphing over tremendous 
obstacles, he carried his task to completion, and his biography of Stirner 
appeared in Berlin in 1898. It is a tribute to the thoroughness of Mackay's 
work that since its publication not one important fact about Stirner has been 
discovered by anybody. During his years of investigation Mackay's advertising 
for information had created a new interest in Stirner, which was enhanced by 
the sudden fame of the writings of Friedrich Nietzsche, an author whose 
intellectual kinship with Stirner has been a subject of much controversy. 
\textit{``Der Einzige,''} previously obtainable only in an expensive form, 
was included in Philipp Reclam's Universal-Bibliothek, and this cheap edition 
has enjoyed a wide and ever-increasing circulation. During the last dozen 
years the book has been translated twice into French, once into Italian, once 
into Russian, and possibly into other languages. The Scandinavian critic, 
Brandes, has written on Stirner. A large and appreciative volume, entitled 
\textit{``L'Individualisme Anarchiste: Max Stirner,''} from the pen of Prof 
Victor Basch, of the University of Rennes, has appeared in Paris. Another 
large and sympathetic volume, ``Max Stirner,'' written by Dr. Anselm Ruest, 
has been published very recently in Berlin. Dr. Paul Eltzbacher, in his work, 
\textit{``Der Anarchismus},'' gives a chapter to Stirner, making him one of 
the seven typical Anarchists, beginning with William Godwin and ending with 
Tolstoi, of whom his book treats. There is hardly a notable magazine or a 
review on the Continent that has not given at least one leading article to the 
subject of Stirner. Upon the initiative of Mackay and with the aid of other 
admirers a suitable stone has been placed above the philosopher's previously 
neglected grave, and a memorial tablet upon the house in Berlin where he died 
in 1856; and this spring another is to be placed upon the house in Bayreuth 
where he was born in 1806. As a result of these various efforts, and though 
but little has been written about Stirner in the English language, his name is 
now known at least to thousands in America and England where formerly it was 
known only to hundreds.

Therefore conditions are now more favorable for the reception of this volume 
than they were when I formed the design of publishing it, more than twenty 
years ago.

The problem of securing a reasonably good translation (for in the case of a 
work presenting difficulties so enormous it was idle to hope for an adequate 
translation) was finally solved by entrusting the task to Steven T. Byington, 
a scholar of remarkable attainments, whose specialty is philology, and who is 
also one of the ablest workers in the propaganda of Anarchism. But, for 
further security from error, it was agreed with Mr. Byington that his 
translation should have the benefit of revision by Dr. Walker, the most 
thorough American student of Stirner, and by Emma Heller Schumm and George 
Schumm, who are not only sympathetic with Stirner, but familiar with the 
history of his time, and who enjoy a knowledge of English and German that 
makes it difficult to decide which is their native tongue. It was also agreed 
that, upon any point of difference between the translator and his revisers 
which consultation might fail to solve, the publisher should decide. This 
method has been followed, and in a considerable number of instances it has 
fallen to me to make a decision. It is only fair to say, therefore, that the 
responsibility for special errors and imperfections properly rests on my 
shoulders, whereas, on the other hand, the credit for whatever general 
excellence the translation may possess belongs with the same propriety to Mr. 
Byington and his coadjutors. One thing is certain: its defects are due to no 
lack of loving care and pains. And I think I may add with confidence, while 
realizing fully how far short of perfection it necessarily falls, that it may 
safely challenge comparison with the translations that have been made into 
other languages.

In particular, I am responsible for the admittedly erroneous rendering of the 
title. ``The Ego and His Own'' is not an exact English equivalent of 
\textit{``Der Einzige und Sein Eigentum.''} But then, there is no exact 
English equivalent. Perhaps the nearest is ``The Unique One and His 
Property.'' But the unique one is not strictly the \textit{Einzige,} for 
uniqueness connotes not only singleness but an admirable singleness, while 
Stirner's \textit{Einzigkeit} is admirable in his eyes only as such, it being 
no part of the purpose of his book to distinguish a particular 
\textit{Einzigkeit} as more excellent than another. Moreover, ``The Unique 
One and His Property '' has no graces to compel our forgiveness of its slight 
inaccuracy. It is clumsy and unattractive. And the same objections may be 
urged with still greater force against all the other renderings that have been 
suggested, --- ``The Single One and His Property,'' ``The Only One and His 
Property,'' ``The Lone One and His Property,'' ``The Unit and His 
Property,'' and, last and least and worst, ``The Individual and His 
Prerogative.'' `` The Ego and His Own,'' on the other hand, if not a 
precise rendering, is at least an excellent title in itself; excellent by its 
euphony, its monosyllabic incisiveness, and its telling--- 
\textit{Einzigkeit}. Another strong argument in its favor is the emphatic 
correspondence of the phrase ``his own'' with Mr. Byington's renderings of 
the kindred words, \textit{Eigenheit} and \textit{Eigner.} Moreover, no reader 
will be led astray who bears in mind Stirner's distinction: ``I am not an ego 
along with other egos, but the sole ego; I am unique.'' And, to help the 
reader to bear this in mind, the various renderings of the word 
\textit{Einzige} that occur through the volume are often accompanied by 
foot-notes showing that, in the German, one and the same word does duty for 
all.

If the reader finds the first quarter of this book somewhat forbidding and 
obscure, he is advised nevertheless not to falter. Close attention will master 
almost every difficulty, and, if he will but give it, he will find abundant 
reward in what follows. For his guidance I may specify one defect in the 
author's style. When controverting a view opposite to his own, he seldom 
distinguishes with sufficient clearness his statement of his own view from his 
re-statement of the antagonistic view. As a result, the reader is plunged into 
deeper and deeper mystification, until something suddenly reveals the cause of 
his misunderstanding, after which he must go back and read again. I therefore 
put him on his guard. The other difficulties lie, as a rule, in the structure 
of the work. As to these I can hardly do better than translate the following 
passage from Prof. Basch's book, alluded to above: ``There is nothing more 
disconcerting than the first approach to this strange work. Stirner does not 
condescend to inform us as to the architecture of his edifice, or furnish us 
the slightest guiding thread. The apparent divisions of the book are few and 
misleading. From the first page to the last a unique thought circulates, but 
it divides itself among an infinity of vessels and arteries in each of which 
runs a blood so rich in ferments that one is tempted to describe them all. 
There is no progress in the development, and the repetitions are 
innumerable... The reader who is not deterred by this oddity, or rather 
absence, of composition gives proof of genuine intellectual courage. At first 
one seems to be confronted with a collection of essays strung together, with a 
throng of aphorisms... But, if you read this book several times; if, after 
having penetrated the intimacy of each of its parts, you then traverse it as a 
whole, ---gradually the fragments weld themselves together, and Stirner's 
thought is revealed in all its unity, in all its force, and in all its 
depth.''

A word about the dedication. Mackay's investigations have brought to light 
that Marie D\"ahnhardt had nothing whatever in common with Stirner, and so was 
unworthy of the honor conferred upon her. She was no \textit{Eigene.} I 
therefore reproduce the dedication merely in the interest of historical 
accuracy.

Happy as I am in the appearance of this book, my joy is not unmixed with 
sorrow. The cherished project was as dear to the heart of Dr. Walker as to 
mine, and I deeply grieve that he is no longer with us to share our delight in 
the fruition. Nothing, however, can rob us of the masterly introduction that 
he wrote for this volume (in 1903, or perhaps earlier), from which I will not 
longer keep the reader. This introduction, no more than the book itself, shall 
that \textit{Einzige}, Death, make his \textit{Eigentum.}

\begin{flushright}
February, 1907.\\
 \textit{B. R. T.}\end{flushright}
