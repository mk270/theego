
\chapter[II. The Owner]{\centering II.\\
THE OWNER}

I -- do I come to myself and mine through liberalism? Whom does the liberal 
look upon as his equal? Man! Be only man -- and that you are anyway -- and the 
liberal calls you his brother. He asks very little about your private opinions 
and private follies, if only he can espy ``Man'' in you.

But, as he takes little heed of what you are \textit{privatim --} nay, in a 
strict following out of his principle sets no value at all on it -- he sees in 
you only what you are \textit{generatim}. In other words, he sees in you, not 
you, but the \textit{species;} not Tom or Jim, but Man; not the real or unique 
one,\footnote{[\textit{Einzigen}] }but your essence or your concept; not the 
bodily man, but the \textit{spirit}.

As Tom you would not be his equal, because he is Jim, therefore not Tom; as 
man you are the same that he is. And, since as Tom you virtually do not exist 
at all for him (so far, to wit, as he is a liberal and not unconsciously an 
egoist), he has really made ``brother-love'' very easy for himself: he loves 
in you not Tom, of whom he knows nothing and wants to know nothing, but Man.

To see in you and me nothing further than ``men,'' that is running the 
Christian way of looking at things, according to which one is for the other 
nothing but a \textit{concept} (\textit{e.g.} a man called to salvation, 
etc.), into the ground.

Christianity properly so called gathers us under a less utterly general 
concept: there we are ``sons of God'' and ``led by the Spirit of 
God.''\footnote{Rom 8. 14.} Yet not all can boast of being God's sons, but 
``the same Spirit which witnesses to our spirit that we are sons of God 
reveals also who are the sons of the devil.''\footnote{Cf. John 3. 10. with 
Rom. 8. 16.} Consequently, to be a son of God one must not be a son of the 
devil; the sonship of God excluded certain men. To be \textit{sons of men} -- 
\textit{i.e.}, men -- on the contrary, we need nothing but to belong to the 
human \textit{species}, need only to be specimens of the same species. What I 
am as this I is no concern of yours as a good liberal, but is my 
\textit{private affair} alone; enough that we are both sons of one and the 
same mother, to wit, the human species: as ``a son of man'' I am your equal.

What am I now to you? Perhaps this \textit{bodily} I as I walk and stand? 
Anything but that. This bodily I, with its thoughts, decisions, and passions, 
is in your eyes a ``private affair'' which is no concern of yours: it is an 
``affair by itself.'' As an ``affair for you'' there exists only my 
concept, my generic concept, only \textit{the Man}, who, as he is called Tom, 
could just as well be Joe or Dick. You see in me not me, the bodily man, but 
an unreal thing, the spook, \textit{i.e.} a \textit{Man}.

In the course of the Christian centuries we declared the most various persons 
to be ``our equals,'' but each time in the measure of that \textit{spirit} 
which we expected from them -- \textit{e.g.} each one in whom the spirit of 
the need of redemption may be assumed, then later each one who has the spirit 
of integrity, finally each one who shows a human spirit and a human face. Thus 
the fundamental principle of ``equality'' varied.

 Equality being now conceived as equality of the \textit{human spirit}, there 
has certainly been discovered an equality that includes \textit{all} men; for 
who could deny that we men have a human spirit, \textit{i.e.}, no other than 
a human!

But are we on that account further on now than in the beginning of 
Christianity? Then we were to have a \textit{divine spirit}, now a 
\textit{human;} but, if the divine did not exhaust us, how should the human 
wholly express what \textit{we} are? Feuerbach \textit{e.g.} thinks, that if 
he humanizes the divine, he has found the truth. No, if God has given us pain, 
``Man'' is capable of pinching us still more torturingly. The long and the 
short of it is this: that we are men is the slightest thing about us, and has 
significance only in so far as it is one of our 
\textit{qualities},\footnote{[\textit{Eigenschaften}]} \textit{i.e.} our 
property.\footnote{[\textit{Eigentum}]} I am indeed among other things a man, 
as I am \textit{e.g.} a living being, therefore an animal, or a European, a 
Berliner, etc.; but he who chose to have regard for me only as a man, or as a 
Berliner, would pay me a regard that would be very unimportant to me. And 
wherefore? Because he would have regard only for one of my \textit{qualities}, 
not for \textit{me}.

It is just so with the \textit{spirit too}. A Christian spirit, an upright 
spirit, etc. may well be my acquired quality, my property, but I am not this 
spirit: it is mine, not I its.

Hence we have in liberalism only the continuation of the old Christian 
depreciation of the I, the bodily Tom. Instead of taking me as I am, one looks 
solely at my property, my qualities, and enters into marriage bonds with me 
only for the sake of my -- possessions; one marries, as it were, what I have, 
not what I am. The Christian takes hold of my spirit, the liberal of my 
humanity.

But, if the spirit, which is not regarded as the \textit{property} of the 
bodily ego but as the proper ego itself, is a ghost, then the Man too, who is 
not recognized as my quality but as the proper I, is nothing but a spook, a 
thought, a concept.

Therefore the liberal too revolves in the same circle as the Christian. 
Because the spirit of mankind, \textit{i.e.} Man, dwells in you, you are a 
man, as when the spirit of Christ dwells in you are a Christian; but, because 
it dwells in you only as a second ego, even though it be as your proper or 
``better'' ego, it remains otherworldly to you, and you have to strive to 
become wholly man. A striving just as fruitless as the Christian's to become 
wholly a blessed spirit!

One can now, after liberalism has proclaimed Man, declare openly that herewith 
was only completed the consistent carrying out of Christianity, and that in 
truth Christianity set itself no other task from the start than to realize 
``man,'' the ``true man.'' Hence, then, the illusion that Christianity 
ascribes an infinite value to the ego (as \textit{e.g.} in the doctrine of 
immortality, in the cure of souls, etc.) comes to light. No, it assigns this 
value to \textit{Man} alone. Only \textit{Man} is immortal, and only because I 
am Man am I too immortal. In fact, Christianity had to teach that no one is 
lost, just as liberalism too puts all on an equality as men; but that 
eternity, like this equality, applied only to the \textit{Man} in me, not to 
me. Only as the bearer and harborer of Man do I not die, as notoriously ``the 
king never dies.'' Louis dies, but the king remains; I die, but my spirit, 
Man, remains. To identify me now entirely with Man the demand has been 
invented, and stated, that I must become a ``real generic 
being.''\footnote{Karl Marx, in the \textit{``Deutsch-franz\"osische 
Jahrbucher},'' p. 197.}

The \textbf{human} \textit{religion} is only the last metamorphosis of the 
Christian religion. For liberalism is a religion because it separates my 
essence from me and sets it above me, because it exalts ``Man'' to the same 
extent as any other religion does its God or idol, because it makes what is 
mine into something otherworldly, because in general it makes out of what is 
mine, out of my qualities and my property, something alien -- to wit, an 
``essence''; in short, because it sets me beneath Man, and thereby creates 
for me a ``vocation.'' But liberalism declares itself a religion in form too 
when it demands for this supreme being, Man, a zeal of faith, ``a faith that 
some day will at last prove its fiery zeal too, a zeal that will be 
invincible.''\footnote{Br. Bauer, \textit{``Judenfrage}'', p. 61.} But, as 
liberalism is a human religion, its professor takes a \textit{tolerant} 
attitude toward the professor of any other (Catholic, Jewish, etc.), as 
Frederick the Great did toward every one who performed his duties as a 
subject, whatever fashion of becoming blest he might be inclined toward. This 
religion is now to be raised to the rank of the generally customary one, and 
separated from the others as mere ``private follies,'' toward which, 
besides, one takes a highly \textit{liberal} attitude on account of their 
unessentialness.

One may call it the \textit{State-religion}, the religion of the ``free 
State,'' not in the sense hitherto current that it is the one favored or 
privileged by the State, but as that religion which the ``free State'' not 
only has the right, but is compelled, to demand from each of those who belong 
to it, let him be \textit{privatim} a Jew, a Christian, or anything else. For 
it does the same service to the State as filial piety to the family. If the 
family is to be recognized and maintained, in its existing condition, by each 
one of those who belong to it, then to him the tie of blood must be sacred, 
and his feeling for it must be that of piety, of respect for the ties of 
blood, by which every blood-relation becomes to him a consecrated person. So 
also to every member of the State-community this community must be sacred, and 
the concept which is the highest to the State must likewise be the highest to 
him.

But what concept is the highest to the State? Doubtless that of being a really 
human society, a society in which every one who is really a man, \textit{i.e.},
\textit{not an un-man}, can obtain admission as a member. Let a State's 
tolerance go ever so far, toward an un-man and toward what is inhuman it 
ceases. And yet this ``un-man'' is a man, yet the ``inhuman'' itself is 
something human, yes, possible only to a man, not to any beast; it is, in 
fact, something ``possible to man.'' But, although every un-man is a man, 
yet the State excludes him; \textit{i.e.} it locks him up, or transforms him 
from a fellow of the State into a fellow of the prison (fellow of the lunatic 
asylum or hospital, according to Communism).

To say in blunt words what an un-man is not particularly hard: it is a man who 
does not correspond to the \textit{concept} man, as the inhuman is something 
human which is not conformed to the concept of the human. Logic calls this a 
``self-contradictory judgment.'' Would it be permissible for one to 
pronounce this judgment, that one can be a man without being a man, if he did 
not admit the hypothesis that the concept of man can be separated from the 
existence, the essence from the appearance? They say, he \textit{appears} 
indeed as a man, but \textit{is} not a man.

Men have passed this ``self-contradictory judgment'' through a long line of 
centuries! Nay, what is still more, in this long time there were only -- 
\textit{un-men}. What individual can have corresponded to his concept? 
Christianity knows only one Man, and this one -- Christ -- is at once an 
un-man again in the reverse sense, to wit, a superhuman man, a ``God.'' Only 
the -- un-man is a \textit{real} man.

Men that are not men, what should they be but \textit{ghosts?} Every real man, 
because he does not correspond to the concept ``man,'' or because he is not 
a ``generic man,'' is a spook. But do I still remain an un-man even if I 
bring Man (who towered above me and remained otherworldly to me only as my 
ideal, my task, my essence or concept) down to be my \textit{quality}, my own 
and inherent in me; so that Man is nothing else than my humanity, my human 
existence, and everything that I do is human precisely because \textit{I} do 
it, but not because it corresponds to the \textit{concept} ``man''? 
\textit{I} am really Man and the un-man in one; for I am a man and at the same 
time more than a man; \textit{i.e.} I am the ego of this my mere quality.

It had to come to this at last, that it was no longer merely demanded of us to 
be Christians, but to become men; for, though we could never really become 
even Christians, but always remained ``poor sinners'' (for the Christian was 
an unattainable ideal too), yet in this the contradictoriness did not come 
before our consciousness so, and the illusion was easier than now when of us, 
who are men act humanly (yes, cannot do otherwise than be such and act so), 
the demand is made that we are to be men, ``real men.''

Our States of today, because they still have all sorts of things sticking to 
them, left from their churchly mother, do indeed load those who belong to them 
with various obligations (\textit{e.g.} churchly religiousness) which 
properly do not a bit concern them, the States; yet on the whole they do not 
deny their significance, since they want to be looked upon as \textit{human 
societies}, in which man as man can be a member, even if he is less privileged 
than other members; most of them admit adherence of every religious sect, and 
receive people without distinction of race or nation: Jews, Turks, Moors, 
etc., can become French citizens. In the act of reception, therefore, the 
State looks only to see whether one is a \textit{man}. The Church, as a 
society of believers, could not receive every man into her bosom; the State, 
as a society of men, can. But, when the State has carried its principle clear 
through, of presupposing in its constituents nothing but that they are men 
(even the North Americans still presuppose in theirs that they have religion, 
at least the religion of integrity, of responsibility), then it has dug its 
grave. While it will fancy that those whom it possesses are without exception 
men, these have meanwhile become without exception \textit{egoists}, each of 
whom utilizes it according to his egoistic powers and ends. Against the 
egoists ``human society'' is wrecked; for they no longer have to do with 
each other as \textit{men}, but appear egoistically as an \textit{I} against a 
\textit{You} altogether different from me and in opposition to me.

If the State must count on our humanity, it is the same if one says it must 
count on our \textit{morality}. Seeing Man in each other, and acting as men 
toward each other, is called moral behavior. This is every whit the 
``spiritual love'' of Christianity. For, if I see Man in you, as in myself I 
see Man and nothing but Man, then I care for you as I would care for myself; 
for we represent, you see, nothing but the mathematical proposition: A = C and 
B = C, consequently A = B -- \textit{i.e.} I nothing but man and you nothing 
but man, consequently I and you the same. Morality is incompatible with 
egoism, because the former does not allow validity to \textit{me}, but only to 
the Man in me. But, if the State is a \textit{society of men}, not a union of 
egos each of whom has only himself before his eyes, then it cannot last 
without morality, and must insist on morality.

Therefore we two, the State and I, are enemies. I, the egoist, have not at 
heart the welfare of this ``human society,'' I sacrifice nothing to it, I 
only utilize it; but to be able to utilize it completely I transform it rather 
into my property and my creature; \textit{i.e.}, I annihilate it, and form in 
its place the \textit{Union of Egoists}.

So the State betrays its enmity to me by demanding that I be a man, which 
presupposes that I may also not be a man, but rank for it as an ``un- man''; 
it imposes being a man upon me as a \textit{duty}. Further, it desires me to 
do nothing along with which \textit{it} cannot last; so \textit{its 
permanence} is to be sacred for me. Then I am not to be an egoist, but a 
``respectable, upright,'' \textit{i.e.} moral, man. Enough: before it and 
its permanence I am to be impotent and respectful.

This State, not a present one indeed, but still in need of being first 
created, is the ideal of advancing liberalism. There is to come into existence 
a true ``society of men,'' in which every ``man'' finds room. Liberalism 
means to realize ``Man,'' \textit{i.e.} create a world for him; and this 
should be the \textit{human} world or the general (Communistic) society of 
men. It was said, ``The Church could regard only the spirit, the State is to 
regard the whole man.''\footnote{Hess, \textit{``Triarchie},'' p. 76.} But 
is not ``Man'' ``spirit''? The kernel of the State is simply ``Man,'' 
this unreality, and it itself is only a ``society of men.'' The world which 
the believer (believing spirit) creates is called Church, the world which the 
man (human or humane spirit) creates is called State. But that is not 
\textit{my} world. I never execute anything \textit{human} in the abstract, 
but always my \textit{own} things; \textit{my} human act is diverse from every 
other human act, and only by this diversity is it a real act belonging to me. 
The human in it is an abstraction, and, as such, spirit, \textit{i.e.} 
abstracted essence.

Bruno Bauer states (\textit{e.g. Judenfrage}, p. 84) that the truth of 
criticism is the final truth, and in fact the truth sought for by Christianity 
itself --to wit, ``Man.'' He says, ``The history of the Christian world is 
the history of the supreme fight for truth, for in it -- and in it only! -- 
the thing at issue is the discovery of the final or the primal truth -- man 
and freedom.''

All right, let us accept this gain, and let us take \textit{man} as the 
ultimately found result of Christian history and of the religious or ideal 
efforts of man in general. Now, who is Man? \textit{I} am! \textit{Man}, the 
end and outcome of Christianity, is, as \textit{I}, the beginning and raw 
material of the new history, a history of enjoyment after the history of 
sacrifices, a history not of man or humanity, but of -- \textit{me. Man} ranks 
as the general. Now then, I and the egoistic are the really general, since 
every one is an egoist and of paramount importance to himself. The Jewish is 
not the purely egoistic, because the Jew still devotes \textit{himself} to 
Jehovah; the Christian is not, because the Christian lives on the grace of God 
and subjects \textit{himself} to him. As Jew and as Christian alike a man 
satisfies only certain of his wants, only a certain need, not 
\textit{himself:} a half-egoism, because the egoism of a half-man, who is half 
he, half Jew, or half his own proprietor, half a slave. Therefore, too, Jew 
and Christian always half-way exclude each other; \textit{i.e.} as men they 
recognize each other, as slaves they exclude each other, because they are 
servants of two different masters. If they could be complete egoists, they 
would exclude each other \textit{wholly} and hold together so much the more 
firmly. Their ignominy is not that they exclude each other, but that this is 
done only \textit{half-way}. Bruno Bauer, on the contrary, thinks Jews and 
Christians cannot regard and treat each other as ``men'' till they give up 
the separate essence which parts them and obligates them to eternal 
separation, recognize the general essence of ``Man,'' and regard this as 
their ``true essence.''

According to his representation the defect of the Jews and the Christians 
alike lies in their wanting to be and have something ``particular'' instead 
of only being men and endeavoring after what is human -- to wit, the 
``general rights of man.'' He thinks their fundamental error consists in the 
belief that they are ``privileged,'' possess ``prerogatives''; in general, 
in the belief in \textit{prerogative}.\footnote{[\textit{Vorrecht}, literally 
``precedent right.''] }In opposition to this he holds up to them the general 
rights of man. The rights of man! --

\textit{Man is man in general}, and in so far every one who is a man. Now 
every one is to have the eternal rights of man, and, according to the opinion 
of Communism, enjoy them in the complete ``democracy,'' or, as it ought more 
correctly to be called -- anthropocracy. But it is I alone who have everything 
that I -- procure for myself; as man I have nothing. People would like to give 
every man an affluence of all good, merely because he has the title ``man.'' 
But I put the accent on \textit{me}, not on my being \textit{man}.

Man is something only as \textit{my} quality\footnote{[\textit{Eigenschaft}]} 
(property\footnote{[\textit{Eigentum}]}), like masculinity or femininity. The 
ancients found the ideal in one's being \textit{male} in the full sense; their 
virtue is \textit{virtus} and \textit{arete} -- \textit{i.e.} manliness. What 
is one to think of a woman who should want only to be perfectly ``woman?'' 
That is not given to all, and many a one would therein be fixing for herself 
an unattainable goal. \textit{Feminine}, on the other hand, she is anyhow, by 
nature; femininity is her quality, and she does not need ``true 
femininity.'' I am a man just as the earth is a star. As ridiculous as it 
would be to set the earth the task of being a ``thorough star,'' so 
ridiculous it is to burden me with the call to be a ``thorough man.''

When Fichte says, ``The ego is all,'' this seems to harmonize perfectly with 
my thesis. But it is not that the ego \textit{is} all, but the ego 
\textit{destroys} all, and only the self-dissolving ego, the never-being ego, 
the -- \textit{finite} ego is really I. Fichte speaks of the ``absolute'' 
ego, but I speak of me, the transitory ego.

How natural is the supposition that \textit{man} and \textit{ego} mean the 
same! And yet one sees, \textit{e.g.}, by Feuerbach, that the expression 
``man'' is to designate the absolute ego, the \textit{species}, not the 
transitory, individual ego. Egoism and humanity (humaneness) ought to mean the 
same, but according to Feuerbach the individual can ``only lift himself above 
the limits of his individuality, but not above the laws, the positive 
ordinances, of his species.''\footnote{``Essence of Christianity,'' 2nd 
ed., p. 401} But the species is nothing, and, if the individual lifts himself 
above the limits of his individuality, this is rather his very self as an 
individual; he exists only in raising himself, he exists only in not remaining 
what he is; otherwise he would be done, dead. Man with the great M is only an 
ideal, the species only something thought of. To be a man is not to realize 
the ideal of \textit{Man}, but to present \textit{oneself}, the individual. It 
is not how I realize the \textit{generally human} that needs to be my task, 
but how I satisfy myself. I am my species, am without norm, without law, 
without model, etc. It is possible that I can make very little out of myself; 
but this little is everything, and is better than what I allow to be made out 
of me by the might of others, by the training of custom, religion, the laws, 
the State. Better -- if the talk is to be of better at all -- better an 
unmannerly child than an old head on young shoulders, better a mulish man than 
a man compliant in everything. The unmannerly and mulish fellow is still on 
the way to form himself according to his own will; the prematurely knowing and 
compliant one is determined by the ``species,'' the general demands -- the 
species is law to him. He is \textit{determined}\footnote{[\textit{bestimmt}]} 
by it; for what else is the species to him but his 
``destiny,''\footnote{[\textit{Bestimmung}]} his ``calling''? Whether I 
look to ``humanity,'' the species, in order to strive toward this ideal, or 
to God and Christ with like endeavor, where is the essential dissimilarity? At 
most the former is more washed-out than the latter. As the individual is the 
whole of nature, so he is the whole of the species too.

Everything that I do, think -- in short, my expression or manifestation -- is 
indeed \textit{conditioned} by what I am. The Jew \textit{e.g.} can will only 
thus or thus, can ``present himself'' only thus; the Christian can present 
and manifest himself only Christianly, etc. If it were possible that you could 
be a Jew or Christian, you would indeed bring out only what was Jewish or 
Christian; but it is not possible; in the most rigorous conduct you yet remain 
an \textit{egoist}, a sinner against that concept -- \textit{i.e.}, 
\textit{you} are not the precise equivalent of Jew. Now, because the egoistic 
always keeps peeping through, people have inquired for a more perfect concept 
which should really wholly express what you are, and which, because it is your 
true nature, should contain all the laws of your activity. The most perfect 
thing of the kind has been attained in ``Man.'' As a Jew you are too little, 
and the Jewish is not your task; to be a Greek, a German, does not suffice. 
But be a -- man, then you have everything; look upon the human as your 
calling.

Now I know what is expected of me, and the new catechism can be written. The 
subject is again subjected to the predicate, the individual to something 
general; the dominion is again secured to an \textit{idea}, and the foundation 
laid for a new \textit{religion}. This is a \textit{step forward} in the 
domain of religion, and in particular of Christianity; not a step out beyond 
it.

To step out beyond it leads into the \textit{unspeakable}. For me paltry 
language has no word, and ``the Word,'' the Logos, is to me a ``mere 
word.''

My \textit{essence} is sought for. If not the Jew, the German, etc., then at 
any rate it is -- the man. ``Man is my essence.''

I am repulsive or repugnant to myself; I have a horror and loathing of myself, 
I am a horror to myself, or, I am never enough for myself and never do enough 
to satisfy myself. From such feelings springs self-dissolution or 
self-criticism. Religiousness begins with self-renunciation, ends with 
completed criticism.

I am possessed, and want to get rid of the ``evil spirit.'' How do I set 
about it? I fearlessly commit the sin that seems to the Christian the most 
dire, the sin and blasphemy against the Holy Spirit. ``He who blasphemes the 
Holy Spirit has no forgiveness forever, but is liable to the eternal 
judgment!''\footnote{Mark 3. 29.} I want no forgiveness, and am not afraid of 
the judgment.

\textit{Man} is the last evil \textit{spirit} or spook, the most deceptive or 
most intimate, the craftiest liar with honest mien, the father of lies.

The egoist, turning against the demands and concepts of the present, executes 
pitilessly the most measureless -- \textit{desecration}. Nothing is holy to 
him!

It would be foolish to assert that there is no power above mine. Only the 
attitude that I take toward it will be quite another than that of the 
religious age: I shall be the \textit{enemy} of -- every higher power, while 
religion teaches us to make it our friend and be humble toward it.

The \textit{desecrator} puts forth his strength against every \textit{fear of 
God}, for fear of God would determine him in everything that he left standing 
as sacred. Whether it is the God or the Man that exercises the hallowing power 
in the God-man -- whether, therefore, anything is held sacred for God's sake 
or for Man's (Humanity's) -- this does not change the fear of God, since Man 
is revered as ``supreme essence,'' as much as on the specifically religious 
standpoint God as ``supreme essence'' calls for our fear and reverence; both 
overawe us.

The fear of God in the proper sense was shaken long ago, and a more or less 
conscious ``atheism,'' externally recognizable by a wide-spread 
``unchurchliness,'' has involuntarily become the mode. But what was taken 
from God has been superadded to Man, and the power of humanity grew greater in 
just the degree that of piety lost weight: ``Man'' is the God of today, and 
fear of Man has taken the place of the old fear of God.

But, because Man represents only another Supreme Being, nothing in fact has 
taken place but a metamorphosis in the Supreme Being, and the fear of Man is 
merely an altered form of the fear of God.

Our atheists are pious people.

If in the so-called feudal times we held everything as a fief from God, in the 
liberal period the same feudal relation exists with Man. God was the Lord, now 
Man is the Lord; God was the Mediator, now Man is; God was the Spirit, now Man 
is. In this three fold regard the feudal relation has experienced a 
transformation. For now, firstly, we hold as a fief from all-powerful Man our 
\textit{power}, which, because it comes from a higher, is not called power or 
might, but ``right'' -- the ``rights of man''; we further hold as a fief 
from him our position in the world, for he, the mediator, mediates our 
\textit{intercourse} with others, which therefore may not be otherwise than 
``human''; finally, we hold as a fief from him ourselves -- to wit, our own 
value, or all that we are worth -- inasmuch as we are worth nothing when 
\textit{he} does not dwell in us, and when or where we are not ``human.'' 
The power is Man's, the world is Man's, I am Man's.

But am I not still unrestrained from declaring \textit{myself} the entitler, 
the mediator, and the own self? Then it runs thus:

My power is \textit{my} property.

My power \textit{gives} me property.

My power \textit{am} I myself, and through it am I my property.

\section[1. My Power]{\centering 1. My Power}

Right\footnote{[This word has also, in German, the meaning of ``common 
law,'' and will sometimes be translated ``law'' in the following 
paragraphs.]} is the \textit{spirit of society}. If society has a 
\textit{will} this will is simply right: society exists only through right. 
But, as it endures only exercising a \textit{sovereignty} over individuals, 
right is its SOVEREIGN WILL. Aristotle says justice is the advantage of 
\textit{society}.

All existing right is -- \textit{foreign law;} some one makes me out to be in 
the right, ``does right by me.'' But should I therefore be in the right if 
all the world made me out so? And yet what else is the right that I obtain in 
the State, in society, but a right of those \textit{foreign} to me? When a 
blockhead makes me out in the right, I grow distrustful of my rightness; I 
don't like to receive it from him. But, even when a wise man makes me out in 
the right, I nevertheless am not in the right on that account. Whether I am in 
the right is completely independent of the fool's making out and of the wise 
man's.

All the same, we have coveted this right till now. We seek for right, and turn 
to the court for that purpose. To what? To a royal, a papal, a popular court, 
etc. Can a sultanic court declare another right than that which the sultan has 
ordained to be right? Can it make me out in the right if I seek for a right 
that does not agree with the sultan's law? Can it, \textit{e.g.}, concede to 
me high treason as a right, since it is assuredly not a right according to the 
sultan's mind? Can it as a court of censorship allow me the free utterance of 
opinion as a right, since the sultan will hear nothing of this \textit{my} 
right? What am I seeking for in this court, then? I am seeking for sultanic 
right, not my right; I am seeking for -- \textit{foreign} right. As long as 
this foreign right harmonizes with mine, to be sure, I shall find in it the 
latter too.

The State does not permit pitching into each other man to man; it opposes the 
\textit{duel}. Even every ordinary appeal to blows, notwithstanding that 
neither of the fighters calls the police to it, is punished; except when it is 
not an I whacking away at a you, but, say, the \textit{head of a family} at 
the child. The \textit{family} is entitled to this, and in its name the 
father; I as Ego am not. The \textit{Vossische Zeitung} presents to us the 
``commonwealth of right.'' There everything is to be decided by the judge 
and a \textit{court}. It ranks the supreme court of censorship as a 
``court'' where ``right is declared.'' What sort of a right? The right of 
the censorship. To recognize the sentences of that court as right one must 
regard the censorship as right. But it is thought nevertheless that this court 
offers a protection. Yes, protection against an individual censor's error: it 
protects only the censorship-legislator against false interpretation of his 
will, at the same time making his statute, by the ``sacred power of right,'' 
all the firmer against writers.

Whether I am in the right or not there is no judge but myself. Others can 
judge only whether they endorse my right, and whether it exists as right for 
them too.

In the meantime let us take the matter yet another way. I am to reverence 
sultanic law in the sultanate, popular law in republics, canon law in Catholic 
communities. To these laws I am to subordinate myself; I am to regard them as 
sacred. A ``sense of right'' and ``law-abiding mind'' of such a sort is so 
firmly planted in people's heads that the most revolutionary persons of our 
days want to subject us to a new ``sacred law,'' the ``law of society,'' 
the law of mankind, the ``right of all,'' and the like. The right of 
``all'' is to go before \textit{my} right. As a right of all it would indeed 
be my right among the rest, since I, with the rest, am included in all; but 
that it is at the same time a right of others, or even of all others, does not 
move me to its upholding. Not as a right \textit{of all} will I defend it, but 
as \textit{my} right; and then every other may see to it how he shall likewise 
maintain it for himself. The right of all (\textit{e.g.,} to eat) is a right 
of every individual. Let each keep this right unabridged for \textit{himself}, 
then all exercise it spontaneously; let him not take care for all though -- 
let him not grow zealous for it as for a right of all.

But the social reformers preach to us a \textit{``law of society''}. There 
the individual becomes society's slave, and is in the right only when society 
\textit{makes him out} in the right, \textit{i.e.} when he lives according to 
society's \textit{statutes} and so is -- \textit{loyal}. Whether I am loyal 
under a despotism or in a ``society'' \textit{\`a la} Weitling, it is the 
same absence of right in so far as in both cases I have not \textit{my} right 
but \textit{foreign} right.

In consideration of right the question is always asked, ``What or who gives 
me the right to it?'' Answer: God, love, reason, nature, humanity, etc. No, 
only \textit{your might, your} power gives you the right (your reason, 
\textit{e.g.,}, may give it to you).

Communism, which assumes that men ``have equal rights by nature,'' 
contradicts its own proposition till it comes to this, that men have no right 
at all by nature. For it is not willing to recognize, \textit{e.g.}, that 
parents have ``by nature'' rights as against their children, or the children 
as against the parents: it abolishes the family. Nature gives parents, 
brothers, etc., no right at all. Altogether, this entire revolutionary or 
Babouvist principle\footnote{Cf. \textit{``Die Kommunisten in der 
Schweiz},'' committee report, p. 3.} rests on a religious, \textit{i.e.}, 
false, view of things. Who can ask after ``right'' if he does not occupy the 
religious standpoint himself? Is not ``right'' a religious concept, 
\textit{i.e.} something sacred? Why, \textit{``equality of rights''}, as the 
Revolution propounded it, is only another name for ``Christian equality,'' 
the ``equality of the brethren,'' ``of God's children,'' ``of 
Christians''; in short, \textit{fraternit\'e}. Each and every inquiry after 
right deserves to be lashed with Schiller's words:

\begin{quotation}

\noindent{} Many a year I've used my nose\\
 To smell the onion and the rose;\\
 Is there any proof which shows\\
 That I've a right to that same nose? \end{quotation}

\noindent{}When the Revolution stamped equality as a ``right,'' it took 
flight into the religious domain, into the region of the sacred, of the ideal. 
Hence, since then, the fight for the ``sacred, inalienable rights of man.'' 
Against the ``eternal rights of man'' the ``well-earned rights of the 
established order'' are quite naturally, and with equal right, brought to 
bear: right against right, where of course one is decried by the other as 
``wrong.'' This has been the \textit{contest of 
rights}\footnote{[\textit{Rechtsstreit}, a word which usually means 
``lawsuit.'']} since the Revolution.

You want to be ``in the right'' as against the rest. That you cannot; as 
against them you remain forever ``in the wrong''; for they surely would not 
be your opponents if they were not in ``their right'' too; they will always 
make you out ``in the wrong.'' But, as against the right of the rest, yours 
is a higher, greater, \textit{more powerful} right, is it not? No such thing! 
Your right is not more powerful if you are not more powerful. Have Chinese 
subjects a right to freedom? Just bestow it on them, and then look how far you 
have gone wrong in your attempt: because they do not know how to use freedom 
they have no right to it, or, in clearer terms, because they have not freedom 
they have not the right to it. Children have no right to the condition of 
majority because they are not of age, \textit{i.e.} because they are children. 
Peoples that let themselves be kept in nonage have no rights to the condition 
of majority; if they ceased to be in nonage, then only would they have the 
right to be of age. This means nothing else than ``What you have the 
\textit{power} to be you have the \textit{right} to.'' I derive all right and 
all warrant from \textit{me ;} I am \textit{entitled} to everything that I 
have in my power. I am entitled to overthrow Zeus, Jehovah, God, etc., if I 
\textit{can ;} if I cannot, then these gods will always remain in the right 
and in power as against me, and what I do will be to fear their right and 
their power in impotent ``god-fearingness,'' to keep their commandments and 
believe that I do right in everything that I do according to \textit{their} 
right, about as the Russian boundary-sentinels think themselves rightfully 
entitled to shoot dead the suspicious persons who are escaping, since they 
murder ``by superior authority,'' \textit{i.e.} ``with right.'' But I am 
entitled by myself to murder if I myself do not forbid it to myself, if I 
myself do not fear murder as a ``wrong.'' This view of things lies at the 
foundation of Chamisso's poem, ``The Valley of Murder,'' where the 
gray-haired Indian murderer compels reverence from the white man whose 
brethren he has murdered. The only thing I am not entitled to is what I do not 
do with a free cheer, \textit{i.e.} what I do not entitle myself to.

I decide whether it is the \textit{right thing} in me; there is no right 
\textit{outside} me. If it is right for \textit{me},\footnote{[A common German 
phrase for ``it suits me.'']} it is right. Possibly this may not suffice to 
make it right for the rest; \textit{i.e.}, their care, not mine: let them 
defend themselves. And if for the whole world something were not right, but it 
were right for me, \textit{i.e.}, I wanted it, then I would ask nothing about 
the whole world. So every one does who knows how to value \textit{himself}, 
every one in the degree that he is an egoist; for might goes before right, and 
that -- with perfect right.

Because I am ``by nature'' a man I have an equal right to the enjoyment of 
all goods, says Babeuf. Must he not also say: because I am ``by nature'' a 
first-born prince I have a right to the throne? The rights of man and the 
``well-earned rights'' come to the same thing in the end, \textit{i.e.} to 
\textit{nature}, which \textit{gives} me a right, \textit{i.e.} to 
\textit{birth} (and, further, inheritance, etc.). ``I am born as a man'' is 
equal to ``I am born as a king's son.'' The natural man has only a natural 
right (because he has only a natural power) and natural claims: he has right 
of birth and claims of birth. But \textit{nature} cannot entitle me, 
\textit{i.e.} give me capacity or might, to that to which only my act entitles 
me. That the king's child sets himself above other children, even this is his 
act, which secures to him the precedence; and that the other children approve 
and recognize this act is their act, which makes them worthy to be -- 
subjects.

Whether nature gives me a right, or whether God, the people's choice, etc., 
does so, all of \textit{i.e.}, the same \textit{foreign} right, a right that 
I do not give or take to myself.

Thus the Communists say, equal labor entitles man to equal enjoyment. Formerly 
the question was raised whether the ``virtuous'' man must not be ``happy'' 
on earth. The Jews actually drew this inference: ``That it may go well with 
thee on earth.'' No, equal labor does not entitle you to it, but equal 
enjoyment alone entitles you to equal enjoyment. Enjoy, then you are entitled 
to enjoyment. But, if you have labored and let the enjoyment be taken from 
you, then -- ``it serves you right.''

If you \textit{take} the enjoyment, it is your right; if, on the contrary, you 
only pine for it without laying hands on it, it remains as before, a, 
``well-earned right'' of those who are privileged for enjoyment. It is 
\textit{their} right, as by laying hands on it would become your right.

The conflict over the ``right of property'' wavers in vehement commotion. 
The Communists affirm\footnote{A. Becker, \textit{``Volksphilosophie}'', p. 
22f.} that ``the earth belongs rightfully to him who tills it, and its 
products to those who bring them out.'' I think it belongs to him who knows 
how to take it, or who does not let it be taken from him, does not let himself 
be deprived of it. If he appropriates it, then not only the earth, but the 
right to it too, belongs to him. This is \textit{egoistic right}: 
\textit{i.e.} it is right for \textit{me}, therefore it is right.

Aside from this, right does have ``a wax nose.'' The tiger that assails me 
is in the right, and I who strike him down am also in the right. I defend 
against him not my \textit{right}, but \textit{myself.}

As human right is always something given, it always in reality reduces to the 
right which men give, \textit{i.e.} ``concede,'' to each other. If the right 
to existence is conceded to new-born children, then they have the right; if it 
is not conceded to them, as was the case among the Spartans and ancient 
Romans, then they do not have it. For only society can give or concede it to 
them; they themselves cannot take it, or give it to themselves. It will be 
objected, the children had nevertheless ``by nature'' the right to exist; 
only the Spartans refused \textit{recognition} to this right. But then they 
simply had no right to this recognition -- no more than they had to 
recognition of their life by the wild beasts to which they were thrown.

People talk so much about \textit{birthright} and complain:

\begin{quotation}

\noindent{} There is alas! -- no mention of the rights\\
 That were born with us.\footnote{[Mephistopheles in ``Faust.'']} 
\end{quotation}

\noindent{}What sort of right, then, is there that was born with me? The right 
to receive an entailed estate, to inherit a throne, to enjoy a princely or 
noble education; or, again, because poor parents begot me, to -- get free 
schooling, be clothed out of contributions of alms, and at last earn my bread 
and my herring in the coal-mines or at the loom? Are these not birthrights, 
rights that have come down to me from my parents through \textit{birth?} You 
think -- no; you think these are only rights improperly so called, it is just 
these rights that you aim to abolish through the \textit{real birthright}. To 
give a basis for this you go back to the simplest thing and affirm that every 
one is by birth \textit{equal} to another -- to wit, a \textit{man}. I will 
grant you that every one is born as man, hence the new-born are therein 
\textit{equal} to each other. Why are they? Only because they do not yet show 
and exert themselves as anything but bare -- \textit{children of men}, naked 
little human beings. But thereby they are at once different from those who 
have already made something out of themselves, who thus are no longer bare 
``children of man,'' but -- children of their own creation. The latter 
possesses more than bare birthrights: they have \textit{earned} rights. What 
an antithesis, what a field of combat! The old combat of the birthrights of 
man and well-earned rights. Go right on appealing to your birthrights; people 
will not fail to oppose to you the well-earned. Both stand on the ``ground of 
right''; for each of the two has a ``right'' against the other, the one the 
birthright of natural right, the other the earned or ``well-earned'' right.

If you remain on the ground of right, you remain in -- 
\textit{Rechthaberei}.\footnote{``I beg you, spare my lungs! He who insists 
on proving himself right, if he but has one of those things called tongues, 
can hold his own in all the world's despite!'' [Faust's words to 
Mephistopheles, slightly misquoted. -- For \textit{Rechthaberei}see note on p. 
185.]} The other cannot give you your right; he cannot ``mete out right'' to 
you. He who has might has -- right; if you have not the former, neither have 
you the latter. Is this wisdom so hard to attain? Just look at the mighty and 
their doings! We are talking here only of China and Japan, of course. Just try 
it once, you Chinese and Japanese, to make them out in the wrong, and learn by 
experience how they throw you into jail. (Only do not confuse with this the 
``well-meaning counsels'' which -- in China and Japan -- are permitted, 
because they do not hinder the mighty one, but possibly \textit{help him on}.) 
For him who should want to make them out in the wrong there would stand open 
only one way thereto, that of might. If he deprives them of their 
\textit{might}, then he has \textit{really} made them out in the wrong, 
deprived them of their right; in any other case he can do nothing but clench 
his little fist in his pocket, or fall a victim as an obtrusive fool.

In short, if you Chinese or Japanese did not ask after right, and in 
particular if you did not ask after the rights ``that were born with you,'' 
then you would not need to ask at all after the well-earned rights either.

You start back in fright before others, because you think you see beside them 
the \textit{ghost of right}, which, as in the Homeric combats, seems to fight 
as a goddess at their side, helping them. What do you do? Do you throw the 
spear? No, you creep around to gain the spook over to yourselves, that it may 
fight on your side: you woo for the ghost's favor. Another would simply ask 
thus: Do I will what my opponent wills? ``No!'' Now then, there may fight 
for him a thousand devils or gods, I go at him all the same!

The ``commonwealth of right,'' as the \textit{Vossische Zeitung} among 
others stands for it, asks that office-holders be removable only by the 
\textit{judge}, not by the \textit{administration}. Vain illusion! If it were 
settled by law that an office-holder who is once seen drunken shall lose his 
office, then the judges would have to condemn him on the word of the 
witnesses. In short, the law-giver would only have to state precisely all the 
possible grounds which entail the loss of office, however laughable they might 
be (\textit{e.g.} he who laughs in his superiors' faces, who does not go to 
church every Sunday, who does not take the communion every four weeks, who 
runs in debt, who has disreputable associates, who shows no determination, 
etc., shall be removed. These things the law-giver might take it into his head 
to prescribe, \textit{e.g.}, for a court of honor); then the judge would 
solely have to investigate whether the accused had ``become guilty'' of 
those ``offenses,'' and, on presentation of the proof, pronounce sentence of 
removal against him ``in the name of the law.''

The judge is lost when he ceases to be \textit{mechanical}, when he ``is 
forsaken by the rules of evidence.'' Then he no longer has anything but an 
opinion like everybody else; and, if he decides according to this 
\textit{opinion}, his action is \textit{no longer an official action}. As 
judge he must decide only according to the law. Commend me rather to the old 
French parliaments, which wanted to examine for themselves what was to be 
matters of right, and to register it only after their own approval. They at 
least judged according to a right of their own, and were not willing to give 
themselves up to be machines of the law-giver, although as judges they must, 
to be sure, become their own machines.

It is said that punishment is the criminal's right. But impunity is just as 
much his right. If his undertaking succeeds, it serves him right, and, if it 
does not succeed, it likewise serves him right. You make your bed and lie in 
it. If some one goes foolhardily into dangers and perishes in them, we are apt 
to say, ``It serves him right; he would have it so.'' But, if he conquered 
the dangers, \textit{i.e.} if his \textit{might} was victorious, then he would 
be in the \textit{right} too. If a child plays with the knife and gets cut, it 
is served right; but, if it doesn't get cut, it is served right too. Hence 
right befalls the criminal, doubtless, when he suffers what he risked; why, 
what did he risk it for, since he knew the possible consequences? But the 
punishment that we decree against him is only our right, not his. Our right 
reacts against his, and he is -- ``in the wrong at last'' because -- we get 
the upper hand.

\myhrule


But what is right, what is matter of right in a society, is voiced too -- in 
the law.\footnote{[\textit{Gesetz}, statute; no longer the same German word as 
``right'']}

Whatever the law may be, it must be respected by the -- loyal citizen. Thus 
the law-abiding mind of Old England is eulogized. To this that Euripidean 
sentiment (Orestes, 418) entirely corresponds: ``We serve the gods, whatever 
the gods are.'' \textit{Law as such, God as such}, thus far we are today.

People are at pains to distinguish \textit{law} from arbitrary \textit{orders}, 
from an ordinance: the former comes from a duly entitled authority. But a law 
over human action (ethical law, State law, etc.) is always a 
\textit{declaration of will}, and so an order. Yes, even if I myself gave 
myself the law, it would yet be only my order, to which in the next moment I 
can refuse obedience. One may well enough declare what he will put up with, 
and so deprecate the opposite of the law, making known that in the contrary 
case he will treat the transgressor as his enemy; but no one has any business 
to command \textit{my} actions, to say what course I shall pursue and set up a 
code to govern it. I must put up with it that he treats me as his 
\textit{enemy}, but never that he makes free with me as his \textit{creature}, 
and that he makes \textit{his} reason, or even unreason, my plumbline.

States last only so long as there is \textit{a ruling will} and this ruling 
will is looked upon as tantamount to the own will. The lord's will is -- law. 
What do your laws amount to if no one obeys them? What your orders, if nobody 
lets himself be ordered? The State cannot forbear the claim to determine the 
individual's will, to speculate and count on this. For the State it is 
indispensable that nobody have an \textit{own will ;} if one had, the State 
would have to exclude (lock up, banish, etc.) this one; if all had, they would 
do away with the State. The State is not thinkable without lordship and 
servitude (subjection); for the State must will to be the lord of all that it 
embraces, and this will is called the ``will of the State.''

He who, to hold his own, must count on the absence of will in others is a 
thing made by these others, as the master is a thing made by the servant. If 
submissiveness ceased, it would be over with all lordship.

The \textit{own will} of Me is the State's destroyer; it is therefore branded 
by the State as ``self-will.'' Own will and the State are powers in deadly 
hostility, between which no ``eternal peace'' is possible. As long as the 
State asserts itself, it represents own will, its ever-hostile opponent, as 
unreasonable, evil; and the latter lets itself be talked into believing this -- nay, it really is such, for no more reason than this, that it still lets 
itself be talked into such belief: it has not yet come to itself and to the 
consciousness of its dignity; hence it is still incomplete, still amenable to 
fine words, etc.

Every State is a \textit{despotism}, be the despot one or many, or (as one is 
likely to imagine about a republic) if all be lords, \textit{i.e.} despotize 
one over another. For this is the case when the law given at any time, the 
expressed volition of (it may be) a popular assembly, is thenceforth to be 
\textit{law} for the individual, to which \textit{obedience is due} from him 
or toward which he has the \textit{duty} of obedience. If one were even to 
conceive the case that every individual in the people had expressed the same 
will, and hereby a complete ``collective will'' had come into being, the 
matter would still remain the same. Would I not be bound today and henceforth 
to my will of yesterday? My will would in this case be \textit{frozen}. 
Wretched \textit{stability!} My creature -- to wit, a particular expression of 
will -- would have become my commander. But I in my will, I the creator, 
should be hindered in my flow and my dissolution. Because I was a fool 
yesterday I must remain such my life long. So in the State-life I am at best -- I might just as well say, at worst -- a bondman of myself. Because I was a 
willer yesterday, I am today without will: yesterday voluntary, today 
involuntary.

How change it? Only be recognizing no \textit{duty}, not \textit{binding} 
myself nor letting myself be bound. If I have no duty, then I know no law 
either.

``But they will bind me!'' My will nobody can bind, and my disinclination 
remains free.

``Why, everything must go topsy-turvy if every one could do what he would!'' 
Well, who says that every one can do everything? What are you there for, pray, 
you who do not need to put up with everything? Defend yourself, and no one 
will do anything to you! He who would break your will has to do with you, and 
is your \textit{enemy}. Deal with him as such. If there stand behind you for 
your protection some millions more, then you are an imposing power and will 
have an easy victory. But, even if as a power you overawe your opponent, still 
you are not on that account a hallowed authority to him, unless he be a 
simpleton. He does not owe you respect and regard, even though he will have to 
consider your might.

We are accustomed to classify States according to the different ways in which 
``the supreme might'' is distributed. If an individual has it -- monarchy; 
if all have it -- democracy; etc. Supreme might then! Might against whom? 
Against the individual and his ``self-will.'' The State practices 
``violence,'' the individual must not do so. The State's behavior is 
violence, and it calls its violence ``law''; that of the individual, 
``crime.'' Crime, then\footnote{[\textit{Verbrechen}]} -- so the 
individual's violence is called; and only by crime does he 
overcome\footnote{[\textit{brechen}]} the State's violence when he thinks that 
the State is not above him, but he is above the State.

Now, if I wanted to act ridiculously, I might, as a well-meaning person, 
admonish you not to make laws which impair my self-development, self-activity, 
self-creation. I do not give this advice. For, if you should follow it, you 
would be unwise, and I should have been cheated of my entire profit. I request 
nothing at all from you; for, whatever I might demand, you would still be 
dictatorial law-givers, and must be so, because a raven cannot sing, nor a 
robber live without robbery. Rather do I ask those who would be egoists what 
they think the more egoistic -- to let laws be given them by you, and to 
respect those that are given, or to practice \textit{refractoriness}, yes, 
complete disobedience. Good-hearted people think the laws ought to prescribe 
only what is accepted in the people's feeling as right and proper. But what 
concern is it of mine what is accepted in the nation and by the nation? The 
nation will perhaps be against the blasphemer; therefore a law against 
blasphemy. Am I not to blaspheme on that account? Is this law to be more than 
an ``order'' to me? I put the question.

Solely from the principle that all \textit{right} and all \textit{authority} 
belong to the \textit{collectivity of the people} do all forms of government 
arise. For none of them lacks this appeal to the collectivity, and the despot, 
as well as the president or any aristocracy, acts and commands ``in the name 
of the State.'' They are in possession of the ``authority of the State,'' 
and it is perfectly indifferent whether, were this possible, the people as a 
\textit{collectivity} (all individuals) exercise this State -- 
\textit{authority}, or whether it is only the representatives of this 
collectivity, be there many of them as in aristocracies or one as in 
monarchies. Always the collectivity is above the individual, and has a power 
which is called \textit{legitimate}, \textit{i.e.} which is \textit{law}.

Over against the sacredness of the State, the individual is only a vessel of 
dishonor, in which ``exuberance, malevolence, mania for ridicule and slander, 
frivolity,'' etc., are left as soon as he does not deem that object of 
veneration, the State, to be worthy of recognition. The spiritual 
\textit{haughtiness} of the servants and subjects of the State has fine 
penalties against unspiritual ``exuberance.''

When the government designates as punishable all play of mind \textit{against} 
the State, the moderate liberals come and opine that fun, satire, wit, humor, 
must have free play anyhow, and \textit{genius} must enjoy freedom. So not the 
\textit{individual man} indeed, but still \textit{genius}, is to be free. Here 
the State, or in its name the government, says with perfect right: He who is 
not for me is against me. Fun, wit, etc. -- in short, the turning of State 
affairs into a comedy -- have undermined States from of old: they are not 
``innocent.'' And, further, what boundaries are to be drawn between guilty 
and innocent wit, etc.? At this question the moderates fall into great 
perplexity, and everything reduces itself to the prayer that the State 
(government) would please not be so \textit{sensitive}, so \textit{ticklish ;} 
that it would not immediately scent malevolence in ``harmless'' things, and 
would in general be a little ``more tolerant.'' Exaggerated sensitiveness is 
certainly a weakness, its avoidance may be praiseworthy virtue; but in time of 
war one cannot be sparing, and what may be allowed under peaceable 
circumstances ceases to be permitted as soon as a state of siege is declared. 
Because the well-meaning liberals feel this plainly, they hasten to declare 
that, considering ``the devotion of the people,'' there is assuredly no 
danger to be feared. But the government will be wiser, and not let itself be 
talked into believing anything of that sort. It knows too well how people 
stuff one with fine words, and will not let itself be satisfied with the 
Barmecide dish.

But they are bound to have their play-ground, for they are children, you know, 
and cannot be so staid as old folks; boys will be boys. Only for this 
playground, only for a few hours of jolly running about, they bargain. They 
ask only that the State should not, like a splenetic papa, be too cross. It 
should permit some Processions of the Ass and plays of fools, as the church 
allowed them in the Middle Ages. But the times when it could grant this 
without danger are past. Children that now once come \textit{into the open}, 
and live through an hour without the rod of discipline, are no longer willing 
to go into the \textit{cell}. For the open is now no longer a 
\textit{supplement} to the cell, no longer a refreshing \textit{recreation}, 
but its \textit{opposite}, an \textit{aut-aut}. In short, the State must 
either no longer put up with anything, or put up with everything and perish; 
it must be either sensitive through and through, or, like a dead man, 
insensitive. Tolerance is done with. If the State but gives a finger, they 
take the whole hand at once. There can be no more ``jesting,'' and all jest, 
such as fun, wit, humor, becomes bitter earnest.

The clamor of the Liberals for freedom of the press runs counter to their own 
principle, their proper \textit{will}. They will what they \textit{do not 
will}, \textit{i.e.} they wish, they would like. Hence it is too that they 
fall away so easily when once so-called freedom of the press appears; then 
they would like censorship. Quite naturally. The State is sacred even to them; 
likewise morals. They behave toward it only as ill-bred brats, as tricky 
children who seek to utilize the weaknesses of their parents. Papa State is to 
permit them to say many things that do not please him, but papa has the right, 
by a stern look, to blue-pencil their impertinent gabble. If they recognize in 
him their papa, they must in his presence put up with the censorship of 
speech, like every child.

\myhrule


If you let yourself be made out in the right by another, you must no less let 
yourself be made out in the wrong by him; if justification and reward come to 
you from him, expect also his arraignment and punishment. Alongside right goes 
wrong, alongside legality \textit{crime}. What are you? -- You are a -- 
\textit{criminal!}

``The criminal is in the utmost degree the State's own crime!'' says 
Bettina.\footnote{``This Book Belongs to the King,'', p. 376.} One may let 
this sentiment pass, even if Bettina herself does not understand it exactly 
so. For in the State the unbridled I -- I, as I belong to myself alone -- 
cannot come to my fulfillment and realization. Every ego is from birth a 
criminal to begin with against the people, the State. Hence it is that it does 
really keep watch over all; it sees in each one an -- egoist, and it is afraid 
of the egoist. It presumes the worst about each one, and takes care, 
police-care, that ``no harm happens to the State,'' \textit{ne quid 
respublica detrimenti capiat}. The unbridled ego -- and this we originally 
are, and in our secret inward parts we remain so always -- is the 
never-ceasing criminal in the State. The man whom his boldness, his will, his 
inconsiderateness and fearlessness lead is surrounded with spies by the State, 
by the people. I say, by the people! The people (think it something wonderful, 
you good-hearted folks, what you have in the people) -- the people is full of 
police sentiments through and through. -- Only he who renounces his ego, who 
practices ``self-renunciation,'' is acceptable to the people.

In the book cited Bettina is throughout good-natured enough to regard the 
State as only sick, and to hope for its recovery, a recovery which she would 
bring about through the ``demagogues'';\footnote{P. 376.} but it is not 
sick; rather is it in its full strength, when it puts from it the demagogues 
who want to acquire something for the individuals, for ``all.'' In its 
believers it is provided with the best demagogues (leaders of the people). 
According to Bettina, the State is to\footnote{P. 374.} ``develop mankind's 
germ of freedom; otherwise it is a raven-mother\footnote{[An unnatural 
mother]} and caring for raven-fodder!'' It cannot do otherwise, for in its 
very caring for ``mankind'' (which, besides, would have to be the 
``humane'' or `` free'' State to begin with) the ``individual'' is 
raven-fodder for it. How rightly speaks the burgomaster, on the other 
hand:\footnote{P. 381.} ``What? the State has no other duty than to be merely 
the attendant of incurable invalids? -- that isn't to the point. From of old 
the healthy State has relieved itself of the diseased matter, and not mixed 
itself with it. It does not need to be so economical with its juices. Cut off 
the robber-branches without hesitation, that the others may bloom. -- Do not 
shiver at the State's harshness; its morality, its policy and religion, point 
it to that. Accuse it of no want of feeling; its sympathy revolts against 
this, but its experience finds safety only in this severity! There are 
diseases in which only drastic remedies will help. The physician who 
recognizes the disease as such, but timidly turns to palliatives, will never 
remove the disease, but may well cause the patient to succumb after a shorter 
or longer sickness.'' Frau Rat's question, ``If you apply death as a drastic 
remedy, how is the cure to be wrought then?'' isn't to the point. Why, the 
State does not apply death against itself, but against an offensive member; it 
tears out an eye that offends it, etc.

``For the invalid State the only way of salvation is to make man flourish in 
it.''\footnote{P. 385.} If one here, like Bettina, understand by man the 
concept ``Man,'' she is right; the ``invalid'' State will recover by the 
flourishing of ``Man,'' for, the more infatuated the individuals are with 
``Man,'' the better it serves the State's turn. But, if one referred it to 
the individuals, to ``all'' (and the authoress half-does this too, because 
about ``Man'' she is still involved in vagueness), then it would sound 
somewhat like the following: For an invalid band of robbers the only way of 
salvation is to make the loyal citizen nourish in it! Why, thereby the band of 
robbers would simply go to ruin as a band of robbers; and, because it 
perceives this, it prefers to shoot every one who has a leaning toward 
becoming a ``steady man.''

In this book Bettina is a patriot, or, what is little more, a philanthropist, 
a worker for human happiness. She is discontented with the existing order in 
quite the same way as is the title-ghost of her book, along with all who would 
like to bring back the good old faith and what goes with it. Only she thinks, 
contrariwise, that the politicians, place-holders, and diplomats ruined the 
State, while those lay it at the door of the malevolent, the ``seducers of 
the people.''

What is the ordinary criminal but one who has committed the fatal mistake of 
endeavoring after what is the people's instead of seeking for what is his? He 
has sought despicable \textit{alien} goods, has done what believers do who 
seek after what is God's. What does the priest who admonishes the criminal do? 
He sets before him the great wrong of having desecrated by his act what was 
hallowed by the State, its property (in which, of course, must be included 
even the life of those who belong to the State); instead of this, he might 
rather hold up to him the fact that he has befouled \textit{himself} in not 
despising the alien thing, but thinking it worth stealing; he could, if he 
were not a parson. Talk with the so-called criminal as with an egoist, and he 
will be ashamed, not that he transgressed against your laws and goods, but 
that he considered your laws worth evading, your goods worth desiring; he will 
be ashamed that he did not -- despise you and yours together, that he was too 
little an egoist. But you cannot talk egoistically with him, for you are not 
so great as a criminal, you -- commit no crime! You do not know that an ego 
who is his own cannot desist from being a criminal, that crime is his life. 
And yet you should know it, since you believe that ``we are all miserable 
sinners''; but you think surreptitiously to get beyond sin, you do not 
comprehend -- for you are devil-fearing -- that guilt is the value of a man. 
Oh, if you were guilty! But now you are 
``righteous.''\footnote{[\textit{Gerechte}]} Well -- just put every thing 
nicely to rights\footnote{[\textit{macht Alles h\"ubsch gerecht}]} for your 
master!

When the Christian consciousness, or the Christian man, draws up a criminal 
code, what can the concept of \textit{crime} be there but simply -- 
\textit{heartlessness?} Each severing and wounding of a \textit{heart 
relation}, each \textit{heartless behavior} toward a sacred being, is crime. 
The more heartfelt the relation is supposed to be, the more scandalous is the 
deriding of it, and the more worthy of punishment the crime. Everyone who is 
subject to the lord should love him; to deny this love is a high treason 
worthy of death. Adultery is a heartlessness worthy of punishment; one has no 
heart, no enthusiasm, no pathetic feeling for the sacredness of marriage. So 
long as the heart or soul dictates laws, only the heartful or soulful man 
enjoys the protection of the laws. That the man of soul makes laws means 
properly that the \textit{moral} man makes them: what contradicts these men's 
``moral feeling,'' this they penalize. How, \textit{e.g.}, should 
disloyalty, secession, breach of oaths -- in short, all \textit{radical 
breaking off}, all tearing asunder of venerable \textit{ties --} not be 
flagitious and criminal in their eyes? He who breaks with these demands of the 
soul has for enemies all the moral, all the men of soul. Only Krummacher and 
his mates are the right people to set up consistently a penal code of the 
heart, as a certain bill sufficiently proves. The consistent legislation of 
the Christian State must be placed wholly in the hands of the -- 
\textit{parsons}, and will not become pure and coherent so long as it is 
worked out only by -- the \textit{parson-ridden}, who are always only 
\textit{half-parsons}. Only then will every lack of soulfulness, every 
heartlessness, be certified as an unpardonable crime, only then will every 
agitation of the soul become condemnable, every objection of criticism and 
doubt be anathematized; only then is the own man, before the Christian 
consciousness, a convicted -- \textit{criminal} to begin with.

The men of the Revolution often talked of the people's ``just revenge'' as 
its ``right.'' Revenge and right coincide here. Is this an attitude of an 
ego to an ego? The people cries that the opposite party has committed 
``crimes'' against it. Can I assume that one commits a crime against me, 
without assuming that he has to act as I see fit? And this action I call the 
right, the good, etc.; the divergent action, a crime. So I think that the 
others must aim at the \textit{same} goal with me; \textit{i.e.}, I do not 
treat them as unique beings\footnote{[\textit{Einzige}]} who bear their law in 
themselves and live according to it, but as beings who are to obey some 
``rational'' law. I set up what ``Man'' is and what acting in a ``truly 
human'' way is, and I demand of every one that this law become norm and ideal 
to him; otherwise he will expose himself as a ``sinner and criminal.'' But 
upon the ``guilty'' falls the ``penalty of the law''!

One sees here how it is ``Man'' again who sets on foot even the concept of 
crime, of sin, and therewith that of right. A man in whom I do not recognize 
``man'' is ``sinner, a guilty one.''

Only against a sacred thing are there criminals; you against me can never be a 
criminal, but only an opponent. But not to hate him who injures a sacred thing 
is in itself a crime, as St. Just cries out against Danton: ``Are you not a 
criminal and responsible for not having hated the enemies of the 
fatherland?'' --

If, as in the Revolution, what ``Man'' is apprehended as ``good citizen,'' 
then from this concept of ``Man'' we have the well-known ``political 
offenses and crimes.''

In all this the individual, the individual man, is regarded as refuse, and on 
the other hand the general man, ``Man,'' is honored. Now, according to how 
this ghost is named -- as Christian, Jew, Mussulman, good citizen, loyal 
subject, freeman, patriot, etc. -- just so do those who would like to carry 
through a divergent concept of man, as well as those who want to put 
\textit{themselves} through, fall before victorious ``Man.''

And with what unction the butchery goes on here in the name of the law, of the 
sovereign people, of God, etc.!

Now, if the persecuted trickily conceal and protect themselves from the stern 
parsonical judges, people stigmatize them as St. Just, \textit{e.g.}, does 
those whom he accuses in the speech against Danton.\footnote{See ``Political 
Speeches,'' 10, p. 153} One is to be a fool, and deliver himself up to their 
Moloch.

Crimes spring from \textit{fixed ideas}. The sacredness of marriage is a fixed 
idea. From the sacredness it follows that infidelity is a \textit{crime}, and 
therefore a certain marriage law imposes upon it a shorter or longer 
\textit{penalty}. But by those who proclaim ``freedom as sacred'' this 
penalty must be regarded as a crime against freedom, and only in this sense 
has public opinion in fact branded the marriage law.

Society would have \textit{every one} come to his right indeed, but yet only 
to that which is sanctioned by society, to the society-right, not really to 
\textit{his} right. But I give or take to myself the right out of my own 
plenitude of power, and against every superior power I am the most impenitent 
criminal. Owner and creator of my right, I recognize no other source of right 
than -- me, neither God nor the State nor nature nor even man himself with his 
``eternal rights of man,'' neither divine nor human right.

Right ``in and for itself.'' Without relation to me, therefore! ``Absolute 
right.'' Separated from me, therefore! A thing that exists in and for itself! 
An absolute! An eternal right, like an eternal truth!

According to the liberal way of thinking, right is to be obligatory for me 
because it is thus established by \textit{human reason}, against which 
\textit{my reason} is ``unreason.'' Formerly people inveighed in the name of 
divine reason against weak human reason; now, in the name of strong human 
reason, against egoistic reason, which is rejected as ``unreason.'' And yet 
none is real but this very ``unreason.'' Neither divine nor human reason, 
but only your and my reason existing at any given time, is real, as and 
because you and I are real.

The thought of right is originally my thought; or, it has its origin in me. 
But, when it has sprung from me, when the ``Word'' is out, then it has 
``become flesh,'' it is a \textit{fixed idea}. Now I no longer get rid of 
the thought; however I turn, it stands before me. Thus men have not become 
masters again of the thought ``right,'' which they themselves created; their 
creature is running away with them. This is absolute right, that which is 
absolved or unfastened from me. We, revering it as absolute, cannot devour it 
again, and it takes from us the creative power: the creature is more than the 
creator, it is ``in and for itself.''

Once you no longer let right run around free, once you draw it back into its 
origin, into you, it is \textit{your} right; and that is right which suits 
you.

\myhrule


Right has had to suffer an attack within itself, \textit{i.e.} from the 
standpoint of right; war being declared on the part of liberalism against 
``privilege.''\footnote{[Literally, ``precedent right.'']}

\textit{Privileged} and \textit{endowed with equal rights --} on these two 
concepts turns a stubborn fight. Excluded or admitted -- would mean the same. 
But where should there be a power -- be it an imaginary one like God, law, or 
a real one like I, you -- of which it should not be true that before it all 
are ``endowed with equal rights,'' \textit{i.e.}, no respect of persons 
holds? Every one is equally dear to God if he adores him, equally agreeable to 
the law \textit{if} only he is a law- abiding person; whether the lover of God 
and the law is humpbacked and lame, whether poor or rich, etc., that amounts 
to nothing for God and the law; just so, when you are at the point of 
drowning, you like a Negro as rescuer as well as the most excellent Caucasian -- yes, in this situation you esteem a dog not less than a man. But to whom 
will not every one be also, contrariwise, a preferred or disregarded person? 
God punishes the wicked with his wrath, the law chastises the lawless, you let 
one visit you every moment and show the other the door.

The ``equality of right'' is a phantom just because right is nothing more 
and nothing less than admission, \textit{a matter of grace}, which, be it 
said, one may also acquire by his desert; for desert and grace are not 
contradictory, since even grace wishes to be ``deserved'' and our gracious 
smile falls only to him who knows how to force it from us.

So people dream of ``all citizens of the State having to stand side by side, 
with equal rights.'' As citizens of the State they are certainly all equal 
for the State. But it will divide them, and advance them or put them in the 
rear, according to its special ends, if on no other account; and still more 
must it distinguish them from one another as good and bad citizens.

Bruno Bauer disposes of the Jew question from the standpoint that 
``privilege'' is not justified. Because Jew and Christian have each some 
point of advantage over the other, and in having this point of advantage are 
exclusive, therefore before the critic's gaze they crumble into nothingness. 
With them the State lies under the like blame, since it justifies their having 
advantages and stamps it as a ``privilege.'' or prerogative, but thereby 
derogates from its calling to become a ``free State.''

But now every one has something of advantage over another -- \textit{viz}., 
himself or his individuality; in this everybody remains exclusive.

And, again, before a third party every one makes his peculiarity count for as 
much as possible, and (if he wants to win him at all) tries to make it appear 
attractive before him.

Now, is the third party to be insensible to the difference of the one from the 
other? Do they ask that of the free State or of humanity? Then these would 
have to be absolutely without self-interest, and incapable of taking an 
interest in any one whatever. Neither God (who divides his own from the 
wicked) nor the State (which knows how to separate good citizens from bad) was 
thought of as so indifferent. But they are looking for this very third party 
that bestows no more ``privilege.'' Then it is called perhaps the free 
State, or humanity, or whatever else it may be.

As Christian and Jew are ranked low by Bruno Bauer on account of their 
asserting privileges, it must be that they could and should free themselves 
from their narrow standpoint by self-renunciation or unselfishness. If they 
threw off their ``egoism,'' the mutual wrong would cease, and with it 
Christian and Jewish religiousness in general; it would be necessary only that 
neither of them should any longer want to be anything peculiar.

But, if they gave up this exclusiveness, with that the ground on which their 
hostilities were waged would in truth not yet be forsaken. In case of need 
they would indeed find a third thing on which they could unite, a ``general 
religion,'' a ``religion of humanity,'' etc.; in short, an equalization, 
which need not be better than that which would result if all Jews became 
Christians, by this likewise the ``privilege'' of one over the other would 
have an end. The \textit{tension}\footnote{[\textit{Spannung}]} would indeed 
be done away, but in this consisted not the essence of the two, but only their 
neighborhood. As being distinguished from each other they must necessarily be 
mutually resistant,\footnote{[\textit{gespannt}]} and the disparity will 
always remain. Truly it is not a failing in you that you 
stiffen\footnote{[\textit{spannen}]} yourself against me and assert your 
distinctness or peculiarity: you need not give way or renounce yourself.

People conceive the significance of the opposition too \textit{formally} and 
weakly when they want only to ``dissolve'' it in order to make room for a 
third thing that shall ``unite.'' The opposition deserves rather to be 
\textit{sharpened}. As Jew and Christian you are in too slight an opposition, 
and are contending only about religion, as it were about the emperor's beard, 
about a fiddlestick's end. Enemies in religion indeed, \textit{in the rest} 
you still remain good friends, and equal to each other, \textit{e.g.} as men. 
Nevertheless the rest too is unlike in each; and the time when you no longer 
merely \textit{dissemble} your opposition will be only when you entirely 
recognize it, and everybody asserts himself from top to toe as 
\textit{unique}.\footnote{[\textit{Einzig}]} Then the former opposition will 
assuredly be dissolved, but only because a stronger has taken it up into 
itself.

Our weakness consists not in this, that we are in opposition to others, but in 
this, that we are not completely so; that we are not entirely \textit{severed} 
from them, or that we seek a ``communion,'' a ``bond,'' that in communion 
we have an ideal. One faith, one God, one idea, one hat, for all! If all were 
brought under one hat, certainly no one would any longer need to take off his 
hat before another.

The last and most decided opposition, that of unique against unique, is at 
bottom beyond what is called opposition, but without having sunk back into 
``unity'' and unison. As unique you have nothing in common with the other 
any longer, and therefore nothing divisive or hostile either; you are not 
seeking to be in the right against him before a \textit{third} party, and are 
standing with him neither ``on the ground of right'' nor on any other common 
ground. The opposition vanishes in complete -- \textit{severance} or 
singleness.\footnote{[\textit{Einzigkeit}]} This might indeed be regarded as 
the new point in common or a new parity, but here the parity consists 
precisely in the disparity, and is itself nothing but disparity, a par of 
disparity, and that only for him who institutes a ``comparison.''

The polemic against privilege forms a characteristic feature of liberalism, 
which fumes against ``privilege'' because it itself appeals to ``right.'' 
Further than to fuming it cannot carry this; for privileges do not fall before 
right falls, as they are only forms of right. But right falls apart into its 
nothingness when it is swallowed up by might, \textit{i.e.} when one 
understands what is meant by ``Might goes before right.'' All right explains 
itself then as privilege, and privilege itself as power, as -- 
\textit{superior power}.

But must not the mighty combat against superior power show quite another face 
than the modest combat against privilege, which is to be fought out before a 
first judge, ``Right,'' according to the judge's mind?

\myhrule


Now, in conclusion, I have still to take back the half-way form of expression 
of which I was willing to make use only so long as I was still rooting among 
the entrails of right, and letting the word at least stand. But, in fact, with 
the concept the word too loses its meaning. What I called ``my right'' is no 
longer ``right'' at all, because right can be bestowed only by a spirit, be 
it the spirit of nature or that of the species, of mankind, the Spirit of God 
or that of His Holiness or His Highness, etc. What I have without an entitling 
spirit I have without right; I have it solely and alone through my power.

I do not demand any right, therefore I need not recognize any either. What I 
can get by force I get by force, and what I do not get by force I have no 
right to, nor do I give myself airs, or consolation, with my imprescriptible 
right.

With absolute right, right itself passes away; the dominion of the ``concept 
of right'' is canceled at the same time. For it is not to be forgotten that 
hitherto concepts, ideas, or principles ruled us, and that among these rulers 
the concept of right, or of justice, played one of the most important parts.

Entitled or unentitled -- that does not concern me, if I am only 
\textit{powerful}, I am of myself \textit{empowered}, and need no other 
empowering or entitling.

Right -- is a wheel in the head, put there by a spook; power -- that am I 
myself, I am the powerful one and owner of power. Right is above me, is 
absolute, and exists in one higher, as whose grace it flows to me: right is a 
gift of grace from the judge; power and might exist only in me the powerful 
and mighty.

\section[2. My Intercourse]{\centering 2. My Intercourse}

In society the human demand at most can be satisfied, while the egoistic must 
always come short. Because it can hardly escape anybody that the present shows 
no such living interest in any question as in the ``social,'' one has to 
direct his gaze especially to society. Nay, if the interest felt in it were 
less passionate and dazzled, people would not so much, in looking at society, 
lose sight of the individuals in it, and would recognize that a society cannot 
become new so long as those who form and constitute it remain the old ones. 
If, \textit{e.g.}, there was to arise in the Jewish people a society which 
should spread a new faith over the earth, these apostles could in no case 
remain Pharisees.

As you are, so you present yourself, so you behave toward men: a hypocrite as 
a hypocrite, a Christian as a Christian. Therefore the character of a society 
is determined by the character of its members: they are its creators. So much 
at least one must perceive even if one were not willing to put to the test the 
concept ``society'' itself.

Ever far from letting \textit{themselves} come to their full development and 
consequence, men have hitherto not been able to found their societies on 
\textit{themselves;} or rather, they have been able only to found 
``societies'' and to live in societies. The societies were always persons, 
powerful persons, so-called ``moral persons,'' \textit{i.e.} ghosts, before 
which the individual had the appropriate wheel in his head, the fear of 
ghosts. As such ghosts they may most suitably be designated by the respective 
names ``people'' and ``peoplet'': the people of the patriarchs, the people 
of the Hellenes, etc., at last the -- people of men, Mankind (Anacharsis 
Clootz was enthusiastic for the ``nation'' of mankind); then every 
subdivision of this ``people,'' which could and must have its special 
societies, the Spanish, French people, etc.; within it again classes, cities, 
in short all kinds of corporations; lastly, tapering to the finest point, the 
little peoplet of the --family. Hence, instead of saying that the person that 
walked as ghost in all societies hitherto has been the people, there might 
also have been named the two extremes -- to wit, either ``mankind'' or the 
``family,'' both the most ``natural-born units.'' We choose the word 
``people''\footnote{[\textit{Volk}; but the etymological remark following 
applies equally to the English word ``people.'' See Liddell \& Scott's Greek 
lexicon, under \textit{pimplemi}.]} because its derivation has been brought 
into connection with the Greek \textit{polloi}, the ``many'' or ``the 
masses,'' but still more because ``national efforts'' are at present the 
order of the day, and because even the newest mutineers have not yet shaken 
off this deceptive person, although on the other hand the latter consideration 
must give the preference to the expression ``mankind,'' since on all sides 
they are going in for enthusiasm over ``mankind.''

The people, then -- mankind or the family -- have hitherto, as it seems, 
played history: no \textit{egoistic} interest was to come up in these 
societies, but solely general ones, national or popular interests, class 
interests, family interests, and ``general human interests.'' But who has 
brought to their fall the peoples whose decline history relates? Who but the 
egoist, who was seeking \textit{his} satisfaction! If once an egoistic 
interest crept in, the society was ``corrupted'' and moved toward its 
dissolution, as Rome, \textit{e.g.} proves with its highly developed system 
of private rights, or Christianity with the incessantly-breaking-in 
``rational self-determination,'' ``self-consciousness,'' the ``autonomy 
of the spirit,'' etc.

The Christian people has produced two societies whose duration will keep equal 
measure with the permanence of that people: these are the societies 
\textit{State} and \textit{Church}. Can they be called a union of egoists? Do 
we in them pursue an egoistic, personal, own interest, or do we pursue a 
popular (\textit{i.e.} an interest of the Christian \textit{people}), to wit, 
a State, and Church interest? Can I and may I be myself in them? May I think 
and act as I will, may I reveal myself, live myself out, busy myself? Must I 
not leave untouched the majesty of the State, the sanctity of the Church?

Well, I may not do so as I will. But shall I find in any society such an 
unmeasured freedom of maying? Certainly no! Accordingly we might be content? 
Not a bit! It is a different thing whether I rebound from an ego or from a 
people, a generalization. There I am my opponent's opponent, born his equal; 
here I am a despised opponent, bound and under a guardian: there I stand man 
to man; here I am a schoolboy who can accomplish nothing against his comrade 
because the latter has called father and mother to aid and has crept under the 
apron, while I am well scolded as an ill-bred brat, and I must not 
``argue'': there I fight against a bodily enemy; here against mankind, 
against a generalization, against a ``majesty,'' against a spook. But to me 
no majesty, nothing sacred, is a limit; nothing that I know how to overpower. 
Only that which I cannot overpower still limits my might; and I of limited 
might am temporarily a limited I, not limited by the might \textit{outside} 
me, but limited by my \textit{own} still deficient might, by my \textit{own 
impotence}. However, ``the Guard dies, but does not surrender!'' Above all, 
only a bodily opponent!

\begin{quotation}

\noindent{} I dare meet every foeman\\
 Whom I can see and measure with my eye,\\
 mettle fires my mettle for the fight -- etc. \end{quotation}

\noindent{}Many privileges have indeed been cancelled with time, but solely 
for the sake of the common weal, of the State and the State's weal, by no 
means for the strengthening of me. Vassalage, \textit{e.g.}, was abrogated 
only that a single liege lord, the lord of the people, the monarchical power, 
might be strengthened: vassalage under the one became yet more rigorous 
thereby. Only in favor of the monarch, be he called ``prince'' or ``law,'' 
have privileges fallen. In France the citizens are not, indeed, vassals of the 
king, but are instead vassals of the ``law'' (the Charter). 
\textit{Subordination} was retained, only the Christian State recognized that 
man cannot serve two masters (the lord of the manor and the prince); therefore 
one obtained all the prerogatives; now he can again \textit{place} one above 
another, he can make ``men in high place.''

But of what concern to me is the common weal? The common weal as such is not 
\textit{my weal}, but only the furthest extremity of \textit{self- 
renunciation}. The common weal may cheer aloud while I must 
``down'';\footnote{[\textit{Kuschen}, a word whose only use is in ordering 
dogs to keep quiet.]} the State may shine while I starve. In what lies the 
folly of the political liberals but in their opposing the people to the 
government and talking of people's rights? So there is the people going to be 
of age, etc. As if one who has no mouth could be 
\textit{m\"undig}!\footnote{This is the word for ``of age''; but it is 
derived from \textit{Mund}, ``mouth,'' and refers properly to the right of 
speaking through one's own mouth, not by a guardian.]} Only the individual is 
able to be \textit{m\"undig}. Thus the whole question of the liberty of the 
press is turned upside down when it is laid claim to as a ``right of the 
people.'' It is only a right, or better the might, of the 
\textit{individual}. If a people has liberty of the press, then I, although in 
the midst of this people, have it not; a liberty of the people is not 
\textit{my} liberty, and the liberty of the press as a liberty of the people 
must have at its side a press law directed against \textit{me}.

This must be insisted on all around against the present-day efforts for 
liberty:

Liberty of the \textit{people} is not \textit{my} liberty!

Let us admit these categories, liberty of the people and right of the people: 
\textit{e.g.}, the right of the people that everybody may bear arms. Does one 
not forfeit such a right? One cannot forfeit his own right, but may well 
forfeit a right that belongs not to me but to the people. I may be locked up 
for the sake of the liberty of the people; I may, under sentence, incur the 
loss of the right to bear arms.

Liberalism appears as the last attempt at a creation of the liberty of the 
people, a liberty of the commune, of ``society,'' of the general, of 
mankind; the dream of a humanity, a people, a commune, a ``society,'' that 
shall be of age.

A people cannot be free otherwise than at the individual's expense; for it is 
not the individual that is the main point in this liberty, but the people. The 
freer the people, the more bound the individual; the Athenian people, 
precisely at its freest time, created ostracism, banished the atheists, 
poisoned the most honest thinker.

How they do praise Socrates for his conscientiousness, which makes him resist 
the advice to get away from the dungeon! He is a fool that he concedes to the 
Athenians a right to condemn him. Therefore it certainly serves him right; why 
then does he remain standing on an equal footing with the Athenians? Why does 
he not break with them? Had he known, and been able to know, what he was, he 
would have conceded to such judges no claim, no right. That \textit{he did not 
escape} was just his weakness, his delusion of still having something in 
common with the Athenians, or the opinion that he was a member, a mere member 
of this people. But he was rather this people itself in person, and could only 
be his own judge. There was no \textit{judge over him}, as he himself had 
really pronounced a public sentence on himself and rated himself worthy of the 
Prytaneum. He should have stuck to that, and, as he had uttered no sentence of 
death against himself, should have despised that of the Athenians too and 
escaped. But he subordinated himself and recognized in the \textit{people} his 
\textit{judge;} he seemed little to himself before the majesty of the people. 
That he subjected himself to \textit{might} (to which alone he could succumb) 
as to a ``right'' was treason against himself: it was \textit{virtue}. To 
Christ, who, it is alleged, refrained from using the power over his heavenly 
legions, the same scrupulousness is thereby ascribed by the narrators. Luther 
did very well and wisely to have the safety of his journey to Worms warranted 
to him in black and white, and Socrates should have known that the Athenians 
were his \textit{enemies}, he alone his judge. The self-deception of a 
``reign of law,'' etc., should have given way to the perception that the 
relation was a relation of \textit{might}.

It was with pettifoggery and intrigues that Greek liberty ended. Why? Because 
the ordinary Greeks could still less attain that logical conclusion which not 
even their hero of thought, Socrates, was able to draw. What then is 
pettifoggery but a way of utilizing something established without doing away 
with it? I might add ``for one's own advantage,'' but, you see, that lies in 
``utilizing.'' Such pettifoggers are the theologians who ``wrest'' and 
``force'' God's word; what would they have to wrest if it were not for the 
``established'' Word of God? So those liberals who only shake and wrest the 
``established order.'' They are all perverters, like those perverters of the 
law. Socrates recognized law, right; the Greeks constantly retained the 
authority of right and law. If with this recognition they wanted nevertheless 
to assert their advantage, every one his own, then they had to seek it in 
perversion of the law, or intrigue. Alcibiades, an intriguer of genius, 
introduces the period of Athenian ``decay''; the Spartan Lysander and others 
show that intrigue had become universally Greek. Greek \textit{law}, on which 
the Greek \textit{States} rested, had to be perverted and undermined by the 
egoists within these States, and the \textit{States} went down that the 
\textit{individuals} might become free, the Greek people fell because the 
individuals cared less for this people than for themselves. In general, all 
States, constitutions, churches, have sunk by the \textit{secession} of 
individuals; for the individual is the irreconcilable enemy of every 
\textit{generality}, every \textit{tie}, \textit{i.e.} every fetter. Yet 
people fancy to this day that man needs ``sacred ties'': he, the deadly 
enemy of every ``tie.'' The history of the world shows that no tie has yet 
remained unrent, shows that man tirelessly defends himself against ties of 
every sort; and yet, blinded, people think up new ties again and again, and 
think, \textit{e.g.}, that they have arrived at the right one if one puts upon 
them the tie of a so-called free constitution, a beautiful, constitutional 
tie; decoration ribbons, the ties of confidence between

``-- -- --,'' do seem gradually to have become somewhat infirm, but people 
have made no further progress than from apron-strings to garters and collars.

\textit{Everything sacred is a tie, a fetter}.

Everything sacred is and must be perverted by perverters of the law; therefore 
our present time has multitudes of such perverters in all spheres. They are 
preparing the way for the break-up of law, for lawlessness.

Poor Athenians who are accused of pettifoggery and sophistry! poor Alcibiades, 
of intrigue! Why, that was just your best point, your first step in freedom. 
Your \AE{}eschylus, Herodotus, etc., only wanted to have a free Greek 
\textit{people;} you were the first to surmise something of \textit{your} 
freedom.

A people represses those who tower above \textit{its majesty}, by ostracism 
against too-powerful citizens, by the Inquisition against the heretics of the 
Church, by the -- Inquisition against traitors in the State.

For the people is concerned only with its self-assertion; it demands 
``patriotic self-sacrifice'' from everybody. To it, accordingly, every one 
\textit{in himself} is indifferent, a nothing, and it cannot do, not even 
suffer, what the individual and he alone must do -- to wit, \textit{turn him 
to account}. Every people, every State, is unjust toward the \textit{egoist}.

As long as there still exists even one institution which the individual may 
not dissolve, the ownness and self-appurtenance of Me is still very remote. 
How can I, \textit{e.g.} be free when I must bind myself by oath to a 
constitution, a charter, a law, ``vow body and soul'' to my people? How can 
I be my own when my faculties may develop only so far as they ``do not 
disturb the harmony of society'' (Weitling)?

The fall of peoples and mankind will invite \textit{me} to my rise.

Listen, even as I am writing this, the bells begin to sound, that they may 
jingle in for tomorrow the festival of the thousand years' existence of our 
dear Germany. Sound, sound its knell! You do sound solemn enough, as if your 
tongue was moved by the presentiment that it is giving convoy to a corpse. The 
German people and German peoples have behind them a history of a thousand 
years: what a long life! O, go to rest, never to rise again -- that all may 
become free whom you so long have held in fetters. -- The \textit{people} is 
dead. -- Up with \textit{me}!

O thou my much-tormented German people -- what was thy torment? It was the 
torment of a thought that cannot create itself a body, the torment of a 
walking spirit that dissolves into nothing at every cock-crow and yet pines 
for deliverance and fulfillment. In me too thou hast lived long, thou dear -- 
thought, thou dear -- spook. Already I almost fancied I had found the word of 
thy deliverance, discovered flesh and bones for the wandering spirit; then I 
hear them sound, the bells that usher thee into eternal rest; then the last 
hope fades out, then the notes of the last love die away, then I depart from 
the desolate house of those who now are dead and enter at the door of the -- 
living one:

\begin{quotation}

\noindent{}For only he who is alive is in the right.\end{quotation}

\noindent{}Farewell, thou dream of so many millions; farewell, thou who hast 
tyrannized over thy children for a thousand years!

Tomorrow they carry thee to the grave; soon thy sisters, the peoples, will 
follow thee. But, when they have all followed, then -- -- mankind is buried, 
and I am my own, I am the laughing heir!

\myhrule


The word \textit{Gesellschaft} (society) has its origin in the word 
\textit{Sal} (hall). If one hall encloses many persons, then the hall causes 
these persons to be in society. They \textit{are} in society, and at most 
constitute a parlor-society by talking in the traditional forms of parlor 
speech. When it comes to real \textit{intercourse}, this is to be regarded as 
independent of society: it may occur or be lacking, without altering the 
nature of what is named society. Those who are in the hall are a society even 
as mute persons, or when they put each other off solely with empty phrases of 
courtesy. Intercourse is mutuality, it is the action, the \textit{commercium}, 
of individuals; society is only community of the hall, and even the statues of 
a museum-hall are in society, they are ``grouped.'' People are accustomed to 
say ``they \textit{haben inne}\footnote{[''Occupy``; literally, ''have 
within``.]} this hall in common,'' but the case is rather that the hall has 
us \textit{inne} or in it. So far the natural signification of the word 
society. In this it comes out that society is not generated by me and you, but 
by a third factor which makes associates out of us two, and that it is just 
this third factor that is the creative one, that which creates society.

Just so a prison society or prison companionship (those who 
enjoy\footnote{[The word \textit{Genosse}, ``companion,'' signifies 
originally a companion in enjoyment.]} the same prison). Here we already hit 
upon a third factor fuller of significance than was that merely local one, the 
hall. Prison no longer means a space only, but a space with express reference 
to its inhabitants: for it is a prison only through being destined for 
prisoners, without whom it would be a mere building. What gives a common stamp 
to those who are gathered in it? Evidently the prison, since it is only by 
means of the prison that they are prisoners. What, then, determines the 
\textit{manner} of life of the prison society? The prison! What determines 
their intercourse? The prison too, perhaps? Certainly they can enter upon 
intercourse only as prisoners, \textit{i.e.} only so far as the prison laws 
allow it; but that \textit{they themselves} hold intercourse, I with you, this 
the prison cannot bring to pass; on the contrary, it must have an eye to 
guarding against such egoistic, purely personal intercourse (and only as such 
is it really intercourse between me and you). That we \textit{jointly} execute 
a job, run a machine, effectuate anything in general -- for this a prison will 
indeed provide; but that I forget that I am a prisoner, and engage in 
intercourse with you who likewise disregard it, brings danger to the prison, 
and not only cannot be caused by it, but must not even be permitted. For this 
reason the saintly and moral-minded French chamber decides to introduce 
solitary confinement, and other saints will do the like in order to cut off 
``demoralizing intercourse.'' Imprisonment is the established and -- sacred 
condition, to injure which no attempt must be made. The slightest push of that 
kind is punishable, as is every uprising against a sacred thing by which man 
is to be charmed and chained.

Like the hall, the prison does form a society, a companionship, a communion 
(\textit{e.g.} communion of labor), but no \textit{intercourse}, no 
reciprocity, no \textit{union}. On the contrary, every union in the prison 
bears within it the dangerous seed of a ``plot,'' which under favorable 
circumstances might spring up and bear fruit.

Yet one does not usually enter the prison voluntarily, and seldom remains in 
it voluntarily either, but cherishes the egoistic desire for liberty. Here, 
therefore, it sooner becomes manifest that personal intercourse is in hostile 
relations to the prison society and tends to the dissolution of this very 
society, this joint incarceration.

Let us therefore look about for such communions as, it seems, we remain in 
gladly and voluntarily, without wanting to endanger them by our egoistic 
impulses.

As a communion of the required sort the \textit{family} offers itself in the 
first place. Parents, husbands and wife, children, brothers and sisters, 
represent a whole or form a family, for the further widening of which the 
collateral relatives also may be made to serve if taken into account. The 
family is a true communion only when the law of the family, 
piety\footnote{[This word in German does not mean religion, but, as in Latin, 
faithfulness to family ties -- as we speak of ``filial piety.'' But the word 
elsewhere translated ``pious'' [\textit{fromm}] means ``religious,'' as 
usually in English.]} or family love, is observed by its members. A son to 
whom parents, brothers, and sisters have become indifferent \textit{has been} 
a son; for, as the sonship no longer shows itself efficacious, it has no 
greater significance than the long-past connection of mother and child by the 
navel-string. That one has once lived in this bodily juncture cannot as a fact 
be undone; and so far one remains irrevocably this mother's son and the 
brother of the rest of her children; but it would come to a lasting connection 
only by lasting piety, this spirit of the family. Individuals are members of a 
family in the full sense only when they make the \textit{persistence} of the 
family their task; only as \textit{conservative} do they keep aloof from 
doubting their basis, the family. To every member of the family one thing must 
be fixed and sacred -- \textit{viz}., the family itself, or, more 
expressively, piety. That the family is to \textit{persist} remains to its 
member, so long as he keeps himself free from that egoism which is hostile to 
the family, an unassailable truth. In a word: -- If the family is sacred, then 
nobody who belongs to it may secede from it; else he becomes a ``criminal'' 
against the family: he may never pursue an interest hostile to the family, 
\textit{e.g.} form a misalliance. He who does this has ``dishonored the 
family,'' ``put it to shame,'' etc.

Now, if in an individual the egoistic impulse has not force enough, he 
complies and makes a marriage which suits the claims of the family, takes a 
rank which harmonizes with its position, etc.; in short, he ``does honor to 
the family.''

If, on the contrary, the egoistic blood flows fierily enough in his veins, he 
prefers to become a ``criminal'' against the family and to throw off its 
laws.

Which of the two lies nearer my heart, the good of the family or my good? In 
innumerable cases both go peacefully together; the advantage of the family is 
at the same time mine, and \textit{vice versa}. Then it is hard to decide 
whether I am thinking \textit{selfishly} or \textit{for the common benefit}, 
and perhaps I complacently flatter myself with my unselfishness. But there 
comes the day when a necessity of choice makes me tremble, when I have it in 
mind to dishonor my family tree, to affront parents, brothers, and kindred. 
What then? Now it will appear how I am disposed at the bottom of my heart; now 
it will be revealed whether piety ever stood above egoism for me, now the 
selfish one can no longer skulk behind the semblance of unselfishness. A wish 
rises in my soul, and, growing from hour to hour, becomes a passion. To whom 
does it occur at first blush that the slightest thought which may result 
adversely to the spirit of the family (piety) bears within it a transgression 
against this? Nay, who at once, in the first moment, becomes completely 
conscious of the matter? It happens so with Juliet in ``Romeo and Juliet.'' 
The unruly passion can at last no longer be tamed, and undermines the building 
of piety. You will say, indeed, it is from self-will that the family casts out 
of its bosom those wilful ones that grant more of a hearing to their passion 
than to piety; the good Protestants used the same excuse with much success 
against the Catholics, and believed in it themselves. But it is just a 
subterfuge to roll the fault off oneself, nothing more. The Catholics had 
regard for the common bond of the church, and thrust those heretics from them 
only because these did not have so much regard for the bond of the church as 
to sacrifice their convictions to it; the former, therefore, held the bond 
fast, because the bond, the Catholic (\textit{i.e.} common and united) church, 
was sacred to them; the latter, on the contrary, disregarded the bond. Just so 
those who lack piety. They are not thrust out, but thrust themselves out, 
prizing their passion, their wilfulness, higher than the bond of the family.

But now sometimes a wish glimmers in a less passionate and wilful heart than 
Juliet's. The pliable girl brings herself as a \textit{sacrifice} to the peace 
of the family. One might say that here too selfishness prevailed, for the 
decision came from the feeling that the pliable girl felt herself more 
satisfied by the unity of the family than by the fulfillment of her wish. That 
might be; but what if there remained a sure sign that egoism had been 
sacrificed to piety? What if, even after the wish that had been directed 
against the peace of the family was sacrificed, it remained at least as a 
recollection of a ``sacrifice'' brought to a sacred tie? What if the pliable 
girl were conscious of having left her self-will unsatisfied and humbly 
subjected herself to a higher power? Subjected and sacrificed, because the 
superstition of piety exercised its dominion over her!

There egoism won, here piety wins and the egoistic heart bleeds; there egoism 
was strong, here it was -- weak. But the weak, as we have long known, are the -- unselfish. For them, for these its weak members, the family cares, because 
they \textit{belong} to the family, do not belong to themselves and care for 
themselves. This weakness Hegel, \textit{e.g.} praises when he wants to have 
match- making left to the choice of the parents.

As a sacred communion to which, among the rest, the individual owes obedience, 
the family has the judicial function too vested in it; such a ``family 
court'' is described \textit{e.g.} in the \textit{Cabanis} \textit{}of 
Wilibald Alexis. There the father, in the name of the ``family council,'' 
puts the intractable son among the soldiers and thrusts him out of the family, 
in order to cleanse the smirched family again by means of this act of 
punishment. -- The most consistent development of family responsibility is 
contained in Chinese law, according to which the whole family has to expiate 
the individual's fault.

Today, however, the arm of family power seldom reaches far enough to take 
seriously in hand the punishment of apostates (in most cases the State 
protects even against disinheritance). The criminal against the family 
(family-criminal) flees into the domain of the State and is free, as the 
State-criminal who gets away to America is no longer reached by the 
punishments of his State. He who has shamed his family, the graceless son, is 
protected against the family's punishment because the State, this protecting 
lord, takes away from family punishment its ``sacredness'' and profanes it, 
decreeing that it is only --``revenge'': it restrains punishment, this 
sacred family right, because before its, the State's, ``sacredness'' the 
subordinate sacredness of the family always pales and loses its sanctity as 
soon as it comes in conflict with this higher sacredness. Without the 
conflict, the State lets pass the lesser sacredness of the family; but in the 
opposite case it even commands crime against the family, charging, \textit{e. 
g.}, the son to refuse obedience to his parents as soon as they want to 
beguile him to a crime against the State.

Well, the egoist has broken the ties of the family and found in the State a 
lord to shelter him against the grievously affronted spirit of the family. But 
where has he run now? Straight into a new \textit{society}, in which his 
egoism is awaited by the same snares and nets that it has just escaped. For 
the State is likewise a society, not a union; it is the broadened 
\textit{family} (``Father of the Country -- Mother of the Country -- children 
of the country'').

\myhrule


What is called a State is a tissue and plexus of dependence and adherence; it 
is a \textit{belonging together}, a holding together, in which those who are 
placed together fit themselves to each other, or, in short, mutually depend on 
each other: it is the \textit{order} of this \textit{dependence}. Suppose the 
king, whose authority lends authority to all down to the beadle, should 
vanish: still all in whom the will for order was awake would keep order erect 
against the disorders of bestiality. If disorder were victorious, the State 
would be at an end.

But is this thought of love, to fit ourselves to each other, to adhere to each 
other and depend on each other, really capable of winning us? According to 
this the State should be \textit{love} realized, the being for each other and 
living for each other of all. Is not self-will being lost while we attend to 
the will for order? Will people not be satisfied when order is cared for by 
authority, \textit{i.e.} when authority sees to it that no one ``gets in the 
way of'' another; when, then, the \textit{herd} is judiciously distributed or 
ordered? Why, then everything is in ``the best order,'' and it is this best 
order that is called -- State!

Our societies and States \textit{are} without our \textit{making} them, are 
united without our uniting, are predestined and established, or have an 
independent standing\footnote{[It should be remembered that the words 
``establish'' and ``State'' are both derived from the root ``stand.'']} 
of their own, are the indissolubly established against us egoists. The fight 
of the world today is, as it is said, directed against the ``established.'' 
Yet people are wont to misunderstand this as if it were only that what is now 
established was to be exchanged for another, a better, established system. But 
war might rather be declared against establishment itself, the \textit{State}, 
not a particular State, not any such thing as the mere condition of the State 
at the time; it is not another State (\textit{e.g.} a ``people's State'') 
that men aim at, but their \textit{union}, uniting, this ever-fluid uniting of 
everything standing. -- A State exists even without my co-operation: I am born 
in it, brought up in it, under obligations to it, and must ``do it 
homage.''\footnote{[\textit{huldigen}]} It takes me up into its 
``favor,''\footnote{[\textit{Huld}]} and I live by its ``grace.'' Thus the 
independent establishment of the State founds my lack of independence; its 
condition as a ``natural growth,'' its organism, demands that my nature do 
not grow freely, but be cut to fit it. That \textit{it} may be able to unfold 
in natural growth, it applies to me the shears of ``civilization''; it gives 
me an education and culture adapted to it, not to me, and teaches me 
\textit{e.g.} to respect the laws, to refrain from injury to State property 
(\textit{i.e.} private property), to reverence divine and earthly highness, 
etc.; in short, it teaches me to be -- \textit{unpunishable}, 
``sacrificing'' my ownness to ``sacredness'' (everything possible is 
sacred; \textit{e.g.} property, others' life, etc.). In this consists the 
sort of civilization and culture that the State is able to give me: it brings 
me up to be a ``serviceable instrument,'' a ``serviceable member of 
society.''

This every State must do, the people's State as well as the absolute or 
constitutional one. It must do so as long as we rest in the error that it is 
an \textit{I}, as which it then applies to itself the name of a ``moral, 
mystical, or political person.'' I, who really am I, must pull off this 
lion-skin of the I from the stalking thistle-eater. What manifold robbery have 
I not put up with in the history of the world! There I let sun, moon, and 
stars, cats and crocodiles, receive the honor of ranking as I; there Jehovah, 
Allah, and Our Father came and were invested with the I; there families, 
tribes, peoples, and at last actually mankind, came and were honored as I's; 
there the Church, the State, came with the pretension to be I -- and I gazed 
calmly on all. What wonder if then there was always a real I too that joined 
the company and affirmed in my face that it was not my \textit{you} but my 
real \textit{I}. Why, \textit{the} Son of Man \textit{par excellence} had done 
the like; why should not \textit{a} son of man do it too? So I saw my I always 
above me and outside me, and could never really come to myself.

I never believed in myself; I never believed in my present, I saw myself only 
in the future. The boy believes he will be a proper I, a proper fellow, only 
when he has become a man; the man thinks, only in the other world will he be 
something proper. And, to enter more closely upon reality at once, even the 
best are today still persuading each other that one must have received into 
himself the State, his people, mankind, and what not, in order to be a real I, 
a ``free burgher,'' a ``citizen,'' a ``free or true man''; they too see 
the truth and reality of me in the reception of an alien I and devotion to it. 
And what sort of an I? An I that is neither an I nor a you, a \textit{fancied} 
I, a spook.

While in the Middle Ages the church could well brook many States living united 
in it, the States learned after the Reformation, especially after the Thirty 
Years' War, to tolerate many churches (confessions) gathering under one crown. 
But all States are religious and, as the case may be, ``Christian States,'' 
and make it their task to force the intractable, the ``egoists,'' under the 
bond of the unnatural, \textit{e.g.}, Christianize them. All arrangements of 
the Christian State have the object of \textit{Christianizing the people}. 
Thus the court has the object of forcing people to justice, the school that of 
forcing them to mental culture -- in short, the object of protecting those who 
act Christianly against those who act un-Christianly, of bringing Christian 
action to \textit{dominion}, of making it \textit{powerful}. Among these means 
of force the State counted the \textit{Church} too, it demanded a -- 
particular religion from everybody. Dupin said lately against the clergy, 
``Instruction and education belong to the State.''

Certainly everything that regards the principle of morality is a State affair. 
Hence it is that the Chinese State meddles so much in family concerns, and one 
is nothing there if one is not first of all a good child to his parents. 
Family concerns are altogether State concerns with us too, only that our State -- puts confidence in the families without painful oversight; it holds the 
family bound by the marriage tie, and this tie cannot be broken without it.

But that the State makes me responsible for my principles, and demands certain 
ones from me, might make me ask, what concern has it with the ``wheel in my 
head'' (principle)? Very much, for the State is the -- \textit{ruling 
principle}. It is supposed that in divorce matters, in marriage law in 
general, the question is of the proportion of rights between Church and 
States. Rather, the question is of whether anything sacred is to rule over 
man, be it called faith or ethical law (morality). The State behaves as the 
same ruler that the Church was. The latter rests on godliness, the former on 
morality.

People talk of the tolerance, the leaving opposite tendencies free, etc., by 
which civilized States are distinguished. Certainly some are strong enough to 
look with complacency on even the most unrestrained meetings, while others 
charge their catchpolls to go hunting for tobacco-pipes. Yet for one State as 
for another the play of individuals among themselves, their buzzing to and 
fro, their daily life, is an \textit{incident} which it must be content to 
leave to themselves because it can do nothing with this. Many, indeed, still 
strain out gnats and swallow camels, while others are shrewder. Individuals 
are ``freer'' in the latter, because less pestered. But \textit{I} am free 
in \textit{no} State. The lauded tolerance of States is simply a tolerating of 
the ``harmless,'' the ``not dangerous''; it is only elevation above 
pettymindedness, only a more estimable, grander, prouder -- despotism. A 
certain State seemed for a while to mean to be pretty well elevated above 
\textit{literary} combats, which might be carried on with all heat; England is 
elevated above \textit{popular turmoil} and -- tobacco-smoking. But woe to the 
literature that deals blows at the State itself, woe to the mobs that 
``endanger'' the State. In that certain State they dream of a ``free 
science,'' in England of a ``free popular life.''

The State does let individuals \textit{play} as freely as possible, only they 
must not be in \textit{earnest}, must not forget \textit{it}. Man must not 
carry on intercourse with man \textit{unconcernedly}, not without ``superior 
oversight and mediation.'' I must not execute all that I am able to, but only 
so much as the State allows; I must not turn to account \textit{my} thoughts, 
nor \textit{my} work, nor, in general, anything of mine.

The State always has the sole purpose to limit, tame, subordinate, the 
individual -- to make him subject to some \textit{generality} or other; it 
lasts only so long as the individual is not all in all, and it is only the 
clearly-marked \textit{restriction of me}, my limitation, my slavery. Never 
does a State aim to bring in the free activity of individuals, but always that 
which is bound to the \textit{purpose of the State}. Through the State nothing 
\textit{in common} comes to pass either, as little as one can call a piece of 
cloth the common work of all the individual parts of a machine; it is rather 
the work of the whole machine as a unit, \textit{machine work}. In the same 
style everything is done by the \textit{State machine} too; for it moves the 
clockwork of the individual minds, none of which follow their own impulse. The 
State seeks to hinder every free activity by its censorship, its supervision, 
its police, and holds this hindering to be its duty, because it is in truth a 
duty of self-preservation. The State wants to make something out of man, 
therefore there live in it only \textit{made} men; every one who wants to be 
his own self is its opponent and is nothing. ``He is nothing'' means as much 
as, the State does not make use of him, grants him no position, no office, no 
trade, etc.

Edgar Bauer,\footnote{What was said in the concluding remarks after Humane 
Liberalism holds good of the following -- to wit, that it was likewise written 
immediately after the appearance of the book cited.} in the \textit{Liberale 
Bestrebungen} (vol. II, p.50), is still dreaming of a ``government which, 
proceeding out of the people, can never stand in opposition to it.'' He does 
indeed (p.69) himself take back the word ``government'': ``In the republic 
no government at all obtains, but only an executive authority. An authority 
which proceeds purely and alone out of the people; which has not an 
independent power, independent principles, independent officers, over against 
the people; but which has its foundation, the fountain of its power and of its 
principles, in the sole, supreme authority of the State, in the people. The 
concept government, therefore, is not at all suitable in the people's 
State.'' But the thing remains the same. That which has ``proceeded, been 
founded, sprung from the fountain'' becomes something ``independent'' and, 
like a child delivered from the womb, enters upon opposition at once. The 
government, if it were nothing independent and opposing, would be nothing at 
all.

``In the free State there is no government,'' etc. (p.94). This surely means 
that the people, when it is the \textit{sovereign}, does not let itself be 
conducted by a superior authority. Is it perchance different in absolute 
monarchy? Is there \textit{there} for the \textit{sovereign}, perchance, a 
government standing over him? \textit{Over} the sovereign, be he called prince 
or people, there never stands a government: that is understood of itself. But 
over \textit{me} there will stand a government in every ``State,'' in the 
absolute as well as in the republican or ``free.'' I am as badly off in one 
as in the other.

The republic is nothing whatever but -- absolute monarchy; for it makes no 
difference whether the monarch is called prince or people, both being a 
``majesty.'' Constitutionalism itself proves that nobody is able and willing 
to be only an instrument. The ministers domineer over their master the prince, 
the deputies over their master the people. Here, then, the \textit{parties} at 
least are already free -- \textit{videlicet}, the office-holders' party 
(so-called people's party). The prince must conform to the will of the 
ministers, the people dance to the pipe of the chambers. Constitutionalism is 
further than the republic, because it is the \textit{State} in incipient 
\textit{dissolution}.

Edgar Bauer denies (p.56) that the people is a ``personality'' in the 
constitutional State; \textit{per contra}, then, in the republic? Well, in the 
constitutional State the people is -- a \textit{party}, and a party is surely 
a ``personality'' if one is once resolved to talk of a ``political'' 
(p.76) moral person anyhow. The fact is that a moral person, be it called 
people's party or people or even ``the Lord,'' is in no wise a person, but a 
spook.

Further, Edgar Bauer goes on (p.69): ``guardianship is the characteristic of 
a government.'' Truly, still more that of a people and ``people's State''; 
it is the characteristic of all \textit{dominion}. A people's State, which 
``unites in itself all completeness of power,'' the ``absolute master,'' 
cannot let me become powerful. And what a chimera, to be no longer willing to 
call the ``people's officials'' ``servants, instruments,'' because they 
``execute the free, rational law-will of the people!'' (p.73). He thinks 
(p.74): ``Only by all official circles subordinating themselves to the 
government's views can unity be brought into the State''; but his ``people's 
State'' is to have ``unity'' too; how will a lack of subordination be 
allowed there? subordination to the -- people's will.

``In the constitutional State it is the regent and his \textit{disposition} 
that the whole structure of government rests on in the end.'' (p. 130.) How 
would that be otherwise in the ``people's State''? Shall \textit{I} not 
there be governed by the people's \textit{disposition} too, and does it make a 
difference \textit{for me} whether I see myself kept in dependence by the 
prince's disposition or by the people's disposition, so-called ``public 
opinion''? If dependence means as much as ``religious relation,'' as Edgar 
Bauer rightly alleges, then in the people's State the people remains 
\textit{for me} the superior power, the ``majesty'' (for God and prince have 
their proper essence in ``majesty'') to which I stand in religious 
relations. -- Like the sovereign regent, the sovereign people too would be 
reached by no \textit{law}. Edgar Bauer's whole attempt comes to a 
\textit{change of masters}. Instead of wanting to make the \textit{people} 
free, he should have had his mind on the sole realizable freedom, his own.

In the constitutional State \textit{absolutism} itself has at last come in 
conflict with itself, as it has been shattered into a duality; the government 
wants to be absolute, and the people wants to be absolute. These two absolutes 
will wear out against each other.

Edgar Bauer inveighs against the determination of the regent \textit{by 
birth}, by \textit{chance}. But, when ``the people'' have become ``the sole 
power in the State'' (p. 132), have \textit{we} not then in it a master from 
\textit{chance?} Why, what is the people? The people has always been only the 
\textit{body} of the government: it is many under one hat (a prince's hat) or 
many under one constitution. And the constitution is the -- prince. Princes 
and peoples will persist so long as both do not \textit{col}lapse, \textit{i.e.}, 
fall \textit{together}. If under one constitution there are many 
``peoples'' -- as in the ancient Persian monarchy and today --then these 
``peoples'' rank only as ``provinces.'' For me the people is in any case 
an --accidental power, a force of nature, an enemy that I must overcome.

What is one to think of under the name of an ``organized'' people (p. 132)? 
A people ``that no longer has a government,'' that governs itself. In which, 
therefore, no ego stands out prominently; a people organized by ostracism. The 
banishment of egos, ostracism, makes the people autocrat.

If you speak of the people, you must speak of the prince; for the people, if 
it is to be a subject\footnote{[In the philosophical sense [a thinking and 
acting being] not in the political sense.]} and make history, must, like 
everything that acts, have a \textit{head}, its ``supreme head.'' Weitling 
sets this forth in [\textit{Die Europ\"aische}] Triarchie, and Proudhon 
declares, \textit{``une soci\'et\'e, pour ainsi dire ac\'ephale, ne peut 
vivre}.''\footnote{[\textit{``Cr\'eation de l'Ordre},'' p.485.]}

The \textit{vox populi} is now always held up to us, and ``public opinion'' 
is to rule our princes. Certainly the \textit{vox populi} is at the same time 
\textit{vox dei;} but is either of any use, and is not the \textit{vox 
principis} also \textit{vox dei}?

At this point the ``Nationals'' may be brought to mind. To demand of the 
thirty-eight States of Germany that they shall act as \textit{one nation} can 
only be put alongside the senseless desire that thirty-eight swarms of bees, 
led by thirty-eight queen-bees, shall unite themselves into one swarm. 
\textit{Bees} they all remain; but it is not the bees as bees that belong 
together and can join themselves together, it is only that the 
\textit{subject} bees are connected with the \textit{ruling} queens. Bees and 
peoples are destitute of will, and the \textit{instinct} of their queens leads 
them.

If one were to point the bees to their beehood, in which at any rate they are 
all equal to each other, one would be doing the same thing that they are now 
doing so stormily in pointing the Germans to their Germanhood. Why, Germanhood 
is just like beehood in this very thing, that it bears in itself the necessity 
of cleavages and separations, yet without pushing on to the last separation, 
where, with the complete carrying through of the process of separating, its 
end appears: I mean, to the separation of man from man. Germanhood does indeed 
divide itself into different peoples and tribes, \textit{i.e.} beehives; but 
the individual who has the quality of being a German is still as powerless as 
the isolated bee. And yet only individuals can enter into union with each 
other, and all alliances and leagues of peoples are and remain mechanical 
compoundings, because those who come together, at least so far as the 
``peoples'' are regarded as the ones that have come together, are 
\textit{destitute of will}. Only with the last separation does separation 
itself end and change to unification.

Now the Nationals are exerting themselves to set up the abstract, lifeless 
unity of beehood; but the self-owned are going to fight for the unity willed 
by their own will, for union. This is the token of all reactionary wishes, 
that they want to set up something \textit{general}, abstract, an empty, 
lifeless \textit{concept}, in distinction from which the self-owned aspire to 
relieve the robust, lively \textit{particular} from the trashy burden of 
generalities. The reactionaries would be glad to smite a \textit{people, a 
nation}, forth from the earth; the self-owned have before their eyes only 
themselves. In essentials the two efforts that are just now the order of the 
day - to wit, the restoration of provincial rights and of the old tribal 
divisions (Franks, Bavarians, Lusatia, etc.), and the restoration of the 
entire nationality -- coincide in one. But the Germans will come into unison, 
\textit{i.e.} unite \textit{themselves}, only when they knock over their 
beehood as well as all the beehives; in other words, when they are more than -- Germans: only then can they form a ``German Union.'' They must not want 
to turn back into their nationality, into the womb, in order to be born again, 
but let every one turn in to \textit{himself}. How ridiculously sentimental 
when one German grasps another's hand and presses it with sacred awe because 
``he too is a German!'' With that he is something great! But this will 
certainly still be thought touching as long as people are enthusiastic for 
``brotherliness,'' \textit{i.e.} as long as they have a \textit{``family 
disposition''}. From the superstition of ``piety,'' from 
``brotherliness'' or ``childlikeness'' or however else the soft-hearted 
piety-phrases run -- from the \textit{family spirit} -- the Nationals, who 
want to have a great \textit{family of Germans}, cannot liberate themselves.

Aside from this, the so-called Nationals would only have to understand 
themselves rightly in order to lift themselves out of their juncture with the 
good-natured Teutomaniacs. For the uniting for material ends and interests, 
which they demand of the Germans, comes to nothing else than a voluntary 
union. Carri\`ere, inspired, cries out,\footnote{[\textit{``K\"olner Dom},'' 
p. 4.]} ``Railroads are to the more penetrating eye the way to a \textit{life 
of the people} \textit{e.g.} has not yet anywhere appeared in such 
significance.'' Quite right, it will be a life of the people that has nowhere 
appeared, because it is not a -- life of the people. -- So Carri\`ere then 
combats himself (p. 10): ``Pure humanity or manhood cannot be better 
represented than by a people fulfilling its mission.'' Why, by this 
nationality only is represented. ``Washed-out generality is lower than the 
form complete in itself, which is itself a whole, and lives as a living member 
of the truly general, the organized.'' Why, the people is this very 
``washed-out generality,'' and it is only a man that is the ``form complete 
in itself.''

The impersonality of what they call ``people, nation,'' is clear also from 
this: that a people which wants to bring its I into view to the best of its 
power puts at its head the ruler \textit{without will}. It finds itself in the 
alternative either to be subjected to a prince who realizes only 
\textit{himself, his individual pleasure --} then it does not recognize in the 
``absolute master'' its own will, the so-called will of the people -- or to 
seat on the throne a prince who gives effect to no will of his own -- then it 
has a prince \textit{without will}, whose place some ingenious clockwork would 
perhaps fill just as well. -- Therefore insight need go only a step farther; 
then it becomes clear of itself that the I of the people is an impersonal, 
``spiritual'' power, the -- law. The people's I, therefore, is a -- spook, 
not an I. I am I only by this, that I make myself; \textit{i.e.} that it is 
not another who makes me, but I must be my own work. But how is it with this I 
of the people? \textit{Chance} plays it into the people's hand, chance gives 
it this or that born lord, accidents procure it the chosen one; he is not its 
(the \textit{``sovereign''} people's) product, as I am \textit{my} product. 
Conceive of one wanting to talk you into believing that you were not your I, 
but Tom or Jack was your I! But so it is with the people, and rightly. For the 
people has an I as little as the eleven planets counted together have an I, 
though they revolve around a common \textit{center}.

Bailly's utterance is representative of the slave-disposition that folks 
manifest before the sovereign people, as before the prince. ``I have,'' says 
he, ``no longer any extra reason when the general reason has pronounced 
itself. My first law was the nation's will; as soon as it had assembled I knew 
nothing beyond its sovereign will.'' He would have no ``extra reason,'' and 
yet this extra reason alone accomplishes everything. Just so Mirabeau inveighs 
in the words, ``No power on earth has the \textit{right} to say to the 
nation's representatives, It is my will!''

As with the Greeks, there is now a wish to make man a \textit{zoon politicon}, 
a citizen of the State or political man. So he ranked for a long time as a 
``citizen of heaven.'' But the Greek fell into ignominy along with his 
\textit{State}, the citizen of heaven likewise falls with heaven; we, on the 
other hand, are not willing to go down along with the \textit{people}, the 
nation and nationality, not willing to be merely \textit{political} men or 
politicians. Since the Revolution they have striven to ``make the people 
happy,'' and in making the people happy, great, etc., they make us unhappy: 
the people's good hap is -- my mishap.

What empty talk the political liberals utter with emphatic decorum is well 
seen again in Nauwerck's ``On Taking Part in the State''. There complaint is 
made of those who are indifferent and do not take part, who are not in the 
full sense citizens, and the author speaks as if one could not be man at all 
if one did not take a lively part in State affairs, \textit{i.e.} if one were 
not a politician. In this he is right; for, if the State ranks as the warder 
of everything ``human,'' we can have nothing human without taking part in 
it. But what does this make out against the egoist? Nothing at all, because 
the egoist is to himself the warder of the human, and has nothing to say to 
the State except ``Get out of my sunshine.'' Only when the State comes in 
contact with his ownness does the egoist take an active interest in it. If the 
condition of the State does not bear hard on the closet-philosopher, is he to 
occupy himself with it because it is his ``most sacred duty?'' So long as 
the State does according to his wish, what need has he to look up from his 
studies? Let those who from an interest of their own want to have conditions 
otherwise busy themselves with them. Not now, nor evermore, will ``sacred 
duty'' bring folks to reflect about the State -- as little as they become 
disciples of science, artists, etc., from ``sacred duty.'' Egoism alone can 
impel them to it, and will as soon as things have become much worse. If you 
showed folks that their egoism demanded that they busy themselves with State 
affairs, you would not have to call on them long; if, on the other hand, you 
appeal to their love of fatherland etc., you will long preach to deaf hearts 
in behalf of this ``service of love.'' Certainly, in your sense the egoists 
will not participate in State affairs at all.

Nauwerck utters a genuine liberal phrase on p. 16: ``Man completely fulfills 
his calling only in feeling and knowing himself as a member of humanity, and 
being active as such. The individual cannot realize the idea of 
\textit{manhood} if he does not stay himself upon all humanity, if he does not 
draw his powers from it like Antaeus.''

In the same place it is said: ``Man's relation to the \textit{res publica} is 
degraded to a purely private matter by the theological view; is, accordingly, 
made away with by denial.'' As if the political view did otherwise with 
religion! There religion is a ``private matter.''

If, instead of ``sacred duty,'' ``man's destiny,'' the ``calling to full 
manhood,'' and similar commandments, it were held up to people that their 
\textit{self-interest} was infringed on when they let everything in the State 
go as it goes, then, without declamations, they would be addressed as one will 
have to address them at the decisive moment if he wants to attain his end. 
Instead of this, the theology-hating author says, ``If there has ever been a 
time when the \textit{State} laid claim to all that are hers, such a time is 
ours. -- The thinking man sees in participation in the theory and practice of 
the State a \textit{duty}, one of the most sacred duties that rest upon him'' -- and then takes under closer consideration the ``unconditional necessity 
that everybody participate in the State.''

He in whose head or heart or both the \textit{State} is seated, he who is 
possessed by the State, or the \textit{believer in the State}, is a 
politician, and remains such to all eternity.

``The State is the most necessary means for the complete development of 
mankind.'' It assuredly has been so as long as we wanted to develop mankind; 
but, if we want to develop ourselves, it can be to us only a means of 
hindrance.

Can State and people still be reformed and bettered now? As little as the 
nobility, the clergy, the church, etc.: they can be abrogated, annihilated, 
done away with, not reformed. Can I change a piece of nonsense into sense by 
reforming it, or must I drop it outright?

Henceforth what is to be done is no longer about the \textit{State} (the form 
of the State, etc.), but about me. With this all questions about the prince's 
power, the constitution, etc., sink into their true abyss and their true 
nothingness. I, this nothing, shall put forth my \textit{creations} from 
myself.

\myhrule


To the chapter of society belongs also ``the party,'' whose praise has of 
late been sung.

In the State the \textit{party} is current. ``Party, party, who should not 
join one!'' But the individual is 
\textit{unique},\footnote{[\textit{einzig}]} not a member of the party. He 
unites freely, and separates freely again. The party is nothing but a State in 
the State, and in this smaller bee- State ``peace'' is also to rule just as 
in the greater. The very people who cry loudest that there must be an 
\textit{opposition} in the State inveigh against every discord in the party. A 
proof that they too want only a --State. All parties are shattered not against 
the State, but against the ego.\footnote{[\textit{am Einzigen}]}

One hears nothing oftener now than the admonition to remain true to his party; 
party men despise nothing so much as a mugwump. One must run with his party 
through thick and thin, and unconditionally approve and represent its chief 
principles. It does not indeed go quite so badly here as with closed 
societies, because these bind their members to fixed laws or statutes 
(\textit{e.g.} the orders, the Society of Jesus, etc.). But yet the party 
ceases to be a union at the same moment at which it makes certain principles 
\textit{binding} and wants to have them assured against attacks; but this 
moment is the very birth-act of the party. As party it is already a 
\textit{born society}, a dead union, an idea that has become fixed. As party 
of absolutism it cannot will that its members should doubt the irrefragable 
truth of this principle; they could cherish this doubt only if they were 
egoistic enough to want still to be something outside their party, 
\textit{i.e.} non-partisans. Non-partisans they cannot be as party-men, but 
only as egoists. If you are a Protestant and belong to that party, you must 
only justify Protestantism, at most ``purge'' it, not reject it; if you are 
a Christian and belong among men to the Christian party, you cannot be beyond 
this as a member of this party, but only when your egoism, \textit{i.e.} 
non-partisanship, impels you to it. What exertions the Christians, down to 
Hegel and the Communists, have put forth to make their party strong! They 
stuck to it that Christianity must contain the eternal truth, and that one 
needs only to get at it, make sure of it, and justify it.

In short, the party cannot bear non-partisanship, and it is in this that 
egoism appears. What matters the party to me? I shall find enough anyhow who 
\textit{unite} with me without swearing allegiance to my flag.

He who passes over from one party to another is at once abused as a 
``turncoat.'' Certainly \textit{morality} demands that one stand by his 
party, and to become apostate from it is to spot oneself with the stain of 
``faithlessness''; but ownness knows no commandment of ``faithlessness''; 
adhesion, etc., ownness permits everything, even apostasy, defection. 
Unconsciously even the moral themselves let themselves be led by this 
principle when they have to judge one who passes over to \textit{their} party -- nay, they are likely to be making proselytes; they should only at the same 
time acquire a consciousness of the fact that one must commit \textit{immoral} 
actions in order to commit his own -- \textit{i.e.} here, that one must break 
faith, yes, even his oath, in order to determine himself instead of being 
determined by moral considerations. In the eyes of people of strict moral 
judgment an apostate always shimmers in equivocal colors, and will not easily 
obtain their confidence; for there sticks to him the taint of 
``faithlessness,'' \textit{i.e.} of an immorality. In the lower man this 
view is found almost generally; advanced thinkers fall here too, as always, 
into an uncertainty and bewilderment, and the contradiction necessarily 
founded in the principle of morality does not, on account of the confusion of 
their concepts, come clearly to their consciousness. They do not venture to 
call the apostate downright immoral, because they themselves entice to 
apostasy, to defection from one religion to another, etc.; still, they cannot 
give up the standpoint of morality either. And yet here the occasion was to be 
seized to step outside of morality.

Are the Own or Unique\footnote{[\textit{Einzigen}]} perchance a party? How 
could they be \textit{own} if they were \textit{e.g.} \textit{belonged} to a 
party?

Or is one to hold with no party? In the very act of joining them and entering 
their circle one forms a union with them that lasts as long as party and I 
pursue one and the same goal. But today I still share the party's tendency, as 
by tomorrow I can do so no longer and I become ``untrue'' to it. The party 
has nothing binding (obligatory) for me, and I do not have respect for it; if 
it no longer pleases me, I become its foe.

In every party that cares for itself and its persistence, the members are 
unfree (or better, unown) in that degree, they lack egoism in that degree, in 
which they serve this desire of the party. The independence of the party 
conditions the lack of independence in the party- members.

A party, of whatever kind it may be, can never do without a \textit{confession 
of faith}. For those who belong to the party must \textit{believe} in its 
principle, it must not be brought in doubt or put in question by them, it must 
be the certain, indubitable thing for the party-member. That is: One must 
belong to a party body and soul, else one is not truly a party-man, but more 
or less -- an egoist. Harbor a doubt of Christianity, and you are already no 
longer a true Christian, you have lifted yourself to the ``effrontery'' of 
putting a question beyond it and haling Christianity before your egoistic 
judgment-seat. You have -- \textit{sinned} against Christianity, this party 
cause (for it is surely not \textit{e.g.} a cause for the Jews, another 
party.) But well for you if you do not let yourself be affrighted: your 
effrontery helps you to ownness.

So then an egoist could never embrace a party or take up with a party? Oh, 
yes, only he cannot let himself be embraced and taken up by the party. For him 
the party remains all the time nothing but a gathering: he is one of the 
party, he takes part.

\myhrule


The best State will clearly be that which has the most loyal citizens, and the 
more the devoted mind for \textit{legality} is lost, so much the more will the 
State, this system of morality, this moral life itself, be diminished in force 
and quality. With the ``good citizens'' the good State too perishes and 
dissolves into anarchy and lawlessness. ``Respect for the law!'' By this 
cement the total of the State is held together. ``The law is \textit{sacred}, 
and he who affronts it a \textit{criminal''}. Without crime no State: the 
moral world -- and this the State is -- is crammed full of scamps, cheats, 
liars, thieves, etc. Since the State is the ``lordship of law,'' its 
hierarchy, it follows that the egoist, in all cases where \textit{his} 
advantage runs against the State's, can satisfy himself only by crime.

The State cannot give up the claim that its \textit{laws} and ordinances are 
\textit{sacred}.\footnote{[\textit{heilig}]} At this the individual ranks as 
the \textit{unholy}\footnote{[\textit{unheilig}]} (barbarian, natural man, 
``egoist'') over against the State, exactly as he was once regarded by the 
Church; before the individual the State takes on the nimbus of a 
saint.\footnote{[\textit{Heiliger}]} Thus it issues a law against dueling. Two 
men who are both at one in this, that they are willing to stake their life for 
a cause (no matter what), are not to be allowed this, because the State will 
not have it: it imposes a penalty on it. Where is the liberty of 
self-determination then? It is at once quite another situation if, as 
\textit{e.g.} in North America, society determines to let the duelists bear 
certain evil \textit{consequences} of their act, \textit{e.g.} withdrawal of 
the credit hitherto enjoyed. To refuse credit is everybody's affair, and, if a 
society wants to withdraw it for this or that reason, the man who is hit 
cannot therefore complain of encroachment on his liberty: the society is 
simply availing itself of its own liberty. That is no penalty for sin, no 
penalty for a \textit{crime}. The duel is no crime there, but only an act 
against which the society adopts counter-measures, resolves on a 
\textit{defense}. The State, on the contrary, stamps the duel as a crime, 
\textit{i.e.} as an injury to its sacred law: it makes it a \textit{criminal 
case}. The society leaves it to the individual's decision whether he will draw 
upon himself evil consequences and inconveniences by his mode of action, and 
hereby recognizes his free decision; the State behaves in exactly the reverse 
way, denying all right to the individual's decision and, instead, ascribing 
the sole right to its own decision, the law of the State, so that he who 
transgresses the State's commandment is looked upon as if he were acting 
against God's commandment -- a view which likewise was once maintained by the 
Church. Here God is the Holy in and of himself, and the commandments of the 
Church, as of the State, are the commandments of this Holy One, which he 
transmits to the world through his anointed and Lords-by-the-Grace-of-God. If 
the Church had \textit{deadly sins}, the State has \textit{capital crimes;} if 
the one had \textit{heretics}, the other has \textit{traitors;} the one 
\textit{ecclesiastical penalties}, the other \textit{criminal penalties;} the 
one \textit{inquisitorial} processes, the other \textit{fiscal;} in short, 
there sins, here crimes, there inquisition and here -- inquisition. Will the 
sanctity of the State not fall like the Church's? The awe of its laws, the 
reverence for its highness, the humility of its ``subjects,'' will this 
remain? Will the ``saint's'' face not be stripped of its adornment?

What a folly, to ask of the State's authority that it should enter into an 
honourable fight with the individual, and, as they express themselves in the 
matter of freedom of the press, share sun and wind equally! If the State, this 
thought, is to be a \textit{de facto} power, it simply must be a superior 
power against the individual. The State is ``sacred'' and must not expose 
itself to the ``impudent attacks'' of individuals. If the State is 
\textit{sacred}, there must be censorship. The political liberals admit the 
former and dispute the inference. But in any case they concede repressive 
measures to it, for -- they stick to this, that State is \textit{more} than 
the individual and exercises a justified revenge, called punishment.

\textit{Punishment} has a meaning only when it is to afford expiation for the 
injuring of a\textit{sacred} thing. If something is sacred to any one, he 
certainly deserves punishment when he acts as its enemy. A man who lets a 
man's life continue in existence \textit{because} to him it is sacred and he 
has a \textit{dread} of touching it is simply a -- \textit{religious} man.

Weitling lays crime at the door of ``social disorder,'' and lives in the 
expectation that under Communistic arrangements crimes will become impossible, 
because the temptations to them, \textit{e.g.} money, fall away. As, however, 
his organized society is also exalted into a sacred and inviolable one, he 
miscalculates in that good-hearted opinion. \textit{e.g.} with their mouth 
professed allegiance to the Communistic society, but worked underhand for its 
ruin, would not be lacking. Besides, Weitling has to keep on with ``curative 
means against the natural remainder of human diseases and weaknesses,'' and 
``curative means'' always announce to begin with that individuals will be 
looked upon as ``called'' to a particular ``salvation'' and hence treated 
according to the requirements of this ``human calling.'' \textit{Curative 
means} or \textit{healing} is only the reverse side of \textit{punishment}, 
the \textit{theory of cure} runs parallel with the \textit{theory of 
punishment;} if the latter sees in an action a sin against right, the former 
takes it for a sin of the man \textit{against himself}, as a decadence from 
his health. But the correct thing is that I regard it either as an action that 
\textit{suits me} or as one that \textit{does not suit me}, as hostile or 
friendly to \textit{me}, \textit{i.e.} that I treat it as my 
\textit{property}, which I cherish or demolish. ``Crime'' or ``disease'' 
are not either of them an \textit{egoistic} view of the matter, \textit{i.e.} 
a judgment \textit{starting from me}, but starting from \textit{another --} to 
wit, whether it injures \textit{right}, general right, or the \textit{health} 
partly of the individual (the sick one), partly of the generality 
(\textit{society}). ``Crime'' is treated inexorably, ``disease'' with 
``loving gentleness, compassion,'' etc.

Punishment follows crime. If crime falls because the sacred vanishes, 
punishment must not less be drawn into its fall; for it too has significance 
only over against something sacred. Ecclesiastical punishments have been 
abolished. Why? Because how one behaves toward the ``holy God'' is his own 
affair. But, as this one punishment, \textit{ecclesiastical punishment}, has 
fallen, so all \textit{punishments} must fall. As sin against the so-called 
God is a man's own affair, so is that against every kind of the so-called 
sacred. According to our theories of penal law, with whose ``improvement in 
conformity to the times'' people are tormenting themselves in vain, they want 
to \textit{punish} men for this or that ``inhumanity''; and therein they 
make the silliness of these theories especially plain by their consistency, 
hanging the little thieves and letting the big ones run. For injury to 
property they have the house of correction, and for ``violence to thought,'' 
suppression of ``natural rights of man,'' only --representations and 
petitions.

The criminal code has continued existence only through the sacred, and 
perishes of itself if punishment is given up. Now they want to create 
everywhere a new penal law, without indulging in a misgiving about punishment 
itself. But it is exactly punishment that must make room for satisfaction, 
which, again, cannot aim at satisfying right or justice, but at procuring 
\textit{us} a satisfactory outcome. If one does to us what we \textit{will not 
put up with}, we break his power and bring our own to bear: we satisfy 
\textit{ourselves} on him, and do not fall into the folly of wanting to 
satisfy right (the spook). It is not the \textit{sacred} that is to defend 
itself against man, but man against man; as \textit{God} too, you know, no 
longer defends himself against man, God to whom formerly (and in part, indeed, 
even now) all the ``servants of God'' offered their hands to punish the 
blasphemer, as they still at this very day lend their hands to the sacred. 
This devotion to the sacred brings it to pass also that, without lively 
participation of one's own, one only delivers misdoers into the hands of the 
police and courts: a non-participating making over to the authorities, ``who, 
of course, will best administer sacred matters.'' The people is quite crazy 
for hounding the police on against everything that seems to it to be immoral, 
often only unseemly, and this popular rage for the moral protects the police 
institution more than the government could in any way protect it.

In crime the egoist has hitherto asserted himself and mocked at the sacred; 
the break with the sacred, or rather of the sacred, may become general. A 
revolution never returns, but a mighty, reckless, shameless, conscienceless. 
proud --\textit{crime}, does it not rumble in distant thunders, and do you not 
see how the sky grows presciently silent and gloomy?

\myhrule


He who refuses to spend his powers for such limited societies as family, 
party, nation, is still always longing for a worthier society, and thinks he 
has found the true object of love, perhaps, in ``human society'' or 
``mankind,'' to sacrifice himself to which constitutes his honor; from now 
on he ``lives for and serves \textit{mankind}.''

\textit{People} is the name of the body, \textit{State} of the spirit, of that 
\textit{ruling person} that has hitherto suppressed me. Some have wanted to 
transfigure peoples and States by broadening them out to ``mankind'' and 
``general reason''; but servitude would only become still more intense with 
this widening, and philanthropists and humanitarians are as absolute masters 
as politicians and diplomats.

Modern critics inveigh against religion because it sets God, the divine, 
moral, etc., \textit{outside} of man, or makes them something objective, in 
opposition to which the critics rather transfer these very subjects 
\textit{into} man. But those critics none the less fall into the proper error 
of religion, to give man a ``destiny,'' in that they too want to have him 
divine, human, and the like: morality, freedom and humanity, etc., are his 
essence. And, like religion politics too wanted to \textit{``educate''} man, 
to bring him to the realization of his ``essence,'' his ``destiny,'' to 
\textit{make} something out of him -- to wit, a ``true man,'' the one in the 
form of the ``true believer,'' the other in that of the ``true citizen or 
subject.'' In fact, it comes to the same whether one calls the destiny the 
divine or human.

Under religion and politics man finds himself at the standpoint of 
\textit{should: he should} become this and that, should be so and so. With 
this postulate, this commandment, every one steps not only in front of another 
but also in front of himself. Those critics say: You should be a whole, free 
man. Thus they too stand in the temptation to proclaim a new 
\textit{religion}, to set up a new absolute, an ideal -- to wit, freedom. Men 
\textit{should} be free. Then there might even arise \textit{missionaries} of 
freedom, as Christianity, in the conviction that all were properly destined to 
become Christians, sent out missionaries of the faith. Freedom would then (as 
have hitherto faith as Church, morality as State) constitute itself as a new 
\textit{community} and carry on a like ``propaganda'' therefrom. Certainly 
no objection can be raised against a getting together; but so much the more 
must one oppose every renewal of the old \textit{care} for us, of culture 
directed toward an end -- in short, the principle of \textit{making something} 
out of us, no matter whether Christians, subjects, or freemen and men.

One may well say with Feuerbach and others that religion has displaced the 
human from man, and has transferred it so into another world that, 
unattainable, it went on with its own existence there as something personal in 
itself, as a ``God'': but the error of religion is by no means exhausted 
with this. One might very well let fall the personality of the displaced 
human, might transform God into the divine, and still remain religious. For 
the religious consists in discontent with the \textit{present} men, in the 
setting up of a ``perfection'' to be striven for, in ``man wrestling for 
his completion.''\footnote{B. Bauer, \textit{``Lit. Ztg}.'' 8,22.} (``Ye 
therefore \textit{should} be perfect as your father in heaven is perfect.'' 
Matt. 5, 48): it consists in the fixation of an ideal, an absolute. Perfection 
is the ``supreme good,'' the \textit{finis bonorum;} every one's ideal is 
the perfect man, the true, the free man, etc.

The efforts of modern times aim to set up the ideal of the ``free man.'' If 
one could find it, there would be a new -- religion, because a new ideal; 
there would be a new longing, a new torment, a new devotion, a new deity, a 
new contrition.

With the ideal of ``absolute liberty,'' the same turmoil is made as with 
everything absolute, and according to Hess, \textit{e.g.}, it is said to 
``be realizable in absolute human society.''\footnote{\textit{``E. u. Z. 
B.},'' p. 89ff.} Nay, this realization is immediately afterward styled a 
``vocation''; just so he then defines liberty as ``morality'': the kingdom 
of ``justice'' (equality) and ``morality'' (\textit{i.e.} liberty) is to 
begin, etc.

Ridiculous is he who, while fellows of his tribe, family, nation, rank high, 
is -- nothing but ``puffed up'' over the merit of his fellows; but blinded 
too is he who wants only to be ``man.'' Neither of them puts his worth in 
\textit{exclusiveness}, but in \textit{connectedness}, or in the ``tie'' 
that conjoins him with others, in the ties of blood, of nationality, of 
humanity.

Through the ``Nationals'' of today the conflict has again been stirred up 
between those who think themselves to have merely human blood and human ties 
of blood, and the others who brag of their special blood and the special ties 
of blood.

If we disregard the fact that pride may mean conceit, and take it for 
consciousness alone, there is found to be a vast difference between pride in 
``belonging to'' a nation and therefore being its property, and that in 
calling a nationality one's property. Nationality is my quality, but the 
nation my owner and mistress. If you have bodily strength, you can apply it at 
a suitable place and have a self-consciousness or pride of it; if, on the 
contrary, your strong body has you, then it pricks you everywhere, and at the 
most unsuitable place, to show its strength: you can give nobody your hand 
without squeezing his.

The perception that one is more than a member of the family, more than a 
fellow of the tribe, more than an individual of the people, has finally led to 
saying, one is more than all this because one is man, or, the man is more than 
the Jew, German, etc. ``Therefore be every one wholly and solely -- man.'' 
Could one not rather say: Because we are more than what has been stated, 
therefore we will be this, as well as that ``more'' also? Man and Germans, 
then, man and Guelph, etc.? The Nationals are in the right; one cannot deny 
his nationality: and the humanitarians are in the right; one must not remain 
in the narrowness of the national. In 
\textit{uniqueness}\footnote{[\textit{Einzigkeit}]} the contradiction is 
solved; the national is my quality. But I am not swallowed up in my quality -- 
as the human too is my quality, but I give to man his existence first through 
my uniqueness.

History seeks for Man: but he is I, you, we. Sought as a mysterious 
\textit{essence}, as the divine, first as \textit{God}, then as Man (humanity, 
humaneness, and mankind), he is found as the individual, the finite, the 
unique one.

I am owner of humanity, am humanity, and do nothing for the good of another 
humanity. Fool, you who are a unique humanity, that you make a merit of 
wanting to live for another than you are.

The hitherto-considered relation of me to the \textit{world of men} offers 
such a wealth of phenomena that it will have to be taken up again and again on 
other occasions, but here, where it was only to have its chief outlines made 
clear to the eye, it must be broken off to make place for an apprehension of 
two other sides toward which it radiates. For, as I find myself in relation 
not merely to men so far as they present in themselves the concept ``man'' 
or are children of men (children of \textit{Man}, as children of God are 
spoken of), but also to that which they have of man and call their own, and as 
therefore I relate myself not only to that which they \textit{are} through 
man, but also to their human \textit{possessions:} so, besides the world of 
men, the world of the senses and of ideas will have to be included in our 
survey, and somewhat said of what men call their own of sensuous goods, and of 
spiritual as well.

According as one had developed and clearly grasped the concept of man, he gave 
it to us to respect as this or that \textit{person of respect}, and from the 
broadest understanding of this concept there proceeded at last the command 
``to respect Man in every one.'' But if I respect Man, my respect must 
likewise extend to the human, or what is Man's.

Men have somewhat of their \textit{own}, and \textit{I} am to recognize this 
own and hold it sacred. Their own consists partly in outward, partly in inward 
\textit{possessions}. The former are things, the latter spiritualities, 
thoughts, convictions, noble feelings, etc. But I am always to respect only 
\textit{rightful} or \textit{human} possessions: the wrongful and unhuman I 
need not spare, for only \textit{Man's} own is men's real own. An inward 
possession of this sort is, \textit{e.g.}, religion; because 
\textit{religion} is free, \textit{i.e.} is Man's, \textit{I} must not strike 
at it. Just so \textit{honor} is an inward possession; it is free and must not 
be struck at my me. (Action for insult, caricatures, etc.) Religion and honor 
are ``spiritual property.'' In tangible property the person stands foremost: 
my person is my first property. Hence freedom of the person; but only the 
\textit{rightful} or human person is free, the other is locked up. Your life 
is your property; but it is sacred for men only if it is not that of an 
inhuman monster.

What a man as such cannot defend of bodily goods, we may take from him: this 
is the meaning of competition, of freedom of occupation. What he cannot defend 
of spiritual goods falls a prey to us likewise: so far goes the liberty of 
discussion, of science, of criticism.

But \textit{consecrated} goods are inviolable. Consecrated and guarantied by 
whom? Proximately by the State, society, but properly by man or the 
``concept,'' the ``concept of the thing''; for the concept of consecrated 
goods is this, that they are truly human, or rather that the holder possesses 
them as man and not as un-man.\footnote{[See note on p. 184.]}

On the spiritual side man's faith is such goods, his honor, his moral feeling -- yes, his feeling of decency, modesty, etc. Actions (speeches, writings) 
that touch honor are punishable; attacks on ``the foundations of all 
religion''; attacks on political faith; in short, attacks on everything that 
a man ``rightly'' has.

How far critical liberalism would extend the sanctity of goods -- on this 
point it has not yet made any pronouncement, and doubtless fancies itself to 
be ill-disposed toward all sanctity; but, as it combats egoism, it must set 
limits to it, and must not let the un-man pounce on the human. To its 
theoretical contempt for the ``masses'' there must correspond a practical 
snub if it should get into power.

What extension the concept ``man'' receives, and what comes to the 
individual man through it -- what, therefore, man and the human are -- on this 
point the various grades of liberalism differ, and the political, the social, 
the humane man are each always claiming more than the other for ``man.'' He 
who has best grasped this concept knows best what is ``man's.'' The State 
still grasps this concept in political restriction, society in social; 
mankind, so it is said, is the first to comprehend it entirely, or ``the 
history of mankind develops it.'' But, if ``man is discovered,'' then we 
know also what pertains to man as his own, man's property, the human.

But let the individual man lay claim to ever so many rights because Man or the 
concept man ``entitles'' him to them, because his being man does it: what do 
I care for his right and his claim? If he has his right only from Man and does 
not have it from \textit{me}, then for \textit{me} he has no right. His life, 
\textit{e.g.}, counts to \textit{me} only for what it is \textit{worth} to 
\textit{me}. I respect neither a so-called right of property (or his claim to 
tangible goods) nor yet his right to the ``sanctuary of his inner nature'' 
(or his right to have the spiritual goods and divinities, his gods, remain 
un-aggrieved). His goods, the sensuous as well as the spiritual, are 
\textit{mine}, and I dispose of them as proprietor, in the measure of my -- 
might.

In the \textit{property question} lies a broader meaning than the limited 
statement of the question allows to be brought out. Referred solely to what 
men call our possessions, it is capable of no solution; the decision is to be 
found in him ``from whom we have everything.'' Property depends on the 
\textit{owner}.

The Revolution directed its weapons against everything which came ``from the 
grace of God,'' \textit{e.g.}, against divine right, in whose place the 
human was confirmed. To that which is granted by the grace of God, there is 
opposed that which is derived ``from the essence of man.''

Now, as men's relation to each other, in opposition to the religious dogma 
which commands a ``Love one another for God's sake,'' had to receive its 
human position by a ``Love each other for man's sake,'' so the revolutionary 
teaching could not do otherwise than, first, as to what concerns the relation 
of men to the things of this world, settle it that the world, which hitherto 
was arranged according to God's ordinance, henceforth belongs to ``Man.''

The world belongs to ``Man,'' and is to be respected by me as his property.

Property is what is mine!

Property in the civic sense means \textit{sacred} property, such that I must 
\textit{respect} your property. ``Respect for property!'' Hence the 
politicians would like to have every one possess his little bit of property, 
and they have in part brought about an incredible parcellation by this effort. 
Each must have his bone on which he may find something to bite.

The position of affairs is different in the egoistic sense. I do not step 
shyly back from your property, but look upon it always as my property, in 
which I need to ``respect'' nothing. Pray do the like with what you call my 
property!

With this view we shall most easily come to an understanding with each other.

The political liberals are anxious that, if possible, all servitudes be 
dissolved, and every one be free lord on his ground, even if this ground has 
only so much area as can have its requirements adequately filled by the manure 
of one person. (The farmer in the story married even in his old age ``that he 
might profit by his wife's dung.'') Be it ever so little, if one only has 
somewhat of his own -- to wit, a \textit{respected} property! The more such 
owners, such cotters,\footnote{[The words ``cot'' and ``dung'' are alike 
in German.]} the more ``free people and good patriots'' has the State.

Political liberalism, like everything religious, counts on \textit{respect}, 
humaneness, the virtues of love. Therefore does it live in incessant vexation. 
For in practice people respect nothing, and every day the small possessions 
are bought up again by greater proprietors, and the ``free people'' change 
into day- laborers.

If, on the contrary, the ``small proprietors'' had reflected that the great 
property was also theirs, they would not have respectfully shut themselves out 
from it, and would not have been shut out.

Property as the civic liberals understand it deserves the attacks of the 
Communists and Proudhon: it is untenable, because the civic proprietor is in 
truth nothing but a property-less man, one who is everywhere \textit{shut 
out}. Instead of owning the world, as he might, he does not own even the 
paltry point on which he turns around.

Proudhon wants not the \textit{propri\'etaire} but the \textit{possesseur} or 
\textit{usufruitier}.\footnote{\textit{e.g.}, \textit{``Qu'est-ce que la 
Propri\'et\'e?}'' p. 83} What does that mean? He wants no one to own the 
land; but the benefit of it -- even though one were allowed only the hundredth 
part of this benefit, this fruit -- is at any rate one's property, which he 
can dispose of at will. He who has only the benefit of a field is assuredly 
not the proprietor of it; still less he who, as Proudhon would have it, must 
give up so much of this benefit as is not required for his wants; but he is 
the proprietor of the share that is left him. Proudhon, therefore, denies only 
such and such property, not \textit{property} itself. If we want no longer to 
leave the land to the landed proprietors, but to appropriate it to 
\textit{ourselves}, we unite ourselves to this end, form a union, a 
\textit{soci\'et\'e}, that makes \textit{itself} proprietor; if we have good 
luck in this, then those persons cease to be landed proprietors. And, as from 
the land, so we can drive them out of many another property yet, in order to 
make it \textit{our} property, the property of the -- \textit{conquerors}. The 
conquerors form a society which one may imagine so great that it by degrees 
embraces all humanity; but so-called humanity too is as such only a thought 
(spook); the individuals are its reality. And these individuals as a 
collective (mass will treat land and earth not less arbitrarily than an 
isolated individual or so-called \textit{propri\'etaire}. Even so, therefore, 
\textit{property} remains standing, and that as exclusive too, in that 
\textit{humanity}, this great society, excludes the \textit{individual} from 
its property (perhaps only leases to him, gives his as a fief, a piece of it) 
as it besides excludes everything that is not humanity, \textit{e.g.} does 
not allow animals to have property. -- So too it will remain, and will grow to 
be. That in which \textit{all} want to have a \textit{share} will be withdrawn 
from that individual who wants to have it for himself alone: it is made a 
\textit{common estate}. As a \textit{common estate} every one has his 
\textit{share} in it, and this share is his \textit{property}. Why, so in our 
old relations a house which belongs to five heirs is their common estate; but 
the fifth part of the revenue is, each one's property. Proudhon might spare 
his prolix pathos if he said: ``There are some things that belong only to a 
few, and to which we others will from now on lay claim or -- siege. Let us 
take them, because one comes to property by taking, and the property of which 
for the present we are still deprived came to the proprietors likewise only by 
taking. It can be utilized better if it is in the hands of us \textit{all} 
than if the few control it. Let us therefore associate ourselves for the 
purpose of this robbery (\textit{vol}).'' -- Instead of this, he tries to get 
us to believe that society is the original possessor and the sole proprietor, 
of imprescriptible right; against it the so-called proprietors have become 
thieves (\textit{La propri\'et\'e c'est le vol}); if it now deprives of his 
property the present proprietor, it robs him of nothing, as it is only 
availing itself of its imprescriptible right. -- So far one comes with the 
spook of society as a \textit{moral person}. On the contrary, what man can 
obtain belongs to him: the world belongs to \textit{me}. Do you say anything 
else by your opposite proposition? ``The world belongs to \textit{all''}? 
All are I and again I, etc. But you make out of the ``all'' a spook, and 
make it sacred, so that then the ``all'' become the individual's fearful 
\textit{master}. Then the ghost of ``right'' places itself on their side.

Proudhon, like the Communists, fights against \textit{egoism}. Therefore they 
are continuations and consistent carryings-out of the Christian principle, the 
principle of love, of sacrifice for something general, something alien. They 
complete in property, \textit{e.g.,} only what has long been extant as a 
matter of fact -- to wit, the propertylessness of the individual. When the 
laws says, \textit{Ad reges potestas omnium pertinet, ad singulos proprietas; 
omnia rex imperio possidet, singuli dominio}, this means: The king is 
proprietor, for he alone can control and dispose of ``everything,'' he has 
\textit{potestas} and \textit{imperium} over it. The Communists make this 
clearer, transferring that \textit{imperium} to the ``society of all.'' 
Therefore: Because enemies of egoism, they are on that account -- Christians, 
or, more generally speaking, religious men, believers in ghosts, dependents, 
servants of some generality (God, society, etc.). In this too Proudhon is like 
the Christians, that he ascribes to God that which he denies to men. He names 
him (\textit{e.g.} page 90) the Propri\'etaire of the earth. Herewith he 
proves that he cannot think away the \textit{proprietor as such;} he comes to 
a proprietor at last, but removes him to the other world.

Neither God nor Man (``human society'') is proprietor, but the individual.

\myhrule


Proudhon (Weitling too) thinks he is telling the worst about property when he 
calls it theft (\textit{vol}). Passing quite over the embarrassing question, 
what well-founded objection could be made against theft, we only ask: Is the 
concept ``theft'' at all possible unless one allows validity to the concept 
``property''? How can one steal if property is not already extant? What 
belongs to no one cannot be \textit{stolen;} the water that one draws out of 
the sea he does \textit{not steal}. Accordingly property is not theft, but a 
theft becomes possible only through property. Weitling has to come to this 
too, as he does regard everything as the \textit{property of all:} if 
something is ``the property of all,'' then indeed the individual who 
appropriates it to himself steals.

Private property lives by grace of the \textit{law}. Only in the law has it 
its warrant -- for possession is not yet property, it becomes ``mine'' only 
by assent of the law; it is not a fact, not \textit{un fait} as Proudhon 
thinks, but a fiction, a thought. This is legal property, legitimate property, 
guarantied property. It is mine not through \textit{me} but through the -- 
\textit{law}.

Nevertheless, property is the expression for \textit{unlimited dominion} over 
somewhat (thing, beast, man) which ``I can judge and dispose of as seems good 
to me.'' According to Roman law, indeed, \textit{jus utendi et abutendi re 
sua, quatenus juris ratio patitur}, an \textit{exclusive} and 
\textit{unlimited right;} but property is conditioned by might. What I have in 
my power, that is my own. So long as I assert myself as holder, I am the 
proprietor of the thing; if it gets away from me again, no matter by what 
power, \textit{e.g.} through my recognition of a title of others to the thing -- then the property is extinct. Thus property and possession coincide. It is 
not a right lying outside my might that legitimizes me, but solely my might: 
if I no longer have this, the thing vanishes away from me. When the Romans no 
longer had any might against the Germans, the world-empire of Rome 
\textit{belonged} to the latter, and it would sound ridiculous to insist that 
the Romans had nevertheless remained properly the proprietors. Whoever knows 
how to take and to defend the thing, to him it belongs till it is again taken 
from him, as liberty belongs to him who \textit{takes} it.--

Only might decides about property, and, as the State (no matter whether State 
or well-to-do citizens or of ragamuffins or of men in the absolute) is the 
sole mighty one, it alone is proprietor; I, the 
unique,\footnote{[\textit{Einzige}]} have nothing, and am only enfeoffed, am 
vassal and as such, servitor. Under the dominion of the State there is no 
property of \textit{mine}.

I want to raise the value of myself, the value of ownness, and should I 
cheapen property? No, as I was not respected hitherto because people, mankind, 
and a thousand other generalities were put higher, so property too has to this 
day not yet been recognized in its full value. Property too was only the 
property of a ghost, \textit{e.g.} the people's property; my whole existence 
``belonged to the fatherland''; \textit{I} belonged to the fatherland, the 
people, the State, and therefore also everything that I called \textit{my 
own}. It is demanded of States that they make away with pauperism. It seems to 
me this is asking that the State should cut off its own head and lay it at its 
feet; for so long as the State is the ego the individual ego must remain a 
poor devil, a non-ego. The State has an interest only in being itself rich; 
whether Michael is rich and Peter poor is alike to it; Peter might also be 
rich and Michael poor. It looks on indifferently as one grows poor and the 
other rich, unruffled by this alternation. As \textit{individuals} they are 
really equal before its face; in this it is just: before it both of them are -- nothing, as we ``are altogether sinners before God''; on the other hand, 
it has a very great interest in this, that those individuals who make it their 
ego should have a part in \textit{its} wealth; it makes them partakers in 
\textit{its property}. Through property, with which it rewards the 
individuals, it tames them; but this remains \textit{its} property, and every 
one has the usufruct of it only so long as he bears in himself the ego of the 
State, or is a ``loyal member of society''; in the opposite case the 
property is confiscated, or made to melt away by vexatious lawsuits. The 
property, then, is and remains \textit{State property}, not property of the 
ego. That the State does not arbitrarily deprive the individual of what he has 
from the State means simply that the State does not rob itself. He who is 
State-ego, \textit{i.e.} a good citizen or subject, holds his fief undisturbed 
as \textit{such an ego}, not as being an ego of his own. According to the 
code, property is what I call mine ``by virtue of God and law.'' But it is 
mine by virtue of God and law only so long as -- the State has nothing against 
it.

In expropriations, disarmaments, etc. (as, when the exchequer confiscates 
inheritances if the heirs do not put in an appearance early enough) how 
plainly the else-veiled principle that only the \textit{people}, ``the 
State,'' is proprietor, while the individual is feoffee, strikes the eye!

The State, I mean to say, cannot intend that anybody should \textit{for his 
own sake} have property or actually be rich, nay, even well-to-do; it can 
acknowledge nothing, yield nothing, grant nothing to me as me. The State 
cannot check pauperism, because the poverty of possession is a poverty of me. 
He who \textit{is} nothing but what chance or another -- to wit, the State -- 
makes out of him also \textit{has} quite rightly nothing but what another 
gives him. And this other will \textit{give} him only what he 
\textit{deserves}, \textit{i.e.} what he is worth by \textit{service}. It is 
not he that realizes a value from himself; the State realizes a value from 
him.

National economy busies itself much with this subject. It lies far out beyond 
the ``national,'' however, and goes beyond the concepts and horizon of the 
State, which knows only State property and can distribute nothing else. For 
this reason it binds the possessions of property to \textit{conditions --} as 
it binds everything to them, \textit{e.g.} marriage, allowing validity only 
to the marriage sanctioned by it, and wresting this out of my power. But 
property is my property only when I hold it \textit{unconditionally} : only I, 
an \textit{unconditional} ego, have property, enter a relation of love, carry 
on free trade.

The State has no anxiety about me and mine, but about itself and its: I count 
for something to it only as its \textit{child}, as ``a son of the country''; 
as \textit{ego} I am nothing at all for it. For the State's understanding, 
what befalls me as ego is something accidental, my wealth as well as my 
impoverishment. But, if I with all that is mine am an accident in the State's 
eyes, this proves that it cannot comprehend \textit{me: I} go beyond its 
concepts, or, its understanding is too limited to comprehend me. Therefore it 
cannot do anything for me either.

Pauperism is the \textit{valuelessness of me}, the phenomenon that I cannot 
realize value from myself. For this reason State and pauperism are one and the 
same. The State does not let me come to my value, and continues in existence 
only through my valuelessness: it is forever intent on \textit{getting 
benefit} from me, \textit{i.e.} exploiting me, turning me to account, using me 
up, even if the use it gets from me consists only in my supplying a 
\textit{proles} (proletariat); it wants me to be ``its creature.''

Pauperism can be removed only when I as ego \textit{realize value} from 
myself, when I give my own self value, and make my price myself. I must rise 
in revolt to rise in the world.

What I produce, flour, linen, or iron and coal, which I toilsomely win from 
the earth, is my work that I want to realize value from. But then I may long 
complain that I am not paid for my work according to its value: the payer will 
not listen to me, and the State likewise will maintain an apathetic attitude 
so long as it does not think it must ``appease'' me that \textit{I} may not 
break out with my dreaded might. But this ``appeasing'' will be all, and, if 
it comes into my head to ask for more, the State turns against me with all the 
force of its lion-paws and eagle-claws: for it is the king of beasts, it is 
lion and eagle. If I refuse to be content with the price that it fixes for my 
ware and labor, if I rather aspire to determine the price of my ware myself, 
\textit{e.g.}, ``to pay myself,'' in the first place I come into a conflict 
with the buyers of the ware. If this were stilled by a mutual understanding, 
the State would not readily make objections; for how individuals get along 
with each other troubles it little, so long as therein they do not get in its 
way. Its damage and its danger begin only when they do not agree, but, in the 
absence of a settlement, take each other by the hair. The State cannot endure 
that man stand in a direct relation to man; it must step between as 
--\textit{mediator}, must \textit{-- intervene}. What Christ was, what the 
saints, the Church were, the State has become -- to wit, ``mediator.'' It 
tears man from man to put itself between them as ``spirit.'' The laborers 
who ask for higher pay are treated as criminals as soon as they want to 
\textit{compel} it. What are they to do? Without compulsion they don't get it, 
and in compulsion the State sees a self-help, a determination of price by the 
ego, a genuine, free realization of value from his property, which it cannot 
admit of. What then are the laborers to do? Look to themselves and ask nothing 
about the State? -- --

But, as is the situation with regard to my material work, so it is with my 
intellectual too. The State allows me to realize value from all my thoughts 
and to find customers for them (I do realize value from them, \textit{e.g.} 
in the very fact that they bring me honor from the listeners, etc.); but only 
so long as \textit{my} thoughts are --\textit{its} thoughts. If, on the other 
hand, I harbor thoughts that it cannot approve (\textit{i.e.} make its own), 
then it does not allow me at all to realize value from them, to bring them 
into \textit{exchange} into \textit{commerce. My} thoughts are free only if 
they are granted to me by the State's \textit{grace}, \textit{i.e.} if they 
are the State's thoughts. It lets me philosophize freely only so far as I 
approve myself a ``philosopher of State''; \textit{against} the State I must 
not philosophize, gladly as it tolerates my helping it out of its 
``deficiencies,'' ``furthering'' it. -- Therefore, as I may behave only as 
an ego most graciously permitted by the State, provided with its testimonial 
of legitimacy and police pass, so too it is not granted me to realize value 
from what is mine, unless this proves to be its, which I hold as fief from it. 
My ways must be its ways, else it distrains me; my thoughts its thoughts, else 
it stops my mouth.

The State has nothing to be more afraid of than the value of me, and nothing 
must it more carefully guard against than every occasion that offers itself to 
me for \textit{realizing value} from myself. \textit{I} am the deadly enemy of 
the State, which always hovers between the alternatives, it or I. Therefore it 
strictly insists not only on not letting \textit{me} have a standing, but also 
on keeping down what is \textit{mine}. In the State there is no property, 
\textit{i.e.} no property of the individual, but only State property. Only 
through the State have I what I have, as I am only through it what I am. My 
private property is only that which the State leaves to me of \textit{its, 
cutting off} others from it (depriving them, making it private); it is State 
property.

But, in opposition to the State, I feel more and more clearly that there is 
still left me a great might, the might over myself, \textit{i.e.} over 
everything that pertains only to me and that \textit{exists} only in being my 
own.

What do I do if my ways are no longer its ways, my thoughts no longer its 
thoughts? I look to myself, and ask nothing about it! In \textit{my} thoughts, 
which I get sanctioned by no assent, grant, or grace, I have my real property, 
a property with which I can trade. For as mine they are my \textit{creatures}, 
and I am in a position to give them away in return for \textit{other} 
thoughts: I give them up and take in exchange for them others, which then are 
my new purchased property.

What then is \textit{my} property? Nothing but what is in my \textit{power!} 
To what property am I entitled? To every property to which I -- 
\textit{empower} myself.\footnote{[A German idiom for ``take upon myself,'' 
``assume.'']} I give myself the right of property in taking property to 
myself, or giving myself the proprietor's \textit{power}, full power, 
empowerment.

Everything over which I have might that cannot be torn from me remains my 
property; well, then let might decide about property, and I will expect 
everything from my might! Alien might, might that I leave to another, makes me 
an owned slave: then let my own might make me an owner. Let me then withdraw 
the might that I have conceded to others out of ignorance regarding the 
strength of my \textit{own} might! Let me say to myself, what my might reaches 
to is my property; and let me claim as property everything that I feel myself 
strong enough to attain, and let me extend my actual property as far as 
\textit{I} entitle, \textit{i.e.} -- empower, myself to take.

Here egoism, selfishness, must decide; not the principle of \textit{love}, not 
love-motives like mercy, gentleness, good-nature, or even justice and equity 
(for \textit{justitia} too is a phenomenon of -- love, a product of love): 
love knows only \textit{sacrifices} and demands ``self-sacrifice.''

Egoism does not think of sacrificing anything, giving away anything that it 
wants; it simply decides, what I want I must have and will procure.

All attempts to enact rational laws about property have put out from the bay 
of \textit{love} into a desolate sea of regulations. Even Socialism and 
Communism cannot be excepted from this. Every one is to be provided with 
adequate means, for which it is little to the point whether one 
socialistically finds them still in a personal property, or communistically 
draws them from the community of goods. The individual's mind in this remains 
the same; it remains a mind of dependence. The distributing \textit{board of 
equity} lets me have only what the sense of equity, its \textit{loving} care 
for all, prescribes. For me, the individual, there lies no less of a check in 
\textit{collective wealth} than in that of \textit{individual others;} neither 
that is mine, nor this: whether the wealth belongs to the collectivity, which 
confers part of it on me, or to individual possessors, is for me the same 
constraint, as I cannot decide about either of the two. On the contrary, 
Communism, by the abolition of all personal property, only presses me back 
still more into dependence on another, \textit{viz}., on the generality or 
collectivity; and, loudly as it always attacks the ``State,'' what it 
intends is itself again a State, a \textit{status}, a condition hindering my 
free movement, a sovereign power over me. Communism rightly revolts against 
the pressure that I experience from individual proprietors; but still more 
horrible is the might that it puts in the hands of the collectivity.

Egoism takes another way to root out the non-possessing rabble. It does not 
say: Wait for what the board of equity will -- bestow on you in the name of 
the collectivity (for such bestowal took place in ``States'' from the most 
ancient times, each receiving ``according to his desert,'' and therefore 
according to the measure in which each was able to \textit{deserve} it, to 
acquire it by \textit{service}), but: Take hold, and take what you require! 
With this the war of all against all is declared. I alone decide what I will 
have.

``Now, that is truly no new wisdom, for self-seekers have acted so at all 
times!'' Not at all necessary either that the thing be new, if only 
\textit{consciousness} of it is present. But this latter will not be able to 
claim great age, unless perhaps one counts in the Egyptian and Spartan law; 
for how little current it is appears even from the stricture above, which 
speaks with contempt of ``self-seekers.'' One is to know just this, that the 
procedure of taking hold is not contemptible, but manifests the pure deed of 
the egoist at one with himself.

Only when I expect neither from individuals nor from a collectivity what I can 
give to myself, only then do I slip out of the snares of --love; the rabble 
ceases to be rabble only when it \textit{takes hold}. Only the dread of taking 
hold, and the corresponding punishment thereof, makes it a rabble. Only that 
taking hold is \textit{sin}, crime -- only this dogma creates a rabble. For 
the fact that the rabble remains what it is, it (because it allows validity to 
that dogma) is to blame as well as, more especially, those who 
``self-seekingly'' (to give them back their favorite word) demand that the 
dogma be respected. In short, the lack of \textit{consciousness} of that 
``new wisdom,'' the old consciousness of sin, alone bears the blame.

If men reach the point of losing respect for property, every one will have 
property, as all slaves become free men as soon as they no longer respect the 
master as master. \textit{Unions} will then, in this matter too, multiply the 
individual's means and secure his assailed property.

According to the Communists' opinion the commune should be proprietor. On the 
contrary, \textit{I} am proprietor, and I only come to an understanding with 
others about my property. If the commune does not do what suits me, I rise 
against it and defend my property. I am proprietor, but property is 
\textit{not sacred}. I should be merely possessor? No, hitherto one was only 
possessor, secured in the possession of a parcel by leaving others also in 
possession of a parcel; but now \textit{everything} belongs to me, I am 
proprietor of \textit{everything that I require} and can get possession of. If 
it is said socialistically, society gives me what I require -- then the egoist 
says, I take what I require. If the Communists conduct themselves as 
ragamuffins, the egoist behaves as proprietor.

All swan-fraternities,\footnote{[Apparently some benevolent scheme of the day; 
compare note on p. 343.]} and attempts at making the rabble happy, that spring 
from the principle of love, must miscarry. Only from egoism can the rabble get 
help, and this help it must give to itself and -- will give to itself. If it 
does not let itself be coerced into fear, it is a power. ``People would lose 
all respect if one did not coerce them into fear,'' says bugbear Law in 
\textit{Der gestiefelte Kater}.

Property, therefore, should not and cannot be abolished; it must rather be 
torn from ghostly hands and become \textit{my} property; then the erroneous 
consciousness, that I cannot entitle myself to as much as I require, will 
vanish. --

``But what cannot man require!'' Well, whoever requires much, and 
understands how to get it, has at all times helped himself to it, as Napoleon 
did with the Continent and France with Algiers. Hence the exact point is that 
the respectful ``rabble'' should learn at last to help itself to what it 
requires. If it reaches out too far for you, why, then defend yourselves. You 
have no need at all to good-heartedly -- bestow anything on it; and, when it 
learns to know itself, it -- or rather: whoever of the rabble learns to know 
himself, he -- casts off the rabble-quality in refusing your alms with thanks. 
But it remains ridiculous that you declare the rabble ``sinful and 
criminal'' if it is not pleased to live from your favors because it can do 
something in its own favor. Your bestowals cheat it and put it off. Defend 
your property, then you will be strong; if, on the other hand, you want to 
retain your ability to bestow, and perhaps actually have the more political 
rights the more alms (poor-rates) you can give, this will work just as long as 
the recipients let you work it.\footnote{In a registration bill for Ireland 
the government made the proposal to let those be electors who pay \pounds{}5 
sterling of poor-rates. He who gives alms, therefore, acquires political 
rights, or elsewhere becomes a swan-knight. [See p. 342.]}

In short, the property question cannot be solved so amicably as the 
Socialists, yes, even the Communists, dream. It is solved only by the war of 
all against all. The poor become free and proprietors only when they -- 
\textit{rise}. Bestow ever so much on them, they will still always want more; 
for they want nothing less than that at last -- nothing more be bestowed.

It will be asked, but how then will it be when the have- nots take heart? Of 
what sort is the settlement to be? One might as well ask that I cast a child's 
nativity. What a slave will do as soon as he has broken his fetters, one must 
--await.

In Kaiser's pamphlet, worthless for lack of form as well as substance 
(\textit{``Die Pers\"onlichkeit des Eigent\"umers in Bezug auf den 
Socialismus und Communismus},'' etc.), he hopes from the \textit{State} that 
it will bring about a leveling of property. Always the State! Herr Papa! As 
the Church was proclaimed and looked upon as the ``mother'' of believers, so 
the State has altogether the face of the provident father.

\myhrule


\textit{Competition} shows itself most strictly connected with the principle 
of civism. Is it anything else than \textit{equality} (\textit{\'egalit\'e})? 
And is not equality a product of that same Revolution which was brought on by 
the commonalty, the middle classes? As no one is barred from competing with 
all in the State (except the prince, because he represents the State itself) 
and working himself up to their height, yes, overthrowing or exploiting them 
for his own advantage, soaring above them and by stronger exertion depriving 
them of their favorable circumstances -- this serves as a clear proof that 
before the State's judgment-seat every one has only the value of a ``simple 
individual'' and may not count on any favoritism. Outrun and outbid each 
other as much as you like and can; that shall not trouble me, the State! Among 
yourselves you are free in competing, you are competitors; that is your 
\textit{social} position. But before me, the State, you are nothing but 
``simple individuals''!\footnote{Minister Stein used this expression about 
Count von Reisach, when he cold-bloodedly left the latter at the mercy of the 
Bavarian government because to him, as he said, ``a government like Bavaria 
must be worth more than a simple individual.'' Reisach had written against 
Montgelas at Stein's bidding, and Stein later agreed to the giving up of 
Reisach, which was demanded by Montgelas on account of this very book. See 
Hinrichs, \textit{``Politische Vorlesungen},'' I, 280.}

What in the form of principle or theory was propounded as the equality of all 
has found here in competition its realization and practical carrying out; for 
\textit{\'egalit\'e} is -- free competition. All are, before the State 
--simple individuals; in society, or in relation to each other -- competitors.

I need be nothing further than a simple individual to be able to compete with 
all others aside from the prince and his family: a freedom which formerly was 
made impossible by the fact that only by means of one's corporation, and 
within it, did one enjoy any freedom of effort.

In the guild and feudality the State is in an intolerant and fastidious 
attitude, granting \textit{privileges;} in competition and liberalism it is in 
a tolerant and indulgent attitude, granting only \textit{patents} (letters 
assuring the applicant that the business stands open (patent) to him) or 
``concessions.'' Now, as the State has thus left everything to the 
\textit{applicants}, it must come in conflict with all, because each and all 
are entitled to make application. It will be ``stormed,'' and will go down 
in this storm.

Is ``free competition'' then really ``free?'' nay, is it really a 
``competition'' -- to wit, one of \textit{persons --} as it gives itself out 
to be because on this title it bases its right? It originated, you know, in 
persons becoming free of all personal rule. Is a competition ``free'' which 
the State, this ruler in the civic principle, hems in by a thousand barriers? 
There is a rich manufacturer doing a brilliant business, and I should like to 
compete with him. ``Go ahead,'' says the State, ``I have no objection to 
make to your \textit{person} as competitor.'' Yes, I reply, but for that I 
need a space for buildings, I need money! ``That's bad; but, if you have no 
money, you cannot compete. You must not take anything from anybody, for I 
protect property and grant it privileges.'' Free competition is not 
``free,'' because I lack the THINGS for competition. Against my 
\textit{person} no objection can be made, but because I have not the things my 
person too must step to the rear. And who has the necessary things? Perhaps 
that manufacturer? Why, from him I could take them away! No, the State has 
them as property, the manufacturer only as fief, as possession.

But, since it is no use trying it with the manufacturer, I will compete with 
that professor of jurisprudence; the man is a booby, and I, who know a hundred 
times more than he, shall make his class-room empty. ``Have you studied and 
graduated, friend?'' No, but what of that? I understand abundantly what is 
necessary for instruction in that department. ``Sorry, but competition is not 
'free' here. Against your person there is nothing to be said, but the 
\textit{thing}, the doctor's diploma, is lacking. And this diploma I, the 
State, demand. Ask me for it respectfully first; then we will see what is to 
be done.''

This, therefore, is the ``freedom'' of competition. The State, \textit{my 
lord}, first qualifies me to compete.

But do \textit{persons} really compete? No, again \textit{things} only! Moneys 
in the first place, etc.

In the rivalry one will always be left behind another (\textit{e.g.} a 
poetaster behind a poet). But it makes a difference whether the means that the 
unlucky competitor lacks are personal or material, and likewise whether the 
material means can be won by \textit{personal energy} or are to be obtained 
only by \textit{grace}, only as a present; as when \textit{e.g.} the poorer 
man must leave, \textit{i.e.} present, to the rich man his riches. But, if I 
must all along wait for the State's \textit{approval} to obtain or to use 
(\textit{e.g.} in the case of graduation) the means, I have the means by the 
\textit{grace of the State}.\footnote{In colleges and universities poor men 
compete with rich. But they are able to do in most eases only through 
scholarships, which -- a significant point -- almost all come down to us from 
a time when free competition was still far from being a controlling principle. 
The principle of competition founds no scholarship, but says, Help yourself; 
provide yourself the means. What the State gives for such purposes it pays out 
from interested motives, to educate ``servants'' for itself.}

Free competition, therefore, has only the following meaning: To the State all 
rank as its equal children, and every one can scud and run to earn the 
\textit{State's goods and largesse}. Therefore all do chase after havings, 
holdings, possessions (be it of money or offices, titles of honor, etc.), 
after the \textit{things}.

In the mind of the commonalty every one is possessor or ``owner.'' Now, 
whence comes it that the most have in fact next to nothing? From this, that 
the most are already joyful over being possessors at all, even though it be of 
some rags, as children are joyful in their first trousers or even the first 
penny that is presented to them. More precisely, however, the matter is to be 
taken as follows. Liberalism came forward at once with the declaration that it 
belonged to man's essence not to be property, but proprietor. As the 
consideration here was about ``man,'' not about the individual, the how-much 
(which formed exactly the point of the individual's special interest) was left 
to him. Hence the individual's egoism retained room for the freest play in 
this how- much, and carried on an indefatigable competition.

However, the lucky egoism had to become a snag in the way of the less 
fortunate, and the latter, still keeping its feet planted on the principle of 
humanity, put forward the question as to how-much of possession, and answered 
it to the effect that ``man must have as much as he requires.''

Will it be possible for \textit{my} egoism to let itself be satisfied with 
that? What ``man'' requires furnishes by no means a scale for measuring me 
and my needs; for I may have use for less or more. I must rather have so much 
as I am competent to appropriate.

Competition suffers from the unfavorable circumstance that the \textit{means} 
for competing are not at every one's command, because they are not taken from 
personality, but from accident. Most are \textit{without means}, and for this 
reason \textit{without goods}.

Hence the Socialists demand the \textit{means} for all, and aim at a society 
that shall offer means. Your money value, say they, we no longer recognize as 
your ``competence''; you must show another competence -- to wit, your 
\textit{working force}. In the possession of a property, or as 
``possessor,'' man does certainly show himself as man; it was for this 
reason that we let the possessor, whom we called ``proprietor,'' keep his 
standing so long. Yet you possess the things only so long as you are not 
``put out of this property.''

The possessor is competent, but only so far as the others are incompetent. 
Since your ware forms your competence only so long as you are competent to 
defend it (\textit{i.e.} as \textit{we} are not competent to do anything with 
it), look about you for another competence; for we now, by our might, surpass 
your alleged competence.

It was an extraordinarily large gain made, when the point of being regarded as 
possessors was put through. Therein bondservice was abolished, and every one 
who till then had been bound to the lord's service, and more or less had been 
his property, now became a ``lord.'' But henceforth your having, and what 
you have, are no longer adequate and no longer recognized; \textit{per 
contra}, your working and your work rise in value. We now respect your 
\textit{subduing} things, as we formerly did your possessing them. Your work 
is your competence! You are lord or possessor only of what comes by 
\textit{work}, not by \textit{inheritance}. But as at the time everything has 
come by inheritance, and every copper that you possess bears not a labor-stamp 
but an inheritance-stamp, everything must be melted over.

But is my work then really, as the Communists suppose, my sole competence? or 
does not this consist rather in everything that I am competent for? And does 
not the workers' society itself have to concede this, \textit{e.g.,} in 
supporting also the sick, children, old men -- in short, those who are 
incapable of work? These are still competent for a good deal, \textit{e.g.} 
for instance, to preserve their life instead of taking it. If they are 
competent to cause you to desire their continued existence, they have a power 
over you. To him who exercised utterly no power over you, you would vouchsafe 
nothing; he might perish.

Therefore, what you are \textit{competent} for is your \textit{competence!} If 
you are competent to furnish pleasure to thousands, then thousands will pay 
you an honorarium for it; for it would stand in your power to forbear doing 
it, hence they must purchase your deed. If you are not competent to 
\textit{captivate} any one, you may simply starve.

Now am I, who am competent for much, perchance to have no advantage over the 
less competent?

We are all in the midst of abundance; now shall I not help myself as well as I 
can, but only wait and see how much is left me in an equal division?

Against competition there rises up the principle of ragamuffin society -- 
\textit{partition}.

To be looked upon as a mere \textit{part}, part of society, the individual 
cannot bear -- because he is \textit{more;} his uniqueness puts from it this 
limited conception.

Hence he does not await his competence from the sharing of others, and even in 
the workers' society there arises the misgiving that in an equal partition the 
strong will be exploited by the weak; he awaits his competence rather from 
himself, and says now, what I am competent to have, that is my competence.

What competence does not the child possess in its smiling, its playing, its 
screaming! in short, in its mere existence! Are you capable of resisting its 
desire? Or do you not hold out to it, as mother, your breast; as father, as 
much of your possessions as it needs? It compels you, therefore it possesses 
what you call yours.

If your person is of consequence to me, you pay me with your very existence; 
if I am concerned only with one of your qualities, then your compliance, 
perhaps, or your aid, has a value (a money value) for me, and I 
\textit{purchase} it.

If you do not know how to give yourself any other than a money value in my 
estimation, there may arise the case of which history tells us, that Germans, 
sons of the fatherland, were sold to America. Should those who let themselves 
to be traded in be worth more to the seller? He preferred the cash to this 
living ware that did not understand how to make itself precious to him. That 
he discovered nothing more valuable in it was assuredly a defect of his 
competence; but it takes a rogue to give more than he has. How should he show 
respect when he did not have it, nay, hardly could have it for such a pack!

You behave egoistically when you respect each other neither as possessors nor 
as ragamuffins or workers, but as a part of your competence, as 
\textit{``useful bodies''}. Then you will neither give anything to the 
possessor (``proprietor'') for his possessions, nor to him who works, but 
only to him whom you \textit{require}. The North Americans ask themselves, Do 
we require a king? and answer, Not a farthing are he and his work worth to us.

If it is said that competition throws every thing open to all, the expression 
is not accurate, and it is better put thus: competition makes everything 
purchasable. In \textit{abandoning}\footnote{[\textit{preisgeben}]} it to 
them, competition leaves it to their appraisal\footnote{[\textit{Preis}]} or 
their estimation, and demands a price\footnote{[\textit{Preis}]} for it.

But the would-be buyers mostly lack the means to make themselves buyers: they 
have no money. For money, then, the purchasable things are indeed to be had 
(``For money everything is to be had!''), but it is exactly money that is 
lacking. Where is one to get money, this current or circulating property? Know 
then, you have as much money\footnote{[\textit{Geld}]} as you have -- might; 
for you count\footnote{[\textit{gelten}]} for as much as you make yourself 
count for.

One pays not with money, of which there may come a lack, but with his 
competence, by which alone we are ``competent'';\footnote{[Equivalent in 
ordinary German use to our ``possessed of a competence.'']} for one is 
proprietor only so far as the arm of our power reaches.

Weitling has thought out a new means of payment -- work. But the true means of 
payment remains, as always, \textit{competence}. With what you have ``within 
your competence'' you pay. Therefore think on the enlargement of your 
competence.

This being admitted, they are nevertheless right on hand again with the motto, 
``To each according to his competence!'' Who is to \textit{give} to me 
according to my competence? Society? Then I should have to put up with its 
estimation. Rather, I shall \textit{take} according to my competence.

``All belongs to all!'' This proposition springs from the same unsubstantial 
theory. To each belongs only what he is competent for. If I say, The world 
belongs to me, properly that too is empty talk, which has a meaning only in so 
far as I respect no alien property. But to me belongs only as much as I am 
competent for, or have within my competence.

One is not worthy to have what one, through weakness, lets be taken from him; 
one is not worthy of it because one is not capable of it.

They raise a mighty uproar over the ``wrong of a thousand years'' which is 
being committed by the rich against the poor. As if the rich were to blame for 
poverty, and the poor were not in like manner responsible for riches! Is there 
another difference between the two than that of competence and incompetence, 
of the competent and incompetent? Wherein, pray, does the crime of the rich 
consist? ``In their hardheartedness.'' But who then have maintained the 
poor? Who have cared for their nourishment? Who have given alms, those alms 
that have even their name from mercy (\textit{eleemosyne})? Have not the rich 
been ``merciful'' at all times? Are they not to this day 
``tender-hearted,'' as poor-taxes, hospitals, foundations of all sorts, 
etc., prove?

But all this does not satisfy you! Doubtless, then, they are to \textit{share} 
with the poor? Now you are demanding that they shall abolish poverty. Aside 
from the point that there might be hardly one among you who would act so, and 
that this one would be a fool for it, do ask yourselves: why should the rich 
let go their fleeces and give up \textit{themselves}, thereby pursuing the 
advantage of the poor rather than their own? You, who have your thaler daily, 
are rich above thousands who live on four groschen. Is it for your interest to 
share with the thousands, or is it not rather for theirs? --

With competition is connected less the intention to do the thing \textit{best} 
than the intention to make it as \textit{profitable}, as productive, as 
possible. Hence people study to get into the civil service (pot-boiling 
study), study cringing and flattery, routine and ``acquaintance with 
business,'' work ``for appearance.'' Hence, while it is apparently a matter 
of doing ``good service,'' in truth only a ``good business'' and earning 
of money are looked out for. The job is done only ostensibly for the job's 
sake, but in fact on account of the gain that it yields. One would indeed 
prefer not to be censor, but one wants to be -- advanced; one would like to 
judge, administer, etc., according to his best convictions, but one is afraid 
of transference or even dismissal; one must, above all things -- live.

Thus these goings-on are a fight for \textit{dear life}, and, in gradation 
upward, for more or less of a ``good living.''

And yet, withal, their whole round of toil and care brings in for most only 
``bitter life'' and ``bitter poverty.'' All the bitter painstaking for 
this!

Restless acquisition does not let us take breath, take a calm 
\textit{enjoyment:} we do not get the comfort of our possessions.

But the organization of labor touches only such labors as others can do for 
us, slaughtering, tillage, etc.; the rest remain egoistic, because no one can 
in your stead elaborate your musical compositions, carry out your projects of 
painting, etc.; nobody can replace Raphael's labors. The latter are labors of 
a unique person,\footnote{[\textit{Einzige}]} which only he is competent to 
achieve, while the former deserved to be called ``human,'' since what is 
anybody's \textit{own} in them is of slight account, and almost ``any man'' 
can be trained to it.

Now, as society can regard only labors for the common benefit, \textit{human} 
labors, he who does anything \textit{unique} remains without its care; nay, he 
may find himself disturbed by its intervention. The unique person will work 
himself forth out of society all right, but society brings forth no unique 
person.

Hence it is at any rate helpful that we come to an agreement about 
\textit{human} labors, that they may not, as under competition, claim all our 
time and toil. So far Communism will bear its fruits. For before the dominion 
of the commonalty even that for which all men are qualified, or can be 
qualified, was tied up to a few and withheld from the rest: it was a 
privilege. To the commonalty it looked equitable to leave free all that seemed 
to exist for every ``man.'' But, because left\footnote{[Literally, 
``given.'']} free, it was yet given to no one, but rather left to each to be 
got hold of by his \textit{human} power. By this the mind was turned to the 
acquisition of the human, which henceforth beckoned to every one; and there 
arose a movement which one hears so loudly bemoaned under the name of 
``materialism.''

Communism seeks to check its course, spreading the belief that the human is 
not worth so much discomfort, and, with sensible arrangements, could be gained 
without the great expense of time and powers which has hitherto seemed 
requisite.

But for whom is time to be gained? For what does man require more time than is 
necessary to refresh his wearied powers of labor? Here Communism is silent.

For what? To take comfort in himself as the unique, after he has done his part 
as man!

In the first joy over being allowed to stretch out their hands toward 
everything human, people forgot to want anything else; and they competed away 
vigorously, as if the possession of the human were the goal of all our wishes.

But they have run themselves tired, and are gradually noticing that 
``possession does not give happiness.'' Therefore they are thinking of 
obtaining the necessary by an easier bargain, and spending on it only so much 
time and toil as its indispensableness exacts. Riches fall in price, and 
contented poverty, the care-free ragamuffin, becomes the seductive ideal.

Should such human activities, that every one is confident of his capacity for, 
be highly salaried, and sought for with toil and expenditure of all 
life-forces? Even in the everyday form of speech, ``If I were minister, or 
even the., then it should go quite otherwise,'' that confidence expresses 
itself -- that one holds himself capable of playing the part of such a 
dignitary; one does get a perception that to things of this sort there belongs 
not uniqueness, but only a culture which is attainable, even if not exactly by 
all, at any rate by many; \textit{i.e.} that for such a thing one need only be 
an ordinary man.

If we assume that, as \textit{order} belongs to the essence of the State, so 
\textit{subordination} too is founded in its nature, then we see that the 
subordinates, or those who have received preferment, disproportionately 
\textit{overcharge} and \textit{overreach} those who are put in the lower 
ranks. But the latter take heart (first from the Socialist standpoint, but 
certainly with egoistic consciousness later, of which we will therefore at 
once give their speech some coloring) for the question, By what then is your 
property secure, you creatures of preferment? -- and give themselves the 
answer, By our refraining from interference! And so by \textit{our} 
protection! And what do you give us for it? Kicks and disdain you give to the 
``common people''; police supervision, and a catechism with the chief 
sentence ``Respect what is \textit{not yours}, what belongs to 
\textit{others!} respect others, and especially your superiors!'' But we 
reply, ``If you want our respect, \textit{buy} it for a price agreeable to 
us. We will leave you your property, if you give a due equivalent for this 
leaving.'' Really, what equivalent does the general in time of peace give for 
the many thousands of his yearly income.? -- another for the sheer 
hundred-thousands and millions yearly? What equivalent do you give for our 
chewing potatoes and looking calmly on while you swallow oysters? Only buy the 
oysters of us as dear as we have to buy the potatoes of you, then you may go 
on eating them. Or do you suppose the oysters do not belong to us as much as 
to you? You will make an outcry over \textit{violence} if we reach out our 
hands and help consume them, and you are right. Without violence we do not get 
them, as you no less have them by doing violence to us.

But take the oysters and have done with it, and let us consider our nearer 
property, labor; for the other is only possession. We distress ourselves 
twelve hours in the sweat of our face, and you offer us a few groschen for it. 
Then take the like for your labor too. Are you not willing? You fancy that our 
labor is richly repaid with that wage, while yours on the other hands is worth 
a wage of many thousands. But, if you did not rate yours so high, and gave us 
a better chance to realize value from ours, then we might well, if the case 
demanded it, bring to pass still more important things than you do for the 
many thousand thalers; and, if you got only such wages as we, you would soon 
grow more industrious in order to receive more. But, if you render any service 
that seems to us worth ten and a hundred times more than our own labor, why, 
then you shall get a hundred times more for it too; we, on the other hand, 
think also to produce for you things for which you will requite us more highly 
than with the ordinary day's wages. We shall be willing to get along with each 
other all right, if only we have first agreed on this -- that neither any 
longer needs to -- \textit{present} anything to the other. Then we may perhaps 
actually go so far as to pay even the cripples and sick and old an appropriate 
price for not parting from us by hunger and want; for, if we want them to 
live, it is fitting also that we -- purchase the fulfillment of our will. I 
say ``purchase,'' and therefore do not mean a wretched ``alms.'' For their 
life is the property even of those who cannot work; if we (no matter for what 
reason) want them not to withdraw this life from us, we can mean to bring this 
to pass only by purchase; nay, we shall perhaps (maybe because we like to have 
friendly faces about us) even want a life of comfort for them. In short, we 
want nothing presented by you, but neither will we present you with anything. 
For centuries we have handed alms to you from goodhearted -- stupidity, have 
doled out the mite of the poor and given to the masters the things that are -- 
not the masters'; now just open your wallet, for henceforth our ware rises in 
price quite enormously. We do not want to take from you anything, anything at 
all, only you are to pay better for what you want to have. What then have you? 
``I have an estate of a thousand acres.'' And I am your plowman, and will 
henceforth attend to your fields only for one thaler a day wages. ``Then I'll 
take another.'' You won't find any, for we plowmen are no longer doing 
otherwise, and, if one puts in an appearance who takes less, then let him 
beware of us. There is the housemaid, she too is now demanding as much, and 
you will no longer find one below this price. ``Why, then it is all over with 
me.'' Not so fast! You will doubtless take in as much as we; and, if it 
should not be so, we will take off so much that you shall have wherewith to 
live like us. ``But I am accustomed to live better.'' We have nothing 
against that, but it is not our look-out; if you can clear more, go ahead. Are 
we to hire out under rates, that you may have a good living?

The rich man always puts off the poor with the words, ``What does your want 
concern me? See to it how you make your way through the world; that is 
\textit{your affair}, not mine.'' Well, let us let it be our affair, then, 
and let us not let the means that we have to realize value from ourselves be 
pilfered from us by the rich. ``But you uncultured people really do not need 
so much.'' Well, we are taking somewhat more in order that for it we may 
procure the culture that we perhaps need. ``But, if you thus bring down the 
rich, who is then to support the arts and sciences hereafter?'' Oh, well, we 
must make it up by numbers; we club together, that gives a nice little sum -- 
besides, you rich men now buy only the most tasteless books and the most 
lamentable Madonnas or a pair of lively dancer's legs. ``O ill-starred 
equality!'' No, my good old sir, nothing of equality. We only want to count 
for what we are worth, and, if you are worth more, you shall count for more 
right along. We only want to be \textit{worth our price}, and think to show 
ourselves worth the price that you will pay.

Is the State likely to be able to awaken so secure a temper and so forceful a 
self-consciousness in the menial? Can it make man feel himself? Nay, may it 
even do so much as set this goal for itself? Can it want the individual to 
recognize his value and realize this value from himself? Let us keep the parts 
of the double question separate, and see first whether the State can bring 
about such a thing. As the unanimity of the plowmen is required, only this 
unanimity can bring it to pass, and a State law would be evaded in a thousand 
ways by competition and in secret.

But can the State bear with it? The State cannot possibly bear with people's 
suffering coercion from another than it; it could not, therefore, admit the 
self-help of the unanimous plowmen against those who want to engage for lower 
wages. Suppose, however, that the State made the law, and all the plowmen were 
in accord with it: could the State bear with it then?

In the isolated case -- yes; but the isolated case is more than that, it is a 
case of \textit{principle}. The question therein is of the whole range of the 
\textit{ego's self-realization of value from himself}, and therefore also of 
his self-consciousness \textit{against} the State. So far the Communists keep 
company; but, as self-realization of value from self necessarily directs 
itself against the State, so it does against \textit{society} too, and 
therewith reaches out beyond the commune and the communistic -- out of egoism.

Communism makes the maxim of the commonalty, that every one is a possessor 
(``proprietor''), into an irrefragable truth, into a reality, since the 
anxiety about \textit{obtaining} now ceases and every one \textit{has} from 
the start what he requires. In his labor-force he \textit{has} his competence, 
and, if he makes no use of it, that is his fault. The grasping and hounding is 
at an end, and no competition is left (as so often now) without fruit, because 
with every stroke of labor an adequate supply of the needful is brought into 
the house. Now for the first time one is a \textit{real possessor}, because 
what one has in his labor-force can no longer escape from him as it was 
continually threatening to do under the system of competition. One is a 
\textit{care-free} and assured possessor. And one is this precisely by seeking 
his competence no longer in a ware, but in his own labor, his competence for 
labor; and therefore by being a \textit{ragamuffin}, a man of only ideal 
wealth. \textit{I}, however, cannot content myself with the little that I 
scrape up by my competence for labor, because my competence does not consist 
merely in my labor.

By labor I can perform the official functions of a president, a minister, 
etc.; these offices demand only a general culture -- to wit, such a culture as 
is generally attainable (for general culture is not merely that which every 
one has attained, but broadly that which every one can attain, and therefore 
every special culture, \textit{e.g.} medical, military, philological, of 
which no ``cultivated man'' believes that they surpass his powers), or, 
broadly, only a skill possible to all.

But, even if these offices may vest in every one, yet it is only the 
individual's unique force, peculiar to him alone. that gives them, so to 
speak, life and significance. That he does not manage his office like an 
``ordinary man.'' but puts in the competence of his uniqueness, this he is 
not yet paid for when he is paid only in general as an official or a minister. 
If he has done it so as to earn your thanks, and you wish to retain this 
thank-worthy force of the unique one, you must not pay him like a mere man who 
performed only what was human, but as one who accomplishes what is unique. Do 
the like with your labor, do!

There cannot be a general schedule-price fixed for my uniqueness as there can 
for what I do as man. Only for the latter can a schedule-price be set.

Go right on, then, setting up a general appraisal for human labors, but do not 
deprive your uniqueness of its desert.

\textit{Human} or \textit{general} needs can be satisfied through society; for 
satisfaction of \textit{unique} needs you must do some seeking. A friend and a 
friendly service, or even an individual's service, society cannot procure you. 
And yet you will every moment be in need of such a service, and on the 
slightest occasions require somebody who is helpful to you. Therefore do not 
rely on society, but see to it that you have the wherewithal to -- purchase 
the fulfillment of your wishes.

Whether money is to be retained among egoists? To the old stamp an inherited 
possession adheres. If you no longer let yourselves be paid with it, it is 
ruined: if you do nothing for this money, it loses all power. Cancel the 
\textit{inheritance}, and you have broken off the executor's court-seal. For 
now everything is an inheritance, whether it be already inherited or await its 
heir. If it is yours, wherefore do you let it be sealed up from you? Why do 
you respect the seal?

But why should you not create a new money? Do you then annihilate the ware in 
taking from it the hereditary stamp? Now, money is a ware, and an essential 
\textit{means} or competence. For it protects against the ossification of 
resources, keeps them in flux and brings to pass their exchange. If you know a 
better medium of exchange, go ahead; yet it will be a ``money'' again. It is 
not the money that does you damage, but your incompetence to take it. Let your 
competence take effect, collect yourselves, and there will be no lack of money -- of your money, the money of \textit{your} stamp. But working I do not call 
``letting your competence take effect.'' Those who are only ``looking for 
work'' and ``willing to work hard'' are preparing for their own selves the 
infallible upshot -- to be out of work.

Good and bad luck depend on money. It is a power in the \textit{bourgeois} 
period for this reason, that it is only wooed on all hands like a girl, 
indissolubly wedded by nobody. All the romance and chivalry of \textit{wooing} 
for a dear object come to life again in competition. Money, an object of 
longing, is carried off by the bold ``knights of industry.''\footnote{[A 
German phrase for sharpers.]}

He who has luck takes home the bride. The ragamuffin has luck; he takes her 
into his household, ``society,'' and destroys the virgin. In his house she 
is no longer bride, but wife; and with her virginity her family name is also 
lost. As housewife the maiden Money is called ``Labor,'' for ``Labor'' is 
her husband's name. She is a possession of her husband's.

To bring this figure to an end, the child of Labor and Money is again a girl, 
an unwedded one and therefore Money but with the certain descent from Labor, 
her father. The form of the face, the ``effigy,'' bears another stamp.

Finally, as regards competition once more, it has a continued existence by 
this very means, that all do not attend to \textit{their affair} and come to 
an \textit{understanding} with each other about it. Bread \textit{e.g.} is a 
need of all the inhabitants of a city; therefore they might easily agree on 
setting up a public bakery. Instead of this, they leave the furnishing of the 
needful to the competing bakers. Just so meat to the butchers, wine to 
wine-dealers, etc.

Abolishing competition is not equivalent to favoring the guild. The difference 
is this: In the \textit{guild} baking, etc., is the affair of the 
guild-brothers; in \textit{competition}, the affair of chance competitors; in 
the \textit{union}, of those who require baked goods, and therefore my affair, 
yours, the affair of neither the guildic nor the concessionary baker, but the 
affair of the \textit{united}.

If \textit{I} do not trouble myself about my affair, I must be 
\textit{content} with what it pleases others to vouchsafe me. To have bread is 
my affair, my wish and desire, and yet people leave that to the bakers and 
hope at most to obtain through their wrangling, their getting ahead of each 
other, their rivalry --in short, their competition -- an advantage which one 
could not count on in the case of the guild-brothers who were lodged 
\textit{entirely} and \textit{alone} in the proprietorship of the baking 
franchise. -- What every one requires, every one should also take a hand in 
procuring and producing; it is \textit{his} affair, his property, not the 
property of the guildic or concessionary master.

Let us look back once more. The world belongs to the children of this world, 
the children of men; it is no longer God's world, but man's. As much as every 
man can procure of it, let him call his; only the true man, the State, human 
society or mankind, will look to it that each shall make nothing else his own 
than what he appropriates as man, \textit{i.e.} in human fashion. Unhuman 
appropriation is that which is not consented to by man, \textit{i.e.}, it is a 
``criminal'' appropriation, as the human, \textit{vice versa}, is a 
``rightful'' one, one acquired in the ``way of law.''

So they talk since the Revolution.

But my property is not a thing, since this has an existence independent of me; 
only my might is my own. Not this tree, but my might or control over it, is 
what is mine.

Now, how is this might perversely expressed? They say I have a \textit{right} 
to this tree, or it is my \textit{rightful} property. So I have 
\textit{earned} it by might. That the might must last in order that the tree 
may also be \textit{held --} or better, that the might is not a thing existing 
of itself, but has existence solely in the \textit{mighty ego}, in me the 
mighty -- is forgotten. Might, like other of my \textit{qualities} (\textit{e. 
g.} humanity, majesty, etc.), is exalted to something existing of itself, so 
that it still exists long after it has ceased to be \textit{my} might. Thus 
transformed into a ghost, might is -- \textit{right}. This 
\textit{eternalized} might is not extinguished even with my death, but is 
transferred to ``bequeathed.''

Things now really belong not to me, but to right.

On the other side, this is nothing but a hallucination of vision. For the 
individual's might becomes permanent and a right only by others joining their 
might with his. The delusion consists in their believing that they cannot 
withdraw their might. The same phenomenon over again; might is separated from 
me. I cannot take back the might that I gave to the possessor. One has 
``granted power of attorney,'' has given away his power, has renounced 
coming to a better mind.

The proprietor can give up his might and his right to a thing by giving the 
thing away, squandering it, etc. And \textit{we} should not be able likewise 
to let go the might that we lend to him?

The rightful man, the \textit{just}, desires to call nothing his own that he 
does not have ``rightly'' or have the right to, and therefore only 
\textit{legitimate property}.

Now, who is to be judge, and adjudge his right to him? At last, surely, Man, 
who imparts to him the rights of man: then he can say, in an infinitely 
broader sense than Terence, \textit{humani nihil a me alienum puto}, 
\textit{e.g.}, \textit{the human is my property}. However he may go about it, 
so long as he occupies this standpoint he cannot get clear of a judge; and in 
our time the multifarious judges that had been selected have set themselves 
against each other in two persons at deadly enmity -- to wit, in God and Man. 
The one party appeal to divine right, the other to human right or the rights 
of man.

So much is clear, that in neither case does the individual do the entitling 
himself.

Just pick me out an action today that would not be a violation of right! Every 
moment the rights of man are trampled under foot by one side, while their 
opponents cannot open their mouth without uttering a blasphemy against divine 
right. Give an alms, you mock at a right of man, because the relation of 
beggar and benefactor is an inhuman relation; utter a doubt, you sin against a 
divine right. Eat dry bread with contentment, you violate the right of man by 
your equanimity; eat it with discontent, you revile divine right by your 
repining. There is not one among you who does not commit a crime at every 
moment; your speeches are crimes, and every hindrance to your freedom of 
speech is no less a crime. Ye are criminals altogether!

Yet you are so only in that you all stand on the \textit{ground of right}, 
\textit{i.e.} in that you do not even know, and understand how to value, the 
fact that you are criminals.

Inviolable or \textit{sacred} property has grown on this very ground: it is a 
\textit{juridical concept}.

A dog sees the bone in another's power, -- and stands off only if it feels 
itself too weak. But man respects the other's \textit{right} to his bone. The 
latter action, therefore, ranks as \textit{human}, the former as 
\textit{brutal} or ``egoistic.''

And as here, so in general, it is called \textit{``human''} when one sees in 
everything something \textit{spiritual} (here right), \textit{i.e.} makes 
everything a ghost and takes his attitude toward it as toward a ghost, which 
one can indeed scare away at its appearance, but cannot kill. It is human to 
look at what is individual not as individual, but as a generality.

In nature as such I no longer respect anything, but know myself to be entitled 
to everything against it; in the tree in that garden, on the other hand, I 
must respect \textit{alienness} (they say in one-sided fashion 
``property''), I must keep my hand off it. This comes to an end only when I 
can indeed leave that tree to another as I leave my stick. etc., to another, 
but do not in advance regard it as alien to me, \textit{i.e.} sacred. Rather, 
I make to myself no \textit{crime} of felling it if I will, and it remains my 
property, however long as I resign it to others: it is and remains 
\textit{mine}. In the banker's fortune I as little see anything alien as 
Napoleon did in the territories of kings: we have no \textit{dread} of 
\textit{``conquering''} it, and we look about us also for the means thereto. 
We strip off from it, therefore, the \textit{spirit} of \textit{alienness}, of 
which we had been afraid.

Therefore it is necessary that I do not lay claim to, anything more \textit{as 
man}, but to everything as I, this I; and accordingly to nothing human, but to 
mine; \textit{i.e.}, nothing that pertains to me as man, but -- what I will 
and because I will it.

Rightful, or legitimate, property of another will be only that which 
\textit{you} are content to recognize as such. If your content ceases, then 
this property has lost legitimacy for you, and you will laugh at absolute 
right to it.

Besides the hitherto discussed property in the limited sense, there is held up 
to our reverent heart another property against which we are far less ``to 
sin.'' This property consists in spiritual goods, in the ``sanctuary of the 
inner nature.'' What a man holds sacred, no other is to gibe at; because, 
untrue as it may be, and zealously as one may ``in loving and modest wise'' 
seek to convince of a true sanctity the man who adheres to it and believes in 
it, yet \textit{the sacred} itself is always to be honored in it: the mistaken 
man does believe in the sacred, even though in an incorrect essence of it, and 
so his belief in the sacred must at least be respected.

In ruder times than ours it was customary to demand a particular faith, and 
devotion to a particular sacred essence, and they did not take the gentlest 
way with those who believed otherwise; since, however, ``freedom of belief'' 
spread itself more and more abroad, the ``jealous God and sole Lord'' 
gradually melted into a pretty general ``supreme being,'' and it satisfied 
humane tolerance if only every one revered ``something sacred.''

Reduced to the most human expression, this sacred essence is ``man himself'' 
and ``the human.'' With the deceptive semblance as if the human were 
altogether our own, and free from all the otherworldliness with which the 
divine is tainted -- yes, as if Man were as much as I or you -- there may 
arise even the proud fancy that the talk is no longer of a ``sacred 
essence'' and that we now feel ourselves everywhere at home and no longer in 
the uncanny,\footnote{[Literally, ``unhomely.'']} \textit{i.e.} in the 
sacred and in sacred awe: in the ecstasy over ``Man discovered at last'' the 
egoistic cry of pain passes unheard, and the spook that has become so intimate 
is taken for our true ego.

But ``Humanus is the saint's name'' (see Goethe), and the humane is only the 
most clarified sanctity.

The egoist makes the reverse declaration. For this precise reason, because you 
hold something sacred, I gibe at you; and, even if I respected everything in 
you, your sanctuary is precisely what I should not respect.

With these opposed views there must also be assumed a contradictory relation 
to spiritual goods: the egoist insults them, the religious man (\textit{i.e.} 
every one who puts his ``essence'' above himself) must consistently -- 
protect them. But what kind of spiritual goods are to be protected, and what 
left unprotected, depends entirely on the concept that one forms of the 
``supreme being''; and he who fears God, \textit{e.g.}, has more to shelter 
than he (the liberal) who fears Man.

In spiritual goods we are (in distinction from the sensuous) injured in a 
spiritual way, and the sin against them consists in a direct 
\textit{desecration}, while against the sensuous a purloining or alienation 
takes place; the goods themselves are robbed of value and of consecration, not 
merely taken away; the sacred is immediately compromised. With the word 
``irreverence'' or ``flippancy'' is designated everything that can be 
committed as \textit{crime} against spiritual goods, \textit{i.e.} against 
everything that is sacred for us; and scoffing, reviling, contempt, doubt, 
etc., are only different shades of \textit{criminal flippancy}.

That desecration can be practiced in the most manifold way is here to be 
passed over, and only that desecration is to be preferentially mentioned which 
threatens the sacred with danger through an \textit{unrestricted press}.

As long as respect is demanded even for one spiritual essence, speech and the 
press must be enthralled in the name of this essence; for just so long the 
egoist might ``trespass'' against it by his \textit{utterances}, from which 
thing he must be hindered by ``due punishment'' at least, if one does not 
prefer to take up the more correct means against it, the preventive use of 
police authority, \textit{e.g.} censorship.

What a sighing for liberty of the press! What then is the press to be 
liberated from? Surely from a dependence, a belonging, and a liability to 
service!

But to liberate himself from that is every one's affair, and it may with 
safety be assumed that, when you have delivered yourself from liability to 
service, that which you compose and write will also belong to you as your 
\textit{own} instead of having been thought and indicted \textit{in} the 
service of some power. What can a believer in Christ say and have printed, 
that should be freer from that belief in Christ than he himself is? If I 
cannot or may not write something, perhaps the primary fault lies with 
\textit{me}. Little as this seems to hit the point, so near is the application 
nevertheless to be found. By a press-law I draw a boundary for my 
publications, or let one be drawn, beyond which wrong and its 
\textit{punishment} follows. I myself \textit{limit} myself.

If the press was to be free, nothing would be so important as precisely its 
liberation from every coercion that could be put on it in the \textit{name of 
a law}. And, that it might come to that, I my own self should have to have 
absolved myself from obedience to the law.

Certainly, the absolute liberty of the press is like every absolute liberty, a 
nonentity. The press can become free from full many a thing, but always only 
from what I too am free from. If we make ourselves free from the sacred, if we 
have become \textit{graceless} and \textit{lawless}, our words too will become 
so.

As little as \textit{we} can be declared clear of every coercion in the world, 
so little can our writing be withdrawn from it. But as free as we are, so free 
we can make it too.

It must therefore become our \textit{own}, instead of, as hitherto, serving a 
spook.

People do not yet know what they mean by their cry for liberty of the press. 
What they ostensibly ask is that the State shall set the press free; but what 
they are really after, without knowing it themselves, is that the press become 
free from the State, or clear of the State. The former is a \textit{petition 
to} the State, the latter an \textit{insurrection against} the State. As a 
``petition for right,'' even as a serious demanding of the right of liberty 
of the press, it presupposes the State as the giver, and can hope only for a 
\textit{present}, a permission, a chartering. Possible, no doubt, that a State 
acts so senselessly as to grant the demanded present; but you may bet 
everything that those who receive the present will not know how to use it so 
long as they regard the State as a truth: they will not trespass against this 
``sacred thing,'' and will call for a penal press-law against every one who 
would be willing to dare this.

In a word, the press does not become free from what I am not free from.

Do I perhaps hereby show myself an opponent of the liberty of the press? On 
the contrary, I only assert that one will never get it if one wants only it, 
the liberty of the press, \textit{i.e.} if one sets out only for an 
unrestricted permission. Only beg right along for this permission: you may 
wait forever for it, for there is no one in the world who could give it to 
you. As long as you want to have yourselves ``entitled'' to the use of the 
press by a permission, \textit{i.e.} liberty of the press, you live in vain 
hope and complaint.

``Nonsense! Why, you yourself, who harbor such thoughts as stand in your 
book, can unfortunately bring them to publicity only through a lucky chance or 
by stealth; nevertheless you will inveigh against one's pressing and 
importuning his own State till it gives the refused permission to print?'' 
But an author thus addressed would perhaps -- for the impudence of such people 
goes far -- give the following reply: ``Consider well what you say! What then 
do I do to procure myself liberty of the press for my book? Do I ask for 
permission, or do I not rather, without any question of legality, seek a 
favorable occasion and grasp it in complete recklessness of the State and its 
wishes? I -- the terrifying word must be uttered -- I cheat the State. You 
unconsciously do the same. From your tribunes you talk it into the idea that 
it must give up its sanctity and inviolability, it must lay itself bare to the 
attacks of writers, without needing on that account to fear danger. But you 
are imposing on it; for its existence is done for as soon as it loses its 
unapproachableness. To \textit{you} indeed it might well accord liberty of 
writing, as England has done; you are \textit{believers in the State} and 
incapable of writing against the State, however much you would like to reform 
it and 'remedy its defects.' But what if opponents of the State availed 
themselves of free utterance, and stormed out against Church, State, morals, 
and everything 'sacred' with inexorable reasons? You would then be the first, 
in terrible agonies, to call into life the \textit{September laws}. Too late 
would you then rue the stupidity that earlier made you so ready to fool and 
palaver into compliance the State, or the government of the State. -- But, I 
prove by my act only two things. This for one, that the liberty of the press 
is always bound to 'favorable opportunities,' and accordingly will never be an 
absolute liberty; but secondly this, that he who would enjoy it must seek out 
and, if possible, create the favorable opportunity, availing himself of his 
\textit{own advantage} against the State; and counting himself and his will 
more than the State and every 'superior' power. Not in the State, but only 
against it, can the liberty of the press be carried through; if it is to be 
established, it is to be obtained not as the consequence of a 
\textit{petition} but as the work of an \textit{insurrection}. Every petition 
and every motion for liberty of the press is already an insurrection, be it 
conscious or unconscious: a thing which Philistine halfness alone will not and 
cannot confess to itself until, with a shrinking shudder, it shall see it 
clearly and irrefutably by the outcome. For the requested liberty of the press 
has indeed a friendly and well-meaning face at the beginning, as it is not in 
the least minded ever to let the 'insolence of the press' come into vogue; but 
little by little its heart grows more hardened, and the inference flatters its 
way in that really a liberty is not a liberty if it stands in the 
\textit{service} of the State, of morals, or of the law. A liberty indeed from 
the coercion of censorship, it is yet not a liberty from the coercion of law. 
The press, once seized by the lust for liberty, always wants to grow freer, 
till at last the writer says to himself, really I am not wholly free till I 
ask about nothing; and writing is free only when it is my \textit{own}, 
dictated to me by no power or authority, by no faith, no dread; the press must 
not be free -- that is too little -- it must be \textit{mine: -- ownness of 
the press} or \textit{property in the press}, that is what I will take.

``Why, liberty of the press is only \textit{permission of the press}, and the 
State never will or can voluntarily permit me to grind it to nothingness by 
the press.''

Let us now, in conclusion, bettering the above language, which is still vague, 
owing to the phrase 'liberty of the press,' rather put it thus: 
\textit{``liberty of the press}, the liberals' loud demand, is assuredly 
possible in the State; yes, it is possible only \textit{in} the State, because 
it is a \textit{permission}, and consequently the permitter (the State) must 
not be lacking. But as permission it has its limit in this very State, which 
surely should not in reason permit more than is compatible with itself and its 
welfare: the State fixes for it this limit as the \textit{law} of its 
existence and of its extension. That one State brooks more than another is 
only a quantitative distinction, which alone, nevertheless, lies at the heart 
of the political liberals: they want in Germany, \textit{i.e.}, only a 
'\textit{more extended, broader} accordance of free utterance.' The liberty of 
the press which is sought for is an affair of the \textit{people's}, and 
before the people (the State) possesses it I may make no use of it. From the 
standpoint of property in the press, the situation is different. Let my 
people, if they will, go without liberty of free press, I will manage to print 
by force or ruse; I get my permission to print only from -- \textit{myself} 
and my strength.

If the press is \textit{my own}, I as little need a permission of the State 
for employing it as I seek that permission in order to blow my nose. The press 
is my \textit{property} from the moment when nothing is more to me than 
myself; for from this moment State, Church, people, society, etc., cease, 
because they have to thank for their existence only the disrespect that I have 
for myself, and with the vanishing of this undervaluation they themselves are 
extinguished: they exist only when they exist \textit{above me}, exist only as 
\textit{powers} and \textit{power-holders}. Or can you imagine a State whose 
citizens one and all think nothing of it? It would be as certainly a dream, an 
existence in seeming, as 'united Germany.'

The press is my own as soon as I myself am my own, a self- owned man: to the 
egoist belongs the world, because he belongs to no power of the world.

With this my press might still be very \textit{unfree}, as \textit{e.g.} at 
this moment. But the world is large, and one helps himself as well as he can. 
If I were willing to abate from the \textit{property} of my press, I could 
easily attain the point where I might everywhere have as much printed as my 
fingers produced. But, as I want to assert my property, I must necessarily 
swindle my enemies. 'Would you not accept their permission if it were given 
you?' Certainly, with joy; for their permission would be to me a proof that I 
had fooled them and started them on the road to ruin. I am not concerned for 
their permission, but so much the more for their folly and their overthrow. I 
do not sue for their permission as if I flattered myself (like the political 
liberals) that we both, they and I, could make out peaceably alongside and 
with each other, yes, probably raise and prop each other; but I sue for it in 
order to make them bleed to death by it, that the permitters themselves may 
cease at last. I act as a conscious enemy, overreaching them and 
\textit{utilizing} their heedlessness.

The press is \textit{mine} when I recognize outside myself no \textit{judge} 
whatever over its utilization, \textit{i.e.} when my writing is no longer 
determined by morality or religion or respect for the State laws or the like, 
but by me and my egoism!''

Now, what have you to reply to him who gives you so impudent an answer? -- We 
shall perhaps put the question most strikingly by phrasing it as follows: 
Whose is the press, the people's (State's) or mine? The politicals on their 
side intend nothing further than to liberate the press from personal and 
arbitrary interferences of the possessors of power, without thinking of the 
point that to be really open for everybody it would also have to be free from 
the laws, from the people's (State's) will. They want to make a ``people's 
affair'' of it.

But, having become the people's property, it is still far from being mine; 
rather, it retains for me the subordinate significance of a 
\textit{permission}. The people plays judge over my thoughts; it has the right 
of calling me to account for them, or, I am responsible to it for them. 
Jurors, when their fixed ideas are attacked, have just as hard heads as the 
stiffest despots and their servile officials.

In the \textit{``Liberale Bestrebungen}''\footnote{II, p. 91ff. (See my note 
above.)} Edgar Bauer asserts that liberty of the press is impossible in the 
absolutist and the constitutional State, whereas in the ``free State'' it 
finds its place. ``Here,'' the statement is, ``it is recognized that the 
individual, because he is no longer an individual but a member of a true and 
rational generality, has the right to utter his mind.'' So not the 
individual, but the ``member,'' has liberty of the press. But, if for the 
purpose of liberty of the press the individual must first give proof of 
himself regarding his belief in the generality, the people; if he does not 
have this liberty \textit{through might of his own --} then it is a 
\textit{people's liberty}, a liberty that he is invested with for the sake of 
his faith, his ``membership.'' The reverse is the case: it is precisely as 
an individual that every one has open to him the liberty to utter his mind. 
But he has not the ``right'': that liberty is assuredly not his ``sacred 
right.'' He has only the \textit{might;} but the might alone makes him owner. 
I need no concession for the liberty of the press, do not need the people's 
consent to it, do not need the ``right'' to it, nor any ``justification.'' 
The liberty of the press too, like every liberty, I must ``take''; the 
people, ``as being the sole judge,'' cannot \textit{give} it to me. It can 
put up with me the liberty that I take, or defend itself against it; give, 
bestow, grant it cannot. I exercise it \textit{despite} the people, purely as 
an individual; \textit{i.e.} I get it by fighting the people, my -- enemy, and 
obtain it only when I really get it by such fighting, \textit{i.e. take} it. 
But I take it because it is my property.

Sander, against whom E. Bauer writes, lays claim (page 99) to the liberty of 
the press ``as the right and the liberty of the \textit{citizens in the 
State''}. What else does Edgar Bauer do? To him also it is only a right of 
the free \textit{citizen}.

The liberty of the press is also demanded under the name of a ``general human 
right.'' Against this the objection was well-founded that not every man knew 
how to use it rightly, for not every individual was truly man. Never did a 
government refuse it to \textit{Man} as such; but \textit{Man} writes nothing, 
for the reason that he is a ghost. It always refused it to 
\textit{individuals} only, and gave it to others, \textit{e.g.} its organs. 
If then one would have it for all, one must assert outright that it is due to 
the individual, me, not to man or to the individual so far as he is man. 
Besides, another than a man (a beast) can make no use of it. The French 
government, \textit{e.g.}, does not dispute the liberty of the press as a 
right of man, but demands from the individual a security for his really being 
man; for it assigns liberty of the press not to the individual, but to man.

Under the exact pretense that it was \textit{not human}, what was mine was 
taken from me! What was human was left to me undiminished.

Liberty of the press can bring about only a \textit{responsible} press; the 
\textit{irresponsible} proceeds solely from property in the press.

\myhrule

For intercourse with men an express law (conformity to which one may venture 
at times sinfully to forget, but the absolute value of which one at no time 
ventures to deny) is placed foremost among all who live religiously: this is 
the law -- of \textit{love}, to which not even those who seem to fight against 
its principle, and who hate its name, have as yet become untrue; for they also 
still have love, yes, they love with a deeper and more sublimated love, they 
love ``man and mankind.''

If we formulate the sense of this law, it will be about as follows: Every man 
must have a something that is more to him than himself. You are to put your 
``private interest'' in the background when it is a question of the welfare 
of others, the weal of the fatherland, of society, the common weal, the weal 
of mankind, the good cause, etc.! Fatherland, society, mankind, must be more 
to you than yourself, and as against their interest your ``private 
interest'' must stand back; for you must not be an --egoist.

Love is a far-reaching religious demand, which is not, as might be supposed, 
limited to love to God and man, but stands foremost in every regard. Whatever 
we do, think, will, the ground of it is always to be love. Thus we may indeed 
judge, but only ``with love.'' The Bible may assuredly be criticized, and 
that very thoroughly, but the critic must before all things \textit{love} it 
and see in it the sacred book. Is this anything else than to say he must not 
criticize it to death, he must leave it standing, and that as a sacred thing 
that cannot be upset? -- In our criticism on men too, love must remain the 
unchanged key-note. Certainly judgments that hatred inspires are not at all 
our \textit{own} judgments, but judgments of the hatred that rules us, 
``rancorous judgments.'' But are judgments that love inspires in us any more 
our \textit{own}? They are judgments of the love that rules us, they are 
``loving, lenient'' judgments, they are not our \textit{own}, and 
accordingly not real judgments at all. He who burns with love for justice 
cries out, \textit{fiat justitia, pereat mundus!} He can doubtless ask and 
investigate what justice properly is or demands, and \textit{in what} it 
consists, but not \textit{whether} it is anything.

It is very true, ``He who abides in love abides in God, and God in him.'' (1 
John 4. 16.) God abides in him, he does not get rid of God, does not become 
godless; and he abides in God, does not come to himself and into his own home, 
abides in love to God and does not become loveless.

``God is love! All times and all races recognize in this word the central 
point of Christianity.'' God, who is love, is an officious God: he cannot 
leave the world in peace, but wants to make it \textit{blest}. ``God became 
man to make men divine.''\footnote{Athanasius.} He has his hand in the game 
everywhere, and nothing happens without it; everywhere he has his ``best 
purposes,'' his ``incomprehensible plans and decrees.'' Reason, which he 
himself is, is to be forwarded and realized in the whole world. His fatherly 
care deprives us of all independence. We can do nothing sensible without its 
being said, God did that, and can bring upon ourselves no misfortune without 
hearing, God ordained that; we have nothing that we have not from him, he 
``gave'' everything. But, as God does, so does Man. God wants perforce to 
make the world \textit{blest}, and Man wants to make it \textit{happy}, to 
make all men happy. Hence every ``man'' wants to awaken in all men the 
reason which he supposes his own self to have: everything is to be rational 
throughout. God torments himself with the devil, and the philosopher does it 
with unreason and the accidental. God lets no being go \textit{its own} gait, 
and Man likewise wants to make us walk only in human wise.

But whoso is full of sacred (religious, moral, humane) love loves only the 
spook, the ``true man,'' and persecutes with dull mercilessness the 
individual, the real man, under the phlegmatic legal title of measures against 
the ``un- man.'' He finds it praiseworthy and indispensable to exercise 
pitilessness in the harshest measure; for love to the spook or generality 
commands him to hate him who is not ghostly, \textit{i.e.} the egoist or 
individual; such is the meaning of the renowned love-phenomenon that is called 
``justice.''

The criminally arraigned man can expect no forbearance, and no one spreads a 
friendly veil over his unhappy nakedness. Without emotion the stern judge 
tears the last rags of excuse from the body of the poor accused; without 
compassion the jailer drags him into his damp abode; without placability, when 
the time of punishment has expired, he thrusts the branded man again among 
men, his good, Christian, loyal brethren, who contemptuously spit on him. Yes, 
without grace a criminal ``deserving of death'' is led to the scaffold, and 
before the eyes of a jubilating crowd the appeased moral law celebrates its 
sublime -- revenge. For only one can live, the moral law or the criminal. 
Where criminals live unpunished, the moral law has fallen; and, where this 
prevails, those must go down. Their enmity is indestructible.

The Christian age is precisely that of \textit{mercy, love}, solicitude to 
have men receive what is due them, yes, to bring them to fulfil their human 
(divine) calling. Therefore the principle has been put foremost for 
intercourse, that this and that is man's essence and consequently his calling, 
to which either God has called him or (according to the concepts of today) his 
being man (the species) calls him. Hence the zeal for conversion. That the 
Communists and the humane expect from man more than the Christians do does not 
change the standpoint in the least. Man shall get what is human! If it was 
enough for the pious that what was divine became his part, the humane demand 
that he be not curtailed of what is human. Both set themselves against what is 
egoistic. Of course; for what is egoistic cannot be accorded to him or vested 
in him (a fief); he must procure it for himself. Love imparts the former, the 
latter can be given to me by myself alone.

Intercourse hitherto has rested on love, \textit{regardful} behavior, doing 
for each other. As one owed it to himself to make himself blessed, or owed 
himself the bliss of taking up into himself the supreme essence and bringing 
it to a \textit{v\'erit\'e} (a truth and reality), so one owed it to 
\textit{others} to help them realize their essence and their calling: in both 
cases one owed it to the essence of man to contribute to its realization.

But one owes it neither to himself to make anything out of himself, nor to 
others to make anything out of them; for one owes nothing to his essence and 
that of others. Intercourse resting on essence is an intercourse with the 
spook, not with anything real. If I hold intercourse with the supreme essence, 
I am not holding intercourse with myself, and, if I hold intercourse with the 
essence of man, I am not holding intercourse with men.

The natural man's love becomes through culture a \textit{commandment}. But as 
commandment it belongs to \textit{Man} as such. not to me; it is my 
\textit{essence},\footnote{[\textit{Wesen}]} about which much 
ado\footnote{[\textit{Wesen}]} is made. not my property. \textit{Man}, 
\textit{i.e.} humanity, presents that demand to me; love \textit{is demanded}, 
it is my \textit{duty}. Instead, therefore, of being really won for 
\textit{me}, it has been won for the generality, \textit{Man}, as his property 
or peculiarity: ``it becomes man, every man, to love; love is the duty and 
calling of man,'' etc.

Consequently I must again vindicate love for \textit{myself}, and deliver it 
out of the power of Man with the great M.

What was originally \textit{mine}, but \textit{accidentally} mine, 
instinctively mine, I was invested with as the property of Man; I became 
feoffee in loving, I became the retainer of mankind, only a specimen of this 
species, and acted, loving, not as \textit{I}, but as \textit{man}, as a 
specimen of man, the humanly. The whole condition of civilization is the 
\textit{feudal system}, the property being Man's or mankind's, not 
\textit{mine}. A monstrous feudal State was founded, the individual robbed of 
everything, everything left to ``man.'' The individual had to appear at last 
as a ``sinner through and through.''

Am I perchance to have no lively interest in the person of another, are 
\textit{his} joy and \textit{his} weal not to lie at my heart, is the 
enjoyment that I furnish him not to be more to me than other enjoyments of my 
own? On the contrary, I can with joy sacrifice to him numberless enjoyments, I 
can deny myself numberless things for the enhancement of \textit{his} 
pleasure, and I can hazard for him what without him was the dearest to me, my 
life, my welfare, my freedom. Why, it constitutes my pleasure and my happiness 
to refresh myself with his happiness and his pleasure. But \textit{myself, my 
own self}, I do not sacrifice to him, but remain an egoist and -- enjoy him. 
If I sacrifice to him everything that but for my love to him I should keep, 
that is very simple, and even more usual in life than it seems to be; but it 
proves nothing further than that this one passion is more powerful in me than 
all the rest. Christianity too teaches us to sacrifice all other passions to 
this. But, if to one passion I sacrifice others, I do not on that account go 
so far as to sacrifice \textit{myself}, nor sacrifice anything of that whereby 
I truly am myself; I do not sacrifice my peculiar value, my \textit{ownness}. 
Where this bad case occurs, love cuts no better figure than any other passion 
that I obey blindly. The ambitious man, who is carried away by ambition and 
remains deaf to every warning that a calm moment begets in him, has let this 
passion grow up into a despot against whom he abandons all power of 
dissolution: he has given up himself, because he cannot \textit{dissolve} 
himself, and consequently cannot absolve himself from the passion: he is 
possessed.

I love men too -- not merely individuals, but every one. But I love them with 
the consciousness of egoism; I love them because love makes \textit{me} happy, 
I love because loving is natural to me, because it pleases me. I know no 
``commandment of love.'' I have a \textit{fellow-feeling} with every feeling 
being, and their torment torments, their refreshment refreshes me too; I can 
kill them, not torture them. \textit{Per contra}, the high-souled, virtuous 
Philistine prince Rudolph in \textit{The Mysteries of Paris}, because the 
wicked provoke his ``indignation,'' plans their torture. That fellow-feeling 
proves only that the feeling of those who feel is mine too, my property; in 
opposition to which the pitiless dealing of the ``righteous'' man 
(\textit{e.g.} against notary Ferrand) is like the unfeelingness of that 
robber [Procrustes] who cut \textit{off} or stretched his prisoners' legs to 
the measure of his bedstead: Rudolph's bedstead, which he cuts men to fit, is 
the concept of the ``good.'' The for right, virtue, etc., makes people 
hard-hearted and intolerant. Rudolph does not feel like the notary, but the 
reverse; he feels that ``it serves the rascal right''; that is no 
fellow-feeling.

You love man, therefore you torture the individual man, the egoist; your 
philanthropy (love of men) is the tormenting of men.

If I see the loved one suffer, I suffer with him, and I know no rest till I 
have tried everything to comfort and cheer him; if I see him glad, I too 
become glad over his joy. From this it does not follow that suffering or joy 
is caused in me by the same thing that brings out this effect in him, as is 
sufficiently proved by every bodily pain which I do not feel as he does; his 
tooth pains him, but his pain pains me.

But, because I cannot bear the troubled crease on the beloved forehead, for 
that reason, and therefore for my sake, I kiss it away. If I did not love this 
person, he might go right on making creases, they would not trouble me; I am 
only driving away \textit{my} trouble.

How now, has anybody or anything, whom and which I do not love, a 
\textit{right} to be loved by me? Is my love first, or is his right first? 
Parents, kinsfolk, fatherland, nation, native town, etc., finally fellowmen in 
general (``brothers, fraternity''), assert that they have a right to my 
love, and lay claim to it without further ceremony. They look upon it as 
\textit{their property}, and upon me, if I do not respect this, as a robber 
who takes from them what pertains to them and is theirs. I \textit{should} 
love. If love is a commandment and law, then I must be educated into it, 
cultivated up to it, and, if I trespass against it, punished. Hence people 
will exercise as strong a ``moral influence'' as possible on me to bring me 
to love. And there is no doubt that one can work up and seduce men to love as 
one can to other passions -- if you like, to hate. Hate runs through whole 
races merely because the ancestors of the one belonged to the Guelphs, those 
of the other to the Ghibellines.

But love is not a commandment, but, like each of my feelings, \textit{my 
property. Acquire}, \textit{i.e.} purchase, my property, and then I will make 
it over to you. A church, a nation, a fatherland, a family, etc., that does 
not know how to acquire my love, I need not love; and I fix the purchase price 
of my love quite at my pleasure.

Selfish love is far distant from unselfish, mystical, or romantic love. One 
can love everything possible, not merely men, but an ``object'' in general 
(wine, one's fatherland, etc.). Love becomes blind and crazy by a 
\textit{must} taking it out of my power (infatuation), romantic by a 
\textit{should} entering into it, \textit{i.e.} by the ``objects'' becoming 
sacred for me, or my becoming bound to it by duty, conscience, oath. Now the 
object no longer exists for me, but I for it.

Love is a possessedness, not as my feeling -- as such I rather keep it in my 
possession as property -- but through the alienness of the object. For 
religious love consists in the commandment to love in the beloved a ``holy 
one,'' or to adhere to a holy one; for unselfish love there are objects 
\textit{absolutely lovable} for which my heart is to beat, \textit{e.g.} 
fellow-men, or my wedded mate, kinsfolk, etc. Holy Love loves the holy in the 
beloved, and therefore exerts itself also to make of the beloved more and more 
a holy one (a ``man'').

The beloved is an object that \textit{should} be loved by me. He is not an 
object of my love on account of, because of, or by, my loving him, but is an 
object of love in and of himself. Not I make him an object of love, but he is 
such to begin with; for it is here irrelevant that he has become so by my 
choice, if so it be (as with a \textit{fianc\'ee}, a spouse, etc.), since even 
so he has in any case, as the person once chosen, obtained a ``right of his 
own to my love,'' and I, because I have loved him, am under obligation to 
love him forever. He is therefore not an object of \textit{my} love, but of 
love in general: an object that \textit{should} be loved. Love appertains to 
him, is due to him, or is his \textit{right}, while I am under 
\textit{obligation} to love him. My love, \textit{i.e.} the toll of love that 
I pay him, is in truth \textit{his} love, which he only collects from me as 
toll.

Every love to which there clings but the smallest speck of obligation is an 
unselfish love, and, so far as this speck reaches, a possessedness. He who 
believes that he \textit{owes} the object of his love anything loves 
romantically or religiously.

Family love, \textit{e.g.} as it is usually understood as ``piety,'' is a 
religious love; love of fatherland, preached as ``patriotism,'' likewise. 
All our romantic loves move in the same pattern: everywhere the hypocrisy, or 
rather self-deception, of an ``unselfish love,'' an interest in the object 
for the object's sake, not for my sake and mine alone.

Religious or romantic love is distinguished from sensual love by the 
difference of the object indeed, but not by the dependence of the relation to 
it. In the latter regard both are possessedness; but in the former the one 
object is profane, the other sacred. The dominion of the object over me is the 
same in both cases, only that it is one time a sensuous one, the other time a 
spiritual (ghostly) one. My love is my own only when it consists altogether in 
a selfish and egoistic interest, and when consequently the object of my love 
is really \textit{my} object or my property. I owe my property nothing, and 
have no duty to it, as little as I might have a duty to my eye; if 
nevertheless I guard it with the greatest care, I do so on my account.

Antiquity lacked love as little as do Christian times; the god of love is 
older than the God of Love. But the mystical possessedness belongs to the 
moderns.

The possessedness of love lies in the alienation of the object, or in my 
powerlessness as against its alienness and superior power. To the egoist 
nothing is high enough for him to humble himself before it, nothing so 
independent that he would live for love of it, nothing so sacred that he would 
sacrifice himself to it. The egoist's love rises in selfishness, flows in the 
bed of selfishness, and empties into selfishness again.

Whether this can still be called love? If you know another word for it, go 
ahead and choose it; then the sweet word love may wither with the departed 
world; for the present I at least find none in our \textit{Christian} 
language, and hence stick to the old sound and ``love'' \textit{my} object, 
my -- property.

Only as one of my feelings do I harbor love; but as a power above me, as a 
divine power, as Feuerbach says, as a passion that I am not to cast off, as a 
religious and moral duty, I -- scorn it. As my feeling it is \textit{mine;} as 
a principle to which I consecrate and ``vow'' my soul it is a dominator and 
\textit{divine}, just as hatred as a principle \textit{is diabolical;} one not 
better than the other. In short, egoistic love, \textit{i.e.} my love, is 
neither holy nor unholy, neither divine nor diabolical.

``A love that is limited by faith is an untrue love. The sole limitation that 
does not contradict the essence of love is the self-limitation of love by 
reason, intelligence. Love that scorns the rigor, the law, of intelligence, is 
theoretically a false love, practically a ruinous one.''\footnote{Feuerbach, 
``Essence of Chr.,'' 394.} So love is in its essence \textit{rational!} So 
thinks Feuerbach; the believer, on the contrary, thinks, Love is in its 
essence \textit{believing}. The one inveighs against \textit{irrational}, the 
other against \textit{unbelieving}, love. To both it can at most rank as a 
\textit{splendidum vitium}. Do not both leave love standing, even in the form 
of unreason and unbelief? They do not dare to say, irrational or unbelieving 
love is nonsense, is not love; as little as they are willing to say, 
irrational or unbelieving tears are not tears. But, if even irrational love, 
etc., must count as love, and if they are nevertheless to be unworthy of man, 
there follows simply this: love is not the highest thing, but reason or faith; 
even the unreasonable and the unbelieving can love; but love has value only 
when it is that of a rational or believing person. It is an illusion when 
Feuerbach calls the rationality of love its ``self-limitation''; the 
believer might with the same right call belief its ``self-limitation.'' 
Irrational love is neither ``false'' nor ``ruinous''; its does its service 
as love.

Toward the world, especially toward men, I am to \textit{assume a particular 
feeling}, and ``meet them with love,'' with the feeling of love, from the 
beginning. Certainly, in this there is revealed far more free-will and 
self-determination than when I let myself be stormed, by way of the world, by 
all possible feelings, and remain exposed to the most checkered, most 
accidental impressions. I go to the world rather with a preconceived feeling, 
as if it were a prejudice and a preconceived opinion; I have prescribed to 
myself in advance my behavior toward it, and, despite all its temptations, 
feel and think about it only as I have once determined to. Against the 
dominion of the world I secure myself by the principle of love; for, whatever 
may come, I -- love. The ugly -- \textit{e.g.} --makes a repulsive impression 
on me; but, determined to love, I master this impression as I do every 
antipathy.

But the feeling to which I have determined and -- condemned myself from the 
start is a \textit{narrow} feeling, because it is a predestined one, of which 
I myself am not able to get clear or to declare myself clear. Because 
preconceived, it is a \textit{prejudice. I} no longer show myself in face of 
the world, but my love shows itself. The \textit{world} indeed does not rule 
me, but so much the more inevitably does the spirit of \textit{love} rule this 
spirit.

If I first said, I love the world, I now add likewise: I do not love it, for I 
\textit{annihilate} it as I annihilate myself; I \textit{dissolve it}. I do 
not limit myself to one feeling for men, but give free play to all that I am 
capable of. Why should I not dare speak it out in all its glaringness? Yes, I 
\textit{utilize} the world and men! With this I can keep myself open to every 
impression without being torn away from myself by one of them. I can love, 
love with a full heart, and let the most consuming glow of passion burn in my 
heart, without taking the beloved one for anything else than the 
\textit{nourishment} of my passion, on which it ever refreshes itself anew. 
All my care for him applies only to the \textit{object of my love}, only to 
him whom my love \textit{requires}, only to him, the ``warmly loved.'' How 
indifferent would he be to me without this -- my love! I feed only my love 
with him, I \textit{utilize} him for this only: I \textit{enjoy} him.

Let us choose another convenient example. I see how men are fretted in dark 
superstition by a swarm of ghosts. If to the extent of my powers I let a bit 
of daylight fall in on the nocturnal spookery, is it perchance because love to 
you inspires this in me? Do I write out of love to men? No, I write because I 
want to procure for \textit{my} thoughts an existence in the world; and, even 
if I foresaw that these thoughts would deprive you of your rest and your 
peace, even if I saw the bloodiest wars and the fall of many generations 
springing up from this seed of thought -- I would nevertheless scatter it. Do 
with it what you will and can, that is your affair and does not trouble me. 
You will perhaps have only trouble, combat, and death from it, very few will 
draw joy from it. If your weal lay at my heart, I should act as the church did 
in withholding the Bible from the laity, or Christian governments, which make 
it a sacred duty for themselves to ``protect the common people from bad 
books.''

But not only not for your sake, not even for truth's sake either do I speak 
out what I think. No --

\begin{quotation}

\noindent{}I sing as the bird sings\\
 That on the bough alights;\\
 The song that from me springs\\
 Is pay that well requites.\end{quotation}

\noindent{}I sing because -- I am a singer. But I 
\textit{use}\footnote{[\textit{gebrauche}]} you for it because I -- 
need\footnote{[\textit{brauche}]} ears.

Where the world comes in my way -- and it comes in my way everywhere -- I 
consume it to quiet the hunger of my egoism. For me you are nothing but --my 
food, even as I too am fed upon and turned to use by you. We have only one 
relation to each other, that of \textit{usableness}, of utility, of use. We 
owe \textit{each other} nothing, for what I seem to owe you I owe at most to 
myself. If I show you a cheery air in order to cheer you likewise, then your 
cheeriness is of consequence to \textit{me}, and my air serves \textit{my} 
wish; to a thousand others, whom I do not aim to cheer, I do not show it.

\myhrule

One has to be educated up to that love which founds itself on the ``essence 
of man'' or, in the ecclesiastical and moral period, lies upon us as a 
``commandment.'' In what fashion moral influence, the chief ingredient of 
our education, seeks to regulate the intercourse of men shall here be looked 
at with egoistic eyes in one example at least.

Those who educate us make it their concern early to break us of lying and to 
inculcate the principle that one must always tell the truth. If selfishness 
were made the basis for this rule, every one would easily understand how by 
lying he fools away that confidence in him which he hopes to awaken in others, 
and how correct the maxim proves, Nobody believes a liar even when he tells 
the truth. Yet, at the same time, he would also feel that he had to meet with 
truth only him whom \textit{he} authorized to hear the truth. If a spy walks 
in disguise through the hostile camp, and is asked who he is, the askers are 
assuredly entitled to inquire after his name, but the disguised man does not 
give them the right to learn the truth from him; he tells them what he likes, 
only not the fact. And yet morality demands, ``Thou shalt not lie!'' By 
morality those persons are vested with the right to expect the truth; but by 
me they are not vested with that right, and I recognize only the right that 
\textit{I} impart. In a gathering of revolutionists the police force their way 
in and ask the orator for his name; everybody knows that the police have the 
right to do so, but they do not have it from the \textit{revolutionist}, since 
he is their enemy; he tells them a false name and --cheats them with a lie. 
The police do not act so foolishly either as to count on their enemies' love 
of truth; on the contrary, they do not believe without further ceremony, but 
have the questioned individual ``identified'' if they can. Nay, the State -- 
everywhere proceeds incredulously with individuals, because in their egoism it 
recognizes its natural enemy; it invariably demands a ``voucher,'' and he 
who cannot show vouchers falls a prey to its investigating inquisition. The 
State does not believe nor trust the individual, and so of itself places 
itself with him in the \textit{convention of lying}; it trusts me only when it 
has \textit{convinced} itself of the truth of my statement, for which there 
often remains to it no other means than the oath. How clearly, too, this (the 
oath) proves that the State does not count on our credibility and love of 
truth, but on our \textit{interest}, our selfishness: it relies on our not 
wanting to fall foul of God by a perjury.

Now, let one imagine a French revolutionist in the year 1788, who among 
friends let fall the now well-known phrase, ``the world will have no rest 
till the last king is hanged with the guts of the last priest.'' The king 
then still had all power, and, when the utterance is betrayed by an accident, 
yet without its being possible to produce witnesses, confession is demanded 
from the accused. Is he to confess or not?

If he denies, he lies and -- remains unpunished; if he confesses, he is candid 
and -- is beheaded. If truth is more than everything else to him, all right, 
let him die. Only a paltry poet could try to make a tragedy out of the end of 
his life; for what interest is there in seeing how a man succumbs from 
cowardice? But, if he had the courage not to be a slave of truth and 
sincerity, he would ask somewhat thus: Why need the judges know what I have 
spoken among friends? If I had \textit{wished} them to know, I should have 
said it to them as I said it to my friends. I will not have them know it. They 
force themselves into my confidence without my having called them to it and 
made them my confidants; they \textit{will} learn what I \textit{will} keep 
secret. Come on then, you who wish to break my will by your will, and try your 
arts. You can torture me by the rack, you can threaten me with hell and 
eternal damnation, you can make me so nerveless that I swear a false oath, but 
the truth you shall not press out of me, for I \textit{will} lie to you 
because I have given you no claim and no right to my sincerity. Let God, 
``who is truth,'' look down ever so threateningly on me, let lying come ever 
so hard to me, I have nevertheless the courage of a lie; and, even if I were 
weary of my life, even if nothing appeared to me more welcome than your 
executioner's sword, you nevertheless should not have the joy of finding in me 
a slave of truth, whom by your priestly arts you make a traitor to his 
\textit{will}. When I spoke those treasonable words, I would not have had you 
know anything of them; I now retain the same will, and do not let myself be 
frightened by the curse of the lie.

Sigismund is not a miserable caitiff because he broke his princely word, but 
he broke the word because he was a caitiff; he might have kept his word and 
would still have been a caitiff, a priest-ridden man. Luther, driven by a 
higher power, became unfaithful to his monastic vow: he became so for God's 
sake. Both broke their oath as possessed persons: Sigismund, because he wanted 
to appear as a \textit{sincere} professor of the divine \textit{truth}, 
\textit{i.e.}, of the true, genuinely Catholic faith; Luther, in order to 
give testimony for the gospel \textit{sincerely} and with entire truth. with 
body and soul; both became perjured in order to be sincere toward the 
``higher truth.'' Only, the priests absolved the one, the other absolved 
himself. What else did both observe than what is contained in those apostolic 
words, ``Thou hast not lied to men, but to God?'' They lied to men, broke 
their oath before the world's eyes, in order not to lie to God, but to serve 
him. Thus they show us a way to deal with truth before men. For God's glory, 
and for God's sake, a -- breach of oath, a lie, a prince's word broken!

How would it be, now, if we changed the thing a little and wrote, A perjury 
and lie for -- \textit{my sake?} Would not that be pleading for every 
baseness? It seems so, assuredly, only in this it is altogether like the 
``for God's sake.'' For was not every baseness committed for God's sake, 
were not all the scaffolds filled for his sake and all the 
\textit{autos-da-f\'e} held for his sake, was not all stupefaction introduced 
for his sake? And do they not today still for God's sake fetter the mind in 
tender children by religious education? Were not sacred vows broken for his 
sake, and do not missionaries and priests still go around every day to bring 
Jews, heathen, Protestants or Catholics, to treason against the faith of their 
fathers -- for his sake? And that should be worse with the \textit{for my 
sake?} What then does \textit{on my account} mean? There people immediately 
think of \textit{``filthy lucre''}. But he who acts from love of filthy 
lucre does it on his own account indeed, as there is nothing anyhow that one 
does not do for his own sake -- among other things, everything that is done 
for God's glory; yet he, for whom he seeks the lucre, is a slave of lucre, not 
raised above lucre; he is one who belongs to lucre, the money-bag, not to 
himself; he is not his own. Must not a man whom the passion of avarice rules 
follow the commands of this \textit{master?} And, if a weak goodnaturedness 
once beguiles him, does this not appear as simply an exceptional case of 
precisely the same sort as when pious believers are sometimes forsaken by 
their Lord's guidance and ensnared by the arts of the ``devil?'' So an 
avaricious man is not a self-owned man, but a servant; and he can do nothing 
for his own sake without at the same time doing it for his lord's sake -- 
precisely like the godly man.

Famous is the breach of oath which Francis I committed against Emperor Charles 
V. Not later, when he ripely weighed his promise, but at once, when he swore 
the oath, King Francis took it back in thought as well as by a secret 
protestation documentarily subscribed before his councillors; he uttered a 
perjury aforethought. Francis did not show himself disinclined to buy his 
release, but the price that Charles put on it seemed to him too high and 
unreasonable. Even though Charles behaved himself in a sordid fashion when he 
sought to extort as much as possible, it was yet shabby of Francis to want to 
purchase his freedom for a lower ransom; and his later dealings, among which 
there occurs yet a second breach of his word, prove sufficiently how the 
huckster spirit held him enthralled and made him a shabby swindler. However, 
what shall we say to the reproach of perjury against him? In the first place, 
surely, this again: that not the perjury, but his sordidness, shamed him; that 
he did not deserve contempt for his perjury, but made himself guilty of 
perjury because he was a contemptible man. But Francis's perjury, regarded in 
itself, demands another judgment. One might say Francis did not respond to the 
confidence that Charles put in him in setting him free. But, if Charles had 
really favored him with confidence, he would have named to him the price that 
he considered the release worth, and would then have set him at liberty and 
expected Francis to pay the redemption-sum. Charles harbored no such trust, 
but only believed in Francis's impotence and credulity, which would not allow 
him to act against his oath; but Francis deceived only this -- credulous 
calculation. When Charles believed he was assuring himself of his enemy by an 
oath, right there he was freeing him from every obligation. Charles had given 
the king credit for a piece of stupidity, a narrow conscience, and, without 
confidence in Francis, counted only on Francis's stupidity, \textit{e.g.}, 
conscientiousness: he let him go from the Madrid prison only to hold him the 
more securely in the prison of conscientiousness, the great jail built about 
the mind of man by religion: he sent him back to France locked fast in 
invisible chains, what wonder if Francis sought to escape and sawed the chains 
apart? No man would have taken it amiss of him if he had secretly fled from 
Madrid, for he was in an enemy's power; but every good Christian cries out 
upon him, that he wanted to loose himself from God's bonds too. (It was only 
later that the pope absolved him from his oath.)

It is despicable to deceive a confidence that we voluntarily call forth; but 
it is no shame to egoism to let every one who wants to get us into his power 
by an oath bleed to death by the failure of his untrustful craft. If you have 
wanted to bind me, then learn that I know how to burst your bonds.

The point is whether I give the confider the right to confidence. If the 
pursuer of my friend asks me where he has fled to, I shall surely put him on a 
false trail. Why does he ask precisely me, the pursued man's friend? In order 
not to be a false, traitorous friend, I prefer to be false to the enemy. I 
might certainly in courageous conscientiousness, answer, ``I will not tell'' 
(so Fichte decides the case); by that I should salve my love of truth and do 
for my friend as much as -- nothing, for, if I do not mislead the enemy, he 
may accidentally take the right street, and my love of truth would have given 
up my friend as a prey, because it hindered me from the --courage for a lie. 
He who has in the truth an idol, a sacred thing, must \textit{humble} himself 
before it, must not defy its demands, not resist courageously; in short, he 
must renounce the \textit{heroism of the lie}. For to the lie belongs not less 
courage than to the truth: a courage that young men are most apt to be 
defective in, who would rather confess the truth and mount the scaffold for it 
than confound the enemy's power by the impudence of a lie. To them the truth 
is ``sacred,'' and the sacred at all times demands blind reverence, 
submission, and self-sacrifice. If you are not impudent, not mockers of the 
sacred, you are tame and its servants. Let one but lay a grain of truth in the 
trap for you, you peck at it to a certainty, and the fool is caught. You will 
not lie? Well, then, fall as sacrifices to the truth and become -- martyrs! 
Martyrs! -- for what? For yourselves, for self-ownership? No, for your goddess -- the truth. You know only two \textit{services}, only two kinds of servants: 
servants of the truth and servants of the lie. Then in God's name serve the 
truth!

Others, again, serve the truth also; but they serve it ``in moderation,'' 
and make, \textit{e.g.} a great distinction between a simple lie and a lie 
sworn to. And yet the whole chapter of the oath coincides with that of the 
lie, since an oath, everybody knows, is only a strongly assured statement. You 
consider yourselves entitled to lie, if only you do not swear to it besides? 
One who is particular about it must judge and condemn a lie as sharply as a 
false oath. But now there has been kept up in morality an ancient point of 
controversy, which is customarily treated of under the name of the ``lie of 
necessity.'' No one who dares plead for this can consistently put from him an 
``oath of necessity.'' If I justify my lie as a lie of necessity, I should 
not be so pusillanimous as to rob the justified lie of the strongest 
corroboration. Whatever I do, why should I not do it entirely and without 
reservations (\textit{reservatio mentalis})? If I once lie, why then not lie 
completely, with entire consciousness and all my might? As a spy I should have 
to swear to each of my false statements at the enemy's demand; determined to 
lie to him, should I suddenly become cowardly and undecided in face of an 
oath? Then I should have been ruined in advance for a liar and spy; for, you 
see, I should be voluntarily putting into the enemy's hands a means to catch 
me. -- The State too fears the oath of necessity, and for this reason does not 
give the accused a chance to swear. But you do not justify the State's fear; 
you lie, but do not swear falsely. If, \textit{e.g.} you show some one a 
kindness, and he is not to know it, but he guesses it and tells you so to your 
face, you deny; if he insists, you say, ``honestly, no!'' If it came to 
swearing, then you would refuse; for, from fear of the sacred, you always stop 
half way. \textit{Against} the sacred you have no \textit{will of your own}. 
You lie in -- moderation, as you are free ``in moderation,'' religious ``in 
moderation'' (the clergy are not to ``encroach''; over this point the most 
rapid of controversies is now being carried on, on the part of the university 
against the church), monarchically disposed ``in moderation'' (you want a 
monarch limited by the constitution, by a fundamental law of the State), 
everything nicely \textit{tempered}, lukewarm, half God's, half the devil's.

There was a university where the usage was that every word of honor that must 
be given to the university judge was looked upon by the students as null and 
void. For the students saw in the demanding of it nothing but a snare, which 
they could not escape otherwise than by taking away all its significance. He 
who at that same university broke his word of honor to one of the fellows was 
infamous; he who gave it to the university judge derided, in union with these 
very fellows, the dupe who fancied that a word had the same value among 
friends and among foes. It was less a correct theory than the constraint of 
practice that had there taught the students to act so, as, without that means 
of getting out, they would have been pitilessly driven to treachery against 
their comrades. But, as the means approved itself in practice, so it has its 
theoretical probation too. A word of honor, an oath, is one only for him whom 
I entitle to receive it; he who forces me to it obtains only a forced, 
\textit{i.e.} a \textit{hostile} word, the word of a foe, whom one has no 
right to trust; for the foe does not give us the right.

Aside from this, the courts of the State do not even recognize the 
inviolability of an oath. For, if I had sworn to one who comes under 
examination that I would not declare anything against him, the court would 
demand my declaration in spite of the fact that an oath binds me, and, in case 
of refusal, would lock me up till I decided to become -- an oath-breaker. The 
court ``absolves me from my oath''; -- how magnanimous! If any power can 
absolve me from the oath, I myself am surely the very first power that has a 
claim to.

As a curiosity, and to remind us of customary oaths of all sorts, let place be 
given here to that which Emperor Paul commanded the captured Poles 
(Kosciuszko, Potocki, Niemcewicz, and others) to take when he released them: 
``We not merely swear fidelity and obedience to the emperor, but also further 
promise to pour out our blood for his glory; we obligate ourselves to discover 
everything threatening to his person or his empire that we ever learn; we 
declare finally that, in whatever part of the earth we may be, a single word 
of the emperor shall suffice to make us leave everything and repair to him at 
once.''

\myhrule

In one domain the principle of love seems to have been long outsoared by 
egoism, and to be still in need only of sure consciousness, as it were of 
victory with a good conscience. This domain is speculation, in its double 
manifestation as thinking and as trade. One thinks with a will, whatever may 
come of it; one speculates, however many may suffer under our speculative 
undertakings. But, when it finally becomes serious, when even the last remnant 
of religiousness, romance, or ``humanity'' is to be done away, then the 
pulse of religious conscience beats, and one at least \textit{professes} 
humanity. The avaricious speculator throws some coppers into the poor-box and 
``does good,'' the bold thinker consoles himself with the fact that he is 
working for the advancement of the human race and that his devastation 
``turns to the good'' of mankind, or, in another case, that he is ``serving 
the idea''; mankind, the idea, is to him that something of which he must say, 
It is more to me than myself.

To this day thinking and trading have been done for -- God's sake. Those who 
for six days were trampling down everything by their selfish aims sacrificed 
on the seventh to the Lord; and those who destroyed a hundred ``good 
causes'' by their reckless thinking still did this in the service of another 
``good cause,'' and had yet to think of another -- besides themselves -- to 
whose good their self-indulgence should turn; of the people, mankind, etc. But 
this other thing is a being above them, a higher or supreme being; and 
therefore I say, they are toiling for God's sake.

Hence I can also say that the ultimate basis of their actions is -- love. Not 
a voluntary love however, not their own, but a tributary love, or the higher 
being's own (God's, who himself is love); in short, not the egoistic, but the 
religious; a love that springs from their fancy that they \textit{must} 
discharge a tribute of love, \textit{i.e.} that they must not be 
``egoists.''

If \textit{we} want to deliver the world from many kinds of unfreedom, we want 
this not on its account but on ours; for, as we are not world-liberators by 
profession and out of ``love,'' we only want to win it away from others. We 
want to make it \textit{our} own; it is not to be any longer \textit{owned as 
serf} by God (the church) nor by the law (State), but to be \textit{our own}; 
therefore we seek to ``win'' it, to ``captivate'' it, and, by meeting it 
halfway and ``devoting'' ourselves to it as to ourselves as soon as it 
belongs to us, to complete and make superfluous the force that it turns 
against us. If the world is ours, it no longer attempts any force 
\textit{against} us, but only \textit{with us}. My selfishness has an interest 
in the liberation of the world, that it may become -- my property.

Not isolation or being alone, but society, is man's original state. Our 
existence begins with the most intimate conjunction, as we are already living 
with our mother before we breathe; when we see the light of the world, we at 
once lie on a human being's breast again, her love cradles us in the lap, 
leads us in the go-cart, and chains us to her person with a thousand ties. 
Society is our \textit{state of nature}. And this is why, the more we learn to 
feel ourselves, the connection that was formerly most intimate becomes ever 
looser and the dissolution of the original society more unmistakable. To have 
once again for herself the child that once lay under her heart, the mother 
must fetch it from the street and from the midst of its playmates. The child 
prefers the \textit{intercourse} that it enters into with \textit{its fellows} 
to the \textit{society} that it has not entered into, but only been born in.

But the dissolution of \textit{society} is \textit{intercourse} or 
\textit{union}. A society does assuredly arise by union too, but only as a 
fixed idea arises by a thought -- to wit, by the vanishing of the energy of 
the thought (the thinking itself, this restless taking back all thoughts that 
make themselves fast) from the thought. If a union\footnote{[\textit{Verein}]} 
has crystallized into a society, it has ceased to be a 
coalition;\footnote{[\textit{Vereinigung}]} for coalition is an incessant 
self-uniting; it has become a unitedness, come to a standstill, degenerated 
into a fixity; it is -- \textit{dead} as a union, it is the corpse of the 
union or the coalition, \textit{i.e.} it is --society, community. A striking 
example of this kind is furnished by the \textit{party}.

That a society (\textit{e.g.} the society of the State) diminishes my 
\textit{liberty} offends me little. Why, I have to let my liberty be limited 
by all sorts of powers and by every one who is stronger; nay, by every 
fellow-man; and, were I the autocrat of all the R......, I yet should not 
enjoy absolute liberty. But \textit{ownness} I will not have taken from me. 
And ownness is precisely what every society has designs on, precisely what is 
to succumb to its power.

A society which I join does indeed take from me many liberties, but in return 
it affords me other liberties; neither does it matter if I myself deprive 
myself of this and that liberty (\textit{e.g.} by any contract). On the other 
hand, I want to hold jealously to my ownness. Every community has the 
propensity, stronger or weaker according to the fullness of its power, to 
become an \textit{authority} to its members and to set \textit{limits} for 
them: it asks, and must ask, for a ``subject's limited understanding''; it 
asks that those who belong to it be subjected to it, be its ``subjects''; it 
exists only by \textit{subjection}. In this a certain tolerance need by no 
means be excluded; on the contrary, the society will welcome improvements, 
corrections, and blame, so far as such are calculated for its gain: but the 
blame must be ``well-meaning,'' it may not be ``insolent and 
disrespectful'' -- in other words, one must leave uninjured, and hold sacred, 
the substance of the society. The society demands that those who belong to it 
shall not \textit{go beyond it} and exalt themselves, but remain ``within the 
bounds of legality,'' \textit{e.g.}, allow themselves only so much as the 
society and its law allow them.

There is a difference whether my liberty or my ownness is limited by a 
society. If the former only is the case, it is a coalition, an agreement, a 
union; but, if ruin is threatened to ownness, it is \textit{a power of 
itself}, a power \textit{above me}, a thing unattainable by me, which I can 
indeed admire, adore, reverence, respect, but cannot subdue and consume, and 
that for the reason that I \textit{am resigned}. It exists by my 
\textit{resignation}, my \textit{self-renunciation}, my 
spiritlessness,\footnote{[\textit{Muthl\"osigkeit}]} called --

HUMILITY.\footnote{[\textit{Demuth}]} My humility makes its 
courage,\footnote{[\textit{Muth}]} my submissiveness gives it its dominion.

 But in reference to \textit{liberty}, State and union are subject to no 
essential difference. The latter can just as little come into existence, or 
continue in existence, without liberty's being limited in all sorts of ways, 
as the State is compatible with unmeasured liberty. Limitation of liberty is 
inevitable everywhere, for one cannot get \textit{rid} of everything; one 
cannot fly like a bird merely because one would like to fly so, for one does 
not get free from his own weight; one cannot live under water as long as he 
likes, like a fish, because one cannot do without air and cannot get free from 
this indispensable necessity; etc. As religion, and most decidedly 
Christianity, tormented man with the demand to realize the unnatural and self- 
contradictory, so it is to be looked upon only as the true logical outcome of 
that religious over-straining and overwroughtness that finally \textit{liberty 
itself, absolute liberty}, was exalted into an ideal, and thus the nonsense of 
the impossible to come glaringly to the light. -- The union will assuredly 
offer a greater measure of liberty, as well as (and especially because by it 
one escapes all the coercion peculiar to State and society life) admit of 
being considered as ``a new liberty''; but nevertheless it will still 
contain enough of unfreedom and involuntariness. For its object is not this -- 
liberty (which on the contrary it sacrifices to ownness), but only 
\textit{ownness}. Referred to this, the difference between State and union is 
great enough. The former is an enemy and murderer of \textit{ownness}, the 
latter a son and co-worker of it; the former a spirit that would be adored in 
spirit and in truth, the latter my work, my product ; the State is the lord of 
my spirit, who demands faith and prescribes to me articles of faith, the creed 
of legality; it exerts moral influence, dominates my spirit, drives away my 
ego to put itself in its place as ``my true ego'' -- in short, the State is 
sacred, and as against me, the individual man, it is the true man, the spirit, 
the ghost; but the union is my own creation, my creature, not sacred, not a 
spiritual power above my spirit, as little as any association of whatever 
sort. As I am not willing to be a slave of my maxims, but lay them bare to my 
continual criticism without \textit{any warrant}, and admit no bail at all for 
their persistence, so still less do I obligate myself to the union for my 
future and pledge my soul to it, as is said to be done with the devil, and is 
really the case with the State and all spiritual authority; but I am and 
remain \textit{more} to myself than State, Church, God, etc.; consequently 
infinitely more than the union too.

That society which Communism wants to found seems to stand nearest to 
\textit{coalition}. For it is to aim at the ``welfare of all,'' oh, yes, of 
all, cries Weitling innumerable times, of all! That does really look as if in 
it no one needed to take a back seat. But what then will this welfare be? Have 
all one and the same welfare, are all equally well off with one and the same 
thing? If that be so, the question is of the ``true welfare.'' Do we not 
with this come right to the point where religion begins its dominion of 
violence? Christianity says, Look not on earthly toys, but seek your true 
welfare, become -- pious Christians; being Christians is the true welfare. It 
is the true welfare of ``all,'' because it is the welfare of Man as such 
(this spook). Now, the welfare of all is surely to be \textit{your} and 
\textit{my} welfare too? But, if you and I do not look upon that welfare as 
\textit{our} welfare, will care then be taken for that in which \textit{we} 
feel well? On the contrary, society has decreed a welfare as the ``true 
welfare,'' if this welfare were called \textit{e.g.} ``enjoyment honestly 
worked for''; but if you preferred enjoyable laziness, enjoyment without 
work, then society, which cares for the ``welfare of all,'' would wisely 
avoid caring for that in which you are well off. Communism, in proclaiming the 
welfare of all, annuls outright the well-being of those who hitherto lived on 
their income from investments and apparently felt better in that than in the 
prospect of Weitling's strict hours of labor. Hence the latter asserts that 
with the welfare of thousands the welfare of millions cannot exist, and the 
former must give up \textit{their} special welfare ``for the sake of the 
general welfare.'' No, let people not be summoned to sacrifice their special 
welfare for the general, for this Christian admonition will not carry you 
through; they will better understand the opposite admonition, not to let their 
\textit{own} welfare be snatched from them by anybody, but to put it on a 
permanent foundation. Then they are of themselves led to the point that they 
care best for their welfare if they \textit{unite} with others for this 
purpose, \textit{e.g.}, ``sacrifice a part of their liberty,'' yet not to 
the welfare of others, but to their own. An appeal to men's self-sacrificing 
disposition end self- renouncing love ought at least to have lost its 
seductive plausibility when, after an activity of thousands of years, it has 
left nothing behind but the -- \textit{mis\`ere} of today. Why then still 
fruitlessly expect self-sacrifice to bring us better time? Why not rather hope 
for them from \textit{usurpation?} Salvation comes no longer from the giver, 
the bestower, the loving one, but from the \textit{taker}, the appropriator 
(usurper), the owner. Communism, and, consciously, egoism-reviling humanism, 
still count on \textit{love}.

If community is once a need of man, and he finds himself furthered by it in 
his aims, then very soon, because it has become his principle, it prescribes 
to him its laws too, the laws of -- society. The principle of men exalts 
itself into a sovereign power over them, becomes their supreme essence, their 
God, and, as such -- law-giver. Communism gives this principle the strictest 
effect, and Christianity is the religion of society, for, as Feuerbach rightly 
says, although he does not mean it rightly, love is the essence of man; 
\textit{e.g.}, the essence of society or of societary (Communistic) man. All 
religion is a cult of society, this principle by which societary (cultivated) 
man is dominated; neither is any god an ego's exclusive god, but always a 
society's or community's, be it of the society, ``family'' (Lar, Penates) or 
of a ``people'' (``national god'') or of ``all men'' (``he is a Father 
of all men'').

Consequently one has a prospect of extirpating religion down to the ground 
only when one antiquates \textit{society} and everything that flows from this 
principle. But it is precisely in Communism that this principle seeks to 
culminate, as in it everything is to become \textit{common} for the 
establishment of -- ``equality.'' If this ``equality'' is won, 
``liberty'' too is not lacking. But whose liberty? \textit{Society's}! 
Society is then all in all, and men are only ``for each other.'' It would be 
the glory of the -- love-State.

But I would rather be referred to men's selfishness than to their 
``kindnesses,''\footnote{[Literally, ``love-services.'']} their mercy, 
pity, etc. The former demands \textit{reciprocity} (as thou to me, so I to 
thee), does nothing ``gratis,'' and may be won and -- \textit{bought}. But 
with what shall I obtain the kindness? It is a matter of chance whether I am 
at the time having to do with a ``loving'' person. The affectionate one's 
service can be had only by -- \textit{begging}, be it by my lamentable 
appearance, by my need of help, my misery, my -- \textit{suffering}. What can 
I offer him for his assistance? Nothing! I must accept it as a --present. Love 
is \textit{unpayable}, or rather, love can assuredly be paid for, but only by 
counter-love (``One good turn deserves another''). What paltriness and 
beggarliness does it not take to accept gifts year in and year out without 
service in return, as they are regularly collected \textit{e.g.} from the 
poor day-laborer? What can the receiver do for him and his donated pennies, in 
which his wealth consists? The day- laborer would really have more enjoyment 
if the receiver with his laws, his institutions, etc., all of which the 
day-laborer has to pay for though, did not exist at all. And yet, with it all, 
the poor wight \textit{loves} his master.

No, community, as the ``goal'' of history hitherto, is impossible. Let us 
rather renounce every hypocrisy of community, and recognize that, if we are 
equal as men, we are not equal for the very reason that we are not men. We are 
equal \textit{only in thoughts}, only when ``we'' are \textit{thought}, not 
as we really and bodily are. I am ego, and you are ego: but I am not this 
thought-of ego; this ego in which we are all equal is only my 
\textit{thought}. I am man, and you are man: but ``man'' is only a thought, 
a generality; neither I nor you are speakable, we are \textit{unutterable}, 
because only \textit{thoughts} are speakable and consist in speaking.

Let us therefore not aspire to community, but to \textit{one-sidedness}. Let 
us not seek the most comprehensive commune, ``human society,'' but let us 
seek in others only means and organs which we may use as our property! As we 
do not see our equals in the tree, the beast, so the presupposition that 
others are \textit{our equals} springs from a hypocrisy. No one is \textit{my 
equal}, but I regard him, equally with all other beings, as my property. In 
opposition to this I am told that I should be a man among ``fellow-men'' 
(\textit{Judenfrage}, p. 60); I should ``respect'' the fellow-man in them. 
For me no one is a person to be respected, not even the fellow-man, but 
solely, like other beings, an \textit{object} in which I take an interest or 
else do not, an interesting or uninteresting object, a usable or unusable 
person.

And, if I can use him, I doubtless come to an understanding and make myself at 
one with him, in order, by the agreement, to strengthen \textit{my power}, and 
by combined force to accomplish more than individual force could effect. In 
this combination I see nothing whatever but a multiplication of my force, and 
I retain it only so long as it is my multiplied force. But thus it is a -- 
union.

Neither a natural ligature nor a spiritual one holds the union together, and 
it is not a natural, not a spiritual league. It is not brought about by one 
\textit{blood}, not by one \textit{faith} (spirit). In a natural league -- 
like a family, a tribe, a nation, yes, mankind -- the individuals have only 
the value of \textit{specimens} of the same species or genus; in a spiritual 
league -- like a commune, a church -- the individual signifies only a 
\textit{member} of the same spirit; what you are in both cases as a unique 
person must be -- suppressed. Only in the union can you assert yourself as 
unique, because the union does not possess you, but you possess it or make it 
of use to you.

Property is recognized in the union, and only in the union, because one no 
longer holds what is his as a fief from any being. The Communists are only 
consistently carrying further what had already been long present during 
religious evolution, and especially in the State; to wit, propertylessness, 
the feudal system.

The State exerts itself to tame the desirous man; in other words, it seeks to 
direct his desire to it alone, and to \textit{content} that desire with what 
it offers. To sate the desire for the desirous man's sake does not come into 
the mind: on the contrary, it stigmatizes as an ``egoistic man'' the man who 
breathes out unbridled desire, and the ``egoistic man'' is its enemy. He is 
this for it because the capacity to agree with him is wanting to the State; 
the egoist is precisely what it cannot ``comprehend.'' Since the State (as 
nothing else is possible) has to do only for itself, it does not take care for 
my needs, but takes care only of how it make away with me, \textit{i.e.} make 
out of me another ego, a good citizen. It takes measures for the 
``improvement of morals.'' -- And with what does it win individuals for 
itself? With itself, \textit{i.e.} with what is the State's, with 
\textit{State property}. It will be unremittingly active in making all 
participants in its ``goods,'' providing all with the ``good things of 
culture''; it presents them its education, opens to them the access to its 
institutions of culture, capacitates them to come to property (\textit{i.e.} 
to a fief) in the way of industry, etc. For all these \textit{fiefs} it 
demands only the just rent of continual \textit{thanks}. But the 
``unthankful'' forget to pay these thanks. -- Now, neither can ``society'' 
do essentially otherwise than the State.

You bring into a union your whole power, your competence, and \textit{make 
yourself count}; in a society you are \textit{employed}, with your working 
power; in the former you live egoistically, in the latter humanly, 
\textit{i.e.} religiously, as a ``member in the body of this Lord''; to a 
society you owe what you have, and are in duty bound to it, are -- possessed 
by ``social duties''; a union you utilize, and give it up undutifully and 
unfaithfully when you see no way to use it further. If a society is more than 
you, then it is more to you than yourself; a union is only your instrument, or 
the sword with which you sharpen and increase your natural force; the union 
exists for you and through you, the society conversely lays claim to you for 
itself and exists even without you, in short, the society is \textit{sacred}, 
the union your \textit{own}; consumes \textit{you, you} consume the union.

Nevertheless people will not be backward with the objection that the agreement 
which has been concluded may again become burdensome to us and limit our 
freedom; they will say, we too would at last come to this, that ``every one 
must sacrifice a part of his freedom for the sake of the generality.'' But 
the sacrifice would not be made for the ``generality's'' sake a bit, as 
little as I concluded the agreement for the ``generality's'' or even for any 
other man's sake; rather I came into it only for the sake of my own benefit, 
from \textit{selfishness}.\footnote{[Literally, ``own-benefit.'']} But, as 
regards the sacrificing, surely I ``sacrifice'' only that which does not 
stand in my power, \textit{i.e.}, I ``sacrifice'' nothing at all.

To come back to property, the lord is proprietor. Choose then whether you want 
to be lord, or whether society shall be! On this depends whether you are to be 
an \textit{owner} or a \textit{ragamuffin}! The egoist is owner, the Socialist 
a ragamuffin. But ragamuffinism or propertylessness is the sense of feudalism, 
of the feudal system which since the last century has only changed its 
overlord, putting ``Man'' in the place of God, and accepting as a fief from 
Man what had before been a fief from the grace of God. That the ragamuffinism 
of Communism is carried out by the humane principle into the absolute or most 
ragamuffinly ragamuffinism has been shown above; but at the same time also, 
how ragamuffinism can only thus swing around into ownness. The \textit{old} 
feudal system was so thoroughly trampled into the ground in the Revolution 
that since then all reactionary craft has remained fruitless, and will always 
remain fruitless, because the dead is -- dead; but the resurrection too had to 
prove itself a truth in Christian history, and has so proved itself: for in 
another world feudalism is risen again with a glorified body, the \textit{new} 
feudalism under the suzerainty of ``Man.''

Christianity is not annihilated, but the faithful are right in having hitherto 
trustfully assumed of every combat against it that this could serve only for 
the purgation and confirmation of Christianity; for it has really only been 
glorified, and ``Christianity exposed'' is the -- \textit{human 
Christianity}. We are still living entirely in the Christian age, and the very 
ones who feel worst about it are the most zealously contributing to 
``complete'' it. The more human, the dearer has feudalism become to us; for 
we the less believe that it still is feudalism, we take it the more 
confidently for ownness and think we have found what is ``most absolutely our 
own'' when we discover ``the human.''

Liberalism wants to give me what is mine, but it thinks to procure it for me 
not under the title of mine, but under that of the ``human.'' As if it were 
attainable under this mask! The rights of man, the precious work of the 
Revolution, have the meaning that the Man in me 
\textit{entitles}\footnote{[Literally, furnishes me with a \textit{right}.]} 
me to this and that; I as individual, \textit{i.e.} as this man, am not 
entitled, but Man has the right and entitles me. Hence as man I may well be 
entitled; but, as I am more than man, to wit, a \textit{special} man, it may 
be refused to this very me, the special one. If on the other hand you insist 
on the \textit{value} of your gifts, keep up their price, do not let 
yourselves be forced to sell out below price, do not let yourselves be talked 
into the idea that your ware is not worth its price. do not make yourself 
ridiculous by a ``ridiculous price,'' but imitate the brave man who says, I 
will \textit{sell} my life (property) dear, the enemy shall not have it at a 
cheap \textit{bargain}; then you have recognized the reverse of Communism as 
the correct thing, and the word then is not ``Give up your property!'' but 
\textit{``Get the value out of} your property!''

Over the portal of our time stands not that ``Know thyself'' of Apollo, but 
a \textit{``Get the value out of thyself!''}

Proudhon calls property ``robbery'' (\textit{le vol}). But alien property -- 
and he is talking of this alone -- is not less existent by renunciation, 
cession, and humility; it is a \textit{present}. Why so sentimentally call for 
compassion as a poor victim of robbery, when one is just a foolish, cowardly 
giver of presents? Why here again put the fault on others as if they were 
robbing us, while we ourselves do bear the fault in leaving the others 
unrobbed? The poor are to blame for there being rich men.

Universally, no one grows indignant at \textit{his}, but at \textit{alien} 
property. They do not in truth attack property, but the alienation of 
property. They want to be able to call \textit{more}, not less, 
\textit{theirs}; they want to call everything \textit{theirs}. They are 
fighting, therefore, against \textit{alienness}, or, to form a word similar to 
property, against alienty. And how do they help themselves therein? Instead of 
transforming the alien into own, they play impartial and ask only that all 
property be left to a third party, \textit{e.g.} human society. They 
revindicate the alien not in their own name but in a third party's. Now the 
``egoistic'' coloring is wiped off, and everything is so clean and -- human!

Propertylessness or ragamuffinism, this then is the ``essence of 
Christianity,'' as it is essence of all religiousness (\textit{i.e.} 
godliness, morality, humanity), and only announced itself most clearly, and, 
as glad tidings, became a gospel capable of development, in the ``absolute 
religion.'' We have before us the most striking development in the present 
fight against property, a fight which is to bring ``Man'' to victory and 
make propertylessness complete: victorious humanity is the victory of 
--Christianity. But the ``Christianity exposed'' thus is feudalism 
completed. the most all-embracing feudal system, \textit{i.e.} perfect 
ragamuffinism.

Once more then, doubtless, a ``revolution'' against the feudal system? --

Revolution and insurrection must not be looked upon as synonymous. The former 
consists in an overturning of conditions, of the established condition or 
status, the State or society, and is accordingly a \textit{political} or 
\textit{social} act; the latter has indeed for its unavoidable consequence a 
transformation of circumstances, yet does not start from it but from men's 
discontent with themselves, is not an armed rising, but a rising of 
individuals, a getting up, without regard to the arrangements that spring from 
it. The Revolution aimed at new \textit{arrangements}; insurrection leads us 
no longer to \textit{let} ourselves be arranged, but to arrange ourselves, and 
sets no glittering hopes on ``institutions.'' It is not a fight against the 
established, since, if it prospers, the established collapses of itself; it is 
only a working forth of me out of the established. If I leave the established, 
it is dead and passes into decay. Now, as my object is not the overthrow of an 
established order but my elevation above it, my purpose and deed are not a 
political or social but (as directed toward myself and my ownness alone) an 
\textit{egoistic} purpose and deed.

The revolution commands one to make \textit{arrangements}, the 
insurrection\footnote{[\textit{Emp\"orung}]} demands that he \textit{rise or 
exalt himself}.\footnote{[\textit{sich auf-oder emp\"orzurichten}]} What 
\textit{constitution} was to be chosen, this question busied the revolutionary 
heads, and the whole political period foams with constitutional fights and 
constitutional questions, as the social talents too were uncommonly inventive 
in societary arrangements (phalansteries etc.). The insurgent\footnote{To 
secure myself against a criminal charge I superfluously make the express 
remark that I choose the word ``insurrection'' on account of its 
etymological sense, and therefore am not using it in the limited sense which 
is disallowed by the penal code.} strives to become constitutionless.

While, to get greater clearness, I am thinking up a comparison, the founding 
of Christianity comes unexpectedly into my mind. On the liberal side it is 
noted as a bad point in the first Christians that they preached obedience to 
the established heathen civil order, enjoined recognition of the heathen 
authorities, and confidently delivered a command, ``Give to the emperor that 
which is the emperor's.'' Yet how much disturbance arose at the same time 
against the Roman supremacy, how mutinous did the Jews and even the Romans 
show themselves against their own temporal government! In short, how popular 
was ``political discontent!'' Those Christians would hear nothing of it; 
would not side with the ``liberal tendencies.'' The time was politically so 
agitated that, as is said in the gospels, people thought they could not accuse 
the founder of Christianity more successfully than if they arraigned him for 
``political intrigue,'' and yet the same gospels report that he was 
precisely the one who took least part in these political doings. But why was 
he not a revolutionist, not a demagogue, as the Jews would gladly have seen 
him? Why was he not a liberal? Because he expected no salvation from a change 
of \textit{conditions}, and this whole business was indifferent to him. He was 
not a revolutionist, like \textit{e.g.} Caesar, but an insurgent; not a 
State-overturner, but one who straightened \textit{himself} up. That was why 
it was for him only a matter of ``Be ye wise as serpents,'' which expresses 
the same sense as, in the special case, that ``Give to the emperor that which 
is the emperor's''; for he was not carrying on any liberal or political fight 
against the established authorities, but wanted to walk his \textit{own} way, 
untroubled about, and undisturbed by, these authorities. Not less indifferent 
to him than the government were its enemies, for neither understood what he 
wanted, and he had only to keep them off from him with the wisdom of the 
serpent. But, even though not a ringleader of popular mutiny, not a demagogue 
or revolutionist, he (and every one of the ancient Christians) was so much the 
more an \textit{insurgent}, who lifted himself above everything that seemed 
sublime to the government and its opponents, and absolved himself from 
everything that they remained bound to, and who at the same time cut off the 
sources of life of the whole heathen world, with which the established State 
must wither away as a matter of course; precisely because he put from him the 
upsetting of the established, he was its deadly enemy and real annihilator; 
for he walled it in, confidently and recklessly carrying up the building of 
\textit{his} temple over it, without heeding the pains of the immured.

Now, as it happened to the heathen order of the world, will the Christian 
order fare likewise? A revolution certainly does not bring on the end if an 
insurrection is not consummated first!

My intercourse with the world, what does it aim at? I want to have the 
enjoyment of it, therefore it must be my property, and therefore I want to win 
it. I do not want the liberty of men, nor their equality; I want only 
\textit{my} power over them, I want to make them my property, \textit{i.e. 
material for enjoyment}. And, if I do not succeed in that, well, then I call 
even the power over life and death, which Church and State reserved to 
themselves -- mine. Brand that officer's widow who, in the flight in Russia, 
after her leg has been shot away, takes the garter from it, strangles her 
child therewith, and then bleeds to death alongside the corpse -- brand the 
memory of the -- infanticide. Who knows, if this child had remained alive, how 
much it might have ``been of use to the world!'' The mother murdered it 
because she wanted to die \textit{satisfied} and at rest. Perhaps this case 
still appeals to your sentimentality, and you do not know how to read out of 
it anything further. Be it so; I on my part use it as an example for this, 
that \textit{my} satisfaction decides about my relation to men, and that I do 
not renounce, from any access of humility, even the power over life and death.

As regards ``social duties'' in general, another does not give me my 
position toward others, therefore neither God nor humanity prescribes to me my 
relation to men, but I give myself this position. This is more strikingly said 
thus: I have no \textit{duty} to others, as I have a duty even to myself 
(\textit{e.g.} that of self-preservation, and therefore not suicide) only so 
long as I distinguish myself from myself (my immortal soul from my earthly 
existence, etc.).

I no longer \textit{humble} myself before any power, and I recognize that all 
powers are only my power, which I have to subject at once when they threaten 
to become a power \textit{against} or \textit{above} me; each of them must be 
only one of \textit{my means} to carry my point, as a hound is our power 
against game, but is killed by us if it should fall upon us ourselves. All 
powers that dominate me I then reduce to serving me. The idols exist through 
me; I need only refrain from creating them anew, then they exist no longer: 
``higher powers'' exist only through my exalting them and abasing myself.

Consequently my relation to the world is this: I no longer do anything for it 
``for God's sake,'' I do nothing ``for man's sake,'' but what I do I do 
``for my sake.'' Thus alone does the world satisfy me, while it is 
characteristic of the religious standpoint, in which I include the moral and 
humane also, that from it everything remains a pious wish (\textit{pium 
desiderium}), \textit{i.e.} an other-world matter, something unattained. Thus 
the general salvation of men, the moral world of a general love, eternal 
peace, the cessation of egoism, etc. ``Nothing in this world is perfect.'' 
With this miserable phrase the good part from it, and take flight into their 
closet to God, or into their proud ``self-consciousness.'' But we remain in 
this ``imperfect'' world, because even so we can use it for our -- 
self-enjoyment.

My intercourse with the world consists in my enjoying it, and so consuming it 
for my self-enjoyment. \textit{Intercourse} is the \textit{enjoyment of the 
world}, and belongs to my -- self-enjoyment.

\section[3. My Self-Enjoyment]{\centering 3. My Self-Enjoyment}

We stand at the boundary of a period. The world hitherto took thought for 
nothing but the gain of life, took care for -- \textit{life}. For whether all 
activity is put on the stretch for the life of this world or of the other, for 
the temporal or for the eternal, whether one hankers for ``daily bread'' 
(``Give us our daily bread'') or for ``holy bread'' (``the true bread 
from heaven'' ``the bread of God, that comes from heaven and \textit{gives 
life} to the world''; ``the bread of life,'' John 6), whether one takes 
care for ``dear life'' or for ``life to eternity'' -- this does not change 
the object of the strain and care, which in the one case as in the other shows 
itself to be \textit{life}. Do the modern tendencies announce themselves 
otherwise? People now want nobody to be embarrassed for the most indispensable 
necessaries of life, but want every one to feel secure as to these; and on the 
other hand they teach that man has this life to attend to and the real world 
to adapt himself to, without vain care for another.

Let us take up the same thing from another side. When one is anxious only to 
live, he easily, in this solicitude, forgets the enjoyment of life. If his 
only concern is for life, and he thinks ``if I only have my dear life,'' he 
does not apply his full strength to using, \textit{i.e.}, enjoying, life. But 
how does one use life? In using it up, like the candle, which one uses in 
burning it up. One uses life, and consequently himself the living one, in 
\textit{consuming} it and himself. \textit{Enjoyment of life} is using life 
up.

Now -- we are in search of the \textit{enjoyment} of life! And what did the 
religious world do? It went in search of life. Wherein consists the true life, 
the blessed life; etc.? How is it to be attained? What must man do and become 
in order to become a truly living man? How does he fulfil this calling? These 
and similar questions indicate that the askers were still seeking for 
\textit{themselves --} to wit, themselves in the true sense, in the sense of 
true living. ``What I am is foam and shadow; what I shall be is my true 
self.'' To chase after this self, to produce it, to realize it, constitutes 
the hard task of mortals, who die only to \textit{rise again}, live only to 
die, live only to find the true life.

Not till I am certain of myself, and no longer seeking for myself, am I really 
my property; I have myself, therefore I use and enjoy myself. On the other 
hand, I can never take comfort in myself as long as I think that I have still 
to find my true self and that it must come to this, that not I but Christ or 
some other spiritual, \textit{i.e.} ghostly, self (\textit{e.g.} the true 
man, the essence of man, etc.) lives in me.

A vast interval separates the two views. In the old I go toward myself, in the 
new I start from myself; in the former I long for myself, in the latter I have 
myself and do with myself as one does with any other property -- I enjoy 
myself at my pleasure. I am no longer afraid for my life, but ``squander'' 
it.

Henceforth, the question runs, not how one can acquire life, but how one can 
squander, enjoy it; or, not how one is to produce the true self in himself, 
but how one is to dissolve himself, to live himself out.

What else should the ideal be but the sought-for ever-distant self? One seeks 
for himself, consequently one doth not yet have himself; one aspires toward 
what one \textit{ought} to be, consequently one \textit{is} not it. One lives 
in \textit{longing} and has lived thousands of years in it, in \textit{hope}. 
Living is quite another thing in -- \textit{enjoyment!}

Does this perchance apply only to the so-called pious? No, it applies to all 
who belong to the departing period of history, even to its men of pleasure. 
For them too the work-days were followed by a Sunday, and the rush of the 
world by the dream of a better world, of a general happiness of humanity; in 
short by an ideal. But philosophers especially are contrasted with the pious. 
Now, have they been thinking of anything else than the ideal, been planning 
for anything else than the absolute self? Longing and hope everywhere, and 
nothing but these. For me, call it romanticism.

If the \textit{enjoyment of life} is to triumph over the \textit{longing for 
life} or hope of life, it must vanquish this in its double significance which 
Schiller introduces in his ``Ideal and Life''; it must crush spiritual and 
secular poverty, exterminate the ideal and -- the want of daily bread. He who 
must expend his life to prolong life cannot enjoy it, and he who is still 
seeking for his life does not have it and can as little enjoy it: both are 
poor, but ``blessed are the poor.''

Those who are hungering for the true life have no power over their present 
life, but must apply it for the purpose of thereby gaining that true life, and 
must sacrifice it entirely to this aspiration and this task. If in the case of 
those devotees who hope for a life in the other world, and look upon that in 
this world as merely a preparation for it, the tributariness of their earthly 
existence, which they put solely into the service of the hoped-for heavenly 
existence, is pretty distinctly apparent; one would yet go far wrong if one 
wanted to consider the most rationalistic and enlightened as less 
self-sacrificing. Oh, there is to be found in the ``true life'' a much more 
comprehensive significance than the ``heavenly'' is competent to express. 
Now, is not -- to introduce the liberal concept of it at once -- the 
``human'' and ``truly human'' life the true one? And is every one already 
leading this truly human life from the start, or must he first raise himself 
to it with hard toil? Does he already have it as his present life, or must he 
struggle for it as his future life, which will become his part only when he 
``is no longer tainted with any egoism''? In this view life exists only to 
gain life, and one lives only to make the essence of man alive in oneself, one 
lives for the sake of this essence. One has his life only in order to procure 
by means of it the ``true'' life cleansed of all egoism. Hence one is afraid 
to make any use he likes of his life: it is to serve only for the ``right 
use.''

In short, one has a \textit{calling in life}, a task in life; one has 
something to realize and produce by his life, a something for which our life 
is only means and implement, a something that is worth more than this life, a 
something to which one \textit{owes} his life. One has a God who asks a 
\textit{living sacrifice}. Only the rudeness of human sacrifice has been lost 
with time; human sacrifice itself has remained unabated, and criminals hourly 
fall sacrifices to justice, and we ``poor sinners'' slay our own selves as 
sacrifices for ``the human essence,'' the ``idea of mankind,'' 
``humanity,'' and whatever the idols or gods are called besides.

But, because we owe our life to that something, therefore --this is the next 
point -- we have no right to take it from us.

The conservative tendency of Christianity does not permit thinking of death 
otherwise than with the purpose to take its sting from it and -- live on and 
preserve oneself nicely. The Christian lets everything happen and come upon 
him if he -- the arch-Jew -- can only haggle and smuggle himself into heaven; 
he must not kill himself, he must only -- preserve himself and work at the 
``preparation of a future abode.'' Conservatism or ``conquest of death'' 
lies at his heart; ``the last enemy that is abolished is death.''\footnote{1 
Cor. 15. 26.} ``Christ has taken the power from death and brought life and 
\textit{imperishable} being to light by the gospel.''\footnote{2 Tim. 1. 10.} 
``Imperishableness,'' stability.

The moral man wants the good, the right; and, if he takes to the means that 
lead to this goal, really lead to it, then these means are not \textit{his} 
means, but those of the good, right, etc., itself. These means are never 
immoral, because the good end itself mediates itself through them: the end 
sanctifies the means. They call this maxim jesuitical, but it is ``moral'' 
through and through. The moral man acts \textit{in the service} of an end or 
an idea: he makes himself the \textit{tool} of the idea of the good, as the 
pious man counts it his glory to be a tool or instrument of God. To await 
death is what the moral commandment postulates as the good; to give it to 
oneself is immoral and bad: \textit{suicide} finds no excuse before the 
judgment-seat of morality. If the religious man forbids it because ``you have 
not given yourself life, but God, who alone can also take it from you again'' 
(as if, even taking in this conception, God did not take it from me just as 
much when I kill myself as when a tile from the roof, or a hostile bullet, 
fells me; for he would have aroused the resolution of death in me too!), the 
moral man forbids it because I owe my life to the fatherland, etc., ``because 
I do not know whether I may not yet accomplish good by my life.'' Of course, 
for in me good loses a tool, as God does an instrument. If I am immoral, the 
good is served in my \textit{amendment}; if I am ``ungodly,'' God has joy in 
my \textit{penitence}. Suicide, therefore, is ungodly as well as nefarious. If 
one whose standpoint is religiousness takes his own life, he acts in 
forgetfulness of God; but, if the suicide's standpoint is morality, he acts in 
forgetfulness of duty, immorally. People worried themselves much with the 
question whether Emilia Galotti's death can be justified before morality (they 
take it as if it were suicide, which it is too in substance). That she is so 
infatuated with chastity, this moral good, as to yield up even her life for it 
is certainly moral; but, again, that she fears the weakness of her flesh is 
immoral.\footnote{[See the next to the last scene of the tragedy:

ODOARDO: Under the pretext of a judicial investigation he tears you out of our 
arms and takes you to Grimaldi. ...

EMILIA: Give me that dagger, father, me! ...

ODOARDO: No, no! Reflect -- You too have only one life to lose.

EMILIA: And only one innocence!

ODOARDO: Which is above the reach of any violence. --

EMILIA: But not above the reach of any seduction. -- Violence! violence! Who 
cannot defy violence? What is called violence is nothing; seduction is the 
true violence. -- I have blood, father; blood as youthful and warm as 
anybody's. My senses are senses. -- I can warrant nothing. I am sure of 
nothing. I know Grimaldi's house. It is the house of pleasure. An hour there, 
under my mother's eyes -- and there arose in my soul so much tumult as the 
strictest exercises of religion could hardly quiet in weeks. -- Religion! And 
what religion? -- To escape nothing worse, thousands sprang into the water and 
are saints. -- Give me that dagger, father, give it to me. ...

EMILIA: Once indeed there was a father who. to save his daughter from shame, 
drove into her heart whatever steel he could quickest find -- gave life to her 
for the second time. But all such deeds are of the past! Of such fathers there 
are no more.

ODOARDO: Yes, daughter, yes! (\textit{Stabs her}.)]

}

Such contradictions form the tragic conflict universally in the moral drama; 
and one must think and feel morally to be able to take an interest in it.

What holds good of piety and morality will necessarily apply to humanity also, 
because one owes his life likewise to man, mankind or the species. Only when I 
am under obligation to no being is the maintaining of life -- my affair. ``A 
leap from this bridge makes me free!''

But, if we owe the maintaining of our life to that being that we are to make 
alive in ourselves, it is not less our duty not to lead this life according to 
\textit{our} pleasure, but to shape it in conformity to that being. All my 
feeling, thinking, and willing, all my doing and designing, belongs to -- him.

What is in conformity to that being is to be inferred from his concept; and 
how differently has this concept been conceived! or how differently has that 
being been imagined! What demands the Supreme Being makes on the Mohammedan; 
what different ones the Christian, again, thinks he hears from him; how 
divergent, therefore, must the shaping of the lives of the two turn out! Only 
this do all hold fast, that the Supreme Being is to 
\textit{judge}\footnote{[Or, ``regulate'' (\textit{richten})]} our life.

But the pious who have their judge in God, and in his word a book of 
directions for their life, I everywhere pass by only reminiscently, because 
they belong to a period of development that has been lived through, and as 
petrifactions they may remain in their fixed place right along; in our time it 
is no longer the pious, but the liberals, who have the floor, and piety itself 
cannot keep from reddening its pale face with liberal coloring. But the 
liberals do not adore their judge in God, and do not unfold their life by the 
directions of the divine word, but regulate\footnote{[\textit{richten}]} 
themselves by man: they want to be not ``divine'' but ``human,'' and to 
live so.

Man is the liberal's supreme being, man the \textit{judge} of his life, 
humanity his \textit{directions}, or catechism. God is spirit, but man is the 
``most perfect spirit,'' the final result of the long chase after the spirit 
or of the ``searching in the depths of the Godhead,'' \textit{i.e.} in the 
depths of the spirit.

Every one of your traits is to be human; you yourself are to be so from top to 
toe, in the inward as in the outward; for humanity is your calling.

Calling -- destiny -- task! --

What one can become he does become. A born poet may well be hindered by the 
disfavor of circumstances from standing on the high level of his time, and, 
after the great studies that are indispensable for this, producing 
\textit{consummate} works of art; but he will make poetry, be he a plowman or 
so lucky as to live at the court of Weimar. A born musician will make music, 
no matter whether on all instruments or only on an oaten pipe. A born 
philosophical head can give proof of itself as university philosopher or as 
village philosopher. Finally, a born dolt, who, as is very well compatible 
with this, may at the same time be a sly-boots, will (as probably every one 
who has visited schools is in a position to exemplify to himself by many 
instances of fellow-scholars) always remain a blockhead, let him have been 
drilled and trained into the chief of a bureau, or let him serve that same 
chief as bootblack. Nay, the born shallow-pates indisputably form the most 
numerous class of men. And why. indeed, should not the same distinctions show 
themselves in the human species that are unmistakable in every species of 
beasts? The more gifted and the less gifted are to be found everywhere.

Only a few, however, are so imbecile that one could not get ideas into them. 
Hence, people usually consider all men capable of having religion. In a 
certain degree they may be trained to other ideas too, \textit{e.g.} to some 
musical intelligence, even some philosophy. At this point then the priesthood 
of religion, of morality, of culture, of science, etc., takes its start, and 
the Communists, \textit{e.g.} want to make everything accessible to all by 
their ``public school.'' There is heard a common assertion that this 
``great mass'' cannot get along without religion; the Communists broaden it 
into the proposition that not only the ``great mass,'' but absolutely all, 
are called to everything.

Not enough that the great mass has been trained to religion, now it is 
actually to have to occupy itself with ``everything human.'' Training is 
growing ever more general and more comprehensive.

You poor beings who could live so happily if you might skip according to your 
mind, you are to dance to the pipe of schoolmasters and bear-leaders, in order 
to perform tricks that you yourselves would never use yourselves for. And you 
do not even kick out of the traces at last against being always taken 
otherwise than you want to give yourselves. No, you mechanically recite to 
yourselves the question that is recited to you: ``What am I called to? What 
\textit{ought} I to do?'' You need only ask thus, to have yourselves 
\textit{told} what you ought to do and \textit{ordered} to do it, to have your 
\textit{calling} marked out for you, or else to order yourselves and impose it 
on yourselves according to the spirit's prescription. Then in reference to the 
will the word is, I will to do what I \textit{ought}.

A man is ``called'' to nothing, and has no ``calling,'' no ``destiny,'' 
as little as a plant or a beast has a ``calling.'' The flower does not 
follow the calling to complete itself, but it spends all its forces to enjoy 
and consume the world as well as it can -- \textit{i.e.} it sucks in as much 
of the juices of the earth, as much air of the ether, as much light of the 
sun, as it can get and lodge. The bird lives up to no calling, but it uses its 
forces as much as is practicable; it catches beetles and sings to its heart's 
delight. But the forces of the flower and the bird are slight in comparison to 
those of a man, and a man who applies his forces will affect the world much 
more powerfully than flower and beast. A calling he has not, but he has forces 
that manifest themselves where they are because their being consists solely in 
their manifestation, and are as little able to abide inactive as life, which, 
if it ``stood still'' only a second, would no longer be life. Now, one might 
call out to the man, ``use your force.'' Yet to this imperative would be 
given the meaning that it was man's task to use his force. It is not so. 
Rather, each one really uses his force without first looking upon this as his 
calling: at all times every one uses as much force as he possesses. One does 
say of a beaten man that he ought to have exerted his force more; but one 
forgets that, if in the moment of succumbing he had the force to exert his 
forces (\textit{e.g.} bodily forces), he would not have failed to do it: even 
if it was only the discouragement of a minute, this was yet a --destitution of 
force, a minute long. Forces may assuredly be sharpened and redoubled, 
especially by hostile resistance or friendly assistance; but where one misses 
their application one may be sure of their absence too. One can strike fire 
out of a stone, but without the blow none comes out; in like manner a man too 
needs ``impact.''

Now, for this reason that forces always of themselves show themselves 
operative, the command to use them would be superfluous and senseless. To use 
his forces is not man's \textit{calling} and task, but is his \textit{act}, 
real and extant at all times. Force is only a simpler word for manifestation 
of force.

Now, as this rose is a true rose to begin with, this nightingale always a true 
nightingale, so I am not for the first time a true man when I fulfil my 
calling, live up to my destiny, but I am a ``true man'' from the start. My 
first babble is the token of the life of a ``true man,'' the struggles of my 
life are the outpourings of his force, my last breath is the last exhalation 
of the force of the ``man.''

The true man does not lie in the future, an object of longing, but lies, 
existent and real, in the present. Whatever and whoever I may be, joyous and 
suffering, a child or a graybeard, in confidence or doubt, in sleep or in 
waking, I am it, I am the true man.

But, if I am Man, and have really found in myself him whom religious humanity 
designated as the distant goal, then everything ``truly human'' is also 
\textit{my} own. What was ascribed to the idea of humanity belongs to me. That 
freedom of trade,

\textit{e.g.}, which humanity has yet to attain -- and which, like an 
enchanting dream, people remove to humanity's golden future -- I take by 
anticipation as my property, and carry it on for the time in the form of 
smuggling. There may indeed be but few smugglers who have sufficient 
understanding to thus account to themselves for their doings, but the instinct 
of egoism replaces their consciousness. Above I have shown the same thing 
about freedom of the press.

Everything is my own, therefore I bring back to myself what wants to withdraw 
from me; but above all I always bring myself back when I have slipped away 
from myself to any tributariness. But this too is not my calling, but my 
natural act.

Enough, there is a mighty difference whether I make myself the starting-point 
or the goal. As the latter I do not have myself, am consequently still alien 
to myself, am my \textit{essence}, my ``true essence,'' and this ``true 
essence,'' alien to me, will mock me as a spook of a thousand different 
names. Because I am not yet I, another (like God, the true man, the truly 
pious man, the rational man, the freeman, etc.) is I, my ego.

Still far from myself, I separate myself into two halves, of which one, the 
one unattained and to be fulfilled, is the true one. The one, the untrue, must 
be brought as a sacrifice; to wit, the unspiritual one. The other, the true, 
is to be the whole man; to wit, the spirit. Then it is said, ``The spirit is 
man's proper essence,'' or, ``man exists as man only spiritually.'' Now, 
there is a greedy rush to catch the spirit, as if one would then have bagged 
\textit{himself}; and so, in chasing after himself, one loses sight of 
himself, whom he is.

And, as one stormily pursues his own self, the never-attained, so one also 
despises shrewd people's rule to take men as they are, and prefers to take 
them as they should be; and, for this reason, hounds every one on after his 
should-be self and ``endeavors to make all into equally entitled, equally 
respectable, equally moral or rational men.''\footnote{\textit{``Der 
Kommunismus in der Schweiz}'', p. 24.}

Yes, ``if men were what they \textit{should} be, \textit{could} be, if all 
men were rational, all loved each other as brothers,'' then it would be a 
paradisiacal life.\footnote{\textit{Ibid}, p. 63} -- All right, men are as 
they should be, can be. What should they be? Surely not more than they can be! 
And what can they be? Not more, again, than they -- can, than they have the 
competence, the force, to be. But this they really are, because what they are 
not they are \textit{incapable} of being; for to be capable means -- really to 
be. One is not capable for anything that one really is not; one is not capable 
of anything that one does not really do. Could a man blinded by cataracts see? 
Oh, yes, if he had his cataracts successfully removed. But now he cannot see 
because he does not see. Possibility and reality always coincide. One can do 
nothing that one does not, as one does nothing that one cannot.

The singularity of this assertion vanishes when one reflects that the words 
``it is possible that.'' almost never contain another meaning than ``I can 
imagine that. . .,'' \textit{e.g.}, It is possible for all men to live 
rationally; \textit{e.g.}, I can imagine that all, etc. Now -- since my 
thinking cannot, and accordingly does not, cause all men to live rationally, 
but this must still be left to the men themselves -- general reason is for me 
only thinkable, a thinkableness, but as such in fact a \textit{reality} that 
is called a possibility only in reference to what I \textit{can} not bring to 
pass, to wit, the rationality of others. So far as depends on you, all men 
might be rational, for you have nothing against it; nay, so far as your 
thinking reaches, you perhaps cannot discover any hindrance either, and 
accordingly nothing does stand in the way of the thing in your thinking; it is 
thinkable to you.

As men are not all rational, though, it is probable that they -- cannot be so.

If something which one imagines to be easily possible is not, or does not 
happen, then one may be assured that something stands in the way of the thing, 
and that it is -- impossible. Our time has its art, science, etc.; the art may 
be bad in all conscience; but may one say that we deserved to have a better, 
and ``could'' have it if we only would? We have just as much art as we can 
have. Our art of today is the \textit{only art possible}, and therefore real, 
at the time.

Even in the sense to which one might at last still reduce the word 
``possible,'' that it should mean ``future,'' it retains the full force of 
the ``real.'' If one says, \textit{e.g.}, ``It is possible that the sun 
will rise tomorrow'' -- this means only, ``for today tomorrow is the real 
future''; for I suppose there is hardly need of the suggestion that a future 
is real ``future'' only when it has not yet appeared.

Yet wherefore this dignifying of a word? If the most prolific misunderstanding 
of thousands of years were not in ambush behind it, if this single concept of 
the little word ``possible'' were not haunted by all the spooks of possessed 
men, its contemplation should trouble us little here.

The thought, it was just now shown, rules the possessed world. Well, then, 
possibility is nothing but thinkableness, and innumerable sacrifices have 
hitherto been made to hideous \textit{thinkableness}. It was 
\textit{thinkable} that men might become rational; thinkable, that they might 
know Christ; thinkable, that they might become moral and enthusiastic for the 
good; thinkable, that they might all take refuge in the Church's lap; 
thinkable, that they might meditate, speak, and do, nothing dangerous to the 
State; thinkable, that they \textit{might} be obedient subjects; but, because 
it was thinkable, it was -- so ran the inference -- possible, and further, 
because it was possible to men (right here lies the deceptive point; because 
it is thinkable to me, it is possible to \textit{men}), therefore they ought 
to be so, it was their \textit{calling}; and finally -- one is to take men 
only according to this calling, only as \textit{called} men, ``not as they 
are, but as they ought to be.''

And the further inference? Man is not the individual, but man is a 
\textit{thought}, an \textit{ideal}, to which the individual is related not 
even as the child to the man, but as a chalk point to a point thought of, or 
as a -- finite creature to the eternal Creator, or, according to modern views, 
as the specimen to the species. Here then comes to light the glorification of 
``humanity,'' the ``eternal, immortal,'' for whose glory (\textit{in 
majorem humanitatis gloriam}) the individual must devote himself and find his 
``immortal renown'' in having done something for the ``spirit of 
humanity.''

Thus the \textit{thinkers} rule in the world as long as the age of priests or 
of schoolmasters lasts, and what they think of is possible, but what is 
possible must be realized. They \textit{think} an ideal of man, which for the 
time is real only in their thoughts; but they also think the possibility of 
carrying it out, and there is no chance for dispute, the carrying out is 
really -- thinkable, it is an -- idea.

But you and I, we may indeed be people of whom a Krummacher can \textit{think} 
that we might yet become good Christians; if, however, he wanted to ``labor 
with'' us, we should soon make it palpable to him that our Christianity is 
only \textit{thinkable}, but in other respects \textit{impossible}; if he 
grinned on and on at us with his obtrusive \textit{thoughts}, his ``good 
belief,'' he would have to learn that we do not at all \textit{need} to 
become what we do not like to become.

And so it goes on, far beyond the most pious of the pious. ``If all men were 
rational, if all did right, if all were guided by philanthropy, etc.''! 
Reason, right, philanthropy, are put before the eyes of men as their calling, 
as the goal of their aspiration. And what does being rational mean? Giving 
oneself a hearing?\footnote{[Cf. note p. 81]} No, reason is a book full of 
laws, which are all enacted against egoism.

History hitherto is the history of the \textit{intellectual} man. After the 
period of sensuality, history proper begins; \textit{i.e.} the period of 
intellectuality,\footnote{[\textit{Geistigkeit}]} 
spirituality,\footnote{[\textit{Geistlichkeit}]} non-sensuality, 
supersensuality, nonsensicality. Man now begins to want to be and become 
\textit{something}. What? Good, beautiful, true; more precisely, moral, pious, 
agreeable, etc. He wants to make of himself a ``proper man,'' ``something 
proper.'' \textit{Man} is his goal, his ought, his destiny, calling, task, 
his -- \textit{ideal}; he is to himself a future, otherworldly he. And 
\textit{what} makes a ``proper fellow'' of him? Being true, being good, 
being moral, etc. Now he looks askance at every one who does not recognize the 
same ``what,'' seek the same morality, have the same faith, he chases out 
``separatists, heretics, sects,'' etc.

No sheep, no dog, exerts itself to become a ``proper sheep, a proper dog''; 
no beast has its essence appear to it as a task, \textit{i.e.} as a concept 
that it has to realize. It realizes itself in living itself out, in dissolving 
itself, passing away. It does not ask to be or to become anything 
\textit{other} than it is.

Do I mean to advise you to be like the beasts? That you ought to become beasts 
is an exhortation which I certainly cannot give you, as that would again be a 
task, an ideal (``How doth the little busy bee improve each shining hour. In 
works of labor or of skill I would be busy too, for Satan finds some mischief 
still for idle hands to do''). It would be the same, too, as if one wished 
for the beasts that they should become human beings. Your nature is, once for 
all, a human one; you are human natures, human beings. But, just because you 
already are so, you do not still need to become so. Beasts too are 
``trained,'' and a trained beast executes many unnatural things. But a 
trained dog is no better for itself than a natural one, and has no profit from 
it, even if it is more companionable for us.

Exertions to ``form'' all men into moral, rational, pious, human, 
``beings'' (\textit{i.e.} training) were in vogue from of yore. They are 
wrecked against the indomitable quality of I, against own nature, against 
egoism. Those who are trained never attain their ideal, and only profess with 
their \textit{mouth} the sublime principles, or make a \textit{profession}, a 
profession of faith. In face of this profession they must in \textit{life} 
``acknowledge themselves sinners altogether,'' and they fall short of their 
ideal, are ``weak men,'' and bear with them the consciousness of ``human 
weakness.''

It is different if you do not chase after an \textit{ideal} as your 
``destiny,'' but dissolve yourself as time dissolves everything. The 
dissolution is not your ``destiny,'' because it is present time.

Yet the \textit{culture}, the religiousness, of men has assuredly made them 
free, but only free from one lord, to lead them to another. I have learned by 
religion to tame my appetite, I break the world's resistance by the cunning 
that is put in my hand by \textit{science}; I even serve no man; ``I am no 
man's lackey.'' But then it comes. You must obey God more than man. Just so I 
am indeed free from irrational determination by my impulses. but obedient to 
the master \textit{Reason}. I have gained ``spiritual freedom,'' ``freedom 
of the spirit.'' But with that I have then become subject to that very 
\textit{spirit}. The spirit gives me orders, reason guides me, they are my 
leaders and commanders. The ``rational,'' the ``servants of the spirit,'' 
rule. But, if \textit{I} am not flesh, I am in truth not spirit either. 
Freedom of the spirit is servitude of me, because I am more than spirit or 
flesh.

Without doubt culture has made me \textit{powerful}. It has given me power 
over all \textit{motives}, over the impulses of my nature as well as over the 
exactions and violences of the world. I know, and have gained the force for it 
by culture, that I need not let myself be coerced by any of my appetites, 
pleasures, emotions, etc.; I am their -- \textit{master}; in like manner I 
become, through the sciences and arts, the \textit{master} of the refractory 
world, whom sea and earth obey, and to whom even the stars must give an 
account of themselves. The spirit has made me \textit{master. --} But I have 
no power over the spirit itself. From religion (culture) I do learn the means 
for the ``vanquishing of the world,'' but not how I am to subdue 
\textit{God} too and become master of him; for God ``is the spirit.'' And 
this same spirit, of which I am unable to become master, may have the most 
manifold shapes; he may be called God or National Spirit, State, Family, 
Reason, also -- Liberty, Humanity, Man.

\textit{I} receive with thanks what the centuries of culture have acquired for 
me; I am not willing to throw away and give up anything of it: I have not 
lived in vain. The experience that I have \textit{power} over my nature, and 
need not be the slave of my appetites, shall not be lost to me; the experience 
that I can subdue the world by culture's means is too dear- bought for me to 
be able to forget it. But I want still more.

People ask, what can man do? What can he accomplish? What goods procure, and 
put down the highest of everything as a calling. As if everything were 
possible to \textit{me!}

If one sees somebody going to ruin in a mania, a passion, etc. (\textit{e.g.} 
in the huckster-spirit, in jealousy), the desire is stirred to deliver him out 
of this possession and to help him to ``self-conquest.'' ``We want to make 
a man of him!'' That would be very fine if another possession were not 
immediately put in the place of the earlier one. But one frees from the love 
of money him who is a thrall to it, only to deliver him over to piety, 
humanity, or some principle else, and to transfer him to a \textit{fixed 
standpoint} anew.

This transference from a narrow standpoint to a sublime one is declared in the 
words that the sense must not be directed to the perishable, but to the 
imperishable alone: not to the temporal, but to the eternal, absolute, divine, 
purely human, etc. -- to the spiritual.

People very soon discerned that it was not indifferent what one set his 
affections on, or what one occupied himself with; they recognized the 
importance of the \textit{object}. An object exalted above the individuality 
of things is the \textit{essence} of things; yes, the essence is alone the 
thinkable in them. it is for the \textit{thinking} man. Therefore direct no 
longer your \textit{sense} to the \textit{things}, but your \textit{thoughts} 
to the \textit{essence}. ``Blessed are they who see not, and yet believe''; 
\textit{i.e.}, blessed are the \textit{thinkers}, for they have to do with 
the invisible and believe in it. Yet even an object of thought, that 
constituted an essential point of contention centuries long, comes at last to 
the point of being ``No longer worth speaking of.'' This was discerned, but 
nevertheless people always kept before their eyes again a self-valid 
importance of the object, an absolute value of it, as if the doll were not the 
most important thing to the child, the Koran to the Turk. As long as I am not 
the sole important thing to myself, it is indifferent of what object I ``make 
much,'' and only my greater or lesser \textit{delinquency} against it is of 
value. The degree of my attachment and devotion marks the standpoint of my 
liability to service, the degree of my sinning shows the measure of my 
ownness.

But finally, and in general, one must know how to ``put everything out of his 
mind,'' if only so as to be able to -- go to sleep. Nothing may occupy us 
with which \textit{we} do not occupy ourselves: the victim of ambition cannot 
run away from his ambitious plans, nor the God-fearing man from the thought of 
God; infatuation and possessedness coincide.

To want to realize his essence or live comfortably to his concept (which with 
believers in God signifies as much as to be ``pious,'' and with believers in 
humanity means living ``humanly'') is what only the sensual and sinful man 
can propose to himself, the man so long as he has the anxious choice between 
happiness of sense and peace of soul, so long as he is a ``poor sinner.'' 
The Christian is nothing but a sensual man who, knowing of the sacred and 
being conscious that he violates it, sees in himself a poor sinner: 
sensualness, recognized as ``sinfulness,'' is Christian consciousness, is 
the Christian himself. And if ``sin'' and ``sinfulness'' are now no longer 
taken into the mouths of moderns, but, instead of that, ``egoism,'' 
``self-seeking,'' ``selfishness,'' etc., engage them; if the devil has 
been translated into the ``un-man'' or ``egoistic man'' -- is the 
Christian less present then than before? Is not the old discord between good 
and evil -- is not a judge over us, man -- is not a calling, the calling to 
make oneself man -- left? If they no longer name it calling, but ``task'' 
or, very likely, ``duty,'' the change of name is quite correct, because 
``man'' is not, like God, a personal being that can ``call''; but outside 
the name the thing remains as of old.

\myhrule


Every one has a relation to objects, and more, every one is differently 
related to them. Let us choose as an example that book to which millions of 
men had a relation for two thousand years, the Bible. What is it, what was it, 
to each? Absolutely, only what he \textit{made out of it!} For him who makes 
to himself nothing at all out of it, it is nothing at all; for him who uses it 
as an amulet, it has solely the value, the significance, of a means of 
sorcery; for him who, like children, plays with it, it is nothing but a 
plaything, etc.

Now, Christianity asks that it shall \textit{be the same for all}: say the 
sacred book or the ``sacred Scriptures.'' This means as much as that the 
Christian's view shall also be that of other men, and that no one may be 
otherwise related to that object. And with this the ownness of the relation is 
destroyed, and one mind, one disposition, is fixed as the ``true'', the 
``only true'' one. In the limitation of the freedom to make of the Bible 
what I will, the freedom of making in general is limited; and the coercion of 
a view or a judgment is put in its place. He who should pass the judgment that 
the Bible was a long error of mankind would judge -- \textit{criminally}.

In fact, the child who tears it to pieces or plays with it, the Inca Atahualpa 
who lays his ear to it and throws it away contemptuously when it remains dumb, 
judges just as correctly about the Bible as the priest who praises in it the 
``Word of God,'' or the critic who calls it a job of men's hands. For how we 
toss things about is the affair of our \textit{option}, our \textit{free 
will}: we use them according to our \textit{heart's pleasure}, or, more 
clearly, we use them just as we \textit{can}. Why, what do the parsons scream 
about when they see how Hegel and the speculative theologians make speculative 
thoughts out of the contents of the Bible? Precisely this, that they deal with 
it according to their heart's pleasure, or ``proceed arbitrarily with it.''

But, because we all show ourselves arbitrary in the handling of objects, 
\textit{i.e.} do with them as we \textit{like} best, at our \textit{liking} 
(the philosopher likes nothing so well as when he can trace out an ``idea'' 
in everything, as the God-fearing man likes to make God his friend by 
everything, and so, \textit{e.g.}, by keeping the Bible sacred), therefore we 
nowhere meet such grievous arbitrariness, such a frightful tendency to 
violence, such stupid coercion, as in this very domain of our -- \textit{own 
free will}. If \textit{we} proceed arbitrarily in taking the sacred objects 
thus or so, how is it then that we want to take it ill of the parson-spirits 
if they take us just as arbitrarily, \textit{in their fashion}, and esteem us 
worthy of the heretic's fire or of another punishment, perhaps of the -- 
censorship?

What a man is, he makes out of things; ``as you look at the world, so it 
looks at you again.'' Then the wise advice makes itself heard again at once, 
You must only look at it ``rightly, unbiasedly,'' etc. As if the child did 
not look at the Bible ``rightly and unbiasedly'' when it makes it a 
plaything. That shrewd precept is given us, \textit{e.g.} by Feuerbach. One 
does look at things rightly when one makes of them what one \textit{will} (by 
things objects in general are here understood, \textit{e.g.} God, our 
fellowmen, a sweetheart, a book, a beast, etc.). And therefore the things and 
the looking at them are not first, but I am, my will is. One \textit{will} 
brings thoughts out of the things, \textit{will} discover reason in the world, 
\textit{will} have sacredness in it: therefore one shall find them. ``Seek 
and ye shall find.'' \textit{What} I will seek, I determine: I want, 
\textit{e.g.}, to get edification from the Bible; it is to be found; I want 
to read and test the Bible thoroughly; my outcome will be a thorough 
instruction and criticism -- to the extent of my powers. I elect for myself 
what I have a fancy for, and in electing I show myself -- arbitrary.

Connected with this is the discernment that every judgment which I pass upon 
an object is the \textit{creature} of my will; and that discernment again 
leads me to not losing myself in the \textit{creature}, the judgment, but 
remaining the \textit{creator}, the judge, who is ever creating anew. All 
predicates of objects are my statements, my judgments, my -- creatures. If 
they want to tear themselves loose from me and be something for themselves, or 
actually overawe me, then I have nothing more pressing to do than to take them 
back into their nothing, into me the creator. God, Christ, Trinity, morality, 
the good, etc., are such creatures, of which I must not merely allow myself to 
say that they are truths, but also that they are deceptions. As I once willed 
and decreed their existence, so I want to have license to will their non- 
existence too; I must not let them grow over my head, must not have the 
weakness to let them become something ``absolute,'' whereby they would be 
eternalized and withdrawn from my power and decision. With that I should fall 
a prey to the \textit{principle of stability}, the proper life-principle of 
religion, which concerns itself with creating ``sanctuaries that must not be 
touched,'' ``eternal truths'' -- in short, that which shall be ``sacred'' -- and depriving you of what is \textit{yours}.

The object makes us into possessed men in its sacred form just as in its 
profane, as a supersensuous object, just as it does as a sensuous one. The 
appetite or mania refers to both, and avarice and longing for heaven stand on 
a level. When the rationalists wanted to win people for the sensuous world, 
Lavater preached the longing for the invisible. The one party wanted to call 
forth \textit{emotion}, the other \textit{motion}, activity.

 The conception of objects is altogether diverse, even as God, Christ, the 
world, were and are conceived of in the most manifold wise. In this every one 
is a ``dissenter,'' and after bloody combats so much has at last been 
attained, that opposite views about one and the same object are no longer 
condemned as heresies worthy of death. The ``dissenters'' reconcile 
themselves to each other. But why should I only dissent (think otherwise) 
about a thing? Why not push the thinking otherwise to its last extremity, that 
of no longer having any regard at all for the thing, and therefore thinking 
its nothingness, crushing it? Then the \textit{conception} itself has an end, 
because there is no longer anything to conceive of. Why am I to say, let us 
suppose, ``God is not Allah, not Brahma, not Jehovah, but -- God''; but not, 
``God is nothing but a deception''? Why do people brand me if I am an 
``atheist''? Because they put the creature above the creator (``They honor 
and serve the creature more than the Creator''\footnote{Rom. 1. 25.}) and 
require a \textit{ruling object}, that the subject may be right 
\textit{submissive}. I am to bend \textit{beneath} the absolute, I 
\textit{ought} to.

By the ``realm of thoughts'' Christianity has completed itself; the thought 
is that inwardness in which all the world's lights go out, all existence 
becomes existenceless, the inward. man (the heart, the head) is all in all. 
This realm of thoughts awaits its deliverance, awaits, like the Sphinx, 
Oedipus's key- word to the riddle, that it may enter in at last to its death. 
I am the annihilator of its continuance, for in the creator's realm it no 
longer forms a realm of its own, not a State in the State, but a creature of 
my creative -- thoughtlessness. Only together and at the same time with the 
benumbed \textit{thinking} world can the world of Christians, Christianity and 
religion itself, come to its downfall; only when thoughts run out are there no 
more believers. To the thinker his thinking is a ``sublime labor, a sacred 
activity,'' and it rests on a firm \textit{faith}, the faith in truth. At 
first praying is a sacred activity, then this sacred ``devotion'' passes 
over into a rational and reasoning ``thinking,'' which, however, likewise 
retains in the ``sacred truth'' its underangeable basis of faith, and is 
only a marvelous machine that the spirit of truth winds up for its service. 
Free thinking and free science busy \textit{me} -- for it is not I that am 
free, not \textit{I} that busy myself, but thinking is free and busies me -- 
with heaven and the heavenly or ``divine''; \textit{e.g.}, properly, with 
the world and the worldly, not this world but ``another'' world; it is only 
the reversing and deranging of the world, a busying with the \textit{essence} 
of the world, therefore a \textit{derangement}. The thinker is blind to the 
immediateness of things, and incapable of mastering them: he does not eat, 
does not drink, does not enjoy; for the eater and drinker is never the 
thinker, nay, the latter forgets eating and drinking, his getting on in life, 
the cares of nourishment, etc., over his thinking; he forgets it as the 
praying man too forgets it. This is why he appears to the forceful son of 
nature as a queer Dick, a \textit{fool --} even if he does look upon him as 
holy, just as lunatics appeared so to the ancients. Free thinking is lunacy, 
because it is \textit{pure movement of the inwardness}, of the merely 
\textit{inward man}, which guides and regulates the rest of the man. The 
shaman and the speculative philosopher mark the bottom and top rounds on the 
ladder of the \textit{inward} man, the -- Mongol. Shaman and philosopher fight 
with ghosts, demons, \textit{spirits}, gods.

Totally different from this \textit{free} thinking is \textit{own} thinking, 
\textit{my} thinking, a thinking which does not guide me, but is guided, 
continued, or broken off, by me at my pleasure. The distinction of this own 
thinking from free thinking is similar to that of own sensuality, which I 
satisfy at pleasure, from free, unruly sensuality to which I succumb.

Feuerbach, in the \textit{Principles of the Philosophy of the Future}, is 
always harping upon \textit{being}. In this he too, with all his antagonism to 
Hegel and the absolute philosophy, is stuck fast in abstraction; for 
``being'' is abstraction, as is even ``the I.'' Only \textit{I am} not 
abstraction alone: \textit{I am} all in all, consequently even abstraction or 
nothing; I am all and nothing; I am not a mere thought, but at the same time I 
am full of thoughts, a thought-world. Hegel condemns the own, 
mine,\footnote{[\textit{das Meinige}]} -- ``opinion.''\footnote{[\textit{die 
--``Meinung''}]} ``Absolute thinking'' is that which forgets that it is 
\textit{my} thinking, that \textit{I} think, and that it exists only through 
\textit{me}. But I, as I, swallow up again what is mine, am its master; it is 
only my \textit{opinion}, which I can at any moment \textit{change}, 
\textit{i.e.} annihilate, take back into myself, and consume. Feuerbach wants 
to smite Hegel's ``absolute thinking'' with \textit{unconquered being}. But 
in me being is as much conquered as thinking is. It is \textit{my} being, as 
the other is \textit{my} thinking.

With this, of course, Feuerbach does not get further than to the proof, 
trivial in itself, that I require the \textit{senses} for everything, or that 
I cannot entirely do without these organs. Certainly I cannot think if I do 
not exist sensuously. But for thinking as well as for feeling, and so for the 
abstract as well as for the sensuous, I need above all things \textit{myself}, 
this quite particular myself, this \textit{unique} myself. If I were not this 
one, \textit{e.g.} Hegel, I should not look at the world as I do look at it, 
I should not pick out of it that philosophical system which just I as Hegel 
do, etc. I should indeed have senses, as do other people too, but I should not 
utilize them as I do.

Thus the reproach is brought up against Hegel by Feuerbach\footnote{P. 47ff.} 
that he misuses language, understanding by many words something else than what 
natural consciousness takes them for; and yet he too commits the same fault 
when he gives the ``sensuous'' a sense of unusual eminence. Thus it is said, 
p. 69, ``the sensuous is not the profane, the destitute of thought, the 
obvious, that which is understood of itself.'' But, if it is the sacred, the 
full of thought, the recondite, that which can be understood only through 
mediation -- well, then it is no longer what people call the sensuous. The 
sensuous is only that which exists for \textit{the senses}; what, on the other 
hand, is enjoyable only to those who enjoy with \textit{more} than the senses, 
who go beyond sense-enjoyment or sense-reception, is at most mediated or 
introduced by the senses, \textit{i.e.}, the senses constitute a 
\textit{condition} for obtaining it, but it is no longer anything sensuous. 
The sensuous, whatever it may be, when taken up into me becomes something 
non-sensuous, which, however, may again have sensuous effects, \textit{e.g.} 
as by the stirring of my emotions and my blood.

It is well that Feuerbach brings sensuousness to honor, but the only thing he 
is able to do with it is to clothe the materialism of his ``new philosophy'' 
with what had hitherto been the property of idealism, the ``absolute 
philosophy.'' As little as people let it be talked into them that one can 
live on the ``spiritual'' alone without bread, so little will they believe 
his word that as a sensuous being one is already everything, and so spiritual, 
full of thoughts, etc.

Nothing at all is justified by \textit{being}. What is thought of \textit{is} 
as well as what is not thought of; the stone in the street \textit{is}, and my 
notion of it \textit{is} too. Both are only in different \textit{spaces}, the 
former in airy space, the latter in my head, in \textit{me}; for I am space 
like the street.

The professionals, the privileged, brook no freedom of thought, \textit{i.e.} 
no thoughts that do not come from the ``Giver of all good,'' be he called 
God, pope, church, or whatever else. If anybody has such illegitimate 
thoughts, he must whisper them into his confessor's ear, and have himself 
chastised by him till the slave-whip becomes unendurable to the free thoughts. 
In other ways too the professional spirit takes care that free thoughts shall 
not come at all: first and foremost, by a wise education. He on whom the 
principles of morality have been duly inculcated never becomes free again from 
moralizing thoughts, and robbery, perjury, overreaching, etc., remain to him 
fixed ideas against which no freedom of thought protects him. He has his 
thoughts ``from above,'' and gets no further.

It is different with the holders of concessions or patents. Every one must be 
able to have and form thoughts as he will. If he has the patent, or the 
concession, of a capacity to think, he needs no special \textit{privilege}. 
But, as ``all men are rational,'' it is free to every one to put into his 
head any thoughts whatever, and, to the extent of the patent of his natural 
endowment, to have a greater or less wealth of thoughts. Now one hears the 
admonitions that one ``is to honor all opinions and convictions,'' that 
``every conviction is authorized,'' that one must be ``tolerant to the 
views of others,'' etc.

But ``your thoughts are not my thoughts, and your ways are not my ways.'' Or 
rather, I mean the reverse: Your thoughts are \textit{my} thoughts, which I 
dispose of as I will, and which I strike down unmercifully; they are my 
property, which I annihilate as I list. I do not wait for authorization from 
you first, to decompose and blow away your thoughts. It does not matter to me 
that you call these thoughts yours too, they remain mine nevertheless, and how 
I will proceed with them is \textit{my affair}, not a usurpation. It may 
please me to leave you in your thoughts; then I keep still. Do you believe 
thoughts fly around free like birds, so that every one may get himself some 
which he may then make good against me as his inviolable property? What is 
flying around is all -- \textit{mine}.

Do you believe you have your thoughts for yourselves and need answer to no one 
for them, or as you do also say, you have to give an account of them to God 
only? No, your great and small thoughts belong to me, and I handle them at my 
pleasure.

The thought is my \textit{own} only when I have no misgiving about bringing it 
in danger of death every moment, when I do not have to fear its loss as a 
\textit{loss for me}, a loss of me. The thought is my own only when I can 
indeed subjugate it, but it never can subjugate me, never fanaticizes me, 
makes me the tool of its realization.

So freedom of thought exists when I can have all possible thoughts; but the 
thoughts become property only by not being able to become masters. In the time 
of freedom of thought, thoughts (ideas) \textit{rule}; but, if I attain to 
property in thought, they stand as my creatures.

If the hierarchy had not so penetrated men to the innermost as to take from 
them all courage to pursue free thoughts, \textit{e.g.}, thoughts perhaps 
displeasing to God, one would have to consider freedom of thought just as 
empty a word as, say, a freedom of digestion.

According to the professionals' opinion, the thought is \textit{given} to me; 
according to the freethinkers', I \textit{seek} the thought. There the 
\textit{truth} is already found and extant, only I must -- receive it from its 
Giver by grace; here the truth is to be sought and is my goal, lying in the 
future, toward which I have to run.

In both cases the truth (the true thought) lies outside me, and I aspire to 
\textit{get} it, be it by presentation (grace), be it by earning (merit of my 
own). Therefore, (1) The truth is a \textit{privilege}; (2) No, the way to it 
is patent to all, and neither the Bible nor the holy fathers nor the church 
nor any one else is in possession of the truth; but one can come into 
possession of it by -- speculating.

Both, one sees, are \textit{property-less} in relation to the truth: they have 
it either as a \textit{fief} (for the ``holy father,'' \textit{e.g.} is not 
a unique person; as unique he is this Sixtus, Clement, but he does not have 
the truth as Sixtus, Clement, but as ``holy father,'' \textit{i.e.} as a 
spirit) or as an \textit{ideal}. As a fief, it is only for a few (the 
privileged); as an ideal, for \textit{all} (the patentees).

Freedom of thought, then, has the meaning that we do indeed all walk in the 
dark and in the paths of error, but every one can on this path approach 
\textit{the truth} and is accordingly on the right path (``All roads lead to 
Rome, to the world's end, etc.''). Hence freedom of thought means this much, 
that the true thought is not my \textit{own}; for, if it were this, how should 
people want to shut me off from it?

Thinking has become entirely free, and has laid down a lot of truths which I 
must accommodate myself to. It seeks to complete itself into a \textit{system} 
and to bring itself to an absolute ``constitution.'' In the State \textit{e. 
g.} it seeks for the idea, say, till it has brought out the ``rational 
State,'' in which I am then obliged to be suited; in man (anthropology), till 
it ``has found man.''

The thinker is distinguished from the believer only by believing much more 
than the latter, who on his part thinks of much less as signified by his faith 
(creed). The thinker has a thousand tenets of faith where the believer gets 
along with few; but the former brings \textit{coherence} into his tenets, and 
takes the coherence in turn for the scale to estimate their worth by. If one 
or the other does not fit into his budget, he throws it out.

The thinkers run parallel to the believers in their pronouncements. Instead of 
``If it is from God you will not root it out,'' the word is ``If it is from 
the \textit{truth}, is true, etc.''; instead of ``Give God the glory'' -- 
``Give truth the glory.'' But it is very much the same to me whether God or 
the truth wins; first and foremost I want to win.

Aside from this, how is an ``unlimited freedom'' to be thinkable inside of 
the State or society? The State may well protect one against another, but yet 
it must not let itself be endangered by an unmeasured freedom, a so-called 
unbridledness. Thus in ``freedom of instruction'' the \textit{State} 
declares only this -- that it is suited with every one who instructs as the 
State (or, speaking more comprehensibly, the political power) would have it. 
The point for the competitors is this ``as the State would have it.'' If the 
clergy, \textit{e.g.}, does not will as the State does, then it itself 
excludes itself from \textit{competition} (\textit{vid.} France). The limit 
that is necessarily drawn in the State for any and all competition is called 
``the oversight and superintendence of the State.'' In bidding freedom of 
instruction keep within the due bounds, the State at the same time fixes the 
scope of freedom of thought; because, as a rule, people do not think farther 
than their teachers have thought.

Hear Minister Guizot: ``The great difficulty of today is the \textit{guiding 
and dominating of the mind}. Formerly the church fulfilled this mission; now 
it is not adequate to it. It is from the university that this great service 
must be expected, and the university will not fail to perform it. We, the 
\textit{government}, have the duty of supporting it therein. The charter calls 
for the freedom of thought and that of conscience.''\footnote{Chamber of 
peers, Apr. 25, 1844.} So, in favor of freedom of thought and conscience, the 
minister demands ``the guiding and dominating of the mind.''

Catholicism haled the examinee before the forum of ecclesiasticism, 
Protestantism before that of biblical Christianity. It would be but little 
bettered if one haled him before that of reason, as Ruge, \textit{e.g.}, 
wants to.\footnote{\textit{``Anekdota},'' 1, 120.} Whether the church, the 
Bible, or reason (to which, moreover, Luther and Huss already appealed) is the 
\textit{sacred authority} makes no difference in essentials.

The ``question of our time'' does not become soluble even when one puts it 
thus: Is anything general authorized, or only the individual? Is the 
generality (\textit{e.g.} State, law, custom, morality, etc.) authorized, or 
individuality? It becomes soluble for the first time when one no longer asks 
after an ``authorization'' at all, and does not carry on a mere fight 
against ``privileges.'' -- A ``rational'' freedom of teaching, which 
recognizes only the conscience of reason, \footnote{\textit{``Anekdota''}, 
1, 127. }does not bring us to the goal; we require an \textit{egoistic} 
freedom of teaching rather, a freedom of teaching for all ownness, wherein 
\textit{I} become audible and can announce myself unchecked. That I make 
myself \textit{``audible''},\footnote{[\textit{vernehmbar}]} this alone is 
``reason,''\footnote{[\textit{Vernunft}]} be I ever so irrational; in my 
making myself heard, and so hearing myself, others as well as I myself enjoy 
me, and at the same time consume me.

What would be gained if, as formerly the orthodox I, the loyal I, the moral I, 
etc., was free, now the rational I should become free? Would this be the 
freedom of me?

If I am free as ``rational I,'' then the rational in me, or reason, is free; 
and this freedom of reason, or freedom of the thought, was the ideal of the 
Christian world from of old. They wanted to make thinking -- and, as 
aforesaid, faith is also thinking, as thinking is faith -- free; the thinkers, 
\textit{i.e.} the believers as well as the rational, were to be free; for the 
rest freedom was impossible. But the freedom of thinkers is the ``freedom of 
the children of God,'' and at the same time the most merciless --hierarchy or 
dominion of the thought; for \textit{I} succumb to the thought. If thoughts 
are free, I am their slave; I have no power over them, and am dominated by 
them. But I want to have the thought, want to be full of thoughts, but at the 
same time I want to be thoughtless, and, instead of freedom of thought, I 
preserve for myself thoughtlessness.

If the point is to have myself understood and to make communications, then 
assuredly I can make use only of \textit{human} means, which are at my command 
because I am at the same time man. And really I have thoughts only as 
\textit{man}; as I, I am at the same time 
\textit{thoughtless.}\footnote{[Literally, ``thought-rid.'']} He who cannot 
get rid of a thought is so far \textit{only} man, is a thrall of 
\textit{language}, this human institution, this treasury of \textit{human} 
thoughts. Language or ``the word'' tyrannizes hardest over us, because it 
brings up against us a whole army of \textit{fixed ideas}. Just observe 
yourself in the act of reflection, right now, and you will find how you make 
progress only by becoming thoughtless and speechless every moment. You are not 
thoughtless and speechless merely in (say) sleep, but even in the deepest 
reflection; yes, precisely then most so. And only by this thoughtlessness, 
this unrecognized ``freedom of thought'' or freedom from the thought, are 
you your own. Only from it do you arrive at putting language to use as your 
\textit{property}.

If thinking is not \textit{my} thinking, it is merely a spun-out thought; it 
is slave work, or the work of a ``servant obeying at the word.'' For not a 
thought, but I, am the beginning for my thinking, and therefore I am its goal 
too, even as its whole course is only a course of my self-enjoyment; for 
absolute or free thinking, on the other hand, thinking itself is the 
beginning, and it plagues itself with propounding this beginning as the 
extremest ``abstraction'' (\textit{e.g.} as being). This very abstraction, 
or this thought, is then spun out further.

Absolute thinking is the affair of the human spirit, and this is a holy 
spirit. Hence this thinking is an affair of the parsons, who have ``a sense 
for it,'' a sense for the ``highest interests of mankind,'' for ``the 
spirit.''

To the believer, truths are a \textit{settled} thing, a fact; to the 
freethinker, a thing that is still to be \textit{settled}. Be absolute 
thinking ever so unbelieving, its incredulity has its limits, and there does 
remain a belief in the truth, in the spirit, in the idea and its final 
victory: this thinking does not sin against the holy spirit. But all thinking 
that does not sin against the holy spirit is belief in spirits or ghosts.

I can as little renounce thinking as feeling, the spirit's activity as little 
as the activity of the senses. As feeling is our sense for things, so thinking 
is our sense for essences (thoughts). Essences have their existence in 
everything sensuous, especially in the word. The power of words follows that 
of things: first one is coerced by the rod, afterward by conviction. The might 
of things overcomes our courage, our spirit; against the power of a 
conviction, and so of the word, even the rack and the sword lose their 
overpoweringness and force. The men of conviction are the priestly men, who 
resist every enticement of Satan.

Christianity took away from the things of this world only their 
irresistibleness, made us independent of them. In like manner I raise myself 
above truths and their power: as I am supersensual, so I am supertrue. 
\textit{Before me} truths are as common and as indifferent as things; they do 
not carry me away, and do not inspire me with enthusiasm. There exists not 
even one truth, not right, not freedom, humanity, etc., that has stability 
before me, and to which I subject myself. They are \textit{words}, nothing but 
words, as to the Christian nothing but ``vain things.'' In words and truths 
(every word is a truth, as Hegel asserts that one cannot \textit{tell} a lie) 
there is no salvation for me, as little as there is for the Christian in 
things and vanities. As the riches of this world do not make me happy, so 
neither do its truths. It is now no longer Satan, but the spirit, that plays 
the story of the temptation; and he does not seduce by the things of this 
world, but by its thoughts, by the ``glitter of the idea.''

Along with worldly goods, all sacred goods too must be put away as no longer 
valuable.

Truths are phrases, ways of speaking, words (l\'ogos); brought into 
connection, or into an articulate series, they form logic, science, 
philosophy.

For thinking and speaking I need truths and words, as I do foods for eating; 
without them I cannot think nor speak. Truths are men's thoughts, set down in 
words and therefore just as extant as other things, although extant only for 
the mind or for thinking. They are human institutions and human creatures, 
and, even if they are given out for divine revelations, there still remains in 
them the quality of alienness for me; yes, as my own creatures they are 
already alienated from me after the act of creation.

The Christian man is the man with faith in thinking, who believes in the 
supreme dominion of thoughts and wants to bring thoughts, so-called 
``principles,'' to dominion. Many a one does indeed test the thoughts, and 
chooses none of them for his master without criticism, but in this he is like 
the dog who sniffs at people to smell out ``his master''; he is always 
aiming at the \textit{ruling} thought. The Christian may reform and revolt an 
infinite deal, may demolish the ruling concepts of centuries; he will always 
aspire to a new ``principle'' or new master again, always set up a higher or 
``deeper'' truth again, always call forth a cult again, always proclaim a 
spirit called to dominion, lay down a \textit{law} for all.

If there is even one truth only to which man has to devote his life and his 
powers because he is man, then he is subjected to a rule, dominion, law; he is 
a servingman. It is supposed that, \textit{e.g.} man, humanity, liberty, 
etc., are such truths.

On the other hand, one can say thus: Whether you will further occupy yourself 
with thinking depends on you; only know that, \textit{if} in your thinking you 
would like to make out anything worthy of notice, many hard problems are to be 
solved, without vanquishing which you cannot get far. There exists, therefore, 
no duty and no calling for you to meddle with thoughts (ideas, truths); but, 
if you will do so, you will do well to utilize what the forces of others have 
already achieved toward clearing up these difficult subjects.

Thus, therefore, he who will think does assuredly have a task, which 
\textit{he} consciously or unconsciously sets for himself in willing that; but 
no one has the task of thinking or of believing. In the former case it may be 
said, ``You do not go far enough, you have a narrow and biased interest, you 
do not go to the bottom of the thing; in short, you do not completely subdue 
it. But, on the other hand, however far you may come at any time, you are 
still always at the end, you have no call to step farther, and you can have it 
as you will or as you are able. It stands with this as with any other piece of 
work, which you can give up when the humor for it wears off. Just so, if you 
can no longer \textit{believe} a thing, you do not have to force yourself into 
faith or to busy yourself lastingly as if with a sacred truth of the faith, as 
theologians or philosophers do, but you can tranquilly draw back your interest 
from it and let it run. Priestly spirits will indeed expound this your lack of 
interest as `laziness, thoughtlessness, obduracy, self-deception,' etc. 
But do you just let the trumpery lie, notwithstanding. No 
thing,\footnote{[\textit{Sache}]} no so-called `highest interest of 
mankind,' no `sacred cause,'\footnote{[\textit{Sache}]} is worth your 
serving it, and occupying yourself with it for \textit{its sake}; you may seek 
its worth in this alone, whether it is worth anything to \textit{you} for your 
sake. Become like children, the biblical saying admonishes us. But children 
have no sacred interest and know nothing of a `good cause.' They know all 
the more accurately what they have a fancy for; and they think over, to the 
best of their powers, how they are to arrive at it.''

Thinking will as little cease as feeling. But the power of thoughts and ideas, 
the dominion of theories and principles, the sovereignty of the spirit, in 
short the -- \textit{hierarchy}, lasts as long as the parsons, \textit{i.e.}, 
theologians, philosophers, statesmen, philistines, liberals, schoolmasters, 
servants, parents, children, married couples, Proudhon, George Sand, 
Bluntschli, etc., etc., have the floor; the hierarchy will endure as long as 
people believe in, think of, or even criticize, principles; for even the most 
inexorable criticism, which undermines all current principles, still does 
finally \textit{believe in the principle}.

Every one criticises, but the criterion is different. People run after the 
``right'' criterion. The right criterion is the first presupposition. The 
critic starts from a proposition, a truth, a belief. This is not a creation of 
the critic, but of the dogmatist; nay, commonly it is actually taken up out of 
the culture of the time without further ceremony, like \textit{e.g.} 
``liberty,'' ``humanity,'' etc. The critic has not ``discovered man,'' 
but this truth has been established as ``man'' by the dogmatist, and the 
critic (who, besides, may be the same person with him) believes in this truth, 
this article of faith. In this faith, and possessed by this faith, he 
criticises.

The secret of criticism is some ``truth'' or other: this remains its 
energizing mystery.

But I distinguish between \textit{servile} and \textit{own} criticism. If I 
criticize under the presupposition of a supreme being, my criticism 
\textit{serves} the being and is carried on for its sake: if \textit{e.g.} I 
am possessed by the belief in a ``free State,'' then everything that has a 
bearing on it I criticize from the standpoint of whether it is suitable to 
this State, for I \textit{love} this State; if I criticize as a pious man, 
then for me everything falls into the classes of divine and diabolical, and 
before my criticism nature consists of traces of God or traces of the devil 
(hence names like Godsgift, Godmount, the Devil's Pulpit), men of believers 
and unbelievers; if I criticize while believing in man as the ``true 
essence,'' then for me everything falls primarily into the classes of man and 
the un-man, etc.

Criticism has to this day remained a work of love: for at all times we 
exercised it for the love of some being. All servile criticism is a product of 
love, a possessedness, and proceeds according to that New Testament precept, 
``Test everything and hold fast the \textit{good.''}\footnote{1 Thess. 5. 
21.} ``The good'' is the touchstone, the criterion. The good, returning 
under a thousand names and forms, remained always the presupposition, remained 
the dogmatic fixed point for this criticism, remained the -- fixed idea.

The critic, in setting to work, impartially presupposes the ``truth,'' and 
seeks for the truth in the belief that it is to be found. He wants to 
ascertain the true, and has in it that very ``good.''

Presuppose means nothing else than put a \textit{thought} in front, or think 
something before everything else and think the rest from the starting-point of 
this that has \textit{been thought}, \textit{i.e.} measure and criticize it by 
this. In other words, this is as much as to say that thinking is to begin with 
something already thought. If thinking began at all, instead of being begun, 
if thinking were a subject, an acting personality of its own, as even the 
plant is such, then indeed there would be no abandoning the principle that 
thinking must begin with itself. But it is just the personification of 
thinking that brings to pass those innumerable errors. In the Hegelian system 
they always talk as if thinking or ``the thinking spirit'' (\textit{i.e.} 
personified thinking, thinking as a ghost) thought and acted; in critical 
liberalism it is always said that ``criticism'' does this and that, or else 
that ``self- consciousness'' finds this and that. But, if thinking ranks as 
the personal actor, thinking itself must be presupposed; if criticism ranks as 
such, a thought must likewise stand in front. Thinking and criticism could be 
active only starting from themselves, would have to be themselves the 
presupposition of their activity, as without being they could not be active. 
But thinking, as a thing presupposed, is a fixed thought, a \textit{dogma}; 
thinking and criticism, therefore, can start only from a \textit{dogma, i.e.} 
from a thought, a fixed idea, a presupposition.

With this we come back again to what was enunciated above, that Christianity 
consists in the development of a world of thoughts, or that it is the proper 
``freedom of thought,'' the ``free thought,'' the ``free spirit.'' The 
``true'' criticism, which I called ``servile,'' is therefore just as much 
``free'' criticism, for it is not \textit{my own}.

The case stands otherwise when what is yours is not made into something that 
is of itself, not personified, not made independent as a ``spirit'' to 
itself. \textit{Your} thinking has for a presupposition not ``thinking,'' 
but \textit{you}. But thus you do presuppose yourself after all? Yes, but not 
for myself, but for my thinking. Before my thinking, there is -- I. From this 
it follows that my thinking is not preceded by a \textit{thought}, or that my 
thinking is without a ``presupposition.'' For the presupposition which I am 
for my thinking is not one \textit{made by thinking}, not one \textit{thought 
of}, but it is \textit{posited} thinking \textit{itself}, it is the 
\textit{owner} of the thought, and proves only that thinking is nothing more 
than -- \textit{property}, \textit{i.e.} that an ``independent'' thinking, 
a ``thinking spirit,'' does not exist at all.

 This reversal of the usual way of regarding things might so resemble an empty 
playing with abstractions that even those against whom it is directed would 
acquiesce in the harmless aspect I give it, if practical consequences were not 
connected with it.

To bring these into a concise expression, the assertion now made is that man 
is not the measure of all things, but I am this measure. The servile critic 
has before his eyes another being, an idea, which he means to serve; therefore 
he only slays the false idols for his God. What is done for the love of this 
being, what else should it be but a -- work of love? But I, when I criticize, 
do not even have myself before my eyes, but am only doing myself a pleasure, 
amusing myself according to my taste; according to my several needs I chew the 
thing up or only inhale its odor.

The distinction between the two attitudes will come out still more strikingly 
if one reflects that the servile critic, because love guides him, supposes he 
is serving the thing (cause) itself.

\textit{The} truth, or ``truth in general,'' people are bound not to give 
up, but to seek for. What else is it but the \textit{\^Etre supr\^eme}, the 
highest essence? Even ``true criticism'' would have to despair if it lost 
faith in the truth. And yet the truth is only a -- \textit{thought}; but it is 
not merely ``a'' thought, but the thought that is above all thoughts, the 
irrefragable thought; it is \textit{the} thought itself, which gives the first 
hallowing to all others; it is the consecration of thoughts, the 
``absolute,'' the ``sacred'' thought. The truth wears longer than all the 
gods; for it is only in the truth's service, and for love of it, that people 
have overthrown the gods and at last God himself. ``The truth'' outlasts the 
downfall of the world of gods, for it is the immortal soul of this transitory 
world of gods, it is Deity itself.

I will answer Pilate's question, What is truth? Truth is the free thought, the 
free idea, the free spirit; truth is what is free from you, what is not your 
own, what is not in your power. But truth is also the completely 
unindependent, impersonal, unreal, and incorporeal; truth cannot step forward 
as you do, cannot move, change, develop; truth awaits and receives everything 
from you, and itself is only through you; for it exists only -- in your head. 
You concede that the truth is a thought, but say that not every thought is a 
true one, or, as you are also likely to express it, not every thought is truly 
and really a thought. And by what do you measure and recognize the thought? By 
\textit{your impotence}, to wit, by your being no longer able to make any 
successful assault on it! When it overpowers you, inspires you, and carries 
you away, then you hold it to be the true one. Its dominion over you certifies 
to you its truth; and, when it possesses you, and you are possessed by it, 
then you feel well with it, for then you have found your -- \textit{lord and 
master}. When you were seeking the truth, what did your heart then long for? 
For your master! You did not aspire to \textit{your} might, but to a Mighty 
One, and wanted to exalt a Mighty One (``Exalt ye the Lord our God!''). The 
truth, my dear Pilate, is -- the Lord, and all who seek the truth are seeking 
and praising the Lord. Where does the Lord exist? Where else but in your head? 
He is only spirit, and, wherever you believe you really see him, there he is a -- ghost; for the Lord is merely something that is thought of, and it was only 
the Christian pains and agony to make the invisible visible, the spiritual 
corporeal, that generated the ghost and was the frightful misery of the belief 
in ghosts.

As long as you believe in the truth, you do not believe in yourself, and you 
are a -- \textit{servant}, a -- \textit{religious man}. You alone are the 
truth, or rather, you are more than the truth, which is nothing at all before 
you. You too do assuredly ask about the truth, you too do assuredly 
``criticize,'' but you do not ask about a ``higher truth'' -- to wit, one 
that should be higher than you -- nor criticize according to the criterion of 
such a truth. You address yourself to thoughts and notions, as you do to the 
appearances of things, only for the purpose of making them palatable to you, 
enjoyable to you, and your own: you want only to subdue them and become their 
\textit{owner}, you want to orient yourself and feel at home in them, and you 
find them true, or see them in their true light, when they can no longer slip 
away from you, no longer have any unseized or uncomprehended place, or when 
they are \textit{right for you}, when they are your \textit{property}. If 
afterward they become heavier again, if they wriggle themselves out of your 
power again, then that is just their untruth -- to wit, your impotence. Your 
impotence is their power, your humility their exaltation. Their truth, 
therefore, is you, or is the nothing which you are for them and in which they 
dissolve: their truth is their \textit{nothingness}.

Only as the property of me do the spirits, the truths, get to rest; and they 
then for the first time really are, when they have been deprived of their 
sorry existence and made a property of mine, when it is no longer said ``the 
truth develops itself, rules, asserts itself; history (also a concept) wins 
the victory,'' etc. The truth never has won a victory, but was always my 
\textit{means} to the victory, like the sword (``the sword of truth''). The 
truth is dead, a letter, a word, a material that I can use up. All truth by 
itself is dead, a corpse; it is alive only in the same way as my lungs are 
alive -- to wit, in the measure of my own vitality. Truths are material, like 
vegetables and weeds; as to whether vegetable or weed, the decision lies in 
me.

Objects are to me only material that I use up. Wherever I put my hand I grasp 
a truth, which I trim for myself. The truth is certain to me, and I do not 
need to long after it. To do the truth a service is in no case my intent; it 
is to me only a nourishment for my thinking head, as potatoes are for my 
digesting stomach, or as a friend is for my social heart. As long as I have 
the humor and force for thinking, every truth serves me only for me to work it 
up according to my powers. As reality or worldliness is ``vain and a thing of 
naught'' for Christians, so is the truth for me. It exists, exactly as much 
as the things of this world go on existing although the Christian has proved 
their nothingness; but it is vain, because it has its \textit{value} not 
\textit{in itself} but \textit{in me. Of itself} it is \textit{valueless}. The 
truth is a -- \textit{creature}.

As you produce innumerable things by your activity, yes, shape the earth's 
surface anew and set up works of men everywhere, so too you may still 
ascertain numberless truths by your thinking, and we will gladly take delight 
in them. Nevertheless, as I do not please to hand myself over to serve your 
newly discovered machines mechanically, but only help to set them running for 
my benefit, so too I will only use your truths, without letting myself be used 
for their demands.

All truths \textit{beneath} me are to my liking; a truth \textit{above} me, a 
truth that I should have to \textit{direct} myself by, I am not acquainted 
with. For me there is no truth, for nothing is more than I! Not even my 
essence, not even the essence of man, is more than I! than I, this ``drop in 
the bucket,'' this ``insignificant man''!

You believe that you have done the utmost when you boldly assert that, because 
every time has its own truth, there is no ``absolute truth.'' Why, with this 
you nevertheless still leave to each time its truth, and thus you quite 
genuinely create an ``absolute truth,'' a truth that no time lacks, because 
every time, however its truth may be, still has a ``truth.''

Is it meant only that people have been thinking in every time, and so have had 
thoughts or truths, and that in the subsequent time these were other than they 
were in the earlier? No, the word is to be that every time had its ``truth of 
faith''; and in fact none has yet appeared in which a ``higher truth'' has 
not been recognized, a truth that people believed they must subject themselves 
to as ``highness and majesty.''

Every truth of a time is its fixed idea, and, if people later found another 
truth, this always happened only because they sought for another; they only 
reformed the folly and put a modern dress on it. For they did want -- who 
would dare doubt their justification for this? -- they wanted to be 
``inspired by an idea.'' They wanted to be dominated -- possessed, by a 
\textit{thought}! The most modern ruler of this kind is ``our essence,'' or 
``man.''

For all free criticism a thought was the criterion; for own criticism I am, I 
the unspeakable, and so not the merely thought-of; for what is merely thought 
of is always speakable, because word and thought coincide. That is true which 
is mine, untrue that whose own I am; true, \textit{e.g.} the union; untrue, 
the State and society. ``Free and true'' criticism takes care for the 
consistent dominion of a thought, an idea, a spirit; ``own'' criticism, for 
nothing but my \textit{self-enjoyment}. But in this the latter is in fact -- 
and we will not spare it this ``ignominy''! -- like the bestial criticism of 
instinct. I, like the criticizing beast, am concerned only for 
\textit{myself}, not ``for the cause.'' I am the criterion of truth, but I 
am not an idea, but more than idea, \textit{e.g.}, unutterable. \textit{My 
criticism} is not a ``free'' criticism, not free from me, and not 
``servile,'' not in the service of an idea, but an \textit{own} criticism.

True or human criticism makes out only whether something is \textit{suitable} 
to man, to the true man; but by own criticism you ascertain whether it is 
suitable to \textit{you}.

Free criticism busies itself with \textit{ideas}, and therefore is always 
theoretical. However it may rage against ideas, it still does not get clear of 
them. It pitches into the ghosts, but it can do this only as it holds them to 
be ghosts. The ideas it has to do with do not fully disappear; the morning 
breeze of a new day does not scare them away.

The critic may indeed come to ataraxia before ideas, but he never gets 
\textit{rid} of them; \textit{i.e.} he will never comprehend that above the 
\textit{bodily man} there does not exist something higher -- to wit, liberty, 
his humanity, etc. He always has a ``calling'' of man still left, 
``humanity.'' And this idea of humanity remains unrealized, just because it 
is an ``idea'' and is to remain such.

If, on the other hand, I grasp the idea as \textit{my} idea, then it is 
already realized, because I am its reality; its reality consists in the fact 
that I, the bodily, have it.

They say, the idea of liberty realizes itself in the history of the world. The 
reverse is the case; this idea is real as a man thinks it, and it is real in 
the measure in which it is idea, \textit{i.e.} in which I think it or 
\textit{have} it. It is not the idea of liberty that develops itself, but men 
develop themselves, and, of course, in this self-development develop their 
thinking too.

In short, the critic is not yet \textit{owner}, because he still fights with 
ideas as with powerful aliens -- as the Christian is not owner of his ``bad 
desires'' so long as he has to combat them; for him who contends against 
vice, vice \textit{exists}.

Criticism remains stuck fast in the ``freedom of knowing,'' the freedom of 
the spirit, and the spirit gains its proper freedom when it fills itself with 
the pure, true idea; this is the freedom of thinking, which cannot be without 
thoughts.

Criticism smites one idea only by another, \textit{e.g.} that of privilege by 
that of manhood, or that of egoism by that of unselfishness.

In general, the beginning of Christianity comes on the stage again in its 
critical end, egoism being combated here as there. I am not to make myself 
(the individual) count, but the idea, the general.

Why, warfare of the priesthood with \textit{egoism}, of the spiritually minded 
with the worldly-minded, constitutes the substance of all Christian history. 
In the newest criticism this war only becomes all-embracing, fanaticism 
complete. Indeed, neither can it pass away till it passes thus, after it has 
had its life and its rage out.

\myhrule


Whether what I think and do is Christian, what do I care? Whether it is human, 
liberal, humane, whether unhuman, illiberal, inhuman, what do I ask about 
that? If only it accomplishes what I want, if only I satisfy myself in it, 
then overlay it with predicates as you will; it is all alike to me.

Perhaps I too, in the very next moment, defend myself against my former 
thoughts; I too am likely to change suddenly my mode of action; but not on 
account of its not corresponding to Christianity, not on account of its 
running counter to the eternal rights of man, not on account of its affronting 
the idea of mankind, humanity, and humanitarianism, but -- because I am no 
longer all in it, because it no longer furnishes me any full enjoyment, 
because I doubt the earlier thought or no longer please myself in the mode of 
action just now practiced. As the world as property has become a 
\textit{material} with which I undertake what I will, so the spirit too as 
property must sink down into a \textit{material} before which I no longer 
entertain any sacred dread. Then, firstly, I shall shudder no more before a 
thought, let it appear as presumptuous and ``devilish'' as it will, because, 
if it threatens to become too inconvenient and unsatisfactory for \textit{me}, 
its end lies in my power; but neither shall I recoil from any deed because 
there dwells in it a spirit of godlessness, immorality, wrongfulness. as 
little as St. Boniface pleased to desist, through religious scrupulousness, 
from cutting down the sacred oak of the heathens. If the \textit{things} of 
the world have once become vain, the thoughts of the spirit must also become 
vain.

No thought is sacred, for let no thought rank as 
``devotions'';\footnote{[\textit{Andacht}, a compound form of the word 
``thought''.]} no feeling is sacred (no sacred feeling of friendship, 
mother's feelings, etc.), no belief is sacred. They are all 
\textit{alienable}, my alienable property, and are annihilated, as they are 
created, by \textit{me}.

The Christian can lose all \textit{things} or objects, the most loved persons, 
these ``objects'' of his love, without giving up himself (\textit{i.e.}, in 
the Christian sense, his spirit, his soul! as lost. The owner can cast from 
him all the \textit{thoughts} that were dear to his heart and kindled his 
zeal, and will likewise ``gain a thousandfold again,'' because he, their 
creator, remains.

Unconsciously and involuntarily we all strive toward ownness, and there will 
hardly be one among us who has not given up a sacred feeling, a sacred 
thought, a sacred belief; nay, we probably meet no one who could not still 
deliver himself from one or another of his sacred thoughts. All our contention 
against convictions starts from the opinion that maybe we are capable of 
driving our opponent out of his entrenchments of thought. But what I do 
unconsciously I half-do, and therefore after every victory over a faith I 
become again the \textit{prisoner} (possessed) of a faith which then takes my 
whole self anew into its \textit{service}, and makes me an enthusiast for 
reason after I have ceased to be enthusiastic for the Bible, or an enthusiast 
for the idea of humanity after I have fought long enough for that of 
Christianity.

Doubtless, as owner of thoughts, I shall cover my property with my shield, 
just as I do not, as owner of things, willingly let everybody help himself to 
them; but at the same time I shall look forward smilingly to the outcome of 
the battle, smilingly lay the shield on the corpses of my thoughts and my 
faith, smilingly triumph when I am beaten. That is the very humor of the 
thing. Every one who has ``sublimer feelings'' is able to vent his humor on 
the pettiness of men; but to let it play with all ``great thoughts, sublime 
feelings, noble inspiration, and sacred faith'' presupposes that I am the 
owner of all.

If religion has set up the proposition that we are sinners altogether, I set 
over against it the other: we are perfect altogether! For we are, every 
moment, all that we can be; and we never need be more. Since no defect cleaves 
to us, sin has no meaning either. Show me a sinner in the world still, if no 
one any longer needs to do what suits a superior! If I only need do what suits 
myself, I am no sinner if I do not do what suits myself, as I do not injure in 
myself a ``holy one''; if, on the other hand, I am to be pious, then I must 
do what suits God; if I am to act humanly, I must do what suits the essence of 
man, the idea of mankind, etc. What religion calls the ``sinner,'' 
humanitarianism calls the ``egoist.'' But, once more: if I need not do what 
suits any other, is the ``egoist,'' in whom humanitarianism has borne to 
itself a new-fangled devil, anything more than a piece of nonsense? The 
egoist, before whom the humane shudder, is a spook as much as the devil is: he 
exists only as a bogie and phantasm in their brain. If they were not 
unsophisticatedly drifting back and forth in the antediluvian opposition of 
good and evil, to which they have given the modern names of ``human'' and 
``egoistic,'' they would not have freshened up the hoary ``sinner'' into 
an ``egoist'' either, and put a new patch on an old garment. But they could 
not do otherwise, for they hold it for their task to be ``men.'' They are 
rid of the Good One; good is left!\footnote{[See note on p. 112.]}

We are perfect altogether, and on the whole earth there is not one man who is 
a sinner! There are crazy people who imagine that they are God the Father, God 
the Son, or the man in the moon, and so too the world swarms with fools who 
seem to themselves to be sinners; but, as the former are not the man in the 
moon, so the latter are -- not sinners. Their sin is imaginary, yet, it is 
insidiously objected, their craziness or their possessedness is at least their 
sin. Their possessedness is nothing but what they -- could achieve, the result 
of their development, just as Luther's faith in the Bible was all that he was -- competent to make out. The one brings himself into the madhouse with his 
development, the other brings himself therewith into the Pantheon and to the 
loss of -- Valhalla.

There is no sinner and no sinful egoism!

Get away from me with your ``philanthropy''! Creep in, you philanthropist, 
into the ``dens of vice,'' linger awhile in the throng of the great city: 
will you not everywhere find sin, and sin, and again sin? Will you not wail 
over corrupt humanity, not lament at the monstrous egoism? Will you see a rich 
man without finding him pitiless and ``egoistic?'' Perhaps you already call 
yourself an atheist, but you remain true to the Christian feeling that a camel 
will sooner go through a needle's eye than a rich man not be an ``un-man.'' 
How many do you see anyhow that you would not throw into the ``egoistic 
mass''? What, therefore, has your philanthropy [love of man] found? Nothing 
but unlovable men! And where do they all come from? From you, from your 
philanthropy! You brought the sinner with you in your head, therefore you 
found him, therefore you inserted him everywhere. Do not call men sinners, and 
they are not: you alone are the creator of sinners; you, who fancy that you 
love men, are the very one to throw them into the mire of sin, the very one to 
divide them into vicious and virtuous, into men and un-men, the very one to 
befoul them with the slaver of your possessedness; for you love not 
\textit{men}, but \textit{man}. But I tell you, you have never seen a sinner, 
you have only -- dreamed of him.

Self-enjoyment is embittered to me by my thinking I must serve another, by my 
fancying myself under obligation to him, by my holding myself called to 
``self-sacrifice,'' ``resignation,'' ``enthusiasm.'' All right: if I no 
longer serve any idea, any ``higher essence,'' then it is clear of itself 
that I no longer serve any man either, but -- under all circumstances -- 
\textit{myself}. But thus I am not merely in fact or in being, but also for my 
consciousness, the -- unique.\footnote{[\textit{Einzige}]}

There pertains to \textit{you} more than the divine, the human, etc.; 
\textit{yours} pertains to you.

Look upon yourself as more powerful than they give you out for, and you have 
more power; look upon yourself as more, and you have more.

You are then not merely \textit{called} to everything divine, 
\textit{entitled} to everything human, but \textit{owner} of what is yours, 
\textit{i.e.} of all that you possess the force to make your 
own;\footnote{[\textit{Eigen}]} \textit{i.e.} you are 
\textit{appropriate}\footnote{[\textit{geeignet}]} and capacitated for 
everything that is yours.

People have always supposed that they must give me a destiny lying outside 
myself, so that at last they demanded that I should lay claim to the human 
because I am -- man. This is the Christian magic circle. Fichte's ego too is 
the same essence outside me, for every one is ego; and, if only this ego has 
rights, then it is ``the ego,'' it is not I. But I am not an ego along with 
other egos, but the sole ego: I am unique. Hence my wants too are unique, and 
my deeds; in short, everything about me is unique. And it is only as this 
unique I that I take everything for my own, as I set myself to work, and 
develop myself, only as this. I do not develop men, nor as man, but, as I, I 
develop -- myself.

This is the meaning of the -- \textit{unique one}.
