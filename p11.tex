
\chapter[III. The Unique One]{\centering III.\\
THE UNIQUE ONE}

Pre-Christian and Christian times pursue opposite goals; the former wants to 
idealize the real, the latter to realize the ideal; the former seeks the 
``holy spirit,'' the latter the ``glorified body.'' Hence the former 
closes with insensitivity to the real, with ``contempt for the world''; the 
latter will end with the casting off of the ideal, with ``contempt for the 
spirit.''

The opposition of the real and the ideal is an irreconcilable one, and the one 
can never become the other: if the ideal became the real, it would no longer 
be the ideal; and, if the real became the ideal, the ideal alone would be, but 
not at all the real. The opposition of the two is not to be vanquished 
otherwise than if some one annihilates both. Only in this \textit{``some 
one},'' the third party, does the opposition find its end; otherwise idea and 
reality will ever fail to coincide. The idea cannot be so realized as to 
remain idea, but is realized only when it dies as idea; and it is the same 
with the real.

But now we have before us in the ancients adherents of the idea, in the 
moderns adherents of reality. Neither can get clear of the opposition, and 
both pine only, the one party for the spirit, and, when this craving of the 
ancient world seemed to be satisfied and this spirit to have come, the others 
immediately for the secularization of this spirit again, which must forever 
remain a ``pious wish.''

The pious wish of the ancients was \textit{sanctity}, the pious wish of the 
moderns is \textit{corporeity}. But, as antiquity had to go down if its 
longing was to be satisfied (for it consisted only in the longing), so too 
corporeity can never be attained within the ring of Christianness. As the 
trait of sanctification or purification goes through the old world (the 
washings, etc.), so that of incorporation goes through the Christian world: 
God plunges down into this world, becomes flesh, and wants to redeem it, 
\textit{e.g.}, fill it with himself; but, since he is ``the idea'' or 
``the spirit,'' people (\textit{e.g.} Hegel) in the end introduce the idea 
into everything, into the world, and prove ``that the idea is, that reason 
is, in everything.'' ``Man'' corresponds in the culture of today to what 
the heathen Stoics set up as ``the wise man''; the latter, like the former, 
a ---\textit{fleshless} being. The unreal ``wise man,'' this bodiless 
``holy one'' of the Stoics, became a real person, a bodily ``Holy One,'' 
in God \textit{made flesh;} the unreal ``man,'' the bodiless ego, will 
become real in the \textit{corporeal ego}, in \textit{me}.

There winds its way through Christianity the question about the ``existence 
of God,'' which, taken up ever and ever again, gives testimony that the 
craving for existence, corporeity, personality, reality, was incessantly 
busying the heart because it never found a satisfying solution. At last the 
question about the existence of God fell, but only to rise up again in the 
proposition that the ``divine'' had existence (Feuerbach). But this too has 
no existence, and neither will the last refuge, that the ``purely human'' is 
realizable, afford shelter much longer. No idea has existence, for none is 
capable of corporeity. The scholastic contention of realism and nominalism has 
the same content; in short, this spins itself out through all Christian 
history, and cannot end \textit{in} it.

The world of Christians is working at \textit{realizing ideas} in the 
individual relations of life, the institutions and laws of the Church and the 
State; but they make resistance, and always keep back something unembodied 
(unrealizable). Nevertheless this embodiment is restlessly rushed after, no 
matter in what degree \textit{corporeity} constantly fails to result.

For realities matter little to the realizer, but it matters everything that 
they be realizations of the idea. Hence he is ever examining anew whether the 
realized does in truth have the idea, its kernel, dwelling in it; and in 
testing the real he at the same time tests the idea, whether it is realizable 
as he thinks it, or is only thought by him incorrectly, and for that reason 
unfeasibly.

The Christian is no longer to care for family, State, etc., as 
\textit{existences;} Christians are not to sacrifice themselves for these 
``divine things'' like the ancients, but these are only to be utilized to 
make the \textit{spirit alive} in them. The \textit{real} family has become 
indifferent, and there is to arise out of it an \textit{ideal} one which would 
then be the ``truly real,'' a sacred family, blessed by God, or, according 
to the liberal way of thinking, a ``rational'' family. With the ancients, 
family, State, fatherland, is divine as a thing \textit{extant;} with the 
moderns it is still awaiting divinity, as extant it is only sinful, earthly, 
and has still to be ``redeemed,'' \textit{i.e.}, to become truly real. This 
has the following meaning: The family, etc., is not the extant and real, but 
the divine, the idea, is extant and real; whether \textit{this} family will 
make itself real by taking up the truly real, the idea, is still unsettled. It 
is not the individual's task to serve the family as the divine, but, 
reversely, to serve the divine and to bring to it the still undivine family, 
to subject everything in the idea's name, to set up the idea's banner 
everywhere, to bring the idea to real efficacy.

But, since the concern of Christianity, as of antiquity, is for the 
\textit{divine}, they always come out at this again on their opposite ways. At 
the end of heathenism the divine becomes the \textit{extramundane}, at the end 
of Christianity the \textit{intramundane}. Antiquity does not succeed in 
putting it entirely outside the world, and, when Christianity accomplishes 
this task, the divine instantly longs to get back into the world and wants to 
``redeem'' the world. But within Christianity it does not and cannot come to 
this, that the divine as \textit{intramundane} should really become the 
\textit{mundane itself:} there is enough left that does and must maintain 
itself unpenetrated as the ``bad,'' irrational, accidental, ``egoistic,'' 
the ``mundane'' in the bad sense. Christianity begins with God's becoming 
man, and carries on its work of conversion and redemption through all time in 
order to prepare for God a reception in all men and in everything human, and 
to penetrate everything with the spirit: it sticks to preparing a place for 
the ``spirit.''

When the accent was at last laid on Man or mankind, it was again the idea that 
they \textit{``pronounced eternal}. '' ``Man does not die!'' They thought 
they had now found the reality of the idea: \textit{Man is} the I of history, 
of the world's history; it is he, this \textit{ideal}, that really develops, 
\textit{i.e. realizes}, himself. He is the really real and corporeal one, for 
history is his body, in which individuals are only members. Christ is the I of 
the world's history, even of the pre-Christian; in modern apprehension it is 
man, the figure of Christ has developed into the \textit{figure of man:} man 
as such, man absolutely, is the ``central point'' of history. In ``man'' 
the imaginary beginning returns again; for ``man'' is as imaginary as Christ 
is. ``Man,'' as the I of the world's history, closes the cycle of Christian 
apprehensions.

Christianity's magic circle would be broken if the strained relation between 
existence and calling, \textit{e.g.}, between me as I am and me as I should 
be, ceased; it persists only as the longing of the idea for its bodiliness, 
and vanishes with the relaxing separation of the two: only when the idea 
remains ---idea, as man or mankind is indeed a bodiless idea, is Christianity 
still extant. The corporeal idea, the corporeal or ``completed'' spirit, 
floats before the Christian as ``the end of the days'' or as the ``goal of 
history''; it is not present time to him.

The individual can only have a part in the founding of the Kingdom of God, or, 
according to the modern notion of the same thing, in the development and 
history of humanity; and only so far as he has a part in it does a Christian, 
or according to the modern expression human, value pertain to him; for the 
rest he is dust and a worm-bag. That the individual is of himself a world's 
history, and possesses his property in the rest of the world's history, goes 
beyond what is Christian. To the Christian the world's history is the higher 
thing, because it is the history of Christ or ``man''; to the egoist only 
\textit{his} history has value, because he wants to develop only 
\textit{himself} not the mankind-idea, not God's plan, not the purposes of 
Providence, not liberty, etc. He does not look upon himself as a tool of the 
idea or a vessel of God, he recognizes no calling, he does not fancy that he 
exists for the further development of mankind and that he must contribute his 
mite to it, but he lives himself out, careless of how well or ill humanity may 
fare thereby. If it were not open to confusion with the idea that a state of 
nature is to be praised, one might recall Lenau's \textit{``Three 
Gypsies.''}- What, am I in the world to realize ideas? To do my part by my 
citizenship, say, toward the realization of the idea ``State,'' or by 
marriage, as husband and father, to bring the idea of the family into an 
existence? What does such a calling concern me! I live after a calling as 
little as the flower grows and gives fragrance after a calling.

The ideal ``Man'' is \textit{realized} when the Christian apprehension turns 
about and becomes the proposition, ``I, this unique one, am man.'' The 
conceptual question, ``what is man?'' ---has then changed into the personal 
question, ``who is man?'' With ``what'' the concept was sought for, in 
order to realize it; with ``who'' it is no longer any question at all, but 
the answer is personally on hand at once in the asker: the question answers 
itself.

They say of God, ``Names name thee not.'' That holds good of me: no 
\textit{concept} expresses me, nothing that is designated as my essence 
exhausts me; they are only names. Likewise they say of God that he is perfect 
and has no calling to strive after perfection. That too holds good of me 
alone.

I am \textit{owner} of my might, and I am so when I know myself as 
\textit{unique}. In the \textit{unique one} the owner himself returns into his 
creative nothing, of which he is born. Every higher essence above me, be it 
God, be it man, weakens the feeling of my uniqueness, and pales only before 
the sun of this consciousness. If I concern myself for 
myself,\footnote{[\textit{Stell' Ich auf Mich meine Sache.} Literally, ``if I 
set my affair on myself.'']} the unique one, then my concern rests on its 
transitory, mortal creator, who consumes himself, and I may say:

All things are nothing to me.\footnote{[\textit{``Ich hab' Mein' Sach' auf 
Nichts gestellt}.'' Literally, ``I have set my affair on nothing.'' See 
note on p. 8.]}

\begin{center}
THE END\end{center}
