
\chapter[Translator's Preface]{\centering TRANSLATOR'S PREFACE}

If the style of this book is found unattractive, it will show that I have done 
my work ill and not represented the author truly; but, if it is found odd, I 
beg that I may not bear all the blame. I have simply tried to reproduce the 
author's own mixture of colloquialisms and technicalities, and his preference 
for the precise expression of his thought rather than the word conventionally 
expected.

One especial feature of the style, however, gives the reason why this preface 
should exist. It is characteristic of Stirner's writing that the thread of 
thought is carried on largely by the repetition of the same word in a modified 
form or sense. That connection of ideas which has guided popular instinct in 
the formation of words is made to suggest the line of thought which the writer 
wishes to follow. If this echoing of words is missed, the bearing of the 
statements on each other is in a measure lost; and, where the ideas are very 
new, one cannot afford to throw away any help in following their connection. 
Therefore, where a useful echo (and then are few useless ones in the book) 
could not be reproduced in English, I have generally called attention to it in 
a note. My notes are distinguished from the author's by being enclosed in 
parentheses.

One or two of such coincidences of language, occurring in words which are 
prominent throughout the book, should be borne constantly in mind as a sort of 
\textit{Keri perpetuum;} for instance, the identity in the original of the 
words ``spirit'' and ``mind,'' and of the phrases ``supreme being'' and 
``highest essence.'' In such cases I have repeated the note where it seemed 
that such repetition might be absolutely necessary, but have trusted the 
reader to carry it in his head where a failure of his memory would not be 
ruinous or likely.

For the same reason--that is, in order not to miss any indication of the drift 
of the thought -- I have followed the original in the very liberal use of 
italics, and in the occasional eccentric use of a punctuation mark, as I might 
not have done in translating a work of a different nature.

I have set my face as a flint against the temptation to add notes that were 
not part of the translation. There is no telling how much I might have 
enlarged the book if I had put a note at every sentence which deserved to have 
its truth brought out by fuller elucidation -- or even at every one which I 
thought needed correction. It might have been within my province, if I had 
been able, to explain all the allusions to contemporary events, but I doubt 
whether any one could do that properly without having access to the files of 
three or four well-chosen German newspapers of Stirner's time. The allusions 
are clear enough, without names and dates, to give a vivid picture of certain 
aspects of German life then. The tone of some of them is explained by the fact 
that the book was published under censorship.

I have usually preferred, for the sake of the connection, to translate 
Biblical quotations somewhat as they stand in the German, rather than conform 
them altogether to the English Bible. I am sometimes quite as near the 
original Greek as if I had followed the current translation.

Where German books are referred to, the pages cited are those of the German 
editions even when (usually because of some allusions in the text) the titles 
of the books are translated.

\begin{flushright}
Steven T. Byington\end{flushright}


\newpage{}

~

\vspace{200pt}

\begin{center}
\textbf{THE EGO AND HIS OWN}
\end{center}

