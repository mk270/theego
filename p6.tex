
\chapter[I. A Human Life]{\centering I.\\
A HUMAN LIFE}

From the moment when he catches sight of the light of the world a man seeks to 
find out \textit{himself} and get hold of \textit{himself} out of its 
confusion, in which he, with everything else, is tossed about in motley 
mixture.

But everything that comes in contact with the child defends itself in turn 
against his attacks, and asserts its own persistence.

Accordingly, because each thing\textit{cares for itself} at the same time 
comes into constant collision with other things, the \textit{combat} of 
self-assertion is unavoidable.

\textit{Victory or defeat} -- between the two alternatives the fate of the 
combat wavers. The victor becomes the \textit{lord,} the vanquished one the 
\textit{subject}: the former exercises \textit{supremacy} and ``rights of 
supremacy,'' the latter fulfills in awe and deference the ``duties of a 
subject.''

But both remain \textit{enemies}, and always lie in wait: they watch for each 
other's \textit{weaknesses} -- children for those of their parents and parents 
for those of their children (\textit{e.g.,} their fear); either the stick 
conquers the man, or the man conquers the stick.

In childhood liberation takes the direction of trying to get to the bottom of 
things, to get at what is ``back of'' things; therefore we spy out the weak 
points of everybody, for which, it is well known, children have a sure 
instinct; therefore we like to smash things, like to rummage through hidden 
corners, pry after what is covered up or out of the way, and try what we can 
do with everything. When we once get at what is back of the things, we know we 
are safe; when, \textit{e.g.,} we have got at the fact that the rod is too 
weak against our obduracy, then we no longer fear it, ``have out-grown it.''

Back of the rod, mightier than it, stands our -- obduracy, our obdurate 
courage. By degrees we get at what is back of everything that was mysterious 
and uncanny to us, the mysteriously-dreaded might of the rod, the father's 
stern look, etc., and back of all we find our ataraxia, \textit{i. e.} 
imperturbability, intrepidity, our counter force, our odds of strength, our 
invincibility. Before that which formerly inspired in us fear and deference we 
no longer retreat shyly, but take \textit{courage}. Back of everything we find 
our \textit{courage}, our superiority; back of the sharp command of parents 
and authorities stands, after all, our courageous choice or our outwitting 
shrewdness. And the more we feel ourselves, the smaller appears that which 
before seemed invincible. And what is our trickery, shrewdness, courage, 
obduracy? What else but -- \textit{mind!}\footnote{\textit{Geist}. This word 
will be translated sometimes ``mind'' and sometimes ``spirit'' in the 
following pages.}

Through a considerable time we are spared a fight that is so exhausting later 
-- the fight against \textit{reason.} The fairest part of childhood passes 
without the necessity of coming to blows with reason. We care nothing at all 
about it, do not meddle with it, admit no reason. We are not to be persuaded 
to anything by \textit{conviction}, and are deaf to good arguments, 
principles, etc.; on the other hand, coaxing, punishment, etc. are hard for us 
to resist.

This stern life-and-death combat with \textit{reason} enters later, and begins 
a new phase; in childhood we scamper about without racking our brains much.

\textit{Mind} is the name of the \textit{first} self-discovery, the first 
self-discovery, the first undeification of the divine; \textit{i. e.}, of the 
uncanny, the spooks, the ``powers above.'' Our fresh feeling of youth, this 
feeling of self, now defers to nothing; the world is discredited, for we are 
above it, we are \textit{mind}.

Now for the first time we see that hitherto we have not looked at the world 
\textit{intelligently} at all, but only stared at it.

We exercise the beginnings of our strength on \textit{natural powers}. We 
defer to parents as a natural power; later we say: Father and mother are to be 
forsaken, all natural power to be counted as riven. They are vanquished. For 
the rational, \textit{i.e.} the ``intellectual'' man, there is no family as 
a natural power; a renunciation of parents, brothers, etc., makes its 
appearance. If these are ``born again'' as \textit{intellectual, rational 
powers}, they are no longer at all what they were before.

And not only parents, but \textit{men in general}, are conquered by the young 
man; they are no hindrance to him, and are no longer regarded; for now he 
says: One must obey God rather than men.

From this high standpoint everything \textit{``earthly''} recedes into 
contemptible remoteness; for the standpoint is -- the \textit{heavenly}.

The attitude is now altogether reversed; the youth takes up an 
\textit{intellectual} position, while the boy, who did not yet feel himself as 
mind, grew up on mindless learning. The former does not try to get hold of 
\textit{things} (\textit{e.g.} to get into his head the \textit{data} of 
history), but of the \textit{thoughts} that lie hidden in things, and so, 
\textit{e.g.}, of the \textit{spirit} of history. On the other hand, the boy 
understands \textit{connections} no doubt, but not ideas, the spirit; 
therefore he strings together whatever can be learned, without proceeding 
\textit{a priori} and theoretically, \textit{i.e.} without looking for ideas.

As in childhood one had to overcome the resistance of the \textit{laws of the 
world}, so now in everything that he proposes he is met by an objection of the 
mind, of reason, of his \textit{own conscience}. ``That is unreasonable, 
unchristian, unpatriotic,'' etc., cries conscience to us, and -- frightens us 
away from it. Not the might of the avenging Eumenides, not Poseidon's wrath, 
not God, far as he sees the hidden, not the father's rod of punishment, do we 
fear, but -- \textit{conscience.}

We ``run after our thoughts'' now, and follow their commands just as before 
we followed parental, human ones. Our course of action is determined by our 
thoughts (ideas, conceptions, \textit{faith}) as it is in childhood by the 
commands of our parents.

For all that, we were already thinking when we were children, only our 
thoughts were not fleshless, abstract, \textit{absolute}, \textit{i. e.}, 
NOTHING BUT THOUGHTS, a heaven in themselves, a pure world of thought, 
\textit{logical} thoughts.

On the contrary, they had been only thoughts that we had about a 
\textit{thing}; we thought of the thing so or so. Thus we may have thought 
``God made the world that we see there,'' but we did not think of 
(``search'') the ``depths of the Godhead itself''; we may have thought 
``that is the truth about the matter,'' but we do not think of Truth itself, 
nor unite into one sentence ``God is truth.'' The ``depths of the Godhead, 
who is truth,'' we did not touch. Over such purely logical, \textit{i.e.} 
theological questions, ``What is truth?'' Pilate does not stop, though he 
does not therefore hesitate to ascertain in an individual case ``what truth 
there is in the thing,'' \textit{i.e.} whether the \textit{thing} is true.

Any thought bound to a \textit{thing} is not yet \textit{nothing but a 
thought}, absolute thought.

To bring to light the \textit{pure thought}, or to be of its party, is the 
delight of youth; and all the shapes of light in the world of thought, like 
truth, freedom, humanity, Man, etc., illumine and inspire the youthful soul.

But, when the spirit is recognized as the essential thing, it still makes a 
difference whether the spirit is poor or rich, and therefore one seeks to 
become rich in spirit; the spirit wants to spread out so as to found its 
empire -- an empire that is not of this world, the world just conquered. Thus, 
then, it longs to become all in all to itself; \textit{i.e.}, although I am 
spirit, I am not yet \textit{perfected} spirit, and must first seek the 
complete spirit.

But with that I, who had just now found myself as spirit, lose myself again at 
once, bowing before the complete spirit as one not my own but 
\textit{supernal}, and feeling my emptiness.

Spirit is the essential point for everything, to be sure; but then is every 
spirit the ``right'' spirit? The right and true spirit is the ideal of 
spirit, the ``Holy Spirit.'' It is not my or your spirit, but just -- an 
ideal, supernal one, it is ``God.'' ``God is spirit.'' And this supernal 
``Father in heaven gives it to those that pray to him.''\footnote{Luke 11, 
13.}

The man is distinguished from the youth by the fact that he takes the world as 
it is, instead of everywhere fancying it amiss and wanting to improve it, 
\textit{i.e.} model it after his ideal; in him the view that one must deal 
with the world according to his \textit{interest,} not according to his 
\textit{ideals}, becomes confirmed.

So long as one knows himself only as \textit{spirit}, and feels that all the 
value of his existence consists in being spirit (it becomes easy for the youth 
to give his life, the ``bodily life,'' for a nothing, for the silliest point 
of honor), so long it is only \textit{thoughts} that one has, ideas that he 
hopes to be able to realize some day when he has found a sphere of action; 
thus one has meanwhile only \textit{ideals}, unexecuted ideas or thoughts.

Not till one has fallen in love with his \textit{corporeal} self, and takes a 
pleasure in himself as a living flesh-and-blood person -- but it is in mature 
years, in the man, that we find it so -- not till then has one a personal or 
\textit{egoistic} interest, \textit{i.e.} an interest not only of our spirit, 
\textit{e. g.}, but of total satisfaction, satisfaction of the whole chap, a 
\textit{selfish} interest. Just compare a man with a youth, and see if he will 
not appear to you harder, less magnanimous, more selfish. Is he therefore 
worse? No, you say; he has only become more definite, or, as you also call it, 
more ``practical.'' But the main point is this, that he makes 
\textit{himself} more the center than does the youth, who is infatuated about 
other things, \textit{e.g.} God, fatherland, etc.

Therefore the man shows a \textit{second} self-discovery. The youth found 
himself as \textit{spirit} and lost himself again in the \textit{general} 
spirit, the complete, holy spirit, Man, mankind -- in short, all ideals; the 
man finds himself as \textit{embodied} spirit.

Boys had only \textit{unintellectual} interests (\textit{i.e.} interests 
devoid of thoughts and ideas), youths only \textit{intellectual} ones; the man 
has bodily, personal, egoistic interests.

If the child has not an \textit{object} that it can occupy itself with, it 
feels \textit{ennui}; for it does not yet know how to occupy itself with 
\textit{itself}. The youth, on the contrary, throws the object aside, because 
for him \textit{thoughts} arose out of the object; he occupies himself with 
his \textit{thoughts}, his dreams, occupies himself intellectually, or ``his 
mind is occupied.''

The young man includes everything not intellectual under the contemptuous name 
of ``externalities.'' If he nevertheless sticks to the most trivial 
externalities (\textit{e.g.} the customs of students' clubs and other 
formalities), it is because, and when, he discovers \textit{mind} in them, 
\textit{i.e.} when they are \textit{symbols} to him.

As I find myself back of things, and that as mind, so I must later find 
\textit{myself} also back of \textit{thoughts} -- to wit, as their creator and 
owner. In the time of spirits thoughts grew till they overtopped my head, 
whose offspring they yet were; they hovered about me and convulsed me like 
fever-phantasies -- an awful power. The thoughts had become \textit{corporeal} 
on their own account, were ghosts, \textit{e. g.} God, Emperor, Pope, 
Fatherland, etc. If I destroy their corporeity, then I take them back into 
mine, and say: ``I alone am corporeal.'' And now I take the world as what it 
is to me, as \textit{mine}, as my property; I refer all to myself.

If as spirit I had thrust away the world in the deepest contempt, so as owner 
I thrust spirits or ideas away into their ``vanity.'' They have no longer 
any power over me, as no ``earthly might'' has power over the spirit.

The child was realistic, taken up with the things of this world, till little 
by little he succeeded in getting at what was back of these very things; the 
youth was idealistic, inspired by thoughts, till he worked his way up to where 
he became the man, the egoistic man, who deals with things and thoughts 
according to his heart's pleasure, and sets his personal interest above 
everything. Finally, the old man? When I become one, there will still be time 
enough to speak of that.
